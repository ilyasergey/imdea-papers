\section{Verifying Exchanger's Client}
\label{sec:cal}
\newcommand{\ts}{\mathit{ts}}
\newcommand{\vvs}{\mathit{vs}}
\newcommand{\acc}{\mathit{ac}}
\newcommand{\ws}{\mathit{ws}}
\newcommand{\sorted}[1]{\mathsf{sorted}\ #1}

%\paragraph{Client definition.}

We now substantiate \textbf{\emph{Step 3}} of our method and
illustrate how the verified exchanger from Section~\ref{sec:overview}
can be used by client programs, and how the \emph{other} component,
asserted by the spec to satisfy
$\gist \subseteq \histo \hunion \mygather{\pending}$, is crucial in
this process.
%
We emphasize that the proof of the client does not see the
implementation details, which are hidden by
spec~\eqref{tag:exchangespec}.
%

While simple, our client is realistic, and has been used
in~\code{java.util.concurrent}~\cite{ExchangerClass}. It is defined as
follows. First, the exchanger loops until it exchanges the value.
%
\vspace{-2pt}
\[
\vspace{-2pt}
{\small{
\begin{array}{rl}
& \esc{exchange'}~(v : A) : A = \{\\[1pt]
&  ~~~~ w' \Asgn \esc{exchange}~v;\\[1pt]
&  ~~~~
  \kw{if}~~w'~~\kw{is}~~\esc{Some}~w~~\kw{then}~~\kw{return}~w~~\kw{else}~~\esc{exchange'}~v~\}
\end{array}
}}
\]
%
Next, $\esc{exchange'}$ is iterated to exchange a sequence in order,
appending the received matches to an accumulator.
%
%\vspace{-2pt}
\[
{\small{
\begin{array}{rl}
& \esc{ex\_seq}~(\vvs, \acc : \esc{seq}~A) : \esc{seq}~A = \{\\[1pt]
& ~~~~ \kw{if}~~\vvs~~\kw{is}~~v{::}\vvs'~~\kw{then}\\[1pt]
& ~~~~ ~~~~ w \Asgn \esc{exchange'}~v;~~\esc{ex\_seq}~(\vvs', \esc{snoc}~\acc~w)\\[1pt]
& ~~~~ \kw{else}~~\kw{return}~\acc~\}
\end{array}
}}
\]
%
Our goal is to prove, out of~\eqref{tag:exchangespec},
% compositionally, \ie~reasoning only out of the
%spec of $\mathtt{exchange}$, 
that the parallel composition
%
\[
e = \esc{ex\_seq}~(\vvs_1, \esc{nil}) \parallel \esc{ex\_seq}~(\vvs_2, \esc{nil})
\]
%
exchanges $\vvs_1$ and $\vvs_2$, \ie,~returns the pair $(\vvs_2,
\vvs_1)$. This holds only under the assumption that $e$ runs without
interference (\ie, quiescently), so that the two threads in $e$ have
no choice but to exchange the values between themselves. 

We make the quiescence assumption explicit using the FCSL $\hide$
constructor, as described in Section~\ref{sec:background}.
%
%The FCSL typechecker prevents $\hide~e$ from being
%composed in parallel with other exchanger threads, thus enforcing
%quiescence. We refer to Section~\ref{sec:background} for details about
%$\hide$, but here just mention that $\hide~e$ also removes selected
%parts of $e$'s auxiliary state, making them invisible outside of
%$\hide$. In particular, the precondition and postcondition of $\hide$
%existentially quantify over selected \emph{self} and \emph{joint}
%components of $e$, while setting the \emph{other} components to
%$\emptyset$ (which is sound due to quiescence). In this example, we
%select to hide the histories and offer sets of $e$, and to fold the
%heap $\heapj$ into the private heap of the surrounding program.
%
Thus, we establish the following Hoare triple:
%
\[
\tag{\arabic{tags}}\refstepcounter{tags}\label{tag:hidespec} 
{\small{
\!\!\!\!\!
\begin{array}{c}
\specK{\{\heaps = g \mapsto\mathsf{null}\}}~~\hide~~e~~\specK{\{g \in
  \mathsf{dom}~\heaps, \res = (\vvs_2, \vvs_1)\}} % @ \cal P
\end{array}
}}
\]
%
It says that we start with a heap where $g$ stores $\mathsf{null}$,
and end with a possibly larger heap (due to the memory leak), but with
the result $(\vvs_2, \vvs_1)$. The auxiliaries $\perms, \permo$,
$\gists, \gisto$, $\heapj, \pending$ are visible inside $\hide$, but
outside, only $\heaps$ persists.

%\an{I deliberately haven't removed the $@ \cal P$ and $@ \cal E$ parts
%  from the displays in this section. This should be done only if we
%  decide to remove the FCSL background section. As of now, just to try
%  it on, I have moved that section to the appendix, but have instead
%  included brief explanations of the rules for hiding and parallel
%  composition in the present section (see the text below). Thus, the
%  present section can also serve as a brief, informal introduction to
%  FCSL, albeit without the description of concurroids (hence, the need
%  to remove $@ \cal P$). I think this is very appropriate for PLDI.
%%
%  It remains to be seen if we can rework the counting network section
%  so that it doesn't rely on these rules, but only invokes the
%  intuitive descriptions that I provided here. Doing so would probably
%  be good for the paper, as it would remove the technicalities, and
%  shine the light on the main ideas. 
%%
%  A starting point for achieving this is to remove the proof outlines
%  in that section, and replace each with a program : spec
%  ascription. Also, each such ascription should have a prose
%  describing why it holds informally, but we already have most of that
%  prose written, so that shouldn't be too much work.}

%$vs$ and $ws$ are exchanged. This is a valid property under the
%assumptions that the two $\esc{ex\_seq}$ threads run in isolation,
%\ie, without interfernce from any other exchanging threads. In that
%case, the two threads have no other options but to exchange values
%between themselves.

\paragraph{Explaining the verification.}
%
We illustrate the verification by listing the specs of selected
subprograms. First, the spec of $\esc{exchange'}$ easily derives
from~(\ref{tag:exchangespec}) by removing the now-impossible failing
case.
%First, $\esc{exchange'}$ is easy to specify,
%as we merely need remove from~(\ref{tag:exchangespec}) the impossible
%case of $\esc{exchange'}$ failing.
%
\[
{\small{
\begin{array}{c}
\specK{\{\heaps = \emptyset, \perms = \emptyset, \hists = \emptyset, \gist \subseteq \histo \hunion \mygather{\pending}\}}\\[2pt]
\esc{exchange'}\ v\\[2pt]
\spec{\!\!
\begin{array}{c}
\heaps = \emptyset, \perms = \emptyset, \gist \subseteq \histo \hunion \mygather{\pending}, \\[1pt]    
\exists t\ldot \hists = t \mapsto (v, \res), \mathsf{last} (\gist) < t, \twin{t}
\end{array}
\!\!}%@\cal E
% \specK{\{\heaps = \emptyset, \perms = \emptyset, \gist \subseteq \gists \hunion \histo \hunion \mygather{\pending}, \hbox{}}\\
% \specK{\exists t\ldot \hists = t \mapsto (v, \res) \hunion \gists, \mathsf{last} (\gist) < t, \twin{t}\}} @ 
\end{array}
}}
\]
%
Next, $\esc{ex\_seq}$ has the following spec:
%
\[
{\small{
\begin{array}{c}
\specK{\{\heaps = \emptyset, \perms = \emptyset, \hists = \emptyset\}}\\[2pt]
\mathtt{ex\_seq}~(vs, \mathsf{nil})\\[2pt]
\spec{\!\!\!
\begin{array}{c}
\exists \ts\ldot \heaps = \emptyset, \perms = \emptyset, 
\hists = \mathsf{zip}~\ts~\vvs~\res,
\\[1pt]
\mathsf{grows\_notwins}~\ts, 
\mathsf{zip}~\overline{\ts}~\res~\vvs \subseteq \histo  \hunion \mygather{\pending}  
\end{array}
\!\!\!}%@ \cal E
\end{array}
}}
\]
%
Here, $\ts$ is a list of time-stamps, and
$\mathsf{zip}\,\ts\,\vvs\,\ws$ joins up the singleton histories
$t\,{\mapsto}\,(v, w)$, for each $t$, $v$, $w$ drawn, in order, from
the lists $\ts$, $\vvs$, $\ws$.
%
%\[
%{\small{
%\mathsf{zip}~ts~vs~ws = \left\{%
%\begin{array}{l}
%t \mapsto (v, w) \hunion \mathsf{zip}~\ts'~\vvs'~\ws', \\
%\hphantom{\emptyset,}\ \mbox{if $\ts=t\,{::}\,\ts', \vvs=v\,{::}\,\vvs', \ws=w\,{::}\,\ws'$}\\
%\emptyset, \mbox{if $\ts = \vvs = \ws = \mathsf{nil}$}\\
%\mbox{undefined}, \mbox{otherwise}
%\end{array}\right.
%}}
%\]
%
The spec says that at the time-stamps from $\ts$, $\esc{ex\_seq}$
exchanged the elements of $\vvs$ for those of $\esc{res}$. That $\ts$
is increasing and contains no twins, follows from the spec of
$\esc{exchange'}$ which says that the time-stamps $t$ and $\bar t$
that populate $\ts$ and $\overline{\ts}$, are larger than anything in
$\gist$, and thus only grow with iteration.
%
%\an{This point is subtle, but it requires
%  discussing framing and validity, so better avoid. The postcondition
%  implies that $\bar{\ts}$ is also increasing, so we don't state it
%  explicitly.}  
%
From the same postcondition, it follows that $\histo \hunion
\mygather{\pending}$ contains all the twin exchanges, by
invariant~(\ref{tag:exchanging}), as commented in
Section~\ref{sec:overview} about the spec for $\esc{exchange}$.

Next, by the FCSL parallel composition rule
(Section~\ref{sec:background}):
%
\[
%\tag{\arabic{tags}}\refstepcounter{tags}\label{tag:e}\\
{\small{
\begin{array}{c}
\specK{\{\heaps = \emptyset, \perms = \emptyset, \hists = \emptyset\}}\\[2pt]
%\begin{array}{c}
%\specK{\{\heaps = \emptyset, \perms = \emptyset, \hists = \emptyset\}}\\[1pt]
\mathsf{ex\_seq}~(\vvs_1, \mathsf{nil}) %\\[1pt]
%\specK{\{Q(\vvs_1)\}}
%\end{array} 
\parallel
%\begin{array}{c}
%\specK{\{\heaps = \emptyset, \perms = \emptyset, \hists = \emptyset\}}\\[1pt]
\mathsf{ex\_seq}~(\vvs_2, \mathsf{nil}) \\[1pt]
%\specK{\{Q(\vvs_2)\}}
%\end{array}\\[2pt]
\spec{\!\!\!
\begin{array}{c}
\exists \ts_1~\ts_2\ldot \mathsf{grows\_notwins}~{\ts_1}, \mathsf{grows\_notwins}~{\ts_2},\\
 \heaps = \emptyset, \perms = \emptyset, \hists = \mathsf{zip}~\ts_1~\vvs_1~\res.1 \hunion \mathsf{zip}~\ts_2~\vvs_2~\res.2,\\
 \mathsf{zip}~\overline{\ts_1}~\res.1~\vvs_1 \subseteq \mathsf{zip}~\ts_2~\vvs_2~\res.2 \hunion \histo \hunion\mygather{\pending}, \\%\tag{\arabic{tags}}\refstepcounter{tags}\label{tag:x}\\
 \mathsf{zip}~\overline{\ts_2}~\res.2~\vvs_2 \subseteq \mathsf{zip}~\ts_1~\vvs_1~\res.1 \hunion \histo \hunion\mygather{\pending}. %\tag{\arabic{tags}}\refstepcounter{tags}\label{tag:y}
\end{array}
\!\!\!}
%\{Q(\vvs_1) \circledast Q(\vvs_2)\}}
%\makebox[0pt]{\quad $@ \cal E$}
\end{array}
}}
\]
%
%\gad{Some unification on notation is due: Note that $\mathsf{ex\_seq}$
%  has been currified here whereas before it took a pair of
%  arguments. Also the prelude of the outline still refers to the
%  appendix (Section~\ref{sec:background}} \an{fixed} 
%
To explain: $ts$ and $\esc{res}$ from the left and right
$\esc{ex\_seq}$ threads become $ts_1$, $ts_2$, $\esc{res}.1$ and
$\esc{res}.2$, respectively. The values of each \emph{self} component
$\heaps$, $\perms$, $\hists$ from the two threads are joined into the
\emph{self} component of the composition. At the same time, the
\emph{other} component $\histo$ of the left (resp.~right) thread
equals the sum of $\hists$ of the right (resp.~left) thread, and the
$\histo$ of the composition.  This formalizes the intuition that upon
forking, the left thread becomes part of the environment for the right
thread, and vice-versa.

The postcondition says that the self history of $e$ contains both
$\mathsf{zip}\,\ts_1\,\vvs_1\,\res.1$ and
$\mathsf{zip}\,\ts_2\,\vvs_2\,\res.2$. Thus, $\vvs_1$ is exchanged for
$\res.1$, and $\vvs_2$ for $\res.2$. But we further want to derive
$\res.1\,{=}\,\vvs_2$ and $\res.2\,{=}\,\vvs_1$, \ie, the lists are
exchanged \emph{for each other}, in the absence of interference.

We next explain how this desired property follows for $\hide~e$, from
the two inequalities in $e$'s postcondition
\begin{align}
\mathsf{zip}~\overline{\ts_1}~\res.1~\vvs_1\,&\subseteq&\!\!\!\!{\mathsf{zip}}~\ts_2~\vvs_2~\res.2\,&\,{\hunion}\,\histo\,{\hunion}\,\mygather{\pending}, \tag{\arabic{tags}}\refstepcounter{tags}\label{tag:x}\\
\mathsf{zip}~\overline{\ts_2}~\res.2~\vvs_2\,&\subseteq&\!\!\!\!{\mathsf{zip}}~\ts_1~\vvs_1~\res.1\,&\,{\hunion}\,\histo\,{\hunion}\,\mygather{\pending}. \tag{\arabic{tags}}\refstepcounter{tags}\label{tag:y}
\end{align}
Notice that $(\ref{tag:x})$ and $(\ref{tag:y})$ are ultimately
instances of the conjunct $\gist \subseteq \histo \hunion
\mygather{\pending}$ that was part of the
specification~(\ref{tag:exchangespec}), thereby justifying the use of
subjective \emph{other} variables.

We know that $\mathsf{dom}\ \pending\,{=}\,\perms \hunion \permo$
(from Section~\ref{sec:overview}), that $\perms\,{=}\,\emptyset$ (from
$e$'s postcondition), and that by hiding,
$\permo\,{=}\,\histo\,{=}\,\emptyset$. Thus, towards deriving the
postcondition of $\hide~e$, we simplify $(\ref{tag:x})$ and
$(\ref{tag:y})$ into:
\begin{align*}
\mathsf{zip}~\overline{\ts_1}~\res.1~\vvs_1 \subseteq \mathsf{zip}~\ts_2~\vvs_2~\res.2\\% \tag{\ref{tag:x}'}\label{tag:x'}\\
\mathsf{zip}~\overline{\ts_2}~\res.2~\vvs_2 \subseteq \mathsf{zip}~\ts_1~\vvs_1~\res.1% \tag{\ref{tag:y}'}\label{tag:y'}
\end{align*}
%
Because $\ts_1$ and $\ts_2$ are increasing lists of time-stamps, and
contain no twins, the above implies $\ts_2 = \overline{\ts_1}$. Hence:
%\an{The reasoning is a bit subtle and goes as follows: From the above
%  inequations, we get: bar(ts1) <= ts2 and bar(ts2) <= ts1, ***as
%  sets***. Because bar is idempotent, we get bar(ts1) = ts2 **as
%  sets**. Now, we need that bar(ts1) is sorted, in order to conclude
%  that bar(ts1) = ts2 **as lists**. But sortedness of bar(ts1) follows
%  from the fact that ts1 has no twins.}
%
\[
\mathsf{zip}~\overline{\ts_1}~\res.1~\vvs_1 = \mathsf{zip}~\ts_2~\vvs_2~\res.2
\]
%
and thus $\res.1\,{=}\,\vvs_2$, $\vvs_1\,{=}\,\res.2$. We omit the
remaining technical argument that explains how the heap $\heapj$, with
the pointer $g$, is folded into $\heaps$, which ultimately
obtains~\eqref{tag:hidespec}.

