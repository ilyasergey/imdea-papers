\section{Verifying Exchanger's Client}
\label{sec:cal}
\newcommand{\ts}{{ts}}
\newcommand{\vvs}{{vs}}
\newcommand{\acc}{{ac}}
\newcommand{\ws}{{ws}}
\newcommand{\sorted}[1]{\mathsf{sorted}\ #1}

%\paragraph{Client definition.}
We next illustrate how the verified exchanger from
Section~\ref{sec:overview} can be used by client programs, and how the
\emph{other} component, asserted by the spec to satisfy $\gist
\subseteq \histo \hunion \mygather{\pending}$, is crucial in this
process.
%
We emphasize that proof of the client does not see the fine-grained
implementation details of the exchanger, which are hidden by
spec~\eqref{tag:exchangespec}.
%
A version of the client we consider is actually used in
\code{java.util.concurrent}~\cite{ExchangerClass}, and is defined as
follows. First, we make the exchanger loop until it succeeds in
exchanging the value.
%
\vspace{-5pt}
\[
\vspace{-5pt}
{\small{
\begin{array}{rl}
& \esc{exchange'}~(v : A) : A = \{\\[1pt]
&  ~~~~ w' \Asgn \esc{exchange}~v;\\[1pt]
&  ~~~~
  \kw{if}~~w'~~\kw{is}~~\esc{Some}~w~~\kw{then}~~\kw{return}~w~~\kw{else}~~\esc{exchange'}~v~\}
\end{array}
}}
\]
%
Next, $\esc{exchange'}$ is iterated over a sequence, exchanging each
element in order, appending the received matches to an accumulator. 
%
\vspace{-5pt}
\[
\vspace{-5pt}
{\small{
\begin{array}{rl}
& \esc{ex\_seq}~(\vvs, \acc : \esc{seq}~A) : \esc{seq}~A = \{\\[1pt]
& ~~~~ \kw{if}~~\vvs~~\kw{is}~~v{::}\vvs'~~\kw{then}\\[1pt]
& ~~~~ ~~~~ w \Asgn \esc{exchange'}~v;~~\esc{ex\_seq}~(\vvs', \esc{snoc}~\acc~w)\\[1pt]
& ~~~~ \kw{else}~~\kw{return}~\acc~\}
\end{array}
}}
\]
%
Our goal is to show
% compositionally, \ie~reasoning only out of the
%spec of $\mathtt{exchange}$, 
that the parallel composition
%
\[
e = \esc{ex\_seq}~(\vvs_1, \esc{nil}) \parallel \esc{ex\_seq}~(\vvs_2, \esc{nil})
\]
%
exchanges $\vvs_1$ and $\vvs_2$, \ie,~returns the pair $(\vvs_2,
\vvs_1)$. This is a valid property, assuming that $e$ runs without
interference, so that the two threads in $e$ have no choice but to
exchange the values between themselves. We make the assumption
explicit by using the $\hide$ constructor. Thus, the Hoare
triple we will prove is:
%
\[
\tag{\arabic{tags}}\refstepcounter{tags}\label{tag:hidespec} 
{\small{
\!\!\!
\begin{array}{c}
\specK{\{\heaps = g \mapsto\mathsf{null}\}}~~\hide~~e~~\specK{\{g \in
  \mathsf{dom}~\heaps, \res = (\vvs_2, \vvs_1)\}} @ \cal P
\end{array}
}}
\]
%
It says that we start with a heap where $g$ stores $\mathsf{null}$,
and end with a possibly larger heap (due to the memory leak of the
exchanger), but with the result $\res = (\vvs_2, \vvs_1)$.

%$vs$ and $ws$ are exchanged. This is a valid property under the
%assumptions that the two $\esc{ex\_seq}$ threads run in isolation,
%\ie, without interfernce from any other exchanging threads. In that
%case, the two threads have no other options but to exchange values
%between themselves.

\paragraph{Explaining the verification.}
%
The proof of $\hide~~e$ involves several stages: verification of
$\esc{exchange'}$, $\esc{ex\_seq}$, $e$ and finally $\hide~e$. We only
list the specs involved in the first two stages, omitting the
associated proof outlines. These are straightforward, as the programs
contain only sequential composition, and can be found in our Coq
scripts. The last two stages employ parallel composition and $\hide$,
and are thus more involved, so we present them in detail.

The following are the specs for $\esc{exchange'}$ and $\esc{ex\_seq}$.
%
\[
{\small{
\begin{array}{c}
\specK{\{\heaps = \emptyset, \perms = \emptyset, \hists = \gists, \gist \subseteq \gists \hunion \histo \hunion \mygather{\pending}\}}\\[2pt]
\esc{exchange'}\ v\\[2pt]
\spec{\!\!
\begin{array}{c}
\heaps = \emptyset, \perms = \emptyset, \gist \subseteq \gists \hunion
  \histo \hunion \mygather{\pending}, \\[1pt]    
\exists t\ldot \hists = t \mapsto (v, \res) \hunion \gists, \mathsf{last} (\gist) < t, \twin{t}
\end{array}
\!\!}@\cal E
% \specK{\{\heaps = \emptyset, \perms = \emptyset, \gist \subseteq \gists \hunion \histo \hunion \mygather{\pending}, \hbox{}}\\
% \specK{\exists t\ldot \hists = t \mapsto (v, \res) \hunion \gists, \mathsf{last} (\gist) < t, \twin{t}\}} @ 
\end{array}
}}
\]
%
%\vspace{-10pt}
%
\[
{\small{
\!\!\!
\begin{array}{c}
\spec{\!\!
\begin{array}{c}
\heaps = \emptyset, \perms = \emptyset, \hists = \mathsf{zip}~\ts~\vvs~\acc, \hbox{}\\[1pt]
\sorted{\ts}, \mathsf{zip}~\overline{\ts}~\acc~\vvs \subseteq \histo \hunion \mygather{\pending}
\end{array}
\!\!}
\\[2pt]
\mathtt{ex\_seq}~\vvs'~\acc\\[2pt]
\spec{\!\!\!\!
\begin{array}{c}
\exists \ts'~\acc'\ldot\heaps = \emptyset, \perms = \emptyset, \res =
  \acc~\esc{++}~\acc',\\[1pt] 
\hists = \mathsf{zip}~(\ts~\esc{++}~\ts')~(\vvs~\esc{++}~\vvs')~(\acc~\esc{++}~\acc'), 
 \sorted{(\ts~\esc{++}~\ts')},\\[1pt]
\mathsf{zip}~(\overline{\ts}~\esc{++}~\overline{\ts'})~(\acc~\esc{++}~\acc')~(\vvs~\esc{++}~\vvs') \subseteq 
      \histo  \hunion \mygather{\pending}
\end{array}
\!\!\!\!}@\cal E
\end{array}
}}
\]

The spec for $\esc{exchange'}$ is immediately derived
from~(\ref{tag:exchangespec}) by removing the now-impossible case of
exchange failing, and then framing the history components by $\gists$,
as explained in Section~\ref{sec:background}.
%
The spec for $\esc{ex\_seq}$ is more complicated.  It starts with two
logical variables $\ts$ and $\vvs$. The conjunct
$\hists = \mathsf{zip}~\ts~\vvs~\acc$ in the precondition says that
$\ts$ is the list of time-stamps currently generated by our thread,
and $\vvs$ is the list of values exchanged for those in $\acc$. Here
$\mathsf{zip}$ creates a history out of a list of time-stamps and
values:
%
\[
{\small{
\mathsf{zip}~ts~vs~ws = \left\{%
\begin{array}{l}
t \mapsto (v, w) \hunion \mathsf{zip}~\ts'~\vvs'~\ws', \\
\hphantom{\emptyset,}\ \mbox{if $\ts=t\,{::}\,\ts', \vvs=v\,{::}\,\vvs', \ws=w\,{::}\,\ws'$}\\
\emptyset, \mbox{if $\ts = \vvs = \ws = \mathsf{nil}$}\\
\mbox{undefined}, \mbox{otherwise}
\end{array}\right.
}}
\]
%
%\[
%\begin{array}{c}
%\mathsf{zip}~(t\,{::}\,ts)~(v\,{::}\,vs)~(w\,{::}\,ws) = 
%  t \mapsto (v, w) \hunion \mathsf{zip}~ts~vs~ws\\
%\mathsf{zip}~\mathsf{nil}~\mathsf{nil}~\mathsf{nil} = \emptyset\\
%\mathsf{zip}~ts~vs~ws = \mbox{undefined otherwise}
%\end{array}
%\]
The precondition further assumes that the time-stamps in $\ts$, when
considered as natural numbers, strictly grow ($\sorted{\ts}$), and
that the \emph{other} history contains the ``twin'' history of
$\hists$: $\mathsf{zip}~\overline{\ts}~\acc~\vvs \subseteq \histo
\hunion \mygather{\pending}$. Here $\overline{\ts}$ denotes the list
of twin time-stamps of $\ts$, and $\acc$ and $\vvs$ appear in the
reversed order compared to $\hists$.

The postcondition says that $\esc{ex\_seq}$ adds a list of time-stamps
$\ts'$, which are all larger than time-stamps in $\ts$ and also
strictly grow. Intuitively, this holds because the postcondition of
$\esc{exchange'}$ ensures the time-stamp $t$ is larger than any
time-stamps used in $\hists$, $\histo$ or $\pending$ (by choosing the
logical variable $\gist$ in the spec of $\esc{exchange'}$ to be the
union of $\hists$, $\histo$ and $\mygather{\pending}$). The self
history $\hists$ is extended with the new time-stamps and values:
$\hists =
\mathsf{zip}~(\ts~\esc{++}~\ts')~(\vvs~\esc{++}~\vvs')~(\acc~\esc{++}~\acc')$. Similarly
for the environment history, which must be a twin of $\hists$ by the
resource invariant~(\ref{tag:exchanging}). The result $\res$ appends
an unknown list of values $\acc'$, supplied by interfering threads, to
the starting list $\acc$.

The above spec for $\esc{ex\_seq}$ is also used as its loop
invariant. For use in clients, we restrict it to
$\acc = \ts = \vvs = \mathsf{nil}$, and derive:
%
\[
{\small{
\begin{array}{c}
\specK{\{\heaps = \emptyset, \perms = \emptyset, \hists = \emptyset\}}\\[2pt]
\mathtt{ex\_seq}~vs~\mathsf{nil}\\[2pt]
\spec{\!\!\!
\begin{array}{c}
\exists \ts\ldot \heaps = \emptyset, \perms = \emptyset, 
\mathsf{sorted}~\ts, \hists = \mathsf{zip}~\ts~\vvs~\res,
\\[1pt]
\mathsf{zip}~\overline{\ts}~\res~\vvs \subseteq \histo  \hunion \mygather{\pending}  
\end{array}
\!\!\!}@ \cal E
\end{array}
}}
\]
%
Naming the postcondition above as $Q(\vvs)$, we can now verify the
parallel composition $e$.
%
\[
\tag{\arabic{tags}}\refstepcounter{tags}\label{tag:e}\\
{\small{
\begin{array}{c}
\specK{\{\heaps = \emptyset, \perms = \emptyset, \hists = \emptyset\}}\\[2pt]
%\specK{\{\heaps = \emptyset\hunion\emptyset, \perms = \emptyset\hunion\emptyset, \hists = \emptyset\hunion\emptyset\}}\\
\specK{\{(\heaps = \emptyset, \perms = \emptyset, \hists = \emptyset) \circledast 
  (\heaps = \emptyset, \perms = \emptyset, \hists = \emptyset)\}}\\[2pt]
\begin{array}{c}
\specK{\{\heaps = \emptyset, \perms = \emptyset, \hists = \emptyset\}}\\[1pt]
\mathsf{ex\_seq}~\vvs_1~\mathsf{nil}\\[1pt]
\specK{\{Q(\vvs_1)\}}
\end{array} \parallel
\begin{array}{c}
\specK{\{\heaps = \emptyset, \perms = \emptyset, \hists = \emptyset\}}\\[1pt]
\mathsf{ex\_seq}~\vvs_2~\mathsf{nil}\\[1pt]
\specK{\{Q(\vvs_2)\}}
\end{array}\\[2pt]
\specK{\{Q(\vvs_1) \circledast Q(\vvs_2)\}}\makebox[0pt]{\quad $@ \cal E$}
\end{array}
}}
\]
Unfolding the definition of $\circledast$, the postcondition obtains:
%
%{\small{
\begin{align*}
\exists \ts_1~& \ts_2\ldot \heaps = \emptyset, \perms = \emptyset, \mathsf{sorted}~{\ts_1}, \mathsf{sorted}~{\ts_2},\\
& \hists = \mathsf{zip}~\ts_1~\vvs_1~\res.1 \hunion \mathsf{zip}~\ts_2~\vvs_2~\res.2,\\
& \mathsf{zip}~\overline{\ts_1}~\res.1~\vvs_1 \subseteq \mathsf{zip}~\ts_2~\vvs_2~\res.2 \hunion \histo \hunion\mygather{\pending}, \tag{\arabic{tags}}\refstepcounter{tags}\label{tag:x}\\
& \mathsf{zip}~\overline{\ts_2}~\res.2~\vvs_2 \subseteq \mathsf{zip}~\ts_1~\vvs_1~\res.1 \hunion \histo \hunion\mygather{\pending}. \tag{\arabic{tags}}\refstepcounter{tags}\label{tag:y}
\end{align*}
%}}
%
Intuitively, the values of each \emph{self} component $\heaps$,
$\perms$, $\hists$ from $Q(\vvs_1)$ and $Q(\vvs_2)$ are joined into
the self component of the joined thread. At the same time, the
\emph{other} component $\histo$ of the left thread equals the sum of
the $\hists$ of the right thread, and the $\histo$ of the joining
thread. Thus, the predicates $\mathsf{zip}~\overline{\ts}~\res~\vvs
\subseteq \histo \hunion \mygather{\pending}$ from $Q_1$ and $Q_2$
become $(\ref{tag:x})$ and $(\ref{tag:y})$, respectively.
%\[
%\begin{array}{c}
%\{\heaps = \emptyset, \perms = \emptyset, \chi_s = \emptyset\}\\
%\{\heaps = \emptyset\hunion\emptyset, \perms = \emptyset\hunion\emptyset, \chi_s = \emptyset\hunion\emptyset\}\\
%\begin{array}{c}
%\{\heaps = \emptyset, \perms = \emptyset, \chi_s = \emptyset\}\\
%\mathsf{ex\_seq}~vs_1~\mathsf{nil}\\
%\{\heaps = \emptyset, \perms = \emptyset, \\
%\exists ts_1\ldot \mathsf{sorted}~ts_1, \\
%  \chi_s = \mathsf{zip}~ts_1~vs_1~\res, \\
% \mathsf{zip}~\overline{ts_1}~\res~vs_1 \subseteq \chi_o \hunion \mygather{\pending}\}
%\end{array} \parallel
%\begin{array}{c}
%\{\heaps = \emptyset, \perms = \emptyset, \chi_s = \emptyset\}\\
%\mathsf{ex\_seq}~vs_2~\mathsf{nil}\\
%\{\heaps = \emptyset, \perms = \emptyset, \\
%\exists ts_2\ldot \mathsf{sorted}~ts_2, \\
%  \chi_s = \mathsf{zip}~ts_2~vs_2~\res, \\
%\mathsf{zip}~\overline{ts_2}~\res~vs_2 \subseteq \chi_o \hunion \mygather{\pending}\}
%\end{array}\\
%\{\heaps = \emptyset, \perms = \emptyset, \exists ts_1~ts_2\ldot \mathsf{sorted}~ts_1, \mathsf{sorted}~ts_2,\\
%\chi_s = \mathsf{zip}~ts_1~vs_1~\res.1 \hunion \mathsf{zip}~ts_2~vs_2~\res.2\\
%\mathsf{zip}~\overline{ts_1}~\res.1~vs_1 \subseteq \mathsf{zip}~ts_2~vs_2~\res.2 \hunion \chi_o \hunion\mygather{\pending}\\
%\mathsf{zip}~\overline{ts_2}~\res.2~vs_2 \subseteq \mathsf{zip}~ts_1~vs_1~\res.1 \hunion \chi_o \hunion\mygather{\pending}\}
%\end{array}
%\]

What does this postcondition say? First, the self history of $e$
contains both $\mathsf{zip}~\ts_1~\vvs_1~\res.1$ and
$\mathsf{zip}~\ts_2~\vvs_2~\res.2$. Thus, $\vvs_1$ is exchanged for
$\res.1$, and $\vvs_2$ for $\res.2$. But we additionally want to
conclude that $\res.1 = \vvs_2$ and $\res.2 = \vvs_1$, \ie, the lists
are exchanged for each other, in the absence of interference.

To derive this desired property, we will use the inequalities
$(\ref{tag:x})$ and $(\ref{tag:y})$. Notice that $(\ref{tag:x})$ and
$(\ref{tag:y})$ are ultimately instances of the conjunct
$\gist \subseteq \histo \hunion \mygather{\pending}$ that was part of
the specification~(\ref{tag:exchangespec}). Hence, this example
justifies the use of subjectivity in specs.

First, we apply $\hide$ to the spec~$(\ref{tag:e})$, to limit the
interference on $e$. We pick substitution
$\Phi_1 = [\emptyset/\heaps, \emptyset/\perms, \emptyset/\hists,
g\,{\mapsto}\,\mathsf{null}/\heapj$,
$\emptyset/\pending, \emptyset/\heapo, \emptyset/\permo,
\emptyset/\histo]$,
and obtain (via the $\hide$ rule~\eqref{eq:ehide}):
%
\[
{\small{
\begin{array}{c}
\specK{\{\heaps = g \mapsto \mathsf{null}\}}\\[2pt]
\specK{\{\heaps = g \mapsto \mathsf{null}, \Phi_1 (\heaps = \emptyset, \perms = \emptyset, \hists = \emptyset)\}}\\[2pt]
\hide_{\Phi_1}~~e \\[2pt]
\specK{\{\exists \Phi_2\ldot \heaps = \Phi_2(\heapj), \Phi_2(Q (\vvs_1) \circledast Q(\vvs_2))\}} @ \cal P
\end{array}
}}
\]
%
From the unfolding of $Q(\vvs_1) \circledast Q(\vvs_2)$, we obtain
that $\Phi_2$ must be such that $\heaps$ and $\perms$ map to
$\emptyset$, and $\hists$ maps to $\mathsf{zip}~\ts_1~\vvs_1~\res.1
\hunion \mathsf{zip}~\ts_2~\vvs_2~\res.2$. Moreover, by the general
conditions on $\Phi$, $\permo$ and $\histo$ also map to
$\emptyset$. The variable $\heapj$ maps to some heap which, by
resource invariant~\ref{exP}, must contain the pointer $g$. Hence,
from $\heaps = \Phi_2(\heapj)$, we derive $g \in
\mathsf{dom}~\heaps$. Also by resource invariant~\ref{matched},
$\mathsf{dom}\ \pending = \perms \hunion \permo$, and thus $\pending$
too must map to $\emptyset$. Hence, we can weaken and simplify the
postcondition into:
\begin{align*}
\exists \ts_1~\ts_2\ldot & g \in \mathsf{dom}\ \heaps, \sorted{\ts_1}, \sorted{\ts_2}\\
& \mathsf{zip}~\overline{\ts_1}~\res.1~\vvs_1 \subseteq \mathsf{zip}~\ts_2~\vvs_2~\res.2 \tag{\ref{tag:x}'}\label{tag:x'}\\
& \mathsf{zip}~\overline{\ts_2}~\res.2~\vvs_2 \subseteq \mathsf{zip}~\ts_1~\vvs_1~\res.1 \tag{\ref{tag:y}'}\label{tag:y'}
\end{align*}
%
Now, histories are finite maps from time-stamps to value pairs. Hence
$(\ref{tag:x'})$ implies that the time-stamps from $\overline{\ts_1}$
are included in the time-stamps of $\ts_2$. Similarly from
$(\ref{tag:y'})$, $\overline{\ts_2}$ is included in $\ts_1$. Thus,
$\ts_1$ and $\ts_2$ are of same size, and moreover, since they both
are strictly increasing, it must be that $\ts_2 =
\overline{\ts_1}$. Therefore, $(\ref{tag:x'})$ can be strengthened
into an equality: 
%
\[
\mathsf{zip}~\overline{\ts_1}~\res.1~\vvs_1 = \mathsf{zip}~\ts_2~\vvs_2~\res.2
\]
%
As the time-stamps in $\overline{\ts_1}$ and $\ts_2$ appear
in the same order, it must be $\res.1 = \vvs_2$ and $\vvs_1 = \res.2$,
leading to the desired (\ref{tag:hidespec}).  
%
% \an{I omitted here some steps. In particular, that $\sorted{\ts_1}$
%   implies $\sorted{\overline{\ts_1}}$ is not straightforward, and
%   requires also deriving that there are no twins in $\ts_1$. That can
%   all been done (and has been done), but should we present it?}



