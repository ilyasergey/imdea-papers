\documentclass[10pt,nocopyrightspace]{sigplanconf}
\usepackage[1stsubmission]{oopsla2016}
%\usepackage[final]{oopsla2016}

\usepackage{amssymb,amsmath,amsthm}
\usepackage{latexsym}
\usepackage{graphicx}
\usepackage[usenames,dvipsnames]{color}
\usepackage{listings}
\usepackage{float}
\usepackage{multirow}
\usepackage[scaled]{helvet}
\usepackage[noend]{algorithmic}
\usepackage{mathrsfs}
\usepackage{mathpartir}
\usepackage{dsfont} 
\usepackage{stmaryrd}
\usepackage{url}
\usepackage{textcomp} 
\usepackage[colorlinks=true,allcolors=blue,breaklinks,draft=false]{hyperref}
\usepackage{titlesec}
\usepackage{parskip}
\usepackage{alltt} 
\usepackage{bbm}
\usepackage{alltt}
\usepackage{verbdef}
\usepackage{xspace}
\usepackage{verbatim}
\usepackage{enumitem}
\usepackage{lipsum}
\usepackage{wrapfig}
\usepackage[usenames,dvipsnames]{xcolor}
\hypersetup{linkcolor=black,citecolor=black,urlcolor=RubineRed}
\newcounter{tags}
\usepackage{flushend}

% \usepackage{natbib}
% \bibpunct();A{},
% \let\cite=\citep

% For turned column headers 
\usepackage{adjustbox} 
\usepackage{array}
\usepackage{booktabs}
\usepackage{multirow}
\usepackage{pifont}
 
% remarks
\newcommand{\todo}[1]{\textcolor{red}{({#1})}}
\newcommand{\is}[1]{\textcolor{blue}{(Ilya: {#1})}}
\newcommand{\an}[1]{\textcolor{red}{(Aleks: {#1})}}

% useful macros
\newcommand{\asgn}{\leftarrow}
\newcommand{\code}[1]{\lstinline{#1}}
\newcommand{\ccode}[1]{\code{#1}}
\newcommand{\var}[1]{\({#1}\)} 
\newcommand{\num}[1]{\({\text{{\scriptsize{#1}}}}\)}
\newcommand{\etc}{\emph{etc}}
\newcommand{\ie}{\emph{i.e.}\xspace}
\newcommand{\Ie}{\emph{I.e.}\xspace}
\newcommand{\eg}{\emph{e.g.}\xspace}
\newcommand{\Eg}{\emph{E.g.}\xspace}
\newcommand{\vs}{\emph{vs.}\xspace}
\newcommand{\etal}{\emph{et~al.}\xspace}
\newcommand{\adhoc}{\emph{ad hoc}\xspace}
\newcommand{\viz}{\emph{viz.}\xspace}
\newcommand{\dom}[1]{\mathsf{dom}(#1)}
\newcommand{\aka}{\textit{a.k.a.}\xspace}
\newcommand{\cf}{\textit{cf.}\xspace}
\newcommand{\wrt}{\emph{wrt.}\xspace}
\newcommand{\loef}{L\"{o}f}
\newcommand{\sep}{\textasteriskcentered}
\newcommand{\res}{\mathsf{res}} 
\newcommand{\ret}{\mathsf{ret}} 
\newcommand{\fix}{\mathsf{fix}} 

%specs
\newcommand{\specK}[1]{\ensuremath{\textcolor{blue}{#1}}}
\newcommand{\spec}[1]{\specK{\left\{{#1}\right\}}}
\newcommand{\specQ}[4]{[#1 , #2 , #3] \, #4}%\specQ{mL}{gL}{gE}{p}
\newcommand{\drspec}[1]{\specK{\langle{#1}\rangle}}
\newcommand{\sspec}[1]{\specK{\{{#1}\}}}




% Keep footnotes on one page
\interfootnotelinepenalty=10000 

\setlist[itemize]{leftmargin=*}

\setlength{\parindent}{0.15in}
\setlength{\topsep}{0cm}
\setlength{\parskip}{0pt}

%\titlespacing*{\section}{0pt}{*1.5}{*1.5} 
\titlespacing*{\subsection}{3pt}{*0.8}{*0.5}
\titlespacing*{\subsubsection}{0pt}{*0.8}{*0.5}
%\titlespacing*{\paragraph}{0pt}{*0.5}{*1.2}

% SSReflect listings 
\input{lstcoq}
\lstset{style=Coq}

% Hyphenation
\hyphenation{Veri-Fast}

% Bibtgex tweaks
\setcitestyle{square}
\defcitealias{Coq-manual}{Coq proof assistant}

\begin{document}

% \authorinfo{Ilya Sergey$^\dag$ \and Aleksandar Nanevski$^\ddag$ \and Anindya Banerje$^\ddag$
% \and Germ\'{a}n Andr\'{e}s Delbianco$^\ddag$} 
% {$^\dag$University College London, UK \and\and $^\ddag$IMDEA Software Institute, Spain}
% {i.sergey@ucl.ac.uk \and \{aleks.nanevski, anindya.banerjee, german.delbianco\}@imdea.org}

%\special{papersize=8.5in,11in}

% \authorinfo{Ilya Sergey}
% {University College London, UK}
% {i.sergey@ucl.ac.uk}

% \authorinfo{Aleksandar Nanevski}
% {IMDEA Software Institute, Spain}
% {aleks.nanevski@imdea.org}


% \authorinfo{Anindya Banerjee}
% {IMDEA Software Institute, Spain}
% {anindya.banerjee@imdea.org }


% \authorinfo{Germ\'{a}n Andr\'{e}s Delbianco}
% {IMDEA Software Institute, Spain}
% {german.delbianco@imdea.org}


% \vspace{-20pt}}



\authorinfo{}{}{}

% \conferenceinfo{OOPSLA~'16} {June 13--17, 2015, Portland, OR, USA}
% \CopyrightYear{2016}
% \copyrightdata{TODO}
% \doi{TODO}


%\title{The Power of Subjectivity}


\title{
  Hoare-style Specifications as Correctness Conditions \\
  for Non-linearizable Concurrent Objects }

\maketitle

\begin{comment}
Linearizability is the de facto correctness criterion for reasoning
with concurrent fine-grained data-structures. It works by relating the
concurrent history of a program with its sequential behaviour. More
precisely, for each concurrent history of an object, linearizability
requires that there exists a mapping to a sequential history, such
that the ordering of matching call/return pairs is preserved either if
they are performed by the same thread, or if they do not overlap. To
prove linearizability, one usually has to identify linearization
points in programs or object methods, showing that this particular
point is the single, atomic, point where the effect of the operation
occurs.

However, for certain complex concurrent objects, proving them to be
linearizable is not a straightforwards task: the linearization points
of their methods are not fixed by the structure of the programs
themselves, but rather depend on intricate interactions with the
environment. Traditionally, verifying such objects requires a
dedicated metatheory, e.g. supporting prophecy variables, capable of
reasoning about their highly speculative nature.

In contrast, in this paper we will present our on-going effort to
verify such objects in an existing concurrent Hoare-style logic, FCSL,
giving them (and proving) intuitive specifications that hide the
speculative reasoning about the environment. We build on the
philosophy of FCSL, where by relying upon simple, but powerful,
abstractions like partial commutative monoids (PCMs), auxiliary
histories, and state transition systems it is possible prove tight
specifications for concurrent fine-grained data-structures that
resemble those of its sequential counterparts.

To that end, we introduce re-sortable histories as an abstraction that
(i) can be implemented in FCSL off-the-shelf, without changes to the
underlying logical framework, and (ii) in combination with other FCSL~
features, such as history PCMs, it can internalize speculative
reasoning about the concurrent environment, hiding it from the library
clients.

We will illustrate our technique by presenting the mechanization in
FCSL of an optimal concurrent-snapshot algorithm originally introduced
by Jayanti, whose correctness itself has a highly-speculative nature,
relying on non-trivial arguments about its non-fixed linearization
points. Furthermore, we will comment on a few interesting details
about its mechanization in the Coq proof assistant and discuss its
pitfalls and the challenges ahead.
\end{comment}

\begin{comment}
Arguments about linearizability of a concurrent data structure are
typically carried out by specifying the structure methods'
linearization points.
%
Carrying proofs out of such specification is often cumbersome, because
the linearization points' position in time is ofthen \emph{dynamic},
that is, it may depend on the interference, run-time values and events
from the past, or even future.

In this paper we propose a new method, based on a Hoare-style logic,
for reasoning about concurrent objects with such linearization
points. The method embraces their dynamic nature, and encodes it as
part of the structure's \emph{auxiliary state}, so that it can be
dynamically changed by \emph{auxiliary code}, \emph{in place}.

The method is implemented in FCSL, a recently proposed fine-grained
variant of concurrent separation logic. The dynamic temporal ordering
is modeled by the same logical features of separation logic, which are
used to specify spatial linking of data structures via pointers,
motivating us to name the idea as \emph{linking in time}. By employing
a variant of separation logic to reason about both space and time, we
ensure modularity of reasoning.

We illustrate the method by verifying an intricate optimal snapshot
algorithm due to Jayanti.
\end{comment}

Arguments about linearizability of a concurrent data structure are
typically carried out by specifying the linearization points of the
data structure's procedures.
%
Proofs that use such specifications are often cumbersome.  This is
because the linearization points' position in time is
often \emph{dynamic}; that is, it may depend on the interference,
run-time values and events from the past, or even future.

In this paper we propose a new method, based on a Hoare-style logic,
for reasoning about concurrent objects with such linearization
points. The method embraces their dynamic nature, and encodes it as
part of the data structure's \emph{auxiliary state}, so that it can be
dynamically modified \emph{in place} by \emph{auxiliary code}, as
needed when some appropriate run-time event occurs.

The method is implemented in FCSL, a recently proposed fine-grained
variant of concurrent separation logic. The dynamic ordering of
linearization points \emph{in time} is modeled by the same features
used by separation logic to reason about pointer linking of data
structures \emph{in space}, motivating us to name the idea
as \emph{linking in time}. By employing a variant of separation logic
to reason about both space and time, we ensure modularity of
reasoning.

We illustrate the method by verifying an intricate optimal snapshot
algorithm due to Jayanti.




%\vspace{-2pt}
\section{Introduction}
\label{sec:introduction}

% Traditionally, the correctness of a concurrent object is reasoned
% using its method call/return histories, and follows two steps. The
% first step is to define a specification set that describes all
% ``basic'' behaviors that the object's client might observe when using
% it. The second step is to fix a consistency condition that
% relates the object's concurrently observable behaviors to the ones in
% the specification set.
% % 
% Typically, the specification set is taken to be the
% sequential behaviors of a concurrent object, in
% which the calls to the object's methods are immediately followed by the 
% corresponding returns.
% %
% \is{The paragraph above doesn't seem to contribute in the followin
%   intro at its present structure. Can we drop it at all or, may be,
%   move to the related work?}
 
Linearizability~\cite{Herlihy-Wing:TOPLAS90} remains the most
well-known correctness condition for concurrent objects. It has been
used to argue the correctness of a variety of concurrent objects such
as stacks, queues, sets, locks, snapshots, and their
combinations---all of which have intuitive sequential specifications
(specs).
%
For each concurrent history of an object, linearizability requires
that there exist a mapping to a sequential history, such that the
ordering of two matching call/ return pairs is preserved either if
they are performed by the same thread, or if they do not overlap.
%
However, enforcement of linearizability is expensive: an
implementation of a concurrent object must inherently incorporate
costly synchronization primitives~\cite{Attiya-al:POPL11}, which then
become points of high contention and prevent efficient parallelization
\cite{Shavit:CACM11}. Moreover, many concurrent objects are inherently
\emph{non-linearizable}. In the presence of concurrent interference,
the behaviors of these programs are observably different from their
sequential behaviors: linearizability is not flexible enough to
capture their behaviors, as observed in several recent works
\cite{Hemed-Rinetzky:PODC14,Shavit:CACM11,Derrick-al:FM14,Jagadeesan-Riely:ICALP14}.

% The way to remedy this situation
% and to enable scalability is to change the semantics of a concurrent
% object and to redefine its admissible behaviors. \ab{Not happy yet with this
% para.}

This paper therefore focuses on reasoning about non-linearizable
concurrent objects and their clients, such as counting
networks~\cite{Aspnes-al:JACM94} or an elimination-based
exchanger~\cite{Scherer-al:SCOOL05}.
%
% In the presence of concurrent interference, the behaviors of these programs are observably different from their sequential
% behaviors. Linearizability is therefore not flexible enough to
% capture their behaviors as observed in several recent works~\cite{Hemed-Rinetzky:PODC14,Shavit:CACM11,Derrick-al:FM14,Jagadeesan-Riely:ICALP14}.
%\ab{Ugh. This sounds weak.}
%
% who term them ``concurrency-aware concurrent objects'',
%
% \is{I don't think Rinetzky is so special that his should work should
%   have a dedicated mentioning here. Hence, I maked this statement a
%   bit more general and supplied some citations.}
%
%
% \an{Maybe say why? \Ie, we can use the criticism from CAL. Because
%   linearizability reduces a concurrent object to a sequential
%   approximation, and sometimes this needs to be relaxed, because the
%   behavior of the program in the presence of concurrent interference,
%   is different from its sequential behavior.}
%
% ``their behavior in the presence of concurrent (overlapping)
% operations is, and should be, \emph{observably different} from their
% behavior in the sequential setting''~\cite{Hemed-Rinetzky:PODC14}.
% 
To reason about such data structures, a variety of novel consistency
conditions and correctness criteria have been developed:
concurrency-aware linearizability (CAL)~\cite{Hemed-Rinetzky:PODC14},
quiescent consistency (QC)~\cite{Aspnes-al:JACM94},
quasi-linearizability (QL)~\cite{Afek-al:OPODIS10}, quantitative
relaxation~\cite{Henzinger-al:POPL13}, quantitative quiescent
consistency (QQC)~\cite{Jagadeesan-Riely:ICALP14}, local
linearizability~\cite{Haas-al-local15}, to name a few.
% \an{Are
%   the phrases consistency condition and correctness criteria
%   synonymous? How about consistency criteria and correctness
%   condition? We seem to use all 4 interchangeably. Should we
%   streamline?} \is{From the related literature, I observe
%   \emph{correctness criteria} and \emph{correctness condition} to be
%   synonyms. In contrast, \emph{consistency condition} is used to
%   denote the way concurrent behaviors are related to ones in
%   specification set (\eg, sequential or CA ones). I somewhat adapted
%   this terminology from Henzinger et al, POPL'13.}
%
These criteria attempt to specify a concurrent program's behavior in
the presence of interference, though some, such as QC devote special
treatment to the sequential case, trying to qualify the behavior in
the quiescent, that is, interference-free, moment.

% However, this variety makes it difficult to establish that the
% conditions are general enough to accommodate highly parallel
% concurrent objects. 
% %
% \is{I don't understand the statement above: how the diversity makes it
% difficult to accommodate algorithms---this is what it aims to help.
% Perhaps, what you want to say here is that each new condition should
% be characterized in some way?}
%

This variety makes it difficult to see when a specific consistency
condition should be used, or how to combine several of them, as may be
needed when different parts of the same program are specified using
different conditions. In particular, in contrast to linearizability,
which has been shown to imply observational refinement\footnote{That is,
  a program can be replaced in any larger context by the set of the
  sequential histories to which it linearizes; every property
  derivable about the replacement applies to the original program as
  well.}~\cite{Filipovic-al:TCS10}, no similar results have been
proven for the aforementioned consistency conditions. While some of
them, say QC and QQC, are known to be compositional in the sense that
the combination of two QC (QQC) objects is QC
(QQC)~\cite{Herlihy-Shavit:08,Jagadeesan-Riely:ICALP14}, such
compositionality is much weaker than observational refinement, and does
not allow transferring a general property (such as, one expressed as a Hoare
triple~\cite{Turon-al:ICFP13,Liang-Feng:PLDI13}) to a program, from
the set of histories with which it is QC (QQC) consistent.
%
While recently linearizability has been generalized to apply to modern
%
concurrent programs, which use higher-orderness, ownership transfer,
and dynamic
allocation~\cite{Gotsman-Yang:CONCUR12,Cerone-al:ICALP14}), the
alternative consistency conditions almost invariably focus on 
simple imperative programs. Finally, the considerations of the
alternative conditions have focused only on their semantics: there
is a lack of syntactic logical methods for checking that a program
satisfies one of them (again, in contrast to the situation for
establishing linearizability for a given
program~\cite{OHearn-al:PODC10,Liang-Feng:PLDI13,Turon-al:ICFP13,Vafeiadis:PhD}). 
%
Such methods are desirable, as they allow one to verify clients and
implementations in a single proof system.
%

% \is{Can we say, why such methods are desirable, and why they are
%   better than reasoning directly in terms of program semantics?}
% %
% \is{Presumably, uniform reasoning about clients in the presence of
%   HO, dynamic state, amenable to scalable computer-aided verification. }

From these observations stems a fundamental question: Can the
alternative correctness conditions be represented in one and the same
logical system, with support for higher-order compositional reasoning
about realistic libraries of modern, possibly non-linearizable,
concurrent programs?
%

%In short, there is a lack of proof methods amenable to structured
%formal verification with non-linearizable objects, and formal results
%that enable these proof methods to apply be compositionally used on clients.

%\ab{How to bring in client reasoning and associated problems?}
%
% \is{Here we should clarify what we mean by compositionality. QC and
%   QQC are also proven to be compositional by their authors, so that
%   statement above is misleading if not wrong.}

%Despite this variety, it is not obvious how these criteria can facilitate the
%verification of modern concurrent programs which use higher-orderness,
%ownership transfer, dynamic allocation, etc. In contrast 
%to linearizability which has been shown to imply observational refinement\footnote{That is, a program can be replaced in any larger context by the set of the sequential histories to which it linearizes.}~\cite{Filipovic-al:TCS10},
%no similar results have been proven for the aforementioned consistency criteria. 
%%
%Neither are syntactic logical methods for establishing the consistency
%criteria known (again, in contrast to the situation for establishing
%linearizability for a given program~\cite{Gotsman-Yang:CONCUR12,Cerone-al:ICALP14,Turon-al:ICFP13}). 
%%
%While QC and QQC are known to be compositional~\cite{What?}, such a property 
%only asserts that the composition of two QC (QQC) objects is QC (QQC). 
%Such compositionality is too weak to be applicable in a situation
%where, say, two procedures verified under different criteria need to be used in 
%the same program, and the program's precondition, involving the different 
%criteria, needs to be established. 
%%
%In other words, there is a lack of compositional proof methods amenable to 
%structured formal verification with non-linearizable objects. 
%\ab{How to bring in client reasoning and associated problems?}
%%
%% \is{Here we should clarify what we mean by compositionality. QC and
%%   QQC are also proven to be compositional by their authors, so that
%%   statement above is misleading if not wrong.}

%
%Moreover, client-side reasoning about programs that incorporate
%several objects specified via \emph{different} criteria becomes
%enormously complicated: clients of a concurrent object, when
%committing to a correctness criterion, need to adopt specific
%reasoning principles to characterize the object's behavior.  It can be
%difficult to ensure compositionality of reasoning when dealing with
%different communicating objects.
%%
%\is{The statement above is a bit vague (now I realize) is and isn't
%  instantiated particularly well in our paper: we don't show examples
%  with several objects (although we could). So how about we say here
%  what the previous intro used to say, that for each new criterion one
%  has to devise a method for exploiting the provided safety guarantees
%  for the sake of reasoning about client code that uses the concurrent
%  object.}
%
%\is{In my opinion, the paragraph above is crucial for the whole story,
%  as it sets the motivation for the paper (like those questions we had
%  previously), so it should be more punchy in describing what the
%  problems are with the state of the art, and why one shoudl care
%  about them.}  \an{How about: Moreover, while syntactic logical
%  methods exist for establishing linearizability for a given program
%  (CITE some stuff by Gotzman, and CaReSL), such methods do not exist
%  for the other criteria. Even if such methods existed, one would have
%  to engineer ways of combining them into a unified framework,
%  whenever two procedures verified by different criteria are to be
%  used in the same program. }

% \is{While I totally agree with all of the things said above, in my
%   opinion, at this very place we need to place a punch-phrase (a
%   slogan), that summarizes the problem we attack. Something, in the
%   style of herr Dreyer, \eg, \emph{These observations beg the
%     question: ...?}}  \an{I suppose we can pull a Dreyer here. In
%   principle, his overselling has been noted by many people, so I'm not
%   too convinced that we should follow his approach. But, we could. We
%   could say something like: These observation beg the question whether
%   all these alternative conditions can be represented in a single
%   unified logical system, with support for higher-order compositional
%   reasoning about realistic libraries of modern concurrent programs.}

\subsection{Our approach: logic-based concurrency specification}

This paper demonstrates a uniform approach---based on a Hoare-style
program logic---for verifying the correctness of highly scalable
concurrent objects and their clients, without recourse to specialized
correctness criteria and consistency conditions. Our approach uses
Fine-grained Concurrent Separation Logic
(FCSL)~\cite{Nanevski-al:ESOP14}. We show, via examples, that the
basic ingredient of FCSL, \emph{subjectivity}, provides the uniformity
we seek. Subjectivity permits that within a spec of a thread, one can
refer to the private state of all other interfering threads \emph{in a
  local manner}. Being able to refer to such state enables one to
directly express the results of a program as a function of the
interference of other threads. It ultimately yields uniform reasoning
principles capable of expressing the essential properties captured by
the various correctness criteria.

%we seek. Subjectivity permits the use of two auxiliary variables
%within the scope of each thread; one names the thread's own local
%state, and the other names the private state of all other interfering
%threads combined. This split view of auxiliary state allows directly
%relating the result of a program to the interference of all other
%concurrent threads over auxiliary state, ultimately yielding uniform
%reasoning principles.


Traditionally, correctness criteria for concurrent objects are
formulated in terms of call/return histories of threads, and their
rearrangements. In contrast, assertions in Hoare-style program logics
constrain \emph{state}, auxiliary or real, in which the program runs.
%
In order to bridge this gap, Hoare-style reasoning has been recently
extended to histories, which were formulated as a specific instance of
auxiliary
state~\cite{Fu-al:CONCUR10,Gotsman-al:ESOP13,Sergey-al:ESOP15,Bell-al:SAS10}.
%
%
% our starting point is the representation of a program's
% history directly as user-specified auxiliary state. Such a
% representation facilitates reasoning about history via Hoare-style
% specs. This is a simple and old idea~\cite{what}, that in FCSL comes
% with a twist.
%
% \is{This statement seems like it's taken directly from the ESOP'15
%   paper intro (including the twis bit). However, there it was
%   referring to histories in concurrency in general (including
%   semantics). However, I don't think that histories in Hoare-style
%   logics are an \emph{old} idea. So, may be, we can just say that
%   histories were used in previous logics to reason about FCD and cite
%   HLRG and Gotsman-Yang?}
%
% \is{The paragraph before should be changed to introduce auxiliary
%   state and related logics and then proceed to elaborate on FCSL.}
%
For instance, instead of call/return histories, FCSL allows one to
employ \emph{time-stamped histories}~\cite{Sergey-al:ESOP15} to reason
about linearizable objects. A time-stamped history consists of entries
of the form $t \mapsto a$, to signify that the (typically atomic)
operation $a$ occurred at time $t$. A Hoare-style spec which shows
that a program's history changes by a singleton $t \mapsto a$ can be
seen as exhibiting a behavior $a$ at a linearization point~$t$.
%
Such specification approach makes fine-grained (\ie, lock-free) object
implementations look like atomic ones to the clients, whose proofs are
carried out only out of the object specs.

%\an{Some comment here on the similarity between histories and heaps.}

In this work, we augment this history-based approach to Hoare-style
specifications in a significant way to handle non-linearizable
objects. In particular, we show that more \emph{general notions of
  time-stamped histories lead to adequately capturing the essence of
  alternative consistency criteria} such as CAL, QC, and QQC/QL.  To
wit, an auxiliary history need not merely identify a point at which an
atomic behavior logically occurred, but additionally can include
information about the program's interference.

For example, the main idea of CAL is that histories with which one
linearizes cannot be sequential, but have to be concurrency-aware
(CA), \ie, allow simultaneous events to be represented. In FCSL we can
do so by picking time-stamped histories with additional imposed
structure that naturally captures the simultaneity of events. In
Section~\ref{sec:overview}, we show how this structure helps in
specifying and verifying---in FCSL---an elimination-based concurrent
exchanger~\cite{Scherer-al:SCOOL05}. In Section~\ref{sec:cal}, we show
how to immediately employ the ascribed specification for the
verification of a client program of the exchanger (adapted directly
from the \code{java.util.concurrent} library
documentation~\cite{ExchangerClass}) in the same logical framework.

% \is{In the following paragraph I replaced stack by counter, which is
%   what we have verified.}

QC requires establishing that at moments of quiescence, \ie, no
interference, programs exhibit some desirable behavior. For example,
at quiescence, a concurrent counter implementation behaves as expected
of a \emph{sequential} counter implementation. We capture this
property by relying on subjectivity: we use time-stamped histories in
which a time-stamp $t$ directly stores the kind of interference
exhibited by the program's environment at time $t$.  One can then
prove, that in the absence of interference, the object behaves
sequentially as expected. In Section~\ref{sec:counting} and
Section~\ref{sec:qclients} we show the specification and
verification---in FCSL---of a simple counting
network~\cite{Aspnes-al:JACM94} and its client, both of whose
correctness relies on QC.

One can also derive stronger, \emph{quantitative}, properties, and
show that a bound on the number of interfering threads implies that
the program exhibits a bounded deviation from the expected sequential
behavior. In the past, this has been addressed using
QL~\cite{Afek-al:OPODIS10} and QQC~\cite{Jagadeesan-Riely:ICALP14} as
correctness criteria. In this paper, we derive it as a consequence of
the choice of the auxiliary state of
histories. Section~\ref{sec:qclients} also shows the verification of a
client of the counting network, whose correctness relies on QQC.

The unifying mechanism behind all these different kinds of histories
(and indeed behind the subjective split of any auxiliary state) is
that they all satisfy the algebraic properties of a \emph{partial
  commutative monoid} (PCM)~\cite{LeyWild-Nanevski:POPL13}. 
%
Thus, FCSL can represent them, in addition to heaps (also a PCM, and
often a shared resource) in a uniform reasoning framework, applying
the same logical infrastructure (such as the rule of frame) to all
kinds of state, auxiliary or real, in the process also incorporating
higher-orderness, ownership transfer, and dynamic
allocation~\cite{Nanevski-al:ESOP14,Sergey-al:ESOP15}.
%
The uniformity of the logical rules, treating all kinds of state
similarly, makes it possible to conduct the verification in a general
computer-assisted framework: all proofs of the examples from this
paper are checked mechanically in Coq~\cite{Coq-manual} and are
available as a supplementary material.

\paragraph{Alternative logic-based approaches.}

Recent concurrent program logics, such as
HOCAP~\cite{Svendsen-al:ESOP13}, iCAP \cite{Svendsen-Birkedal:ESOP14},
TaDA~\cite{ArrozPincho-al:ECOOP14}, and Iris~\cite{Jung-al:POPL15}
have shown, using the technique of parametrizing programs with
\emph{first-class auxiliary code}~\cite{Jacobs-Piessens:POPL11} or
\emph{atomic tracking resources} (see Section~\ref{sec:related} for
details), that Hoare-style program logics can adequately specify and
verify tricky linearizable concurrent objects and their clients. In
contrast, this work addresses non-linearizable objects and their
clients---but without the use of such parameterization or atomic
tracking resources, which both seem to require identifying
\emph{synchronization points} within libraries, making it non-trivial
to apply the listed above logics to the objects we consider. In the
process we also derive properties which have hitherto been obtained
only via dedicated alternative correctness criteria.

\paragraph{Observational refinement and compositional reasoning.}
The fact that linearizability implies observational
refinement~\cite{Filipovic-al:TCS10, Cerone-al:ICALP14,
  Bouajjani-al:POPL15, Emmi-al:PLDI15} justifies compositional
reasoning, whereby a program can be substituted by its sequential spec
\emph{no matter the property being verified}. Here, we consider
objects whose correctness criteria do not necessarily imply such
observational refinement. Hence, we fix our properties of interest to
be partial correctness Hoare-style specs only. In that setting,
compositionality of the reasoning is justified by the substitution
principle of FCSL on both programs and
proofs~\cite{Nanevski-al:ESOP14}.
%
% which says that a program $e$, and the proof that $e$ satisfies a
% spec $s$, can always be substituted into a context with a hole of
% spec $s$.



% \subsection{Paper outline}

% Sections~\ref{sec:overview} and~\ref{sec:cal}--\ref{sec:qclients} of
% the paper present the case studies, substantiating our proposal to use
% a Hoare-style logic for reasoning uniformly about non-linearizable
% concurrent objects and their clients.
% %
% Section~\ref{sec:background} provides necessary minimal background on
% program logics for concurrency and key concepts of
% FCSL~\cite{Nanevski-al:ESOP14}. Section~\ref{sec:discussion} discusses
% other possible applications of the proposed approach for reasoning
% about concurrent objects. Relevant related work appears in
% Section~\ref{sec:related}. Section~\ref{sec:conclusion} discusses
% future work and concludes.



%  in the
%   process several of the aforementioned correctness criteria are
%   subsumed.
% %
% \is{Isn't it a bit strong of a statement? We don't have the formal
%   correspondence proved.}
% \begin{comment}
% \an{Hmm, the remainder of this section reads a bit unfocused. We need
%   to say how we do the job, and what's essential. Thus, I would say,
%   let's focus the remainder to explaining what is subjectivity (having
%   two kinds of auxiliary state: self and other). Also, say something
%   about histories. But I would first go with subjectivity; thus, move
%   the section Why FCSL up, and then talk about histories.  But I would
%   definitely drop the section on protocols. They are not original to
%   us, and they don't particularly clarify the main message of the
%   intro, which is that subjectivity suffices for all these
%   criteria. In this particular paper, protocols are largely a
%   technicality, so why not just leave them for the background
%   section?}
% %
% \is{I disagree with the remark on unimportance of protocols. While,
%   indeed, they appeared before FCSL (and subjectivity has appeared in
%   SCSL, which we don't use in this paper), they are equally important
%   for what's achieved in this work. We don't claim protocols as our
%   contribution, but they are crucial for the essence of the
%   verification patterns that we are describing in the rest of the
%   paper (or at least for QC and QQC). I think, the paragraph ``why
%   FCSL?'' gives the right amount of intuition on what subjectivity is,
%   so let's move it up. The protocols can also appear in the same
%   paragraph, later, and with more references to the work, prior of
%   FCSL. Essentially we should plant a message that all these fellas
%   before us knew about protocols, but we now teach how to use them
%   right. :-)} 
% %
% \an{I didn't say protocols are unimportant in principle. They should
%   be covered. Just not in the \emph{introduction}, as they just don't
%   seem to be adding much to it. I was reading the paragraph on
%   protocols, and it came of deadening. It talks about stability,
%   locality, whatnot, but not about why the reader should care about
%   protocols when thinking of \emph{alternative consistency
%     criteria}. If you can rewrite the paragraph to liven it up and
%   crisply define why protocols are important for these criteria
%   specifically, then fine. But, notice, the paragraphs leading up to
%   this one say that they key to this paper is subjectivity and PCMs,
%   which is setting the tone quite a lot.}  
% %
% \is{Now I re-read it again, perhaps, we should say what exactly we mean
%   by uniformity there and how does it relate to the problems with the
%   existing approaches. As it's written now it is way too general and
%   looks like a twin of the same part in the intro of the PLDI paper
%   and I think, we should make it fresher by telling what
%   subjectivity/PCMs helped us to discover this time, which is totally
%   novel. And for this, check my comment below.}
% \is{Let me add more to it. While subjectivity, histories and protocols
%   are all important for what we're doing here, I would suggest us to
%   emphasize in the intro the verification patterns we discovered by
%   using the combination of these three, namely: histories with twin
%   contributions (CAL) and interference-capturing histories (QC and
%   QQC). I don't have a better name of these at this moment, but I
%   believe, something along these lines without too much specifics,
%   should go to the intro, as it summarizes the instances of the
%   subjectivity/histories/protocol magic, that we came up with to
%   specify our examples in this work. And the community would benefit
%   from understanding these patterns as it earlier benefited from
%   understanding Hindsight and (to some extent) fractional
%   permissions.}  
% %
% \an{Sorry, the talk of ``all three'' reminds me of ``Ghost, Protocols,
%   Shit'' paper, which I found very annoying. In general, I don't mind
%   talking about the verification patterns that you mention; I think
%   that would be preferable to the non-germane discussion on
%   protocols. But, that aside, I want to emphasize one principle, which
%   I think it important. When deciding what to say in the intro, we
%   should focus on just the most important things. If you can't say why
%   PCMs, or protocols, or natural numbers, or what have you, are
%   important to the alternative consistency criteria, which are the
%   topic from the title, then none of them should be mentioned in the
%   intro!!! We don't mention nats in the intro, so why bother with
%   protocols? Quite frankly, ditto for the PCMs!!! All these things
%   just water-down the discussion, and de-focus from the main point! I
%   think a short, but razor-focused, intro of 1-or-so page would be
%   quite refreshing; it would give us a chance to immediately jump to
%   describing the interesting stuff in the technical sections.
% %
%   Now, regarding this principle, why is subjectivity something
%   \emph{to be mentioned}? Why is it the main point? Well, here's why:
%   Subjectivity allows providing directly a concurrent specification of
%   a program, as a function of its concurrent environment. This is in
%   stark contrast to the consistency criteria approach which usually
%   (actually, all but CAL) seek to specify the program in terms of some
%   equivalent-looking sequential behavior, and then spend time
%   justifying why the sequential behavior is really
%   ``equivalent-looking''. That's why subjectivity is our secret
%   weapon, and that's why it's the key to our approach, and that's why
%   our approach works.
% %
%   I recalled this point when I saw Eraz Petrank's picture on FB
%   15-mins ago :-) I recall he was intrigued by this point that we
%   specify programs in a concurrent manner directly, when Ilya
%   mentioned to him at SPAA.}  
% %
% \is{Okay, point taken, let's then be more specific about the secret
%   weapons and the goals we're trying to achieve by this
%   paper. Subjectivity, indeed, shines here, so let's give it the usual
%   praise in the introduction.  Is the use of subjectivity the ultimate
%   essence of what we're going to demonstrate by this paper? I mean, is
%   that the case that one, given subjectivity, will come up with
%   everything else that we describe here? Sorry, but I don't think
%   so. So how about right after subjectivity we say \emph{how} exactly
%   we use it, which is (I hope you agree) is quite far from the
%   previous appearances in the four papers Aleks and company published
%   on subjectivity. So, this is why, agreeing to drop PCMs and
%   protocols from the intro (as the points made about them are fair), I
%   suggest us to focus the rest of this part of the intro on the
%   \emph{how} part, outlining our findings wrt histories and using
%   them. This ideas \emph{on top of subjectivity} are our tools and
%   \emph{primary} contributions for this paper (not the examples we
%   verified, as those are just means of demonstration!), and I didn't
%   realize this point until we had this discussion. So, just like
%   O'Hearn and company celebrate \emph{the idea of hindsight} (instead
%   of CSL or whatever logic they use there) in their PODC'10 paper, we
%   should celebrate the formalization principles enabled by
%   subjectivity that allow us to subsume CAL and (Q)QC. In contrast, if
%   we just say ``subjectivity is all you need'', we risk to end up like
%   Dodds et al with their POPL'11 paper (not to mention that it was
%   buggy), who said: ``here are the barriers and CAP is all you need to
%   verify them''. Even after reading their recent TOPLAS submission, I
%   still have no clue about what should I take out of their approach
%   except for the fact that they are very clever and know how to use
%   (i)CAP. So, let's not make this mistake and, I repeat it again,
%   emphasize \emph{how} to use subjectivity for the win. }
% %
% \an{Ok, let me try some synthesis of this discussion. We start the
%   ``our approach'' part of the intro as follows. We say that we have a
%   Hoare logic FCSL, which is subjective. Subjectivity means that we
%   have two auxiliary variables in the scope of assertions, giving us a
%   way of specifying the behavior of the program as a function of its
%   concurrent environment. This is in stark contrast to consistency
%   criteria approach that ... cut+paste+from+above. Then go on to
%   introduce the patterns you wanted, and say something like: For
%   example, in the case of CAL, we specify the behavior by combining
%   the subjective view with the auxiliary state of a special kind of
%   ``twin-symmetric'' histories that capture the inherent symmetry of
%   an exchanging program. In the case of QC and QC we use histories
%   that directly store the relevant aspects of the interfering
%   threads, and allow us to derive, as a consequence out of the spec,
%   the expected QC and QQC properties. Expand here as much as you
%   want. Then say something brief like: all these patterns can be
%   uniformly expressed as special cases of the algebraic structure of
%   PCMs, further supported by concurrent protocols that specify
%   application-specific interference. But don't dwell too much on that
%   part?}
% %
% \is{Okay, good, this seems like more or less what I'd write for
%   now. What about histories? When should we state precisely what are
%   they for us (currently, this comes in the second paragraph below)?
%   How about we start this subsection by saying right away that
%   \emph{subjectivity} and \emph{histories-as-state} are the enabling
%   tools we take from previous works? } 
% %
% \an{Let me not refer to the paragraphs below, as things look quite out
%   of place now. So its bound to be confusing. In general, histories
%   should come right after subjectivity, but before we start describing
%   the flavors of ``twin-symmetric'', or ``interference-dependent''
%   histories, etc. One can introduce them briefly: time-stamped
%   histories describe what happened at a time-stamp $t$. The
%   ``contents'' of $t$ can be diverse: atomic operation, twin-symmetric
%   exchange, some information about the behavior of threads at that
%   moment. In the previous work, we only explored the atomic-operation
%   aspect, but in this paper, we see that different definitions of
%   histories can encode different consistency criteria. The common
%   thread to all, however, is the PCM algebraic structure, etc, etc.}

% \an{I'm vacillating a bit about saying that subjectivity is ``in stark
%   contrast'' to the correctness-condition approach, because
%   correctness-conditions approach approximates by sequential
%   behavior. It is in stark contrast to linearizability (and we should
%   say that), but certainly not to CAL. Have to think more about what
%   the real contrast is with all these conditions? Maybe its auxiliary
%   state of histories giving us a way to capture temporal reasoning
%   that's the foundation of all these criteria?}

% In the FCSL program logic, specifications (specs) contain
% preconditions and postconditions, as well as a ``concurrent protocol''
% or ``concurrent resource''. This last component defines the invariants
% of the shared state that are respected by all concurrent threads,
% and/or the allowed state changes that the threads can make, as
% directed by the
% protocol~\cite{Jones:TOPLAS83,OHearn:TCS07,Turon-al:ICFP13}. To ensure
% thread-locality of the specs, the preconditions and postconditions
% must be stable with respect to the concurrent protocol, \ie, they
% should be invariant under possible changes that interfering
% environment threads can make to the shared state according to the
% protocol.

% As observed in recent works on program
% logics~\cite{Sergey-al:ESOP15,Gotsman-al:ESOP13,Fu-al:CONCUR10,Bell-al:SAS10},
% histories can be represented as instances of auxiliary state. Instead
% of call/return histories, FCSL specs use time-stamped histories. Such
% histories are indexed by discrete time-stamps that ``point to'' atomic
% operations: the histories describe relevant atomic changes in the
% logical state of a concurrent object. Moreover, as recently
% established~\cite{Sergey-al:ESOP15}, reasoning about time-stamped
% histories follows exactly the same patterns that reasoning about heaps
% follows in separation logic~\cite{Reynolds:LICS02}. Histories and
% heaps thus share the same assertion logic, the same rules of
% inference, and thus the same style of local reasoning.
% %
% In contrast to previous work, this paper considers histories that need not be
% linearizable. Indeed since they are instances of auxiliary state, histories can
% store additional logical information, such as \emph{quantitative aspects} of
% the expected interference. We demonstrate how to define
% invariants of concurrent objects in a manner that constrains the real
% and auxiliary state, and to thereby capture a suitable specification
% set of histories, such as, \eg, a \emph{concurrency-aware}
% one~\cite{Hemed-Rinetzky:PODC14}. In combination with the possibility
% to describe the allowed changes in the real and auxiliary state of an
% object, this yields a technique to express and verify the diverse
% consistency conditions on histories such as CAL, QC and QQC.
% % \ab{Not sure about this last
% %   sentence. Need more
% %   ammo here. Do the examples only show extension with additional
% %   logical information? What are the novel reasoning patterns
% %   uncovered?}
% %
% % \is{By the way, should we emphasize somewhere around here that our
% %   histories are not of calls/returns. Even though it's somewhat said
% %   above, someone can get confused, as the word ``history'' is highly
% %   overloaded in the literature.}

% \paragraph{Why FCSL?} 

% The key ingredient of FCSL that captures the essence of most of these
% consistency conditions is
% \emph{subjectivity}~\cite{LeyWild-Nanevski:POPL13}. The notion
% facilitates local reasoning by differentiating between thread-local
% contributions and the contributions of the thread's concurrent
% environment. This split leads to a direct way of relating the
% functional behavior of a program to the interference of its
% environment. Subjectivity allows, in \emph{thread-local} assertions,
% quantification over the arbitrary effects produced by \emph{other}
% interfering threads, as well as the ``shape'' of the environment (\eg,
% the number of threads running concurrently with the one being
% specified). In addition to its native support of subjectivity, FCSL
% also possesses a uniform model of thread-local resources, based on
% PCMs. The PCMs can be instantiated to reason about arbitrary state,
% auxiliary or real, such as heaps, thread capabilities, and histories.
% Finally, FCSL has been implemented as a mechanized tool for concurrency
% verification~\cite{Sergey-al:PLDI15}, enabling provably sound
% computer-aided reasoning about concurrent objects, whose state
% invariants can be complex.
% \end{comment}


% \an{It sounds like we also need to work in the word
%   \emph{subjectivity}, in order to express what's new in our approach,
%   and why the previous work didn't quite succeed in capturing what we
%   propose. The point should be that quiescent consistency, and others,
%   just naturally want to have access to the contributions of others.}
%
% \is{okay, check the paragraphs above and below.}
%
%

%
% \an{Throw in the word "separation logic" somewhere here, to make the
%   point of our observation more precise. Strictly speaking, that you
%   can have histories as ghosts was an old observation (they even call
%   them "history variables" in the model checking works). Our
%   observation was that \emph{separation logic} can natually and
%   compositionally handle history variables, using the algebraic
%   structure of PCMs.}
%

% \an{This section reads meekly, but it's the
%   most important one.  It should be strenghtened by bringing up the
%   points about subjectivity further above, as I said.  Subjectivity as
%   a crucial and most important idea, which makes all the
%   difference. If we don't mention it, peple like Noam Rinetzky or
%   Cesar Sanches will not get the feel of what's different.
%   Subjectivity is the delta that makes us succeed where everyone else
%   went into wrong directions.  Also, we should be more proud of our
%   work. It doesn't matter if "other logics" could have been
%   used. Subjectivity has been invented by SCSL and in the fine-grained
%   setting by FCSL. Just because some copy-cats later decided to
%   implement it in their logics doesn't mean we should bow to them. So
%   I think we don't need to mention "other logics can do it" part. Just
%   say that we use FCSL to explain the ideas of this paper, because
%   FCSL is based on, and introduced the idea of subjectivity.  Of
%   course, this should be done after subjectivity has been promoted
%   further above as the main idea that makes everything fly.}

% \paragraph{Outline.}
% %\label{sec:contr-paper-outl}
% \an{The examples that we cover could be folded into the above
%   description of the different kind of histories we use. That would
%   probably make a separate outline section unnecessary.}
% Recent concurrent program logics such as
% HOCAP~\cite{Svendsen-al:ESOP13}, iCAP~\cite{Svendsen-Birkedal:ESOP14}
% have shown, using the parametrization technique (see
% Section~\ref{sec:related}), that when reasoning about concurrent
% objects, linearizability can be replaced by Hoare-style program
% specifications. It is an open question whether parametrization can
% scale to the alternative consistency criteria such as CAL, QC and QQC.
% The main contribution of this paper is to show that FCSL can be used
% to reason about highly parallel, non-linearizable concurrent objects
% as well, while incorporating CAL, QC and QQC.

% In the remainder of the paper, % we demonstrate the viability of the
% % logic-based approach for defining correctness conditions for highly
% % parallel concurrent objects by
% we formally specify and verify two
% concurrent data structures: an \emph{elimination-based exchange
%   channel}~\cite{Scherer-al:SCOOL05} and a simple \emph{counting
%   network}~\cite{Aspnes-al:JACM94}, whose behavior was previously
% described only in terms of dedicated
% criteria~\cite{Hemed-Rinetzky:PODC14,Derrick-al:FM14,Jagadeesan-Riely:ICALP14}.
% We then argue for the adequacy of the provided specs by modularly
% verifying a series of concurrent client programs, which employ these
% data structures. Specifically in this work we

% \vspace{2pt}

% \begin{itemize}

% % \item describe a series of novel reasoning patterns that unify
% %   state-based and history-based approaches for specification and
% %   verification of concurrent objects. \is{With that part of the intro removed,
% %     it's not clear how these two approaches are different.}

% \item provide the first formal logic-based spec of an
%   \emph{elimination-based concurrent
%     exchanger}~\cite{Scherer-al:SCOOL05} in the spirit of
%   CAL~\cite{Hemed-Rinetzky:PODC14} (Section~\ref{sec:overview});

% \item specify and formally verify a realistic client of the
%   exchanger adapted directly from the \code{java.util.concurrent}
%   library documentation~\cite{ExchangerClass} (Section~\ref{sec:cal}).

% \item give the first logical spec to a simple \emph{counting
%     network}~\cite{Aspnes-al:JACM94} (Section~\ref{sec:counting}) and
%   verify two its clients, making use of the proved specification in
%   the spirit of QC~\cite{Derrick-al:FM14} and
%   QL/QQC~\cite{Afek-al:OPODIS10,Jagadeesan-Riely:ICALP14}
%   (Section~\ref{sec:qclients}).

% \item supply all examples from the paper with proof scripts that
%   were mechanically checked in the Coq proof
%   assistant~\cite{Coq-manual,Bertot-Casteran:BOOK,Sergey-al:PLDI15}.

% \end{itemize}

% \vspace{2pt}


% \an{We should have in the intro with a paragraph like follows: That
%   linearizability can adequatly be replaced by Hoare-style reasoning
%   has been already argued by the previous work on HOCAP and ICAP,
%   which employ the method of parametrization (to be skethed in the
%   related work section). In this paper, we argue that similar
%   replacement can be carried out for three other alternative
%   consistency criteria such as Concurrency-aware linearizability
%   (CAL), quiescent consistency, and quantitative quiescent
%   consistency. Moreover, the last two consistency criteria seem
%   impossible to address by a method of parametrization, at least
%   without some significant and complicated meta-theoretic additions
%   (e.g., prophecy variables), whereas here we show how they can be
%   easily supplanted using reasoning based on subjectivity combined
%   with histories.}

% \is{I disagree with the remark by Aleks almost entirely. First, I'm
%   sick and tired of giving credit to CAP-like approaches for something
%   they don't have enev a slight idea how to do. While I'll most
%   certainly put something on them into the related work, I don't want
%   to have enything on them in the intro, otherwise there will be
%   another round of what we've seen already a year ago.
% %  
%   Second, I don't think that remark on prophecies is sound: the
%   counting network example doesn't have anything reminiscent to
%   prophecy-requiring linearization points. That is, any comparison to
%   LP-based methods, as they're done in CAP, is dangerous and
%   misleading, as (a) they might be able to do it via some other
%   callback-related mumbo-jumbo (or they might not, but we'll get a
%   strong reject on that grounds anyway, just like the last tim), or
%   (b) they indeed cannot do it, but this fact has nothing to with with
%   their prophecies-related troubles. 
% %
%   Finally, the only people who will be able to understand this
%   comments are those who will give us strong reject because of it
%   (just like the last time). That is why I suggest us to focus bringin
%   Henzinger and Rinetzky-like crown on our side by speaking the
%   language they understand.
% %
%   So, the bottom line: I'm strongly against any specific on
%   CAP-related logics or any of this stuff in the intro.
% }

% \an{Gee, that's a strong sentiment. I will just reiterate that HOCAP
%   and ICAP have suggested that Hoare logic should replace
%   linearizability. It seems prudent to be generous and give them
%   credit for that, especially as it doesn't cost us anything
%   (basically, just one line of space), and doesn't diminish from our
%   contribution at all. A year ago, we got rejected form POPL precise
%   because we didn't have a forward-pointer in the intro to the related
%   work section, where the comparison was done. They accused us in the
%   after-rebutal comment of setting '`wrong expectations'' in our
%   intro. After that, in the ESOP version, we did put the forward
%   pointer. So, I really don't think a forward pointer would hurt us
%   here, and can only help. As for prophecies, I just put them as a
%   side comment. I don't think we really have to mention them. We could
%   simply say that its unclear that parametrization can be used to
%   handle quiescense and quantitativeness, wihtout speculiating what's
%   needed to fix that. I do personally think that prophecies will
%   probably be OK for that, if they can get them. But then, I think
%   they'll never get them, precisely because subjectivity and histories
%   is what you need here :-). As for Henzinger and Rinetzky crowd:
%   well, Philippa will be reviewing this, so expect to get someone from
%   the CAP crowd too.}
% \an{I removed the discussion on the forward pointer. The new intro
%   quite directly side-steps linearizability in the third paragraph, so
%   discussing other Hoare logics for linearizability seems
%   unnecessary.}



%\lipsum[1-4]

\section{Main Ideas and Overview}
\label{sec:overview}

We begin by outlining the high-level intuition of our specification
approach, and summarize the main formalization steps.
%
As the first motivating example, we consider the concurrent exchanger
structure from $\mathtt{java.util.concurrent}$
\cite{Scherer-al:SCOOL05,ExchangerClass}. The main purpose of the
exchanger is to allow two threads to efficiently swap values in a
non-blocking way via a globally shared channel. The exchange might
fail, if a thread trying to swap a value does not encounter a peer to
do that in a predefined period of time.
%
% In the case of a successful outcome, two ``colliding'' threads
% simultaneously exchange values, otherwise a thread with no matching
% exchange proposal fails.

For instance, the result of the two-thread program
%
\[
\tag{\arabic{tags}}\refstepcounter{tags}\label{tag:ex1} 
\begin{array}{c@{\ }c@{\ }c} 
  \boxed{T_1} & & \boxed{T_2}
  \\[5pt] 
   r_1 := \esc{exchange}~1 & || & r_2 := \esc{exchange}~2
\end{array}
\]
%
can be described by the following assertion:\footnote{We
  use ML-style \texttt{option} data type with two constructors,
  \texttt{Some} and \texttt{None} to indicate success and failure of
  an operation, correspondingly.}

%
\[
\tag{\arabic{tags}}\refstepcounter{tags}\label{tag:exsp} 
r_1 = r_2 = \esc{None} \oor r_1 = \esc{Some}~2 \wedge r_2 =
\esc{Some}~1
\]

\noindent
That is, $r_1$ and $r_2$ store the results of the execution of
subthreads $T_1$ and $T_2$ correspondingly, and both threads either
succeed, exchanging the values, or fail. The ascribed outcome is only
correct under the assumption that no other threads besides $T_1$ and
$T_2$ attempt to use the very same exchange channel concurrently.

Why is the exchanger not a linearizable data structure? To see that,
recall that linearizability reduces the concurrent behavior to a
sequential one~\cite{Herlihy-Wing:TOPLAS90}. If the exchanger were
linearizable, all possible outcomes of the program~\eqref{tag:ex1}
would be captured by the following two sequential programs, modelling
selected interleavings of the threads $T_1$ and $T_2$:
%
\[
\tag{\arabic{tags}}\refstepcounter{tags}\label{tag:ex2} 
\begin{array}{c}
r_1 := \esc{exchange}~1;~r_2 := \esc{exchange}~2;
\\[3pt]
\text{and}
\\[3pt]
r_2 := \esc{exchange}~2;~r_1 := \esc{exchange}~1;  
\end{array}
\]
%
However, both programs~\eqref{tag:ex2} will \emph{always} result in
$r_1 = r_2 = \esc{None}$, as, in order to succeed, a call to the
exchanger needs another thread, running concurrently, with which to
exchange values.
%
% \ab{Check rewording.}
% \is{Looks good}
%
This observation demonstrates that linearizability with respect to a
sequential specification is too weak a correctness criterion to
capture the \code{exchanger}'s behavior observed in a truly concurrent
context~\cite{Hemed-Rinetzky:PODC14}: an adequate notion of
correctness for \code{exchange} must mention the effect
of {interference}.
%
% \ab{Check rewording.}
% \is{Looks good}

Consider another structure, whose concurrent behavior cannot be
related to sequential executions via linearizability:
%
\[
\tag{\arabic{tags}}\refstepcounter{tags}\label{tag:flip} 
\begin{array}{l}
\esc{flip2}~(x : \esc{ptr~nat})~:~\esc{nat}~=~\{ \\[2pt]
~~ a := \esc{flip}~x; \\[2pt]
~~ b := \esc{flip}~x;\\[2pt]
~~\kw{return}~a + b ~~\}  
\end{array}
\]
%
The procedure \code{flip2} takes a pointer $x$, whose value is
either~0 or~1 and changes its value to the opposite, twice, via the
\emph{atomic} operation \code{flip}, returning the sum of the previous
values. Assuming that $x$ is being modified only by the calls to
\code{flip2}, what is the outcome $r$ of the following program?
%
\[
\tag{\arabic{tags}}\refstepcounter{tags}\label{tag:flip2} 
r := \esc{flip2}~x;
\]
%
The answer depends on the presence or absence of interfering threads
that invoke \code{flip2} concurrently with the
program~\eqref{tag:flip2}. Indeed, in the absence of interference,
\code{flip2} will flip the value of $x$ twice, returning the sum of 0
and 1, \ie,~1. However, in the presence of other threads calling
\code{flip2} in parallel, the value of $r$ may vary from 0 to 2.
%
% depending on the concurrent calls to \code{flip}, changing the value
% of~$x$.

What are the intrinsic properties of \code{flip2} to be specified?
%
Since the effect of \code{flip2} is distributed between two internal
calls to \code{flip}, both subject to interference, the specification
should capture that the variation in \code{flip2}'s result is
subject to interference.
%
% \an{The above paragraph promises to say what an adequate spec for
% flip2 is, but doesn't do it. It just says what such a spec should
% enable infering. Either weaken the first sentence to set up the
% right expectations, or strenghten the rest, to deliver the right
% spec.}
%
% \is{Done}
%
% \ab{Shortened.}
%
Furthermore, the specification should be expressive enough to allow
reasoning under bounded interference. For example, in the absence of
interference from any other threads besides $T_1$ and $T_2$ that
invoke \code{flip2} concurrently, the program below will always result
in~$r = 2$:
%
% \ab{Rephrased.}
% \is{Looks good}
%
\[
\tag{\arabic{tags}}\refstepcounter{tags}\label{tag:flip3} 
\begin{array}{c@{\ }c@{\ }c@{\ }l@{\ }l} 
  \boxed{T_1} & & \boxed{T_2}
  \\[5pt] 
  r_1 := \esc{flip2}~x & || & r_2 := \esc{flip2}~x; \\[3pt]
  \multicolumn{3}{c}{r :=  r_1 + r_2}
\end{array}
\]

\subsection{Abstract Histories of Non-linearizable Objects}
\label{sec:hist}

Execution histories capture the traces of a concurrent object's
interaction with various threads, and are a central notion for
specifying concurrent data structures.
%
For example, linearizability specifies the behavior of an object by
mapping the object's global history of method invocations and returns
to a sequence of operations that can be observed when the object is
used sequentially~\cite{Herlihy-Wing:TOPLAS90}. However, as we have
shown, neither \code{exchange} nor \code{flip2} can be understood in
terms of sequential executions.

We propose to specify the behavior and outcome of such objects in
terms of \emph{abstract concurrent histories}, as follows. Instead of
tracking method invocations and returns, our histories track the
``interesting'' changes to the object's state. What is ``interesting''
is determined by the user, depending on the intended clients of
the concurrent object.
%
% \ab{What's interesting state?}  
%
% \is{Hmm... Do we really have to be more formal here? I tried to
% clarify}
%
%to particular threads. 
%
Moreover, our specifications are subjective (\ie, thread-relative) in
the following sense. Our histories do not identify threads by their
thread IDs. Instead, each method is specified by relating two
different history variables: the history of the invoking thread
(aka.~\emph{self}-history), and the history of its concurrent
environment (aka.~\emph{other}-history). In each thread, these two
variables have different values.
%
%This subjective view will allow us to easily specify programs with
%dynamically-changing number of threads.

%This will also allow us to relate the results of concurrent calls of
%object's methods to the history contributions, made by a thread and
%its interference, correspondingly.
%
% \ab{I am finding it difficult to parse this sentence.}
%
% \is{Rephrased}

For example, in the case of the exchanger, the interesting changes to
the object's state are the exchanges themselves. Thus, the global
history $\hist_{\cal E}$ tracks the successful exchanges in the form
of pairs of values, as shown in below:
%
\[
%\tag{\arabic{tags}}\refstepcounter{tags}\label{tag:hist1} 
\!\!\!\begin{array}{r@{\ }c@{}c@{\ }c@{\ }c@{\ }c@{\ }c@{\ }c@{\ }c@{}l}
  & & \boxed{T_1} & \graybox{\boxed{T_2}} & \graybox{\boxed{T_2}} & \graybox{\boxed{T_3}} & \graybox{\boxed{T_2}}
  & \boxed{T_1} 
  \\[5pt]  
  \hist_{\cal E} \!=\! \text{[} &\!\!...,~& (1, 2), & \graybox{(2, 1)}, & 
                                 \graybox{(4, 5)}, & \graybox{(5, 4)}, &  \graybox{(9, 8)}, & (8, 9), &
                                                                          ...
                     &
                       \text{]}   
\\[-5pt]   
&& \multicolumn{2}{l}{\underbracket{\phantom{aaaaaaaaa}}_{\mathtt{exchange~ok}}} & \multicolumn{2}{c}{\underbracket{\phantom{aaaaaaaaa}}_{\mathtt{exchange~ok}}} & \multicolumn{2}{c}{\underbracket{\phantom{aaaaaaaaa}}_{\mathtt{exchange~ok}}}
                  & & 
\end{array}
\]
%
%
The diagram presents the history from the viewpoint of thread
$T_1$. The exchanges made by $T_1$ are colored white, determining the
\emph{self}-history of $T_1$. The gray parts are the exchanges made by
the other threads (e.g., $T_2$, $T_3$, \etc.), and determine the
\emph{other}-history for $T_1$. 

% \an{We need here a diagram that colors the same total history, but
%   from the point of view of $T_2$.}

% \is{No, I think, we don't...}

The subjective division between \emph{self} and \emph{other} histories
emphasizes that a successful exchange is actually represented by
\emph{two} pairs of numbers $(x, y)$ and $(y, x)$, that appear
consecutively in $\hist_{\cal E}$, and encode the two ends of an
exchange from the viewpoint of the exchanging threads. We call such
pairs \emph{twins}.
%
%and will show in Section~\ref{sec:exchanger} how to encode that two
%pairs are twins.
As an illustration, the white entry $(2, 1)$ from the self-history of
$T_1$, is matched by a twin gray entry $(1, 2)$ from the other-history
of $T_1$, encoding that $T_1$ exchanging $2$ for $1$ corresponds to
$T_1$'s environment exchanging $1$ for $2$.

The subjective division is important, because it will enable us to
specify threads \emph{locally}, \ie, without referring to the code of
other threads. For example, in the case of program~(\ref{tag:ex1}), we
will specify that $T_1$, in the case of a successful exchange, adds a
pair $(1, r_1)$ to its self history, where $\esc{Some}~r_1$ is $T_1$'s
return value. Similarly, $T_2$ adds a pair $(2, r_2)$ to
its self history, where $\esc{Some}~r_2$ is $T_2$'s return value.
%
% \an{Fix the option values above please.}

On the other hand, it is an important invariant of the exchanger
object---but not of any individual thread---that twin entries are
symmetric pairs encoding different viewpoints of the one-and-the-same
exchange. This object invariant will allow us to reason about clients
containing combinations of exchanging threads. Taking
program~(\ref{tag:ex1}) as an example again, the object invariant will
imply of the individual specifications of $T_1$ and $T_2$, that $r_1$
must equal $2$, and $r_2$ must equal $1$, if no threads interfered
with $T_1$ and $T_2$.

%We emphasize here that the object invariants are, as the name
%suggests, object-specific. Our Hoare-style approach will allow the
%user to declare the above invariant in the case of the exchanger, but
%will not mandate it for examples for which it is not relevant. This is
%in contrast to the specification by correctness conditions, which are
%typically not parametrizable by user-defined properties, and are thus
%less flexible (\eg, linearizability does not allow users to declare
%history invariants on a per-object basis).
%\an{Is the above paragraph too strong?}

%This invariant is what makes possible the verification of the
%example~\eqref{tag:ex1} with respect to the
%assertion~\eqref{tag:exsp}.

%Indeed, in the case of a successful exchange, each of the threads
%$T_1$ and $T_2$, can now attribute to itself an entry $(x, r)$ in the
%total exchanger history $\hist_{\cal E}$, where $x$ is a
%\emph{self}-contributed value and $r$ is a thread-specific result of
%calling \code{exchange}, corresponding to a value contributed by some
%\emph{other} thread.
%%
%Furthermore, the ``closed-world'' assumption, stating the absence of
%external interference, allows us to deduce that, in the case of
%success, the \emph{overall} interaction history of the exchanger for
%the time span of the program~\eqref{tag:ex1}, consists only of the two
%``twin'' entries, $(1, 2)$ and $(2, 1)$, which correspond to a single
%exchange transaction. Hence, by the history invariant,
%$r_1 = \esc{Some}~2$ and $r_2 = \esc{Some}~1$.

We can similarly employ abstract histories to specify
\code{flip2}. One way to do it is to notice that the value of the
shared counter $x$ will be changing as $0, 1, 0, 1, \ldots$, and
exactly \emph{two} of these values will be contributed by each call to
\code{flip2} made by some thread. We can depict a particular total
history $\hist_{\cal F}$ of the \code{flip2} structure as follows:
%
\[
%\tag{\arabic{tags}}\refstepcounter{tags}\label{tag:hist2}
%
\!\!\!\!\!\!\!\!
%
\begin{array}{r@{\ }c@{\ }c@{\ }c@{\ }c@{\ }c@{\ }c@{\ }c@{\ }c@{\ }l@{\ }}
  & & \boxed{T_1} & \graybox{\boxed{T_2}} & {\boxed{T_1}} & \graybox{\boxed{T_2}} & \graybox{\boxed{T_3}}
  & \graybox{\boxed{T_3}} 
  \\[5pt] 
  \hist_{\cal F} = \text{[} & ..., & 1, & \graybox{0}, & 
                                 1, & \graybox{0}, &  \graybox{1}, & \graybox{0}, &
                                                                          \ldots
                     &
                       \text{]}   
\\[-5pt] 
&& \multicolumn{3}{l}{{\underbracket{\phantom{aaaaaa}}_{T_1.\mathtt{flip2}}}}
\end{array}
\]
%
The two ``white'' contributions are made by thread $T_1$'s call to
\code{flip2}, while the rest (gray) are contributions by $T_1$'s
environment. Since the atomic \code{flip} operation returns the
\emph{complementary} (\ie, previous) value of the counter, the overall
result of $T_1$'s call in this case is $\bar{1} + \bar{1} = 0 + 0
=~0$.

The invariant for the \code{flip2} structure postulates the
interleaving 0/1-shape of the history and also ensures that the last
history entry is $x$'s current value. This will allow us to reason
about clients of \code{flip2}, such as~\eqref{tag:flip3}.
%
% \an{We're not specifying this shortly, so better say here what the
%   invariant is, and how it helps with deducing anything in the absence
%   of interference.}
%
In the absence of interference, we can deduce that the two parallel
calls to \code{flip2} have contributed four \emph{consecutive} entries
to the history $\hist_{\cal F}$, with each thread contributing
precisely two of them. For each of the two calls, the result equals
the sum of the two complementary values for what the corresponding
thread has contributed to the history, hence, the overall sum
$r_1 + r_2$ is $2$.

% \paragraph{Histories as subjective auxiliary state}

\subsection{Hoare-style Specifications for \texttt{exchange} and
  \texttt{flip2}}
\label{sec:hoare}

The above examples illustrate that subjectivity and object invariants
are two sides of the same coin. In tandem, they allow us to specify
threads individually, but also reason about thread combinations. We
emphasize that in our approach, the invariants are
\emph{object-specific} and \emph{provided by the user}. For example,
we can associate the invariant about twin entries with the exchanger
structure, but our method will not mandate the same invariant for
other structures for which it is not relevant. This is in contrast to
using a fixed correctness condition, such as linearizability, QC, or
CAL, which cannot be parametrized by user-defined
properties.\footnote{For example, linearizability does not allow users
  to declare history invariants on a per-object basis. The exchanger
  example motivated the introduction of the correctness condition
  CAL~\cite{Hemed-Rinetzky:PODC14}, which relaxes linearizability, and
  makes it somewhat more general in this respect, but still falls
  short of admitting user-defined invariants. \texttt{flip2} can be
  specified using a variation of QC~\cite{Jagadeesan-Riely:ICALP14},
  but we show that a similar property can be expressed via
  subjectivity and a user-defined invariant.}

%suggests, object-specific. Our Hoare-style approach will allow the
%user to declare the above invariant in the case of the exchanger, but
%will not mandate it for examples for which it is not relevant. This is
%in contrast to the specification by correctness conditions, which are
%typically not parametrizable by user-defined properties, and are thus
%less flexible (\eg, linearizability does not allow users to declare
%history invariants on a per-object basis).
%\an{Is the above paragraph too strong?}


%main idea demonstrated by the two examples above is to specify the
%effects of a thread's interaction with a concurrent object via its
%contributions to the abstract history, subject to a
%\emph{object-specific} invariant, instead of using a fixed correctness
%condition, such as linearizability, QC or CAL.
%\footnote{The exchanger
%  example motivated the introduction of the correctness condition
%  CAL~\cite{Hemed-Rinetzky:PODC14}, strengthening the classical notion
%  of linearizability, and \texttt{flip2} can be specified using a
%  variation of QC~\cite{Jagadeesan-Riely:ICALP14}.}

Subjective histories can be encoded in our approach as \emph{auxiliary
  state}~\cite{Sergey-al:ESOP15,Owicki-Gries:CACM76}. Our Hoare
triples will specify how programs modify their histories, while the
invariants are declared as properties of a chunk of shared state
(\eg,~resource invariants of~\cite{Owicki-Gries:CACM76}). With the two
components, we will be able to describe the effects and results of
programs \emph{declaratively}, \ie, without exposing program
implementations.

%possible to specify the behavior quantify over history components in Hoare triples, and
%ascribed to the corresponding programs.
%%
%Therefore, treating the (auxiliary) histories as a part of the program
%state, we can provide FCSL-style subjective (\ie, {thread-local})
%Hoare-style specifications to our example programs. By doing so, we
%manage to describe the programs' effects and results
%\emph{declaratively}, \ie, without exposing their actual
%implementations.


% Recent work on Hoare-style specification of concurrent data
% structures~\cite{Sergey-al:ESOP15} has demonstrated how to encode
% observable execution histories as a form of auxiliary
% state~\cite{Owicki-Gries:CACM76} and constrain their object-specific
% invariants. As the work~\cite{Sergey-al:ESOP15} focused solely on
% linearizable objects, it did not consider the objects, whose behavior,
% observed by a thread, inherently depends on the effects of the
% thread's \emph{environment}. We will now show how the very same
% specification approach can be employed to give concurrent Hoare-style
% specification to our non-linearizable examples.

% The typical pattern in specifying concurrent programs with auxiliary
% histories is to take \emph{subjective}~\cite{LeyWild-Nanevski:POPL13}
% (\ie, thread-local) point of view and consider the object-specific
% history $\hist$ as a disjoint union $\hist = \hists \hunion \histo$,
% where $\hists$ is a part of the history contributed by the thread
% being executed (dubbed \emph{self}-history, white parts of
% histories~\eqref{tag:hist1} and~\eqref{tag:hist2}) and $\histo$ is a
% cumulative contribution of its environment (\emph{other}-history, gray
% parts of histories~\eqref{tag:hist1} and~\eqref{tag:hist2}). The
% self/other dichotomy allows to make the resoning independed of the
% shape of thread-forking pattern, avoid using explicit thread IDs in
% the specifications, and also provides a uniform way to restrict the
% interference when reasoning about the clients.

% We can now specify the exchange procedure 


% We now give the simplified specifications to \code{exchange} and
% \code{flip2}, describing their contributions to the corresponding
% abstract histories

\begin{figure*}
\centering
{\small{
\[
\begin{tabular}{r@{\ \ \ \ \ \ \ \ }c || c}
\Num{1} & \multicolumn{2}{c}{\specK{\{\hist_{\cal F} = \emptyset,  \hist_{\cal E} = \emptyset\}}}
\\[3pt]
\Num{2} & \specK{\{\hist_{\cal F} = [\graybox{\ldots}] \}}
&
\specK{\{\hist_{\cal F} = [\graybox{\ldots}] \}}
\\[2pt]
\Num{3} & $r_1 := \esc{flip2}~x$ & $r_2 := \esc{flip2}~x$
\\[2pt]
\Num{4} & \spec{
  \begin{array}{c}
    \exists a~b,
   \hist_{\cal F} = [\graybox{\ldots}, a, \graybox{\ldots}, b,
    \graybox{\ldots}], 
    r_1 := \bar{a} + \bar{b}
  \end{array}
}
&
\spec{
  \begin{array}{c}
    \exists c~d,
   \hist_{\cal F} = [\graybox{\ldots}, c, \graybox{\ldots}, d,
    \graybox{\ldots}], 
    r_2 := \bar{c} + \bar{d}
  \end{array}
}
\\ & \multicolumn{2}{c}{}\\[-5pt]
\Num{5} & \multicolumn{2}{c}{\spec{\hist_{\cal F} = \text{perm}(a, b, c, d) = [1, 0,
  1, 0], r_1 = \bar{a} + \bar{b}, r_2 = \bar{c} + \bar{d}}}
\\[2pt]
\Num{6} & \multicolumn{2}{c}{\spec{r_1 + r_2 = 2}}
% End of the first parallel composition
\\ & \multicolumn{2}{c}{}\\[-5pt]
%
%
\Num{7} &
\specK{\{\hist_{\cal E} = [\graybox{\ldots}] \}} & 
\specK{\{\hist_{\cal E} = [\graybox{\ldots}] \}}
\\[2pt]
\Num{8} & $s_1 := \esc{exchange}~r_1$ & $s_2 := \esc{exchange}~r_2$
\\[2pt]
\Num{9} & 
  \spec{\!\!
  \begin{array}{l}
    \mathsf{if}\ s_1\ \mathsf{is}\ \mathsf{Some}\ v_1\ \mathsf{then}~\\[1pt]
    \hist_{\cal E} = [\graybox{\ldots}, (r_1, v_1), \graybox{\ldots}]
    ~\mathsf{else}\ \hist_{\cal E} = [\graybox{\ldots}]
  \end{array}
  \!\!}
&
  \spec{\!\!
  \begin{array}{l}
    \mathsf{if}\ s_2\ \mathsf{is}\ \mathsf{Some}\ v_2\ \mathsf{then}~\\[1pt]
    \hist_{\cal E} = [\graybox{\ldots}, (r_2, v_2), \graybox{\ldots}]
    ~\mathsf{else}\ \hist_{\cal E} = [\graybox{\ldots}]
  \end{array}
  \!\!}%
%
\\ & \multicolumn{2}{c}{}\\[-5pt]
\Num{10} &
\multicolumn{2}{c}{
\spec{s_1 = \mathsf{Some}\ v_2 \wedge s_2 = \mathsf{Some}\ v_2
  \implies
\hist_{\cal E} = \text{perm}((r_1, v_1), (r_2, v_2)) = \text{perm}((v_1, r_1),
  (v_2, r_2))
}}
\\[2pt]
\Num{11} &
\multicolumn{2}{c}{
\spec{s_1 = \mathsf{Some}\ v_2 \wedge s_2 = \mathsf{Some}\ v_2
  \implies
v_1 = r_2 \wedge v_2 = r_1
}}
\\[2pt]
\Num{12} &
\multicolumn{2}{l}
{$\kw{if}~s_1~\kw{is}~\esc{Some}~v_1~\kw{and}~s_2~\kw{is}~\esc{Some}~v_2~\kw{then}$}
\\[2pt]
\Num{13} &
\multicolumn{2}{l}
{\spec{v_1 = r_2,  v_2 = r_1, r_1 + r_2 = 2}}
\\[2pt]
\Num{14} &
\multicolumn{2}{l}
{$t := v_1 + v_2$~~~{\spec{t = 2}}~~~$ \kw{else}~t := 2$~~~{\spec{t = 2}}}
% \\[2pt]
% \Num{15} & 
% \multicolumn{2}{l}
% {$ \kw{else}~t := 2$~~~~~~{\spec{t = 2}}}
\end{tabular} 
\]
}}  
\caption{Verification of a concurrent client program using
  \code{exchange} and \code{flip2} in the absence of external
  interference.}
\label{fig:verif1}
\end{figure*}


A semi-formal and partial spec of \code{exchange} looks as follows,
with the white/gray parts denoting \emph{self}/\emph{other}
contributions to history, from the point of view of the thread being
specified (we postpone the full presentation until
Section~\ref{sec:exchanger}):
%
% \an{Remove the twin $(w, v)$ from the equation below.}
%
{\small{
\[
\tag{\arabic{tags}}\refstepcounter{tags}\label{tag:exsimpl} 
{\small{
\begin{array}{c}
\specK{\{\hist_{\cal E} = [\graybox{\ldots}] \}}\\[2pt]
\esc{exchange}\ v\\[2pt]
  \spec{\!\!
  \begin{array}{c}
    \mathsf{if}\ \res\ \mathsf{is}\ \mathsf{Some}\ w\ \mathsf{then}~\\[1pt]
    \hist_{\cal E} = [\graybox{\ldots}, (v, w), \graybox{\ldots}]
    ~\mathsf{else}\ \hist_{\cal E} = [\graybox{\ldots}]
  \end{array}
  \!\!}
\end{array}
}}
\]
}}
%
\hspace{-5pt}
%
The ellipsis ($\ldots$) stands for an existentially-quantified chunk
of the history.
%
%
%By employing the coloring scheme to indicate the \emph{disjoint}
%contributions made by this thread (\emph{self}) and by its environment
%(\emph{other}), we avoid using thread IDs in the spec, making it
%agnostic to the shape of the concurrent context with dynamic forking.
%
%
%
The spec~\eqref{tag:exsimpl} says that a successful exchange adds an
entry $(v, w)$ to the \emph{self}-history (hence, the entry is
white). In the case of failed exchange, no entry is added. In the
complete and formal specification in Section~\ref{sec:exchanger}, we
will have to add a timing aspect, and say that the new entry appears
\emph{after} all the history entries from the precondition. We will
also have to say that no entries are removed from the \emph{other}
history (\ie, the exchanger cannot erase the behavior of other
threads), but we elide those details here.


%the call to \code{exchange} result
%in the case of succesff $\res$ of calling \code{exchange} to the
%self-contribution made to the final history $\hist_{\cal E}$ by this
%thread. The spec also \emph{explicitly} mentions the environment's
%contribution $\graybox{$(w, v)$}$ in the successful branch of the
%postcondition and accounts for possible effects of the interference by
%adding trailing $\graybox{\ldots}$ to $\hist_{\cal E}$.

%
%The spec~\eqref{tag:exsimpl} relates the result $\res$ of calling
%\code{exchange} to the self-contribution made to the final history
%$\hist_{\cal E}$ by this thread. The spec also \emph{explicitly} mentions the
%environment's contribution $\graybox{$(w, v)$}$ in the successful
%branch of the postcondition and accounts for possible effects of the
%interference by adding trailing $\graybox{\ldots}$ to $\hist_{\cal E}$.
%%
%For simplicity of presentation, we omit the imposed invariant on the
%history $\hist_{\cal E}$ from the spec, and consider equal histories
%$\hist_{\cal E}$ that are isomorphic up to permutations of subsequent
%``twin'' contributions $(v, w), (w, v)$.\footnote{There will be no
%  such ambiguity in a more rigorous encoding of history entries via
%  timestamps, which we present in Section~\ref{sec:exchanger}.}
%

The spec of \code{flip2} is defined with respect to history
$\hist_{\cal F}$:
%
\[
\tag{\arabic{tags}}\refstepcounter{tags}\label{tag:flipsimpl} 
{\small{
\begin{array}{c}
\specK{\{\hist_{\cal F} = [\graybox{\ldots}] \}}\\[2pt]
\esc{flip2}\ x\\[2pt]
  \spec{\!\!
  \begin{array}{c}
   \exists a~b, \hist_{\cal F} = [\graybox{\ldots}, a, \graybox{\ldots}, b,
    \graybox{\ldots}], \res = \bar{a} + \bar{b}
  \end{array}
  \!\!}
\end{array}
}}
\]
%
It says that the return value $\res$ is equal to the sum of binary
complements $\bar{a} + \bar{b}$ for the thread's two separate
\emph{self}-contributions to the history. Due to the effects of the
interference, the history entries $a$ and $b$ may be separated in the
overall history by the contributions of the environment, as indicated
by \graybox{\ldots} between them.

\subsection{Using Subjective Specifications in the Client Code}
\label{sec:clients}

The immediate benefit of using Hoare logic is that one can easily
reason about programs whose components use different object
invariants, whereas there is not much one can say about programs whose
components require different correctness conditions.
%
For example, Figure~\ref{fig:verif1} shows a proof sketch for a toy
program that uses both \code{exchange} and \code{flip2}.  As each of
these methods requires its own auxiliary history variable
($\hist_{\cal E}$ for the exchanger, and $\hist_{\cal F}$ for
\code{flip2}), the combined program uses both, but the proof simply
ignores those histories that are not relevant for any specific method
(\ie, we can ``frame'' the specs~\eqref{tag:exsimpl}
and~\eqref{tag:flipsimpl} wrt.~the histories of the objects that they
do not depend upon).

%
%To give an example of compositional reasoning with our Hoare-style
%specs, we focus on verification of a concurrent program, using both
%\code{exchange} and \code{flip2}. The program code and the proof
%outline are presented in Figure~\ref{fig:verif1}.

The program first forks two instances of \code{flip2}, storing the
results in $r_1$ and $r_2$ (line~4). Next, two new threads are forked,
trying to exchange $r_1$ and $r_2$ (line~8). The conditional (line~12)
checks if the exchange was successful, and if so, assigns the sum of
exchanged values to $t$ (line~14); otherwise $t$ gets assigned 2. We
want to prove via the specs~\eqref{tag:exsimpl}
and~\eqref{tag:flipsimpl}, that in the absence of external
interference on the \code{flip2}'s pointer $x$ and the exchanger, the
outcome is always $t = 2$.

\paragraph{Explaining the verification}

In addition to the absence of external interference, we assume that
the initial value of $x$ is $0$, and the initial \emph{self}-histories
for both \code{flip2} and \code{exchange} are empty (line~1).
%
% \an{There was a footnote here about framing, but I think it's too
%   detailed for being here. Moreover, that framing may be confused with
%   the framing of concurroids, which we called injection in previous
%   work. Let's see if shephard calls for an explanation.}
%
%\footnote{These restrictions
%  can be lifted via the FCSL framing
%  rule~\cite{Nanevski-al:ESOP14}.}
%
%Therefore, both initial histories $\hist_{\cal F}$ and $\hist_{\cal F}$ are empty
%(line~1).
%
%
Once the \code{flip2} threads are forked, we employ
spec~\eqref{tag:flipsimpl} for each of them, simply ignoring (i.e.,
framing out) $\hist_{\cal E}$, as this history variable does not apply
to them \code{flip2}. Upon finishing, the postconditions of
\code{flip2} in line~4 capture the relationship between the
contributions to the history $\hist_{\cal F}$ and the results $r_1$
and $r_2$ of the two calls.

Both postconditions in line~4 talk about the very same history
$\hist_{\cal F}$, just using different colors to express that the
contributions of the two threads are \emph{disjoint}: $a$ and $b$
being white in the left thread, implies that $a$ and $b$ are history
entries added by the left thread. Thus, they \emph{must} be gray in
the right thread, as they cannot overlap with the entries contributed
by the right thread. The right thread cannot explicitly specify in its
postcondition that $a$ and $b$ are gray, since the right thread is
unaware of the specific contributions of the left thread.
%
% \is{This is redundant}
% But, such inference will be enabled by the FCSL rule for
% parallel composition, as we explain in Section~\ref{sec:background}.

Dually, $c$ and $d$ being white in the right thread in line~4, implies
that they must be gray on the left. Thus, overall, in line~5, we know
that $\hist_{\cal F}$ contains all four entries in some permutation,
and in the absence of intereference, it contains no other entries but
these four. From the object invariant on $\hist_{\cal F}$ it then
follows that the entries are some permutation of $[1,0,1,0]$, which
makes their sum total $r_1 + r_2 = 2$. 
%
% \an{Make a reference to this object invariant, once it is added in the
%   previous section.}

%threads, the subjective assertions in line~4 of the proof outline
%constrain, in fact, the \emph{very same} history $\hist_{\cal F}$,
%specifying its disjoint subparts.
%%
%Upon joining the first pair of threads, we reconcile two thread-local
%views to $\hist_{\cal F}$ (line~5), inferring that $\hist_{\cal F}$ is a permutation
%of $(0, 1, 0, 1)$, where each of the four components correspond to one
%of the contributions made in parallel.
%%
%The assumption about absence of external interference is crucial, as
%it makes it possible to ensure that the \emph{entire} history
%$\hist_{\cal F}$ consists only of the two thread's contributions.
%%
%Therefore, the sum $r_1 + r_2$ is the sum of complements to the
%history entries, \ie,~$2$.

Similarly, we ignore $\hist_{\cal F}$ while reasoning about calls to
\code{exchange} via spec~\eqref{tag:exsimpl} (lines~7 and 9). As
before, we know that the entry $(r_1, v_1)$, which is white in the
left postcondition in line~9, must be gray on the right, and dually
for $(r_2, v_2)$. In total, the history $\hist_{\cal E}$ must contain
both of the entries, but, by the invariant, it must also contain their
twins. In the absence of any other interference, it therefore must be
that $(r_1, v_1)$ is a twin for $(r_2, v_2)$, \ie, $r_1 = v_2$ and
$r_2 = v_1$, as line~11 expresses for the case of a succesful
exchange.
%
% \an{It seems to me that the figure should say this by using
%   $\hist_{\cal E} = perm ((r_1, v_1), (r_2, v_2)) = perm ((v_1, r_1),
%   (v_2, r_2))$, rather than the current $\hist_{\cal E} = perm ((r_2,
%   v_2), (v_2, r_2)) = perm ((r_1, v_1), (v_1, r_1))$.}  
%
The rest of the proof is then trivial.

The sketch relied on several important aspects of program verification
in FCSL: \emph{(i)} the invariants constraining $\hist_{\cal F}$ and
$\hist_{\cal E}$ were preserved by the methods, \emph{(ii)} upon
joining the threads, we can rely on the disjointness of history
contributions of the two threads, in order to combine the thread-local
views into a specification of the parent thread, and, \emph{(iii)} we
could guarantee the absence of the external interference.

The aspect \emph{(i)} is a significant component of what it means to
specify and verify a concurrent object. As we will show in
Sections~\ref{sec:exchanger} and~\ref{sec:counting}, defining a
sufficiently strong object invariant, and then proving that it is
indeed an invariant, \ie, that it is preserved by the implementation
of the program, is a major part of the verification challenge.
%
We will explain FCSL rules for \emph{parallel composition} and
\emph{hiding} in Section~\ref{sec:background}, justifying the
reasoning principles~\emph{(ii)} and \emph{(iii)}.

\subsection{Specifying Non-linearizable Objects in Three Steps}
\label{sec:three-steps-reas}

As shown by Sections~\ref{sec:hist}--\ref{sec:clients}, our method for
specifying and verifying non-linearizable concurrent objects and their
clients boils down to the following three systematic steps.

\paragraph{Step 1 (\S\ref{sec:hist}):} 

\emph{Define object-specific auxiliary state and its invariants. The
  auxiliary state will typically include a specific notion of abstract
  histories, recording whatever behavior is perceived as essential by
  the implementor of the object}.
%
%The concurrent object implementor should first define the shape and
%key invariants of the object's auxiliary state, in particular, of its
%histories.
%
To account for the variety of object-specific correctness conditions,
we do not fix a specific shape for the histories. We do not restrict
them to always record pairs of numbers (as in the exchanger), or
record single numbers (as in \code{flip2}). The only requirement that
we impose on auxiliary state in general, and on histories in
particular, is that the chosen type of auxiliary state is an instance
of the PCM algebraic structure~\cite{Sergey-al:ESOP15}, thus providing
an abstract, and user-defined, notion of \emph{disjointness} between
\emph{self}/\emph{other} contributions.

\paragraph{Step 2 (\S\ref{sec:hoare}):} 

\emph{Formulate Hoare-style specifications, para-metrized by
  interference, and verify them}.
%
This step provides a suitable ``interface'' for the methods of the
concurrent object, which the clients use to reason, without
knowing the details of the object and method implementations.
%
Naturally, the interface can refer to the auxiliary state and
histories defined in the previous step.
%
When dealing with non-linearizable objects in FCSL, it is customary to
formulate the spec in a subjective way (\ie, using
\emph{self}/\emph{other}, dually white/gray division between history
entries) so that the specification has a way to refer to the effects
of the interfering calls to the same object. 
%
The amount of interference can be later instantiated with more
specific information, once we know more about the context of
concurrent threads in which the specified program is being run.

\paragraph{Step 3 (\S\ref{sec:clients}):} 

\emph{Restrict the interference when using object specs for
  verification of clients}.
%
Eventually, thread-local knowledge about effects of individual clients
of one and the same object, should be combined into a cumulative
knowledge about the effect of the composition.
%
To measure this effect, one usually considers the object in a
\emph{quiescent} (interference-free) moment~\cite{Rinard:RACES}.
%
%\ab{Not clear what quiescent means. Also provide a citation?} 
%
To model quiescent situations, FCSL provides a program-level
constructor for \emph{hiding}. In particular, $\esc{hide}\ e$ executes
$e$, but statically prevents other threads from interfering with $e$,
by making $e$'s auxiliary history invisible. Program $e$'s
\emph{other} contribution is fixed to be empty, thus modeling
quiescence.

\section{Verifying the Exchanger Implementation}
\label{sec:exchanger}

We now proceed with more rigorous development of the invariants and
specification for the exchanger data structure, necessary to verify
its real-world implementation~\cite{ExchangerClass}, which was so far
elided from the overview of the approach.

% This example previously motivated the introduction of the correctness
% condition CAL~\cite{Hemed-Rinetzky:PODC14}. The exchanger is not
% linearizable to a set of sequential histories, as its behavior
% crucially depends on interaction with interfering threads.
% %
% CAL works around this problem by employing new
% \emph{concurrency-aware} (CA) histories as a specification set.


\newcommand{\Unmatched}{{\mathsf{U}}}
\newcommand{\Matched}[1]{{\mathsf{M}\ #1}}
\newcommand{\Retired}{{\mathsf{R}}}


The exchanger implementation is presented in ML-style pseudo-code in
Figure~\ref{fig:exchanger}. It takes a value $v\,{:}\,A$ and creates
an \emph{offer} from it (line 2). An offer is a pointer $p$ to two
consecutive locations in the heap.\footnote{In our mechanization, we
  simplify a bit by making $p$ point to a pair instead.}
%\footnote{In our mechanization, we
%  use a pointer to a pair, but for presentation here adopt a more
%  common style of consecutive heap locations.} 
%
The first location stores $v$, and the second is a ``hole'' which the
interfering thread tries to fill with a matching value. The hole is
drawn from the type
$\esc{hole}\,{=}\,\Unmatched\,{\mid}\,\Retired\,{\mid}\,\Matched
w$. Constructor $\Unmatched$ signals that the offer is unmatched;
$\Retired$ that the exchanger retired (\ie, withdrew) the offer, and
does not expect any matches on it; and $\Matched w$ that the offer has
been matched with a value $w$.

The global pointer $g$ stores the latest offer proposed for
matching. The exchanger proposes $p$ for matching by making $g$ point
to $p$ via the atomic compare-and-set instruction \code{CAS} (line
3). We assume that \code{CAS} returns the value read, which can be
used to determine if it failed or succeeded. If \code{CAS} succeeds,
exchanger waits a bit, then checks if the offer has been matched by
some $w$ (lines 6, 7). If so, $\esc{Some}\ w$ is returned (line
7). Otherwise, the offer is retired by storing $\Retired$ into its
hole (line 6). Retired offers remain allocated (thus, exchanger has a
memory leak) in order to avoid the ABA problem, as usual in many
concurrent structures~\cite{Herlihy-Shavit:08,Treiber:TR}.
%
If the exchanger fails to link $p$ into $g$ in line 3, it deallocates
the offer $p$ (line 10), and instead tries to match the offer $cur$
that is current in $g$. If no offer is current, perhaps because
another thread already matched the offer that made the \code{CAS} in
line 3 fail, the exchanger returns $\mathsf{None}$ (line
12). Otherwise, the exchanger tries to make a match, by changing the
hole of $cur$ into $\Matched v$ (line 14). If successful (line 16), it
reads the value $w$ stored in $cur$ that was initially proposed for
matching, and returns it. In any case, it unlinks $cur$ from $g$ (line
15) to make space for other offers.

{
%\setlength{\belowcaptionskip}{-10pt} 
\begin{figure}
\centering
\[
{\small{
\begin{array}{rl}
 \Num{1} & \esc{exchange}~(v : A) : \esc{option}~A~=~\{ 
\\ 
 \Num{2} & ~~~~ p \Asgn \esc{alloc}~(v, \Unmatched);\\
 \Num{3} & ~~~~ b \Asgn \esc{CAS}~(g, \esc{null}, p);\\
 \Num{4} & ~~~~ \kw{if}~~b~\esc{==}~\esc{null}~~\kw{then}\\
 \Num{5} & ~~~~ ~~~~ \esc{sleep}~(50);\\
 \Num{6} & ~~~~ ~~~~ x \Asgn \esc{CAS}~(p\esc{+}1, \Unmatched, \Retired);\\
 \Num{7} & ~~~~ ~~~~ \kw{if}~~x~~\kw{is}~~\Matched w~~\kw{then}~~\kw{return}~~(\esc{Some}~w)\\
 \Num{8} & ~~~~ ~~~~ \kw{else}~~\kw{return}~~\esc{None}\\
 \Num{9} & ~~~~ \kw{else}\\
\Num{10} & ~~~~ ~~~~ \esc{dealloc}~p;\\
\Num{11} & ~~~~ ~~~~ cur \Asgn \esc{read}~g;\\
\Num{12} & ~~~~ ~~~~ \kw{if}~~cur~\esc{==}~\esc{null}~~\kw{then}~~\kw{return}~{\esc{None}}\\
\Num{13} & ~~~~ ~~~~ \kw{else}\\
\Num{14} & ~~~~ ~~~~ ~~~~ x \Asgn \esc{CAS}~(cur\esc{+}1, \Unmatched, \Matched v);\\
\Num{15} & ~~~~ ~~~~ ~~~~ \esc{CAS}~(g, cur, \esc{null});\\
\Num{16} & ~~~~ ~~~~ ~~~~ \kw{if}~~x~~\kw{is}~~\Unmatched~~\kw{then}~~w\Asgn \esc{read}~cur;\kw{return}~(\esc{Some}\ w)\\
\Num{17} & ~~~~ ~~~~ ~~~~ \kw{else}~~\kw{return}~\esc{None}\}
\end{array}
}}
\]
%\vspace{-5pt}  
\caption{Elimination-based exchanger procedure.}
\label{fig:exchanger}
%\vspace{-10pt}   
\end{figure} 
}




\subsection{Step 1: Defining Auxiliary State and Invariants}

%
%The subjective division suffices to specify the exchanger in Hoare
%logic, and will be essential for client reasoning in
%Section~\ref{sec:cal}.
%
%In Section~\ref{sec:counting}, we show how the approach directly
%scales to naturally specify programs that have heretofore only been
%addressed using quiescent and quantitative quiescent consistency.

To formally specify the exchanger, we decorate it with auxiliary
state. 
%
In addition to histories, necessary for specifying the observable
behavior, the auxiliary state is used for capturing the coherence
constraints of the actual implementation, \eg, with respect to memory
allocation and management of outstanding offers.
%
The state is subjective as described in Section~\ref{sec:overview}: it
keeps thread-local auxiliary variables that name the thread's private
state (\emph{self}), but also the private state of all other threads
combined (\emph{other}).
%
% The subjective division will be essential for client reasoning in
% Section~\ref{sec:cal}.

The subjective state of the exchanger for each thread in this example
consists of three groups of two components: (1) thread-private heap
$\heaps$ of the thread, and of the environment $\heapo$, (2) a set of
outstanding offers $\perms$ created by the thread, and by the
environment $\permo$, and (3) a time-stamped history of values
$\hists$ that the thread exchanged so far, and dually $\histo$ for the
environment. In Section~\ref{sec:overview}, we illustrated
subjectivity by means of histories, white we used white and gray
entries, respectively, to describe what here we name $\hists$ and
$\histo$, respectively. Now we see that the dichotomy extends beyond
histories, and this example requires the dichotomy applied to heaps,
and to sets of offers as well. In addition to \emph{self}/\emph{other}
components of heaps, permissions and histories, we also need shared
(aka.~\emph{joint}) state consisting of two components: a heap
$\heapj$ of storing the offers that have been made, and a map
$\pending$ of offers that have been matched, but not yet collected by
the thread that made them.

Heaps, sets and histories are all PCMs under the operation of disjoint
union, with empty heap/set/history as a unit. We overload the notation
and write $x\,{\mapsto}\,v$ for a singleton heap with a pointer $x$
storing value $v$, and $t\,{\mapsto}\,a$ for a singleton
history. Similarly, we apply disjoint union $\hunion$ and subset
$\subseteq$, to all three types uniformly.

We next describe how the exchanger manipulates the above variables.
%for maps, and write $x \mapsto y$ for a singleton map, no matter
%its type. For example, in the case of time-stamped histories, $t
%\mapsto b$ denotes that at time-stamp $t$, some abstract behavior $b$
%took place. In the case of heaps, $x \mapsto v$ for a singleton heap
%with a pointer $x$ storing value $v$. We also overload the operator
%$\hunion$ for disjoint union, and apply it to histories, heaps, and
%set of offers, uniformly. Thus, each history or heap will be a
%disjoint union of a number of singleton entries, and we use $x
%\subseteq y$ to say that the map $x$ is a submap (or subheap, or
%subhistory) of $y$.
First, $\heapj$ is a heap that serves as the ``staging'' area for the
offers. It includes the global pointer $g$. Whenever a thread wants to
make an offer, it allocates a pointer $p$ in $\heaps$, and then tries
to move $p$ from $\heaps$ into $\heapj$, simultaneously linking $g$ to
$p$, via the \code{CAS} in line~3 of Figure~\ref{fig:exchanger}.

Second, $\perms$ and $\permo$ are sets of offers (hence, sets of
pointers) that determine offer ownership. A thread that has the offer
$p \in \perms$ is the one that created it, and thus has the
\emph{sole} right to retire $p$, or to collect the value that $p$ was
matched with. Upon collection or retirement, $p$ is removed from
$\perms$.

%To ensure that only one thread has such a right, one of the important
%state invariants of the exchanger is that $\perms$ and $\permo$ are
%disjoint sets.

Third, $\hists$ and $\histo$ are exchanger-specific histories, each
mapping a time-stamp (isomorphic to nats), to a pair of exchanged
values. A singleton history $t \mapsto (v, w)$ symbolizes that a
thread having this singleton as a subcomponent of $\hists$, has
exchanged $v$ for $w$ at time $t$.
%
%Histories have been used in FCSL before~\cite{Sergey-al:ESOP15}, but
%only on linearizable examples, where each time-stamp $t$ ``stored'' an
%atomic operation performed at time $t$. In the exchanger example,
%which is not linearizable, a time-stamp $t$ stores only a \emph{half}
%of an atomic operation, \ie, a pair $(v, w)$ signaling that a thread
%exchanged $v$ for $w$. 
%
As we describe below, the most important invariant of the exchanger is
that each such singleton is matched by a ``symmetric'' one to capture
that another thread has \emph{simultaneously} exchanged $w$ for
$v$. Classical linearizability cannot express this simultaneous
behavior, making the exchanger non-linearizable.

Fourth, $\pending$ is a map storing the offers that were matched, but
not yet acknowledged and collected. Thus,
$\mathsf{dom}\ \pending = \perms \hunion \permo$. A singleton entry in
$\pending$ has the form $p \mapsto (t, v, w)$ and denotes that offer
$p$, initially storing $v$, was matched at time $t$ with $w$. A
singleton entry is entered into $\pending$ when a thread on the one
end of matching, matches $v$ with $w$. Such a thread also places the
\emph{twin} entry $\twin{t} \mapsto (w, v)$, with inverted order of
$v$ and $w$, into its own private history $\hists$, where:
%
\[
\begin{array}{c}
%\vspace{-5pt}
\twin{t} = \left\{%
\begin{array}{ll}
t+1 & \mbox{if $t$ is odd}\\
t-1 & \mbox{if $t > 0$ and $t$ is even}
\end{array}\right.  
\end{array}
\]
%
For technical reasons, $0$ is not a valid time-stamp, and has no
distinct twin. The pending entry for $p$ resides in $\pending$ until
the thread that created the offer $p$ decides to ``collect'' it. It
removes $p$ from $\pending$, and simultaneously adds the entry $t
\mapsto (v, w)$ into its own $\hists$, thereby logically completing
the exchange. Since twin time-stamps are consecutive integers, a
history cannot contain entries \emph{between} twins.

Thus, two twin entries in the combined history including $\hists$,
$\histo$ and $\pending$, jointly represent a single exchange, as if it
occurred \emph{atomically}. 
%
For example, the entries $1 \mapsto (v_1, w_1)$ and
$2 \mapsto (w_1, v_1)$ will encode the end-points of the first
exchange; the entries $3 \mapsto (v_2, w_2)$ and
$4 \mapsto (w_2, v_2)$ will encode the end-points of the second
exchange, etc., the entries at timestamps $t$ and $t+1$, for odd $t$,
will encode the end-points of the $\frac{t+1}{2}$-th exchange.
%
\emph{Concurrency-aware}
histories~\cite{Hemed-Rinetzky:PODC14} capture this by making the ends
of an exchange occur as simultaneous events. We capture it via twin
time-stamps. More formally, consider
$\hist = \hists \hunion \histo \hunion \mygather{\pending}$. Then, the
exchanger's main invariant is that $\hist$ always contains matching
twin entries:
%
%\vspace{-5pt}
\[
%\vspace{-5pt}
\tag{\arabic{tags}}\refstepcounter{tags}\label{tag:exchanging} 
t \mapsto (v, w) \subseteq \hist \iff \twin t \mapsto (w, v) \subseteq \hist
\]
%
Here $\mygather{\pending}$ is the collection of all the entries in
$\pending$. That is, $\mygather{\emptyset} = \emptyset$, and
$\mygather {p \mapsto (t, v, w) \hunion \pending'} = t \mapsto (v, w)
\hunion \mygather{\pending'}$.

In our implementation, we prove that atomic actions, such as
\code{CAS}, preserve the invariant, therefore, the whole program,
being just a composition of actions, doesn't violate~it.

%One consequence of this property is that the history $\hist$ always
%has an even number of entries. Thus, if $t$ is the smallest unused
%time-stamp in $\hist$ (not counting $0$), then $t$ must be odd, and
%hence $t < \twin t = t + 1$.

\subsection{Step 2: Hoare-style Specification of Exchanger}

We can now give the desired formal Hoare-style spec.
%
\[
\tag{\arabic{tags}}\refstepcounter{tags}\label{tag:exchangespec} 
{\small{
\begin{array}{c}
\specK{\{\heaps = \emptyset, \perms = \emptyset, \hists = \emptyset, \gist \subseteq \histo \hunion \mygather{\pending}\}}\\
\esc{exchange}\ v\\
  \spec{\!\!
  \begin{array}{c}
\heaps = \emptyset, \perms = \emptyset, \gist \subseteq
  \histo \hunion \mygather{\pending}, \hbox{}\\[1pt]
\mathsf{if}\ \res\ \mathsf{is}\ \mathsf{Some}\ w\ \mathsf{then}~\\[1pt]
\exists t\ldot \hists = t \mapsto (v, w), \mathsf{last} (\gist) < t, \twin{t}
~\mathsf{else}\ \hists = \emptyset 
  \end{array}
  \!\!}
\end{array}
}}
\]
%
The precondition says that the exchanger starts with the empty private
heap $\heaps$, set of offers $\perms$ and history $\hists$; hence by
framing, it can start with any value for these
components.\footnote{Framing in FCSL is similar to that of separation
  logic, allowing extensions to the initial state that remain
  invariant by program execution. In FCSL, however, framing applies to
  any PCM-valued state component (\eg, heaps, histories, \etc.),
  whereas in separation logic, it applies just to heaps.} The logical
variable $\gist$ names the initial history of all threads,
$\histo \hunion \mygather{\pending}$, which may grow during the call,
thus, we use subset instead of equality to make the precondition
stable under other threads adding
 entries to $\histo$ or $\pending$.

In the postcondition, the self heap $\heaps$ and the set of offers
$\perms$ didn't change. Hence, if $\mathtt{exchange}$ made an offer
during its execution, it also collected or retired it by the end.
%
The history $\gist$ is still a subset of the ending value for $\histo
\hunion \mygather{\pending}$, signifying that the environment history
only grows by interference. We will make a crucial use of this part of
the spec when verifying a client of the exchanger in
Section~\ref{sec:cal}.

If the exchange fails (\ie, $\mathsf{res}$ is $\mathsf{None}$), then
$\hists$ remains empty.  If it succeeds (either in line 7 or line 16
in Figure~\ref{fig:exchanger}), \ie, if the result $\mathsf{res}$ is
$\mathsf{Some}\ w$, then there exists a time-stamp $t$, such that
self-history $\hists$ contains the entry $t \mapsto (v, w)$,
symbolizing that $v$ and $w$ were exchanged at time $t$.

Importantly, the postcondition implies, by
invariant~(\ref{tag:exchanging}), that in the success case, the twin
entry $\twin t \mapsto (w, v)$ must belong to $\histo \hunion
\mygather{\pending}$, \ie, \emph{another} thread matched the exchange
(this was made explicit by the spec~\eqref{tag:exsimpl}).
Moreover, the exchange occurred \emph{after} the call to
$\mathsf{exchange}$: whichever $\gist$ we chose in the pre-state, both
$t$ and $\twin t$ are larger than the last time-stamp in $\gist$.

The proof outline for the exchanger is available in
Appendix~\ref{app:exch}.
%
In Section~\ref{sec:cal}, after introducing necessary FCSL background,
we will illustrate \textbf{\emph{Step 3}} of our method and show how
to employ the subjective Hoare spec~\eqref{tag:exchangespec} for
modular verification of a concurrent client.

% \is{Where is the discussed ``general pattern'', which we wanted to
%   summarize around this moment of the paper (Anindya also suggests it
%   in his very first comment)? I believe, this is a good place to put
%   it.}
% \an{It's in the second paragraph of Section 2.}


\section{Background on FCSL}
\label{sec:background}

A Hoare specification in FCSL has the form $\spec{P}\ e\ \spec{Q} @
\rcon$. $P$ and $Q$ are pre- and postcondition for partial
correctness, and $\rcon$ defines the \emph{shared resource} on which
$e$ operates. 
%
%We have elided $\rcon$ from the specs in
%Section~\ref{sec:overview}, but explain it now. 
%
The latter is a state transition system describing the state (real and
auxiliary) and atomic operations that can be invoked by the threads
that simultaneously operate on that state. We elide the transition
system aspect of resources here, and refer
to~\cite{Nanevski-al:ESOP14} for detailed treatment.

An important secondary role of a resource is to declare the variables
that $P$ and $Q$ may scope over. For example, in the case of
exchanger, we use the variables $\heaps, \perms, \hists$, $\heapo,
\permo, \histo$, and $\heapj, \pending$.
%
The mechanism by which the variables are declared is as
follows. Underneath, a resource comes with only three variables:
$a_\lcl$, $a_\env$ and $a_\joint$ standing for abstract self state,
other state, and shared (joint) state, but the user can pick their
types depending on the application. In the case of exchanger, $a_\lcl$
and $a_\env$ are triples containing a heap, an offer-set and a
history. The variables we used in Section~\ref{sec:overview} are
projections out of such triples: $a_\lcl\,{=}\,(\heaps, \perms,
\hists)$, and $a_\env\,{=}\,(\heapo, \permo, \histo)$. Similarly,
$a_\joint\,{=}\,(\heapj, \pending)$.

It is essential that $a_\lcl$ and $a_\env$ have a common type
exhibiting the algebraic structure of a PCM, under a partial binary
operation $\hunion$.
%
% In the case of $\cal E$, each of
%the three components of $a_\lcl$ and $a_\env$---heaps, offer-sets and
%histories---form a PCM, where $\bullet$ is disjoint union $\hunion$,
%and $\emptyset$ is the unit. Hence, the product of the three is a PCM
%as well, with $\bullet$ and unit defined point-wise. 
PCMs give a way, generic in $\rcon$, to define the
inference rule for parallel composition.
%
\[
\tag{\normalsize \arabic{tags}}\refstepcounter{tags}\label{eq:parrule}
{\small{
\begin{array}{c}
\specK{\{P_1\}}\ e_1\ \specK{\{Q_1\}} @ \rcon \quad \specK{\{P_2\}}\ e_2\ \specK{\{Q_2\}} @ \rcon\\[2pt]
\hline\\[-7pt]
\specK{\{P_1 \circledast P_2\}}\ e_1 \parallel e_2\ \specK{\{[\res.1/\res]Q_1 \circledast [\res.2/\res]Q_2\}} @ \rcon
\end{array}
}}
\]
%
Here, $\circledast$ is defined as follows.
\[
%\tag{\normalsize \arabic{tags}}\refstepcounter{tags}\label{eq:ssep}
\begin{array}{c}
(P_1 \circledast P_2)(a_\lcl, a_\joint, a_\env) \iff \exists x_1~x_2\ldot a_\lcl = x_1 \hunion x_2, \hbox{}\\
 P_1 (x_1, a_\joint, x_2 \hunion a_\env), P_2 (x_2, a_\joint, x_1 \hunion a_\env)
\end{array}
\]
%
%
Thereby, when a parent thread forks $e_1$ and $e_2$, then $e_1$
becomes part of the environment for $e_2$, and vice-versa. This is so
because the \emph{self} component $a_\lcl$ of the parent is split into
$x_1$ and $x_2$; $x_1$ becomes the \emph{self} part of $e_1$, but
$x_2$ is added to the \emph{other} part $a_\env$ of $e_1$ (and
symmetrically for $e_2$).
%
%Also note that parallel composition returns a pair of the outputs
%produced by $e_1$ and $e_2$. Thus, the variable $\res$ in $Q_1$ and
%$Q_2$ has to be appropriately renamed by the projections $\res.1$ and
%$\res.2$ in the postcondition of the parallel composition.

%The rule of frame of FCSL is a special case of parallel composition,
%when $e_2$ is the idle thread.
%%
%\[
%\tag{\normalsize \arabic{tags}}\refstepcounter{tags}\label{eq:frame}
%{\small{
%\begin{array}{c}
%\specK{\{P_1\}}\ e\ \specK{\{P_2\}} @ \rcon\\[2pt]
%\hline\\[-7pt]
%\specK{\{P_1 \circledast Q\}}\ e\ \specK{\{P_1 \circledast Q\}} @ \rcon
%\end{array}\qquad 
%\begin{array}{c}
%\mbox{$Q$ stable under}\\
%\mbox{$\rcon$'s transitions}
%\end{array}
%}}
%\]
%A notable difference from the frame rules of other separation logics
%is that FCSL's definition of $\circledast$ forces that the value being
%framed onto \emph{self} component is \emph{subtracted} from the
%\emph{other} component, whereas in other separation logic, the frame
%value materializes out of nowhere. To illustrate, we can frame
%$\gists$ onto the history $\hists$ in the the
%spec~(\ref{tag:exchangespec}), by taking
%$\rcon\,{\eqdef}\,a_\lcl\,{=}\,(\heaps, \hists,
%\perms)\,{=}\,(\emptyset, \gists, \emptyset)$.
%We obtain, after some simplification:
%%
%\[
%{\small{
%\begin{array}{c}
%\specK{\{\heaps = \emptyset, \perms = \emptyset, \hists = \gists, \gist \subseteq \gists \hunion \histo \hunion \mygather{\pending}\}}\\[2pt]
%\mathtt{exchange}\ v \\[2pt]
%\spec{\!\!
%  \begin{array}{c}
%    \heaps = \emptyset, \perms = \emptyset, \gist \subseteq \gists \hunion \histo \hunion \mygather{\pending}, \hbox{}\\[1pt]
%    \mathsf{if}\ \res\ \mathsf{is}\ \mathsf{Some}\ w\ \mathsf{then}\\[1pt]
%    \exists t\ldot \hists = t \mapsto (v, w) \hunion \gists, 
%    \mathsf{last} (\gist) < t, \twin{t}~\mathsf{else}\ \hists = \gists    
%  \end{array}
%\!\!}@\cal E
%\end{array}
%}}
%\]
%But notice how the spec now says that $\gist \subseteq \gists \hunion
%\histo \hunion \mygather{\pending}$, whereas
%in~(\ref{tag:exchangespec}) it said $\gist \subseteq \histo \hunion
%\mygather{\pending}$. The addition of $\gists$ compensates for
%$\gists$ having been subtracted out of $\histo$, to be moved to
%$\hists$.

To reason about quiescent moments, we use one more constructor of
FCSL: \emph{hiding}. The program $\mathsf{hide}\ e$ operationally
executes $e$, but logically installs a resource within the scope of
$e$. In the case of the exchanger, $\mathsf{hide}\ e$ starts only with
private heaps $\heaps$ and $\heapo$, then takes a chunk of heap out of
$\heaps$ and ``installs'' an exchanger in this heap, allowing the
threads in $e$ to exchange values. $\mathsf{hide}\ e$ is
\emph{quiescent} wrt.~exchanger, as the typechecker will prevent
composing $\mathsf{hide}\ e$ with threads that want to exchange values
with $e$.

The auxiliaries $\perms, \hists$, $\permo, \histo$, and $\heapj,
\pending$, belonging to the exchanger (denoted as resource $\cal E$)
are visible within $\mathsf{hide}$, but outside, only $\heaps$
persists (denoted as a resource $\cal P$ for private state).  We elide
the general hiding rule~\cite{Nanevski-al:ESOP14}, and just show the
special case for the exchanger.
\[
%\tag{\normalsize \arabic{tags}}\refstepcounter{tags}\label{eq:ehide}
{\small{
\begin{array}{c}
\specK{\{P\}}\ e\ \specK{\{Q\}} @ \cal E\\[2pt]
\hline\\[-7pt]
\specK{\{\heaps = \Phi_1(\heapj), \Phi_1(P)\}}\ \mathsf{hide}~e\ \specK{\{\exists \Phi_2\ldot \heaps = \Phi_2(\heapj), \Phi_2(Q)\}} @ \cal P
\end{array}
}}
\]
Read bottom-up, the rule says that we can install the exchanger $\cal
E$ in the scope of a thread that works with $\cal P$, but then we need
substitutions $\Phi_1$ and $\Phi_2$, to map variables of $\cal E$
($\heaps, \perms, \hists$, \etc) to values expressed with variables
from $\cal P$ ($\heaps$ and $\heapo$). $\Phi_1$ is an initial such
substitution (user provided), and the rule guarantees the existence of
an ending substitution $\Phi_2$. The substitutions have to satisfy a
number of side conditions, which we elide here for brevity. The most
important one is that \emph{other} variable $a_\env = (\heapo, \permo,
\histo)$ is fixed to be the PCM unit (\ie,~a triple of empty
sets). Fixing $a_\env$ to unit captures that $\mathsf{hide}$ protects
$e$ from interference.

At the beginning of $\mathsf{hide}~e$, the private heap equals the
value that $\Phi_1$ gives to $\heapj$ ($\heaps = \Phi_1(\heapj)$). In
other words, the $\mathsf{hide}$ rule takes the private heap of a
thread, and makes it shared, \ie, gives it to the $\heapj$ component
of $\cal E$. Upon finishing, $\mathsf{hide}~e$ makes $\heapj$ private
again.
%
%($\heaps = \Phi_2(\heapj)$).

%\an{Should I say something about compositionality? Why is FCSL
%  compositional? Maybe say, soundness of FCSL has been established by
%  shallow embeding in Coq. Thus, the logic immediately inherits the
%  substitution principle, thereby allowing that clients can reason
%  only out of the Hoare spec of an object.}

%\subsection{FCSL basics}
%\label{sec:fcsl-basics}
%
%\todo{A short overview of FCSL: mostly, concerning subjectivity and hiding}
%
%\subsection{Histories as auxiliary state}
%\label{sec:hist-state} 


In the subsequent text we elide the resources from specs.


\section{Verifying Exchanger's Client}
\label{sec:cal}
\newcommand{\ts}{\mathit{ts}}
\newcommand{\vvs}{\mathit{vs}}
\newcommand{\acc}{\mathit{ac}}
\newcommand{\ws}{\mathit{ws}}
\newcommand{\sorted}[1]{\mathsf{sorted}\ #1}

%\paragraph{Client definition.}

We next illustrate how the formally specified exchanger from
Section~\ref{sec:exchanger} can be used by real-world client programs,
and how the \emph{other} component, asserted by the spec to satisfy
$\gist \subseteq \histo \hunion \mygather{\pending}$, is crucial for
their verification.
%
We emphasize that the proof of the client does not see the
implementation details, which are hidden by the
spec~\eqref{tag:exchangespec}.
%

While simple, our client is realistic, and has been used
in~\code{java.util.concurrent}~\cite{ExchangerClass}. It is defined as
follows. First, the exchanger loops until it exchanges the value.
%
\vspace{-2pt}
\[
\vspace{-2pt}
{\small{
\begin{array}{rl}
& \esc{exchange'}~(v : A) : A = \{\\[1pt]
&  ~~~~ w' \Asgn \esc{exchange}~v;\\[1pt]
&  ~~~~
  \kw{if}~~w'~~\kw{is}~~\esc{Some}~w~~\kw{then}~~\kw{return}~w~~\kw{else}~~\esc{exchange'}~v~\}
\end{array}
}}
\]
%
Next, $\esc{exchange'}$ is iterated to exchange a sequence in order,
appending the received matches to an accumulator.
%
%\vspace{-2pt}
\[
{\small{
\begin{array}{rl}
& \esc{ex\_seq}~(\vvs, \acc : \esc{seq}~A) : \esc{seq}~A = \{\\[1pt]
& ~~~~ \kw{if}~~\vvs~~\kw{is}~~v{::}\vvs'~~\kw{then}\\[1pt]
& ~~~~ ~~~~ w \Asgn \esc{exchange'}~v;~~\esc{ex\_seq}~(\vvs', \esc{snoc}~\acc~w)\\[1pt]
& ~~~~ \kw{else}~~\kw{return}~\acc~\}
\end{array}
}}
\]
%
Our goal is to prove, via~\eqref{tag:exchangespec},
% compositionally, \ie~reasoning only out of the
%spec of $\mathtt{exchange}$, 
that the parallel composition
%
\[
e = \esc{ex\_seq}~(\vvs_1, \esc{nil}) \parallel \esc{ex\_seq}~(\vvs_2, \esc{nil})
\]
%
exchanges $\vvs_1$ and $\vvs_2$, \ie,~returns the pair $(\vvs_2,
\vvs_1)$. This holds only under the assumption that $e$ runs without
interference (\ie, quiescently), so that the two threads in $e$ have
no choice but to exchange the values between themselves. 

We make the quiescence assumption explicit using the FCSL $\hide$
constructor, as described in Section~\ref{sec:background}.
%
%The FCSL typechecker prevents $\hide~e$ from being
%composed in parallel with other exchanger threads, thus enforcing
%quiescence. We refer to Section~\ref{sec:background} for details about
%$\hide$, but here just mention that $\hide~e$ also removes selected
%parts of $e$'s auxiliary state, making them invisible outside of
%$\hide$. In particular, the precondition and postcondition of $\hide$
%existentially quantify over selected \emph{self} and \emph{joint}
%components of $e$, while setting the \emph{other} components to
%$\emptyset$ (which is sound due to quiescence). In this example, we
%select to hide the histories and offer sets of $e$, and to fold the
%heap $\heapj$ into the private heap of the surrounding program.
%
Thus, we establish the following Hoare triple:
%
\[
\tag{\arabic{tags}}\refstepcounter{tags}\label{tag:hidespec} 
{\small{
\!\!\!\!\!
\begin{array}{c}
\specK{\{\heaps = g \mapsto\mathsf{null}\}}~~\hide~~e~~\specK{\{g \in
  \mathsf{dom}~\heaps, \res = (\vvs_2, \vvs_1)\}} % @ \cal P
\end{array}
}}
\]
%
It says that we start with a heap where $g$ stores $\mathsf{null}$,
and end with a possibly larger heap (due to the memory leak), but with
the result $(\vvs_2, \vvs_1)$. The auxiliaries $\perms, \permo$,
$\gists, \gisto$, $\heapj, \pending$ are visible inside $\hide$, but
outside, only $\heaps$ persists.

%\an{I deliberately haven't removed the $@ \cal P$ and $@ \cal E$ parts
%  from the displays in this section. This should be done only if we
%  decide to remove the FCSL background section. As of now, just to try
%  it on, I have moved that section to the appendix, but have instead
%  included brief explanations of the rules for hiding and parallel
%  composition in the present section (see the text below). Thus, the
%  present section can also serve as a brief, informal introduction to
%  FCSL, albeit without the description of concurroids (hence, the need
%  to remove $@ \cal P$). I think this is very appropriate for PLDI.
%%
%  It remains to be seen if we can rework the counting network section
%  so that it doesn't rely on these rules, but only invokes the
%  intuitive descriptions that I provided here. Doing so would probably
%  be good for the paper, as it would remove the technicalities, and
%  shine the light on the main ideas. 
%%
%  A starting point for achieving this is to remove the proof outlines
%  in that section, and replace each with a program : spec
%  ascription. Also, each such ascription should have a prose
%  describing why it holds informally, but we already have most of that
%  prose written, so that shouldn't be too much work.}

%$vs$ and $ws$ are exchanged. This is a valid property under the
%assumptions that the two $\esc{ex\_seq}$ threads run in isolation,
%\ie, without interfernce from any other exchanging threads. In that
%case, the two threads have no other options but to exchange values
%between themselves.

\paragraph{Explaining the verification.}
%
We illustrate the verification by listing the specs of selected
subprograms. First, the spec of $\esc{exchange'}$ easily derives
from~(\ref{tag:exchangespec}) by removing the now-impossible failing
case.
%First, $\esc{exchange'}$ is easy to specify,
%as we merely need remove from~(\ref{tag:exchangespec}) the impossible
%case of $\esc{exchange'}$ failing.
%
\[
{\small{
\begin{array}{c}
\specK{\{\heaps = \emptyset, \perms = \emptyset, \hists = \emptyset, \gist \subseteq \histo \hunion \mygather{\pending}\}}\\[2pt]
\esc{exchange'}\ v\\[2pt]
\spec{\!\!
\begin{array}{c}
\heaps = \emptyset, \perms = \emptyset, \gist \subseteq \histo \hunion \mygather{\pending}, \\[1pt]    
\exists t\ldot \hists = t \mapsto (v, \res), \mathsf{last} (\gist) < t, \twin{t}
\end{array}
\!\!}%@\cal E
% \specK{\{\heaps = \emptyset, \perms = \emptyset, \gist \subseteq \gists \hunion \histo \hunion \mygather{\pending}, \hbox{}}\\
% \specK{\exists t\ldot \hists = t \mapsto (v, \res) \hunion \gists, \mathsf{last} (\gist) < t, \twin{t}\}} @ 
\end{array}
}}
\]
%
Next, $\esc{ex\_seq}$ has the following spec:
%
\[
{\small{
\begin{array}{c}
\specK{\{\heaps = \emptyset, \perms = \emptyset, \hists = \emptyset\}}\\[2pt]
\mathtt{ex\_seq}~(vs, \mathsf{nil})\\[2pt]
\spec{\!\!\!
\begin{array}{c}
\exists \ts\ldot \heaps = \emptyset, \perms = \emptyset, 
\hists = \mathsf{zip}~\ts~\vvs~\res,
\\[1pt]
\mathsf{grows\_notwins}~\ts, 
\mathsf{zip}~\overline{\ts}~\res~\vvs \subseteq \histo  \hunion \mygather{\pending}  
\end{array}
\!\!\!}%@ \cal E
\end{array}
}}
\]
%
Here, $\ts$ is a list of time-stamps, and
$\mathsf{zip}\,\ts\,\vvs\,\ws$ joins up the singleton histories
$t\,{\mapsto}\,(v, w)$, for each $t$, $v$, $w$ drawn, in order, from
the lists $\ts$, $\vvs$, $\ws$.
%
%\[
%{\small{
%\mathsf{zip}~ts~vs~ws = \left\{%
%\begin{array}{l}
%t \mapsto (v, w) \hunion \mathsf{zip}~\ts'~\vvs'~\ws', \\
%\hphantom{\emptyset,}\ \mbox{if $\ts=t\,{::}\,\ts', \vvs=v\,{::}\,\vvs', \ws=w\,{::}\,\ws'$}\\
%\emptyset, \mbox{if $\ts = \vvs = \ws = \mathsf{nil}$}\\
%\mbox{undefined}, \mbox{otherwise}
%\end{array}\right.
%}}
%\]
%
The spec says that at the time-stamps from $\ts$, $\esc{ex\_seq}$
exchanged the elements of $\vvs$ for those of $\esc{res}$. That $\ts$
is increasing and contains no twins, follows from the spec of
$\esc{exchange'}$ which says that the time-stamps $t$ and $\bar t$
that populate $\ts$ and $\overline{\ts}$, are larger than anything in
$\gist$, and thus only grow with iteration.
%
%\an{This point is subtle, but it requires
%  discussing framing and validity, so better avoid. The postcondition
%  implies that $\bar{\ts}$ is also increasing, so we don't state it
%  explicitly.}  
%
From the same postcondition, it follows that $\histo \hunion
\mygather{\pending}$ contains all the twin exchanges, by
invariant~(\ref{tag:exchanging}), as commented in
Section~\ref{sec:overview} about the spec for $\esc{exchange}$.

Next, by the FCSL parallel composition rule
(Section~\ref{sec:background}):
%
\[
%\tag{\arabic{tags}}\refstepcounter{tags}\label{tag:e}\\
{\small{
\begin{array}{c}
\specK{\{\heaps = \emptyset, \perms = \emptyset, \hists = \emptyset\}}\\[2pt]
%\begin{array}{c}
%\specK{\{\heaps = \emptyset, \perms = \emptyset, \hists = \emptyset\}}\\[1pt]
\mathsf{ex\_seq}~(\vvs_1, \mathsf{nil}) %\\[1pt]
%\specK{\{Q(\vvs_1)\}}
%\end{array} 
\parallel
%\begin{array}{c}
%\specK{\{\heaps = \emptyset, \perms = \emptyset, \hists = \emptyset\}}\\[1pt]
\mathsf{ex\_seq}~(\vvs_2, \mathsf{nil}) \\[1pt]
%\specK{\{Q(\vvs_2)\}}
%\end{array}\\[2pt]
\spec{\!\!\!
\begin{array}{c}
\exists \ts_1~\ts_2\ldot \mathsf{grows\_notwins}~{\ts_1}, \mathsf{grows\_notwins}~{\ts_2},\\
 \heaps = \emptyset, \perms = \emptyset, \hists = \mathsf{zip}~\ts_1~\vvs_1~\res.1 \hunion \mathsf{zip}~\ts_2~\vvs_2~\res.2,\\
 \mathsf{zip}~\overline{\ts_1}~\res.1~\vvs_1 \subseteq \mathsf{zip}~\ts_2~\vvs_2~\res.2 \hunion \histo \hunion\mygather{\pending}, \\%\tag{\arabic{tags}}\refstepcounter{tags}\label{tag:x}\\
 \mathsf{zip}~\overline{\ts_2}~\res.2~\vvs_2 \subseteq \mathsf{zip}~\ts_1~\vvs_1~\res.1 \hunion \histo \hunion\mygather{\pending}. %\tag{\arabic{tags}}\refstepcounter{tags}\label{tag:y}
\end{array}
\!\!\!}
%\{Q(\vvs_1) \circledast Q(\vvs_2)\}}
%\makebox[0pt]{\quad $@ \cal E$}
\end{array}
}}
\]
%
%\gad{Some unification on notation is due: Note that $\mathsf{ex\_seq}$
%  has been currified here whereas before it took a pair of
%  arguments. Also the prelude of the outline still refers to the
%  appendix (Section~\ref{sec:background}} \an{fixed} 
%
To explain: $ts$ and $\esc{res}$ from the left and right
$\esc{ex\_seq}$ threads become $ts_1$, $ts_2$, $\esc{res}.1$ and
$\esc{res}.2$, respectively. The values of each \emph{self} component
$\heaps$, $\perms$, $\hists$ from the two threads are joined into the
\emph{self} component of the composition. At the same time, the
\emph{other} component $\histo$ of the left (resp.~right) thread
equals the sum of $\hists$ of the right (resp.~left) thread, and the
$\histo$ of the composition.  This formalizes the intuition that upon
forking, the left thread becomes part of the environment for the right
thread, and vice-versa.

The postcondition says that the self history of $e$ contains both
$\mathsf{zip}\,\ts_1\,\vvs_1\,\res.1$ and
$\mathsf{zip}\,\ts_2\,\vvs_2\,\res.2$. Thus, $\vvs_1$ is exchanged for
$\res.1$, and $\vvs_2$ for $\res.2$. But we further want to derive
$\res.1\,{=}\,\vvs_2$ and $\res.2\,{=}\,\vvs_1$, \ie, the lists are
exchanged \emph{for each other}, in the absence of interference.

We next explain how this desired property follows for $\hide~e$, from
the two inequalities in $e$'s postcondition
\begin{align}
\mathsf{zip}~\overline{\ts_1}~\res.1~\vvs_1\,&\subseteq&\!\!\!\!{\mathsf{zip}}~\ts_2~\vvs_2~\res.2\,&\,{\hunion}\,\histo\,{\hunion}\,\mygather{\pending}, \tag{\arabic{tags}}\refstepcounter{tags}\label{tag:x}\\
\mathsf{zip}~\overline{\ts_2}~\res.2~\vvs_2\,&\subseteq&\!\!\!\!{\mathsf{zip}}~\ts_1~\vvs_1~\res.1\,&\,{\hunion}\,\histo\,{\hunion}\,\mygather{\pending}. \tag{\arabic{tags}}\refstepcounter{tags}\label{tag:y}
\end{align}
Notice that $(\ref{tag:x})$ and $(\ref{tag:y})$ are ultimately
instances of the conjunct $\gist \subseteq \histo \hunion
\mygather{\pending}$ that was part of the
specification~(\ref{tag:exchangespec}), thereby justifying the use of
subjective \emph{other} variables.

We know that $\mathsf{dom}\ \pending\,{=}\,\perms \hunion \permo$
(from Section~\ref{sec:overview}), that $\perms\,{=}\,\emptyset$ (from
$e$'s postcondition), and that by hiding,
$\permo\,{=}\,\histo\,{=}\,\emptyset$. Thus, towards deriving the
postcondition of $\hide~e$, we simplify $(\ref{tag:x})$ and
$(\ref{tag:y})$ into:
\begin{align*}
\mathsf{zip}~\overline{\ts_1}~\res.1~\vvs_1 \subseteq \mathsf{zip}~\ts_2~\vvs_2~\res.2\\% \tag{\ref{tag:x}'}\label{tag:x'}\\
\mathsf{zip}~\overline{\ts_2}~\res.2~\vvs_2 \subseteq \mathsf{zip}~\ts_1~\vvs_1~\res.1% \tag{\ref{tag:y}'}\label{tag:y'}
\end{align*}
%
Because $\ts_1$ and $\ts_2$ are increasing lists of time-stamps, and
contain no twins, the above implies $\ts_2 = \overline{\ts_1}$. Hence:
%\an{The reasoning is a bit subtle and goes as follows: From the above
%  inequations, we get: bar(ts1) <= ts2 and bar(ts2) <= ts1, ***as
%  sets***. Because bar is idempotent, we get bar(ts1) = ts2 **as
%  sets**. Now, we need that bar(ts1) is sorted, in order to conclude
%  that bar(ts1) = ts2 **as lists**. But sortedness of bar(ts1) follows
%  from the fact that ts1 has no twins.}
%
\[
\mathsf{zip}~\overline{\ts_1}~\res.1~\vvs_1 = \mathsf{zip}~\ts_2~\vvs_2~\res.2
\]
%
and thus $\res.1\,{=}\,\vvs_2$, $\vvs_1\,{=}\,\res.2$. We omit the
remaining technical argument that explains how the heap $\heapj$, with
the pointer $g$, is folded into $\heaps$, which ultimately
obtains~\eqref{tag:hidespec}.


\section{Specifying Counting Networks in Hoare Style}
\label{sec:counting}

We now show how one can use histories and subjectivity to specify
another class of non-linearizable objects---\emph{counting networks}.

Counting networks are a special case of \emph{balancing networks}
introduced by Aspnes \etal~\cite{Aspnes-al:JACM94}, themselves
building on sorting networks~\cite{Ajtai-al:STOC83}. Counting networks
implement concurrent counters in a way free from synchronization
bottlenecks.
%
The key idea of counting networks is to decompose the workload between
\emph{several} counters, so that each of them is responsible for a
disjoint set of values. A thread trying to perform an incrementation
first approaches the \emph{balancer}, which is a logical ``switch''
that ``directs'' the thread, \ie, provides it with the address of the
counter to increment.
%
The balancers make counting networks' operations
\emph{non-linearizable}, as in the presence of interference the
results of increments might be observed out of order.
%
% \wrt~a sequential specification.

\begin{figure}%[18]{r}{4cm} 
\begin{tabular}{c@{\ \ \ \ \ \ }c}
\begin{minipage}[c]{2.5cm}
\includegraphics[width=2.1cm]{counter.pdf} 
\end{minipage}
&
\begin{minipage}[l]{4.9cm}
\centering
{\small{
\[
\begin{array}{rl}
\Num{1} & \esc{getAndInc()} : \esc{nat}~=~\esc{\{}  \\[2pt] 
\Num{2} & ~~~~ b \Asgn \esc{flip(}\bal\esc{)};\\[2pt]
\Num{3} & ~~~~ \res \Asgn \esc{fetchAndAdd2(}c_b\esc{)};\\[2pt]
\Num{4} & ~~~~ \kw{return}~\res~\esc{\}}
\end{array}
\]
}}
\end{minipage} 
\\
(a) & (b)
%
\end{tabular}
%
\caption{Simple counting network: intuition and pseudo-code.}
\label{fig:counter-fig} 
\end{figure}

Figure~\ref{fig:counter-fig} presents a schematic outline (a) and a
pseudo-code implementation (b) of a counting network with a single
balancer.
%
The implementation contains three pointers: the balancer $\bal$, which
stores either 0 or 1, thus directing threads to the shared pointers
$c_0$ or $c_1$, which count the even and odd values,
respectively. Threads increment by calling \code{getAndInc}, which
works as follows. It first atomically changes the bit value of the
balancer via a call to atomic operation \code{flip} (line 2). The
\code{flip} operation returns the \emph{previous} value $b$ of the
balancer as a result, thus determining which of the counters, $c_0$ or
$c_1$, should be incremented. The thread proceeds to atomically add 2
to the value of $c_b$ via \code{fetchAndAdd2} (line 3). The old value
of $c_b$ is returned as the result of the procedure.\footnote{In the
  counting network from Figure~\ref{fig:counter-fig}, the balancer
  itself might seem like a contention point. However, the \code{flip}
  operation is much less expensive than \code{CAS} as a
  synchronization mechanism. The performance can be further improved
  by constructing a \emph{diffracting tree} of several
  balancers~\cite[\S 12.6]{Herlihy-Shavit:08}, but we do not consider
  diffracting trees here.}

Assuming that $c_0$ and $c_1$ are initialized with $0$ and $1$, it is
easy to see that in a single-threaded program, the network will behave
as a conventional counter; that is, consecutive invocations of
\code{getAndInc} return consecutive nats.
%
However, in the concurrent setting, \code{getAndInc} may return
results out of order, as follows. 
%
% which historically led to the definition of quiescent
% consistency~\cite[\S 3.3]{Herlihy-Shavit:08} in order to specify the
% network's concurrent behavior.

\vspace{3pt}
\begin{example}
\label{ex:t1t2}
%
Consider two threads, $T_1$ and $T_2$ operating on the network
initialized with $\bal\,{\mapsto}\,0$, $c_b\,{\mapsto}\,b$. $T_1$
calls \code{getAndInc} and executes its line~2 to set $\bal$ to 1. It
gets suspended, so $T_2$ proceeds to execute lines~2 and~3, therefore
setting $\bal$ back to $0$ and returning $1$. While $T_1$ is still
suspended, $T_2$ calls \code{getAndInc} again, gets directed to $c_0$,
and returns 0, after it has just returned 1.
%
\end{example}
\vspace{3pt}

\noindent

This out-of-order behavior, however, is not random, and can be
precisely characterized as a function of the number of threads
operating on the
network~\cite{Afek-al:OPODIS10,Jagadeesan-Riely:ICALP14}. In the rest
of this section and in Section~\ref{sec:qc-client}, we show how to
capture such bounds precisely using auxiliary state of (subjective)
histories in a client-sensitive manner. As a form of road map, we
first list the desired requirements for the spec of \code{getAndInc},
%
adapting the design goals of the criteria, such as QC, QQC and
QL~\cite{Aspnes-al:JACM94,Afek-al:OPODIS10,Jagadeesan-Riely:ICALP14},
which we will proceed to verify formally, and then employ in
client-side reasoning.
%
\vspace{2pt}
\begin{itemize}

\item \textbf{R1:} Two different calls to \code{getAndInc}
  should return distinct results (\emph{strong concurrent
    counter semantics}).

\item \textbf{R2:} The results of calls to \code{getAndInc},
  separated by a period of quiescence (\ie, absence of interference),
  should appear in their sequential order (\emph{quiescent
    consistency}).

\item \textbf{R3:} The results of two sequential calls $C_1$ and
  $C_2$, in a single thread should be out of order by no more than
  $2\ N$, where $N$ is the number of interfering calls that overlap
  with $C_1$ and $C_2$ (\emph{quantitative quiescent
    consistency/quasi-linearizability}).
%\an{Can we chose one of the two here: either qqc or ql?}

\end{itemize}

\begin{comment}
\noindent 
In the rest of this and in the next section, we will illustrate how to
achieve all these goals by employing (a) \emph{subjective auxiliary
  state} as a mechanism for \emph{{explicitly referring to and
    quantifying over}} the effects of currently interfering threads
(via its \emph{other}-component) in combination with (b)
\emph{histories}, providing a way to \emph{{logically record relevant
    pieces of state information}} (including witnessed interference)
in Hoare-style program specifications.
\end{comment}

%\vspace{2pt}
%\lipsum[1]

\subsection{Formalizing the counting network}
\label{sec:counting-intuition}

To formalize the necessary invariants, we elaborate the counting
network with auxiliary state: \emph{tokens} (isomorphic to nats) and
\emph{interference-capturing histories}.

A \emph{token} provides a thread that owns it with the right to
increment an appropriate counter~\cite{Aspnes-al:JACM94}. In our
example, a thread that performs the \code{flip} in line 2 of
\code{getAndInc} will be awarded a token which it can then spend to
execute \code{fetchAndAdd2}.
%
Thus, any individual token represents a ``pending'' call to
\code{getAndInc}, and the set of unspent tokens serves as a bound on
the out-of-order behavior that the network exhibits. We will have four
different auxiliary variables tracking the different classes of
tokens: $\tkns^0$ and $\tkns^1$ keep the tokens owned by the
\emph{self} thread, administering access to $c_0$ and $c_1$,
respectively. Similarly, $\tkno^0$ and $\tkno^1$ keep the tokens owned
by the \emph{other} thread. We abbreviate $\tkn^i = \tkns^i \hunion
\tkno^i$, $i=0,1$, $\tkns = \tkns^0 \hunion \tkns^1$, $\tkno = \tkno^0
\hunion \tkno^1$.

Figure~\ref{fig:chist} illustrates a network with three \emph{even}
tokens: $x^0, y^0, z^0 \in \tkn^0$, held by threads that will
increment $c_0$, and one \emph{odd} token $u^1 \in \tkn^1$, whose
owner will increment $c_1$.
%
%\an{Removed: We also point out here that token names (and their
%  uniqueness) will be of critical importance for the specifications we
%  give further. This point was never emphasized later on, so why
%  bother drawing attention to it.}

\begin{figure}
\centering
\includegraphics[width=8.2cm]{chist.pdf}      
\caption{Tokens and histories of the simple counting network.}
\label{fig:chist}
\end{figure}

A \emph{history} of the counting network is an auxiliary finite map,
consisting of entries of the form
\[
\tag{\normalsize{\arabic{tags}}}\refstepcounter{tags}\label{eq:cn-entry} 
%
t \mapsto (\tknh^0, \tknh^1, z)
\]
Such an entry records that the value $t$ has been written into an
appropriate counter ($c_0$ or $c_1$, depending on the parity of $t$),
at the moment when $\tkn^0$ and $\tkn^1$ held values $\tknh^0$ and
$\tknh^1$, respectively. Moreover, in order to write $t$ into a
counter, the token $z$ was spent by the thread invoking
\code{getAndInc}. We will refer to $z$ as the \emph{spent}
token. Notice that the entries in the history contain tokens held by
both \emph{self} and \emph{other} threads. Thus, a history captures
the behavior of a thread subjectively, \ie, as a function of the
behavior of the threads interfering with it.

Similarly to tokens, network histories are split into four different
auxiliary variables, which are manipulated differently by our
proofs. Thus, we have $\hists^0$ and $\hists^1$ to track the even and
odd counter updates performed by the \emph{self} thread, and dually
$\histo^0$ and $\histo^1$ for the \emph{other} thread. We abbreviate
$\hist^i = \hists^i \hunion \histo^i$, $i = 0,1$, and $\hists =
\hists^0 \hunion \hists^1$, and $\histo = \histo^0 \hunion \histo^1$.

Figure~\ref{fig:chist} illustrates a moment in network's history and
how it relates to the state of the counters. Only $0$ has been written
to $c_0$ so far (upon initialization), hence $\hist^0$ only contains
an entry for $t = 0$ (we ignore at the moment the \emph{contents} of
the history entries). On the other hand, $\hist^1$ has entries for $1$
and $3$, because after initialization, one thread has increased $c_1$.
%
The gray boxes indicate that $0$ and $3$ are the current values of
$c_0$ and $c_1$, and thus also the latest entries in $\hist^0$ and
$\hist^1$, respectively. In particular, these values will be returned
by the next invocations of \code{fetchAndAdd2}. The dashed boxes
correspond to the entries to be added to the history by the currently
running threads holding the tokens $x^0$, $y^0$, $z^0$,
$u^1$. However, as thread scheduling is non-deterministic, we cannot
predict which of the tokens will be spent to, say, write 2 into $c_0$
(it may be any of the even tokens). On the other hand, in the absence
of other threads joining the network, we know that token $u^1$ will be
spent to write $5$ into $c_1$.

\subsubsection{Resource invariants of the counting network}
\label{sec:count-netw-invar}
We next formally list the invariants that describe the interdependence
between the various components of the real and auxiliary state. In
addition to $\tkn$ and $\hist$ which come in flavors private to
\emph{self} and \emph{other} threads, we require the following joint
variables: (1) $\heapj$ which stands for the joint heap of the
network, and (2) $b_\joint$, $n^0_\joint$ and $n^1_\joint$ which stand
for the contents of $\bal$, $c_o$ and $c_1$, respectively. 

The main invariant of the network relates the number of tokens, the
size of histories and the value of the balancer as follows.
%
\[
\tag{\normalsize{\arabic{tags}}}\refstepcounter{tags}\label{cn:si} 
%
|\hist^0| + |\tkn^0| =
|\hist^1| + |\tkn^1| + b_\joint
\]
%
The equation directly motivates our design of the auxiliary state, and
leads to establishing the requirements \textbf{R1}--\textbf{R3}. It
formalizes the intuition that out-of-order anomalies of the counting
network appear if one of the two counters is too far ahead of the
other one.
%
The invariant~(\ref{cn:si}) provides a bound on such a situation. One
counter can get ahead temporarily, but then the other counter must
have a number of threads waiting to spend their tokens, and increment
it. Thus, the other counter will eventually catch up.

The approaches such as quiescent and quantitative quiescent
consistency describe this situation by referring to the number of
\emph{unmatched} call events in an event
history~\cite{Derrick-al:FM14,Jagadeesan-Riely:ICALP14}. In contrast,
we formalize this property by relying on auxiliary state. In our case,
the sets of tokens $\tknh^0$ and $\tknh^1$ recorded in the entry for
the number $t$ determine the environment's capability to add new
history entries, and thus ``run ahead'' or ``catch up'' after $t$ has
been returned. Such auxiliary state will let us directly specify the
network's behavior in the moments of quiescence (\ie,~when $\tkno$ is
empty), but also \emph{quantitatively} bound the out-of-orderness as a
function of $\tkno$.

We next list the other resource invariants.
\vspace{2pt}
\begin{enumerate}[label=(\roman*)]

% %
% \[
% \tag{\normalsize{\arabic{tags}}}\refstepcounter{tags}\label{eq:cn-states} 
% {\small
% \begin{array}{r@{\ }c@{\ }l} 
% {\!\!\!\!\!\!\!\!}W_{\ccon} & \!\eqdef & \exists \tkns~\tkno~ \hists~\histo~b~n_0~n_1\ldot 
% %  
% \qcl \spts (\tkns, \hists)\aand \qcl \opts (\tkno, \histo) 
%   \\[4pt] 
% &\aand & \qcl \jpts \bal \hpts b \hunion c_0 \hpts n_0 \hunion c_1 \hpts n_1   
%  \aand \hvalid~(\hists \hunion \histo)         \\[3pt] 
% &\aand & \SI~\tkn^0~\tkn^1~\hist^0~\hist^1~b ~~~~\aand
%          \CI~\hist^0~\hist^1~n_0~n_1 \\[3pt]  
% &\aand & \TI~\tkn^0~\tkn^1~(\hist^0 \hunion \hist^1)\aand \AI~\hist^0~\hist^1~\tkn^0~\tkn^1~n_0~n_1.
% \end{array}
% }
% \]
%

% where $\hist^i = (\hists \hunion \histo)^i$ and
% $\tkn^i = (\tkns \hunion \tkno)^i$ for $i \in \set{0,1}$.

\item\label{cn:state} $\heapj = \bal \mapsto b_\joint \hunion c_0 \mapsto n^0_\joint
  \hunion c_1 \mapsto n^1_\joint$.

\item\label{cn:hvalid} The histories contain disjoint time-stamps. % (\ie $\hists \hunion \histo$ is always defined);
 
% The state-space invariant $W_{\ccon}$ fixes the auxiliary self/other
% components to be pairs of tokens and histories $(\tkns, \hists)$ and
% $(\tkno, \histo)$, which are held/contributed by the thread and its
% environment, correspondingly. The invarian $\hvalid~(\hists \hunion
% \histo)$ ensures that at any moment 
% %
% The joint part of the state contains the pointers $\bal$, $c_0$ and
% $c_1$, and the relations between all these components are specified by
% the invariants $\SI$, $\CI$, $\TI$ and $\AI$.

\item\label{cn:ci} 
%
  The history $\hist^0$ (resp. $\hist^1$) contains \emph{all} even
  (resp. odd) values in the interval $[0, n^0_\joint]$ (resp. $[1,
    n^1_\joint]$). In other words, the network does not ``skip''
  values. This further automatically ensures that $n^0_\joint$ and
  $n^1_\joint$ are the last time-stamps in $\hist^0$ and $\hist^1$,
  respectively.

\item\label{cn:ti}  
%
  $\tkn^0$, $\tkn^1$ and $\Tomb~(\hists \hunion \histo)$ contain
  mutually disjoint tokens, where $\Tomb~(t \mapsto (\tknh^0,\tknh^1,
  z) \hunion \hist') = \{z\} \hunion \Tomb~\hist'$, and
  $\Tomb~\emptyset = \emptyset$. In other words, a spent token
  never appears among the ``alive'' ones (\ie, in $\tkn^0 \hunion
  \tkn^1$).

%As a consequence, $\Tomb~(\hists \hunion \histo)$ is always defined.

\item\label{cn:ti1}
%
  $t \mapsto (\tknh^0, \tknh^1, z) \subseteq \hists \hunion \histo
  \implies z \in \tknh^0 \hunion \tknh^1$. \\[-7pt]

\item\label{cn:ai} 
%
For any $t$, $\tknh^0$, $\tknh^1$, $z$: \\[-7pt]
% 
  \begin{itemize}
  \item   $t \hpts (\tknh^0, \tknh^1, z) \subseteq \hist^0 ~\implies$\\[2pt]
    $t + 2\ |\tkn^0 \cap \tknh^0| < n^1_\joint + 2\ |\tkn^1 \cap
    \tknh^1| + 2$, and \\[-7pt]
  \item
    $t \hpts (\tknh^0, \tknh^1, z) \subseteq \hist^1 ~\implies$\\[2pt]
    $t + 2\ |\tkn^1 \cap \tknh^1| < n^0_\joint + 2\ |\tkn^0 \cap
    \tknh^0| + 2$.
  \end{itemize}
%
\end{enumerate}
\vspace{2pt}

We comment on the invariant~\ref{cn:ai}, as it is the only non-trivial
one. This invariant provides quantitative information about the
network history by relating the actual ($n^0_\joint$, $n^1_\joint$)
and the past ($t$) counter values, via the current amount of
interference ($\tkn^0$, $\tkn^1$) and the snapshot interference
($\tknh^0$, $\tknh^1$). 
%
%Whereas~(\ref{cn:si}) provides bounds for the network's present,
%\ref{cn:ai} does so for the counter values \emph{from the past}. 
%
To explain~\ref{cn:ai}, we resort to the intuition provided by the
following equality, which, however, being \emph{not quite valid},
cannot be used as an invariant, as we shall see. Focusing on the
first clause in~\ref{cn:ai}, if
$t \mapsto (\tknh^0, \tknh^1, z) \subseteq \hist^0$, then,
intuitively:
%
\[
{\small{
t + 2\ |\tknh^0 \setminus \tkn^0 | + 2\ |\tkn^0 \cap \tknh^0| =
n^1_\joint + 2\ |\tkn^1 \cap \tknh^1| + (2 b_\joint - 1)
}}
\]
%
The equality says the following. When $t$ is snapshot from $c_0$ and
placed into the history $\hist^0$, the set of outstanding even tokens
was $\tknh^0$. By the present time, $c_0$ has been increased
$|\tknh^0 \setminus \tkn^0|$ times, each time by $2$, thus
$n^0_\joint = t + 2\ |\tknh^0 \setminus \tkn^0|$. What is left to add
to $c_0$ to reach the \emph{period of quiescence}, when no threads
interfere with us, is $2\ |\tknh^0 \cap \tkn^0|$. Similar reasoning
applies to $c_1$. It is easy to see at the period of quiescence, $c_0$
and $c_1$ differ by $2 b_\joint - 1$; that is, the counter pointed to
by $\bal$ is behind by $1$. However, the equality is invalid, as
$b_\joint$ can be read off only in the present, whereas the
``intuitive'' reasoning behind the equality requires a value of
$b_\joint$ from a quiescent period \emph{in the future}. Hence, in
order to get a valid property, we bound $2 b_\joint - 1$ by 2. For
simplicity, we even further weaken the bounds by dropping
$|\tknh^0 \setminus \tkn^0|$ to obtain~\ref{cn:ai}; as it will turn
out, even such a simpler bound will suffice for proving
\textbf{R1}--\textbf{R3}.

% in Section~\ref{sec:qc-client}.

%As already aparent from our explanation, the invariant gives us a way
%to formally model when the network is in the period of quiescence, as
%required in \textbf{R2}, which we verify in
%Section~\ref{sec:qc-client}.
%%

\subsubsection{Transitions of the counting network}
\label{sec:count-netw-prot}
The STS $\cal C$ for the counting network admits the following two
transitions between states. They lift the atomic operations
\code{flip} and \code{fetchAndAdd2} from
Figure~\ref{fig:counter-fig}~(b), to work with auxiliary state.

%\noindent
\paragraph{Flipping transition}
%
changes the bit value $b_\joint$ of $\bal$ to the complementary one,
$1 - b_\joint$.
%
It also generates a token $z$ (of parity $b_\joint$) and stores it
into $\tkns$. The token is fresh, \ie, distinct from all alive and
spent tokens in $\tkns \hunion \tkno \hunion{\Tomb~(\hists
  \hunion \histo)}$.
%
% \[
% \small{
% \!\!\!\!
% %
% \begin{array}{r@{\ }c@{\ }l@{\ \ }c}
%   \tauflip & \eqdef &\qcl \jpts (\bal \hpts b \hunion c_0 \hpts n_0
%                       \hunion c_1 \hpts n_1)\aand \qcl \spts (\tkns, \hists) & 
%   \\[3pt]
%            &\rightsquigarrow &
%               \qcl \jpts (\bal \hpts (1-{b}) \hunion c_0 \hpts n_0
%               \hunion c_1 \hpts n_1) \aand \\[2pt]
%            && \qcl \spts (\tkns \hunion \set{\tfr{\tkns \hunion \tkno
%               \hunion{\Tomb~(\hists \hunion \histo)}}{b}}, \hists)
% \end{array}
% }
% \]
%\noindent
\paragraph{Incrementing transition}
%
takes a token $z$ as input. Depending on the parity $i$ of $z$, it
atomically increases the value $n^i_\joint$ of $c_i$ by two, while
simultaneously removing $z$ from $\tkns$ (thus, the precondition is
that $z \in \tkns$). The transition adds the entry $(n^i_\joint + 2)
\hpts (\tkn^0, \tkn^1, z^i)$ to $\hists$, thus snapshoting the values
of $\tkn^0$ and $\tkn^1$ from the pre-state.
%
% \[
% \small{
% \!\!\!\!
% %
% \begin{array}{r@{\ }c@{\ }l@{\ \ }c}
%   \tauadd({z^i}) & \eqdef &\qcl \jpts (\bal \hpts b \hunion c_i \hpts n_i
%                       \hunion c_{1-{i}} \hpts n_{1-i}) \aand \\[2pt]
% & &\qcl \spts (\tkns \hunion \set{z^i}, \hists) & 
%   \\[3pt]
%   & \rightsquigarrow &\qcl \jpts (\bal \hpts b \hunion c_i \hpts
%                             (n_i + 2)\hunion c_{1-{i}} \hpts n_{1-i}) \aand \\[2pt]
% & &\qcl \spts (\tkns, \hists \hunion (n_i + 2) \hpts (\tkns, \tkno, z^i))
% \end{array}
% }
% \]

%As required by FCSL metatheory, neither of the transitions modifies
%the other-state.
%
It is easy to check that both transitions preserve the state-space
invariants~(\ref{cn:si}), \ref{cn:state}--\ref{cn:ai}, and that their
effect on real state (with auxiliary state erased) are those of
\code{flip} and \code{fetchAndAdd2}.

\subsection{A Hoare spec for \texttt{getAndInc}}
\label{sec:spec-gaa}

We next provide a Hoare-style spec for \code{getAndInc}, and show how
it can be inferred from the specs of \code{flip} and
\code{fetchAndAdd2}, defined further in this section.  We use the
logical variable $\ikn$ and its variants to range over token sets, and
$\gist$ to range over histories.
%
\begin{comment}
In order to formulate these specs in an intuitive way, we will first
introduce some new notational conventions.

% \paragraph{Local naming conventions} 
% \label{sec:notat-state-proj}

In our Hoare-style assertions about network states, we will keep
referring to components and values of the \emph{current} state, which
is being constrained, as $\tkns$, $\hists$, \etc.
%
We will refer to a state's sets of even and odd tokens as $\tkn^0$ and
$\tkn^1$ and to actual values of its $c_0$ and $c_1$ as $n_0$
and~$n_1$.
%
For the ghost (logical) variables, appearing in Hoare triples, $\gist$
will range over histories, and $\ikn$ will range over sets of tokens.
\end{comment}
%
% We will use the logical assertion $\This{s}$ to capture the current
% state via a logical variable $s$. For stability, $\This{s}$ will hold
% on \emph{any} state, reachable from the \emph{fixed} state $s$ by
% taking an arbitrary number of transitions $\tauflip$ and $\tauadd$,
% executed by the environment (\ie, modifying \emph{other} and
% \emph{joint} parts of the state).
% %
% For such fixed ``snapshot'' states $s$, we will use the dot-notation,
% reminiscent to addressing object fields in Java, when referring to
% their components.
% %
% For instance, we will refer to the sets of tokens in self/other
% auxiliary state of $s$ as $s.\tkns$ and $s.\tkno$.
% %
% The introduction of $\Thisz$ is done any by weakening any state
% assertion $P$ to to $\exists s \ldot (\This{s}, P)$, where $s$ is
% fresh in $P$.
% %
% Indeed, all assertions about \emph{self}-components of the current
% state can be rephrased in terms of~$s$.
%
%\subsubsection{The spec of \code{getAndInc} and its components}
%\label{sec:spec-codeg-its}
%
%We ascribe the following Hoare-style spec to \code{getAndInc}:
%
\[
%
\tag{\normalsize \arabic{tags}}\refstepcounter{tags}\label{eq:qc-spec}
{\small
\!\!\!\!\!\!\!\! 
\begin{array}{c}
  \spec{\!\!
  \begin{array}{c}
    \tkns = \emptyset,
    \hists = \gists,
    \gisto \subseteq \histo,\\[2pt]
    \ikno \subseteq \tkno \hunion (\Tomb~\histo \setminus \Tomb~\gisto)
  \end{array}
  \!\!}
  \\\\[-6pt]
  \texttt{getAndInc()}
  % 
  \\[3pt]
  \spec{\!\!\!
  \begin{array}{c}
    \exists \iknh^0~\iknh^1~z \ldot \tkns = \emptyset, 
    \hists = \gists \hunion (\res + 2) \hpts (\iknh^0, \iknh^1,z), 
    \\[2pt]
    \gisto \subseteq \histo, \ikno \subseteq \tkno \hunion (\Tomb~\histo \setminus \Tomb~\gisto), 
    \\[2pt]
    \last~(\gists \hunion \gisto) < 
    \res + 2 + 2\ (|\iknh^0 \cap \ikno^0| + |\iknh^1 \cap
    \ikno^1|), 
    \\[2pt]
    \happrox~(\gists \hunion \gisto)~\res~\iknh^0~\iknh^1~z
  \end{array} 
  \!\!\!}@\ccon
%
\end{array}
}
\]
%
The precondition starts with an empty token set ($\tkns = \emptyset$),
and hence by framing, any set of tokens. The initial self-history
$\hists$ is set to an arbitrary $\gists$.\footnote{Alternatively, we
  could have also taken $\hists = \emptyset$, but the clients will
  require generalizing to $\hists = \gists$ by the FCSL's frame
  rule~\eqref{eq:frame}. To save space and simplify the discussion, we
  immediately frame \wrt the auxiliary $\hists$. Our examples do not
  require such client-side framing \wrt~$\tkns$.} The precondition
records the \emph{other} components of the initial state as
follows. First, $\gisto$ names (a subset of) $\histo$, to make it
stable under interference, as in Section~\ref{sec:overview}. Next, we
use $\ikno$ to name the (subset of) initially live tokens
$\tkno$. However, as $\tkno$ may shrink due to other threads spending
tokens, simply writing $\ikno \subseteq \tkno$ is unstable. Instead,
we write $\ikno \subseteq \tkno \hunion (\Tomb~\histo \setminus
\Tomb~\gisto)$ to account for the tokens spent by other threads as
well. The set $\tkno \hunion (\Tomb~\histo \setminus \Tomb~\gisto)$
only grows under interference, as new live tokens are generated, or
old live tokens are spent, making the inclusion of $\ikno$
stable.

%

The postcondition asserts that the final token set $\tkns$ is also
empty (\ie, the token that \code{getAndInc} generates by \code{flip},
is spent by the end). The history $\hists$ is increased by an entry
$(\res + 2) \hpts (\iknh^0, \iknh^1, z)$, corresponding to writing the
value of the result (plus two) into one of the network's counters,
snapshoting the tokens of that moment into $\iknh^0$ and $\iknh^1$,
and spending the token $z$ on the write. $\gisto$ is a subset of the
new value of $\histo$, and $\ikno$ is a subset of the new value of
$\tkno \hunion (\Tomb~\histo \setminus \Tomb~\gisto)$, by the already
discussed stability.

The next inequality describes where the history entry for $\res + 2$
is placed \wrt~the pre-state history $\gist = \gists \hunion
\gisto$. $\gist$ may have gaps arising due to out-of-order behavior of
the network, and $\res + 2$ may fill one such gap. However, there is a
bound on how far $\res$ (and hence $\res+2$) may be from the tail of
$\gist$, which we express as a function of the token-sets $\ikno$,
$\iknh^0$, $\iknh^1$. The intuition is similar to that of the bounds
in~\ref{cn:ai}. To explain it, let us assume that $\res$ is written
into $c_1$ and the last entry of $\hist$ (let us call it $t$) was
written into $c_0$. Then, we have a similar ``equation'' as in the
explanation in~\ref{cn:ai}:
\[
t + 2\ |\ikno^0 \setminus \iknh^0| + 2\ |\iknh^0 \cap \ikno^0| = \res
+ |\iknh^1 \cap \ikno^1| + (2 b_\joint -1)
\]
To advance $c_0$ to the moment when $\iknh^0$ and $\iknh^1$ were
recorded, we need to increase $t$ by $2\ |\ikno^0 \setminus
\iknh^0|$. After that, to advance both $c_0$ and $c_1$ to a quiescent
period, we have to spend the tokens in $|\iknh^0 \cap \ikno^0|$ (for
$c_0$) and $|\iknh^1 \cap \ikno^1|$ (for $c_1$). In the quiescent
period, $c_0$ and $c_1$ differ by $2 b_\joint - 1$. Moving $|\iknh^0
\cap \ikno^0|$ to the other side of the equation (while not changing
the sign), omitting $|\ikno^0 \setminus \iknh^0|$, and bounding the
value of $2 b_\joint - 1$ from above by $2$, we get:
\[
t < \res + 2\ |\iknh^0 \cap \ikno^0| + 2 \ |\iknh^1 \cap \ikno^1| + 2
\]
which we use in~(\ref{eq:qc-spec}). Being symmetric in $\ikn^0$ and
$\ikn^1$, the inequality has the pleasant property that it also holds
in the other three cases: when $\res$ is written into $c_0$ and $t$
into $c_1$, and when both $\res$ and $t$ are written into the same
counter, $c_0$ or $c_1$.

% The predicate $\strapprox$, stated next, summarizes several properties
% of the result and the newly introduced history entry, which will be
% crucial for reasoning about clients in Section~\ref{sec:qclients}:
% %
% \[
% \tag{\normalsize \arabic{tags}}\refstepcounter{tags}\label{eq:strapprox}
% %
% \!\!\!\!
% {\small{
% \begin{array}{l}
% \strapprox~\gists~\gisto~\ikno~m_0~m_1~\res~\iknh^0~\iknh^1~z
% ~\eqdef
% \\[2pt]
% ~~~~~~~~~~~~~~~~~~ \sapprox~\ikno~m_0~m_1~\res~\iknh^0~\iknh^1, \\[2pt]
% ~~~~~~~~~~~~~~~~~~ \happrox~(\gists \hunion \gisto)~\res~\iknh^0~\iknh^1~z,\\[2pt]
% ~~~~~~~~~~~~~~~~~~ \tapprox~(\gists \hunion \gisto)~\iknh^0~\iknh^1~z
% \end{array}
% }}
% \]
% %
% \[
% %
% \tag{\normalsize \arabic{tags}}\refstepcounter{tags}\label{eq:sapprox}
% %
% {\small{
% \begin{array}{l}
% \sapprox~\ikn~m_0~m_1~\res~\iknh^0~\iknh^1 ~\eqdef \\[2pt]
% %
% ~~~~  m_0 < \res + 2 + 2 \times (|\iknh^0 \cap \ikn^0| + |\iknh^1 \cap
%   \ikn^1|), \\[2pt]
% ~~~~ m_1 < \res + 2 + 2 \times (|\iknh^0 \cap \ikn^0| + |\iknh^1 \cap
%   \ikn^1|)
% \end{array}
% \hfill
% }}
% \]
% 

%\noindent
Finally, the predicate $\happrox$ provides some further bounds that we
will need in the proofs of the client code's properties.
%
\[ 
%
\tag{\normalsize \arabic{tags}}\refstepcounter{tags}\label{eq:happrox}
%
\!\!\!\!\!
{\small{
\begin{array}{l}
\happrox~\gist~\res~\iknh^0~\iknh^1~z \eqdef \hbox{}\\
% (\iknh^0 \hunion \iknh^1) \cap \Tomb~\gist = \emptyset,\\[2pt]
~~~~~~ \iknh^0 \hunion \iknh^1 \subseteq \tkno \hunion (\Tomb~\histo) \hunion
  \set{z},\\[2pt]
~~~~~~ \forall t~\ikn^0~\ikn^1 \ldot t \hpts (\ikn^0, \ikn^1, -) \subseteq \gist ~\implies\\[2pt]
~~~~~~~~~~~~
  z \notin \ikn^0 \hunion \ikn^1,~~ 
  t < \res + 2 + 2 \ (|\iknh^0 \cap \ikn^0| + |\iknh^1
  \cap \ikn^1|)
\end{array}
}}
\]
%
When instantiated with $\gist = \gists \hunion \gisto$, $\happrox$
says the following. The token set $\iknh^0 \hunion \iknh^1$ snapshot
when $\res+2$ was committed to history, is a subset of all the tokens
in post-state, including the live ones ($\tkno$), and
spent ones ($\Tomb~\histo \hunion \{z\}$).
%
Moreover, if $t$ is an entry in $\gist$, with contents $(\ikn^0,
\ikn^1, -)$, then: (1) $z \notin \ikn^0 \hunion \ikn^1$, because $z$
is a token generated when \code{getAndInc} executed
\code{flip}. Hence, $z$ must be fresh \wrt~any token sets used in the
pre-state history $\gist$; and (2) $t$, $\ikn^0$ and $\ikn^1$ must
satisfy the same bounds \wrt~$\res+2$, as those described for the last
history entry and sets $\ikno^0$ and $\ikno^1$.


%\noindent

%
% \begin{comment}
% \paragraph{Why the spec~\eqref{eq:qc-spec} is stable?}
% \label{sec:why-spec-eqrefeq:qc}

% The stability of the spec we ascribed to \code{getAndInc} follows from
% the following observations. First, all clauses in the pre- and
% postconditions that contsrain only \emph{self}-components of the state
% (\eg, $s.\hists = \gists$ or $\tkns = \emptyset$) are stable, since
% they cannot be affected by interference (which might change only
% \emph{other} and \emph{joint} components), as ensured by FCSL's
% meta-theory.
% %
% Second, the stability of all other clauses that also mention the
% \emph{other} component, follows from their \emph{monotonicity} with
% respect to interference. In particular, the union
% $\tkno \hunion \Tomb~\histo$, appearing also in the definition of
% $\happrox$, can only grow, while the union
% $\ikno \hunion \Tomb~\gisto$ is fixed.
% %
% Finally, the rest of the clauses mentions only values that are not
% components of the state being constrained (\eg, $\ikno$, $\gisto$,
% \etc) and, hence, are also unaffected by interference.
% %
% All these stability arguments are carried out as formal proofs in our
% Coq development, accompanying the paper.
% \end{comment}

\paragraph{How will the spec~\eqref{eq:qc-spec} be used?}

The clause $\hists\,{=}\,\gists \hunion (\res+2)\,{\mapsto}\,-$ of
\eqref{eq:qc-spec}, in conjunction with the resource
invariant~\ref{cn:hvalid}, ensures that any two calls to
\code{getAndInc}, sequential or concurrent, yield different history
entries, and hence a different result. This establishes the
requirement~\textbf{R1}, which we will not discuss further.

The inequality $\mathsf{last}(\gists \hunion \gisto)$ will provide
for~\textbf{R2} in client reasoning. To see how, consider the
particular case when $\ikno$ is empty, \ie, the pre-state is
quiescent. In that case, all the intersection in the inequality are
empty, and we can infer that the result (more precisely, $\res + 2$),
is larger than either counter's value in the pre-state. As we shall
see in Section~\ref{sec:qclients}, this captures the essence of QC.

Finally, the predicate $\happrox$~\eqref{eq:happrox} establishes a
bound for the ``out-of-order'' discrepancy between $\res$ and any
value $t$ committed to the history in the past, via the summand
$2\ (|\iknh^0 \cap \ikno^0| + |\iknh^1 \cap \ikno^1|)$. We will further
bound this summand using the expression $|\iknh^0 \hunion \iknh^1|$,
and the inclusion $\iknh^0 \hunion \iknh^1 \subseteq \tkno \hunion
\Tomb~\histo$ from~\eqref{eq:happrox}. All these bounds will
ultimately enable us to derive the requirement \textbf{R3}.
%in the clients.
%
% The approximation of the size of interference can be obtained via the
% relation, established by $\tapprox$~\eqref{eq:tapprox}, as we will
% soon show.

\subsubsection{Specifications of {\code{flip}} and
  {\code{fetchAndAdd2}}}
\label{sec:qacts}

The formal verification of the spec~\eqref{eq:qc-spec} follows by
sequential composition of its operations, \code{flip} and
\code{fetchAndAdd2}, to which we ascribe the following specs.
%
Both specs are obtained by relaxing the definitions of the transitions
from Section~\ref{sec:count-netw-prot}, \wrt~stability.
%
%
\[
%
%\tag{\normalsize \arabic{tags}}\refstepcounter{tags}\label{eq:flip-spec}
{\small
%\!\!\!\!\!\!\!\! 
\begin{array}{c}
  \spec{\!\!
  \begin{array}{c}
    \tkns = \emptyset,
    \hists = \gists,
    \gisto \subseteq \histo,\\[2pt]
    \ikno  \subseteq \tkno \hunion (\Tomb~\histo \setminus \Tomb~\gisto)
  \end{array}
  \!\!}
  \\\\[-6pt]
  \texttt{flip(}\bal\texttt{)}
  %  
  \\[3pt]
  \spec{\!\!
  \begin{array}{c}
    \exists b~z^b \ldot \res = (b, z^b)\aand
    \tkns = \set{z^b}\aand \hists = \gists \aand   \\[2pt]
    \gisto \subseteq \histo, \ikno \subseteq \tkno \hunion (\Tomb~\histo \setminus \Tomb~\gisto),\\[2pt]    
    \forall t~\ikn^0~\ikn^1 \ldot
    t \hpts (\ikn^0, \ikn^1, -) \subseteq (\gists \hunion \gisto) \Rightarrow z^b \notin \ikn^0 \hunion \ikn^1,
    \\[2pt]     
    \bapprox~(\last~(\gists \hunion \gisto)^0)~(\last~(\gists \hunion \gisto)^1)~\ikno 
  \end{array}
  \!\!}@\ccon
%
\end{array}
}
\]

\noindent
The precondition of \code{flip}'s matches the one of
\code{getAndInc}. The postcondition contains a clause with a new
predicate $\bapprox$, relating the last entries $m_0$ and $m_1$ of
either parity of the initial history $\gist = \gists \hunion \gisto$,
to the current values $n_\joint^0$ and $n_\joint^1$ of $c_0$ and
$c_1$.
%
\[
%
\!\!\!\!
{\small{
\begin{array}{l}
\bapprox~m_0~m_1~\ikno \eqdef \\[2pt]
%
  \begin{array}{l}
   m_0 \le n_\joint^0\aand
    m_1 + 2 \times |\ikno^1 \cap \tkn^1| < n_\joint^0 + 2 \times
  |\ikn^0 \cap  \tkn^0| + 2, \\[2pt]
   m_1 \le n_\joint^1\aand m_0 + 2 \times |\ikno^0 \cap \tkn^0| < n_\joint^1 + 2 \times
  |\ikn^1 \cap  \tkn^1| + 2 
  \end{array}
\end{array}
\hfill
}}
\]
%
The predicate says that the contents of $c_0$ and $c_1$ increases,
hence $m_0$ and $m_1$ are smaller or equal to the current values
$n_\joint^0$ and $n_\joint^1$, respectively. Moreover, when comparing
values of different parities (\ie, $m_1$ with $n_\joint^0$ and $m_0$
with $n_\joint^1$), we require bounds similar to the ones already
discussed in~\ref{cn:ai} and~\eqref{eq:qc-spec}, and expressed in
terms of token set $\ikno$ and $\tkn = \tkns \hunion \tkno$, that
capture the interference in the pre-state and post-state,
respectively. The predicate is internal to \esc{getAndInc}, and is not
used by, or even visible to, the clients.

The precondition of \code{fetchAndAdd2} is the same as \code{flip}'s
postcondition, and \code{fetchAndAdd2}'s post is the one of
\code{getAndInc}, so verifying the sequential composition is
straightforward.
%
% We note that the $\bapprox$ property is essential for deriving the
% inequalities from the postcondition~\eqref{eq:qc-spec}.
%
\[
%
%\tag{\normalsize \arabic{tags}}\refstepcounter{tags}\label{eq:add-spec}
{\small
%\!\!\!\!\!\!\!\!\!\! 
\begin{array}{c}
  \spec{\!\!
  \begin{array}{c}
   \tkns = \set{z^b}\aand \hists = \gists \aand   \\[2pt]
    \gisto \subseteq \histo, \ikno  \subseteq \tkno \hunion (\Tomb~\histo \setminus \Tomb~\gisto),\\[2pt]
    \forall t~\ikn^0~\ikn^1 \ldot
    t \hpts (\ikn^0, \ikn^1, -) \subseteq (\gists \hunion \gisto) \Rightarrow z^b \notin \ikn^0 \hunion \ikn^1,
    \\[2pt]    
    \bapprox~(\last~(\gists \hunion \gisto)^0)~(\last~(\gists \hunion \gisto)^1)~\ikno 
  \end{array}
  \!\!}
  \\\\[-5pt]
  \texttt{fetchAndAdd2($c_b, \specK{z^b}$)} 
  % 
  \\[3pt]
  {\normalsize{ 
  \specK{\{}~ {\small\texttt{getAndInc}}\specK{\text{'s post~\eqref{eq:qc-spec},
  instantiated with}~\gists, \ikno, \gisto\}}@\ccon
  }}
%
\end{array}
}
\]

\noindent
We note one peculiarity, however. In order to provide a provable spec
for \code{fetchAndAdd2}, we had to augment its signature with a
\emph{logical} parameter $\specK{z^b}$, representing the token,
obtained by executing \code{flip}, to be spent in incrementation of
$c_b$. While in most Hoare-style specs, logical variables scope over
the precondition and the postcondition, but do not appear in the code,
here we had to pass $z^b$ as a function argument.
%
This logical parameter serves purely for verification purposes, and
does not affect the result of the execution. Hence, in principle, it
can be safely erased, though our current formalization of FCSL in Coq
does not support such erasure.




\section{Verifying Counting Network's Clients}
\label{sec:qclients}

% We next demonstrate how the spec~\eqref{eq:qc-spec} can be used in
% clients, to establish properties \textbf{R2} and \textbf{R3}. These
% properties have been addressed in the previous work using dedicated
% consistency criteria of quiescent
% consistency~\cite{Aspnes-al:JACM94,Derrick-al:FM14} and quantitative
% quiescent consistency and
% quasi-linearizability~\cite{Afek-al:OPODIS10,Jagadeesan-Riely:ICALP14},
% but here we derive them compositionally, \ie, out of the spec of
% \code{getAndInc}. This will demonstrate the usefulness of Hoare logic
% in deriving properties related to the various flavors of quiescent
% consistency.

Following \textbf{\emph{Step 3}} of our verification method, we now
illustrate requirements \textbf{R2} and \textbf{R3} from the previous
section via two different clients which execute two sequential calls
to \code{getAndInc}. Both clients are higher-order, \ie, they are
parametrized by subprograms, which can be ``plugged in''.
%
The first client will exhibit a quiescence between the two calls, and
we will prove that the call results appear in order, as required by
\textbf{R2}. The second client will experience interference of a
program with a $N$ concurrent calls to \code{getAndInc}, and we will
derive a bound on the results in terms of $N$, as required by
\textbf{R3}.

Both our examples will rely on the general mechanism of hiding,
presented in Section~\ref{sec:background}, as a way to logically restrict the
interference on a concurrent object, in this case, a counting network,
in a lexically-scoped way.
%
To ``initialize'' the counting network data structure, we provide the
starting values for the shared heap ($h_0$) and for the history
($\gist_0$), assuming that the initial set of tokens is empty:
%
% Specifically, we will use the following derived rule:
% %
% {\small{
% \[
% \begin{array}{c}
% \spec{P}~e~\spec{Q} @ \ccon\\[2pt]
% \hline\\[-7pt]
% \!\!\!
% \specK{\{\heaps = \Phi_1(\heapj), \Phi_1(P)\}} \hide_{\Phi_1}~e \specK{\{\exists \Phi_2\ldot \heaps = \Phi_2(\heapj), \Phi_2(Q)\}} @ \cal P
% \end{array}
% \]
% }}
%
\[
\tag{\normalsize \arabic{tags}}\refstepcounter{tags}\label{eq:hide2}
{\small{
\begin{array}{r@{\ }c@{\ }l}
% \text{\normalsize{where}} &
% \Phi_1 & \eqdef & [\emptyset/\tkns, \gist_0/\hists, \heap_0/\heapj,
% \emptyset/\tkno, \histo/\hists] 
% \\[2pt]
\heap_0 & \eqdef & \bal \hpts 0 \hunion c_0 \hpts 0 \hunion c_1 \hpts 1     
\\[2pt]
\gist_0 & \eqdef & \set{0 \hpts (\set{0}, 0), 1 \hpts (\set{1}, 1)}
\end{array}
}}
\]
%
That is, $\gist_0$ provides the ``default'' history for the initial
values 0 and 1 of $c_0$ and $c_1$, with the corresponding tokens
represented by numbers 0 and 1.  As always with hiding, the
postcondition of the hidden program will imply that $\tkno$ and
$\histo$ are both empty, as there is no interference at the end.

% \gad{I think that here, again, the quotation marks around ``plugged in'',
%   ``initialize'', and ``default'' are unnecessary.}

\subsection{Exercising Quiescent Consistency}
\label{sec:qc-client}

\begin{figure}
\centering
\[
{\small{
\!\!\!\!\!\!\!\!
\begin{array}{c}
  \spec{\!\!
  \begin{array}{c}
    \tkns = \emptyset,
    \hists = \gists,
    \gisto \subseteq \histo, \Ic{\gisto}{\ikno}, \\[2pt]
    \ikno \subseteq \tkno \hunion (\Tomb~\histo \setminus \Tomb~\gisto)
  \end{array}
  \!\!}
\\\\[-5pt]
  \begin{tabular}{c || c}
   $\esc{getAndInc()}$ & ${\small{e_i}}$ 
\end{tabular}
\\\\[-5pt]
~~~~\spec{\!\!
\begin{array}{c}
  \exists \iknh~\gist_i \ldot  
  \tkns = \emptyset\aand \hists = \gbm{\gists \hunion \gist_i \hunion (\res.1 + 2) \hpts (\iknh, -)},\\[1pt]
  \gisto \subseteq \histo\aand \ikno \subseteq \tkno \hunion
  (\Tomb~\histo \setminus \Tomb~\gisto), \Ic{\gisto}{\ikno},\\[1pt]
  \last~(\gists \hunion \gisto)  < \gbm{\res.1} + 2 +
  2~|\iknh \cap \ikno|
%
\end{array}
\!\!} %@\ccon
%
\end{array}
}}  
\]
%
\caption{Parallel composition of \code{getAndInc} and~$e_i$ in~\eqref{eq:eqc}.}
  \label{fig:example1} 
\end{figure}
%



Our first client is the following program~$\eqc$:
%
\[
\tag{\normalsize \arabic{tags}}\refstepcounter{tags}\label{eq:eqc}
{\small{
\begin{array}{ll} 
\Num{1} & (\res_1, -) \Asgn (\esc{getAndInc()} ~||~ e_1) \esc{;} \\[1pt]
\Num{2} & (\res_2, -) \Asgn (\esc{getAndInc()} ~||~ e_2) \esc{;} \\[1pt]
\Num{3} &  \kw{return}~(\res_1, \res_2) 
\end{array}
}}
\]
%
Each of the calls to \esc{getAndInc} interferes with either $e_1$ or
$e_2$, but in the absence of \emph{external} interference, the
quiescent state is reached between the lines 1 and 2. Hence, after
executing $\hide~\eqc$, it should be $\res_1 < \res_2$, following
\textbf{R2}.

The programs $e_1$ and $e_2$ can invoke \code{getAndInc} and modify
the counters concurrently with the two calls of $\eqc$, which we
capture by giving both the following generic spec:
%
\[
%
\tag{\normalsize \arabic{tags}}\refstepcounter{tags}\label{eq:eispec}
{\small
\!\!\!\!\!\!\!\! 
\begin{array}{c}
  \spec{~
  \hists = \emptyset\aand
  \tkns = \emptyset\aand
   \ikn \subseteq \tkno \hunion \Tomb~\histo
  ~}
  \\[1pt]
  e_i
  % 
  \\[1pt]
  \spec{\!\!\!
  \begin{array}{c}
    \exists \gist_i \ldot \hists = \gist_i\aand 
    \tkns = \emptyset\aand 
    \ikn \subseteq \tkno \hunion \Tomb~\histo
  \end{array} 
  \!\!\!} %@\ccon
%
\end{array}
}
\]
%
The postcondition allows for a number of increments via calls to
\esc{getAndInc}, which is reflected in the addition $\gist_i$ to
$\hists$. However, all such calls are required to be \emph{finished}
by the end of $e_i$ ($\tkns = \emptyset$). As customary by now, we use
the logical variable $\ikn$ to name the initial set of \emph{other}
tokens.

Figure~\ref{fig:example1} provides a spec for each of the parallel
compositions in the program~\eqref{eq:eqc}, proved via the
corresponding FCSL inference rule for parallel
composition~\eqref{eq:parrule}.
%
The spec is very similar to~\eqref{eq:qc-spec} with the differences
highlighted via gray boxes: (a) the self-history $\hists$ is increased
by $e_i$'s contribution $\gist_i$ in addition to the entry, introduced
by \code{getAndInc}, (b) the result of the parallel composition is a
pair, but we only constrain its first component $\res.1$, resulting
from the left subprogram. We also drop the last conjunct with
$\happrox$ from~\eqref{eq:qc-spec}, which we won't require for this
example.

%



Next, we use the spec from Figure~\ref{fig:example1} to specify and
verify the program $\eqc$, so far \emph{assuming} external
interference.
%
\[
\!\!\!
{\small{
\begin{array}{c}
\!\!\!\!\!
\spec{\!\!
  \begin{array}{c}
    % \tkns = \emptyset,
    % \gisto \subseteq \histo, \ikno \subseteq \tkno \hunion (\Tomb~\histo \setminus \Tomb~\gisto)\\[2pt]
    % \hists = \gist_0,
    \mbox{Fig.~\ref{fig:example1}'s precondition with $\gists := \gist_0$, $\gisto :=
      \histo$, and $\ikno := \tkno$}
  \end{array}
  \!\!}
  ~\comm{P}
  \\\\[-6pt]
  (\res_1, -) \Asgn (\esc{getAndInc()} ~||~ e_1) \esc{;}
  \\[3pt]
\!\!\!\!{{
\spec{\!\!\!\!
\begin{array}{c}
 \exists \gist_1\ldot \tkns = \emptyset\aand \hists = \gists',~\ldots
% \ikno \subseteq \tkno \hunion (\Tomb~\histo \setminus \Tomb~\gisto),
\\[2pt]
\mbox{where 
 $\gbm{\gists' = \gist_0 \hunion
\gist_1 \hunion (\res_1 + 2)\mapsto -}$, $\gisto := \histo$ and $\ikno :=
\tkno$} 
% \\[2pt]
% \mbox{hence $\Ic{\gisto}{\ikno}$}
  \end{array}
\!\!\!\!}
}}
%
\\\\[-5pt]
(\res_2, -) \Asgn (\esc{getAndInc()} ~||~ e_2) \esc{;}      
\\[3pt]
\spec{\!\!\!\!
\begin{array}{c}
\exists \gist_1~\gist_2~\iknh \ldot     
%
\tkns = \emptyset \aand 
%%\hists = \gists' \hunion \gist_2 \hunion (\res_2 + 2)\mapsto (\iknh, -)\aand\hbox{}\\[2pt] 
\gbm{\ikno \subseteq \tkno \hunion (\Tomb~\histo \setminus \Tomb~\gisto)},\\[1pt]
\gbm{\last~(\gists' \hunion \gisto) < \res_2 + 2 + 2~|\iknh \cap \ikno|},~\ldots 
\end{array}
\!\!\!\!\!}~\comm{Q}
\\\\[-7pt]
~~~~~~~~~~~\kw{return}~(\res_1, \res_2); ~\comm{=: \res} 
\\[2pt]
\spec{~Q(\res.1/\res_1, \res.2/\res_2)~} %@\ccon
\end{array}
}} 
\]
%
We start by instantiating the logical variables $\gists$, $\gisto$ and
$\ikno$ from Figure~\ref{fig:example1} with $\gist_0$, \emph{current}
$\histo$ and $\tkno$, respectively, naming the obtained precondition
$P$.
%
In the following assertion we focus on the clauses constraining
$\tkns$ and $\hists$. To verify the second call, we instantiate
$\gists$, $\gisto$ and $\ikno$ from Figure~\ref{fig:example1} with
$\gists' = \gist_0 \hunion \gist_1 \hunion (\res_1 + 2)\mapsto -$,
\emph{current} $\histo$ and $\tkno$, correspondingly, obtaining the
postcondition, which we name~$Q$.

The inequality in the postcondition $Q$ gives the boundary on the
out-of-order position of $\res_2$ with respect to the \emph{last}
value in the history captured in between the two parallel
compositions. The boundary is given via the size of intersection of
the two sets of tokens: snapshot ($\iknh$) and ``alive'' between the
calls ($\ikno$).
%
Now, to ensure the absence of external interference, we consider the
program $(\hide~\eqc)$.
%
By the general property of hiding (Section~\ref{sec:background}), we
know that at the final state there is no interference, hence $\tkno =
\emptyset$ and $\histo = \emptyset$ in $Q$.
%
Therefore, from the set inclusion on $\ikno$ in $Q$ (the grayed part),
we deduce that $\ikno = \emptyset$.
%
As a consequence, the intersection $\iknh \cap \ikno = \emptyset$, so
from the inequality we obtain
%
\[
%
\tag{\normalsize \arabic{tags}}\refstepcounter{tags}\label{eq:tada1}
%
%{\small{
\begin{array}{c}
 \last~(\gists' \hunion \gisto) < \res.2 + 2
\end{array}
\hfill
%}}
\]
%
%\noindent
%
But $\gists'$ is defined as $(\res.1+2)\mapsto -~\hunion\ldots$,
hence, $\res.1 + 2 \in \mathsf{dom}\ \gists'$, and thus $\res.1 + 2
\le \mathsf{last}\ \gists'$. Even more:
%
\[
%
\tag{\normalsize \arabic{tags}}\refstepcounter{tags}\label{eq:tada2}
%
%{\small{
\begin{array}{c}
\res.1 + 2 \le \last~(\gists' \hunion \gisto).
\end{array}
\hfill
%}}
\]
%
From~\eqref{eq:tada1} and~\eqref{eq:tada2} follows the result
\textbf{R2}: $\res.1 < \res.2$.

\subsection{Proving Quantitative Bounds}
\label{sec:qqc-client}

We next show how the spec~\eqref{eq:qc-spec} also obtains quantitative
bounds on the out-of-order anomalies in terms of a number of running
threads in the following program $\eqqc$:
%
\[
\tag{\normalsize \arabic{tags}}\refstepcounter{tags}\label{eq:eqqc}
{\small{
\begin{tabular}{l || l}
$
\begin{array}{ll} 
\Num{1} & \res_1 \Asgn \esc{getAndInc();} \\[1pt]
\Num{2} & \res_2 \Asgn \esc{getAndInc();}  \\[1pt]
\Num{3} & \kw{return}~(\res_1, \res_2)
\end{array}
$  
&
$~~~e$
\end{tabular} 
}}
\]
%
The $e$'s spec says that the \emph{number} of calls to \esc{getAndInc}
in~$e$ (\ie, the size of interference $e$ exhibits) is some fixed $N$:
%
\[
%
\tag{\normalsize \arabic{tags}}\refstepcounter{tags}\label{eq:espec}
{\small
\!\!\!\!\!\!\!\!\!  
\begin{array}{c}
  \spec{
  \tkns = \emptyset,
  \hists = \gists }
~  e
~  \spec{\!\!\!
  \begin{array}{c}
    \exists \gist \ldot 
    \tkns = \emptyset,
    \hists = \gists \hunion \gist,
    |\gist| = N
  \end{array} 
  \!\!\!} %@\ccon
%
\end{array}
}
\]
%
Our goal is to prove that in the absence of external interference for
$\eqqc$, $\res_1 < \res_2 + 2 \ N$ (requirement \textbf{R3}).

\begin{figure}
\centering
%    
\[
\!\!\!
{\small{
\begin{array}{c}
  \spec{~
    \mbox{{\normalsize{\eqref{eq:qc-spec}'}}s precondition with $\gists := \gist_0$, $\gisto :=
      \histo$, and $\ikno := \tkno$}~}
% \\\\[-6pt]
% \spec{\!\!
%   \begin{array}{c}
%     \tkns = \emptyset,
%     \hists = \gists,
%     \gisto \subseteq \histo,
%     \ikno \subseteq \tkno \hunion (\Tomb~\histo \setminus \Tomb~\gisto)\\[2pt]
%     \mbox{where $\gisto = \histo$ and $\ikno = \tkno$, hence $\Ic{\gisto}{\ikno}$}
% %    \histo \subseteq \histo,\\[2pt]
% %    \tkno \hunion \Tomb~\histo \subseteq \tkno \hunion \Tomb~\histo
%   \end{array}
%   \!\!}
\\\\[-6pt]
\res_1 \Asgn \esc{getAndInc();}
\\[3pt]
\spec{\!\!
\begin{array}{c}
   \exists \ikn \ldot 
   \tkns = \emptyset, 
   \hists = \gists',
   % \gisto \subseteq \histo, \ikno \subseteq \tkno \hunion
   % (\Tomb~\histo \setminus \Tomb~\gisto)
   \ldots
   \\[2pt]
   \mbox{where $\gbm{\gists' = \gist_0 \hunion (\res_1 + 2) \hpts
       (\ikn, -)}$} 
   % , $\gisto = \histo$ and $\ikno = \tkno$, \ldots}
% \\[2pt]
%    \mbox{hence $\Ic{\gisto}{\ikno}$}
  \end{array}
  \!\!}%
%\\\\[-6pt]
%\spec{\!\!\!
%  \begin{array}{c}
%  \tkns = \emptyset,  
%  \hists = \gists \hunion (\res_1 \!+\! 2) \!\hpts\! (\ikn^0, \ikn^1, -), 
%  \gists' := \hists,\\[2pt] 
%  \gisto := \histo, 
%  \ikno := \tkno,
%  \ikno \hunion \Tomb~\gisto \subseteq \tkno \hunion \Tomb~\histo
%  \end{array}
%  \!\!\!}%
\\\\[-5pt] 
\res_2 \Asgn \esc{getAndInc();}
\\[3pt]
\spec{\!\!\!
\begin{array}{c}
  \exists \iknh~z \ldot     
  % \tkns = \emptyset, \hists = \gists' \hunion (\res_2+2) \mapsto (\iknh, z),\\[2pt]
  \happrox (\gists' \hunion \gisto)~\res_2~\iknh~z, \ldots
\end{array}
\!\!\!}
\\\\[-5pt]
\spec{\!\!\!
\begin{array}{c}
  \exists \iknh~z \ldot     
  % \tkns = \emptyset, \hists = \gists' \hunion (\res_2+2) \mapsto (\iknh^0, \iknh^1, z),\\[2pt]
  \gbm{\iknh \subseteq \tkno \hunion (\Tomb~\histo) \hunion \set{z}},
  z \notin \ikn,\\[2pt] 
  \res_1 + 2 < \res_2 + 2 + 2~|\iknh \cap \ikn|
\end{array}
\!\!\!}
\\\\[-5pt]  
~~~~~~~~~~~~~~~
\kw{return}~(\res_1, \res_2) ~\comm{=: \res}
\\\\[-5pt]
\spec{\!\!\!
\begin{array}{c}
  \res.1 < \res.2 + 2 \ |\tkno \hunion \Tomb~\histo| 
  %
  %\aand\\[1pt] 
  % \tkns = \emptyset\aand
  % \hists = \gists \hunion (\res_1 + 2) \hpts -
  % \hunion (\res_2 + 2) \hpts -
  \end{array}
  \!\!\!} %@\ccon
\end{array}
}} 
\]
%
%
\caption{Proof outline of sequential composition in~\eqref{eq:eqqc}.}
\label{fig:proof2}
\end{figure}

We first verify the sequential composition of the two calls
in~\eqref{eq:eqqc}; the proof outline is in
Figure~\ref{fig:proof2}. 
%
% \ab{Can this proof outline be cut significantly/removed? Can we just
%   give the overall pre-post specs and go directly to the verification
%   of $\eqqc$?}
%
As previously, we start by instantiating the logical variables
$\gists$, $\gisto$ and $\ikno$ from spec~\eqref{eq:qc-spec} with
$\gists$, $\histo$ and $\tkno$, respectively. In the assertion,
resulting by of the first \esc{getAndInc}, we keep only the clauses
involving $\tkns$ and $\hists$, dropping the rest.
%
To verify the second \esc{getAndInc} call, we instantiate $\gists$,
$\gisto$ and $\ikno$ with $\gists' = \gists \hunion (\res_1+2) \mapsto
(\ikn, -)$, current $\histo$ and $\tkno$.

In the postcondition of the second call to \esc{getAndInc}, we focus
on the $\happrox~(\gists' \hunion \gisto)~\res_2~\iknh~z$ clause,
where $\iknh$ is the set of tokens snapshot when contributing
$\res_2+2$.
%
Unfolding the definition of $\happrox$ from~\eqref{eq:happrox}, we
obtain $\iknh \subseteq \tkno \hunion \Tomb~\histo
\hunion\{z\}$. Also, using $(\res_1 +2)\mapsto (\ikn, -)$ in the
implication that the unfolding obtains, we get $z \notin \ikn$ and
\[
\tag{\normalsize \arabic{tags}}\refstepcounter{tags}\label{eq:le0}
{\small{\res_1 + 2 < \res_2 + 2 + 2~|\iknh \cap \ikn|
}}
\]
%
Now we use the following trivial fact to simplify.
%
\vspace{8pt}
%
\begin{lemma}
\label{lm:intersect2}
If $z \in \iknh$ and $z \notin \ikn$, then $|\iknh \cap \ikn| \le
|\iknh| - 1$.
\end{lemma}
%
%
\vspace{8pt}
%
\noindent
Using the invariant~\ref{cn:ti1}, Lemma~\ref{lm:intersect2}
derives
$|\iknh \cap \ikn| \le |\iknh| - 1$
%
after which, the inclusion $\iknh \subseteq \tkno
\hunion \Tomb~\histo \hunion \set{z}$ leads to
%
\[
\tag{\normalsize \arabic{tags}}\refstepcounter{tags}\label{eq:le5}
{\small{
|\iknh \cap \ikn| \le |\tkno \hunion \Tomb~\histo|}}
\]
%
Combined with~\eqref{eq:le0}, this gives us $\res_1 < \res_2 + 2 \
|\tkno \hunion \Tomb~\histo|$, as shown in Figure~\ref{fig:proof2}'s
postcondition. In words, it asserts that the discrepancy between
$\res.1$ and $\res.2$ is bounded by the size of the tokens, which are
either held by the interfering threads at the end or are spent.
%
% \gad{low-priority right now, but still: the inequality above spreads
%   across a linebreak, and worse, in the middle of the arguments for an
%   operator.}
%
% \is{won't fix}

\begin{figure}[t]
  \centering
\[
{\small{
\!\!\!\!\!\!\!\!
\begin{array}{c}
  ~~~~~~~~\spec{~
  \tkns = \emptyset,
  \hists = \gist_0, \ldots
~} ~\comm{P}
\\[2pt]
  \begin{tabular}{c || c}
$
\spec{\!\!\!
    \begin{array}{c}
    \tkns = \emptyset,
    \hists = \gist_0
  \end{array}\!\!\!}
$
&
$
\spec{\!\!\!\begin{array}{c}
    \tkns = \emptyset,
    \hists = \emptyset
  \end{array}\!\!\!}
$
\\[3pt]
   $\begin{array}{l}
      \res_1 \Asgn \esc{getAndInc();}\\[1pt]
      \res_2 \Asgn \esc{getAndInc();}\\[1pt]
      \kw{return}~(\res_1, \res_2) ~\comm{ =: \res}\!\!\!
    \end{array}$
\!\!\!\!\!\!
& ${\small{e}}$ 
\\\\[-5pt] 
$
\spec{\!\!\!
{\small{
  \begin{array}{c}
    \res.1 < \res.2 + 2 \ |\tkno \hunion \Tomb~\histo|
  \end{array}
}}
  \!\!\!}\!\!$
\!\!
&
\!\!$\spec{\!\!\!
{{
  \begin{array}{c}
    \exists \gist \ldot 
    \hists = \gist, 
    |\gist| = N,
    \ldots
  \end{array}
}}
\!\!\!}$
\end{tabular}
\\\\[-5pt]
\comm{\res_1 := \res.1.1, \res_2 := \res.1.2}
\\\\[-6pt]
\spec{
\res_1 < \res_2 + 2 \ |\tkno \hunion \Tomb~(\histo \hunion \gist)|
}
\\[3pt]
~~~~~~~~\spec{
\res_1 < \res_2 + 2 \ |\tkno \hunion \Tomb~\histo| + 2 \ N
} ~\comm{Q}
%
\end{array}
}}  
\]
  \caption{Proof outline for the $\eqqc$ program.}
  \label{fig:eqqcproof}
\end{figure}

Figure~\ref{fig:eqqcproof} shows the proof outline for $\eqqc$ via the
spec from Figure~\ref{fig:proof2}.
%
By the parallel composition rule~\eqref{eq:parrule}, the precondition
splits into two subjective views, where we send the initial history
$\gist_0$ to the left thread, and the empty history to the right
thread. The proof from Figure~\ref{fig:proof2} then applies to the
left thread, and the spec~\eqref{eq:espec} applies to the right
one. Final $\histo$ of the left thread is the union of $\histo$ from
the joined thread with $\gist$, since the environment of the left
thread includes the right thread and of the join. Rewriting by this
property in the postcondition of the left thread gives us the post of
the joint thread: $\res_1 < \res_2 +2\ |\tkno \hunion \Tomb~(\histo
\hunion \gist)|$, which we can next simplify into
\[
\res_1 < \res_2 + 2\ |\tkno \hunion \Tomb~\histo| + 2\ N
\]
because $\Tomb$ distributes over $\hunion$, and $|\Tomb~\gist| =
|\gist| = N$. Finally, we restrict the external interference by
considering $(\hide~\eqqc)$. From the properties of hiding,
we deduce that $\tkno$ and $\histo$ in $Q$ are empty, hence we can
simplify into $\res_1 < \res_2 + 2 \ N$, which is the desired
result~\textbf{R3}.
%
%\gad{The same happens here, just before the highlighted math.}

\section{Discussion}
%\section{Client-side reasoning}
\label{sc:discussion}

\begin{comment}
%\gad{Section 4 should emphasize that we are hiding colors,
%  end-points, and whatever `` internal'' aspects of the concurroid in
%  client reasoning. This might require to connet with the proof
%  outline from Figure~\ref{fig:weird:code} in the appendix. We should
%  say that this internal only ``leak'' through the
%  definition/instantiations of $\stableorder$.}

%% \gad{This section now needs two parts: (i) a first part about the
%%   clients and (ii) a second part discussing the \emph{principality} of
%%   scan's specs}

%% We already argued in Section~\ref{sc:formal} that the Hoare triples
%% from Figure~\ref{fig:specs} capture precisely what linearizability
%% would be used for, namely that the operations of the snapshot object
%% can be sequenced. The triples do so without the intermediary
%% sequential specifications for {\tt write} and {\tt scan}, and the
%% attending simulation proofs.  In this section, we argue that our
%% approach via Hoare logic can still do a bit more, specifically in the
%% cases of client side reasoning, and reasoning about parallel
%% composition.

%% By relying on FCSL, we immediately inherit the separation logic
%% mechanism for reasoning about programs with nested parallel
%% composition (i.e., dynamic, well-bracketed, forking and joining). We
%% refer the interested reader to the Appendix~\ref{sc:background} for a
%% brief background on FCSL, and the separation-logic style inference
%% rule for parallel composition that requires that program state, real
%% and auxiliary, satisfies the properties of a Partial Commutative
%% Monoid, or PCM (and indeed, our histories form a PCM under the
%% operation of disjoint union, with the empty history as a unit). This
%% gave us a thread-local way to reason about programs, via local
%% variables such as $\histS$ and $\histO$, rather than relying on global
%% abstractions, such as \emph{thread-id}s, to specify the behavior of
%% each thread.  This is in contrast to linearizability, whose very
%% definition requires identifying threads by thread-ids, thus focusing
%% on programs that are a top-level parallel composition of a fixed
%% number of sequential threads, but not providing convenient
%% abstractions for reasoning about nested parallel compositions.

%As a first point, we argue that the pre- and postcondition ascribed
%to {\tt write} and {\tt scan} in Section~\ref{sc:formal} are
%\emph{principal}, in that they can be used in larger program contexts
%without modification. This is not the case in most other verification
%methods, where programs, once verified, typically still have to be
%refactored wrt.~auxiliary state and code \emph{on a per-context
%basis}~\cite{Owicki-Gries:CACM76,Jacobs-Piessens:POPL11}. By using
%\emph{local} variables such as $\histS$ and $\histO$ to name the
%per-thread history, we avoid the need for refactoring. Such locality
%is supported by the FCSL inference rule for parallel
%composition~\cite{LeyWild-Nanevski:POPL13,Nanevski-al:ESOP14}. The
%rule is somewhat unusual, but please see Appendix~\ref{sc:background}
%for a brief description. What matters, however, is that reasoning in
%clients is done strictly out of the specs of {\tt write} and {\tt
%scan}. Notice that, while these expose the existence of the logical
%ordering $\stableorder$ and how some individual events may be related
%in it, they \emph{do not} expose the actual definition of
%$\stableorder$ in terms of colors, or any other internal of the
%auxiliary state and code. Other snapshot implementations can provide
%their own definition of $\stableorder$.

\gad{This part was not useful in Section~\ref{sc:clients}.}

As a brief example of reasoning about parallel composition, it is
possible to prove that the program
%
$ 
P_1 = {\mathtt{write}}\ (x, 1) \parallel {\mathtt{write}}\ (y, 2)
$
%
satisfies the spec which says that two events are executed, in an
unspecified order. Then we can reason about a larger program, say
$
  P_2 = P_1; {\mathtt{write}}\ (x, 3)
$
out of that spec, to prove that $P_2$ executes three events, with the
write of $3$ appearing last. 
%
Moreover, the \emph{substitution principle} holds; that is, the proof
remains valid, even if we replace $P_1$ by another program, say
%
$
P'_1 = {\mathtt{write}}\ (x, 1); {\mathtt{write}}\ (y, 2)
$
which satisfies a stronger spec (explicitly ordering the two events),
but which \emph{can be weakened} to the spec of $P_1$ by forgetting the
ordering by means of strengthening the precondition and weakening the
postcondition, as customary in Hoare logic. Thus, our client proofs
\emph{can ignore} the internal thread-structure of component programs.

\gad{The rest of this section will have to be merged either with the
  Related Work, or the conclusions}

\end{comment}

\begin{wrapfigure}[11]{t}[15pt]{0.5\textwidth} 
\vspace{-15pt}
%% \begin{figure}
%
%\centering  
%\begin{tabular}{l}
% 
\begin{minipage}[t]{.5\textwidth}
\[
\begin{array}{rl}
\num{1} & \esc{scan}\ () : ( A \times A )~ \{ \\ 
\num{2} & ~~~ (\var{cx}, \var{vx}) \tbnd \act{read}(\x);\\
\num{3} & ~~~ (\var{cy}, \_ ) \tbnd \act{read}(\y);\\
\num{5} & ~~~ (\_ , \var{tx}) \tbnd \act{read}(\x);\\
\num{5} & ~~~ \kw{if}\ vx = tx \\
\num{6} & ~~~ \kw{then}\ \kw{return}\ (\var{cx},\var{cy})\\
\num{7} & ~~~ \kw{else}\ \esc{scan}\ (); \}
\end{array}
\]
\end{minipage}
%
%\end{tabular}
%
\caption{{\tt Scan} using versions.}
\label{fig:readpair}
%\end{figure}
\end{wrapfigure}

% \gad{\textbf seems to have no efect whatsoever on alttt}
% \gad{unified action-names not to leave spaces before arguments}





The substitution principle can be pushed further. In particular, we
can use a different snapshot algorithm, without modifying the proofs
of $P_1$, $P_2$, $P'_1$ or any other client, as long as \jywrite\ and
\jyscan\ satisfy the expected specs.
%
We have confirmed this property on the toy example given in
Figure~\ref{fig:readpair} (we present only \jyscan, as \jywrite\ is
trivial)~\cite{SergeyNB+ESOP15}. In this example, the snapshot
structure consists of pointers $x$ and $y$ storing tuples $(c_x, v_x)$
and $(c_y, v_y)$, respectively. $c_x$ and $c_y$ are the payload of $x$
and $y$, whereas $v_x$ and $v_y$ are version numbers, internal to the
structure. Writes to $x$ and $y$ increment the version number. {\tt
  Scan} reads $x$, $y$ and $x$ again, in succession, but avoids
snapshot inconsistency by restarting if the two version numbers of $x$
differ. The definition of $\stableorder$ used to satisfy the specs
equals the real time one, as no dynamic reordering is needed.

%\gad{Why are we not calling read-pair by it's name?}
%%
%\an{Well, because we're trying to make the parallel with the Jayanti's
%  code. By calling it {\tt scan}, the connection is immediately
%  made. Speaking of names, I'm more concerned that we name with {\tt
%    write} two different things. The primitive memory operation should
%  probably be renamed into $x := v$ or {\tt assign}.}
%%
%\gad{I understand. But I was refering to the algorithm and not the
%  methods.}

%% \gad{Add HERE!: Clients}

\gad{Add HERE!: Readpair's proof outline w/ this paper's spec and
  instance of getters/ stable order etc.}

\gad{Reviewers --and Ralf-- complain about its not quite obvious that
  our spec is useful. We need to make an effort here to convice them
  that our assertions are useful for clients.}

  %% \gad{The simple sequential {\tt two\_scan} example will help here,
  %%   as it showcases the need for ``sequential`` and the stable ideal
  %%   inclusion on histories. }

\gad{Aleks' comments on 04/10 wrt. why our spec is better than what
  linearizability provides, even if the spec looks a priory more
  complicated. \\ We could reduce the spec by (rule of consequence)
  weakening, to get the assertion of the sequential spec $r
  =\mathsf{eval}\ \hist $, but that would not be useful for client
  reasoning, as you would be losing the information needed to combine
  the client in parallel with others: take the example in Figure~2 --
  the client, we've (almost) proved in Coq-- and a linearization of
  the client on the left and the center {\tt l: write x 2; c: scan; l:
    write y 1 } and do a proof using the sequential spec. Now if you
  are given a different linearization, {\tt l: write x 2; l: write y
    1; c: scan () }, you need to re do the proof. Moreover, there is
  no way to scale up adding more threads, if you add the client on the
  right, r, none of the proofs youy have can be reused. \\ In our
  setting, and with our specs, client reasoning depends solely on the
  API for scan and write, regardless of the different linearizations
  of a program. We use this to our advantage while building up the
  proof for the client in Figure~2.  }

%% \gad{Structure the proof outlines for the clients in the same why we
%%   have done in coq}
%%
%% \paragraph{Clients} \gad{Let's see what we have right now:}  

%% \begin{enumerate}
%%   \item[i] two sequential scans, and a proof that whatever the second
%%     scanner returned is at least as new as the second, and moreover
%%     the two snapshots are consistent, ie, the snapshot in the first
%%     call is still a snapshot with the histories and orderings defined
%%     by the second call. Morever, the stable order 

%%   \item[ii] The composite client building up to Figure~2.
    
%%   \item[iii] The latter with hide? \gad{We might not have time to hack
%%     it in in Coq, but we could do it by hand, and then upload it}.
 
%% \end{enumerate}

\section{Mechanization and Evaluation}
\label{sec:evaluation}

% \ab{This is an important section. It needs to convey what was
%   difficult, what was easy in the implementation. How were the
%   difficulties mitigated via the techniques developed in the previous
%   sections? What interesting proof engineering that needed to be done?
%   Ideally, these discussions should provide context/understanding of
%   the numbers in the Table. What are the current limitations of the
%   implementation? Make a separate paragraph for limitations perhaps
%   starting ``At the moment'' below.}
%
% \is{Mostly addressed in revised writing.}

We have mechanized the specs and the proofs of all the examples from
this paper~\cite{fcsl-site}, taking advantage of the fact that FCSL
has been recently implemented as a tool for concurrency
verification~\cite{Sergey-al:PLDI15} on top of the Coq proof
assistant~\cite{Coq-manual}.

Table~\ref{tab:locs} summarizes the statistics with respect to our
mechanization in terms of lines of code and compilation times. 
%
The examples were proof-checked on a 3.1~GHz Intel Core~i7 OS~X
machine with 16 Gb RAM, using Coq~8.5pl2 and Ssreflect
1.6~\cite{Gonthier-al:TR}.
%
As the table indicates, a large fraction of the implementation is
dedicated to proofs of preservation of resource
invariants~(\textsf{Inv}), \ie, checking that the actual
implementations do not ``go wrong''.
%
In our experience, these parts of the development are the most tricky,
as they require library-specific insights to define and reason about
auxiliary histories.
%
Since FCSL is a general-purpose verification framework, which does not
target any specific class of programs or properties, we had to prove
problem-specific facts, \eg, lemmas about histories of a particular
kind (\textsf{Facts}), and to establish the specs of interest stable
(\textsf{Stab}). Once this infrastructure has been developed, the
proofs of main procedures turned out to be relatively small
(\textsf{Main}).
%
% \gad{I don't see the point of having the {\bf Build} column in
%   Table~\ref{tab:locs}: First of all, I don't think that time to
%   compile a proof in Coq is relevant at all. From an engineering
%   perspective, it might perhaps make more sense to estimate the amount
%   of man-hours that took to discharge them. Second, because we then
%   would have to explain why the exchanger takes half the time than the
%   counter, given that they are similar in size. And I don't think we
%   want to get into describing the design, the differences in ``style''
%   between the two, and discussing why that might be affecting the
%   buildtime in detail here.}
%
% \is{Sorry, but since it's PLDI, we have to give build times, even
%   though we're out of space to explain some specific performance
%   phenomena}

{
%\setlength{\belowcaptionskip}{-1pt} 
\begin{table}
{%\footnotesize
\sffamily\small % tabular data either 10pt times, or 9pt helvetica
\centering
\begin{tabular}{|@{\ }l@{\ }||@{\ }c@{\ }|@{\ }c@{\ }|@{\ }c@{\ }|@{\ }c@{\ }|@{\ }c@{\ }||@{\ }r@{\ }|}
  \hline
  \textbf{Program} &  
                     {Facts} & {Inv} &
                                       {Stab} & {Main} & \textbf{Total}
  & \textbf{Build~~~}    
  \\ \hline \hline 
  Exchanger \hfill (\S \ref{sec:exchanger}) & 365 & 1085 & 446 & 162 & 2058 & 4m~~46s
  \\
  Exch. Client \hfill (\S \ref{sec:cal}) & 258 & -- &--& 182 & 440 & 57s
  \\
  Count. Netw. \hfill (\S \ref{sec:counting}) & 379 & 785 & 688 & 27 & 1879 & 12m~23s
  \\
  CN Client 1 \hfill (\S \ref{sec:qc-client}) & 141 &--&--& 180  & 321 & 3m~11s
  \\
  CN Client 2 \hfill (\S \ref{sec:qqc-client})& 115 &--&--& 259 & 374 & 3m~~~9s 
  \\[2pt] \hline
\end{tabular}
}
\caption{
  Mechanization of the examples: lines of code for program-specific facts \intab{Facts},
  resource invariants and transitions \intab{Inv}, 
  stability proofs for desired specs \intab{Stab}, spec and proof sizes for main
  functions \intab{Main}, total LOC count \intab{\textbf{Total}}, and build
  times \intab{\textbf{Build}}. The ``--'' entries indicate the
  components that were not needed for the example.
} 
\label{tab:locs}
%\vspace{-15pt}
\end{table}}

Fortunately, trickiness in libraries is invisible to clients, as FCSL
proofs are compositional. Indeed, because specs are encoded as Coq
types~\cite{Sergey-al:PLDI15}, the substitution principle
automatically applies to programs \emph{and proofs}.
%
%Thus, the trickiness of library proofs is not visible to the clients.
%
At the moment, our goal was not to optimize the proof sizes, but to
demonstrate that FCSL as a tool is suitable \emph{off-the-shelf} for
machine-checked verification of properties in the spirit of novel
correctness
conditions~\cite{Hemed-al:DISC15,Aspnes-al:JACM94,Jagadeesan-Riely:ICALP14}.
Therefore, we didn't invest into building advanced
tactics~\cite{McCreight:TPHOL09} for specific classes of
programs~\cite{Zee-al:PLDI08} or
properties~\cite{Dragoi-al:CAV13,Vafeiadis:CAV10,Bouajjani-al:POPL15}.
%
% and we leave developing such automation for future work.

% Our verification is compositional, because the proofs of the clients'
% specs are derived only from the specifications of concurrent objects,
% without relying on their implementation.


\section{Related work}
\label{sc:related}

% Early times

The proof method for establishing linearizability of concurrent
objects based on the notion of \emph{linearization points} has been
presented in the original paper by Herlihy and
Wing~\cite{Herlihy-Wing:TOPLAS90}. The first Hoare-style logic,
employing this method for compositional proofs of linearizability was
introduced in Vafeiadis' PhD thesis~\cite{Vafeiadis:PhD}. However,
that logic, while being inspired by the
combination~\cite{Vafeiadis-Parkinson:CONCUR07} of Rely-Guarantee
reasoning~\cite{Jones:TOPLAS83} and Concurrent Separation
logic~\cite{OHearn:TCS07} with syntactic treatment of linearization
points~\cite{Vafeiadis-al:PPoPP06}, did not have a soundness proof
with respect to any program semantics. Furthermore, the
work~\cite{Vafeiadis:PhD} did not connect reasoning about
linearizability to the verification of client programs, which make use
of linearizable objects in a concurrent environment (\cf
Section~\ref{sec:clients}).

% Modern logics for linearizability

Both these shortcomings were addressed in more recent works on program
logics for establishing linearizability~\cite{Liang-Feng:PLDI13}, or,
equivalently~\cite{Filipovic-al:TCS10}, \emph{observational
  refinement}~\cite{Turon-al:ICFP13}, which provided semantically
sound methodologies for (a) verifying linearizability/refinement of
concurrent objects \emph{as well as} for (b) giving the objects
Hoare-style specifications, useful for the clients.
%
However, in the both
approaches~\cite{Liang-Feng:PLDI13,Turon-al:ICFP13} establishing (a)
and (b) essentially requires one to prove \emph{two different} facts
about a program, and, if one is interested only in the Hoare-style
reasoning by means of composing program specifications, verifying
linearizability (a) is a detour, which might be avoided.

% No-linearizability

This observation has been recognized in a series of more recent works
on program logics for concurrency that all focused on establishing
Hoare-style specifications for concurrent objects (b) without
resorting to
linearizability~\cite{Sergey-al:ESOP15,Svendsen-Birkedal:ESOP14,ArrozPincho-al:ECOOP14,Jung-al:POPL15}.
%
In this paper, we are following the same way of thinking, building on
the ideas from the prior work~\cite{Sergey-al:ESOP15}, which explored
some patterns of assigning \emph{subjective} Hoare-style concurrent
specifications with auxiliary histories to concurrent objects
(including \emph{higher-order} ones, such as {flat
  combiner}~\cite{Hendler-al:SPAA10}) in
FCSL~\cite{Nanevski-al:ESOP14}. The work~\cite{Sergey-al:ESOP15} has
generalized earlier results on history-based Hoare-style
logics~\cite{Fu-al:CONCUR10, Gotsman-al:ESOP13,Bell-al:SAS10}, yet it
has not provided a way to reason about concurrent objects, featuring
future-dependent linearization points.
%

The key novelty of this work with respect to previous results
involving Hoare-style reasoning about histories~\cite{Fu-al:CONCUR10,
  Gotsman-al:ESOP13,Bell-al:SAS10,Sergey-al:ESOP15,Hemed-al:DISC15} is
the idea of dynamically \emph{re-linking} the auxiliary histories,
enabling efficient constructive reasoning about non-local and
future-dependent linearization points.
%
Since re-linking as we presented it is just manipulation with
otherwise standard auxiliary state, we did not have to extend the
metatheory of FCSL, and were able to use it
\emph{off-the-shelf}. Furthermore, relying on the auxiliary state
makes it possible to extend our verification method for reasoning
about higher-order (\ie, parametrized by another data structure)
snapshot-based concurrent constructions~\cite{Petrank-Timnat:DISC13},
which is our immediate future work.
%
In contrast, alternative modern programming
logics~\cite{ArrozPincho-al:ECOOP14,Jung-al:POPL15,Svendsen-Birkedal:ESOP14}
would require introduction of prophecy variables in order to verify
Jayanti's snapshot construction, and, to the best of our knowledge,
none of these extensions has been implemented yet.

Related to our result, O'Hearn \etal have demonstrated how to employ
history-based reasoning and Hoare-style logic for proving
\emph{non-constructively} existence of linearization points for
concurrent objects out of the data structure
invariants~\cite{OHearn-al:PODC10}---the result is known as \emph{the
  Hindsight Lemma}. The reasoning principle presented in this paper
generalizes that idea, since the Hindsight Lemma is only applicable to
``pure'' concurrent methods (\eg, concurrent set's
\texttt{contains}~\cite{Heller-al:OPODIS05}), which do not determine
position of other threads' linearization points. In contrast, our idea
of re-linking histories also handles the structures, where a
linearization point of a method call (e.g., \texttt{write}) might
depend on the (future) outcome of another operation (e.g.,
\texttt{scan}), as was showcased by Jayanti's construction.



% Most recent related work relies on parametrization to avoid reasoning
% about linearizability. But, that has its drawbacks. In particular,
% while it can handle situations in which linearization points are
% placed in different places, depending on the run-time infomration
% (speculiation), it is not currently strong enough to formalize
% examples where linearizatin points appear in different
% proceedures.\an{Hmm, are we super sure of this?} Thus, we don't
% believe they can handle Jayanti's algorithm.  \is{How about this time
%   we just mention these people briefly and instead do a comparison to
%   the PODC crowd and their reasoning methods, which are all about
%   harnessing the vanialla definition of liearizability. This way, the
%   whole discussion will be more relevant to the audience, as nobody
%   knows the concurrency logics anyway, so we'll just waste valuable
%   space, talking about them in detail.}  \an{I agree that we should
%   just mention them briefly.}

% Independently of us, Kyzha et al. have developed an a method whereby
% linearizability is proved by reordering time-stamped histories,
% similar to the basis of our approach. However, there are many
% differences. 

% \begin{enumerate}
% \item While linearizability does not say anything about clioent side
%   proofs, beyond the ability to replace the two programs in it, our
%   method also gives a way to reason about clients, as we illustrated
%   in Section 4.

% \item While they present a new logic, for us, it is all a mode of use
%   of auxiliary state.

% \item Our PCM of histories let us reuse separation logic in
%   infrastructure (e.g., frame rule) to reason about histories locally,
%   whereas Kyzha et al. use global histories only. Our setting also
%   immediatley lends itself to higher-order programming. This is
%   particularly important for snaphsots, as one of their major
%   application is in iterators -- a prototypical higher-order
%   program~\cite{PetrankT+disc13}. However, we don't explore iterators
%   in this particular paper.

% \item Kyzha et al. track the ordering of timestamps quite differnetly
%   from us. Where we keep an ``existential'' witness for the total
%   ordering of timestamps, at all stages of evaluation, they do so
%   ``universially''. Thus, they require proving that all possible
%   completions of a partial order into a total order are valid for
%   establishing the relation with the linearization program. We believe
%   this leads to larger proofs and more complicated proofs than
%   necessary.
% \end{enumerate}

% Liang et al~\cite{LiangF+pldi13} present a dedicated meta-theory to
% reason with future-dependent linearization points based on
% speculations. \gad{Are we sure they can't do Jayanti here? Need to
%   think what to say about their works as they do have a program logic
%   as we do}
  

%\vspace{-4pt}

\section{Conclusion and Future Work}
\label{sec:conclusion}

%\vspace{-2pt}

We have presented a number of formalization techniques, enabling
specification and verification of highly scalable non-linearizable
concurrent objects and their clients in Hoare-style program logics.
%
In particular, we have explored several reasoning patterns, all
involving the idea of formulating execution histories as auxiliary
state, capturing the expected concurrent object behavior.
%
We have discovered that quantitative logic-based reasoning about
concurrent behaviors can be done by storing relevant information about
interference directly into the entries of a logical history.

We believe that our results help to bring the Hoare-style reasoning
into the area of non-linearizable concurrent objects and open a number
of exciting opportunities for the field of mechanized logic-based
concurrency verification.

For instance, in this paper we have deliberately chosen to focus on
simple client programs to showcase the specs we gave to concurrent
libraries. However, any larger program incorporating these examples
can be verified compositionally in FCSL, out of \emph{these clients'
  specs}, via the substitution principles of
FCSL~\cite{Nanevski-al:ESOP14,Sergey-al:PLDI15}, without the need to
deal with concepts such as histories and tokens that are specific to
particular libraries. Given the bounds, which we formally proved in
Section~\ref{sec:qclients}, we believe that the reasoning patterns we
have described will be useful for mechanical verification of larger
weakly-synchronized approximate parallel
computations~\cite{Rinard:RACES}, exploiting the QC and QQC-like
behavior.

Furthermore, by ascribing interference-sensitive quantitative specs in
the spirit of~\eqref{eq:qc-spec} to relaxed concurrent
libraries~\cite{Henzinger-al:POPL13}, one can assess the applicability
of a library implementation for its clients: the clients should
tolerate the anomalies caused by interference, as long as they can
logically infer the desired safety assertions from a library spec,
which is fine-tuned for particular usage scenarios.


% Since logical approaches enable reasoning about higher-order
% concurrent data
% structures~\cite{Svendsen-al:ESOP13,Turon-al:ICFP13,Sergey-al:ESOP15},
% we envision the possibility of giving parametric logical specs to such
% generic relaxed constructions as diffracting/elimination
% trees~\cite{Shavit-Touitou:TCS97,Shavit:CACM11} that, once
% instantiated with suitably specified stacks or pools on the leaves,
% would yield a provably correct, highly scalable concurrent container
% implementation.


% Acknowledgements:

% Michael Emmi
% Pierre Ganty
% Andrea Cerone
% Anton Podkopaev

% \todo{Generalizing the construction of the counting network to
%   arbitrary diffracting trees}

% \todo{Elimination and diffracting trees~\cite{Shavit-Touitou:TCS97}.}
\paragraph{Acknowledgements}
We thank the anonymous reviewers from OOPSLA'16 PC and AEC for their feedback. We are also grateful to Sophia Drossopoulou for shepherding our paper. This research was partially supported by \ab{Please fill in} and the US National Science Foundation (NSF). Any opinion, findings, and conclusions or recommendations expressed in the material are those of the authors and do not necessarily reflect the views of NSF.

\appendix
%\newpage

\section{Exchanger Invariants and Proof Outline}
\label{app:exch}

% In this section, we formally define the exchanger's state invariants,
% and present the proof outline for its spec~\eqref{tag:exchangespec}.

% \vspace{3pt}
% \noindent
% \emph{\textbf{Exchanger invariants}}~~~
%

\paragraph{Additional exchanger invariants}

The states in the exchanger state-space must satisfy other invariants
in addition to (\ref{tag:exchanging}). These properties arise from our
description of how the exchanger behaves on decorated state. We
abbreviate with $p \mapsto (x; y)$ the heap
$p \mapsto x \hunion p\!+\!1 \mapsto y$.

\begin{enumerate}[label=(\roman*)]
\item\label{exP} $\heapj$ contains a pointer $g$ and a number of
  offers $p \mapsto (v; x)$, and $g$ points to either $\mathsf{null}$
  or to some offer in $\heapj$.

\item $\hists$, $\histo$ and $\mygather{\pending}$ contain only
  disjoint time-stamps. Similarly, $\perms$ is disjoint from $\permo$.

\item\label{matched} All offers in $\pending$ are matched and owned
  by some thread:
%
  {\small$\exists t\ldot p \mapsto (t, v, w) \subseteq \pending
    \Leftrightarrow p \in \perms \hunion \permo, p \mapsto (v;
    \Matched w) \subseteq \heapj $}.

\item There is at most one unmatched offer; it is the one linked
  from~$g$. It is owned by someone:
%
  {\small{
      $p \mapsto (v; \Unmatched) \subseteq \heapj \Longrightarrow p
      \in \perms \hunion \permo, g \mapsto p \subseteq \heapj.  $ }}.

\item Retired offers aren't owned:
  {\small{$p \mapsto (v; \Retired)\!\subseteq\!\heapj\!\Rightarrow
    p\!\notin\!\perms\!\hunion\!\permo$}}.

\item The outstanding offers are included in the joint heap, \ie, if
  $p \in \perms \hunion \permo$ then $p \in \mathsf{dom}\ \heapj$.

\item\label{ex:gapless} The combined history
  $\hists \hunion \histo \hunion \mygather{\pending}$ is gapless: if
  it contains a time-stamp $t$, it also contains all the smaller
  time-stamps (sans 0).

\end{enumerate}

% \vspace{3pt}
% \noindent
% \textbf{\emph{Explaining the proof outline}}~~~
%

\paragraph{Explaining the proof outline}

Figure~\ref{fig:exchanger_proof} presents the proof outline for the
spec~\eqref{tag:exchangespec}.
%
We start with the precondition, and after allocation in line 2,
$\heaps$ stores the offer $p$ in line 3.

If \code{CAS} at line 4 succeeds, the program ``installs'' the offer;
that is, the state (real and auxiliary) is changed simultaneously to
the modification of $g$. In particular, $p$ is added to $\perms$, and
the offer $p$ changes ownership, to move from $\heaps$ to $\heapj$.
Since $b$ will be bound to $\mathsf{null}$, this leads us to the
assertion in line 7. We explain in Section~\ref{sec:background} how
these kinds of changes to the auxiliary state, which are supposed to
occur simultaneously with some atomic operation (in this case,
\code{CAS}), are specified and verified in FCSL. The assertion in line
7 further states $\mathsf{bounded}\ p\ v\ \gist$. We do not formally
define $\mathsf{bounded}$ here (it is in the proof scripts,
accompanying the paper), but it says that $p$ has been moved to
$\heapj$, \ie, $p \mapsto (v; -) \subseteq \heapj$, and that any
time-stamp $t$ at which another thread may match $p$, and thus place
the entry $p \mapsto (t, v,-)$ into $\pending$, must satisfy
$\mathsf{last}(\gist) < t, \twin{t}$. Intuitively, this property is
valid, and stable under interference, because entries in $\pending$
can be added only by generating fresh time-stamps wrt.~the collective
history $\histo \hunion \mygather{\pending}$, and $\gist$ is a subset
of it.
%
If \code{CAS} in line 4 fails, then nothing changes, so we move to the
spec in line~15.

At line 8, \code{CAS} succeeds if $x\,{=}\,\Unmatched$, and fails if
$x\,{=}\,\Matched w$. Notice that $x$ cannot be $\Retired$; since we
own $p \in \perms$, no other thread could retire $p$.
%
If \code{CAS} fails, then the offer has been matched with~$w$. \code{CAS}
simultaneously ``collects'' the offer as follows. By invariant (iii),
and $\mathsf{bounded}\ p\ v\ \gist$, the auxiliary map $\pending$
contains an entry $p \mapsto (t, v, w)$, where $\mathsf{last}(\gist) <
t, \twin{t}$. The auxiliary state is changed to remove $p$ from
$\pending$, and simultaneously place $t \mapsto (v, w)$ into $\hists$.
%
If \code{CAS} succeeds, the offer was unmatched, and is ``retired'' by
removing $p$ from $\perms$.
%
Lines 12-13 branch on $x$, selecting either the assertion 10 or 11, so
the postcondition follows.

%Deallocation in line 16 removes the offer $p$ from the heap $\heaps$.

After reading $cur$ in line 18, by invariant (i), we know that $cur$
either points to $\mathsf{null}$, or to some offer $p \mapsto (w; -)
\subseteq \heapj$.

%The null-check in line 21, together with line 19, directly establishes
%the postcondition.

{
%\setlength{\belowcaptionskip}{-15pt} 
\begin{figure}
\[
{\footnotesize{
\begin{array}{rl}
 \Num{1} & \specK{\{\heaps = \emptyset, \perms = \emptyset, \hists = \emptyset, \gist \subseteq \histo \hunion \mygather{\pending} \}}
\\ 
 \Num{2} & ~~~~ p \Asgn \esc{alloc}~(v, \Unmatched);\\
 \Num{3} & \specK{\{\heaps = p \mapsto (v; \Unmatched), \perms = \emptyset, \hists = \emptyset, \gist \subseteq \histo \hunion \mygather{\pending} \}}\\
 \Num{4} & ~~~~ b \Asgn \esc{CAS}~(g, \esc{null}, p);\\
 \Num{5} & ~~~~ \kw{if}~~b~\esc{==}~\esc{null}~~\kw{then}\\
 \Num{6} & ~~~~ ~~~~ \esc{sleep}~(50);\\
 \Num{7} & \specK{\{\heaps = \emptyset, \perms = \{p\}, \hists = \emptyset, \gist \subseteq \histo \hunion \mygather{\pending}, \esc{bounded}\ p\ v\ \gist \}}\\
 \Num{8} & ~~~~ ~~~~ x \Asgn \esc{CAS}~(p\esc{+}1, \Unmatched, \Retired);\\
 \Num{9} & \specK{\{\heaps = \emptyset, \perms = \emptyset, \gist \subseteq \histo \hunion \mygather{\pending},\hbox{}} \\
\Num{10} & \specK{\hphantom{\}}x = \Matched w \implies \exists t\ldot \hists = t \mapsto (v, w), \mathsf{last}(\gist) < t, \twin t,}\\
\Num{11} & \specK{\hphantom{\}}x = \Unmatched \implies \hists = \emptyset \}}\\
\Num{12} & ~~~~ ~~~~ \kw{if}~~x~~\kw{is}~~\Matched w~~\kw{then}~~\kw{return}~~(\esc{Some}~w)\\
\Num{13} & ~~~~ ~~~~ \kw{else}~~\kw{return}~~\esc{None}\\
\Num{14} & ~~~~ \kw{else}\\
\Num{15} & \specK{\{\heaps = p \mapsto (v; \Unmatched), \perms = \emptyset, \hists = \emptyset, \gist \subseteq \histo \hunion \mygather{\pending}\}}\\
\Num{16} & ~~~~ ~~~~ \esc{dealloc}~p;\\
\Num{17} & \specK{\{\heaps = \emptyset, \perms = \emptyset, \hists = \emptyset, \gist \subseteq \histo \hunion \mygather{\pending}\}}\\
\Num{18} & ~~~~ ~~~~ cur \Asgn \esc{read}~g;\\
\Num{19} & \specK{\{\heaps = \emptyset, \perms = \emptyset, \hists = \emptyset, \gist \subseteq \histo \hunion \mygather{\pending},}\\
\Num{20} & \specK{\hphantom{\}}cur = \esc{null} \vee cur \mapsto (w; -) \subseteq \heapj\}}\\
\Num{21} & ~~~~ ~~~~ \kw{if}~~cur~\esc{==}~\esc{null}~~\kw{then}~~\kw{return}~{\esc{None}}\\
%& \specK{\{ \mbox{obvious} \}}\\
\Num{22} & ~~~~ ~~~~ \kw{else}\\
\Num{23} & \specK{\{\heaps = \emptyset, \perms = \emptyset, \hists = \emptyset, \gist \subseteq \histo \hunion \mygather{\pending}, cur \mapsto (w; -) \subseteq \heapj\}}\\
\Num{24} & ~~~~ ~~~~ ~~~~ x \Asgn \esc{CAS}(cur\esc{+}1, \Unmatched, \Matched v);\\
\Num{25} & \specK{\{\heaps = \emptyset, \perms = \emptyset, \gist \subseteq \histo \hunion \mygather{\pending}, cur \mapsto (w; y) \subseteq \heapj,} \\
\Num{26} & \specK{\hphantom{\}}x = \Unmatched \implies y = \Matched v, \exists t\ldot \hists = t \mapsto (v, w), \mathsf{last}(\gist) < t, \twin t},\\
\Num{27} & \specK{\hphantom{\}}x \neq \Unmatched \implies \hists = \emptyset, y \neq \Unmatched \}}\\
\Num{28} & ~~~~ ~~~~ ~~~~ \esc{CAS}~(g, cur, \esc{null});\\
\Num{29} & \specK{\{\mbox{same as above; the state satisfies (iv) because $y \neq \Unmatched$}\}}\\
%& \specK{\{\heaps = \emptyset, \perms = \emptyset, \gist \subseteq \histo \hunion \mygather{\pending}, 
%    \exists w hw. cur \mapsto (w, hw) \in \heapj,} \\
%& \specK{x = \Unmatched, hw = \Matched v, \exists t. \hists = t \mapsto (w, v), last \gist < t, \twin t}\\
%& \specK{x \neq \Unmatched, \hists = \emptyset, hw \neq \Unmatched \}}\\
\Num{30} & ~~~~ ~~~~ ~~~~ \kw{if}~~x~\esc{==}~\Unmatched~~\kw{then}~~w\Asgn \esc{read}~cur;\kw{return}~(\esc{Some}\ w)\\
\Num{31} & \specK{\{ \heaps = \emptyset, \perms = \emptyset, \gist \subseteq \histo \hunion \mygather{\pending}, \res=\esc{Some}\ w}, \\
\Num{32} & \specK{\hphantom{\}}\exists t. \hists = t \mapsto (w, v), \mathsf{last}(\gist) < t, \twin t\}}\\
\Num{33} & ~~~~ ~~~~ ~~~~ \kw{else}~~\kw{return}~\esc{None}\}\\
\Num{34} & \specK{\{\heaps = \emptyset, \perms = \emptyset, \hists = \emptyset, \gist \subseteq \histo \hunion \mygather{\pending}, \res=\esc{None}\}} 
\end{array}
}}
\]
%\vspace{-15pt}  
\caption{Proof outline for the exchanger.}
\label{fig:exchanger_proof}
\end{figure} 
}


At line 24, the \code{CAS} succeeds if $x = \Unmatched$ and fails
otherwise. If \code{CAS} succeeded, then it ``matches'' the offer in $cur$;
that is, it writes $\Matched w$ into the hole of $cur$, and changes the
auxiliary state as follows. It takes $t$ to be the smallest unused
time-stamp in the history $\hist = \hists \hunion \histo \hunion
\mygather{\pending}$. Thus $\mathsf{last}(\hist) < t$, and because
$\hist$ has even size by invariant~(\ref{tag:exchanging}), $t$ must be
odd, and hence $t < \twin t = t+1$. The $t \mapsto (v, w)$ is placed
into $\hists$, giving us assertion 26.  To preserve the invariant
(iii), \code{CAS} simultaneously puts the entry $p \mapsto (t, w, v)$ into
$\pending$, for future collection by the thread that introduced offer
$cur$. But, we do not need to reflect this in line 26.
%
If the \code{CAS} fails, the history $\hists$ remains empty, as no matching is
done. However, the hole $y$ associated with $cur$ cannot be
$\Unmatched$, as then \code{CAS} would have succeded.
%
Therefore, it is sound in line 28 to ``unlink'' $cur$ from $g$, as the
unlinking will not violate the invariant (iv), which says that an
unmatched offer must be pointed to by $g$.
%
Finally, lines 30 and 33 select the assertion 26 or 27, and either
way, directly imply the postcondition.


% \acks
% \todo{Acknowledgments, if needed.}

%\newpage

\setlength{\bibsep}{2.1pt} 
\bibliographystyle{abbrv}
\softraggedright 
\bibliography{bibmacros,references,proceedings}

\end{document}


