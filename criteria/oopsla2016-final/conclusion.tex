%\vspace{-4pt}

\section{Conclusion and Future Work}
\label{sec:conclusion}

%\vspace{-2pt}

We have presented a number of formalization techniques, enabling
specification and verification of non-linearizable concurrent objects
and their clients in Hoare-style program logics.
%
All the explored reasoning patterns involve the idea of formulating
execution histories as auxiliary state, capturing the expected
concurrent behavior.
%
We have discovered that quantitative logic-based reasoning about
concurrent behaviors can be done by storing relevant information about
interference directly into the entries of a logical history.

We believe that our results help to bring the Hoare-style reasoning
into the area of non-linearizable concurrent objects and open a number
of exciting opportunities for the field of mechanized logic-based
concurrency verification.

For instance, in this paper we have deliberately chosen to focus on
simple client programs to showcase the specs we gave to concurrent
libraries. However, any larger program incorporating these examples
can be verified compositionally in FCSL, out of \emph{these clients'
  specs}, via the substitution principles of
FCSL~\cite{Nanevski-al:ESOP14,Sergey-al:PLDI15}, without the need to
deal with concepts such as histories and tokens that are specific to
particular libraries. We believe that the reasoning patterns we have
described will be useful for mechanical verification of larger
weakly-synchronized approximate parallel
computations~\cite{Rinard:RACES}, exploiting the QC and QQC-like
behavior.

Furthermore, by ascribing interference-sensitive quantitative specs in
the spirit of~\eqref{eq:qc-spec} to relaxed concurrent
libraries~\cite{Henzinger-al:POPL13}, one can assess the applicability
of a library implementation for its clients: the clients should
tolerate the anomalies caused by interference, as long as they can
logically infer the desired safety assertions from a library spec,
which is fine-tuned for particular usage scenarios.


% Since logical approaches enable reasoning about higher-order
% concurrent data
% structures~\cite{Svendsen-al:ESOP13,Turon-al:ICFP13,Sergey-al:ESOP15},
% we envision the possibility of giving parametric logical specs to such
% generic relaxed constructions as diffracting/elimination
% trees~\cite{Shavit-Touitou:TCS97,Shavit:CACM11} that, once
% instantiated with suitably specified stacks or pools on the leaves,
% would yield a provably correct, highly scalable concurrent container
% implementation.

%\vspace{5pt}
\paragraph{Acknowledgements}

%\acks

We thank the anonymous reviewers from OOPSLA'16 PC and AEC for their
feedback. 
%
We are grateful to Yannis Smaragdakis for his efforts as OOPSLA PC
chair and to Sophia Drossopoulou for her dedication to bring out the
best of the paper.
%
This research was partially supported by the Spanish MINECO project
RISCO (TIN2015-71819-P) and the US National Science Foundation
(NSF). Any opinion, findings, and conclusions or recommendations
expressed in the material are those of the authors and do not
necessarily reflect the views of NSF.



