\section{Mechanization and Evaluation}
\label{sec:evaluation}

% \ab{This is an important section. It needs to convey what was
%   difficult, what was easy in the implementation. How were the
%   difficulties mitigated via the techniques developed in the previous
%   sections? What interesting proof engineering that needed to be done?
%   Ideally, these discussions should provide context/understanding of
%   the numbers in the Table. What are the current limitations of the
%   implementation? Make a separate paragraph for limitations perhaps
%   starting ``At the moment'' below.}
%
% \is{Mostly addressed in revised writing.}

We have mechanized the specs and the proofs of all the examples from
this paper~\cite{fcsl-site}, taking advantage of the fact that FCSL
has been recently implemented as a tool for concurrency
verification~\cite{Sergey-al:PLDI15} on top of the Coq proof
assistant~\cite{Coq-manual}.

Table~\ref{tab:locs} summarizes the statistics with respect to our
mechanization in terms of lines of code and compilation times. 
%
The examples were proof-checked on a 3.1~GHz Intel Core~i7 OS~X
machine with 16 Gb RAM, using Coq~8.5pl2 and Ssreflect
1.6~\cite{Gonthier-al:TR}.
%
As the table indicates, a large fraction of the implementation is
dedicated to proofs of preservation of resource
invariants~(\textsf{Inv}), \ie, checking that the actual
implementations do not ``go wrong''.
%
In our experience, these parts of the development are the most tricky,
as they require library-specific insights to define and reason about
auxiliary histories.
%
Since FCSL is a general-purpose verification framework, which does not
target any specific class of programs or properties, we had to prove
problem-specific facts, \eg, lemmas about histories of a particular
kind (\textsf{Facts}), and to establish the specs of interest stable
(\textsf{Stab}). Once this infrastructure has been developed, the
proofs of main procedures turned out to be relatively small
(\textsf{Main}).
%
% \gad{I don't see the point of having the {\bf Build} column in
%   Table~\ref{tab:locs}: First of all, I don't think that time to
%   compile a proof in Coq is relevant at all. From an engineering
%   perspective, it might perhaps make more sense to estimate the amount
%   of man-hours that took to discharge them. Second, because we then
%   would have to explain why the exchanger takes half the time than the
%   counter, given that they are similar in size. And I don't think we
%   want to get into describing the design, the differences in ``style''
%   between the two, and discussing why that might be affecting the
%   buildtime in detail here.}
%
% \is{Sorry, but since it's PLDI, we have to give build times, even
%   though we're out of space to explain some specific performance
%   phenomena}

{
%\setlength{\belowcaptionskip}{-1pt} 
\begin{table}
{%\footnotesize
\sffamily\small % tabular data either 10pt times, or 9pt helvetica
\centering
\begin{tabular}{|@{\ }l@{\ }||@{\ }c@{\ }|@{\ }c@{\ }|@{\ }c@{\ }|@{\ }c@{\ }|@{\ }c@{\ }||@{\ }r@{\ }|}
  \hline
  \textbf{Program} &  
                     {Facts} & {Inv} &
                                       {Stab} & {Main} & \textbf{Total}
  & \textbf{Build~~~}    
  \\ \hline \hline 
  Exchanger \hfill (\S \ref{sec:exchanger}) & 365 & 1085 & 446 & 162 & 2058 & 4m~~46s
  \\
  Exch. Client \hfill (\S \ref{sec:cal}) & 258 & -- &--& 182 & 440 & 57s
  \\
  Count. Netw. \hfill (\S \ref{sec:counting}) & 379 & 785 & 688 & 27 & 1879 & 12m~23s
  \\
  CN Client 1 \hfill (\S \ref{sec:qc-client}) & 141 &--&--& 180  & 321 & 3m~11s
  \\
  CN Client 2 \hfill (\S \ref{sec:qqc-client})& 115 &--&--& 259 & 374 & 3m~~~9s 
  \\[2pt] \hline
\end{tabular}
}
\caption{
  Mechanization of the examples: lines of code for program-specific facts \intab{Facts},
  resource invariants and transitions \intab{Inv}, 
  stability proofs for desired specs \intab{Stab}, spec and proof sizes for main
  functions \intab{Main}, total LOC count \intab{\textbf{Total}}, and build
  times \intab{\textbf{Build}}. The ``--'' entries indicate the
  components that were not needed for the example.
} 
\label{tab:locs}
%\vspace{-15pt}
\end{table}}

Fortunately, trickiness in libraries is invisible to clients, as FCSL
proofs are compositional. Indeed, because specs are encoded as Coq
types~\cite{Sergey-al:PLDI15}, the substitution principle
automatically applies to programs \emph{and proofs}.
%
%Thus, the trickiness of library proofs is not visible to the clients.
%
At the moment, our goal was not to optimize the proof sizes, but to
demonstrate that FCSL as a tool is suitable \emph{off-the-shelf} for
machine-checked verification of properties in the spirit of novel
correctness
conditions~\cite{Hemed-al:DISC15,Aspnes-al:JACM94,Jagadeesan-Riely:ICALP14}.
Therefore, we didn't invest into building advanced
tactics~\cite{McCreight:TPHOL09} for specific classes of
programs~\cite{Zee-al:PLDI08} or
properties~\cite{Dragoi-al:CAV13,Vafeiadis:CAV10,Bouajjani-al:POPL15}.
%
% and we leave developing such automation for future work.

% Our verification is compositional, because the proofs of the clients'
% specs are derived only from the specifications of concurrent objects,
% without relying on their implementation.

