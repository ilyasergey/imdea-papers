
\section{Verifying Counting Network's Clients}
\label{sec:qclients}

% We next demonstrate how the spec~\eqref{eq:qc-spec} can be used in
% clients, to establish properties \textbf{R2} and \textbf{R3}. These
% properties have been addressed in the previous work using dedicated
% consistency criteria of quiescent
% consistency~\cite{Aspnes-al:JACM94,Derrick-al:FM14} and quantitative
% quiescent consistency and
% quasi-linearizability~\cite{Afek-al:OPODIS10,Jagadeesan-Riely:ICALP14},
% but here we derive them compositionally, \ie, out of the spec of
% \code{getAndInc}. This will demonstrate the usefulness of Hoare logic
% in deriving properties related to the various flavors of quiescent
% consistency.

Following \textbf{\emph{Step 3}} of our verification method, we now
illustrate requirements \textbf{R2} and \textbf{R3} from the previous
section via two different clients which execute two sequential calls
to \code{getAndInc}. Both clients are higher-order, \ie, they are
parametrized by subprograms, which can be ``plugged in''.
%
The first client will exhibit a quiescence between the two calls, and
we will prove that the call results appear in order, as required by
\textbf{R2}. The second client will experience interference of a
program with a $N$ concurrent calls to \code{getAndInc}, and we will
derive a bound on the results in terms of $N$, as required by
\textbf{R3}.

Both our examples will rely on the general mechanism of hiding,
presented in Section~\ref{sec:background}, as a way to logically restrict the
interference on a concurrent object, in this case, a counting network,
in a lexically-scoped way.
%
To ``initialize'' the counting network data structure, we provide the
starting values for the shared heap ($h_0$) and for the history
($\gist_0$), assuming that the initial set of tokens is empty:
%
% Specifically, we will use the following derived rule:
% %
% {\small{
% \[
% \begin{array}{c}
% \spec{P}~e~\spec{Q} @ \ccon\\[2pt]
% \hline\\[-7pt]
% \!\!\!
% \specK{\{\heaps = \Phi_1(\heapj), \Phi_1(P)\}} \hide_{\Phi_1}~e \specK{\{\exists \Phi_2\ldot \heaps = \Phi_2(\heapj), \Phi_2(Q)\}} @ \cal P
% \end{array}
% \]
% }}
%
\[
\tag{\normalsize \arabic{tags}}\refstepcounter{tags}\label{eq:hide2}
{\small{
\begin{array}{r@{\ }c@{\ }l}
% \text{\normalsize{where}} &
% \Phi_1 & \eqdef & [\emptyset/\tkns, \gist_0/\hists, \heap_0/\heapj,
% \emptyset/\tkno, \histo/\hists] 
% \\[2pt]
\heap_0 & \eqdef & \bal \hpts 0 \hunion c_0 \hpts 0 \hunion c_1 \hpts 1     
\\[2pt]
\gist_0 & \eqdef & \set{0 \hpts (\set{0}, 0), 1 \hpts (\set{1}, 1)}
\end{array}
}}
\]
%
That is, $\gist_0$ provides the ``default'' history for the initial
values 0 and 1 of $c_0$ and $c_1$, with the corresponding tokens
represented by numbers 0 and 1.  As always with hiding, the
postcondition of the hidden program will imply that $\tkno$ and
$\histo$ are both empty, as there is no interference at the end.

% \gad{I think that here, again, the quotation marks around ``plugged in'',
%   ``initialize'', and ``default'' are unnecessary.}

\subsection{Exercising Quiescent Consistency}
\label{sec:qc-client}

\begin{figure}
\centering
\[
{\small{
\!\!\!\!\!\!\!\!
\begin{array}{c}
  \spec{\!\!
  \begin{array}{c}
    \tkns = \emptyset,
    \hists = \gists,
    \gisto \subseteq \histo, \Ic{\gisto}{\ikno}, \\[2pt]
    \ikno \subseteq \tkno \hunion (\Tomb~\histo \setminus \Tomb~\gisto)
  \end{array}
  \!\!}
\\\\[-5pt]
  \begin{tabular}{c || c}
   $\esc{getAndInc()}$ & ${\small{e_i}}$ 
\end{tabular}
\\\\[-5pt]
~~~~\spec{\!\!
\begin{array}{c}
  \exists \iknh~\gist_i \ldot  
  \tkns = \emptyset\aand \hists = \gbm{\gists \hunion \gist_i \hunion (\res.1 + 2) \hpts (\iknh, -)},\\[1pt]
  \gisto \subseteq \histo\aand \ikno \subseteq \tkno \hunion
  (\Tomb~\histo \setminus \Tomb~\gisto), \Ic{\gisto}{\ikno},\\[1pt]
  \last~(\gists \hunion \gisto)  < \gbm{\res.1} + 2 +
  2~|\iknh \cap \ikno|
%
\end{array}
\!\!} %@\ccon
%
\end{array}
}}  
\]
%
\caption{Parallel composition of \code{getAndInc} and~$e_i$ in~\eqref{eq:eqc}.}
  \label{fig:example1} 
\end{figure}
%



Our first client is the following program~$\eqc$:
%
\[
\tag{\normalsize \arabic{tags}}\refstepcounter{tags}\label{eq:eqc}
{\small{
\begin{array}{ll} 
\Num{1} & (\res_1, -) \Asgn (\esc{getAndInc()} ~||~ e_1) \esc{;} \\[1pt]
\Num{2} & (\res_2, -) \Asgn (\esc{getAndInc()} ~||~ e_2) \esc{;} \\[1pt]
\Num{3} &  \kw{return}~(\res_1, \res_2) 
\end{array}
}}
\]
%
Each of the calls to \esc{getAndInc} interferes with either $e_1$ or
$e_2$, but in the absence of \emph{external} interference, the
quiescent state is reached between the lines 1 and 2. Hence, after
executing $\hide~\eqc$, it should be $\res_1 < \res_2$, following
\textbf{R2}.

The programs $e_1$ and $e_2$ can invoke \code{getAndInc} and modify
the counters concurrently with the two calls of $\eqc$, which we
capture by giving both the following generic spec:
%
\[
%
\tag{\normalsize \arabic{tags}}\refstepcounter{tags}\label{eq:eispec}
{\small
\!\!\!\!\!\!\!\! 
\begin{array}{c}
  \spec{~
  \hists = \emptyset\aand
  \tkns = \emptyset\aand
   \ikn \subseteq \tkno \hunion \Tomb~\histo
  ~}
  \\[1pt]
  e_i
  % 
  \\[1pt]
  \spec{\!\!\!
  \begin{array}{c}
    \exists \gist_i \ldot \hists = \gist_i\aand 
    \tkns = \emptyset\aand 
    \ikn \subseteq \tkno \hunion \Tomb~\histo
  \end{array} 
  \!\!\!} %@\ccon
%
\end{array}
}
\]
%
The postcondition allows for a number of increments via calls to
\esc{getAndInc}, which is reflected in the addition $\gist_i$ to
$\hists$. However, all such calls are required to be \emph{finished}
by the end of $e_i$ ($\tkns = \emptyset$). As customary by now, we use
the logical variable $\ikn$ to name the initial set of \emph{other}
tokens.

Figure~\ref{fig:example1} provides a spec for each of the parallel
compositions in the program~\eqref{eq:eqc}, proved via the
corresponding FCSL inference rule for parallel
composition~\eqref{eq:parrule}.
%
The spec is very similar to~\eqref{eq:qc-spec} with the differences
highlighted via gray boxes: (a) the self-history $\hists$ is increased
by $e_i$'s contribution $\gist_i$ in addition to the entry, introduced
by \code{getAndInc}, (b) the result of the parallel composition is a
pair, but we only constrain its first component $\res.1$, resulting
from the left subprogram. We also drop the last conjunct with
$\happrox$ from~\eqref{eq:qc-spec}, which we won't require for this
example.

%



Next, we use the spec from Figure~\ref{fig:example1} to specify and
verify the program $\eqc$, so far \emph{assuming} external
interference.
%
\[
\!\!\!
{\small{
\begin{array}{c}
\!\!\!\!\!
\spec{\!\!
  \begin{array}{c}
    % \tkns = \emptyset,
    % \gisto \subseteq \histo, \ikno \subseteq \tkno \hunion (\Tomb~\histo \setminus \Tomb~\gisto)\\[2pt]
    % \hists = \gist_0,
    \mbox{Fig.~\ref{fig:example1}'s precondition with $\gists := \gist_0$, $\gisto :=
      \histo$, and $\ikno := \tkno$}
  \end{array}
  \!\!}
  ~\comm{P}
  \\\\[-6pt]
  (\res_1, -) \Asgn (\esc{getAndInc()} ~||~ e_1) \esc{;}
  \\[3pt]
\!\!\!\!{{
\spec{\!\!\!\!
\begin{array}{c}
 \exists \gist_1\ldot \tkns = \emptyset\aand \hists = \gists',~\ldots
% \ikno \subseteq \tkno \hunion (\Tomb~\histo \setminus \Tomb~\gisto),
\\[2pt]
\mbox{where 
 $\gbm{\gists' = \gist_0 \hunion
\gist_1 \hunion (\res_1 + 2)\mapsto -}$, $\gisto := \histo$ and $\ikno :=
\tkno$} 
% \\[2pt]
% \mbox{hence $\Ic{\gisto}{\ikno}$}
  \end{array}
\!\!\!\!}
}}
%
\\\\[-5pt]
(\res_2, -) \Asgn (\esc{getAndInc()} ~||~ e_2) \esc{;}      
\\[3pt]
\spec{\!\!\!\!
\begin{array}{c}
\exists \gist_1~\gist_2~\iknh \ldot     
%
\tkns = \emptyset \aand 
%%\hists = \gists' \hunion \gist_2 \hunion (\res_2 + 2)\mapsto (\iknh, -)\aand\hbox{}\\[2pt] 
\gbm{\ikno \subseteq \tkno \hunion (\Tomb~\histo \setminus \Tomb~\gisto)},\\[1pt]
\gbm{\last~(\gists' \hunion \gisto) < \res_2 + 2 + 2~|\iknh \cap \ikno|},~\ldots 
\end{array}
\!\!\!\!\!}~\comm{Q}
\\\\[-7pt]
~~~~~~~~~~~\kw{return}~(\res_1, \res_2); ~\comm{=: \res} 
\\[2pt]
\spec{~Q(\res.1/\res_1, \res.2/\res_2)~} %@\ccon
\end{array}
}} 
\]
%
We start by instantiating the logical variables $\gists$, $\gisto$ and
$\ikno$ from Figure~\ref{fig:example1} with $\gist_0$, \emph{current}
$\histo$ and $\tkno$, respectively, naming the obtained precondition
$P$.
%
In the following assertion we focus on the clauses constraining
$\tkns$ and $\hists$. To verify the second call, we instantiate
$\gists$, $\gisto$ and $\ikno$ from Figure~\ref{fig:example1} with
$\gists' = \gist_0 \hunion \gist_1 \hunion (\res_1 + 2)\mapsto -$,
\emph{current} $\histo$ and $\tkno$, correspondingly, obtaining the
postcondition, which we name~$Q$.

The inequality in the postcondition $Q$ gives the boundary on the
out-of-order position of $\res_2$ with respect to the \emph{last}
value in the history captured in between the two parallel
compositions. The boundary is given via the size of intersection of
the two sets of tokens: snapshot ($\iknh$) and ``alive'' between the
calls ($\ikno$).
%
Now, to ensure the absence of external interference, we consider the
program $(\hide~\eqc)$.
%
By the general property of hiding (Section~\ref{sec:background}), we
know that at the final state there is no interference, hence $\tkno =
\emptyset$ and $\histo = \emptyset$ in $Q$.
%
Therefore, from the set inclusion on $\ikno$ in $Q$ (the grayed part),
we deduce that $\ikno = \emptyset$.
%
As a consequence, the intersection $\iknh \cap \ikno = \emptyset$, so
from the inequality we obtain
%
\[
%
\tag{\normalsize \arabic{tags}}\refstepcounter{tags}\label{eq:tada1}
%
%{\small{
\begin{array}{c}
 \last~(\gists' \hunion \gisto) < \res.2 + 2
\end{array}
\hfill
%}}
\]
%
%\noindent
%
But $\gists'$ is defined as $(\res.1+2)\mapsto -~\hunion\ldots$,
hence, $\res.1 + 2 \in \mathsf{dom}\ \gists'$, and thus $\res.1 + 2
\le \mathsf{last}\ \gists'$. Even more:
%
\[
%
\tag{\normalsize \arabic{tags}}\refstepcounter{tags}\label{eq:tada2}
%
%{\small{
\begin{array}{c}
\res.1 + 2 \le \last~(\gists' \hunion \gisto).
\end{array}
\hfill
%}}
\]
%
From~\eqref{eq:tada1} and~\eqref{eq:tada2} follows the result
\textbf{R2}: $\res.1 < \res.2$.

\subsection{Proving Quantitative Bounds}
\label{sec:qqc-client}

We next show how the spec~\eqref{eq:qc-spec} also obtains quantitative
bounds on the out-of-order anomalies in terms of a number of running
threads in the following program $\eqqc$:
%
\[
\tag{\normalsize \arabic{tags}}\refstepcounter{tags}\label{eq:eqqc}
{\small{
\begin{tabular}{l || l}
$
\begin{array}{ll} 
\Num{1} & \res_1 \Asgn \esc{getAndInc();} \\[1pt]
\Num{2} & \res_2 \Asgn \esc{getAndInc();}  \\[1pt]
\Num{3} & \kw{return}~(\res_1, \res_2)
\end{array}
$  
&
$~~~e$
\end{tabular} 
}}
\]
%
The $e$'s spec says that the \emph{number} of calls to \esc{getAndInc}
in~$e$ (\ie, the size of interference $e$ exhibits) is some fixed $N$:
%
\[
%
\tag{\normalsize \arabic{tags}}\refstepcounter{tags}\label{eq:espec}
{\small
\!\!\!\!\!\!\!\!\!  
\begin{array}{c}
  \spec{
  \tkns = \emptyset,
  \hists = \gists }
~  e
~  \spec{\!\!\!
  \begin{array}{c}
    \exists \gist \ldot 
    \tkns = \emptyset,
    \hists = \gists \hunion \gist,
    |\gist| = N
  \end{array} 
  \!\!\!} %@\ccon
%
\end{array}
}
\]
%
Our goal is to prove that in the absence of external interference for
$\eqqc$, $\res_1 < \res_2 + 2 \ N$ (requirement \textbf{R3}).

\begin{figure}
\centering
%    
\[
\!\!\!
{\small{
\begin{array}{c}
  \spec{~
    \mbox{{\normalsize{\eqref{eq:qc-spec}'}}s precondition with $\gists := \gist_0$, $\gisto :=
      \histo$, and $\ikno := \tkno$}~}
% \\\\[-6pt]
% \spec{\!\!
%   \begin{array}{c}
%     \tkns = \emptyset,
%     \hists = \gists,
%     \gisto \subseteq \histo,
%     \ikno \subseteq \tkno \hunion (\Tomb~\histo \setminus \Tomb~\gisto)\\[2pt]
%     \mbox{where $\gisto = \histo$ and $\ikno = \tkno$, hence $\Ic{\gisto}{\ikno}$}
% %    \histo \subseteq \histo,\\[2pt]
% %    \tkno \hunion \Tomb~\histo \subseteq \tkno \hunion \Tomb~\histo
%   \end{array}
%   \!\!}
\\\\[-6pt]
\res_1 \Asgn \esc{getAndInc();}
\\[3pt]
\spec{\!\!
\begin{array}{c}
   \exists \ikn \ldot 
   \tkns = \emptyset, 
   \hists = \gists',
   % \gisto \subseteq \histo, \ikno \subseteq \tkno \hunion
   % (\Tomb~\histo \setminus \Tomb~\gisto)
   \ldots
   \\[2pt]
   \mbox{where $\gbm{\gists' = \gist_0 \hunion (\res_1 + 2) \hpts
       (\ikn, -)}$} 
   % , $\gisto = \histo$ and $\ikno = \tkno$, \ldots}
% \\[2pt]
%    \mbox{hence $\Ic{\gisto}{\ikno}$}
  \end{array}
  \!\!}%
%\\\\[-6pt]
%\spec{\!\!\!
%  \begin{array}{c}
%  \tkns = \emptyset,  
%  \hists = \gists \hunion (\res_1 \!+\! 2) \!\hpts\! (\ikn^0, \ikn^1, -), 
%  \gists' := \hists,\\[2pt] 
%  \gisto := \histo, 
%  \ikno := \tkno,
%  \ikno \hunion \Tomb~\gisto \subseteq \tkno \hunion \Tomb~\histo
%  \end{array}
%  \!\!\!}%
\\\\[-5pt] 
\res_2 \Asgn \esc{getAndInc();}
\\[3pt]
\spec{\!\!\!
\begin{array}{c}
  \exists \iknh~z \ldot     
  % \tkns = \emptyset, \hists = \gists' \hunion (\res_2+2) \mapsto (\iknh, z),\\[2pt]
  \happrox (\gists' \hunion \gisto)~\res_2~\iknh~z, \ldots
\end{array}
\!\!\!}
\\\\[-5pt]
\spec{\!\!\!
\begin{array}{c}
  \exists \iknh~z \ldot     
  % \tkns = \emptyset, \hists = \gists' \hunion (\res_2+2) \mapsto (\iknh^0, \iknh^1, z),\\[2pt]
  \gbm{\iknh \subseteq \tkno \hunion (\Tomb~\histo) \hunion \set{z}},
  z \notin \ikn,\\[2pt] 
  \res_1 + 2 < \res_2 + 2 + 2~|\iknh \cap \ikn|
\end{array}
\!\!\!}
\\\\[-5pt]  
~~~~~~~~~~~~~~~
\kw{return}~(\res_1, \res_2) ~\comm{=: \res}
\\\\[-5pt]
\spec{\!\!\!
\begin{array}{c}
  \res.1 < \res.2 + 2 \ |\tkno \hunion \Tomb~\histo| 
  %
  %\aand\\[1pt] 
  % \tkns = \emptyset\aand
  % \hists = \gists \hunion (\res_1 + 2) \hpts -
  % \hunion (\res_2 + 2) \hpts -
  \end{array}
  \!\!\!} %@\ccon
\end{array}
}} 
\]
%
%
\caption{Proof outline of sequential composition in~\eqref{eq:eqqc}.}
\label{fig:proof2}
\end{figure}

We first verify the sequential composition of the two calls
in~\eqref{eq:eqqc}; the proof outline is in
Figure~\ref{fig:proof2}. 
%
% \ab{Can this proof outline be cut significantly/removed? Can we just
%   give the overall pre-post specs and go directly to the verification
%   of $\eqqc$?}
%
As previously, we start by instantiating the logical variables
$\gists$, $\gisto$ and $\ikno$ from spec~\eqref{eq:qc-spec} with
$\gists$, $\histo$ and $\tkno$, respectively. In the assertion,
resulting by of the first \esc{getAndInc}, we keep only the clauses
involving $\tkns$ and $\hists$, dropping the rest.
%
To verify the second \esc{getAndInc} call, we instantiate $\gists$,
$\gisto$ and $\ikno$ with $\gists' = \gists \hunion (\res_1+2) \mapsto
(\ikn, -)$, current $\histo$ and $\tkno$.

In the postcondition of the second call to \esc{getAndInc}, we focus
on the $\happrox~(\gists' \hunion \gisto)~\res_2~\iknh~z$ clause,
where $\iknh$ is the set of tokens snapshot when contributing
$\res_2+2$.
%
Unfolding the definition of $\happrox$ from~\eqref{eq:happrox}, we
obtain $\iknh \subseteq \tkno \hunion \Tomb~\histo
\hunion\{z\}$. Also, using $(\res_1 +2)\mapsto (\ikn, -)$ in the
implication that the unfolding obtains, we get $z \notin \ikn$ and
\[
\tag{\normalsize \arabic{tags}}\refstepcounter{tags}\label{eq:le0}
{\small{\res_1 + 2 < \res_2 + 2 + 2~|\iknh \cap \ikn|
}}
\]
%
Now we use the following trivial fact to simplify.
%
\vspace{8pt}
%
\begin{lemma}
\label{lm:intersect2}
If $z \in \iknh$ and $z \notin \ikn$, then $|\iknh \cap \ikn| \le
|\iknh| - 1$.
\end{lemma}
%
%
\vspace{8pt}
%
\noindent
Using the invariant~\ref{cn:ti1}, Lemma~\ref{lm:intersect2}
derives
$|\iknh \cap \ikn| \le |\iknh| - 1$
%
after which, the inclusion $\iknh \subseteq \tkno
\hunion \Tomb~\histo \hunion \set{z}$ leads to
%
\[
\tag{\normalsize \arabic{tags}}\refstepcounter{tags}\label{eq:le5}
{\small{
|\iknh \cap \ikn| \le |\tkno \hunion \Tomb~\histo|}}
\]
%
Combined with~\eqref{eq:le0}, this gives us $\res_1 < \res_2 + 2 \
|\tkno \hunion \Tomb~\histo|$, as shown in Figure~\ref{fig:proof2}'s
postcondition. In words, it asserts that the discrepancy between
$\res.1$ and $\res.2$ is bounded by the size of the tokens, which are
either held by the interfering threads at the end or are spent.
%
% \gad{low-priority right now, but still: the inequality above spreads
%   across a linebreak, and worse, in the middle of the arguments for an
%   operator.}
%
% \is{won't fix}

\begin{figure}[t]
  \centering
\[
{\small{
\!\!\!\!\!\!\!\!
\begin{array}{c}
  ~~~~~~~~\spec{~
  \tkns = \emptyset,
  \hists = \gist_0, \ldots
~} ~\comm{P}
\\[2pt]
  \begin{tabular}{c || c}
$
\spec{\!\!\!
    \begin{array}{c}
    \tkns = \emptyset,
    \hists = \gist_0
  \end{array}\!\!\!}
$
&
$
\spec{\!\!\!\begin{array}{c}
    \tkns = \emptyset,
    \hists = \emptyset
  \end{array}\!\!\!}
$
\\[3pt]
   $\begin{array}{l}
      \res_1 \Asgn \esc{getAndInc();}\\[1pt]
      \res_2 \Asgn \esc{getAndInc();}\\[1pt]
      \kw{return}~(\res_1, \res_2) ~\comm{ =: \res}\!\!\!
    \end{array}$
\!\!\!\!\!\!
& ${\small{e}}$ 
\\\\[-5pt] 
$
\spec{\!\!\!
{\small{
  \begin{array}{c}
    \res.1 < \res.2 + 2 \ |\tkno \hunion \Tomb~\histo|
  \end{array}
}}
  \!\!\!}\!\!$
\!\!
&
\!\!$\spec{\!\!\!
{{
  \begin{array}{c}
    \exists \gist \ldot 
    \hists = \gist, 
    |\gist| = N,
    \ldots
  \end{array}
}}
\!\!\!}$
\end{tabular}
\\\\[-5pt]
\comm{\res_1 := \res.1.1, \res_2 := \res.1.2}
\\\\[-6pt]
\spec{
\res_1 < \res_2 + 2 \ |\tkno \hunion \Tomb~(\histo \hunion \gist)|
}
\\[3pt]
~~~~~~~~\spec{
\res_1 < \res_2 + 2 \ |\tkno \hunion \Tomb~\histo| + 2 \ N
} ~\comm{Q}
%
\end{array}
}}  
\]
  \caption{Proof outline for the $\eqqc$ program.}
  \label{fig:eqqcproof}
\end{figure}

Figure~\ref{fig:eqqcproof} shows the proof outline for $\eqqc$ via the
spec from Figure~\ref{fig:proof2}.
%
By the parallel composition rule~\eqref{eq:parrule}, the precondition
splits into two subjective views, where we send the initial history
$\gist_0$ to the left thread, and the empty history to the right
thread. The proof from Figure~\ref{fig:proof2} then applies to the
left thread, and the spec~\eqref{eq:espec} applies to the right
one. Final $\histo$ of the left thread is the union of $\histo$ from
the joined thread with $\gist$, since the environment of the left
thread includes the right thread and of the join. Rewriting by this
property in the postcondition of the left thread gives us the post of
the joint thread: $\res_1 < \res_2 +2\ |\tkno \hunion \Tomb~(\histo
\hunion \gist)|$, which we can next simplify into
\[
\res_1 < \res_2 + 2\ |\tkno \hunion \Tomb~\histo| + 2\ N
\]
because $\Tomb$ distributes over $\hunion$, and $|\Tomb~\gist| =
|\gist| = N$. Finally, we restrict the external interference by
considering $(\hide~\eqqc)$. From the properties of hiding,
we deduce that $\tkno$ and $\histo$ in $Q$ are empty, hence we can
simplify into $\res_1 < \res_2 + 2 \ N$, which is the desired
result~\textbf{R3}.
%
%\gad{The same happens here, just before the highlighted math.}
