%\newpage

\section{Exchanger Invariants and Proof Outline}
\label{app:exch}

In this section, we formally define the exchanger's state invariants,
and present the proof outline for its spec~\eqref{tag:exchangespec}.

% \vspace{3pt}
% \noindent
% \emph{\textbf{Exchanger invariants}}~~~
%

\paragraph{Additional exchanger invariants}

The states in the exchanger state-space must satisfy other invariants
in addition to (\ref{tag:exchanging}). These properties arise from our
description of how the exchanger behaves on decorated state. We
abbreviate with $p \mapsto (x; y)$ the heap
$p \mapsto x \hunion p\!+\!1 \mapsto y$.

\begin{enumerate}[label=(\roman*)]
\item\label{exP} $\heapj$ contains a pointer $g$ and a number of
  offers $p \mapsto (v; x)$, and $g$ points to either $\mathsf{null}$
  or to some offer in $\heapj$.

\item $\hists$, $\histo$ and $\mygather{\pending}$ contain only
  disjoint time-stamps. Similarly, $\perms$ is disjoint from $\permo$.

\item\label{matched} All offers in $\pending$ are matched and owned
  by some thread:
%
  {\small$\exists t\ldot p \mapsto (t, v, w) \subseteq \pending
    \Leftrightarrow p \in \perms \hunion \permo, p \mapsto (v;
    \Matched w) \subseteq \heapj $}.

\item There is at most one unmatched offer; it is the one linked
  from~$g$. It is owned by someone:
%
  {\small{
      $p \mapsto (v; \Unmatched) \subseteq \heapj \Longrightarrow p
      \in \perms \hunion \permo, g \mapsto p \subseteq \heapj.  $ }}.

\item Retired offers aren't owned:
  {\small{$p \mapsto (v; \Retired)\!\subseteq\!\heapj\!\Rightarrow
    p\!\notin\!\perms\!\hunion\!\permo$}}.

\item The outstanding offers are included in the joint heap, \ie, if
  $p \in \perms \hunion \permo$ then $p \in \mathsf{dom}\ \heapj$.

\item\label{ex:gapless} The combined history
  $\hists \hunion \histo \hunion \mygather{\pending}$ is gapless: if
  it contains a time-stamp $t$, it also contains all the smaller
  time-stamps (sans 0).

\end{enumerate}

% \vspace{3pt}
% \noindent
% \textbf{\emph{Explaining the proof outline}}~~~
%

\paragraph{Explaining the proof outline}

Figure~\ref{fig:exchanger_proof} presents the proof outline for the
spec~\eqref{tag:exchangespec}.
%
We start with the precondition, and after allocation in line 2,
$\heaps$ stores the offer $p$ in line 3.

If \code{CAS} at line 4 succeeds, the program ``installs'' the offer;
that is, the state (real and auxiliary) is changed simultaneously to
the modification of $g$. In particular, $p$ is added to $\perms$, and
the offer $p$ changes ownership, to move from $\heaps$ to $\heapj$.
Since $b$ will be bound to $\mathsf{null}$, this leads us to the
assertion in line 7. We explain in Section~\ref{sec:background} how
these kinds of changes to the auxiliary state, which are supposed to
occur simultaneously with some atomic operation (in this case,
\code{CAS}), are specified and verified in FCSL. The assertion in line
7 further states $\mathsf{bounded}\ p\ v\ \gist$. We do not formally
define $\mathsf{bounded}$ here (it is in the proof scripts,
accompanying the paper), but it says that $p$ has been moved to
$\heapj$, \ie, $p \mapsto (v; -) \subseteq \heapj$, and that any
time-stamp $t$ at which another thread may match $p$, and thus place
the entry $p \mapsto (t, v,-)$ into $\pending$, must satisfy
$\mathsf{last}(\gist) < t, \twin{t}$. Intuitively, this property is
valid, and stable under interference, because entries in $\pending$
can be added only by generating fresh time-stamps wrt.~the collective
history $\histo \hunion \mygather{\pending}$, and $\gist$ is a subset
of it.
%
If \code{CAS} in line 4 fails, then nothing changes, so we move to the
spec in line~15.

At line 8, \code{CAS} succeeds if $x\,{=}\,\Unmatched$, and fails if
$x\,{=}\,\Matched w$. Notice that $x$ cannot be $\Retired$; since we
own $p \in \perms$, no other thread could retire $p$.
%
If \code{CAS} fails, then the offer has been matched with~$w$. \code{CAS}
simultaneously ``collects'' the offer as follows. By invariant (iii),
and $\mathsf{bounded}\ p\ v\ \gist$, the auxiliary map $\pending$
contains an entry $p \mapsto (t, v, w)$, where $\mathsf{last}(\gist) <
t, \twin{t}$. The auxiliary state is changed to remove $p$ from
$\pending$, and simultaneously place $t \mapsto (v, w)$ into $\hists$.
%
If \code{CAS} succeeds, the offer was unmatched, and is ``retired'' by
removing~$p$ from~$\perms$.
%
Lines 12-13 branch on $x$, selecting either the assertion 10 or 11, so
the postcondition follows.

%Deallocation in line 16 removes the offer $p$ from the heap $\heaps$.

After reading $cur$ in line 18, by invariant (i), we know that $cur$
either points to $\mathsf{null}$, or to some offer $p \mapsto (w; -)
\subseteq \heapj$.

%The null-check in line 21, together with line 19, directly establishes
%the postcondition.

{
%\setlength{\belowcaptionskip}{-15pt} 
\begin{figure}
\[
{\footnotesize{
\begin{array}{rl}
 \Num{1} & \specK{\{\heaps = \emptyset, \perms = \emptyset, \hists = \emptyset, \gist \subseteq \histo \hunion \mygather{\pending} \}}
\\ 
 \Num{2} & ~~~~ p \Asgn \esc{alloc}~(v, \Unmatched);\\
 \Num{3} & \specK{\{\heaps = p \mapsto (v; \Unmatched), \perms = \emptyset, \hists = \emptyset, \gist \subseteq \histo \hunion \mygather{\pending} \}}\\
 \Num{4} & ~~~~ b \Asgn \esc{CAS}~(g, \esc{null}, p);\\
 \Num{5} & ~~~~ \kw{if}~~b~\esc{==}~\esc{null}~~\kw{then}\\
 \Num{6} & ~~~~ ~~~~ \esc{sleep}~(50);\\
 \Num{7} & \specK{\{\heaps = \emptyset, \perms = \{p\}, \hists = \emptyset, \gist \subseteq \histo \hunion \mygather{\pending}, \esc{bounded}\ p\ v\ \gist \}}\\
 \Num{8} & ~~~~ ~~~~ x \Asgn \esc{CAS}~(p\esc{+}1, \Unmatched, \Retired);\\
 \Num{9} & \specK{\{\heaps = \emptyset, \perms = \emptyset, \gist \subseteq \histo \hunion \mygather{\pending},\hbox{}} \\
\Num{10} & \specK{\hphantom{\}}x = \Matched w \implies \exists t\ldot \hists = t \mapsto (v, w), \mathsf{last}(\gist) < t, \twin t,}\\
\Num{11} & \specK{\hphantom{\}}x = \Unmatched \implies \hists = \emptyset \}}\\
\Num{12} & ~~~~ ~~~~ \kw{if}~~x~~\kw{is}~~\Matched w~~\kw{then}~~\kw{return}~~(\esc{Some}~w)\\
\Num{13} & ~~~~ ~~~~ \kw{else}~~\kw{return}~~\esc{None}\\
\Num{14} & ~~~~ \kw{else}\\
\Num{15} & \specK{\{\heaps = p \mapsto (v; \Unmatched), \perms = \emptyset, \hists = \emptyset, \gist \subseteq \histo \hunion \mygather{\pending}\}}\\
\Num{16} & ~~~~ ~~~~ \esc{dealloc}~p;\\
\Num{17} & \specK{\{\heaps = \emptyset, \perms = \emptyset, \hists = \emptyset, \gist \subseteq \histo \hunion \mygather{\pending}\}}\\
\Num{18} & ~~~~ ~~~~ cur \Asgn \esc{read}~g;\\
\Num{19} & \specK{\{\heaps = \emptyset, \perms = \emptyset, \hists = \emptyset, \gist \subseteq \histo \hunion \mygather{\pending},}\\
\Num{20} & \specK{\hphantom{\}}cur = \esc{null} \vee cur \mapsto (w; -) \subseteq \heapj\}}\\
\Num{21} & ~~~~ ~~~~ \kw{if}~~cur~\esc{==}~\esc{null}~~\kw{then}~~\kw{return}~{\esc{None}}\\
%& \specK{\{ \mbox{obvious} \}}\\
\Num{22} & ~~~~ ~~~~ \kw{else}\\
\Num{23} & \specK{\{\heaps = \emptyset, \perms = \emptyset, \hists = \emptyset, \gist \subseteq \histo \hunion \mygather{\pending}, cur \mapsto (w; -) \subseteq \heapj\}}\\
\Num{24} & ~~~~ ~~~~ ~~~~ x \Asgn \esc{CAS}(cur\esc{+}1, \Unmatched, \Matched v);\\
\Num{25} & \specK{\{\heaps = \emptyset, \perms = \emptyset, \gist \subseteq \histo \hunion \mygather{\pending}, cur \mapsto (w; y) \subseteq \heapj,} \\
\Num{26} & \specK{\hphantom{\}}x = \Unmatched \implies y = \Matched v, \exists t\ldot \hists = t \mapsto (v, w), \mathsf{last}(\gist) < t, \twin t},\\
\Num{27} & \specK{\hphantom{\}}x \neq \Unmatched \implies \hists = \emptyset, y \neq \Unmatched \}}\\
\Num{28} & ~~~~ ~~~~ ~~~~ \esc{CAS}~(g, cur, \esc{null});\\
\Num{29} & \specK{\{\mbox{same as above; the state satisfies (iv) because $y \neq \Unmatched$}\}}\\
%& \specK{\{\heaps = \emptyset, \perms = \emptyset, \gist \subseteq \histo \hunion \mygather{\pending}, 
%    \exists w hw. cur \mapsto (w, hw) \in \heapj,} \\
%& \specK{x = \Unmatched, hw = \Matched v, \exists t. \hists = t \mapsto (w, v), last \gist < t, \twin t}\\
%& \specK{x \neq \Unmatched, \hists = \emptyset, hw \neq \Unmatched \}}\\
\Num{30} & ~~~~ ~~~~ ~~~~ \kw{if}~~x~\esc{==}~\Unmatched~~\kw{then}~~w\Asgn \esc{read}~cur;\kw{return}~(\esc{Some}\ w)\\
\Num{31} & \specK{\{ \heaps = \emptyset, \perms = \emptyset, \gist \subseteq \histo \hunion \mygather{\pending}, \res=\esc{Some}\ w}, \\
\Num{32} & \specK{\hphantom{\}}\exists t. \hists = t \mapsto (w, v), \mathsf{last}(\gist) < t, \twin t\}}\\
\Num{33} & ~~~~ ~~~~ ~~~~ \kw{else}~~\kw{return}~\esc{None}\}\\
\Num{34} & \specK{\{\heaps = \emptyset, \perms = \emptyset, \hists = \emptyset, \gist \subseteq \histo \hunion \mygather{\pending}, \res=\esc{None}\}} 
\end{array}
}}
\]
%\vspace{-15pt}  
\caption{Proof outline for the exchanger.}
\label{fig:exchanger_proof}
\end{figure} 
}


At line 24, the \code{CAS} succeeds if $x = \Unmatched$ and fails
otherwise. If \code{CAS} succeeded, then it ``matches'' the offer in $cur$;
that is, it writes $\Matched w$ into the hole of $cur$, and changes the
auxiliary state as follows. It takes $t$ to be the smallest unused
time-stamp in the history $\hist = \hists \hunion \histo \hunion
\mygather{\pending}$. Thus $\mathsf{last}(\hist) < t$, and because
$\hist$ has even size by invariant~(\ref{tag:exchanging}), $t$ must be
odd, and hence $t < \twin t = t+1$. The $t \mapsto (v, w)$ is placed
into $\hists$, giving us assertion 26.  To preserve the invariant
(iii), \code{CAS} simultaneously puts the entry $p \mapsto (t, w, v)$ into
$\pending$, for future collection by the thread that introduced offer
$cur$. But, we do not need to reflect this in line 26.
%
If the \code{CAS} fails, the history $\hists$ remains empty, as no matching is
done. However, the hole $y$ associated with $cur$ cannot be
$\Unmatched$, as then \code{CAS} would have succeded.
%
Therefore, it is sound in line 28 to ``unlink'' $cur$ from $g$, as the
unlinking will not violate the invariant (iv), which says that an
unmatched offer must be pointed to by $g$.
%
Finally, lines 30 and 33 select the assertion 26 or 27, and either
way, directly imply the postcondition.
