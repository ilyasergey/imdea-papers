

\section{Introduction}
\label{sec:introduction}

Linearizability~\cite{Herlihy-Wing:TOPLAS90} remains the
most well-known correctness condition for concurrent objects.
% 
It works by relating a concurrent object to a sequential behavior.
More precisely, for each concurrent history of an object,
linearizability requires that there exists a mapping to a sequential
history, such that the ordering of matching call/return pairs is
preserved either if they are performed by the same thread, or if they
do not overlap.
%
As such, linearizability has been used to establish the correctness of
a variety of concurrent objects such as stacks, queues, sets, locks,
and snapshots---all of which have intuitive sequential specs.
%

However, as argued by Shavit~\cite{Shavit:CACM11}, efficient
parallelization may require the development of concurrent objects
that are inherently \emph{non-linearizable}: in the presence of
interference, such objects exhibit behavior that is not reducible to
any sequential behavior via linearizability. To reason about such
objects, a variety of novel conditions has been developed:
concurrency-aware linearizability (CAL)~\cite{Hemed-Rinetzky:PODC14},
quiescent consistency (QC)~\cite{Aspnes-al:JACM94,Derrick-al:FM14},
quasi-linearizability (QL)~\cite{Afek-al:OPODIS10}, quantitative
relaxation~\cite{Henzinger-al:POPL13}, quantitative quiescent
consistency (QQC)~\cite{Jagadeesan-Riely:ICALP14}, and local
linearizability~\cite{Haas-al-local15}, to name a few.
%
These conditions, formulated as relations on execution traces, specify
a program's behavior under concurrent interference. Some, such as QC,
devote special treatment to the sequential case, qualifying the
behavior in the quiescent (\ie, interference-free) moments.
%
% \gad{ A minor comment about language: Along the paper, we are using
%   ``novel'' and ``non-standard'' intercheangably to refer to QC, CAL,
%   QCC, etc. --- compare, .e.g., the asbtract and the previous
%   paragraph--. In some places it reads oddly, specially when talking
%   about QC where the reference provided~\cite{Aspnes-al:JACM94} is old
%   enough to buy its own beer.}
% \an{Ok, changed novel into alternative.}

This proliferation of alternative conditions is problematic, as it
makes all of them non-canonical. For any specific example, it is
difficult to determine which condition to use, or if a new one should
be developed. Worse, each new condition requires a development of its
own dedicated program logic or verification tool. Furthermore, it is
unclear how to combine the conditions/logics/tools, when different
ones have been used for different subprograms.
%
Finally, having criteria defined \emph{semantically}, \eg, in terms of
execution traces, makes it challenging to employ them directly for
reasoning about clients of the corresponding data structures.

%In particular, in contrast to linearizability, which has been shown to
%imply observational
%refinement~\cite{Filipovic-al:TCS10,Emmi-al:PLDI15} (that is, a
%program can be replaced in any larger context by the set of the
%sequential histories to which it linearizes; every property derivable
%about the replacement applies to the original program as well), no
%similar results have been proven for the aforementioned consistency
%conditions.

%
% While some of them, say QC and QQC, are known to be compositional in
% the sense that the combination of two QC (QQC) objects is QC
% (QQC)~\cite{Herlihy-Shavit:08,Jagadeesan-Riely:ICALP14}, such
% compositionality is much weaker than observational refinement, and
% does not allow transferring a general property (such as, one expressed
% as a Hoare triple~\cite{Turon-al:ICFP13,Liang-Feng:PLDI13}) to a
% program, from the set of histories with which it is QC (QQC)
% consistent.
%

%While linearizability has been generalized to apply to modern
%%
%concurrent programs, which use higher-orderness and dynamic
%allocation~\cite{Gotsman-Yang:CONCUR12,Cerone-al:ICALP14}, the
%alternative consistency conditions almost invariably focus on simple
%imperative programs. Finally, the considerations of the alternative
%conditions have focused only on their semantics: there is a lack of
%syntactic logical methods for checking that a program satisfies one of
%them (again, in contrast to the situation for establishing
%linearizability for a given
%program~\cite{OHearn-al:PODC10,Liang-Feng:PLDI13,Turon-al:ICFP13,Vafeiadis:PhD}).
%%
%Such methods are desirable, as they allow one to verify clients and
%implementations in a single proof system, and are amenable for
%efficient support for constructing formal machine-checked
%proofs~\cite{Appel-al:BOOK14,Chlipala:PLDI11}.
%
%

% \is{Can we say, why such methods are desirable, and why they are
%   better than reasoning directly in terms of program semantics?}
% %
% \is{Presumably, uniform reasoning about clients in the presence of
%   HO, dynamic state, amenable to scalable computer-aided
%   verification. }

\subsection{Concurrency Specification via Program Logics}

In this paper, we propose an alternative, uniform, approach: a Hoare
logic equipped with special \emph{subjective} kind of auxiliary
state~\cite{LeyWild-Nanevski:POPL13} that makes it possible to name
the amount of concurrent interference, and relate it to the program's
inputs and outputs \emph{directly}, without reducing to sequential
behavior. We use Fine-grained Concurrent Separation Logic
(FCSL)~\cite{Nanevski-al:ESOP14}, designed to reason about
higher-order concurrent programs, and has been recently implemented as
a verification tool on top of Coq~\cite{Sergey-al:PLDI15}, but whose
ability to address non-linearizable programs has not been observed
previously.
%
% \gad{I'd rephrase the last part. Right now, it reads apologetic of the
%   fact we're just giving new uses to FCSL and, imho, it sets the wrong
%   tone for the next paragraphs, by making it us sound defensive about
%   not having done something completely/radically new, instead of
%   making that point work in our advantage from the beginning. Also it
%   doesn't say right away why we are using FCSL.}  
% %
% \an{I didn't think this is apologetic; rather, it just says that we
%   were surprised to notice that we can do this in FCSL. Others didn't
%   see this either, e.g., CAL people explicitly say that ``FCSL cannot
%   do this''. Reviewers always pick on this anyhow, so we need to
%   address.}
%
%From these observations stems a fundamental question: Can the
%alternative correctness conditions be represented in one and the same
%logical system, with support for higher-order, compositional,
%machine-assisted reasoning about realistic libraries of modern,
%possibly non-linearizable, concurrent programs and their clients?
%%
%
%%In short, there is a lack of proof methods amenable to structured
%%formal verification with non-linearizable objects, and formal results
%%that enable these proof methods to apply be compositionally used on clients.
%
%%\ab{How to bring in client reasoning and associated problems?}
%%
%% \is{Here we should clarify what we mean by compositionality. QC and
%%   QQC are also proven to be compositional by their authors, so that
%%   statement above is misleading if not wrong.}
%
%%Despite this variety, it is not obvious how these criteria can
%%facilitate the verification of modern concurrent programs which use
%%higher-orderness, ownership transfer, dynamic allocation, etc. In
%%contrast to linearizability which has been shown to imply
%%observational refinement\footnote{That is, a program can be replaced
%%in any larger context by the set of the sequential histories to
%%which it linearizes.}~\cite{Filipovic-al:TCS10}, no similar results
%%have been proven for the aforementioned consistency criteria.
%%%
%%Neither are syntactic logical methods for establishing the
%%consistency criteria known (again, in contrast to the situation for
%%establishing linearizability for a given
%%program~\cite{Gotsman-Yang:CONCUR12,Cerone-al:ICALP14,Turon-al:ICFP13}).
%%%
%%While QC and QQC are known to be compositional~\cite{What?}, such a property 
%%only asserts that the composition of two QC (QQC) objects is QC (QQC). 
%%Such compositionality is too weak to be applicable in a situation
%%where, say, two procedures verified under different criteria need to be used in 
%%the same program, and the program's precondition, involving the different 
%%criteria, needs to be established. 
%%%
%%In other words, there is a lack of compositional proof methods amenable to 
%%structured formal verification with non-linearizable objects. 
%%\ab{How to bring in client reasoning and associated problems?}
%%%
%%% \is{Here we should clarify what we mean by compositionality. QC and
%%%   QQC are also proven to be compositional by their authors, so that
%%%   statement above is misleading if not wrong.}
%
%%
%%Moreover, client-side reasoning about programs that incorporate
%%several objects specified via \emph{different} criteria becomes
%%enormously complicated: clients of a concurrent object, when
%%committing to a correctness criterion, need to adopt specific
%%reasoning principles to characterize the object's behavior.  It can be
%%difficult to ensure compositionality of reasoning when dealing with
%%different communicating objects.
%%%
%%\is{The statement above is a bit vague (now I realize) is and isn't
%%  instantiated particularly well in our paper: we don't show examples
%%  with several objects (although we could). So how about we say here
%%  what the previous intro used to say, that for each new criterion one
%%  has to devise a method for exploiting the provided safety guarantees
%%  for the sake of reasoning about client code that uses the concurrent
%%  object.}
%%
%%\is{In my opinion, the paragraph above is crucial for the whole story,
%%  as it sets the motivation for the paper (like those questions we had
%%  previously), so it should be more punchy in describing what the
%%  problems are with the state of the art, and why one shoudl care
%%  about them.}  \an{How about: Moreover, while syntactic logical
%%  methods exist for establishing linearizability for a given program
%%  (CITE some stuff by Gotzman, and CaReSL), such methods do not exist
%%  for the other criteria. Even if such methods existed, one would have
%%  to engineer ways of combining them into a unified framework,
%%  whenever two procedures verified by different criteria are to be
%%  used in the same program. }
%
%% \is{While I totally agree with all of the things said above, in my
%%   opinion, at this very place we need to place a punch-phrase (a
%%   slogan), that summarizes the problem we attack. Something, in the
%%   style of herr Dreyer, \eg, \emph{These observations beg the
%%     question: ...?}}  \an{I suppose we can pull a Dreyer here. In
%%   principle, his overselling has been noted by many people, so I'm not
%%   too convinced that we should follow his approach. But, we could. We
%%   could say something like: These observation beg the question whether
%%   all these alternative conditions can be represented in a single
%%   unified logical system, with support for higher-order compositional
%%   reasoning about realistic libraries of modern concurrent programs.}
%   a
%\subsection{Our approach: logic-based specifications}

%This paper demonstrates a uniform approach---based on a Hoare-style
%program logic---for verifying the correctness of highly scalable
%concurrent objects and their clients, without recourse to specialized
%correctness criteria and consistency conditions. Our approach uses
%Fine-grained Concurrent Separation Logic
%(FCSL)~\cite{Nanevski-al:ESOP14}, which was designed to reason about
%modern higher-order concurrent programs and has been recently
%implemented as a verification tool~\cite{Sergey-al:PLDI15}.

%We show, via examples, that the basic ingredient of FCSL,
%\emph{subjectivity}~\cite{LeyWild-Nanevski:POPL13}, provides the
%uniformity we seek. Subjectivity permits that within a spec of a
%thread, one can refer to the private state of all other interfering
%threads \emph{in a local manner}. This allows one to directly express
%the results of a program as a function of the interference of other
%threads, and ultimately yields uniform reasoning principles capable of
%expressing the essential properties captured by the various
%non-canonical correctness conditions for concurrency.

More specifically, subjective auxiliary state permits that within a
spec of a thread, one can refer to the private state (real or
auxiliary) of \emph{other} interfering threads \emph{in a local
  manner}. This private state can have arbitrary user-specified
structure, as long as it satisfies the properties of a partial
commutative monoid (PCM).
%
A particularly important PCM is that of \emph{time-stamped histories},
which has previously been applied to linearizable
objects~\cite{Sergey-al:ESOP15}, where it replaced call/return
histories. A (logically) time-stamped history consists of entries of
the form $t\,{\mapsto}\,a$, signifying that an atomic behavior $a$
occurred at a time (or linearization point)~$t$. A subjective
specification further distinguishes the histories of the thread and
its interfering environment, and usefully relates both to the thread's
input and output.

Of course, Hoare-style reasoning about histories is a natural idea,
exploited recently in several
works~\cite{Fu-al:CONCUR10,Gotsman-al:ESOP13,Bell-al:SAS10,Hemed-al:DISC15}. Here,
however, we rely on the unifying power of PCMs, in combination with
subjective specifications, to show that by generalizing histories in
different ways---though all subject to PCM laws---we can capture the
essence of several different conditions, such as CAL, QC and QQC in
one-and-same \emph{off-the-shelf} logical system and tool.
%
More precisely, our histories need not merely identify a point at
which an atomic behavior logically occurred, but can also include
information about interference, or lack thereof. Moreover, we will use
generic FCSL constructs for delimiting the scope of auxiliary state,
to reason about quiescent moments.
%
%
% \gad{The ``unifying power of PCMs' and subjectivity'' thing might
%   sound a bit vague to a lot of people.}  \an{Changed ``to wit'' to
%   ``more precisely'', to make it less vague, and more, well, precise
%   :-)}
%Since our approach encompasses a number of correctness criteria. 

\subsection{Contributions and Outline}
\label{sec:chall-contr}

The ability to use FCSL for specifying and verifying
\emph{linearizable} objects (\eg, fine-grained stacks and atomic
snapshots) has been recognized before~\cite{Sergey-al:ESOP15}.
%
In contrast, the main conceptual contribution of this work is an
observation that the \emph{very same} abstractions provided by FCSL
are sufficient to ascribe non-trivial \emph{non-linearizable} objects
with specs that can hide object implementation details, but are
sufficiently strong to be used in proofs of concurrent client
programs, as we demonstrate in Section~\ref{sec:overview}.
%
Specifically, we recognize that auxiliary histories can be subject of
user-defined invariants beyond mere adherence to sequential executions
(\eg, be concurrency-aware~\cite{Hemed-Rinetzky:PODC14}), and can be
used to capture intermediate interference, allowing for quantitative
reasoning about outcomes of concurrent executions (\eg, in the spirit
of QQC~\cite{Jagadeesan-Riely:ICALP14}). These observations,
surprisingly, enabled reasoning about non-linearizable data structures
and their clients, which were never previously approached from the
perspective of program logics or mechanically verified.

In this unified approach based on program logic, it seems inherently
impossible (and contrary to the whole idea) to classify Hoare triples
as corresponding to this or that correctness condition. Thus, instead
of providing theorems that relate Hoare triples to existing
conditions, we justify the adequacy of our approach by
proof-of-concept verifications of concurrent objects and their
clients.

Hence, as key technical contributions, we present \emph{subjective
  specs and the first mechanized proofs} (in Coq) of (1) an
elimination-based exchanger~\cite{Scherer-al:SCOOL05}
(Section~\ref{sec:exchanger}), previously specified using CAL, and (2)
a simple counting network~\cite{Aspnes-al:JACM94}
(Section~\ref{sec:counting}) that inspired definitions of QC and
QQC. We then employ these specs to verify client
programs~(Sections~\ref{sec:cal} and~\ref{sec:qclients}).
%
% \ifext{}{\footnote{The proof scripts are
%     available in supplementary material for the paper.}}
%
We discuss alternative design choices for specs and further
applications of our verification approach in
Section~\ref{sec:discussion}, and summarize our mechanization
experience in Section~\ref{sec:evaluation}. Section~\ref{sec:related}
compares to related work and Section~\ref{sec:conclusion} concludes.

%

%Traditionally, correctness criteria for concurrent objects are
%formulated in terms of call/return histories of threads, and their
%rearrangements. In contrast, assertions in Hoare-style program logics
%constrain \emph{state}, auxiliary or real, in which the program runs.
%%
%To bridge this gap, Hoare-style reasoning has been recently extended
%to histories, which were formulated as a specific instance of
%auxiliary
%state~\cite{Fu-al:CONCUR10,Gotsman-al:ESOP13,Sergey-al:ESOP15,Bell-al:SAS10}.
%
%
% our starting point is the representation of a program's
% history directly as user-specified auxiliary state. Such a
% representation facilitates reasoning about history via Hoare-style
% specs. This is a simple and old idea~\cite{what}, that in FCSL comes
% with a twist.
%
% \is{This statement seems like it's taken directly from the ESOP'15
%   paper intro (including the twis bit). However, there it was
%   referring to histories in concurrency in general (including
%   semantics). However, I don't think that histories in Hoare-style
%   logics are an \emph{old} idea. So, may be, we can just say that
%   histories were used in previous logics to reason about FCD and cite
%   HLRG and Gotsman-Yang?}
%
% \is{The paragraph before should be changed to introduce auxiliary
%   state and related logics and then proceed to elaborate on FCSL.}
%
%For instance, instead of call/return histories, FCSL allows one to
%employ \emph{time-stamped histories}~\cite{Sergey-al:ESOP15} to reason
%about linearizable objects. Such histories consist of entries of the
%form $t \mapsto a$, to signify that the (typically atomic) operation
%$a$ occurred at time $t$. A Hoare-style spec that shows that a
%program's history changes by a singleton $t \mapsto a$ can be seen as
%exhibiting a behavior $a$ at a linearization point~$t$.
%%
%Such specification approach makes fine-grained (\ie, lock-free)
%objects look like atomic ones to the clients, whose proofs are carried
%out only out of the object specs.
%
%%\an{Some comment here on the similarity between histories and heaps.}
%
%In this work, we augment this history-based approach to Hoare-style
%specs in a significant way to handle non-linearizable objects. We show
%that more \emph{general notions of time-stamped histories lead to
%  adequately capturing the essence of alternative consistency
%  criteria} such as CAL, QC, and QQC/QL.  
%To wit, such histories need
%not merely identify a point at which an atomic behavior logically
%occurred, but also can include information about thread interference.

% For example, the main idea of CAL is that histories with which one
% linearizes cannot be sequential, but have to be concurrency-aware
% (CA), \ie, allow simultaneous events to be represented. In FCSL we can
% do so by picking time-stamped histories with additional imposed
% structure that naturally captures the simultaneity of events. In
% Section~\ref{sec:overview}, we show how this structure helps in
% specifying and verifying---in FCSL---an elimination-based concurrent
% exchanger~\cite{Scherer-al:SCOOL05}. In Section~\ref{sec:cal}, we show
% how to immediately employ the ascribed specification for the
% verification of a client program of the exchanger (adapted directly
% from the \code{java.util.concurrent} library
% documentation~\cite{ExchangerClass}) in the same logical framework.

% QC requires establishing that at moments of quiescence, \ie, no
% interference, programs exhibit some desirable behavior. For example,
% at quiescence, a concurrent counter implementation behaves as expected
% of a \emph{sequential} counter implementation. We capture this
% property by relying on subjectivity: we use time-stamped histories in
% which a time-stamp $t$ directly stores the kind of interference
% exhibited by the program's environment at time $t$.  One can then
% prove, that in the absence of interference, the object behaves
% sequentially as expected. In Section~\ref{sec:counting} and
% Section~\ref{sec:qclients} we show the specification and
% verification---in FCSL---of a simple counting
% network~\cite{Aspnes-al:JACM94} and its client, both of whose
% correctness relies on QC.

% One can also derive stronger, \emph{quantitative}, properties, and
% show that a bound on the number of interfering threads implies that
% the program exhibits a bounded deviation from the expected sequential
% behavior. In the past, this has been addressed using
% QL~\cite{Afek-al:OPODIS10} and QQC~\cite{Jagadeesan-Riely:ICALP14} as
% correctness criteria. In this paper, we derive it as a consequence of
% the choice of the auxiliary state of
% histories. Section~\ref{sec:qclients} also shows the verification of a
% client of the counting network, whose correctness relies on QQC.
%
%Thus, our theoretical contribution is in recognizing that Hoare logic 
%Hoare-logic specification patterns that can be used to verify 
%Hoare-style program logic, properties of concurrent objects, which
%have been captured previously only via dedicated correctness criteria.
%% 
%The central practical contribution is a full mechanization of the
%specs and proofs for a series of highly-scalable non-linearizable
%concurrent objects and their clients, employing the presented
%reasoning patterns, in an existing logic-based framework for
%concurrency verification, taken \emph{off-the-shelf}.
%%







% The unifying mechanism behind all these different kinds of histories
% (and indeed behind the subjective split of any auxiliary state) is
% that they all satisfy the algebraic properties of a \emph{partial
%   commutative monoid} (PCM)~\cite{LeyWild-Nanevski:POPL13}. 

% Thus, FCSL can represent them, in addition to heaps (also a PCM, and
% often a shared resource) in a uniform reasoning framework, applying
% the same logical infrastructure (such as the rule of frame) to all
% kinds of state, auxiliary or real, in the process also incorporating
% higher-orderness, ownership transfer, and dynamic
% allocation~\cite{Nanevski-al:ESOP14,Sergey-al:ESOP15}.
% %
% The uniformity of the logical rules, treating all kinds of state
% similarly, makes it possible to conduct the verification in a general
% computer-assisted framework: all proofs of the examples from this
% paper are checked mechanically in Coq~\cite{Coq-manual} and are
% available as a supplementary material.

% \paragraph{Alternative logic-based approaches.}

% Recent concurrent program logics, such as
% HOCAP~\cite{Svendsen-al:ESOP13}, iCAP \cite{Svendsen-Birkedal:ESOP14},
% TaDA~\cite{ArrozPincho-al:ECOOP14}, and Iris~\cite{Jung-al:POPL15}
% have shown, using the technique of parametrizing programs with
% \emph{first-class auxiliary code}~\cite{Jacobs-Piessens:POPL11} or
% \emph{atomic tracking resources} (see Section~\ref{sec:related} for
% details), that Hoare-style program logics can adequately specify and
% verify tricky linearizable concurrent objects and their clients. In
% contrast, this work addresses non-linearizable objects and their
% clients---but without the use of such parameterization or atomic
% tracking resources, which both seem to require identifying
% \emph{synchronization points} within libraries, making it non-trivial
% to apply the listed above logics to the objects we consider. In the
% process we also derive properties which have hitherto been obtained
% only via dedicated alternative correctness criteria.

% \paragraph{Observational refinement and compositional reasoning.}
% The fact that linearizability implies observational
% refinement~\cite{Filipovic-al:TCS10, Cerone-al:ICALP14,
%   Bouajjani-al:POPL15, Emmi-al:PLDI15} justifies compositional
% reasoning, whereby a program can be substituted by its sequential spec
% \emph{no matter the property being verified}. Here, we consider
% objects whose correctness criteria do not necessarily imply such
% observational refinement. Hence, we fix our properties of interest to
% be partial correctness Hoare-style specs only. In that setting,
% compositionality of the reasoning is justified by the substitution
% principle of FCSL on both programs and
% proofs~\cite{Nanevski-al:ESOP14}.









