\section{Main Ideas and Overview}
\label{sec:overview}

We begin by outlining the high-level intuition of our specification
approach, and summarize the main formalization steps.
%
As the first motivating example, we consider the concurrent exchanger
structure from $\mathtt{java.util.concurrent}$
\cite{Scherer-al:SCOOL05,ExchangerClass}. The main purpose of the
exchanger is to allow two threads to efficiently swap values in a
non-blocking way via a globally shared channel. The exchange might
fail, if a thread trying to swap a value does not encounter a peer to
do that in a predefined period of time.
%
% In the case of a successful outcome, two ``colliding'' threads
% simultaneously exchange values, otherwise a thread with no matching
% exchange proposal fails.

For instance, the result of the two-thread program
%
\[
\tag{\arabic{tags}}\refstepcounter{tags}\label{tag:ex1} 
\begin{array}{c@{\ }c@{\ }c} 
  \boxed{T_1} & & \boxed{T_2}
  \\[5pt] 
   r_1 := \esc{exchange}~1 & || & r_2 := \esc{exchange}~2
\end{array}
\]
%
can be described by the following assertion:\footnote{We
  use ML-style \texttt{option} data type with two constructors,
  \texttt{Some} and \texttt{None} to indicate success and failure of
  an operation, correspondingly.}

%
\[
\tag{\arabic{tags}}\refstepcounter{tags}\label{tag:exsp} 
r_1 = r_2 = \esc{None} \oor r_1 = \esc{Some}~2 \wedge r_2 =
\esc{Some}~1
\]

\noindent
That is, $r_1$ and $r_2$ store the results of the execution of
subthreads $T_1$ and $T_2$ correspondingly, and both threads either
succeed, exchanging the values, or fail. The ascribed outcome is only
correct under the assumption that no other threads besides $T_1$ and
$T_2$ attempt to use the very same exchange channel concurrently.

Why is the exchanger not a linearizable data structure? To see that,
recall that linearizability reduces the concurrent behavior to a
sequential one~\cite{Herlihy-Wing:TOPLAS90}. If the exchanger were
linearizable, all possible outcomes of the program~\eqref{tag:ex1}
would be captured by the following two sequential programs, modelling
selected interleavings of the threads $T_1$ and $T_2$:
%
\[
\tag{\arabic{tags}}\refstepcounter{tags}\label{tag:ex2} 
\begin{array}{c}
r_1 := \esc{exchange}~1;~r_2 := \esc{exchange}~2;
\\[3pt]
\text{and}
\\[3pt]
r_2 := \esc{exchange}~2;~r_1 := \esc{exchange}~1;  
\end{array}
\]
%
However, both programs~\eqref{tag:ex2} will \emph{always} result in
$r_1 = r_2 = \esc{None}$, as, in order to succeed, a call to the
exchanger needs another thread, running concurrently, with which to
exchange values.
%
% \ab{Check rewording.}
% \is{Looks good}
%
This observation demonstrates that linearizability with respect to a
sequential specification is too weak a correctness criterion to
capture the \code{exchanger}'s behavior observed in a truly concurrent
context~\cite{Hemed-Rinetzky:PODC14}: an adequate notion of
correctness for \code{exchange} must mention the effect
of {interference}.
%
% \ab{Check rewording.}
% \is{Looks good}

Consider another structure, whose concurrent behavior cannot be
related to sequential executions via linearizability:
%
\[
\tag{\arabic{tags}}\refstepcounter{tags}\label{tag:flip} 
\begin{array}{l}
\esc{flip2}~(x : \esc{ptr~nat})~:~\esc{nat}~=~\{ \\[2pt]
~~ a := \esc{flip}~x; \\[2pt]
~~ b := \esc{flip}~x;\\[2pt]
~~\kw{return}~a + b ~~\}  
\end{array}
\]
%
The procedure \code{flip2} takes a pointer $x$, whose value is
either~0 or~1 and changes its value to the opposite, twice, via the
\emph{atomic} operation \code{flip}, returning the sum of the previous
values. Assuming that $x$ is being modified only by the calls to
\code{flip2}, what is the outcome $r$ of the following program?
%
\[
\tag{\arabic{tags}}\refstepcounter{tags}\label{tag:flip2} 
r := \esc{flip2}~x;
\]
%
The answer depends on the presence or absence of interfering threads
that invoke \code{flip2} concurrently with the
program~\eqref{tag:flip2}. Indeed, in the absence of interference,
\code{flip2} will flip the value of $x$ twice, returning the sum of 0
and 1, \ie,~1. However, in the presence of other threads calling
\code{flip2} in parallel, the value of $r$ may vary from 0 to 2.
%
% depending on the concurrent calls to \code{flip}, changing the value
% of~$x$.

What are the intrinsic properties of \code{flip2} to be specified?
%
Since the effect of \code{flip2} is distributed between two internal
calls to \code{flip}, both subject to interference, the specification
should capture that the variation in \code{flip2}'s result is
subject to interference.
%
% \an{The above paragraph promises to say what an adequate spec for
% flip2 is, but doesn't do it. It just says what such a spec should
% enable infering. Either weaken the first sentence to set up the
% right expectations, or strenghten the rest, to deliver the right
% spec.}
%
% \is{Done}
%
% \ab{Shortened.}
%
Furthermore, the specification should be expressive enough to allow
reasoning under bounded interference. For example, in the absence of
interference from any other threads besides $T_1$ and $T_2$ that
invoke \code{flip2} concurrently, the program below will always result
in~$r = 2$:
%
% \ab{Rephrased.}
% \is{Looks good}
%
\[
\tag{\arabic{tags}}\refstepcounter{tags}\label{tag:flip3} 
\begin{array}{c@{\ }c@{\ }c@{\ }l@{\ }l} 
  \boxed{T_1} & & \boxed{T_2}
  \\[5pt] 
  r_1 := \esc{flip2}~x & || & r_2 := \esc{flip2}~x; \\[3pt]
  \multicolumn{3}{c}{r :=  r_1 + r_2}
\end{array}
\]

\subsection{Abstract Histories of Non-linearizable Objects}
\label{sec:hist}

Execution histories capture the traces of a concurrent object's
interaction with various threads, and are a central notion for
specifying concurrent data structures.
%
For example, linearizability specifies the behavior of an object by
mapping the object's global history of method invocations and returns
to a sequence of operations that can be observed when the object is
used sequentially~\cite{Herlihy-Wing:TOPLAS90}. However, as we have
shown, neither \code{exchange} nor \code{flip2} can be understood in
terms of sequential executions.

We propose to specify the behavior and outcome of such objects in
terms of \emph{abstract concurrent histories}, as follows. Instead of
tracking method invocations and returns, our histories track the
``interesting'' changes to the object's state. What is ``interesting''
is determined by the user, depending on the intended clients of
the concurrent object.
%
% \ab{What's interesting state?}  
%
% \is{Hmm... Do we really have to be more formal here? I tried to
% clarify}
%
%to particular threads. 
%
Moreover, our specifications are subjective (\ie, thread-relative) in
the following sense. Our histories do not identify threads by their
thread IDs. Instead, each method is specified by relating two
different history variables: the history of the invoking thread
(aka.~\emph{self}-history), and the history of its concurrent
environment (aka.~\emph{other}-history). In each thread, these two
variables have different values.
%
%This subjective view will allow us to easily specify programs with
%dynamically-changing number of threads.

%This will also allow us to relate the results of concurrent calls of
%object's methods to the history contributions, made by a thread and
%its interference, correspondingly.
%
% \ab{I am finding it difficult to parse this sentence.}
%
% \is{Rephrased}

For example, in the case of the exchanger, the interesting changes to
the object's state are the exchanges themselves. Thus, the global
history $\hist_{\cal E}$ tracks the successful exchanges in the form
of pairs of values, as shown in below:
%
\[
%\tag{\arabic{tags}}\refstepcounter{tags}\label{tag:hist1} 
\!\!\!\begin{array}{r@{\ }c@{}c@{\ }c@{\ }c@{\ }c@{\ }c@{\ }c@{\ }c@{}l}
  & & \boxed{T_1} & \graybox{\boxed{T_2}} & \graybox{\boxed{T_2}} & \graybox{\boxed{T_3}} & \graybox{\boxed{T_2}}
  & \boxed{T_1} 
  \\[5pt]  
  \hist_{\cal E} \!=\! \text{[} &\!\!...,~& (1, 2), & \graybox{(2, 1)}, & 
                                 \graybox{(4, 5)}, & \graybox{(5, 4)}, &  \graybox{(9, 8)}, & (8, 9), &
                                                                          ...
                     &
                       \text{]}   
\\[-5pt]   
&& \multicolumn{2}{l}{\underbracket{\phantom{aaaaaaaaa}}_{\mathtt{exchange~ok}}} & \multicolumn{2}{c}{\underbracket{\phantom{aaaaaaaaa}}_{\mathtt{exchange~ok}}} & \multicolumn{2}{c}{\underbracket{\phantom{aaaaaaaaa}}_{\mathtt{exchange~ok}}}
                  & & 
\end{array}
\]
%
%
The diagram presents the history from the viewpoint of thread
$T_1$. The exchanges made by $T_1$ are colored white, determining the
\emph{self}-history of $T_1$. The gray parts are the exchanges made by
the other threads (e.g., $T_2$, $T_3$, \etc.), and determine the
\emph{other}-history for $T_1$. 

% \an{We need here a diagram that colors the same total history, but
%   from the point of view of $T_2$.}

% \is{No, I think, we don't...}

The subjective division between \emph{self} and \emph{other} histories
emphasizes that a successful exchange is actually represented by
\emph{two} pairs of numbers $(x, y)$ and $(y, x)$, that appear
consecutively in $\hist_{\cal E}$, and encode the two ends of an
exchange from the viewpoint of the exchanging threads. We call such
pairs \emph{twins}.
%
%and will show in Section~\ref{sec:exchanger} how to encode that two
%pairs are twins.
As an illustration, the white entry $(2, 1)$ from the self-history of
$T_1$, is matched by a twin gray entry $(1, 2)$ from the other-history
of $T_1$, encoding that $T_1$ exchanging $2$ for $1$ corresponds to
$T_1$'s environment exchanging $1$ for $2$.

The subjective division is important, because it will enable us to
specify threads \emph{locally}, \ie, without referring to the code of
other threads. For example, in the case of program~(\ref{tag:ex1}), we
will specify that $T_1$, in the case of a successful exchange, adds a
pair $(1, r_1)$ to its self history, where $\esc{Some}~r_1$ is $T_1$'s
return value. Similarly, $T_2$ adds a pair $(2, r_2)$ to
its self history, where $\esc{Some}~r_2$ is $T_2$'s return value.
%
% \an{Fix the option values above please.}

On the other hand, it is an important invariant of the exchanger
object---but not of any individual thread---that twin entries are
symmetric pairs encoding different viewpoints of the one-and-the-same
exchange. This object invariant will allow us to reason about clients
containing combinations of exchanging threads. Taking
program~(\ref{tag:ex1}) as an example again, the object invariant will
imply of the individual specifications of $T_1$ and $T_2$, that $r_1$
must equal $2$, and $r_2$ must equal $1$, if no threads interfered
with $T_1$ and $T_2$.

%We emphasize here that the object invariants are, as the name
%suggests, object-specific. Our Hoare-style approach will allow the
%user to declare the above invariant in the case of the exchanger, but
%will not mandate it for examples for which it is not relevant. This is
%in contrast to the specification by correctness conditions, which are
%typically not parametrizable by user-defined properties, and are thus
%less flexible (\eg, linearizability does not allow users to declare
%history invariants on a per-object basis).
%\an{Is the above paragraph too strong?}

%This invariant is what makes possible the verification of the
%example~\eqref{tag:ex1} with respect to the
%assertion~\eqref{tag:exsp}.

%Indeed, in the case of a successful exchange, each of the threads
%$T_1$ and $T_2$, can now attribute to itself an entry $(x, r)$ in the
%total exchanger history $\hist_{\cal E}$, where $x$ is a
%\emph{self}-contributed value and $r$ is a thread-specific result of
%calling \code{exchange}, corresponding to a value contributed by some
%\emph{other} thread.
%%
%Furthermore, the ``closed-world'' assumption, stating the absence of
%external interference, allows us to deduce that, in the case of
%success, the \emph{overall} interaction history of the exchanger for
%the time span of the program~\eqref{tag:ex1}, consists only of the two
%``twin'' entries, $(1, 2)$ and $(2, 1)$, which correspond to a single
%exchange transaction. Hence, by the history invariant,
%$r_1 = \esc{Some}~2$ and $r_2 = \esc{Some}~1$.

We can similarly employ abstract histories to specify
\code{flip2}. One way to do it is to notice that the value of the
shared counter $x$ will be changing as $0, 1, 0, 1, \ldots$, and
exactly \emph{two} of these values will be contributed by each call to
\code{flip2} made by some thread. We can depict a particular total
history $\hist_{\cal F}$ of the \code{flip2} structure as follows:
%
\[
%\tag{\arabic{tags}}\refstepcounter{tags}\label{tag:hist2}
%
\!\!\!\!\!\!\!\!
%
\begin{array}{r@{\ }c@{\ }c@{\ }c@{\ }c@{\ }c@{\ }c@{\ }c@{\ }c@{\ }l@{\ }}
  & & \boxed{T_1} & \graybox{\boxed{T_2}} & {\boxed{T_1}} & \graybox{\boxed{T_2}} & \graybox{\boxed{T_3}}
  & \graybox{\boxed{T_3}} 
  \\[5pt] 
  \hist_{\cal F} = \text{[} & ..., & 1, & \graybox{0}, & 
                                 1, & \graybox{0}, &  \graybox{1}, & \graybox{0}, &
                                                                          \ldots
                     &
                       \text{]}   
\\[-5pt] 
&& \multicolumn{3}{l}{{\underbracket{\phantom{aaaaaa}}_{T_1.\mathtt{flip2}}}}
\end{array}
\]
%
The two ``white'' contributions are made by thread $T_1$'s call to
\code{flip2}, while the rest (gray) are contributions by $T_1$'s
environment. Since the atomic \code{flip} operation returns the
\emph{complementary} (\ie, previous) value of the counter, the overall
result of $T_1$'s call in this case is $\bar{1} + \bar{1} = 0 + 0
=~0$.

The invariant for the \code{flip2} structure postulates the
interleaving 0/1-shape of the history and also ensures that the last
history entry is $x$'s current value. This will allow us to reason
about clients of \code{flip2}, such as~\eqref{tag:flip3}.
%
% \an{We're not specifying this shortly, so better say here what the
%   invariant is, and how it helps with deducing anything in the absence
%   of interference.}
%
In the absence of interference, we can deduce that the two parallel
calls to \code{flip2} have contributed four \emph{consecutive} entries
to the history $\hist_{\cal F}$, with each thread contributing
precisely two of them. For each of the two calls, the result equals
the sum of the two complementary values for what the corresponding
thread has contributed to the history, hence, the overall sum
$r_1 + r_2$ is $2$.

% \paragraph{Histories as subjective auxiliary state}

\subsection{Hoare-style Specifications for \texttt{exchange} and
  \texttt{flip2}}
\label{sec:hoare}

The above examples illustrate that subjectivity and object invariants
are two sides of the same coin. In tandem, they allow us to specify
threads individually, but also reason about thread combinations. We
emphasize that in our approach, the invariants are
\emph{object-specific} and \emph{provided by the user}. For example,
we can associate the invariant about twin entries with the exchanger
structure, but our method will not mandate the same invariant for
other structures for which it is not relevant. This is in contrast to
using a fixed correctness condition, such as linearizability, QC, or
CAL, which cannot be parametrized by user-defined
properties.\footnote{For example, linearizability does not allow users
  to declare history invariants on a per-object basis. The exchanger
  example motivated the introduction of the correctness condition
  CAL~\cite{Hemed-Rinetzky:PODC14}, which relaxes linearizability, and
  makes it somewhat more general in this respect, but still falls
  short of admitting user-defined invariants. \texttt{flip2} can be
  specified using a variation of QC~\cite{Jagadeesan-Riely:ICALP14},
  but we show that a similar property can be expressed via
  subjectivity and a user-defined invariant.}

%suggests, object-specific. Our Hoare-style approach will allow the
%user to declare the above invariant in the case of the exchanger, but
%will not mandate it for examples for which it is not relevant. This is
%in contrast to the specification by correctness conditions, which are
%typically not parametrizable by user-defined properties, and are thus
%less flexible (\eg, linearizability does not allow users to declare
%history invariants on a per-object basis).
%\an{Is the above paragraph too strong?}


%main idea demonstrated by the two examples above is to specify the
%effects of a thread's interaction with a concurrent object via its
%contributions to the abstract history, subject to a
%\emph{object-specific} invariant, instead of using a fixed correctness
%condition, such as linearizability, QC or CAL.
%\footnote{The exchanger
%  example motivated the introduction of the correctness condition
%  CAL~\cite{Hemed-Rinetzky:PODC14}, strengthening the classical notion
%  of linearizability, and \texttt{flip2} can be specified using a
%  variation of QC~\cite{Jagadeesan-Riely:ICALP14}.}

Subjective histories can be encoded in our approach as \emph{auxiliary
  state}~\cite{Sergey-al:ESOP15,Owicki-Gries:CACM76}. Our Hoare
triples will specify how programs modify their histories, while the
invariants are declared as properties of a chunk of shared state
(\eg,~resource invariants of~\cite{Owicki-Gries:CACM76}). With the two
components, we will be able to describe the effects and results of
programs \emph{declaratively}, \ie, without exposing program
implementations.

%possible to specify the behavior quantify over history components in Hoare triples, and
%ascribed to the corresponding programs.
%%
%Therefore, treating the (auxiliary) histories as a part of the program
%state, we can provide FCSL-style subjective (\ie, {thread-local})
%Hoare-style specifications to our example programs. By doing so, we
%manage to describe the programs' effects and results
%\emph{declaratively}, \ie, without exposing their actual
%implementations.


% Recent work on Hoare-style specification of concurrent data
% structures~\cite{Sergey-al:ESOP15} has demonstrated how to encode
% observable execution histories as a form of auxiliary
% state~\cite{Owicki-Gries:CACM76} and constrain their object-specific
% invariants. As the work~\cite{Sergey-al:ESOP15} focused solely on
% linearizable objects, it did not consider the objects, whose behavior,
% observed by a thread, inherently depends on the effects of the
% thread's \emph{environment}. We will now show how the very same
% specification approach can be employed to give concurrent Hoare-style
% specification to our non-linearizable examples.

% The typical pattern in specifying concurrent programs with auxiliary
% histories is to take \emph{subjective}~\cite{LeyWild-Nanevski:POPL13}
% (\ie, thread-local) point of view and consider the object-specific
% history $\hist$ as a disjoint union $\hist = \hists \hunion \histo$,
% where $\hists$ is a part of the history contributed by the thread
% being executed (dubbed \emph{self}-history, white parts of
% histories~\eqref{tag:hist1} and~\eqref{tag:hist2}) and $\histo$ is a
% cumulative contribution of its environment (\emph{other}-history, gray
% parts of histories~\eqref{tag:hist1} and~\eqref{tag:hist2}). The
% self/other dichotomy allows to make the resoning independed of the
% shape of thread-forking pattern, avoid using explicit thread IDs in
% the specifications, and also provides a uniform way to restrict the
% interference when reasoning about the clients.

% We can now specify the exchange procedure 


% We now give the simplified specifications to \code{exchange} and
% \code{flip2}, describing their contributions to the corresponding
% abstract histories

 

A semi-formal and partial spec of \code{exchange} looks as follows,
with the white/gray parts denoting \emph{self}/\emph{other}
contributions to history, from the point of view of the thread being
specified (we postpone the full presentation until
Section~\ref{sec:exchanger}):
%
% \an{Remove the twin $(w, v)$ from the equation below.}
%
{\small{
\[
\tag{\arabic{tags}}\refstepcounter{tags}\label{tag:exsimpl} 
{\small{
\begin{array}{c}
\specK{\{\hist_{\cal E} = [\graybox{\ldots}] \}}\\[2pt]
\esc{exchange}\ v\\[2pt]
  \spec{\!\!
  \begin{array}{c}
    \mathsf{if}\ \res\ \mathsf{is}\ \mathsf{Some}\ w\ \mathsf{then}~\\[1pt]
    \hist_{\cal E} = [\graybox{\ldots}, (v, w), \graybox{\ldots}]
    ~\mathsf{else}\ \hist_{\cal E} = [\graybox{\ldots}]
  \end{array}
  \!\!}
\end{array}
}}
\]
}}
%
\hspace{-5pt}
%
The ellipsis ($\ldots$) stands for an existentially-quantified chunk
of the history.
%
%
%By employing the coloring scheme to indicate the \emph{disjoint}
%contributions made by this thread (\emph{self}) and by its environment
%(\emph{other}), we avoid using thread IDs in the spec, making it
%agnostic to the shape of the concurrent context with dynamic forking.
%
%
%
The spec~\eqref{tag:exsimpl} says that a successful exchange adds an
entry $(v, w)$ to the \emph{self}-history (hence, the entry is
white). In the case of failed exchange, no entry is added. In the
complete and formal specification in Section~\ref{sec:exchanger}, we
will have to add a timing aspect, and say that the new entry appears
\emph{after} all the history entries from the precondition. We will
also have to say that no entries are removed from the \emph{other}
history (\ie, the exchanger cannot erase the behavior of other
threads), but we elide those details here.


%the call to \code{exchange} result
%in the case of succesff $\res$ of calling \code{exchange} to the
%self-contribution made to the final history $\hist_{\cal E}$ by this
%thread. The spec also \emph{explicitly} mentions the environment's
%contribution $\graybox{$(w, v)$}$ in the successful branch of the
%postcondition and accounts for possible effects of the interference by
%adding trailing $\graybox{\ldots}$ to $\hist_{\cal E}$.

%
%The spec~\eqref{tag:exsimpl} relates the result $\res$ of calling
%\code{exchange} to the self-contribution made to the final history
%$\hist_{\cal E}$ by this thread. The spec also \emph{explicitly} mentions the
%environment's contribution $\graybox{$(w, v)$}$ in the successful
%branch of the postcondition and accounts for possible effects of the
%interference by adding trailing $\graybox{\ldots}$ to $\hist_{\cal E}$.
%%
%For simplicity of presentation, we omit the imposed invariant on the
%history $\hist_{\cal E}$ from the spec, and consider equal histories
%$\hist_{\cal E}$ that are isomorphic up to permutations of subsequent
%``twin'' contributions $(v, w), (w, v)$.\footnote{There will be no
%  such ambiguity in a more rigorous encoding of history entries via
%  timestamps, which we present in Section~\ref{sec:exchanger}.}
%

The spec of \code{flip2} is defined with respect to history
$\hist_{\cal F}$:
%
\[
\tag{\arabic{tags}}\refstepcounter{tags}\label{tag:flipsimpl} 
{\small{
\begin{array}{c}
\specK{\{\hist_{\cal F} = [\graybox{\ldots}] \}}\\[2pt]
\esc{flip2}\ x\\[2pt]
  \spec{\!\!
  \begin{array}{c}
   \exists a~b, \hist_{\cal F} = [\graybox{\ldots}, a, \graybox{\ldots}, b,
    \graybox{\ldots}], \res = \bar{a} + \bar{b}
  \end{array}
  \!\!}
\end{array}
}}
\]
%
It says that the return value $\res$ is equal to the sum of binary
complements $\bar{a} + \bar{b}$ for the thread's two separate
\emph{self}-contributions to the history. Due to the effects of the
interference, the history entries $a$ and $b$ may be separated in the
overall history by the contributions of the environment, as indicated
by \graybox{\ldots} between them.

\subsection{Using Subjective Specifications in the Client Code}
\label{sec:clients}

The immediate benefit of using Hoare logic is that one can easily
reason about programs whose components use different object
invariants, whereas there is not much one can say about programs whose
components require different correctness conditions.
%
For example, Figure~\ref{fig:verif1} shows a proof sketch for a toy
program that uses both \code{exchange} and \code{flip2}.  As each of
these methods requires its own auxiliary history variable
($\hist_{\cal E}$ for the exchanger, and $\hist_{\cal F}$ for
\code{flip2}), the combined program uses both, but the proof simply
ignores those histories that are not relevant for any specific method
(\ie, we can ``frame'' the specs~\eqref{tag:exsimpl}
and~\eqref{tag:flipsimpl} wrt.~the histories of the objects that they
do not depend upon).

%
%To give an example of compositional reasoning with our Hoare-style
%specs, we focus on verification of a concurrent program, using both
%\code{exchange} and \code{flip2}. The program code and the proof
%outline are presented in Figure~\ref{fig:verif1}.

The program first forks two instances of \code{flip2}, storing the
results in $r_1$ and $r_2$ (line~4). Next, two new threads are forked,
trying to exchange $r_1$ and $r_2$ (line~8). The conditional (line~12)
checks if the exchange was successful, and if so, assigns the sum of
exchanged values to $t$ (line~14); otherwise $t$ gets assigned 2. We
want to prove via the specs~\eqref{tag:exsimpl}
and~\eqref{tag:flipsimpl}, that in the absence of external
interference on the \code{flip2}'s pointer $x$ and the exchanger, the
outcome is always $t = 2$.

\paragraph{Explaining the verification}

In addition to the absence of external interference, we assume that
the initial value of $x$ is $0$, and the initial \emph{self}-histories
for both \code{flip2} and \code{exchange} are empty (line~1).
%
% \an{There was a footnote here about framing, but I think it's too
%   detailed for being here. Moreover, that framing may be confused with
%   the framing of concurroids, which we called injection in previous
%   work. Let's see if shephard calls for an explanation.}
%
%\footnote{These restrictions
%  can be lifted via the FCSL framing
%  rule~\cite{Nanevski-al:ESOP14}.}
%
%Therefore, both initial histories $\hist_{\cal F}$ and $\hist_{\cal F}$ are empty
%(line~1).
%
%
Once the \code{flip2} threads are forked, we employ
spec~\eqref{tag:flipsimpl} for each of them, simply ignoring (i.e.,
framing out) $\hist_{\cal E}$, as this history variable does not apply
to them \code{flip2}. Upon finishing, the postconditions of
\code{flip2} in line~4 capture the relationship between the
contributions to the history $\hist_{\cal F}$ and the results $r_1$
and $r_2$ of the two calls.

Both postconditions in line~4 talk about the very same history
$\hist_{\cal F}$, just using different colors to express that the
contributions of the two threads are \emph{disjoint}: $a$ and $b$
being white in the left thread, implies that $a$ and $b$ are history
entries added by the left thread. Thus, they \emph{must} be gray in
the right thread, as they cannot overlap with the entries contributed
by the right thread. The right thread cannot explicitly specify in its
postcondition that $a$ and $b$ are gray, since the right thread is
unaware of the specific contributions of the left thread.
%
% \is{This is redundant}
% But, such inference will be enabled by the FCSL rule for
% parallel composition, as we explain in Section~\ref{sec:background}.

Dually, $c$ and $d$ being white in the right thread in line~4, implies
that they must be gray on the left. Thus, overall, in line~5, we know
that $\hist_{\cal F}$ contains all four entries in some permutation,
and in the absence of intereference, it contains no other entries but
these four. From the object invariant on $\hist_{\cal F}$ it then
follows that the entries are some permutation of $[1,0,1,0]$, which
makes their sum total $r_1 + r_2 = 2$. 
%
% \an{Make a reference to this object invariant, once it is added in the
%   previous section.}

%threads, the subjective assertions in line~4 of the proof outline
%constrain, in fact, the \emph{very same} history $\hist_{\cal F}$,
%specifying its disjoint subparts.
%%
%Upon joining the first pair of threads, we reconcile two thread-local
%views to $\hist_{\cal F}$ (line~5), inferring that $\hist_{\cal F}$ is a permutation
%of $(0, 1, 0, 1)$, where each of the four components correspond to one
%of the contributions made in parallel.
%%
%The assumption about absence of external interference is crucial, as
%it makes it possible to ensure that the \emph{entire} history
%$\hist_{\cal F}$ consists only of the two thread's contributions.
%%
%Therefore, the sum $r_1 + r_2$ is the sum of complements to the
%history entries, \ie,~$2$.

Similarly, we ignore $\hist_{\cal F}$ while reasoning about calls to
\code{exchange} via spec~\eqref{tag:exsimpl} (lines~7 and 9). As
before, we know that the entry $(r_1, v_1)$, which is white in the
left postcondition in line~9, must be gray on the right, and dually
for $(r_2, v_2)$. In total, the history $\hist_{\cal E}$ must contain
both of the entries, but, by the invariant, it must also contain their
twins. In the absence of any other interference, it therefore must be
that $(r_1, v_1)$ is a twin for $(r_2, v_2)$, \ie, $r_1 = v_2$ and
$r_2 = v_1$, as line~11 expresses for the case of a succesful
exchange.
%
% \an{It seems to me that the figure should say this by using
%   $\hist_{\cal E} = perm ((r_1, v_1), (r_2, v_2)) = perm ((v_1, r_1),
%   (v_2, r_2))$, rather than the current $\hist_{\cal E} = perm ((r_2,
%   v_2), (v_2, r_2)) = perm ((r_1, v_1), (v_1, r_1))$.}  
%
The rest of the proof is then trivial.

The sketch relied on several important aspects of program verification
in FCSL: \emph{(i)} the invariants constraining $\hist_{\cal F}$ and
$\hist_{\cal E}$ were preserved by the methods, \emph{(ii)} upon
joining the threads, we can rely on the disjointness of history
contributions of the two threads, in order to combine the thread-local
views into a specification of the parent thread, and, \emph{(iii)} we
could guarantee the absence of the external interference.

The aspect \emph{(i)} is a significant component of what it means to
specify and verify a concurrent object. As we will show in
Sections~\ref{sec:exchanger} and~\ref{sec:counting}, defining a
sufficiently strong object invariant, and then proving that it is
indeed an invariant, \ie, that it is preserved by the implementation
of the program, is a major part of the verification challenge.
%
We will explain FCSL rules for \emph{parallel composition} and
\emph{hiding} in Section~\ref{sec:background}, justifying the
reasoning principles~\emph{(ii)} and \emph{(iii)}.

\subsection{Specifying Non-linearizable Objects in Three Steps}
\label{sec:three-steps-reas}

As shown by Sections~\ref{sec:hist}--\ref{sec:clients}, our method for
specifying and verifying non-linearizable concurrent objects and their
clients boils down to the following three systematic steps.

\paragraph{Step 1 (\S\ref{sec:hist}):} 

\emph{Define object-specific auxiliary state and its invariants. The
  auxiliary state will typically include a specific notion of abstract
  histories, recording whatever behavior is perceived as essential by
  the implementor of the object}.
%
%The concurrent object implementor should first define the shape and
%key invariants of the object's auxiliary state, in particular, of its
%histories.
%
To account for the variety of object-specific correctness conditions,
we do not fix a specific shape for the histories. We do not restrict
them to always record pairs of numbers (as in the exchanger), or
record single numbers (as in \code{flip2}). The only requirement that
we impose on auxiliary state in general, and on histories in
particular, is that the chosen type of auxiliary state is an instance
of the PCM algebraic structure~\cite{Sergey-al:ESOP15}, thus providing
an abstract, and user-defined, notion of \emph{disjointness} between
\emph{self}/\emph{other} contributions.

\paragraph{Step 2 (\S\ref{sec:hoare}):} 

\emph{Formulate Hoare-style specifications, para-metrized by
  interference, and verify them}.
%
This step provides a suitable ``interface'' for the methods of the
concurrent object, which the clients use to reason, without
knowing the details of the object and method implementations.
%
Naturally, the interface can refer to the auxiliary state and
histories defined in the previous step.
%
When dealing with non-linearizable objects in FCSL, it is customary to
formulate the spec in a subjective way (\ie, using
\emph{self}/\emph{other}, dually white/gray division between history
entries) so that the specification has a way to refer to the effects
of the interfering calls to the same object. 
%
The amount of interference can be later instantiated with more
specific information, once we know more about the context of
concurrent threads in which the specified program is being run.

\paragraph{Step 3 (\S\ref{sec:clients}):} 

\emph{Restrict the interference when using object specs for
  verification of clients}.
%
Eventually, thread-local knowledge about effects of individual clients
of one and the same object, should be combined into a cumulative
knowledge about the effect of the composition.
%
To measure this effect, one usually considers the object in a
\emph{quiescent} (interference-free) moment~\cite{Rinard:RACES}.
%
%\ab{Not clear what quiescent means. Also provide a citation?} 
%
To model quiescent situations, FCSL provides a program-level
constructor for \emph{hiding}. In particular, $\esc{hide}\ e$ executes
$e$, but statically prevents other threads from interfering with $e$,
by making $e$'s auxiliary history invisible. Program $e$'s
\emph{other} contribution is fixed to be empty, thus modeling
quiescence.

\section{Verifying the Exchanger Implementation}
\label{sec:exchanger}

We now proceed with more rigorous development of the invariants and
specification for the exchanger data structure, necessary to verify
its real-world implementation~\cite{ExchangerClass}, which was so far
elided from the overview of the approach.

% This example previously motivated the introduction of the correctness
% condition CAL~\cite{Hemed-Rinetzky:PODC14}. The exchanger is not
% linearizable to a set of sequential histories, as its behavior
% crucially depends on interaction with interfering threads.
% %
% CAL works around this problem by employing new
% \emph{concurrency-aware} (CA) histories as a specification set.


\newcommand{\Unmatched}{{\mathsf{U}}}
\newcommand{\Matched}[1]{{\mathsf{M}\ #1}}
\newcommand{\Retired}{{\mathsf{R}}}


The exchanger implementation is presented in ML-style pseudo-code in
Figure~\ref{fig:exchanger}. It takes a value $v\,{:}\,A$ and creates
an \emph{offer} from it (line 2). An offer is a pointer $p$ to two
consecutive locations in the heap.\footnote{In our mechanization, we
  simplify a bit by making $p$ point to a pair instead.}
%\footnote{In our mechanization, we
%  use a pointer to a pair, but for presentation here adopt a more
%  common style of consecutive heap locations.} 
%
The first location stores $v$, and the second is a ``hole'' which the
interfering thread tries to fill with a matching value. The hole is
drawn from the type
$\esc{hole}\,{=}\,\Unmatched\,{\mid}\,\Retired\,{\mid}\,\Matched
w$. Constructor $\Unmatched$ signals that the offer is unmatched;
$\Retired$ that the exchanger retired (\ie, withdrew) the offer, and
does not expect any matches on it; and $\Matched w$ that the offer has
been matched with a value $w$.

The global pointer $g$ stores the latest offer proposed for
matching. The exchanger proposes $p$ for matching by making $g$ point
to $p$ via the atomic compare-and-set instruction \code{CAS} (line
3). We assume that \code{CAS} returns the value read, which can be
used to determine if it failed or succeeded. If \code{CAS} succeeds,
exchanger waits a bit, then checks if the offer has been matched by
some $w$ (lines 6, 7). If so, $\esc{Some}\ w$ is returned (line
7). Otherwise, the offer is retired by storing $\Retired$ into its
hole (line 6). Retired offers remain allocated (thus, exchanger has a
memory leak) in order to avoid the ABA problem, as usual in many
concurrent structures~\cite{Herlihy-Shavit:08,Treiber:TR}.
%
If the exchanger fails to link $p$ into $g$ in line 3, it deallocates
the offer $p$ (line 10), and instead tries to match the offer $cur$
that is current in $g$. If no offer is current, perhaps because
another thread already matched the offer that made the \code{CAS} in
line 3 fail, the exchanger returns $\mathsf{None}$ (line
12). Otherwise, the exchanger tries to make a match, by changing the
hole of $cur$ into $\Matched v$ (line 14). If successful (line 16), it
reads the value $w$ stored in $cur$ that was initially proposed for
matching, and returns it. In any case, it unlinks $cur$ from $g$ (line
15) to make space for other offers.

{
%\setlength{\belowcaptionskip}{-10pt} 
\begin{figure}
\centering
\[
{\small{
\begin{array}{rl}
 \Num{1} & \esc{exchange}~(v : A) : \esc{option}~A~=~\{ 
\\ 
 \Num{2} & ~~~~ p \Asgn \esc{alloc}~(v, \Unmatched);\\
 \Num{3} & ~~~~ b \Asgn \esc{CAS}~(g, \esc{null}, p);\\
 \Num{4} & ~~~~ \kw{if}~~b~\esc{==}~\esc{null}~~\kw{then}\\
 \Num{5} & ~~~~ ~~~~ \esc{sleep}~(50);\\
 \Num{6} & ~~~~ ~~~~ x \Asgn \esc{CAS}~(p\esc{+}1, \Unmatched, \Retired);\\
 \Num{7} & ~~~~ ~~~~ \kw{if}~~x~~\kw{is}~~\Matched w~~\kw{then}~~\kw{return}~~(\esc{Some}~w)\\
 \Num{8} & ~~~~ ~~~~ \kw{else}~~\kw{return}~~\esc{None}\\
 \Num{9} & ~~~~ \kw{else}\\
\Num{10} & ~~~~ ~~~~ \esc{dealloc}~p;\\
\Num{11} & ~~~~ ~~~~ cur \Asgn \esc{read}~g;\\
\Num{12} & ~~~~ ~~~~ \kw{if}~~cur~\esc{==}~\esc{null}~~\kw{then}~~\kw{return}~{\esc{None}}\\
\Num{13} & ~~~~ ~~~~ \kw{else}\\
\Num{14} & ~~~~ ~~~~ ~~~~ x \Asgn \esc{CAS}~(cur\esc{+}1, \Unmatched, \Matched v);\\
\Num{15} & ~~~~ ~~~~ ~~~~ \esc{CAS}~(g, cur, \esc{null});\\
\Num{16} & ~~~~ ~~~~ ~~~~ \kw{if}~~x~~\kw{is}~~\Unmatched~~\kw{then}~~w\Asgn \esc{read}~cur;\kw{return}~(\esc{Some}\ w)\\
\Num{17} & ~~~~ ~~~~ ~~~~ \kw{else}~~\kw{return}~\esc{None}\}
\end{array}
}}
\]
%\vspace{-5pt}  
\caption{Elimination-based exchanger procedure.}
\label{fig:exchanger}
%\vspace{-10pt}   
\end{figure} 
}




\subsection{Step 1: Defining Auxiliary State and Invariants}

%
%The subjective division suffices to specify the exchanger in Hoare
%logic, and will be essential for client reasoning in
%Section~\ref{sec:cal}.
%
%In Section~\ref{sec:counting}, we show how the approach directly
%scales to naturally specify programs that have heretofore only been
%addressed using quiescent and quantitative quiescent consistency.

To formally specify the exchanger, we decorate it with auxiliary
state. 
%
In addition to histories, necessary for specifying the observable
behavior, the auxiliary state is used for capturing the coherence
constraints of the actual implementation, \eg, with respect to memory
allocation and management of outstanding offers.
%
The state is subjective as described in Section~\ref{sec:overview}: it
keeps thread-local auxiliary variables that name the thread's private
state (\emph{self}), but also the private state of all other threads
combined (\emph{other}).
%
% The subjective division will be essential for client reasoning in
% Section~\ref{sec:cal}.

The subjective state of the exchanger for each thread in this example
consists of three groups of two components: (1) thread-private heap
$\heaps$ of the thread, and of the environment $\heapo$, (2) a set of
outstanding offers $\perms$ created by the thread, and by the
environment $\permo$, and (3) a time-stamped history of values
$\hists$ that the thread exchanged so far, and dually $\histo$ for the
environment. In Section~\ref{sec:overview}, we illustrated
subjectivity by means of histories, white we used white and gray
entries, respectively, to describe what here we name $\hists$ and
$\histo$, respectively. Now we see that the dichotomy extends beyond
histories, and this example requires the dichotomy applied to heaps,
and to sets of offers as well. In addition to \emph{self}/\emph{other}
components of heaps, permissions and histories, we also need shared
(aka.~\emph{joint}) state consisting of two components: a heap
$\heapj$ of storing the offers that have been made, and a map
$\pending$ of offers that have been matched, but not yet collected by
the thread that made them.

Heaps, sets and histories are all PCMs under the operation of disjoint
union, with empty heap/set/history as a unit. We overload the notation
and write $x\,{\mapsto}\,v$ for a singleton heap with a pointer $x$
storing value $v$, and $t\,{\mapsto}\,a$ for a singleton
history. Similarly, we apply disjoint union $\hunion$ and subset
$\subseteq$, to all three types uniformly.

We next describe how the exchanger manipulates the above variables.
%for maps, and write $x \mapsto y$ for a singleton map, no matter
%its type. For example, in the case of time-stamped histories, $t
%\mapsto b$ denotes that at time-stamp $t$, some abstract behavior $b$
%took place. In the case of heaps, $x \mapsto v$ for a singleton heap
%with a pointer $x$ storing value $v$. We also overload the operator
%$\hunion$ for disjoint union, and apply it to histories, heaps, and
%set of offers, uniformly. Thus, each history or heap will be a
%disjoint union of a number of singleton entries, and we use $x
%\subseteq y$ to say that the map $x$ is a submap (or subheap, or
%subhistory) of $y$.
First, $\heapj$ is a heap that serves as the ``staging'' area for the
offers. It includes the global pointer $g$. Whenever a thread wants to
make an offer, it allocates a pointer $p$ in $\heaps$, and then tries
to move $p$ from $\heaps$ into $\heapj$, simultaneously linking $g$ to
$p$, via the \code{CAS} in line~3 of Figure~\ref{fig:exchanger}.

Second, $\perms$ and $\permo$ are sets of offers (hence, sets of
pointers) that determine offer ownership. A thread that has the offer
$p \in \perms$ is the one that created it, and thus has the
\emph{sole} right to retire $p$, or to collect the value that $p$ was
matched with. Upon collection or retirement, $p$ is removed from
$\perms$.

%To ensure that only one thread has such a right, one of the important
%state invariants of the exchanger is that $\perms$ and $\permo$ are
%disjoint sets.

Third, $\hists$ and $\histo$ are exchanger-specific histories, each
mapping a time-stamp (isomorphic to nats), to a pair of exchanged
values. A singleton history $t \mapsto (v, w)$ symbolizes that a
thread having this singleton as a subcomponent of $\hists$, has
exchanged $v$ for $w$ at time $t$.
%
%Histories have been used in FCSL before~\cite{Sergey-al:ESOP15}, but
%only on linearizable examples, where each time-stamp $t$ ``stored'' an
%atomic operation performed at time $t$. In the exchanger example,
%which is not linearizable, a time-stamp $t$ stores only a \emph{half}
%of an atomic operation, \ie, a pair $(v, w)$ signaling that a thread
%exchanged $v$ for $w$. 
%
As we describe below, the most important invariant of the exchanger is
that each such singleton is matched by a ``symmetric'' one to capture
that another thread has \emph{simultaneously} exchanged $w$ for
$v$. Classical linearizability cannot express this simultaneous
behavior, making the exchanger non-linearizable.

Fourth, $\pending$ is a map storing the offers that were matched, but
not yet acknowledged and collected. Thus,
$\mathsf{dom}\ \pending = \perms \hunion \permo$. A singleton entry in
$\pending$ has the form $p \mapsto (t, v, w)$ and denotes that offer
$p$, initially storing $v$, was matched at time $t$ with $w$. A
singleton entry is entered into $\pending$ when a thread on the one
end of matching, matches $v$ with $w$. Such a thread also places the
\emph{twin} entry $\twin{t} \mapsto (w, v)$, with inverted order of
$v$ and $w$, into its own private history $\hists$, where:
%
\[
\begin{array}{c}
%\vspace{-5pt}
\twin{t} = \left\{%
\begin{array}{ll}
t+1 & \mbox{if $t$ is odd}\\
t-1 & \mbox{if $t > 0$ and $t$ is even}
\end{array}\right.  
\end{array}
\]
%
For technical reasons, $0$ is not a valid time-stamp, and has no
distinct twin. The pending entry for $p$ resides in $\pending$ until
the thread that created the offer $p$ decides to ``collect'' it. It
removes $p$ from $\pending$, and simultaneously adds the entry $t
\mapsto (v, w)$ into its own $\hists$, thereby logically completing
the exchange. Since twin time-stamps are consecutive integers, a
history cannot contain entries \emph{between} twins.

Thus, two twin entries in the combined history including $\hists$,
$\histo$ and $\pending$, jointly represent a single exchange, as if it
occurred \emph{atomically}. 
%
For example, the entries $1 \mapsto (v_1, w_1)$ and
$2 \mapsto (w_1, v_1)$ will encode the end-points of the first
exchange; the entries $3 \mapsto (v_2, w_2)$ and
$4 \mapsto (w_2, v_2)$ will encode the end-points of the second
exchange, etc., the entries at timestamps $t$ and $t+1$, for odd $t$,
will encode the end-points of the $\frac{t+1}{2}$-th exchange.
%
\emph{Concurrency-aware}
histories~\cite{Hemed-Rinetzky:PODC14} capture this by making the ends
of an exchange occur as simultaneous events. We capture it via twin
time-stamps. More formally, consider
$\hist = \hists \hunion \histo \hunion \mygather{\pending}$. Then, the
exchanger's main invariant is that $\hist$ always contains matching
twin entries:
%
%\vspace{-5pt}
\[
%\vspace{-5pt}
\tag{\arabic{tags}}\refstepcounter{tags}\label{tag:exchanging} 
t \mapsto (v, w) \subseteq \hist \iff \twin t \mapsto (w, v) \subseteq \hist
\]
%
Here $\mygather{\pending}$ is the collection of all the entries in
$\pending$. That is, $\mygather{\emptyset} = \emptyset$, and
$\mygather {p \mapsto (t, v, w) \hunion \pending'} = t \mapsto (v, w)
\hunion \mygather{\pending'}$.

In our implementation, we prove that atomic actions, such as
\code{CAS}, preserve the invariant, therefore, the whole program,
being just a composition of actions, doesn't violate~it.

%One consequence of this property is that the history $\hist$ always
%has an even number of entries. Thus, if $t$ is the smallest unused
%time-stamp in $\hist$ (not counting $0$), then $t$ must be odd, and
%hence $t < \twin t = t + 1$.

\subsection{Step 2: Hoare-style Specification of Exchanger}

We can now give the desired formal Hoare-style spec.
%
\[
\tag{\arabic{tags}}\refstepcounter{tags}\label{tag:exchangespec} 
{\small{
\begin{array}{c}
\specK{\{\heaps = \emptyset, \perms = \emptyset, \hists = \emptyset, \gist \subseteq \histo \hunion \mygather{\pending}\}}\\
\esc{exchange}\ v\\
  \spec{\!\!
  \begin{array}{c}
\heaps = \emptyset, \perms = \emptyset, \gist \subseteq
  \histo \hunion \mygather{\pending}, \hbox{}\\[1pt]
\mathsf{if}\ \res\ \mathsf{is}\ \mathsf{Some}\ w\ \mathsf{then}~\\[1pt]
\exists t\ldot \hists = t \mapsto (v, w), \mathsf{last} (\gist) < t, \twin{t}
~\mathsf{else}\ \hists = \emptyset 
  \end{array}
  \!\!}
\end{array}
}}
\]
%
The precondition says that the exchanger starts with the empty private
heap $\heaps$, set of offers $\perms$ and history $\hists$; hence by
framing, it can start with any value for these
components.\footnote{Framing in FCSL is similar to that of separation
  logic, allowing extensions to the initial state that remain
  invariant by program execution. In FCSL, however, framing applies to
  any PCM-valued state component (\eg, heaps, histories, \etc.),
  whereas in separation logic, it applies just to heaps.} The logical
variable $\gist$ names the initial history of all threads,
$\histo \hunion \mygather{\pending}$, which may grow during the call,
thus, we use subset instead of equality to make the precondition
stable under other threads adding
 entries to $\histo$ or $\pending$.

In the postcondition, the self heap $\heaps$ and the set of offers
$\perms$ didn't change. Hence, if $\mathtt{exchange}$ made an offer
during its execution, it also collected or retired it by the end.
%
The history $\gist$ is still a subset of the ending value for $\histo
\hunion \mygather{\pending}$, signifying that the environment history
only grows by interference. We will make a crucial use of this part of
the spec when verifying a client of the exchanger in
Section~\ref{sec:cal}.

If the exchange fails (\ie, $\mathsf{res}$ is $\mathsf{None}$), then
$\hists$ remains empty.  If it succeeds (either in line 7 or line 16
in Figure~\ref{fig:exchanger}), \ie, if the result $\mathsf{res}$ is
$\mathsf{Some}\ w$, then there exists a time-stamp $t$, such that
self-history $\hists$ contains the entry $t \mapsto (v, w)$,
symbolizing that $v$ and $w$ were exchanged at time $t$.

Importantly, the postcondition implies, by
invariant~(\ref{tag:exchanging}), that in the success case, the twin
entry $\twin t \mapsto (w, v)$ must belong to $\histo \hunion
\mygather{\pending}$, \ie, \emph{another} thread matched the exchange
(this was made explicit by the spec~\eqref{tag:exsimpl}).
Moreover, the exchange occurred \emph{after} the call to
$\mathsf{exchange}$: whichever $\gist$ we chose in the pre-state, both
$t$ and $\twin t$ are larger than the last time-stamp in $\gist$.

The proof outline for the exchanger is available in
Appendix~\ref{app:exch}.
%
In Section~\ref{sec:cal}, after introducing necessary FCSL background,
we will illustrate \textbf{\emph{Step 3}} of our method and show how
to employ the subjective Hoare spec~\eqref{tag:exchangespec} for
modular verification of a concurrent client.

% \is{Where is the discussed ``general pattern'', which we wanted to
%   summarize around this moment of the paper (Anindya also suggests it
%   in his very first comment)? I believe, this is a good place to put
%   it.}
% \an{It's in the second paragraph of Section 2.}

