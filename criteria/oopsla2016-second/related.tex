
\section{Related Work}
\label{sec:related}

% Existing methods for specifying the behavior of concurrent data
% structures and programs are either history-based or state-based.
% %
% The former ones describe the behavior of an object by describing and
% characterizing call/return strings that can be obtained by invoking
% the methods concurrently by several threads. 
% %
% The later ones define the object's behavior by posing requirements to
% a state, in which object's methods are safe to run, and stating
% assertions over a state, resulting from the object's method call. The
% second group of approaches employs program logics as a way to specify
% the state of interest as well as a mechanism to compose the program
% specifications.

% This paper contributes to the goal of unifying the history- and
% state-based views to specification and verification of concurrent
% objects.

% In this section we describe related approaches for reasoning about
% concurrency that have motivated our work.

\paragraph{Linearizability and history-based criteria.}

% In the past 25 years, linearizability has been widely applied to
% capture the behavior of concurrent objects with intuitive sequential
% specifications, and even suitable for automatic synthesis of some
% concurrent objects~\cite{Vechev-Yahav:PLDI08}. Thanks to the
% compositional proof method, based on \emph{linearization points},
% proofs of linearizability in most of the cases are amenable for
% practical computer-aided
% verification~\cite{Burckhardt-al:PLDI10,Derrick-al:TOPLAS11,Vafeiadis:CAV10,Amit-al:CAV07,Shacham-al:OOPSLA11,Dragoi-al:CAV13}.

The need for correctness criteria alternative to
linearizability~\cite{Herlihy-Wing:TOPLAS90}, which is more relaxed
yet compositional, was recognized in the work on counting
networks~\cite{Aspnes-al:JACM94}.
%
The suggested notion of quiescent
consistency~\cite{Shavit-Zemah:TOPLAS96} required the operations
separated by a quiescent state to take effect in their logical order.
%
%
A more refined correctness condition, \emph{quasi-linearizability},
implementing a relaxed version of linearizability with an upper bound
on nondeterminism, was proposed by Afek~\etal~\cite{Afek-al:OPODIS10},
allowing them to obtain the quantitative boundaries similar to what we
proved in Section~\ref{sec:qqc-client}.
%
The idea of relaxed linearizability was later used in the work on
\emph{quantitative relaxation} (QR)~\cite{Henzinger-al:POPL13} for
designing scalable concurrent data structures by changing the
specification set of sequential histories.
%
Most recently, \emph{quantitative quiescent consistency} has been
proposed as another criterion incorporating the possibility to reason
about effects of bounded thread
interference~\cite{Jagadeesan-Riely:ICALP14}.
%
It is worth noticing that some of these correctness criteria are
incomparable (\eg, QC and QR~\cite{Henzinger-al:POPL13}, QL and
QQC~\cite{Jagadeesan-Riely:ICALP14}) hence, for a particular
concurrent object, choosing one or another criterion should be
justified by the needs of the object's client. Therefore, a suitable
correctness condition is essentially ``\emph{in the eye of the
  beholder}'', as is typical in programming, when designing libraries
and abstract data structures, and the logic-based approach we advocate
provides precisely this flexibility in choosing desired specs.

% However, how suitable is one or another relaxed correctness
% condition %from the listed ones
% for reasoning about safety properties of a particular client program
% is still largely an open question.

\paragraph{Hoare-style specifications of concurrent objects.}
\label{sec:related-logic-based}

% Program logics for concurrency allow one to capture in the
% specification precisely those bits of information about the program's
% state, which are relevant for the program's clients, making it
% possible to abstract over the irrelevant details of the
% implementation~\cite{DinsdaleYoung-al:ECOOP10}.

Hoare-style program logics were used with great success to verify a
number of concurrent data structures and algorithms, which are much
more natural to specify in terms of observable state modifications,
rather than via call/return histories. The examples of such objects
and programs include
barriers~\cite{Dodds-al:POPL11,Hobor-Gherghina:ESOP11}, concurrent
indices~\cite{ArrozPincho-al:OOPSLA11}, flat
combiner~\cite{Turon-al:ICFP13,Sergey-al:ESOP15}, event
handlers~\cite{Svendsen-Birkedal:ESOP14}, shared graph
manipulations~\cite{Raad-al:ESOP15,Sergey-al:PLDI15}, as well as their
multiple client programs.
%
The observation about a possibility of using program logics as a
correctness criterion, alternative to linearizability, has been made
in some of the prior
works~\cite{Jacobs-Piessens:POPL11,ArrozPincho-al:OOPSLA11,Svendsen-al:ESOP13}.
%
Their criticism of linearizability addressed its inability to capture
the state-based properties, such as dynamic memory
ownership~\cite{Jacobs-Piessens:POPL11}---something that
linearizability indeed cannot tackle, unless it's
extended~\cite{Gotsman-Yang:CONCUR12}.
%
However, we are not aware of any prior attempts to capture CAL, QC and
QQC-like properties of concurrent executions by means of \emph{one and
  the same} program logic and employ them in client-side
reasoning.

%
% Verification in such logics is done structurally, \ie, by
% systematically applying syntax-directed inference rules, until the
% spec is proved. Several mechanized tools for logic-based
% concurrent reasoning have been
% released~\cite{Sergey-al:PLDI15,Appel-al:BOOK14}.

Several logics for proving linearizability or, equivalently,
observational refinement~\cite{Filipovic-al:TCS10,Turon-al:POPL13},
have been proposed
recently~\cite{Turon-al:ICFP13,Liang-Feng:PLDI13,Vafeiadis:PhD}, all
employing variations of the idea of using \emph{specifications as
  resources}, and identifying (possibly, non-fixed or non-local)
linearization points, at which such specification should be ``run''.
%
In these logics, after establishing linearizability of an operation,
one must still devise its Hoare-style spec, such that the spec is useful for
the clients.
% %
% To avoid this detour, a number of other logic-based approaches have
% suggested to assign Hoare-style specifications \emph{directly} to
% fine-grained object
% implementations~\cite{Svendsen-al:ESOP13,Svendsen-Birkedal:ESOP14,ArrozPincho-al:ECOOP14}
% and reason out of these specs.

Similarly to the way linearizability allows one to replace a
concurrent operation by an atomic one, several logics have implemented
the notion of \emph{logical atomicity}, allowing the clients of a data
structure to implement application-specific synchronization on top of
the data structure operations.
%
Logical atomicity can be implemented either by parametrizing specs
with client-specific auxiliary
code~\cite{Jacobs-Piessens:POPL11,Svendsen-al:ESOP13,Svendsen-Birkedal:ESOP14,Jung-al:POPL15}
or by engineering dedicated rules relying on the simulation between
the actual implementation and the ``atomic''
one~\cite{ArrozPincho-al:ECOOP14}.
%
% Both approaches to logical atomicity require one to identify precisely
% a \emph{synchronization point}~\cite{Svendsen-al:ESOP13} within the
% structure being verified, which makes it non-trivial to apply them for
% specifying non-linearizable objects, especially when conducting
% quantitative logic-based reasoning, which we demonstrated in
% Section~\ref{sec:qqc-client}.

Instead of trying to extend the existing approaches for logical
atomicity to non-linearizable objects (for which the notion of
atomicity is not intuitive), we relied on a general mechanism of
auxiliary state, provided by FCSL~\cite{Nanevski-al:ESOP14}. 
%
Specifically, we adopted the idea of histories as auxiliary
state~\cite{Sergey-al:ESOP15}, which, however, was previously explored
in the context of FCSL only for specifying linearizable structures.
%
% Similarly to logical atomicity, the histories-as-state specification
% approach makes fine-grained concurrent programs look like atomic ones,
% since due to compositionality, the specification clients only see the
% history entries.
%
% 
We introduced enhanced notation for referring directly to histories
(\eg, $\hists$, $\histo$), although FCSL's initial logical
infrastructure and inference rules remained unchanged.

% Recently, attempts were made to unify the common idioms occurring in a
% number of concurrency logics in a generic framework of
% \emph{Views}~\cite{DinsdaleYoung-al:POPL13}.
% %
% However, that result is orthogonal to our findings, as \emph{Views}
% are a framework for proving logics sound, not to prove programs, and
% this paper, we focused on using a particular logic (FCSL) for specifying a
% new class of concurrent data structures.
The purpose of \emph{Views}~\cite{DinsdaleYoung-al:POPL13} is to provide a PCM-based semantic framework for proving soundness of logics and type systems, and thus to unify common idioms occurring in several recent concurrency logics. However, the framework has not used PCMs for auxiliary state and/or specification of user programs. \ab{Please check.}


In this work, we do not argue that FCSL is the only logic capable of
encoding custom correctness conditions and their combinations, though,
we are not aware of any other work exploring a similar possibility.
%
However, we believe that FCSL's explicit \emph{other}
subjective state component provides the most straightforward way to do
so.
%
The logics like CAP~\cite{DinsdaleYoung-al:ECOOP10} and
TaDA~\cite{ArrozPincho-al:ECOOP14}, from our experience and personal
communication with their authors, may be capable of implementing our
approach at the expense of engineering a much more complicated
structure of capabilities to encode histories and their invariants,
and ``snapshot'' interference of an environment.
%
Other logics incorporating the generic PCM
structure~\cite{Raad-al:ESOP15,Jung-al:POPL15,Jung-al:ICFP16,Turon-al:OOPSLA14}
might be able to implement our approach, although none of these logics
provide an FCSL-style rule for hiding~\eqref{eq:ehide} as a uniform
mechanism to express explicit quiescence.

%\todo{State that in our cases histories are per-object.}

Concurrently with this work, Hemed~\etal developed a (not yet
mechanized) verification technique for CAL~\cite{Hemed-al:DISC15},
which they applied to the exchanger and the elimination
stack. Similarly to our proposal, they specify CAL-objects via Hoare
logic, but using one global auxiliary history, rather than subjective
auxiliary state. 
%
This tailors their system specifically to CAL (without a possibility
to incorporate reasoning about other, non CA-linearizable, concurrent
structures), and to programs with a \emph{fixed} number of threads. In
contrast, FCSL supports dynamic thread creation, and is capable of
uniformly expressing and mechanically verifying several different
criteria, with CAL merely a special case, obtained by a special choice
of PCM. Moreover, in FCSL the criteria combine, as illustrated in
Section~\ref{sec:cal}, where we combined quiescence with CAL via
hiding. Hiding is crucial for verifying clients with explicit
concurrency, but is currently unsupported by Hemed~\etal's method.
%
% Related is the property of our histories that no event may appear
% between two twin timestamps. We used this property to verify the
% client in Section~\ref{sec:cal}, but this does not seem to hold for
% histories used by Hemed~\etal

%Concurrently with our work, Hemed~\etal developed a (not yet
%mechanized) verification technique for
%CAL~\cite{Hemed-al:DISC15}. Similarly to our proposal, they employed
%the idea of specifying CAL-objects via Hoare triples with auxiliary
%histories. Unlike our approach, their choice of abstractions is
%tailored for conducting proofs about CAL, and they don't establish a
%soundness result of their Hoare-style specs, with respect to CAL or
%any other semantics. In contrast, what we propose is using a uniform
%framework, proved sound from first principles~\cite{Sergey-al:PLDI15},
%capable of capturing the essence of multiple correctness conditions
%(including CAL) and verifying them mechanically. Our spec of the
%exchanger~\eqref{tag:exchangespec} can be easily generalized to tackle
%verification of the elimination stack (main example
%from~\cite{Hemed-al:DISC15}) by parametrizing it with a
%client-provided invariant on histories. Furthermore, by using FCSL as
%a logical foundation, we can also reason about quiescence (via hiding)
%and dynamic thread spawning (via the rule~\eqref{eq:parrule}). These
%components are crucial for verifying clients featuring explicit
%concurrency (Section~\ref{sec:cal}), and we currently don't see how to
%conduct such proofs out of the spec from~\cite{Hemed-al:DISC15}.


% \paragraph{Reasoning about linearizability in program logics}
% \label{sec:rel-linear}

% \begin{itemize}

% \item Logics for linearizability (viktor's PhD), LF, Turon, contextual refinement

% \item Hindsight paper by O'Hearn and company

% \item Observation from the HOCAP paper about the push2 method, that,
%   however, never were extended to any interesting structures, so we
%   did it.

% \item Abstract atomicity

% \end{itemize}

