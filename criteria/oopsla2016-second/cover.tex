\documentclass{article}
%\usepackage[2ndsubmission]{oopsla2016}
%\usepackage[final]{oopsla2016}
%\usepackage[final]{oopsla2016}

\usepackage{amssymb,amsmath,amsthm}
\usepackage{mathtools} 
\usepackage{latexsym}
\usepackage{graphicx}
\usepackage[usenames,dvipsnames]{color}
\usepackage{listings}
\usepackage{float}
\usepackage{multirow}
\usepackage[scaled]{helvet}
\usepackage[noend]{algorithmic}
\usepackage{mathrsfs}
\usepackage{mathpartir}
\usepackage{dsfont} 
\usepackage{stmaryrd}
\usepackage{url}
\usepackage{textcomp} 
\usepackage[colorlinks=true,allcolors=blue,breaklinks,draft=false]{hyperref}
\usepackage{titlesec}
\usepackage{parskip}
\usepackage{alltt} 
\usepackage{bbm}
\usepackage{alltt}
\usepackage{verbdef}
\usepackage{xspace}
\usepackage{verbatim}
\usepackage{enumitem}
\usepackage{lipsum}
\usepackage{wrapfig}
\usepackage{a4wide}
\usepackage[usenames,dvipsnames]{xcolor}
\hypersetup{linkcolor=black,citecolor=black,urlcolor=black}
\newcounter{tags}
%\usepackage{flushend}

% \usepackage{natbib}
% \bibpunct();A{},
% \let\cite=\citep

% For turned column headers 
\usepackage{adjustbox} 
\usepackage{array}
\usepackage{booktabs}
\usepackage{multirow}
\usepackage{pifont}
 
% remarks
\newcommand{\todo}[1]{\textcolor{red}{({#1})}}
\newcommand{\is}[1]{\textcolor{blue}{(Ilya: {#1})}}
\newcommand{\an}[1]{\textcolor{red}{(Aleks: {#1})}}

% useful macros
\newcommand{\asgn}{\leftarrow}
\newcommand{\code}[1]{\lstinline{#1}}
\newcommand{\ccode}[1]{\code{#1}}
\newcommand{\var}[1]{\({#1}\)} 
\newcommand{\num}[1]{\({\text{{\scriptsize{#1}}}}\)}
\newcommand{\etc}{\emph{etc}}
\newcommand{\ie}{\emph{i.e.}\xspace}
\newcommand{\Ie}{\emph{I.e.}\xspace}
\newcommand{\eg}{\emph{e.g.}\xspace}
\newcommand{\Eg}{\emph{E.g.}\xspace}
\newcommand{\vs}{\emph{vs.}\xspace}
\newcommand{\etal}{\emph{et~al.}\xspace}
\newcommand{\adhoc}{\emph{ad hoc}\xspace}
\newcommand{\viz}{\emph{viz.}\xspace}
\newcommand{\dom}[1]{\mathsf{dom}(#1)}
\newcommand{\aka}{\textit{a.k.a.}\xspace}
\newcommand{\cf}{\textit{cf.}\xspace}
\newcommand{\wrt}{\emph{wrt.}\xspace}
\newcommand{\loef}{L\"{o}f}
\newcommand{\sep}{\textasteriskcentered}
\newcommand{\res}{\mathsf{res}} 
\newcommand{\ret}{\mathsf{ret}} 
\newcommand{\fix}{\mathsf{fix}} 

%specs
\newcommand{\specK}[1]{\ensuremath{\textcolor{blue}{#1}}}
\newcommand{\spec}[1]{\specK{\left\{{#1}\right\}}}
\newcommand{\specQ}[4]{[#1 , #2 , #3] \, #4}%\specQ{mL}{gL}{gE}{p}
\newcommand{\drspec}[1]{\specK{\langle{#1}\rangle}}
\newcommand{\sspec}[1]{\specK{\{{#1}\}}}




% Keep footnotes on one page
\interfootnotelinepenalty=10000 

\setlist[itemize]{leftmargin=*}

% \setlength{\parindent}{0.15in}
% \setlength{\topsep}{0cm}
% \setlength{\parskip}{0pt}

% \titlespacing*{\section}{0pt}{*1.5}{*1.5} 
% \titlespacing*{\subsection}{3pt}{*0.8}{*0.5}
% \titlespacing*{\subsubsection}{0pt}{*0.8}{*0.5}
% \titlespacing*{\paragraph}{0pt}{*0.5}{*1.2}

% SSReflect listings 
\input{lstcoq}
\lstset{style=Coq}

\usepackage{geometry}
\geometry{margin=2cm}

% Hyphenation
\hyphenation{Veri-Fast}

% Bibtgex tweaks
% \setcitestyle{square}
% \defcitealias{Coq-manual}{Coq proof assistant}

\begin{document}

\author{Ilya Sergey ~~ Aleksandar Nanevski ~~ Anindya Banerjee ~~
  Germ\'{a}n Andr\'{e}s Delbianco} 
% {$^\dag$University College London, UK \and\and $^\ddag$IMDEA Software Institute, Spain}
% {i.sergey@ucl.ac.uk \and \{aleks.nanevski, anindya.banerjee, german.delbianco\}@imdea.org}

%\special{papersize=8.5in,11in}


% \authorinfo{}{}{}

% \conferenceinfo{OOPSLA~'16} {June 13--17, 2015, Portland, OR, USA}
% \CopyrightYear{2016}
% \copyrightdata{TODO}
% \doi{TODO}


%\title{The Power of Subjectivity}



\title{Cover Letter for OOPSLA'16 Submission \#49}

\maketitle

\noindent
%
This cover letter lists the main revisions and additions made to the
paper \#49 entitled \textbf{"Hoare-style Specifications as Correctness Conditions for
    Non-linearizable Concurrent Objects"}, for the second round of
  OOPSLA'16 submission.

\section*{Summary of the Reviewers' concerns and suggested changes}
\label{sec:summ-revi-conerns}

The reviewers had the following concerns with respect to the initial
version of the submission:

\begin{enumerate}
\item Unclear contributions wrt. the previous work
  (\ie, Nanevski \etal, ESOP'14, Sergey \etal, ESOP15).

\item Lack of discussion with respect to related logics for
  concurrency (\eg, CAP and TaDA).

\item Lack of an introductory example motivating the development.

\item Comparison to the \emph{Views} framework.
\end{enumerate}

We addressed these concerns (as well as other, less critical ones) as
by making the changes outlined below.

\section*{Major changes}
\label{sec:major-changes}

\begin{itemize}

\item (\textbf{Section 1.2}) The introduction now better emphasizes
  the contribution of the work, stating some parts of it as
  "conceptual" (the method) and some other parts as "technical" (the
  verified case studies). It also positions the paper's contributions
  with respect to previous works on FCSL and its applications.

\item (\textbf{Section 2}) The \textbf{Main Ideas and Overview}
  section has
  been split into two new sections, with the verification of the
  exchanger implementation moved to a separate \textbf{Section 3}.

  The revised Section 2 now discusses two simple motivating examples:
  the exchanger (focusing on its use, without exploring the
  implementation) and a simplified version of the counting network,
  featuring two un-synchronized atomic updates to a shared
  pointer via the atomic \texttt{flip} operation. Both concurrent
  structures are first presented from the
  practical perspective of using them.

    The section then explains how to \emph{(i)} define the temporal/spatial
    invariants for them (using a coloring scheme as an intuition for
    subjective state dichotomy), \emph{(ii)} capture the effect on history
    with respect to interference in Hoare-style specs, and
    \emph{(iii)}
    immediately employ the specifications for verification of
    heterogeneous client programs.

    The section is concludes with the 3-step summary of the
    formalization pattern.

  \item (\textbf{Section 10}) The Related Work section has been
    expanded with a discussion on related logics and approaches for
    specifying concurrent programs (including the Views framework),
    incorporating the corresponding parts of the first round rebuttal.

\end{itemize}

\section*{Other Changes}
\label{sec:other-changes}

\begin{itemize}

\item (\textbf{Section 9}) The case studies are re-run with the lates
  Coq version to date (8.5pl2), which slightly changed the compilation
  times (see Table 1). Some of the times have increased, though, but
  more detailed forensic work will be required to understand why.

\item Explanations from the rebuttal are incorporated to the test
    (see the accompanying diff file).

\item Multiple small clarifications in the text of sections
    describing the case studies and relating the overall presentation
    to the initial toy examples.

\end{itemize}


%\newpage



\end{document}


