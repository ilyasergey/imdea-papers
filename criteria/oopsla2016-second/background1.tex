\section{Background on FCSL}
\label{sec:background}

In order to formally present \textbf{\emph{Step 3}} of our method, we
first need to introduce some important parts of FCSL.

A Hoare specification in FCSL has the form $\spec{P}\ e\ \spec{Q} @
\rcon$. $P$ and $Q$ are pre- and postcondition for partial
correctness, and $\rcon$ defines the \emph{shared resource} on which
$e$ operates. 
%
%We have elided $\rcon$ from the specs in
%Section~\ref{sec:overview}, but explain it now. 
%
The latter is a state transition system describing the invariants of
the state (real and auxiliary) and atomic operations that can be
invoked by the threads that simultaneously operate on that state.
%
% For instance, for the \code{flip2} structure from
% Section~\ref{sec:overview}, the shared resource would 
%
We elide the transition system aspect of resources here, and refer
to~\cite{Nanevski-al:ESOP14} for detailed treatment.

An important secondary role of a resource is to declare the variables
that $P$ and $Q$ may scope over. For example, in the case of
exchanger, we use the variables $\heaps, \perms, \hists$, $\heapo,
\permo, \histo$, and $\heapj, \pending$.
%
The mechanism by which the variables are declared is as
follows. Underneath, a resource comes with only three variables:
$a_\lcl$, $a_\env$ and $a_\joint$ standing for abstract self state,
other state, and shared (joint) state, but the user can pick their
types depending on the application. In the case of exchanger, $a_\lcl$
and $a_\env$ are triples containing a heap, an offer-set and a
history. The variables we used in Section~\ref{sec:overview} are
projections out of such triples: $a_\lcl\,{=}\,(\heaps, \perms,
\hists)$, and $a_\env\,{=}\,(\heapo, \permo, \histo)$. Similarly,
$a_\joint\,{=}\,(\heapj, \pending)$.

It is essential that $a_\lcl$ and $a_\env$ have a common type
exhibiting the algebraic structure of a PCM, under a partial binary
operation $\hunion$.
%
% In the case of $\cal E$, each of
%the three components of $a_\lcl$ and $a_\env$---heaps, offer-sets and
%histories---form a PCM, where $\bullet$ is disjoint union $\hunion$,
%and $\emptyset$ is the unit. Hence, the product of the three is a PCM
%as well, with $\bullet$ and unit defined point-wise. 
PCMs give a way, generic in $\rcon$, to define the
inference rule for parallel composition.
%
\[
\tag{\normalsize \arabic{tags}}\refstepcounter{tags}\label{eq:parrule}
{\small{
\begin{array}{c}
\specK{\{P_1\}}\ e_1\ \specK{\{Q_1\}} @ \rcon \quad \specK{\{P_2\}}\ e_2\ \specK{\{Q_2\}} @ \rcon\\[2pt]
\hline\\[-7pt]
\specK{\{P_1 \circledast P_2\}}\ e_1 \parallel e_2\ \specK{\{[\res.1/\res]Q_1 \circledast [\res.2/\res]Q_2\}} @ \rcon
\end{array}
}}
\]
%
Here, $\circledast$ is defined as follows.
\[
%\tag{\normalsize \arabic{tags}}\refstepcounter{tags}\label{eq:ssep}
\begin{array}{c}
(P_1 \circledast P_2)(a_\lcl, a_\joint, a_\env) \iff \exists x_1~x_2\ldot a_\lcl = x_1 \hunion x_2, \hbox{}\\
 P_1 (x_1, a_\joint, x_2 \hunion a_\env), P_2 (x_2, a_\joint, x_1 \hunion a_\env)
\end{array}
\]
%
%
Thereby, when a parent thread forks $e_1$ and $e_2$, then $e_1$
becomes part of the environment for $e_2$, and vice-versa. This is so
because the \emph{self} component $a_\lcl$ of the parent is split into
$x_1$ and $x_2$; $x_1$ becomes the \emph{self} part of $e_1$, but
$x_2$ is added to the \emph{other} part $a_\env$ of $e_1$ (and
symmetrically for $e_2$).
%
%Also note that parallel composition returns a pair of the outputs
%produced by $e_1$ and $e_2$. Thus, the variable $\res$ in $Q_1$ and
%$Q_2$ has to be appropriately renamed by the projections $\res.1$ and
%$\res.2$ in the postcondition of the parallel composition.

%The rule of frame of FCSL is a special case of parallel composition,
%when $e_2$ is the idle thread.
%%
%\[
%\tag{\normalsize \arabic{tags}}\refstepcounter{tags}\label{eq:frame}
%{\small{
%\begin{array}{c}
%\specK{\{P_1\}}\ e\ \specK{\{P_2\}} @ \rcon\\[2pt]
%\hline\\[-7pt]
%\specK{\{P_1 \circledast Q\}}\ e\ \specK{\{P_1 \circledast Q\}} @ \rcon
%\end{array}\qquad 
%\begin{array}{c}
%\mbox{$Q$ stable under}\\
%\mbox{$\rcon$'s transitions}
%\end{array}
%}}
%\]
%A notable difference from the frame rules of other separation logics
%is that FCSL's definition of $\circledast$ forces that the value being
%framed onto \emph{self} component is \emph{subtracted} from the
%\emph{other} component, whereas in other separation logic, the frame
%value materializes out of nowhere. To illustrate, we can frame
%$\gists$ onto the history $\hists$ in the the
%spec~(\ref{tag:exchangespec}), by taking
%$\rcon\,{\eqdef}\,a_\lcl\,{=}\,(\heaps, \hists,
%\perms)\,{=}\,(\emptyset, \gists, \emptyset)$.
%We obtain, after some simplification:
%%
%\[
%{\small{
%\begin{array}{c}
%\specK{\{\heaps = \emptyset, \perms = \emptyset, \hists = \gists, \gist \subseteq \gists \hunion \histo \hunion \mygather{\pending}\}}\\[2pt]
%\mathtt{exchange}\ v \\[2pt]
%\spec{\!\!
%  \begin{array}{c}
%    \heaps = \emptyset, \perms = \emptyset, \gist \subseteq \gists \hunion \histo \hunion \mygather{\pending}, \hbox{}\\[1pt]
%    \mathsf{if}\ \res\ \mathsf{is}\ \mathsf{Some}\ w\ \mathsf{then}\\[1pt]
%    \exists t\ldot \hists = t \mapsto (v, w) \hunion \gists, 
%    \mathsf{last} (\gist) < t, \twin{t}~\mathsf{else}\ \hists = \gists    
%  \end{array}
%\!\!}@\cal E
%\end{array}
%}}
%\]
%But notice how the spec now says that $\gist \subseteq \gists \hunion
%\histo \hunion \mygather{\pending}$, whereas
%in~(\ref{tag:exchangespec}) it said $\gist \subseteq \histo \hunion
%\mygather{\pending}$. The addition of $\gists$ compensates for
%$\gists$ having been subtracted out of $\histo$, to be moved to
%$\hists$.

To reason about quiescent moments, we use one more constructor of
FCSL: \emph{hiding}. The program $\mathsf{hide}\ e$ operationally
executes $e$, but logically installs a resource within the scope of
$e$. In the case of the exchanger, $\mathsf{hide}\ e$ starts only with
private heaps $\heaps$ and $\heapo$, then takes a chunk of heap out of
$\heaps$ and ``installs'' an exchanger in this heap, allowing the
threads in $e$ to exchange values. $\mathsf{hide}\ e$ is
\emph{quiescent} wrt.~exchanger, as the typechecker will prevent
composing $\mathsf{hide}\ e$ with threads that want to exchange values
with $e$.

The auxiliaries $\perms, \hists$, $\permo, \histo$, and $\heapj,
\pending$, belonging to the exchanger (denoted as resource $\cal E$)
are visible within $\mathsf{hide}$, but outside, only $\heaps$
persists (denoted as a resource $\cal P$ for private state).  We elide
the general hiding rule~\cite{Nanevski-al:ESOP14}, and just show the
special case for the exchanger.

\[
\tag{\normalsize \arabic{tags}}\refstepcounter{tags}\label{eq:ehide}
{\small{
\begin{array}{c}
\specK{\{P\}}\ e\ \specK{\{Q\}} @ \cal E\\[2pt]
\hline\\[-7pt]
\specK{\{\heaps = \Phi_1(\heapj), \Phi_1(P)\}}\ \mathsf{hide}~e\ \specK{\{\exists \Phi_2\ldot \heaps = \Phi_2(\heapj), \Phi_2(Q)\}} @ \cal P
\end{array}
}}
\]

Read bottom-up, the rule says that we can install the exchanger $\cal
E$ in the scope of a thread that works with $\cal P$, but then we need
substitutions $\Phi_1$ and $\Phi_2$, to map variables of $\cal E$
($\heaps, \perms, \hists$, \etc) to values expressed with variables
from $\cal P$ ($\heaps$ and $\heapo$). $\Phi_1$ is an initial such
substitution (user provided), and the rule guarantees the existence of
an ending substitution $\Phi_2$. The substitutions have to satisfy a
number of side conditions, which we elide here for brevity. The most
important one is that \emph{other} variable $a_\env = (\heapo, \permo,
\histo)$ is fixed to be the PCM unit (\ie,~a triple of empty
sets). Fixing $a_\env$ to unit captures that $\mathsf{hide}$ protects
$e$ from interference.

At the beginning of $\mathsf{hide}~e$, the private heap equals the
value that $\Phi_1$ gives to $\heapj$ ($\heaps = \Phi_1(\heapj)$). In
other words, the $\mathsf{hide}$ rule takes the private heap of a
thread, and makes it shared, \ie, gives it to the $\heapj$ component
of $\cal E$. Upon finishing, $\mathsf{hide}~e$ makes $\heapj$ private
again.
%
%($\heaps = \Phi_2(\heapj)$).

%\an{Should I say something about compositionality? Why is FCSL
%  compositional? Maybe say, soundness of FCSL has been established by
%  shallow embeding in Coq. Thus, the logic immediately inherits the
%  substitution principle, thereby allowing that clients can reason
%  only out of the Hoare spec of an object.}

%\subsection{FCSL basics}
%\label{sec:fcsl-basics}
%
%\todo{A short overview of FCSL: mostly, concerning subjectivity and hiding}
%
%\subsection{Histories as auxiliary state}
%\label{sec:hist-state} 


In the subsequent text we elide the resources from specs.

