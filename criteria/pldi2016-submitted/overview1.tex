\section{Overview}
\label{sec:overview}

%\paragraph{Example background.}

We introduce the main ideas behind our method via an example:
concurrent exchanger from
$\mathtt{java.util.concurrent}$~\cite{Scherer-al:SCOOL05,ExchangerClass}. This
example previously motivated the introduction of the correctness
condition CAL~\cite{Hemed-Rinetzky:PODC14}. The exchanger is not
linearizable to a set of sequential histories, as its behavior
crucially depends on interaction with interfering threads.
%
CAL works around this problem by employing new
\emph{concurrency-aware} (CA) histories as a specification set.

We sketch here a Hoare spec of the exchanger that relies on
subjectivity to describe the interaction with the interfering
environment, and compare the spec to that of
CAL~\cite{Hemed-al:DISC15} in Section~\ref{sec:related}.  Our
presentation uses the following order, which is also the order used by
FCSL verifications in general~\cite{Sergey-al:PLDI15}: 


\begin{itemize}

\item First, we explain the auxiliary state needed for expressing the
  object's state invariants and allowed state changes. 

\item Second, we describe the specifications and the main ideas of
  their proofs for the object's methods.

\item Third, we show how to use the spec to verify client properties,
  typically, considering the client in quiescent moments.

\end{itemize}


\subsection{Exchanger program} 
\label{sec:exchanger}

\newcommand{\Unmatched}{{\mathsf{U}}}
\newcommand{\Matched}[1]{{\mathsf{M}\ #1}}
\newcommand{\Retired}{{\mathsf{R}}}


The exchanger program is presented in ML-style pseudo-code in
Figure~\ref{fig:exchanger}. It takes a value $v\,{:}\,A$ and creates
an \emph{offer} from it (line 2). An offer is a pointer $p$ to two
consecutive locations in the heap.\footnote{In our mechanization, we
  simplify a bit by making $p$ point to a pair instead.}
%\footnote{In our mechanization, we
%  use a pointer to a pair, but for presentation here adopt a more
%  common style of consecutive heap locations.} 
%
The first location stores $v$, and the second is a ``hole'' which the
interfering thread try to fill with a matching value. The hole is
drawn from the type
$\esc{hole}\,{=}\,\Unmatched\,{\mid}\,\Retired\,{\mid}\,\Matched
w$. Constructor $\Unmatched$ signals that the offer is unmatched;
$\Retired$ that the exchanger retired (\ie, withdrew) the offer, and
does not expect any matches on it; and $\Matched w$ that the offer has
been matched with a value $w$.

The global pointer $g$ stores the latest offer proposed for
matching. The exchanger proposes $p$ for matching by making $g$ point
to $p$ via \code{CAS} (line 3). We assume that \code{CAS} returns the
value read, which can be used to determine if it failed or
succeeded. If \code{CAS} succeeds, exchanger waits a bit, then checks
if the offer has been matched by some $w$ (lines 6, 7). If so,
$\esc{Some}\ w$ is returned (line 7). Otherwise, the offer is retired
by storing $\Retired$ into its hole (line 6). Retired offers remain
allocated (thus, exchanger has a memory leak) in order to avoid the
ABA problem, as usual in many concurrent
structures~\cite{Herlihy-Shavit:08,Treiber:TR}.
%
If the exchanger fails to link $p$ into $g$ in line 3, it deallocates
the offer $p$ (line 10), and instead tries to match the offer $cur$
that is current in $g$. If no offer is current, perhaps because
another thread already matched the offer that made the \code{CAS} in
line 3 fail, the exchanger returns $\mathsf{None}$ (line
12). Otherwise, the exchanger tries to make a match, by changing the
hole of $cur$ into $\Matched v$ (line 14). If successful (line 16), it
reads the value $w$ stored in $cur$ that was initially proposed for
matching, and returns it. In any case, it unlinks $cur$ from $g$ (line
15) to make space for other offers.
%
% \gad{About Figure~\ref{fig:exchanger}: remember that a reviewer asked
%   about the meaning of $!$. Even if we are already declaring our
%   sintax to be ``ML-style'', we could mention this somehow in the
%   description above. Also, and perhaps I'm being too picky here, is
%   there a reason why atomic actions don't have a space between their
%   name and the arguments? So far the paper has not mentioned actions
%   or given details about programs in FCSL, so why make the
%   distinction?}
% \an{fixed}

{
\setlength{\belowcaptionskip}{-10pt} 
\begin{figure}
\centering
\[
{\small{
\begin{array}{rl}
 \Num{1} & \esc{exchange}~(v : A) : \esc{option}~A~=~\{ 
\\ 
 \Num{2} & ~~~~ p \Asgn \esc{alloc}~(v, \Unmatched);\\
 \Num{3} & ~~~~ b \Asgn \esc{CAS}~(g, \esc{null}, p);\\
 \Num{4} & ~~~~ \kw{if}~~b~\esc{==}~\esc{null}~~\kw{then}\\
 \Num{5} & ~~~~ ~~~~ \esc{sleep}~(50);\\
 \Num{6} & ~~~~ ~~~~ x \Asgn \esc{CAS}~(p\esc{+}1, \Unmatched, \Retired);\\
 \Num{7} & ~~~~ ~~~~ \kw{if}~~x~~\kw{is}~~\Matched w~~\kw{then}~~\kw{return}~~(\esc{Some}~w)\\
 \Num{8} & ~~~~ ~~~~ \kw{else}~~\kw{return}~~\esc{None}\\
 \Num{9} & ~~~~ \kw{else}\\
\Num{10} & ~~~~ ~~~~ \esc{dealloc}~p;\\
\Num{11} & ~~~~ ~~~~ cur \Asgn \esc{read}~g;\\
\Num{12} & ~~~~ ~~~~ \kw{if}~~cur~\esc{==}~\esc{null}~~\kw{then}~~\kw{return}~{\esc{None}}\\
\Num{13} & ~~~~ ~~~~ \kw{else}\\
\Num{14} & ~~~~ ~~~~ ~~~~ x \Asgn \esc{CAS}~(cur\esc{+}1, \Unmatched, \Matched v);\\
\Num{15} & ~~~~ ~~~~ ~~~~ \esc{CAS}~(g, cur, \esc{null});\\
\Num{16} & ~~~~ ~~~~ ~~~~ \kw{if}~~x~~\kw{is}~~\Unmatched~~\kw{then}~~w\Asgn \esc{read}~cur;\kw{return}~(\esc{Some}\ w)\\
\Num{17} & ~~~~ ~~~~ ~~~~ \kw{else}~~\kw{return}~\esc{None}\}
\end{array}
}}
\]
\vspace{-5pt}  
\caption{Elimination-based exchanger procedure.}
\label{fig:exchanger}
\vspace{-10pt}  
\end{figure} 
}


\subsection{Auxiliary state and invariants}

To specify the exchanger, we first decorate it with auxiliary
state. The state is \emph{subjective}; that is, it keeps thread-local
auxiliary variables that name the thread's private state
(aka.~\emph{self}), but also the private state of all other threads
combined (aka.~\emph{other}).  The subjective division will be
essential for client reasoning in Section~\ref{sec:cal}.
%
%The subjective division suffices to specify the exchanger in Hoare
%logic, and will be essential for client reasoning in
%Section~\ref{sec:cal}.
%
%In Section~\ref{sec:counting}, we show how the approach directly
%scales to naturally specify programs that have heretofore only been
%addressed using quiescent and quantitative quiescent consistency.

The subjective state for each thread in this example consists of three
groups of two components: (1) thread-private heap $\heaps$ of the
thread, and of the environment $\heapo$, (2) a set of outstanding
offers $\perms$ created by the thread, and by the environment
$\permo$, and (3) a time-stamped history of values $\hists$ that the
thread exchanged so far, and dually $\histo$ for the environment.
This is thread-private state; we also need shared (aka.~\emph{joint})
state consisting of two components: a heap $\heapj$ of made offers,
and a map $\pending$ of ``matched offers''.

Heaps, sets and histories are all PCMs under the operation of disjoint
union, with empty heap/set/history as a unit. We overload the notation
and write $x\,{\mapsto}\,v$ for a singleton heap with a pointer $x$
storing value $v$, and $t\,{\mapsto}\,a$ for a singleton
history. Similarly, we apply disjoint union $\hunion$ and subset
$\subseteq$, to all three types uniformly.

We next describe how the exchanger manipulates the above variables.
%for maps, and write $x \mapsto y$ for a singleton map, no matter
%its type. For example, in the case of time-stamped histories, $t
%\mapsto b$ denotes that at time-stamp $t$, some abstract behavior $b$
%took place. In the case of heaps, $x \mapsto v$ for a singleton heap
%with a pointer $x$ storing value $v$. We also overload the operator
%$\hunion$ for disjoint union, and apply it to histories, heaps, and
%set of offers, uniformly. Thus, each history or heap will be a
%disjoint union of a number of singleton entries, and we use $x
%\subseteq y$ to say that the map $x$ is a submap (or subheap, or
%subhistory) of $y$.
First, $\heapj$ is a heap that serves as the ``staging'' area for the
offers. It includes the global pointer $g$. Whenever a thread wants to
make an offer, it allocates a pointer $p$ in $\heaps$, and then tries
to move $p$ from $\heaps$ into $\heapj$, simultaneously linking $g$ to
$p$.

Second, $\perms$ and $\permo$ are sets of offers (hence, sets of
pointers) that determine offer ownership. A thread that has the offer
$p \in \perms$ is the one that created it, and thus has the
\emph{sole} right to retire $p$, or to collect the value that $p$ was
matched with. Upon collection or retirement, $p$ is removed from
$\perms$.

%To ensure that only one thread has such a right, one of the important
%state invariants of the exchanger is that $\perms$ and $\permo$ are
%disjoint sets.

Third, $\hists$ and $\histo$ are histories, each mapping a time-stamp
(isomorphic to nats), to a pair of exchanged values. A singleton
history $t \mapsto (v, w)$ symbolizes that a thread having this
singleton as a subcomponent of $\hists$, has exchanged $v$ for $w$ at
time $t$.
%
%Histories have been used in FCSL before~\cite{Sergey-al:ESOP15}, but
%only on linearizable examples, where each time-stamp $t$ ``stored'' an
%atomic operation performed at time $t$. In the exchanger example,
%which is not linearizable, a time-stamp $t$ stores only a \emph{half}
%of an atomic operation, \ie, a pair $(v, w)$ signaling that a thread
%exchanged $v$ for $w$. 
%
As we describe below, the most important invariant of the exchanger is
that each such singleton is matched by a ``symmetric'' one to capture
that another thread has \emph{simultaneously} exchanged $w$ for
$v$. Classical linearizability cannot express this simultaneous
behavior, making the exchanger non-linearizable.

Fourth, $\pending$ is a map storing the offers that were matched, but
not yet collected. Thus, $\mathsf{dom}\ \pending = \perms \hunion
\permo$. A singleton entry in $\pending$ has the form $p \mapsto (t,
v, w)$ and denotes that offer $p$, initially storing $v$, was matched
at time $t$ with $w$. A singleton entry is entered into $\pending$
when a thread on the one end of matching, matches $v$ with $w$. Such a
thread also places the \emph{twin} entry $\twin{t} \mapsto (w, v)$,
with inverted order of $v$ and $w$, into its own private history
$\hists$, where:
%
\vspace{-5pt}
\[
\begin{array}{c}
\vspace{-5pt}
\twin{t} = \left\{%
\begin{array}{ll}
t+1 & \mbox{if $t$ is odd}\\
t-1 & \mbox{if $t > 0$ and $t$ is even}
\end{array}\right.  
\end{array}
\]
%
For technical reasons, $0$ is not a valid time-stamp, and has no
distinct twin. The pending entry for $p$ resides in $\pending$ until
the thread that created the offer $p$ decides to ``collect'' it. It
removes $p$ from $\pending$, and simultaneously adds the entry $t
\mapsto (v, w)$ into its own $\hists$, thereby logically completing
the exchange. Since twin time-stamps are consecutive integers, a
history cannot contain entries \emph{between} twins.

Thus, two twin entries in the combined history including $\hists$,
$\histo$ and $\pending$, jointly represent a single exchange, as if it
occurred \emph{atomically}. CA-histories~\cite{Hemed-Rinetzky:PODC14}
capture this by making the ends of an exchange occur as simultaneous
events. We capture it via twin time-stamps. More formally, consider
$\hist = \hists \hunion \histo \hunion \mygather{\pending}$. Then, the
exchanger's main invariant is that $\hist$ always contains matching
twin entries:
%
\vspace{-5pt}
\[
\vspace{-5pt}
\tag{\arabic{tags}}\refstepcounter{tags}\label{tag:exchanging} 
t \mapsto (v, w) \subseteq \hist \iff \twin t \mapsto (w, v) \subseteq \hist
\]
%
Here $\mygather{\pending}$ is the collection of all the entries in
$\pending$. That is, $\mygather{\emptyset} = \emptyset$, and
$\mygather {p \mapsto (t, v, w) \hunion \pending'} = t \mapsto (v, w)
\hunion \mygather{\pending'}$.

%One consequence of this property is that the history $\hist$ always
%has an even number of entries. Thus, if $t$ is the smallest unused
%time-stamp in $\hist$ (not counting $0$), then $t$ must be odd, and
%hence $t < \twin t = t + 1$.

\gad{Even if this subsection is quite thorough in the explanation of
  the ghosts, I would consider adding a dedicated {\bf Notation}
  summary paragraph in the end. First, because I think it wouldn't
  hurt to list the make a summary of the ghost variables with their
  type .e.g $\chi, \hists, \histo$ denote histories, etc. Also, and
  perhaps foremost, because I would like to explicitly remind the
  reader/reviewer that we always use $_{\lcl}$ subscript to denote
  self ghost variables $\hists$, $\heaps$, etc. and, respectively for
  $_{\env}$ and $_{\joint}$. I think that when we read our specs, we
  have that distinction wired down, whereas the occasional reader
  doesn't and reminding them might help them grock our specs. Also, it
  would be a place to gather all the different notation overloadings.}

\subsection{Specification}
We can now give the desired Hoare spec. 
%
\[
\tag{\arabic{tags}}\refstepcounter{tags}\label{tag:exchangespec} 
{\small{
\begin{array}{c}
\specK{\{\heaps = \emptyset, \perms = \emptyset, \hists = \emptyset, \gist \subseteq \histo \hunion \mygather{\pending}\}}\\
\esc{exchange}\ v\\
  \spec{\!\!
  \begin{array}{c}
\heaps = \emptyset, \perms = \emptyset, \gist \subseteq
  \histo \hunion \mygather{\pending}, \hbox{}\\[1pt]
\mathsf{if}\ \res\ \mathsf{is}\ \mathsf{Some}\ w\ \mathsf{then}~\\[1pt]
\exists t\ldot \hists = t \mapsto (v, w), \mathsf{last} (\gist) < t, \twin{t}
~\mathsf{else}\ \hists = \emptyset 
  \end{array}
  \!\!}
\end{array}
}}
\]
%
The precondition says that the exchanger starts with the empty private
heap $\heaps$, set of offers $\perms$ and history $\hists$; hence by
framing, it can start with any value for these
components.\footnote{Framing in FCSL is similar to that of separation
  logic, allowing extensions to the initial state that remain
  invariant by program execution. In FCSL, however, framing applies to
  any PCM-valued state component (\eg, heaps, histories, \etc.),
  whereas in separation logic, it applies just to heaps.} The logical
variable $\gist$ names the initial history of all threads, $\histo
\hunion \mygather{\pending}$. We use subset instead of equality to
make the precondition stable under other threads adding history
entries to $\histo$ or $\pending$.

In the postcondition, the self heap $\heaps$ and the set of offers
$\perms$ didn't change. Hence, if $\mathtt{exchange}$ made an offer
during its execution, it also collected or retired it by the end.
%
The history $\gist$ is still a subset of the ending value for $\histo
\hunion \mygather{\pending}$, signifying that the environment history
only grows by interference. We will make a crucial use of this part of
the spec when verifying a client of the exchanger in
Section~\ref{sec:cal}.

If the exchange fails (\ie, $\mathsf{res}$ is $\mathsf{None}$), then
$\hists$ remains empty.  If it succeeds (either in line 7 or line 16
in Figure~\ref{fig:exchanger}), \ie, if the result $\mathsf{res}$ is
$\mathsf{Some}\ w$, then there exists a time-stamp $t$, such that
self-history $\hists$ contains the entry $t \mapsto (v, w)$,
symbolizing that $v$ and $w$ were exchanged at time $t$.

Importantly, the postcondition implies, by
invariant~(\ref{tag:exchanging}), that in the success case, the twin
entry $\twin t \mapsto (w, v)$ must belong to $\histo \hunion
\mygather{\pending}$, \ie, \emph{another} thread matched the exchange.
Moreover, the exchange occurred \emph{after} the call to
$\mathsf{exchange}$: whichever $\gist$ we chose in the pre-state, both
$t$ and $\twin t$ are larger than then the last time-stamp in $\gist$.
%
The proof outline for the exchanger, together with detailed
commentary, is available in 
%
\ifext{Appendix~\ref{app:exch}}{the supporting material}.

In Section~\ref{sec:cal}, after introducing necessary FCSL background,
we will show how to employ the subjective Hoare
spec~\eqref{tag:exchangespec} for modular verification of a concurrent
client.

% \is{Where is the discussed ``general pattern'', which we wanted to
%   summarize around this moment of the paper (Anindya also suggests it
%   in his very first comment)? I believe, this is a good place to put
%   it.}
% \an{It's in the second paragraph of Section 2.}
