%\vspace{-4pt}

\section{Conclusion and Future Work}
\label{sec:conclusion}

%\vspace{-2pt}

In this paper we have presented a number of formalization techniques,
enabling specification and verification of highly scalable
non-linearizable concurrent objects and their clients in Hoare-style
program logics.
%
Specifically, we have explored several reasoning patterns, all
involving the idea of formulating execution histories as an instance
of auxiliary state and then making these histories to be a subject of
object-specific invariants, capturing the expected concurrent object
behavior.
%
In particular, we have discovered that quantitative logic-based
reasoning about concurrent behaviors can be done by storing relevant
information about interference directly into the entries of a logical
auxiliary history, a pattern which we later demonstrated to be
beneficial in the client-side proofs.

We believe that our results help to bring the Hoare-style reasoning
approach into the area of non-linearizable concurrent data structures
and open a number of exciting opportunities for the field of
logic-based concurrency verification.

For instance, by ascribing interference-sensitive quantitative specs
in the spirit of~\eqref{eq:qc-spec} to relaxed concurrent
libraries~\cite{Henzinger-al:POPL13}, one can assess the applicability
of a particular library implementation for its clients, that can
tolerate the anomalies caused by interference, as long as they can
logically infer the desired safety assertions from the library spec,
as we did in Section~\ref{sec:qclients}.
%
Since logical approaches enable reasoning about higher-order
concurrent data
structures~\cite{Svendsen-al:ESOP13,Turon-al:ICFP13,Sergey-al:ESOP15},
we envision the possibility of giving parametric logical specs to such
generic relaxed constructions as diffracting/elimination
trees~\cite{Shavit-Touitou:TCS97,Shavit:CACM11} that, once
instantiated with suitably specified stacks or pools on the leaves,
would yield a provably correct, highly scalable concurrent container
implementation.


% Acknowledgements:

% Michael Emmi
% Pierre Ganty
% Andrea Cerone
% Anton Podkopaev

% \todo{Generalizing the construction of the counting network to
%   arbitrary diffracting trees}

% \todo{Elimination and diffracting trees~\cite{Shavit-Touitou:TCS97}.}
