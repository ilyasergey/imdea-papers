
\section{Related Work}
\label{sec:related}

% Existing methods for specifying the behavior of concurrent data
% structures and programs are either history-based or state-based.
% %
% The former ones describe the behavior of an object by describing and
% characterizing call/return strings that can be obtained by invoking
% the methods concurrently by several threads. 
% %
% The later ones define the object's behavior by posing requirements to
% a state, in which object's methods are safe to run, and stating
% assertions over a state, resulting from the object's method call. The
% second group of approaches employs program logics as a way to specify
% the state of interest as well as a mechanism to compose the program
% specifications.

% This paper contributes to the goal of unifying the history- and
% state-based views to specification and verification of concurrent
% objects.

In this section we describe related approaches for reasoning about
concurrency that have motivated our work.

\paragraph{Linearizability and relaxed history-based criteria.}

In the twenty five years since its invention, linearizability has been
shown to be remarkably scalable as a correctness condition, widely
applicable to capture the behavior of implementations of multiple
concurrent objects with intuitive sequential specifications, such as
stacks, queues, sets, locks, snapshots and their combinations, and
even suitable for automatic synthesis of some concurrent
objects~\cite{Vechev-Yahav:PLDI08}. Thanks to the compositional proof
method, based on \emph{linearization points}, proofs of
linearizability in most of the cases are amenable for practical
computer-aided
verification~\cite{Burckhardt-al:PLDI10,Derrick-al:TOPLAS11,Vafeiadis:CAV10,Amit-al:CAV07,Shacham-al:OOPSLA11}.

The need for alternative, more relaxed yet compositional, correctness
criteria for describing the behavior of some concurrent objects, has
been recognized in the work that introduced counting
networks~\cite{Aspnes-al:JACM94}, a structure, that later has been
proved to be impossible to implement in a both efficient and
linearizable way~\cite{Herlihy-al:DC96}.
%
The suggested criterion, dubbed quiescent
consistency~\cite{Shavit-Zemah:TOPLAS96}, required the operations
separated by a quiescent state to take effect in their logical order. 
%
%\ab{In the order in which they arrived?}
%
%\is{I rephrased this as logical order}
%
The need for a quantitative correctness condition which relaxes
linearizability, allowing one to reason about ``almost'' linearizable
behaviors ($\varepsilon$-linearizability), was hinted later by Shavit
and Touitou~\cite{Shavit-Touitou:SPAA95}, although no formal definition
was provided.

Such a correctness condition, implementing a relaxed version of
linearizability with an upper bound on nondeterminism, was proposed by
Afek~\etal under the name
\emph{quasi-linearizability}~\cite{Afek-al:OPODIS10}, allowing them to
obtain the quantitative boundary similar to what we proved in
Section~\ref{sec:qqc-client} for a bitonic counting network of a fixed
width.
%
The idea of relaxed linearizability was later employed by
Henzinger~\etal in the work on \emph{quantitative
  relaxation} (QR)~\cite{Henzinger-al:POPL13} to drive the methodology
for designing novel scalable concurrent data structures by changing
the specification set of sequential histories, so it would include
\emph{out-of-order} or \emph{stuttering} behaviors.
%
Most recently, \emph{quantitative quiescent consistency} has been
proposed as another criterion incorporating the possibility to reason
about effects of bounded interference in relaxed data
structures~\cite{Jagadeesan-Riely:ICALP14}.

It is worth noticing that some of the listed correctness criteria are
incomparable (\eg, QC and QR~\cite{Henzinger-al:POPL13}, QL and
QQC~\cite{Jagadeesan-Riely:ICALP14}) hence, for a particular
concurrent object, choosing one or another criterion should be
justified by the needs of the object's client. However, how suitable
is one or another relaxed correctness condition %from the listed ones
for reasoning about safety properties of a particular client program
is still largely an open question.

\paragraph{Hoare-style specifications of concurrent objects.}
\label{sec:related-logic-based}

Program logics for concurrency allow one to capture in the
specification precisely those bits of information about the program's
state, which are relevant for the program's clients, making it
possible to abstract over the irrelevant details of the
implementation~\cite{DinsdaleYoung-al:ECOOP10}.

Hoare-style program logics were used with great success to specify and
verify a number of concurrent data structures and algorithms, which
are much more natural to specify in terms of observable state
modifications, rather than via call/return histories. The examples of
such objects and programs include
barriers~\cite{Dodds-al:POPL11,Hobor-Gherghina:ESOP11}, concurrent
indices~\cite{ArrozPincho-al:OOPSLA11}, flat
combiner~\cite{Turon-al:ICFP13,Sergey-al:ESOP15}, event
handlers~\cite{Svendsen-Birkedal:ESOP14}, shared graph
manipulations~\cite{Raad-al:ESOP15,Sergey-al:PLDI15}, as well as their
multiple client programs.
%
The observation about a possibility of using Hoare-style logics as a
correctness criterion, alternative to linearizability, has been made
in some of the prior
works~\cite{ArrozPincho-al:OOPSLA11,Svendsen-al:ESOP13}, but it was
never substantiated by demonstrating Hoare-style reasoning about
objects whose behavior is specified via dedicated
criteria~\cite{Hemed-Rinetzky:PODC14,Aspnes-al:JACM94,Jagadeesan-Riely:ICALP14}.
%
% Verification in such logics is done structurally, \ie, by
% systematically applying syntax-directed inference rules, until the
% spec is proved. Several mechanized tools for logic-based
% concurrent reasoning have been
% released~\cite{Sergey-al:PLDI15,Appel-al:BOOK14}.

Several logics for proving linearizability or, equivalently,
observational refinement~\cite{Filipovic-al:TCS10}, have been proposed
recently~\cite{Turon-al:ICFP13,Liang-Feng:PLDI13,Vafeiadis:PhD}, all
employing variations of the idea of using \emph{specifications as
  resources}, and identifying (possibly, non-fixed or non-local)
linearization points, at which such specification should be ``run''.
%
In these logics, after establishing linearizability of an object's
operation, one should still devise a suitable Hoare-style
specification for it, which would be useful for the clients.
%
To avoid this detour, a number of other logic-based approaches have
suggested to assign Hoare-style specifications \emph{directly} to
fine-grained object
implementations~\cite{Svendsen-al:ESOP13,Svendsen-Birkedal:ESOP14,ArrozPincho-al:ECOOP14}
and reason out of these specs.

Similarly to the way linearizability allows one to think of a
concurrent operation in terms of an atomic one, these logics implemented the
notion of \emph{logical atomicity}, allowing the clients of a data
structure, to implement application-specific synchronization on top of
the data structure operations, suitably specified in the Hoare style.
%
Different logics implement abstract atomicity either by parametrizing
program specs with client-specific auxiliary
code~\cite{Jacobs-Piessens:POPL11} (usually dubbed
\emph{view-shift}~\cite{DinsdaleYoung-al:POPL13})~\cite{Svendsen-al:ESOP13,Svendsen-Birkedal:ESOP14,Jung-al:POPL15}
or by engineering dedicated rules for resource coupling, relying on
the simulation between the actual implementation and the ``atomic''
one~\cite{ArrozPincho-al:ECOOP14}.
%
Both approaches to logical atomicity require one to identify precisely
a \emph{synchronization point}~\cite{Svendsen-al:ESOP13} within the
structure being verified, which makes it non-trivial to apply them for
specifying non-linearizable objects, especially when conducting
quantitative logic-based reasoning, which we demonstrated in
Section~\ref{sec:qqc-client}.

Instead of trying to extend the existing approaches for logical
atomicity to non-linearizable objects (for which the notion of
atomicity is not intuitive), we relied on a general mechanism of
auxiliary state, provided by FCSL~\cite{Nanevski-al:ESOP14}. 
%
Specifically, we adopted the idea of histories as auxiliary
state~\cite{Sergey-al:ESOP15}, which, however, was previously explored
in the context of FCSL only for specifying linearizable structures.
%
% Similarly to logical atomicity, the histories-as-state specification
% approach makes fine-grained concurrent programs look like atomic ones,
% since due to compositionality, the specification clients only see the
% history entries.
%
% 
To make the proof outlines more intuitive, in this work we introduced
enhanced notation for referring directly to histories (\eg, $\hists$,
$\histo$) instead of generic components of concurrent resources,
although FCSL's initial logical infrastructure, notion of concurrent
resources, and inference rules remained unchanged.


% \paragraph{Reasoning about linearizability in program logics}
% \label{sec:rel-linear}

% \begin{itemize}

% \item Logics for linearizability (viktor's PhD), LF, Turon, contextual refinement

% \item Hindsight paper by O'Hearn and company

% \item Observation from the HOCAP paper about the push2 method, that,
%   however, never were extended to any interesting structures, so we
%   did it.

% \item Abstract atomicity

% \end{itemize}

