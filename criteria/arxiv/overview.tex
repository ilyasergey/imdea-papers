\section{Overview}
\label{sec:overview}

%\paragraph{Example background.}

We introduce the main ideas behind our method via an example:
concurrent exchanger from the $\mathtt{java.util.concurrent}$
library~\cite{Scherer-al:SCOOL05,ExchangerClass}. This example
previously motivated the introduction of a new correctness criteria of
concurrency-aware linearizability~\cite{Hemed-Rinetzky:PODC14}. The
exchanger cannot be related by linearizability to a set of
\emph{sequential} histories, as its behavior crucially depends on
concurrent interaction with interfering threads.
%
CAL works around this restrictions by employing new
\emph{concurrency-aware} (CA) histories as a specification set
instead.
%
In contrast, we here show how to specify and verify the exchanger
using Hoare-style pre- and postconditions in FCSL, and rely on
subjectivity to capture the interaction between the exchanger and the
interfering threads.

%
%\an{Some of this should be put into intro? FCSL will provide a syntactic verification method (something not
%currently available for CAL), and enable client-side reasoning out of
%Hoare-style specs, as we illustrate in Section~\ref{sec:cal}. In the
%case of linearizability, client-side reasoning is justified by a
%theorem that linearizability implies contextual refinement
%(Mijajlovic, O'Hearn, et al.); thus a client can replace the
%fine-grained concurrent program with a simpler atomic, typically
%sequential program. However, a contextual refinement result is also
%not currently available for CAL.}
%
%\is{Instead of linearizability wrt \emph{sequential program}, let's
%  talk about linearizability wrt \emph{sequential set of
%    histories}. This is more accurate and follows the story from the
%  intro. Speking of seuqntial programs also puts us under risk of
%  getting someone Gotsman-like who will tell us to use linearizablity
%  wrt. \emph{non-sequential program} (as he mentioned to me once), and
%  god knows what it is.
%    %
%  How about just saying that one cannot specify it using sequential
%  histories and hence, need to adopt new specification set (\ie,
%  CA-histories), then making the clients aware of this fact (and
%  nobody has seen it done, so here we come, client-friendly, as will
%  be shown).}

\subsection{Exchanger program} 
\label{sec:exchanger}

\newcommand{\Unmatched}{{\mathsf{U}}}
\newcommand{\Matched}[1]{{\mathsf{M}\ #1}}
\newcommand{\Retired}{{\mathsf{R}}}

{
\setlength{\belowcaptionskip}{-10pt} 
\begin{figure}
\centering
\[
{\small{
\begin{array}{rl}
 \Num{1} & \esc{exchange}~(v : A) : \esc{option}~A~=~\{ 
\\ 
 \Num{2} & ~~~~ p \Asgn \esc{alloc}(v, \Unmatched);\\
 \Num{3} & ~~~~ b \Asgn \esc{CAS}(g, \esc{null}, p);\\
 \Num{4} & ~~~~ \kw{if}~~b~\esc{==}~\esc{null}~~\kw{then}\\
 \Num{5} & ~~~~ ~~~~ \esc{sleep}(50);\\
 \Num{6} & ~~~~ ~~~~ x \Asgn \esc{CAS}(p\esc{+}1, \Unmatched, \Retired);\\
 \Num{7} & ~~~~ ~~~~ \kw{if}~~x~~\kw{is}~~\Matched w~~\kw{then}~~\kw{return}~~(\esc{Some}~w)\\
 \Num{8} & ~~~~ ~~~~ \kw{else}~~\kw{return}~~\esc{None}\\
 \Num{9} & ~~~~ \kw{else}\\
\Num{10} & ~~~~ ~~~~ \esc{dealloc}~~p;\\
\Num{11} & ~~~~ ~~~~ cur \Asgn !g;\\
\Num{12} & ~~~~ ~~~~ \kw{if}~~cur~\esc{==}~\esc{null}~~\kw{then}~~\kw{return}~{\esc{None}}\\
\Num{13} & ~~~~ ~~~~ \kw{else}\\
\Num{14} & ~~~~ ~~~~ ~~~~ x \Asgn \esc{CAS}(cur\esc{+}1, \Unmatched, \Matched v);\\
\Num{15} & ~~~~ ~~~~ ~~~~ \esc{CAS}(g, cur, \esc{null});\\
\Num{16} & ~~~~ ~~~~ ~~~~ \kw{if}~~x~~\kw{is}~~\Unmatched~~\kw{then}~~w\Asgn !cur;\kw{return}~(\esc{Some}\ w)\\
\Num{17} & ~~~~ ~~~~ ~~~~ \kw{else}~~\kw{return}~\esc{None}\}
\end{array}
}}
\]
%\vspace{-10pt}  
\caption{Elimination-based exchanger procedure.}
\label{fig:exchanger}
\end{figure} 
}

The exchanger program is presented in a higher-typed ML-style
pseudo-code in Figure~\ref{fig:exchanger}. It takes a value $v : A$
and creates an \emph{offer} from it. An offer is a pointer $p$ to two
consecutive locations in the heap.\footnote{In our Coq sources, we use
  a pointer to a pair, but for presentation here, we adopt a more
  standard style with consecutive memory locations.} The first
location stores $v$, and the second is a ``hole'' which the
interfering thread try to fill with a matching value. The hole is
drawn from the type
$\esc{hole}\,{=}\,\Unmatched\,{\mid}\,\Retired\,{\mid}\,\Matched
w$. Constructor $\Unmatched$ signals that the offer is unmatched;
$\Retired$ that the exchanger retired (\ie, withdrew) the offer, and
does not expect any matches on it; and $\Matched w$ that the offer has
been matched with a value $w$.

The global pointer $g$ stores the latest offer proposed for
matching. The exchanger first allocates an offer $p$ (line 2), and
tries to propose it for matching by making $g$ point to $p$ via \code{CAS}
(line 3). We assume that \code{CAS} returns the value read, which can be used
to determine if it failed or succeeded. If \code{CAS} succeeds, exchanger
waits a bit, then checks if the offer has been matched by some $w$
(lines 6, 7). If so, $\esc{Some}\ w$ is returned (line 7). Otherwise,
the offer is retired by storing $\Retired$ into its hole (line
6). Notice that the exchanger leaves the offer allocated, and thus
exhibits a memory leak. This is a common idiom in concurrent data
structures, introduced to avoid the ABA
problem~\cite{Herlihy-Shavit:08,Treiber:TR}, and the exchanger is
similar in this respect.
%
If the exchanger fails to link $p$ into $g$ in line 3, it deallocates
the offer $p$ (line 10), and instead tries to match the offer $cur$
that is current in $g$. If no offer is current, perhaps because
another thread already matched the offer that made the \code{CAS} in line 3
fail, the exchanger returns $\mathsf{None}$ (line 12). Otherwise, the
exchanger tries to make a match, by changing the hole of $cur$ into
$\Matched v$ (line 14). If successful (line 16), it reads the value $w$
stored in $cur$ that was initially proposed for matching, and returns
it. In any case, it unlinks $cur$ from $g$ (line 15) to make space for
other offers.
%
%\an{The proof has actually been done this way, because I got lazy to
%  work around the details of using \code{CAS}. Do we want to keep the
%  presentation like this? It may cause reviewers to complain, so I may
%  just as well change the code to use \code{CAS} instead of \code{CAS}', at the
%  expense of one more read. I'm not sure it's worth it, however. If
%  someone decides to criticize us on being unrealistic, there will be
%  other properties of our memory model that one can complain about
%  (e.g., that we assume \code{CAS} to work over pointers storing arbitrary
%  types).}
%%
%\is{I think, the issue you describe is not an issue at all, as \code{CAS}
%  returns the previous value (\ie, ($w$, U)), so you just spend an
%  extra line or two to do the pattern matching on the \code{CAS}'s result and
%  return Some ... or None depending on it.}  
%%
%\an{I didn't understand the point of \code{CAS} returning the previous
%  value. The first \code{CAS} that's reading a pointer for being null returns
%  a boolean, doesn't it? So the two \code{CAS}-es used in this program are
%  different. Reading on Wikipedia: the boolean-returning version is
%  called compare-and-set, the old-value-returning version is called
%  compare-and-swap (I didn't know this). Both abbreviated as
%  \code{CAS}. Either way, the two are different. Should I make that explicit
%  by different naming? Or make the first return the old pointer, and
%  then branch on that pointer being null. Or, have both return a bool,
%  and then have an extra read.}

\subsection{Exchanger invariants and specification}

To specify the example, we first decorate the state space of the
program with auxiliary state. Our state will be \emph{subjective} in
the sense that we will keep \emph{thread-local} auxiliary variables
which name, within the scope of each thread, the thread's private
state (aka.~\emph{self} state), but also the private state of all
other threads combined (aka.~\emph{other} state). This subjective
division will suffice to specify and verify the exchanger in Hoare
logic, without relying on CAL, and will be essential for reasoning
about the clients in Section~\ref{sec:cal}.
%
%In Section~\ref{sec:counting}, we show how the approach directly
%scales to naturally specify programs that have heretofore only been
%addressed using quiescent and quantitative quiescent consistency.

The subjective state for each thread in this example consists of three
groups of two components: (1) thread-private heap $\heaps$ of the
thread, and of the environment $\heapo$, (2) a set of outstanding
offers $\perms$ created by the thread, and by the environment
$\permo$, and (3) a time-stamped history of values $\hists$ that the
thread exchanged so far, and dually $\histo$ for the environment.  In
addition to the subjective state, which is private to the specified
thread or to the environment, we need shared (aka.~\emph{joint})
state. The exchanger requires two joint components: a heap $\heapj$
and a map of ``matched offers'' $\pending$.

We next describe the intuition behind all these variables, and how the
proof of the exchanger manipulates them. In the description, we
overload the notation for maps, and write $x \mapsto y$ for a
singleton map, no matter its type. For example, in the case of
time-stamped histories, $t \mapsto b$ denotes that at time-stamp $t$,
some abstract behavior $b$ took place. In the case of heaps, $x
\mapsto v$ for a singleton heap with a pointer $x$ storing value
$v$. We also overload the operator $\hunion$ for disjoint union, and
apply it to histories, heaps, and set of offers, uniformly. Thus, each
history or heap will be a disjoint union of a number of singleton
entries, and we use $x \subseteq y$ to say that the map $x$ is a
submap (or subheap, or subhistory) of $y$.

The behavior of the exchanger on decorated state is as follows. 
%
First, $\heapj$ is a heap that serves as the ``staging'' area for the
offers. It includes the global pointer $g$. Whenever a thread wants to
make an offer, it allocates a pointer $p$ in $\heaps$, and then tries
to move $p$ from $\heaps$ into $\heapj$, simultaneously linking $g$ to
$p$.

Second, $\perms$ and $\permo$ are sets of offers (hence, sets of
pointers) that determine offer ownership. A thread that has the offer
$p \in \perms$ is the one that created it, and thus has the
\emph{sole} right to retire $p$, or to collect the value that $p$ was
matched with. Upon collection or retirement, $p$ is removed from
$\perms$.

%To ensure that only one thread has such a right, one of the important
%state invariants of the exchanger is that $\perms$ and $\permo$ are
%disjoint sets.

Third, $\hists$ and $\histo$ are histories, each mapping a time-stamp
(isomorphic to nats), to a pair of exchanged values. A singleton
history $t \mapsto (v, w)$ symbolizes that a thread having this
singleton as a subcomponent of $\hists$, has exchanged $v$ for $w$ at
time $t$. Histories have been used in FCSL
before~\cite{Sergey-al:ESOP15}, but only on linearizable examples,
where each time-stamp $t$ ``stored'' an atomic operation performed at
time $t$. In the exchanger example, which is not linearizable, a
time-stamp $t$ stores only a \emph{half} of an atomic operation, \ie,
a pair $(v, w)$ signaling that a thread exchanged $v$ for $w$. The
most important invariant of the exchanger is that each such half-entry
is matched by a symmetric half-entry, to capture that another thread
must have simultaneously exchanged $w$ for $v$.

Fourth, $\pending$ is a map storing the offers that were matched, but
not yet collected. A singleton entry in $\pending$ has the form $p
\mapsto (t, v, w)$ and denotes that offer $p$, initially storing $v$,
was matched at time $t$ with $w$. A singleton entry is entered into
$\pending$ when a thread on the one end of matching, matches $v$ with
$w$. Such a thread also places the \emph{twin} entry $\twin{t} \mapsto
(w, v)$, with inverted order of $v$ and $w$, into its own private
history $\hists$, where:
\[
\twin{t} = \left\{%
\begin{array}{ll}
t+1 & \mbox{if $t$ is odd}\\
t-1 & \mbox{if $t > 0$ and $t$ is even}
\end{array}\right.
\]
For technical reasons, $0$ is not a valid time-stamp, and has no
distinct twin. The pending entry for $p$ resides in $\pending$ until
the thread that created the offer $p$ decides to ``collect'' it. It
removes $p$ from $\pending$, and simultaneously adds the entry $t
\mapsto (v, w)$ into its own $\hists$, thereby logically completing
the exchange. Since twin time-stamps are consecutive nats, a history
cannot contain entries \emph{between} twins.

Thus, two twin entries in the combined history, including $\pending$,
jointly represent a single exchange, as if it occurred
\emph{atomically}. CA-histories~\cite{Hemed-Rinetzky:PODC14} capture
this by making the ends of an exchange occur as simultaneous
events. We capture it via twin time-stamps. More formally, let $\hist
= \hists \hunion \histo \hunion \mygather{\pending}$. The exchanger's
main invariant is that $\hist$ always contains matching twin entries:
\[
\tag{\arabic{tags}}\refstepcounter{tags}\label{tag:exchanging} 
t \mapsto (v, w) \subseteq \hist \iff \twin t \mapsto (w, v) \subseteq \hist
\]
Here $\mygather{\pending}$ denotes the collection of all the entries
in $\pending$. That is, $\mygather{\emptyset} = \emptyset$, and
$\mygather {p \mapsto (t, v, w) \hunion \pending'} = t \mapsto (v, w)
\hunion \mygather{\pending'}$.

%One consequence of this property is that the history $\hist$ always
%has an even number of entries. Thus, if $t$ is the smallest unused
%time-stamp in $\hist$ (not counting $0$), then $t$ must be odd, and
%hence $t < \twin t = t + 1$.

\paragraph{Specification.}
The following is the Hoare spec we desire.
%
\[
\tag{\arabic{tags}}\refstepcounter{tags}\label{tag:exchangespec} 
{\small{
\begin{array}{c}
\specK{\{\heaps = \emptyset, \perms = \emptyset, \hists = \emptyset, \gist \subseteq \histo \hunion \mygather{\pending}\}}\\
\esc{exchange}\ v\\
  \spec{\!\!
  \begin{array}{c}
\heaps = \emptyset, \perms = \emptyset, \gist \subseteq
  \histo \hunion \mygather{\pending}, \hbox{}\\[1pt]
\mathsf{if}\ \res\ \mathsf{is}\ \mathsf{Some}\ w\ \mathsf{then}~\\[1pt]
\exists t\ldot \hists = t \mapsto (v, w), \mathsf{last} (\gist) < t, \twin{t}
~\mathsf{else}\ \hists = \emptyset 
  \end{array}
  \!\!}
\end{array}
}}
\]
%
The precondition says that the exchanger starts with the empty private
heap $\heaps$, set of offers $\perms$ and history $\hists$; hence by
framing (see Section~\ref{sec:background} for details), it can start
with any value for these components. The logical variable $\gist$
names (a subset of) the initial history of all threads,
$\histo \hunion \mygather{\pending}$. We use subset instead of
equality in order to make the precondition stable under interference;
that is, stable under other threads adding history entries to $\histo$
or $\pending$.

In the postcondition, the self heap $\heaps$ and the set of offers
$\perms$ didn't change (hence, if $\mathtt{exchange}$ made an offer
during its execution, it also collected or retired it by the end).
%
The history $\gist$ is still a subset of the ending value for $\histo
\hunion \mygather{\pending}$, signifying that the environment history
only grows by the interference. We will make a crucial use of this
part of the spec when verifying a client of the exchanger in
Section~\ref{sec:cal}.
%
If the exchange succeeded (either in line 7 or line 16 in
Figure~\ref{fig:exchanger}), \ie, if the result $\mathsf{res}$ equals
$\mathsf{Some}\ w$, then there exists a time-stamp $t$, such that
self-history $\hists$ contains the entry $t \mapsto (v, w)$,
symbolizing that $v$ and $w$ were exchanged at time $t$.  Observe that
by invariant~(\ref{tag:exchanging}), it must be that the twin entry
$\twin t \mapsto (w, v)$ belongs to $\histo \hunion
\mygather{\pending}$ for the ending state.  Moreover, the exchange
occurred \emph{after} the call to $\mathsf{exchange}$; that is,
whichever subset $\gist$ we chose in the pre-state, both $t$ and
$\twin t$ are larger than then the last time-stamp in $\gist$. If the
exchange fails (\ie, $\mathsf{res}$ is $\mathsf{None}$), then
$\hists$ remains empty.

%\is{Okay, I liked the subsection above, as it seems to cover
%  everything what's needed to state the specification.}
%
%\is{How about moving the rest of this section for after the background
%section? Right now, I think, the reader got enough to be impressed
%about the spec, so let's show him the formal rules of FCSL, and then
%proceed with the rest of the exchanger protocol and its client.}
%
%\an{Hmm, that would make things a bit too schizophrenic, going back
%  and forth all the time. I would like to at least squeeze the proof
%  outline before going to background. So I need to introduce the least
%  amount of info in order to be able to explain the proof outline. It
%  seems, at the very least, I should cover the transitions, if not the
%  spatial invariants.}

\paragraph{Further invariants.}
%
The states in the exchanger state-space must satisfy other invariants
in addition to (\ref{tag:exchanging}). All these properties naturally
arise from our description of how the exchanger behaves on decorated
state. Thus, we simply list them here without much commentary, so that
we can refer to them further on. We abbreviate with $p \mapsto (x; y)$
the heap $p \mapsto x \hunion p\!+\!1 \mapsto y$.

\begin{enumerate}[label=(\roman*)]
\item\label{exP} $\heapj$ contains a pointer $g$ and a number of
  offers $p \mapsto (v; x)$, and $g$ points to either $\mathsf{null}$
  or to some offer in $\heapj$.

\item $\hists$, $\histo$ and $\mygather{\pending}$ contain only
  disjoint time-stamps. Similarly, $\perms$ is disjoint from $\permo$.

\item\label{matched} The offers in $\pending$ are all matched, and all
  owned by some thread:
\[
\!\!\!\!\!\!
{\small{
\exists t\ldot p \mapsto (t, v, w) \subseteq \pending \iff p \in
\perms \hunion \permo, p \mapsto (v; \Matched w) \subseteq \heapj.
}}
\]
\item There is at most one unmatched offer; it is the one linked
  from~$g$. It is owned by someone:
\[
{\small{
p \mapsto (v; \Unmatched) \subseteq \heapj \Longrightarrow p \in \perms \hunion
\permo, g \mapsto p \subseteq \heapj.
}}
\]
\item Retired offers are not owned: $p \mapsto (v; \Retired)
  \subseteq \heapj \Rightarrow p \notin \perms \hunion \permo$.

\item The outstanding offers are included in the joint heap, \ie, if
  $p \in \perms \hunion \permo$ then $p \in \mathsf{dom}\ \heapj$.

\item\label{ex:gapless} The combined history $\hists \hunion \histo \hunion
  \mygather{\pending}$ is gapless, that is, if it contains a
  time-stamp $t$, it also contains all the smaller time-stamps (sans
  0).

\end{enumerate}

%\an{There's nothing I can really say here as to the relevance of the
%  various properties. It may be possible to remove this section later
%  on. Let's see if I need any of these properties when describing the
%  proof outline.}

\begin{comment}
\paragraph{Temporal invariants.}
The proof of $\mathtt{exchange}$ also requires limiting the kind of
interference that the threads can make. In particular, we require that
the each threads can only make the following transitions on the
state-space of the exchanger. We will show in
Section~\ref{sec:background} how those are formally captured in FCSL.
All the transitions correspond to performing at most one atomic
operation on a real state (reading/writing/allocating/deallocating a
pointer), while also allowing for arbitrary changes to the auxiliary
state, or changing of ownership a pointer.
\begin{itemize}
\item $\mathsf{install\_trans}\ p\ v$ takes a pointer $p$ and an offer
  represented as $p \mapsto (v; \Unmatched)$, moves the offer into
  $\heapj$.  It requires that $g \mapsto \mathsf{null}$ in
  $\heapj$. It adds $p$ to $\perms$, and sets $g \mapsto p$. More
  succinctly, the transition changes $[\,\heaps\,{:}\,p \mapsto (v;
    \Unmatched)\,{\mid}\,\perms\,{:}\,\emptyset\,{\mid}\,\heapj\,{:}\,g
    \mapsto \mathsf{null}\,]$ into
  $[\,\heaps\,{:}\,\emptyset\,{\mid}\,\perms\,{:}\,\{p\}\,{\mid}\,\heapj\,{:}\,g
    \mapsto p \hunion p \mapsto (v; \Unmatched)\,]$, allowing that by
  framing each of these components can be enlarged by a value that
  remains invariant upon transition.
%$\mathsf{match\_trans}\ v\ w$ 
%starts in a heap $\heapj$ in which $g$ points to some offer $p$, such
%that $p \mapsto (v, \Unmatched)$ is also in $\heapj$. It atomically
%changes the contents of $p$ to $(v, \Matched w)$. It also modifies the
%auxiliary state as follows. It adds the history entry $t \mapsto (w,
%v)$ to $\hists$, where $t$ is the next unused timestamp in $\hists
%\hunion \histo \hunion \mygather(\pending)$. It also places the entry
%$p \mapsto (t+1 , v, w)$ into $\pending$.
\item $\mathsf{match\_trans}\ v\ w$ encodes matching of the offer
  pointed to by $g$ with $w$, and the appropriate changes to the
  auxiliary state. It changes $[\,\heapj\,{:}\,g \mapsto p \hunion p
    \mapsto (v; \Unmatched)\,{\mid}\,\hists\,{:}\,\emptyset\,{\mid}\,\pending\,{:}\,\emptyset\,]$
  to $[\,\heapj\,{:}\,g \mapsto p \hunion p \mapsto (v; \Matched
    w)\,{\mid}\, \hists\,{:}\,t \mapsto (w, v)\,{\mid}\,
    \pending\,{:}\,p \mapsto (t+1, v, w)\,]$. Here $t$ is the next
  unused time-stamp in $\hists \hunion \histo \hunion
  \mygather{\pending}$. Thus, a history entry $t \mapsto (w, v)$ is
  placed into $\hists$, and $p \mapsto (t+1, v, w)$ is placed into
  $\pending$, to be collected later by the thread that made the offer
  $p$.
\item $\mathsf{retire\_trans}\ p\ v$ modifies $[\,\heapj\,{:}\,p
  \mapsto (v; \Unmatched)\,{\mid}\,\perms\,{:}\,\{p\}\,]$ into
  $[\,\heapj\,{:}\,p \mapsto (v; \Retired)\,{\mid}\,\perms\,{:}\,\emptyset\,]$. Thus, an offer $p$ is
  retired, but it can only be done so by a thread that installed it,
  \ie, a thread that contains $p$ in $\perms$. Upon retirement, $p$
  is removed from $\perms$.
\item $\mathsf{collect\_trans}\ p$ modifies
  $[\,\perms\,{:}\,\{p\}\,{\mid}\,\pending\,{:}\,p \mapsto (t, v,
  w)\,{\mid}\,\hists\,{:}\,\emptyset\,]$ to
  $[\,\perms\,{:}\,\emptyset\,{\mid}\,\pending\,{:}\,\emptyset\,{\mid}\,\hists\,{:}\,t
  \mapsto (v, w)\,]$. Thus, an entry left in $\pending$ by a thread
  which achieved matching is collected and moved to the self history
  $\hists$ of a thread that originally installed the offer $p$.
\item $\mathsf{unlink\_trans}$ unlinks the current offer from $g$. It
  changes $[\,\heapj\,{:}\,g \mapsto p \hunion p \mapsto (v; x)\,]$ to
  $[\heapj\,{:}\,g \mapsto \mathsf{null} \hunion p \mapsto (v; x)\,]$,
  but only if $x \neq \Unmatched$. Unmatched offers can't be unlinked.

%\item $\mathsf{match\_trans}\ v\ w$ starts in a heap $\heapj$ in which
%  $g$ points to some offer $p$, such that $p \mapsto (v, \Unmatched)$
%  is also in $\heapj$. It atomically changes the contents of $p$ to $(v,
%  \Matched w)$. It also modifies the auxiliary state as follows. It
%  adds the history entry $t \mapsto (w, v)$ to $\hists$, where $t$ is
%  the next unused timestamp in $\hists \hunion \histo \hunion
%  \mygather(\pending)$. It also places the entry $p \mapsto (t+1 , v,
%  w)$ into $\pending$.
%
%\item $\mathsf{retire\_trans}\ p\ v$ starts in a heap $\heapj$ in which
%  $p \mapsto (v, \Unmatched)$. Moreover, $p \in \perms$, \ie, a thread
%  that wants to perform the transition has a permission to do so; it
%  has $p$ in its self set of offers, so it is the thread that
%  installed the offer in the first place. The transition removes $p$
%  from $\perms$ and changes $p \mapsto (v, \Retired)$.
%
%\item $\mathsf{collect\_trans}\ p$ starts in a state in which $p \in
%  \perms$ as well as $p \in dom (\pending)$. By invariant (blah) from
%  above, the second propery means that $p$ has been matched by some
%  thread, and thus $p \mapsto (t, v, w) \in \pending$, where $v$ was
%  the original offer. The transition removes $p$ from $\perms$ and
%  $\pending$, and adds the history entry $t \mapsto (v, w)$ to
%  $\hists$.
%
%\item $\mathsf{unlink\_trans}$ starts in a state in which exists $p$,
%  $v$, $x$, such that $g \mapsto p$ and $p \mapsto (v, x)$ are in
%  $\heapj$. Moreover, $x$ is either $\Matched w$ or
%  $\Retired$, but not $\Unmatched$. Hence, unmatched
%  offers can't be unlinked. It simply sets $g \mapsto null$, without
%  changing any auxiliary state. 
\end{itemize}
 
%\paragraph{Atomic actions.}
%
%Atomic actions are primitive programs that are obtained as ``linear
%combinations'' of transitions. They generalize the atomic concurrency
%primitives from the original program, with auxiliary code.  Here we
%present the 5 primitive actions that we use in the exchanger via their
%Hoare logic specs. We omit the proofs that the actions actually
%satisfy the provided Hoare specs, and that the specs are stable under
%interference of other threads that respect the transitions listed
%above. These could be found in our Coq sources.
%
%The first action is $\mathsf{tryinstall}$, with the following spec. 
%%\[
%%\begin{array}{c}
%%\{\heaps = p \mapsto (v, \Unmatched) \aand \perms = \emptyset \aand \hists = \emptyset \aand
%%        \gist \subseteq \histo \hunion \mygather{\pending}\}\\
%%\mathsf{tryinstall}\\
%%\{\hists = \emptyset \aand \gist \subseteq \histo \hunion \mygather(\pending) \aand \hbox{} \\
%%  \mathsf{if}\ \res\ \mathsf{then}\ \heaps = \emptyset \aand \perms = \{p\} \aand
%%         \exists x\ldot p \mapsto (v, x) \in \heapj \aand \mathsf{bounded} p (\mathsf{last}\ \gist) (\pending) \\
%%\mathsf{else} \heaps = p \mapsto (v, \Unmatched) \aand \perms = \emptyset\}
%%\end{array}
%%\]
%
%Intuitively, when the auxiliary state is erased, $\mathsf{tryinstall}$
%corresponds to the \code{CAS} operation from line (blah) in Figure (blah). In
%the presence of auxiliary state, $\mathsf{tryinstall}$ either takes
%the $\mathsf{install\_transition}$ (corresponds to the situation when
%the \code{CAS} succeeds), or an idle transition (corresponds to the sitaution
%when the \code{CAS} fails). The postcondition says that under any
%interference, there will always be an entry $x$, such that $p \mapsto
%(v, x)$. $x$ may initially $\Unmatched$, but may later be
%moved to $\Matched w$, if some environment thread matches the
%offer $p$. At any rate, is such a match is achieved, and an entry $p
%\mapsto (t, v, w)$ stored into $\pending$, the timestamps $t$ and
%$\twin t$ will be larger than $\mathsf{last}$ timestamp in $\gist$.
%

\end{comment}


{
%\setlength{\belowcaptionskip}{-10pt} 
\begin{figure}
\centering
\[
{\small{
\begin{array}{rl}
 \Num{1} & \specK{\{\heaps = \emptyset, \perms = \emptyset, \hists = \emptyset, \gist \subseteq \histo \hunion \mygather{\pending} \}}
\\ 
 \Num{2} & ~~~~ p \Asgn \esc{alloc}(v, \Unmatched);\\
 \Num{3} & \specK{\{\heaps = p \mapsto (v; \Unmatched), \perms = \emptyset, \hists = \emptyset, \gist \subseteq \histo \hunion \mygather{\pending} \}}\\
 \Num{4} & ~~~~ b \Asgn \esc{CAS}(g, \esc{null}, p);\\
 \Num{5} & ~~~~ \kw{if}~~b~\esc{==}~\esc{null}~~\kw{then}\\
 \Num{6} & ~~~~ ~~~~ \esc{sleep}(50);\\
 \Num{7} & \specK{\{\heaps = \emptyset, \perms = \{p\}, \hists = \emptyset, \gist \subseteq \histo \hunion \mygather{\pending}, \esc{bounded}\ p\ v\ \gist \}}\\
 \Num{8} & ~~~~ ~~~~ x \Asgn \esc{CAS}(p\esc{+}1, \Unmatched, \Retired);\\
 \Num{9} & \specK{\{\heaps = \emptyset, \perms = \emptyset, \gist \subseteq \histo \hunion \mygather{\pending},\hbox{}} \\
\Num{10} & \specK{\hphantom{\}}x = \Matched w \implies \exists t\ldot \hists = t \mapsto (v, w), \mathsf{last}(\gist) < t, \twin t,}\\
\Num{11} & \specK{\hphantom{\}}x = \Unmatched \implies \hists = \emptyset \}}\\
\Num{12} & ~~~~ ~~~~ \kw{if}~~x~~\kw{is}~~\Matched w~~\kw{then}~~\kw{return}~~(\esc{Some}~w)\\
\Num{13} & ~~~~ ~~~~ \kw{else}~~\kw{return}~~\esc{None}\\
\Num{14} & ~~~~ \kw{else}\\
\Num{15} & \specK{\{\heaps = p \mapsto (v; \Unmatched), \perms = \emptyset, \hists = \emptyset, \gist \subseteq \histo \hunion \mygather{\pending}\}}\\
\Num{16} & ~~~~ ~~~~ \esc{dealloc}~~p;\\
\Num{17} & \specK{\{\heaps = \emptyset, \perms = \emptyset, \hists = \emptyset, \gist \subseteq \histo \hunion \mygather{\pending}\}}\\
\Num{18} & ~~~~ ~~~~ cur \Asgn !g;\\
\Num{19} & \specK{\{\heaps = \emptyset, \perms = \emptyset, \hists = \emptyset, \gist \subseteq \histo \hunion \mygather{\pending},}\\
\Num{20} & \specK{\hphantom{\}}cur = \esc{null} \vee cur \mapsto (w; -) \subseteq \heapj\}}\\
\Num{21} & ~~~~ ~~~~ \kw{if}~~cur~\esc{==}~\esc{null}~~\kw{then}~~\kw{return}~{\esc{None}}\\
%& \specK{\{ \mbox{obvious} \}}\\
\Num{22} & ~~~~ ~~~~ \kw{else}\\
\Num{23} & \specK{\{\heaps = \emptyset, \perms = \emptyset, \hists = \emptyset, \gist \subseteq \histo \hunion \mygather{\pending},}\\
\Num{24} & \specK{\hphantom{\}} cur \mapsto (w; -) \subseteq \heapj\}}\\
\Num{25} & ~~~~ ~~~~ ~~~~ x \Asgn \esc{CAS}(cur\esc{+}1, \Unmatched, \Matched v);\\
\Num{26} & \specK{\{\heaps = \emptyset, \perms = \emptyset, \gist \subseteq \histo \hunion \mygather{\pending}, cur \mapsto (w; y) \subseteq \heapj,} \\
\Num{27} & \specK{\hphantom{\}}x = \Unmatched \implies y = \Matched v, \exists t\ldot \hists = t \mapsto (v, w), \mathsf{last}(\gist) < t, \twin t}\\
\Num{28} & \specK{\hphantom{\}}x \neq \Unmatched \implies \hists = \emptyset, y \neq \Unmatched \}}\\
\Num{29} & ~~~~ ~~~~ ~~~~ \esc{CAS}(g, cur, \esc{null});\\
\Num{30} & \specK{\{\mbox{same as above; the state satisfies (iv) because $y \neq \Unmatched$}\}}\\
%& \specK{\{\heaps = \emptyset, \perms = \emptyset, \gist \subseteq \histo \hunion \mygather{\pending}, 
%    \exists w hw. cur \mapsto (w, hw) \in \heapj,} \\
%& \specK{x = \Unmatched, hw = \Matched v, \exists t. \hists = t \mapsto (w, v), last \gist < t, \twin t}\\
%& \specK{x \neq \Unmatched, \hists = \emptyset, hw \neq \Unmatched \}}\\
\Num{31} & ~~~~ ~~~~ ~~~~ \kw{if}~~x~\esc{==}~\Unmatched~~\kw{then}~~w\Asgn !cur;\kw{return}~(\esc{Some}\ w)\\
\Num{32} & \specK{\{ \heaps = \emptyset, \perms = \emptyset, \gist \subseteq \histo \hunion \mygather{\pending}, \res=\esc{Some}\ w}, \\
\Num{33} & \specK{\hphantom{\}}\exists t. \hists = t \mapsto (w, v), \mathsf{last}(\gist) < t, \twin t\}}\\
\Num{34} & ~~~~ ~~~~ ~~~~ \kw{else}~~\kw{return}~\esc{None}\}\\
\Num{35} & \specK{\{\heaps = \emptyset, \perms = \emptyset, \hists = \emptyset, \gist \subseteq \histo \hunion \mygather{\pending}, \res=\esc{None}\}} 
\end{array}
}}
\]
%\vspace{-10pt}  
\caption{Proof outline for the exchanger.}
\label{fig:exchanger_proof}
\end{figure} 
}

\subsection{Proof outline for the exchanger}
\label{sec:exproof}

The proof outline for the spec~\eqref{tag:exchangespec} is given in
Figure~\ref{fig:exchanger_proof}, and we proceed to explain it.
%
We start with the desired precondition, and after allocation in line
2, $\heaps$ stores the offer $p$ in line 3.

If the \code{CAS} at line 4 succeeds, the program ``installs'' the
offer; that is, the state (real and auxiliary) is changed
simultaneously to the modification of $g$. In particular, $p$ is added
to $\perms$, and the offer $p$ changes ownership, to move from
$\heaps$ to $\heapj$.  Since $b$ will be bound to $\mathsf{null}$,
this leads us to the assertion in line 7. We explain in
Section~\ref{sec:background} how these kinds of changes to the
auxiliary state, which are supposed to occur simultaneously with some
atomic operation (in this case, \code{CAS}), are specified and
verified in FCSL. The assertion in line 7 further states
$\mathsf{bounded}\ p\ v\ \gist$. We do not formally define
$\mathsf{bounded}$ here (it is in the Coq scripts, accompanying the
paper), but it says that $p$ has been moved to $\heapj$, \ie,
$p \mapsto (v; -) \subseteq \heapj$, and that any time-stamp $t$ at
which another thread may match $p$, and thus place the entry
$p \mapsto (t, v,-)$ into $\pending$, must satisfy
$\mathsf{last}(\gist) < t, \twin{t}$. Intuitively, this property is
valid, and stable under interference, because entries in $\pending$
can be added only by generating fresh time-stamps wrt.~the collective
history $\histo \hunion \mygather{\pending}$, and $\gist$ is a subset
of it.
%
If the \code{CAS} in line 4 fails, then nothing changes, and we simply move
to the spec in line 15.

At line 8, the \code{CAS} succeeds if $x\,{=}\,\Unmatched$, and fails if
$x\,{=}\,\Matched w$. Notice that $x$ cannot be $\Retired$; since we
own $p \in \perms$, no other thread could retire $p$.
%
If \code{CAS} fails, then the offer has been matched with~$w$. \code{CAS}
simultaneously ``collects'' the offer as follows. By invariant (iii),
and $\mathsf{bounded}\ p\ v\ \gist$, the auxiliary map $\pending$
contains an entry $p \mapsto (t, v, w)$, where $\mathsf{last}(\gist) <
t, \twin{t}$. The auxiliary state is changed to remove $p$ from
$\pending$, and simultaneously place $t \mapsto (v, w)$ into $\hists$.
%
If \code{CAS} succeeds, the offer was unmatched, and is ``retired'' by
removing $p$ from $\perms$.

Lines 12 and 13 branch on $x$, and select either the assertion 10 or
11. Either way, the desired postcondition directly follows.

%Deallocation in line 16 removes the offer $p$ from the heap $\heaps$.

After reading $cur$ in line 18, by invariant (i), we know that $cur$
either points to $\mathsf{null}$, or to some offer $p \mapsto (w; -)
\subseteq \heapj$.

%The null-check in line 21, together with line 19, directly establishes
%the postcondition.

At line 25, the \code{CAS} succeeds if $x = \Unmatched$ and fails
otherwise. If \code{CAS} succeeded, then it ``matches'' the offer in $cur$;
that is, it writes $\Matched w$ into the hole of $cur$, and changes the
auxiliary state as follows. It takes $t$ to be the smallest unused
time-stamp in the history $\hist = \hists \hunion \histo \hunion
\mygather{\pending}$. Thus $\mathsf{last}(\hist) < t$, and because
$\hist$ has even size by invariant~(\ref{tag:exchanging}), $t$ must be
odd, and hence $t < \twin t = t+1$. The $t \mapsto (v, w)$ is placed
into $\hists$, giving us assertion 27.  To preserve the invariant
(iii), \code{CAS} simultaneously puts the entry $p \mapsto (t, w, v)$ into
$\pending$, for future collection by the thread that introduced offer
$cur$. But, we do not need to reflect this in line 27.
%
If the \code{CAS} fails, the history $\hists$ remains empty, as no matching is
done. However, the hole $y$ associated with $cur$ cannot be
$\Unmatched$, as then \code{CAS} would have succeded.
%
Therefore, it is sound in line 29 to ``unlink'' $cur$ from $g$, as the
unlinking will not violate the invariant (iv), which says that an
unmatched offer must be pointed to by $g$.
%
Finally, lines 31 and 34 select the assertion 27 or 28, and either
way, directly imply the desired postcondition.



