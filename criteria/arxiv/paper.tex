\documentclass[preprint,9pt,nocopyrightspace]{sigplanconf}

\usepackage{amssymb,amsmath,amsthm}
\usepackage{latexsym}
\usepackage{graphicx}
\usepackage[usenames,dvipsnames]{color}
\usepackage{listings}
\usepackage{float}
\usepackage{multirow}
\usepackage[scaled]{helvet}
\usepackage[noend]{algorithmic}
\usepackage{mathrsfs}
\usepackage{mathpartir}
\usepackage{dsfont}
\usepackage{stmaryrd}
\usepackage{url}
\usepackage{textcomp} 
\usepackage[colorlinks=true,allcolors=blue,breaklinks,draft=false]{hyperref}
\usepackage{titlesec}
\usepackage{parskip}
\usepackage{alltt}
\usepackage{bbm}
\usepackage{alltt}
\usepackage{verbdef}
\usepackage{xspace}
\usepackage{verbatim}
\usepackage{enumitem}
\usepackage{lipsum}
\usepackage{wrapfig}
%\usepackage[strut=on,labelfont=bf]{caption}
\usepackage[usenames,dvipsnames]{xcolor}
\hypersetup{linkcolor=black,citecolor=black,urlcolor=RubineRed}

\newcounter{tags}


% \usepackage{natbib}
% \bibpunct();A{},
% \let\cite=\citep

% For turned column headers 
\usepackage{adjustbox} 
\usepackage{array}
\usepackage{booktabs}
\usepackage{multirow}
\usepackage{pifont}
 
% remarks
\newcommand{\todo}[1]{\textcolor{red}{({#1})}}
\newcommand{\is}[1]{\textcolor{blue}{(Ilya: {#1})}}
\newcommand{\an}[1]{\textcolor{red}{(Aleks: {#1})}}

% useful macros
\newcommand{\asgn}{\leftarrow}
\newcommand{\code}[1]{\lstinline{#1}}
\newcommand{\ccode}[1]{\code{#1}}
\newcommand{\var}[1]{\({#1}\)} 
\newcommand{\num}[1]{\({\text{{\scriptsize{#1}}}}\)}
\newcommand{\etc}{\emph{etc}}
\newcommand{\ie}{\emph{i.e.}\xspace}
\newcommand{\Ie}{\emph{I.e.}\xspace}
\newcommand{\eg}{\emph{e.g.}\xspace}
\newcommand{\Eg}{\emph{E.g.}\xspace}
\newcommand{\vs}{\emph{vs.}\xspace}
\newcommand{\etal}{\emph{et~al.}\xspace}
\newcommand{\adhoc}{\emph{ad hoc}\xspace}
\newcommand{\viz}{\emph{viz.}\xspace}
\newcommand{\dom}[1]{\mathsf{dom}(#1)}
\newcommand{\aka}{\textit{a.k.a.}\xspace}
\newcommand{\cf}{\textit{cf.}\xspace}
\newcommand{\wrt}{\emph{wrt.}\xspace}
\newcommand{\loef}{L\"{o}f}
\newcommand{\sep}{\textasteriskcentered}
\newcommand{\res}{\mathsf{res}} 
\newcommand{\ret}{\mathsf{ret}} 
\newcommand{\fix}{\mathsf{fix}} 

%specs
\newcommand{\specK}[1]{\ensuremath{\textcolor{blue}{#1}}}
\newcommand{\spec}[1]{\specK{\left\{{#1}\right\}}}
\newcommand{\specQ}[4]{[#1 , #2 , #3] \, #4}%\specQ{mL}{gL}{gE}{p}
\newcommand{\drspec}[1]{\specK{\langle{#1}\rangle}}
\newcommand{\sspec}[1]{\specK{\{{#1}\}}}




% Keep footnotes on one page
\interfootnotelinepenalty=10000 

\setlist[itemize]{leftmargin=*}
\setlength{\parindent}{0.15in}
\setlength{\topsep}{0cm}
\setlength{\parskip}{0pt}

\titlespacing*{\section}{0pt}{*1.5}{*1.5} 
\titlespacing*{\subsection}{0pt}{*0.8}{*0.5}
\titlespacing*{\subsubsection}{0pt}{*0.8}{*0.5}
\titlespacing*{\paragraph}{0pt}{*0.5}{*1.2}

% SSReflect listings 
\input{lstcoq}
\lstset{style=Coq}

% Hyphenation
\hyphenation{Veri-Fast}

% Bibtgex tweaks
\setcitestyle{square}
\defcitealias{Coq-manual}{Coq proof assistant}

\begin{document}

%\special{papersize=8.5in,11in}


\authorinfo{Ilya Sergey \and
  Aleksandar Nanevski \and 
  Anindya Banerjee \and 
  Germ\'{a}n Andr\'{e}s Delbianco}
{IMDEA Software Institute, Madrid, Spain}
{
\{ilya.sergey, aleks.nanevski, anindya.banerjee,
  german.delbianco\}@imdea.org 
}

% \vspace{-20pt}}



% \authorinfo{}{}{}

% \conferenceinfo{PLDI~'15} {June 13--17, 2015, Portland, OR, USA}
% \CopyrightYear{2015}
% \copyrightdata{TODO}
% \doi{TODO}


%\title{The Power of Subjectivity}

\title{
  Hoare-style Specifications as Correctness Conditions\\
  for Non-linearizable Concurrent Objects
% 
%\vspace{-42pt}
}
 

\maketitle

\begin{abstract}

%
% 1. The setting
%

  Designing scalable concurrent objects, which can be efficiently used
  on multicore processors, often requires one to abandon standard
  specification techniques, such as \emph{linearizability}, in favor
  of more relaxed consistency requirements.
  %
  However, the variety of alternative correctness conditions makes it
  difficult to choose which one to employ in a particular case, and to
  compose them when using objects whose behaviors are specified via
  different criteria. The lack of syntactic verification methods for
  most of these criteria poses challenges in their systematic adoption
  and application.

  % 
  % To specify different useful classes of non-linearizable concurrent
  % objects, a number of alternative correctness criteria have been
  % proposed recently: \emph{Quiescent} and \emph{Quantitative
  % Quiescent Consistency}, \emph{Concurrency-Aware Linearizability},
  % and \emph{Quasi-Linearizability}, to name just a~few.
%
%
% 2. The problem(s): too many of correctness criteria
%
  % However, it is not trivial to design new concurrent correctness
  % conditions, which are compositional and general enough to
  % accommodate highly parallel concurrent objects.
  % 
  % It is even less trivial to supply these conditions with
  % compositional proof methods, making them amenable for structured
  % formal program verification.
  %
  % Finally, the diversity of existing specification approaches makes it
  % tricky to reason uniformly about client code that employs various
  % concurrent objects, whose semantics is defined via different
  % correctness criteria.

  % Furthermore, the clients of a concurrent data structure, when
  % committing to one or another correctness criterion, need to adopt
  % specific reasoning principles necessary to characterize the
  % structure's behavior. Therefore, client-side reasoning about
  % programs that incorporate several structures, specified via
  % \emph{different} criteria, might be tricky.
%
% 3. Our solution
%  

  In this paper, we argue for using Hoare-style program logics as an
  alternative and uniform approach for specification and compositional
  formal verification of safety properties for concurrent objects and
  their client programs.
  %
  Through a series of case studies, we demonstrate how an existing
  program logic for concurrency can be employed off-the-shelf to
  capture important state and history invariants, allowing one to
  \emph{explicitly quantify over interference} of environment threads
  and provide intuitive and expressive Hoare-style specifications for
  several \emph{non-linearizable} concurrent objects that were
  previously specified only via dedicated correctness criteria.
  %
  We illustrate the adequacy of our specifications by verifying a
  number of concurrent client scenarios, that make use of the
  previously specified concurrent objects, capturing the essence of
  such correctness conditions as \emph{concurrency-aware
    linearizability}, \emph{quiescent}, and \emph{quantitative
    quiescent consistency}.
  %
  % The compositionality of our approach follows from the
  % compositionality of the logical specifications, that abstract away
  % the implementation details and enable program substitution.
  % 
  % Using an existing logic off-the-shelf provides us with a general
  % structural verification method.
  %
  All examples described in this paper are verified mechanically
  in~Coq.

% Points to emphasize:

% \begin{itemize}

% \item Algorithms of multicore era: better scalability requires
%   rethinkin correctness

% \item Multiple ways to define correctness in some relaxed flavours, by
%   relating it to the sequential executions - name a few correctness
%   conditions

% \item These conditions are difficult to design and they are not even
%   supplied with the corresponding proof methods, which are easy
%   amenable for mechanization

% \item Furthermore, clients need to adapt their reasoning principles to
%   each of this criteria, and reasoning about programs that incorporate
%   structures with several criteria might be tricky

% \item We suggest program logic as a uniform framework for specifying
%   safety  aspects of concurrent objects 

% \item We illustrate the generality of the approach by providing
%   intuitive specifications to two concurrent objects, each of which
%   required a dedicated correctness condition to be specified

% \item Out of provided specifications, we verify a series of concurrent
%   client programs, which make use of the above described concurrent
%   objects

% \item All examples are verified mechanically in Coq

% \end{itemize}

  % \todo{Rephrase the abstract as follows: (1) keep the first paragraph
  %   (2) Outline the situation with the variety of alternative criteria
  %   and implications of this fact; (3) say, which alternative (Hoare
  %   logic) we consider and what benefits it brings; (4) mention the
  %   key observation that we need quantify over interference in order
  %   to provide concurrent specs for all these interesting concurrent
  %   objects. }

\end{abstract}

%\section{Introduction}
\label{sec:introduction}

Formally defining the behavior of highly parallel concurrent objects
is a fundamental challenge, which requires the program designer to
find a trade-off between the desired performance on a multicore
processor, possibly enabled by reduced contention, and the safety
guarantees, implied by the chosen correctness conditions.
 
Traditionally, the correctness of concurrent objects is defined using
\emph{event histories},\footnote{Here, \emph{event} stands for a call
  to or a return from an object's method.}  by (\emph{i}) providing a
\emph{specification set}, describing all ``basic'' behaviors that the
object's client might observe when using it, and (\emph{ii}) fixing a
\emph{consistency condition} that relates the concurrently observable
behaviors to the ones in the specification set.
% 
In the majority of the cases, the specification set is taken to be the
histories of the concurrent object's \emph{sequential} behaviors, in
which the calls are immediately followed by the corresponding returns.
 
The most well-known concurrent correctness condition is
\emph{linearizability}~\cite{Herlihy-Wing:TOPLAS90}, which requires
for each concurrent history of an object to exist a mapping to a
sequential history, such that the ordering of two operations, defined
as matching call/return pairs, is preserved if they are performed by
the same thread, or if they do not overlap.
% 
In the twenty five years since it invention, linearizability has been
shown to be remarkably scalable as a correctness condition, widely
applicable to capture the behavior of implementations of multiple
concurrent objects with intuitive sequential specifications, such as
stacks, queues, sets, locks, snapshots and their combinations, and
even suitable for automatic synthesis of some concurrent
objects~\cite{Vechev-Yahav:PLDI08}. Thanks to the compositional proof
method, based on \emph{linearization points}, proofs of
linearizability in most of the cases are amenable for practical
computer-aided
verification~\cite{Burckhardt-al:PLDI10,Derrick-al:TOPLAS11,Vafeiadis:CAV10,Amit-al:CAV07,Shacham-al:OOPSLA11}.
%
Furthermore, linearizability with respect to a sequential
specification implies observational
refinement~\cite{Filipovic-al:TCS10,Emmi-al:PLDI15}, which allows one
to verify the client code of linearizable concurrent objects
efficiently by replacing the actual object implementation with its
\emph{atomic} counterpart.

However, in order to be linearizable with respect to a sequential
specification, an implementation of a concurrent object should
inherently incorporate certain costly synchronization
primitives~\cite{Attiya-al:POPL11}, which then become points of high
contention and prevent efficient
parallelization~\cite{Shavit:CACM11}. The way to remedy this situation
for enabling better scalability is to change the semantics of a
concurrent object, \ie, the correctness criterion, redefining its
admissible behaviors.

The outlined \emph{history-based} approach for concurrency
specification allows one to relax the semantics of a concurrent object
by either changing the specification set to include other histories
than just the sequential ones, or by changing the consistency
condition, typically quantifying over the possible interference
patterns, which can be observed while using the object. The first
option is taken, for example, in the
works~\cite{Hemed-Rinetzky:PODC14,Henzinger-al:POPL13}, resulting in
such correctness-defining frameworks as \emph{concurrency-aware
  linearizability} and \emph{quantitative relaxation},
correspondingly. The second option is exemplified by such concurrency
correctness criteria as \emph{quiescent
  consistency}~\cite{Aspnes-al:JACM94,Derrick-al:FM14},
\emph{quantitative quiescent
  consistency}~\cite{Jagadeesan-Riely:ICALP14},
\emph{quasi-linearizability}~\cite{Afek-al:OPODIS10}, and \emph{local
  linearizability}~\cite{Haas-al-local15}.

Adopting a new correctness condition for concurrent objects comes with
a price. First, in order to make the newly developed criterion
\textbf{(a)} \emph{scalable}, one has to prove it to be compositional
with respect to combination of multiple concurrent data
structures. Second, in order to make it \textbf{(b)} \emph{practical},
one needs to supply it with a principled proof technique, enabling
efficient correctness checking. Finally, to make it \textbf{(c)}
\emph{relevant for the program verification at large}, one has to
devise a method for exploiting the provided safety guarantees for the
sake of reasoning about client code that uses the concurrent object.
%
While compositionality was formally established for most of the listed
above
criteria~\cite{Aspnes-al:JACM94,Jagadeesan-Riely:ICALP14,Afek-al:OPODIS10,Haas-al-local15},
only few of them were equipped with proof
techniques~\cite{Derrick-al:FM14,Zhang-al:ASE13}, and we are not aware
of any correctness conditions other than linearizability being
employed for client-side reasoning. This state of affairs makes us
pose the following research question:

\vspace{4pt}
\noindent
\textbf{Q:} \emph{Is there a generic specification approach that
  allows one to formally define adequate correctness conditions for
  arbitrary concurrent objects, immediately enjoying the properties
  \emph{\textbf{(a)}--\textbf{(c)}}}?
%
\vspace{4pt}

\noindent
We seek an answer to this question by exploring the opportunities,
opened by the recent advances in the \emph{state-based} approach for
concurrency verification using \emph{Hoare-style program logics}.

In modern program logics for
concurrency~\cite{Feng-al:ESOP07,Vafeiadis-Parkinson:CONCUR07,Feng:POPL09,DinsdaleYoung-al:ECOOP10,Nanevski-al:ESOP14,Svendsen-Birkedal:ESOP14,ArrozPincho-al:ECOOP14,Jung-al:POPL15,Raad-al:ESOP15,Fu-al:CONCUR10},
specifications (or \emph{specs}) for methods of concurrent objects
(and programs in general) are represented by \emph{Hoare tuples}
$\spec{P}~e~\spec{Q}@\ucon$, where $e$ is a program, $P$ is a
precondition that constrains a state in which the program is safe to
run, and $Q$ is a postcondition, describing a state upon the program's
termination. The last component of the tuple, $\ucon$, whose exact
syntax is different for various program logics, is typically referred
to as a \emph{concurrent protocol} or \emph{concurrent resource}.
%
% \an{Why not call $U$ a \emph{resource}, instead of concurrent
%   protocol.  Protocol is the name that has been pushed by CAP, but
%   resource is what everyone \emph{before} such as Brookes and O'Hearn,
%   have used for the concept.}
% 
% \is{Okay, I added the concurrent resource as an alternative. I don't
%   think it's important which name has appeared earlier, but in my
%   opion the world ``protocol" better explains the purpose of the
%   concept, whereas ``resource'' can be confused with the actual state,
%   e.g., heap.}
%
It defines the invariants of the shared state, that are respected by
all threads working with it
concurrently~\cite{Nanevski-al:ESOP14,OHearn:TCS07} and/or the allowed
state changes that the threads can make depending on the roles they
take in the protocol~\cite{Jones:TOPLAS83}.
%
To ensure thread-locality of the concurrent Hoare-style specs, the
assertions $P$ and $Q$, should be proven \emph{stable} with respect to
the concurrent protocol, \ie, they should be invariant under possible
changes that interfering environment threads can make to the state
according to $\ucon$.

Program logics provide a naturally compositional way to specify
concurrency: once a concurrent object is verified against a suitable
spec, its code is not required to be re-examined ever
again. Therefore, specification and verification of the client
programs (including other data structures that use the object) can be
performed out of the object's spec.
%
Program logics were used with a great success to specify and verify
such concurrent data structures and algorithms as
barriers~\cite{Dodds-al:POPL11,Hobor-Gherghina:ESOP11},
indices~\cite{ArrozPincho-al:OOPSLA11}, flat
combiner~\cite{Turon-al:ICFP13,Sergey-al:ESOP15}, shared graph
manipulation~\cite{Raad-al:ESOP15,Sergey-al:PLDI15}, as well as their
multiple client programs.
%
Program verification in such logics is done structurally, \ie, by
systematically applying syntax-directed inference rules, until the
spec is proven, and by now several mechanized tools for logic-based
reasoning about concurrency has been
released~\cite{Sergey-al:PLDI15,Appel-al:BOOK14}.

That is, the program logics-based approach seems like an answer to the
question we posed above, as logical specs satisfy the properties
\textbf{(a)}--\textbf{(c)}. But in order to fully adopt it, we first
need to figure out how to express the existing patterns of
\emph{history-based} reasoning in terms of \emph{state-based} logical
specs.
%
Several attempts to do so have been taken
recently~\cite{Fu-al:CONCUR10,Gotsman-al:ESOP13,Sergey-al:ESOP15,OHearn-al:PODC10},
however, all these works were focusing exclusively on specifying
\emph{linearizable} concurrent objects, thus, making us to rephrase
the question \textbf{Q} as follows:

\vspace{4pt}
\noindent
\textbf{Q$'$:} \emph{Can we employ existing program logics for
  specifying \textbf{non-linearizable} concurrent objects and
  reasoning about their clients?}
% 
\vspace{4pt}

\vspace{-8pt}
\noindent
In this paper we answer to this question affirmatively.

\subsection{Logic-based specifications for concurrent objects}
\label{sec:logic-based-spec}

We build our solution on the recently made observation that histories,
describing relevant atomic changes in the logical state of a
concurrent object, can be expressed as an instance of \emph{auxiliary
  state}~\cite{Sergey-al:ESOP15} (a generalization of auxiliary, or
\emph{ghost}, variables~\cite{Owicki-Gries:CACM76}, customary in
logic-based concurrency verification).
%
That is, reasoning about histories follows exactly the same patterns
that reasoning about \emph{heaps} follows in separation
logic~\cite{Reynolds:LICS02}.
%
We then use the expressive power of concurrent protocols to leverage
this observation for specifying and verifying non-linearizable
concurrent objects.  
%
% \an{Throw in the word "separation logic" somewhere here, to make the
%   point of our observation more precise. Strictly speaking, that you
%   can have histories as ghosts was an old observation (they even call
%   them "history variables" in the model checking works). Our
%   observation was that \emph{separation logic} can natually and
%   compositionally handle history variables, using the algebraic
%   structure of PCMs.}
%
% \is{okay, done}
%

In particular, we show how to define invariants of a concurrent object
in a way that they constrain the real and auxiliary state, capturing a
suitable specification set of histories (\cf (\emph{i})), \eg,
\emph{concurrency-aware} one.
%
The freedom to define histories in any way we need (as they are just
an instance of auxiliary state) allows us to record additional logical
information into them, capturing quantitative aspects of the expected
interference. In combination with a possibility to describe the
allowed changes in the real and auxiliary state of an object, it
provides us with a technique to express and verify diverse consistency
conditions on the histories (\cf (\emph{ii})).  

What is crucial for capturing the essence of most of these conditions
is the ability to formally quantify in \emph{thread-local} assertions
over the arbitrary effects produced by \emph{other} interfering
threads and the ``shape'' of the environment (\eg, a number of threads
running concurrently with the one being specified).
%
% \an{It sounds like we also need to work in the word
%   \emph{subjectivity}, in order to express what's new in our approach,
%   and why the previous work didn't quite succeed in capturing what we
%   propose. The point should be that quiescent consistency, and others,
%   just naturally want to have access to the contributions of others.}
%
% \is{okay, check the paragraphs above and below.}
%
Such ability, dubbed \emph{subjectivity}, has been introduced in the
recently developed \emph{Subjective} and \emph{Fine-grained Concurrent
  Separation Logics}, SCSL~\cite{LeyWild-Nanevski:POPL13} and
FCSL~\cite{Nanevski-al:ESOP14}. This is the reason why we have chosen
to use FCSL as a specification and verification framework.
%
% While some other logics could have been employed for this role (see
% Section~\ref{sec:related} for discussion on alternatives),

In addition to the native support of subjectivity, FCSL was appealing
to us because of its uniform model of thread-local resources, which is
based on \emph{partial commutative monoids} (PCMs) and can be
instantiated to reason about arbitrary state, auxiliary or real, such
as heaps, thread capabilities, and, indeed, histories.
%
Finally, FCSL has been implemented as a mechanized tool for concurency
verification~\cite{Sergey-al:PLDI15}, enabling provably sound
computer-aided reasoning about highly optimized concurrent objects,
whose state invariants tend to be quite complicated.

% \an{This section reads meekly, but it's the
%   most important one.  It should be strenghtened by bringing up the
%   points about subjectivity further above, as I said.  Subjectivity as
%   a crucial and most important idea, which makes all the
%   difference. If we don't mention it, peple like Noam Rinetzky or
%   Cesar Sanches will not get the feel of what's different.
%   Subjectivity is the delta that makes us succeed where everyone else
%   went into wrong directions.  Also, we should be more proud of our
%   work. It doesn't matter if "other logics" could have been
%   used. Subjectivity has been invented by SCSL and in the fine-grained
%   setting by FCSL. Just because some copy-cats later decided to
%   implement it in their logics doesn't mean we should bow to them. So
%   I think we don't need to mention "other logics can do it" part. Just
%   say that we use FCSL to explain the ideas of this paper, because
%   FCSL is based on, and introduced the idea of subjectivity.  Of
%   course, this should be done after subjectivity has been promoted
%   further above as the main idea that makes everything fly.}

\subsection{Contributions and paper outline}
\label{sec:contr-paper-outl}

In the remainder of the paper, we demonstrate the viability of the
logic-based approach for defining correctness conditions for highly
parallel concurrent objects by formally specifying and verifying two
concurrent data structures: an \emph{elimination-based exchange
  channel}~\cite{Scherer-al:SCOOL05} (or simply \emph{exchanger}) and
a simple \emph{counting network}~\cite{Aspnes-al:JACM94}, whose
behavior was previously described only in terms of dedicated
criteria~\cite{Hemed-Rinetzky:PODC14,Derrick-al:FM14,Jagadeesan-Riely:ICALP14}. We
then argue for the adequacy of the provided specs by modularly
verifying a series of concurrent client programs, which employ these
data structures.

\noindent
Specifically, in this work we make the following contributions:

\vspace{2pt}

\begin{itemize}

\item We describe a series of novel reasoning patterns that unify
  state-based and history-based approaches for specification and
  verification of concurrent objects.

\item We provide the first formal logic-based specification of an
  \emph{elimination-based concurrent
    exchanger}~\cite{Scherer-al:SCOOL05} in the spirit of
  concurrency-aware linearizability~\cite{Hemed-Rinetzky:PODC14}
  (Section~\ref{sec:overview});

\item We specify and formally verify of a realistic client of the
  exchanger adapted directly from the \code{java.util.concurrent}
  library documentation~\cite{ExchangerClass} (Section~\ref{sec:cal}).

\item We give the first logical specification to a simple
  \emph{counting network}~\cite{Aspnes-al:JACM94} and verify two its
  clients, exploiting the implications of the proved spec in the
  spirit of \emph{quiescent consistency}~\cite{Derrick-al:FM14} and
  \emph{quantitative quiescent
    consistency}~\cite{Jagadeesan-Riely:ICALP14}
  (Section~\ref{sec:counting}).

\item We supply all examples from the paper with proof scripts that
  were mechanically checked in the Coq proof
  assistant~\cite{Coq-manual,Bertot-Casteran:BOOK,Sergey-al:PLDI15}.

\end{itemize}

\vspace{2pt}

\noindent
Section~\ref{sec:background} provides necessary minimal background on
program logics for concurrency and key concepts of
FCSL~\cite{Nanevski-al:ESOP14}. In Section~\ref{sec:discussion} we
discuss other possible applications of the proposed approach for
reasoning about concurrent objects. We cover the relevant related work
in Section~\ref{sec:related} and conclude in
Section~\ref{sec:conclusion}.

\an{We should have in the intro with a paragraph like follows: That
  linearizability can adequatly be replaced by Hoare-style reasoning
  has been already argued by the previous work on HOCAP and ICAP,
  which employ the method of parametrization (to be skethed in the
  related work section). In this paper, we argue that similar
  replacement can be carried out for three other alternative
  consistency criteria such as Concurrency-aware linearizability
  (CAL), quiescent consistency, and quantitative quiescent
  consistency. Moreover, the last two consistency criteria seem
  impossible to address by a method of parametrization, at least
  without some significant and complicated meta-theoretic additions
  (e.g., prophecy variables), whereas here we show how they can be
  easily supplanted using reasoning based on subjectivity combined
  with histories.}

\is{I disagree with the remark by Aleks almost entirely. First, I'm
  sick and tired of giving credit to CAP-like approaches for something
  they don't have enev a slight idea how to do. While I'll most
  certainly put something on them into the related work, I don't want
  to have enything on them in the intro, otherwise there will be
  another round of what we've seen already a year ago.
%  
  Second, I don't think that remark on prophecies is sound: the
  counting network example doesn't have anything reminiscent to
  prophecy-requiring linearization points. That is, any comparison to
  LP-based methods, as they're done in CAP, is dangerous and
  misleading, as (a) they might be able to do it via some other
  callback-related mumbo-jumbo (or they might not, but we'll get a
  strong reject on that grounds anyway, just like the last tim), or
  (b) they indeed cannot do it, but this fact has nothing to with with
  their prophecies-related troubles. 
%
  Finally, the only people who will be able to understand this
  comments are those who will give us strong reject because of it
  (just like the last time). That is why I suggest us to focus bringin
  Henzinger and Rinetzky-like crown on our side by speaking the
  language they understand.
%
  So, the bottom line: I'm strongly against any specific on
  CAP-related logics or any of this stuff in the intro.
}

\an{Gee, that's a strong sentiment. I will just reiterate that HOCAP
  and ICAP have suggested that Hoare logic should replace
  linearizability. It seems prudent to be generous and give them
  credit for that, especially as it doesn't cost us anything
  (basically, just one line of space), and doesn't diminish from our
  contribution at all. A year ago, we got rejected form POPL precise
  because we didn't have a forward-pointer in the intro to the related
  work section, where the comparison was done. They accused us in the
  after-rebutal comment of setting '`wrong expectations'' in our
  intro. After that, in the ESOP version, we did put the forward
  pointer. So, I really don't think a forward pointer would hurt us
  here, and can only help. As for prophecies, I just put them as a
  side comment. I don't think we really have to mention them. We could
  simply say that its unclear that parametrization can be used to
  handle quiescense and quantitativeness, wihtout speculiating what's
  needed to fix that. I do personally think that prophecies will
  probably be OK for that, if they can get them. But then, I think
  they'll never get them, precisely because subjectivity and histories
  is what you need here :-). As for Henzinger and Rinetzky crowd:
  well, Philippa will be reviewing this, so expect to get someone from
  the CAP crowd too.}


%\lipsum[1-4]


%\vspace{-2pt}
\section{Introduction}
\label{sec:introduction}

% Traditionally, the correctness of a concurrent object is reasoned
% using its method call/return histories, and follows two steps. The
% first step is to define a specification set that describes all
% ``basic'' behaviors that the object's client might observe when using
% it. The second step is to fix a consistency condition that
% relates the object's concurrently observable behaviors to the ones in
% the specification set.
% % 
% Typically, the specification set is taken to be the
% sequential behaviors of a concurrent object, in
% which the calls to the object's methods are immediately followed by the 
% corresponding returns.
% %
% \is{The paragraph above doesn't seem to contribute in the followin
%   intro at its present structure. Can we drop it at all or, may be,
%   move to the related work?}
 
Linearizability~\cite{Herlihy-Wing:TOPLAS90} remains the most
well-known correctness condition for concurrent objects. It has been
used to argue the correctness of a variety of concurrent objects such
as stacks, queues, sets, locks, snapshots, and their
combinations---all of which have intuitive sequential specifications
(specs).
%
For each concurrent history of an object, linearizability requires
that there exist a mapping to a sequential history, such that the
ordering of two matching call/ return pairs is preserved either if
they are performed by the same thread, or if they do not overlap.
%
However, enforcement of linearizability is expensive: an
implementation of a concurrent object must inherently incorporate
costly synchronization primitives~\cite{Attiya-al:POPL11}, which then
become points of high contention and prevent efficient parallelization
\cite{Shavit:CACM11}. Moreover, many concurrent objects are inherently
\emph{non-linearizable}. In the presence of concurrent interference,
the behaviors of these programs are observably different from their
sequential behaviors: linearizability is not flexible enough to
capture their behaviors, as observed in several recent works
\cite{Hemed-Rinetzky:PODC14,Shavit:CACM11,Derrick-al:FM14,Jagadeesan-Riely:ICALP14}.

% The way to remedy this situation
% and to enable scalability is to change the semantics of a concurrent
% object and to redefine its admissible behaviors. \ab{Not happy yet with this
% para.}

This paper therefore focuses on reasoning about non-linearizable
concurrent objects and their clients, such as counting
networks~\cite{Aspnes-al:JACM94} or an elimination-based
exchanger~\cite{Scherer-al:SCOOL05}.
%
% In the presence of concurrent interference, the behaviors of these programs are observably different from their sequential
% behaviors. Linearizability is therefore not flexible enough to
% capture their behaviors as observed in several recent works~\cite{Hemed-Rinetzky:PODC14,Shavit:CACM11,Derrick-al:FM14,Jagadeesan-Riely:ICALP14}.
%\ab{Ugh. This sounds weak.}
%
% who term them ``concurrency-aware concurrent objects'',
%
% \is{I don't think Rinetzky is so special that his should work should
%   have a dedicated mentioning here. Hence, I maked this statement a
%   bit more general and supplied some citations.}
%
%
% \an{Maybe say why? \Ie, we can use the criticism from CAL. Because
%   linearizability reduces a concurrent object to a sequential
%   approximation, and sometimes this needs to be relaxed, because the
%   behavior of the program in the presence of concurrent interference,
%   is different from its sequential behavior.}
%
% ``their behavior in the presence of concurrent (overlapping)
% operations is, and should be, \emph{observably different} from their
% behavior in the sequential setting''~\cite{Hemed-Rinetzky:PODC14}.
% 
To reason about such data structures, a variety of novel consistency
conditions and correctness criteria have been developed:
concurrency-aware linearizability (CAL)~\cite{Hemed-Rinetzky:PODC14},
quiescent consistency (QC)~\cite{Aspnes-al:JACM94},
quasi-linearizability (QL)~\cite{Afek-al:OPODIS10}, quantitative
relaxation~\cite{Henzinger-al:POPL13}, quantitative quiescent
consistency (QQC)~\cite{Jagadeesan-Riely:ICALP14}, local
linearizability~\cite{Haas-al-local15}, to name a few.
% \an{Are
%   the phrases consistency condition and correctness criteria
%   synonymous? How about consistency criteria and correctness
%   condition? We seem to use all 4 interchangeably. Should we
%   streamline?} \is{From the related literature, I observe
%   \emph{correctness criteria} and \emph{correctness condition} to be
%   synonyms. In contrast, \emph{consistency condition} is used to
%   denote the way concurrent behaviors are related to ones in
%   specification set (\eg, sequential or CA ones). I somewhat adapted
%   this terminology from Henzinger et al, POPL'13.}
%
These criteria attempt to specify a concurrent program's behavior in
the presence of interference, though some, such as QC devote special
treatment to the sequential case, trying to qualify the behavior in
the quiescent, that is, interference-free, moment.

% However, this variety makes it difficult to establish that the
% conditions are general enough to accommodate highly parallel
% concurrent objects. 
% %
% \is{I don't understand the statement above: how the diversity makes it
% difficult to accommodate algorithms---this is what it aims to help.
% Perhaps, what you want to say here is that each new condition should
% be characterized in some way?}
%

This variety makes it difficult to see when a specific consistency
condition should be used, or how to combine several of them, as may be
needed when different parts of the same program are specified using
different conditions. In particular, in contrast to linearizability,
which has been shown to imply observational refinement\footnote{That is,
  a program can be replaced in any larger context by the set of the
  sequential histories to which it linearizes; every property
  derivable about the replacement applies to the original program as
  well.}~\cite{Filipovic-al:TCS10}, no similar results have been
proven for the aforementioned consistency conditions. While some of
them, say QC and QQC, are known to be compositional in the sense that
the combination of two QC (QQC) objects is QC
(QQC)~\cite{Herlihy-Shavit:08,Jagadeesan-Riely:ICALP14}, such
compositionality is much weaker than observational refinement, and does
not allow transferring a general property (such as, one expressed as a Hoare
triple~\cite{Turon-al:ICFP13,Liang-Feng:PLDI13}) to a program, from
the set of histories with which it is QC (QQC) consistent.
%
While recently linearizability has been generalized to apply to modern
%
concurrent programs, which use higher-orderness, ownership transfer,
and dynamic
allocation~\cite{Gotsman-Yang:CONCUR12,Cerone-al:ICALP14}), the
alternative consistency conditions almost invariably focus on 
simple imperative programs. Finally, the considerations of the
alternative conditions have focused only on their semantics: there
is a lack of syntactic logical methods for checking that a program
satisfies one of them (again, in contrast to the situation for
establishing linearizability for a given
program~\cite{OHearn-al:PODC10,Liang-Feng:PLDI13,Turon-al:ICFP13,Vafeiadis:PhD}). 
%
Such methods are desirable, as they allow one to verify clients and
implementations in a single proof system.
%

% \is{Can we say, why such methods are desirable, and why they are
%   better than reasoning directly in terms of program semantics?}
% %
% \is{Presumably, uniform reasoning about clients in the presence of
%   HO, dynamic state, amenable to scalable computer-aided verification. }

From these observations stems a fundamental question: Can the
alternative correctness conditions be represented in one and the same
logical system, with support for higher-order compositional reasoning
about realistic libraries of modern, possibly non-linearizable,
concurrent programs?
%

%In short, there is a lack of proof methods amenable to structured
%formal verification with non-linearizable objects, and formal results
%that enable these proof methods to apply be compositionally used on clients.

%\ab{How to bring in client reasoning and associated problems?}
%
% \is{Here we should clarify what we mean by compositionality. QC and
%   QQC are also proven to be compositional by their authors, so that
%   statement above is misleading if not wrong.}

%Despite this variety, it is not obvious how these criteria can facilitate the
%verification of modern concurrent programs which use higher-orderness,
%ownership transfer, dynamic allocation, etc. In contrast 
%to linearizability which has been shown to imply observational refinement\footnote{That is, a program can be replaced in any larger context by the set of the sequential histories to which it linearizes.}~\cite{Filipovic-al:TCS10},
%no similar results have been proven for the aforementioned consistency criteria. 
%%
%Neither are syntactic logical methods for establishing the consistency
%criteria known (again, in contrast to the situation for establishing
%linearizability for a given program~\cite{Gotsman-Yang:CONCUR12,Cerone-al:ICALP14,Turon-al:ICFP13}). 
%%
%While QC and QQC are known to be compositional~\cite{What?}, such a property 
%only asserts that the composition of two QC (QQC) objects is QC (QQC). 
%Such compositionality is too weak to be applicable in a situation
%where, say, two procedures verified under different criteria need to be used in 
%the same program, and the program's precondition, involving the different 
%criteria, needs to be established. 
%%
%In other words, there is a lack of compositional proof methods amenable to 
%structured formal verification with non-linearizable objects. 
%\ab{How to bring in client reasoning and associated problems?}
%%
%% \is{Here we should clarify what we mean by compositionality. QC and
%%   QQC are also proven to be compositional by their authors, so that
%%   statement above is misleading if not wrong.}

%
%Moreover, client-side reasoning about programs that incorporate
%several objects specified via \emph{different} criteria becomes
%enormously complicated: clients of a concurrent object, when
%committing to a correctness criterion, need to adopt specific
%reasoning principles to characterize the object's behavior.  It can be
%difficult to ensure compositionality of reasoning when dealing with
%different communicating objects.
%%
%\is{The statement above is a bit vague (now I realize) is and isn't
%  instantiated particularly well in our paper: we don't show examples
%  with several objects (although we could). So how about we say here
%  what the previous intro used to say, that for each new criterion one
%  has to devise a method for exploiting the provided safety guarantees
%  for the sake of reasoning about client code that uses the concurrent
%  object.}
%
%\is{In my opinion, the paragraph above is crucial for the whole story,
%  as it sets the motivation for the paper (like those questions we had
%  previously), so it should be more punchy in describing what the
%  problems are with the state of the art, and why one shoudl care
%  about them.}  \an{How about: Moreover, while syntactic logical
%  methods exist for establishing linearizability for a given program
%  (CITE some stuff by Gotzman, and CaReSL), such methods do not exist
%  for the other criteria. Even if such methods existed, one would have
%  to engineer ways of combining them into a unified framework,
%  whenever two procedures verified by different criteria are to be
%  used in the same program. }

% \is{While I totally agree with all of the things said above, in my
%   opinion, at this very place we need to place a punch-phrase (a
%   slogan), that summarizes the problem we attack. Something, in the
%   style of herr Dreyer, \eg, \emph{These observations beg the
%     question: ...?}}  \an{I suppose we can pull a Dreyer here. In
%   principle, his overselling has been noted by many people, so I'm not
%   too convinced that we should follow his approach. But, we could. We
%   could say something like: These observation beg the question whether
%   all these alternative conditions can be represented in a single
%   unified logical system, with support for higher-order compositional
%   reasoning about realistic libraries of modern concurrent programs.}

\subsection{Our approach: logic-based concurrency specification}

This paper demonstrates a uniform approach---based on a Hoare-style
program logic---for verifying the correctness of highly scalable
concurrent objects and their clients, without recourse to specialized
correctness criteria and consistency conditions. Our approach uses
Fine-grained Concurrent Separation Logic
(FCSL)~\cite{Nanevski-al:ESOP14}. We show, via examples, that the
basic ingredient of FCSL, \emph{subjectivity}, provides the uniformity
we seek. Subjectivity permits that within a spec of a thread, one can
refer to the private state of all other interfering threads \emph{in a
  local manner}. Being able to refer to such state enables one to
directly express the results of a program as a function of the
interference of other threads. It ultimately yields uniform reasoning
principles capable of expressing the essential properties captured by
the various correctness criteria.

%we seek. Subjectivity permits the use of two auxiliary variables
%within the scope of each thread; one names the thread's own local
%state, and the other names the private state of all other interfering
%threads combined. This split view of auxiliary state allows directly
%relating the result of a program to the interference of all other
%concurrent threads over auxiliary state, ultimately yielding uniform
%reasoning principles.


Traditionally, correctness criteria for concurrent objects are
formulated in terms of call/return histories of threads, and their
rearrangements. In contrast, assertions in Hoare-style program logics
constrain \emph{state}, auxiliary or real, in which the program runs.
%
In order to bridge this gap, Hoare-style reasoning has been recently
extended to histories, which were formulated as a specific instance of
auxiliary
state~\cite{Fu-al:CONCUR10,Gotsman-al:ESOP13,Sergey-al:ESOP15,Bell-al:SAS10}.
%
%
% our starting point is the representation of a program's
% history directly as user-specified auxiliary state. Such a
% representation facilitates reasoning about history via Hoare-style
% specs. This is a simple and old idea~\cite{what}, that in FCSL comes
% with a twist.
%
% \is{This statement seems like it's taken directly from the ESOP'15
%   paper intro (including the twis bit). However, there it was
%   referring to histories in concurrency in general (including
%   semantics). However, I don't think that histories in Hoare-style
%   logics are an \emph{old} idea. So, may be, we can just say that
%   histories were used in previous logics to reason about FCD and cite
%   HLRG and Gotsman-Yang?}
%
% \is{The paragraph before should be changed to introduce auxiliary
%   state and related logics and then proceed to elaborate on FCSL.}
%
For instance, instead of call/return histories, FCSL allows one to
employ \emph{time-stamped histories}~\cite{Sergey-al:ESOP15} to reason
about linearizable objects. A time-stamped history consists of entries
of the form $t \mapsto a$, to signify that the (typically atomic)
operation $a$ occurred at time $t$. A Hoare-style spec which shows
that a program's history changes by a singleton $t \mapsto a$ can be
seen as exhibiting a behavior $a$ at a linearization point~$t$.
%
Such specification approach makes fine-grained (\ie, lock-free) object
implementations look like atomic ones to the clients, whose proofs are
carried out only out of the object specs.

%\an{Some comment here on the similarity between histories and heaps.}

In this work, we augment this history-based approach to Hoare-style
specifications in a significant way to handle non-linearizable
objects. In particular, we show that more \emph{general notions of
  time-stamped histories lead to adequately capturing the essence of
  alternative consistency criteria} such as CAL, QC, and QQC/QL.  To
wit, an auxiliary history need not merely identify a point at which an
atomic behavior logically occurred, but additionally can include
information about the program's interference.

For example, the main idea of CAL is that histories with which one
linearizes cannot be sequential, but have to be concurrency-aware
(CA), \ie, allow simultaneous events to be represented. In FCSL we can
do so by picking time-stamped histories with additional imposed
structure that naturally captures the simultaneity of events. In
Section~\ref{sec:overview}, we show how this structure helps in
specifying and verifying---in FCSL---an elimination-based concurrent
exchanger~\cite{Scherer-al:SCOOL05}. In Section~\ref{sec:cal}, we show
how to immediately employ the ascribed specification for the
verification of a client program of the exchanger (adapted directly
from the \code{java.util.concurrent} library
documentation~\cite{ExchangerClass}) in the same logical framework.

% \is{In the following paragraph I replaced stack by counter, which is
%   what we have verified.}

QC requires establishing that at moments of quiescence, \ie, no
interference, programs exhibit some desirable behavior. For example,
at quiescence, a concurrent counter implementation behaves as expected
of a \emph{sequential} counter implementation. We capture this
property by relying on subjectivity: we use time-stamped histories in
which a time-stamp $t$ directly stores the kind of interference
exhibited by the program's environment at time $t$.  One can then
prove, that in the absence of interference, the object behaves
sequentially as expected. In Section~\ref{sec:counting} and
Section~\ref{sec:qclients} we show the specification and
verification---in FCSL---of a simple counting
network~\cite{Aspnes-al:JACM94} and its client, both of whose
correctness relies on QC.

One can also derive stronger, \emph{quantitative}, properties, and
show that a bound on the number of interfering threads implies that
the program exhibits a bounded deviation from the expected sequential
behavior. In the past, this has been addressed using
QL~\cite{Afek-al:OPODIS10} and QQC~\cite{Jagadeesan-Riely:ICALP14} as
correctness criteria. In this paper, we derive it as a consequence of
the choice of the auxiliary state of
histories. Section~\ref{sec:qclients} also shows the verification of a
client of the counting network, whose correctness relies on QQC.

The unifying mechanism behind all these different kinds of histories
(and indeed behind the subjective split of any auxiliary state) is
that they all satisfy the algebraic properties of a \emph{partial
  commutative monoid} (PCM)~\cite{LeyWild-Nanevski:POPL13}. 
%
Thus, FCSL can represent them, in addition to heaps (also a PCM, and
often a shared resource) in a uniform reasoning framework, applying
the same logical infrastructure (such as the rule of frame) to all
kinds of state, auxiliary or real, in the process also incorporating
higher-orderness, ownership transfer, and dynamic
allocation~\cite{Nanevski-al:ESOP14,Sergey-al:ESOP15}.
%
The uniformity of the logical rules, treating all kinds of state
similarly, makes it possible to conduct the verification in a general
computer-assisted framework: all proofs of the examples from this
paper are checked mechanically in Coq~\cite{Coq-manual} and are
available as a supplementary material.

\paragraph{Alternative logic-based approaches.}

Recent concurrent program logics, such as
HOCAP~\cite{Svendsen-al:ESOP13}, iCAP \cite{Svendsen-Birkedal:ESOP14},
TaDA~\cite{ArrozPincho-al:ECOOP14}, and Iris~\cite{Jung-al:POPL15}
have shown, using the technique of parametrizing programs with
\emph{first-class auxiliary code}~\cite{Jacobs-Piessens:POPL11} or
\emph{atomic tracking resources} (see Section~\ref{sec:related} for
details), that Hoare-style program logics can adequately specify and
verify tricky linearizable concurrent objects and their clients. In
contrast, this work addresses non-linearizable objects and their
clients---but without the use of such parameterization or atomic
tracking resources, which both seem to require identifying
\emph{synchronization points} within libraries, making it non-trivial
to apply the listed above logics to the objects we consider. In the
process we also derive properties which have hitherto been obtained
only via dedicated alternative correctness criteria.

\paragraph{Observational refinement and compositional reasoning.}
The fact that linearizability implies observational
refinement~\cite{Filipovic-al:TCS10, Cerone-al:ICALP14,
  Bouajjani-al:POPL15, Emmi-al:PLDI15} justifies compositional
reasoning, whereby a program can be substituted by its sequential spec
\emph{no matter the property being verified}. Here, we consider
objects whose correctness criteria do not necessarily imply such
observational refinement. Hence, we fix our properties of interest to
be partial correctness Hoare-style specs only. In that setting,
compositionality of the reasoning is justified by the substitution
principle of FCSL on both programs and
proofs~\cite{Nanevski-al:ESOP14}.
%
% which says that a program $e$, and the proof that $e$ satisfies a
% spec $s$, can always be substituted into a context with a hole of
% spec $s$.



% \subsection{Paper outline}

% Sections~\ref{sec:overview} and~\ref{sec:cal}--\ref{sec:qclients} of
% the paper present the case studies, substantiating our proposal to use
% a Hoare-style logic for reasoning uniformly about non-linearizable
% concurrent objects and their clients.
% %
% Section~\ref{sec:background} provides necessary minimal background on
% program logics for concurrency and key concepts of
% FCSL~\cite{Nanevski-al:ESOP14}. Section~\ref{sec:discussion} discusses
% other possible applications of the proposed approach for reasoning
% about concurrent objects. Relevant related work appears in
% Section~\ref{sec:related}. Section~\ref{sec:conclusion} discusses
% future work and concludes.



%  in the
%   process several of the aforementioned correctness criteria are
%   subsumed.
% %
% \is{Isn't it a bit strong of a statement? We don't have the formal
%   correspondence proved.}
% \begin{comment}
% \an{Hmm, the remainder of this section reads a bit unfocused. We need
%   to say how we do the job, and what's essential. Thus, I would say,
%   let's focus the remainder to explaining what is subjectivity (having
%   two kinds of auxiliary state: self and other). Also, say something
%   about histories. But I would first go with subjectivity; thus, move
%   the section Why FCSL up, and then talk about histories.  But I would
%   definitely drop the section on protocols. They are not original to
%   us, and they don't particularly clarify the main message of the
%   intro, which is that subjectivity suffices for all these
%   criteria. In this particular paper, protocols are largely a
%   technicality, so why not just leave them for the background
%   section?}
% %
% \is{I disagree with the remark on unimportance of protocols. While,
%   indeed, they appeared before FCSL (and subjectivity has appeared in
%   SCSL, which we don't use in this paper), they are equally important
%   for what's achieved in this work. We don't claim protocols as our
%   contribution, but they are crucial for the essence of the
%   verification patterns that we are describing in the rest of the
%   paper (or at least for QC and QQC). I think, the paragraph ``why
%   FCSL?'' gives the right amount of intuition on what subjectivity is,
%   so let's move it up. The protocols can also appear in the same
%   paragraph, later, and with more references to the work, prior of
%   FCSL. Essentially we should plant a message that all these fellas
%   before us knew about protocols, but we now teach how to use them
%   right. :-)} 
% %
% \an{I didn't say protocols are unimportant in principle. They should
%   be covered. Just not in the \emph{introduction}, as they just don't
%   seem to be adding much to it. I was reading the paragraph on
%   protocols, and it came of deadening. It talks about stability,
%   locality, whatnot, but not about why the reader should care about
%   protocols when thinking of \emph{alternative consistency
%     criteria}. If you can rewrite the paragraph to liven it up and
%   crisply define why protocols are important for these criteria
%   specifically, then fine. But, notice, the paragraphs leading up to
%   this one say that they key to this paper is subjectivity and PCMs,
%   which is setting the tone quite a lot.}  
% %
% \is{Now I re-read it again, perhaps, we should say what exactly we mean
%   by uniformity there and how does it relate to the problems with the
%   existing approaches. As it's written now it is way too general and
%   looks like a twin of the same part in the intro of the PLDI paper
%   and I think, we should make it fresher by telling what
%   subjectivity/PCMs helped us to discover this time, which is totally
%   novel. And for this, check my comment below.}
% \is{Let me add more to it. While subjectivity, histories and protocols
%   are all important for what we're doing here, I would suggest us to
%   emphasize in the intro the verification patterns we discovered by
%   using the combination of these three, namely: histories with twin
%   contributions (CAL) and interference-capturing histories (QC and
%   QQC). I don't have a better name of these at this moment, but I
%   believe, something along these lines without too much specifics,
%   should go to the intro, as it summarizes the instances of the
%   subjectivity/histories/protocol magic, that we came up with to
%   specify our examples in this work. And the community would benefit
%   from understanding these patterns as it earlier benefited from
%   understanding Hindsight and (to some extent) fractional
%   permissions.}  
% %
% \an{Sorry, the talk of ``all three'' reminds me of ``Ghost, Protocols,
%   Shit'' paper, which I found very annoying. In general, I don't mind
%   talking about the verification patterns that you mention; I think
%   that would be preferable to the non-germane discussion on
%   protocols. But, that aside, I want to emphasize one principle, which
%   I think it important. When deciding what to say in the intro, we
%   should focus on just the most important things. If you can't say why
%   PCMs, or protocols, or natural numbers, or what have you, are
%   important to the alternative consistency criteria, which are the
%   topic from the title, then none of them should be mentioned in the
%   intro!!! We don't mention nats in the intro, so why bother with
%   protocols? Quite frankly, ditto for the PCMs!!! All these things
%   just water-down the discussion, and de-focus from the main point! I
%   think a short, but razor-focused, intro of 1-or-so page would be
%   quite refreshing; it would give us a chance to immediately jump to
%   describing the interesting stuff in the technical sections.
% %
%   Now, regarding this principle, why is subjectivity something
%   \emph{to be mentioned}? Why is it the main point? Well, here's why:
%   Subjectivity allows providing directly a concurrent specification of
%   a program, as a function of its concurrent environment. This is in
%   stark contrast to the consistency criteria approach which usually
%   (actually, all but CAL) seek to specify the program in terms of some
%   equivalent-looking sequential behavior, and then spend time
%   justifying why the sequential behavior is really
%   ``equivalent-looking''. That's why subjectivity is our secret
%   weapon, and that's why it's the key to our approach, and that's why
%   our approach works.
% %
%   I recalled this point when I saw Eraz Petrank's picture on FB
%   15-mins ago :-) I recall he was intrigued by this point that we
%   specify programs in a concurrent manner directly, when Ilya
%   mentioned to him at SPAA.}  
% %
% \is{Okay, point taken, let's then be more specific about the secret
%   weapons and the goals we're trying to achieve by this
%   paper. Subjectivity, indeed, shines here, so let's give it the usual
%   praise in the introduction.  Is the use of subjectivity the ultimate
%   essence of what we're going to demonstrate by this paper? I mean, is
%   that the case that one, given subjectivity, will come up with
%   everything else that we describe here? Sorry, but I don't think
%   so. So how about right after subjectivity we say \emph{how} exactly
%   we use it, which is (I hope you agree) is quite far from the
%   previous appearances in the four papers Aleks and company published
%   on subjectivity. So, this is why, agreeing to drop PCMs and
%   protocols from the intro (as the points made about them are fair), I
%   suggest us to focus the rest of this part of the intro on the
%   \emph{how} part, outlining our findings wrt histories and using
%   them. This ideas \emph{on top of subjectivity} are our tools and
%   \emph{primary} contributions for this paper (not the examples we
%   verified, as those are just means of demonstration!), and I didn't
%   realize this point until we had this discussion. So, just like
%   O'Hearn and company celebrate \emph{the idea of hindsight} (instead
%   of CSL or whatever logic they use there) in their PODC'10 paper, we
%   should celebrate the formalization principles enabled by
%   subjectivity that allow us to subsume CAL and (Q)QC. In contrast, if
%   we just say ``subjectivity is all you need'', we risk to end up like
%   Dodds et al with their POPL'11 paper (not to mention that it was
%   buggy), who said: ``here are the barriers and CAP is all you need to
%   verify them''. Even after reading their recent TOPLAS submission, I
%   still have no clue about what should I take out of their approach
%   except for the fact that they are very clever and know how to use
%   (i)CAP. So, let's not make this mistake and, I repeat it again,
%   emphasize \emph{how} to use subjectivity for the win. }
% %
% \an{Ok, let me try some synthesis of this discussion. We start the
%   ``our approach'' part of the intro as follows. We say that we have a
%   Hoare logic FCSL, which is subjective. Subjectivity means that we
%   have two auxiliary variables in the scope of assertions, giving us a
%   way of specifying the behavior of the program as a function of its
%   concurrent environment. This is in stark contrast to consistency
%   criteria approach that ... cut+paste+from+above. Then go on to
%   introduce the patterns you wanted, and say something like: For
%   example, in the case of CAL, we specify the behavior by combining
%   the subjective view with the auxiliary state of a special kind of
%   ``twin-symmetric'' histories that capture the inherent symmetry of
%   an exchanging program. In the case of QC and QC we use histories
%   that directly store the relevant aspects of the interfering
%   threads, and allow us to derive, as a consequence out of the spec,
%   the expected QC and QQC properties. Expand here as much as you
%   want. Then say something brief like: all these patterns can be
%   uniformly expressed as special cases of the algebraic structure of
%   PCMs, further supported by concurrent protocols that specify
%   application-specific interference. But don't dwell too much on that
%   part?}
% %
% \is{Okay, good, this seems like more or less what I'd write for
%   now. What about histories? When should we state precisely what are
%   they for us (currently, this comes in the second paragraph below)?
%   How about we start this subsection by saying right away that
%   \emph{subjectivity} and \emph{histories-as-state} are the enabling
%   tools we take from previous works? } 
% %
% \an{Let me not refer to the paragraphs below, as things look quite out
%   of place now. So its bound to be confusing. In general, histories
%   should come right after subjectivity, but before we start describing
%   the flavors of ``twin-symmetric'', or ``interference-dependent''
%   histories, etc. One can introduce them briefly: time-stamped
%   histories describe what happened at a time-stamp $t$. The
%   ``contents'' of $t$ can be diverse: atomic operation, twin-symmetric
%   exchange, some information about the behavior of threads at that
%   moment. In the previous work, we only explored the atomic-operation
%   aspect, but in this paper, we see that different definitions of
%   histories can encode different consistency criteria. The common
%   thread to all, however, is the PCM algebraic structure, etc, etc.}

% \an{I'm vacillating a bit about saying that subjectivity is ``in stark
%   contrast'' to the correctness-condition approach, because
%   correctness-conditions approach approximates by sequential
%   behavior. It is in stark contrast to linearizability (and we should
%   say that), but certainly not to CAL. Have to think more about what
%   the real contrast is with all these conditions? Maybe its auxiliary
%   state of histories giving us a way to capture temporal reasoning
%   that's the foundation of all these criteria?}

% In the FCSL program logic, specifications (specs) contain
% preconditions and postconditions, as well as a ``concurrent protocol''
% or ``concurrent resource''. This last component defines the invariants
% of the shared state that are respected by all concurrent threads,
% and/or the allowed state changes that the threads can make, as
% directed by the
% protocol~\cite{Jones:TOPLAS83,OHearn:TCS07,Turon-al:ICFP13}. To ensure
% thread-locality of the specs, the preconditions and postconditions
% must be stable with respect to the concurrent protocol, \ie, they
% should be invariant under possible changes that interfering
% environment threads can make to the shared state according to the
% protocol.

% As observed in recent works on program
% logics~\cite{Sergey-al:ESOP15,Gotsman-al:ESOP13,Fu-al:CONCUR10,Bell-al:SAS10},
% histories can be represented as instances of auxiliary state. Instead
% of call/return histories, FCSL specs use time-stamped histories. Such
% histories are indexed by discrete time-stamps that ``point to'' atomic
% operations: the histories describe relevant atomic changes in the
% logical state of a concurrent object. Moreover, as recently
% established~\cite{Sergey-al:ESOP15}, reasoning about time-stamped
% histories follows exactly the same patterns that reasoning about heaps
% follows in separation logic~\cite{Reynolds:LICS02}. Histories and
% heaps thus share the same assertion logic, the same rules of
% inference, and thus the same style of local reasoning.
% %
% In contrast to previous work, this paper considers histories that need not be
% linearizable. Indeed since they are instances of auxiliary state, histories can
% store additional logical information, such as \emph{quantitative aspects} of
% the expected interference. We demonstrate how to define
% invariants of concurrent objects in a manner that constrains the real
% and auxiliary state, and to thereby capture a suitable specification
% set of histories, such as, \eg, a \emph{concurrency-aware}
% one~\cite{Hemed-Rinetzky:PODC14}. In combination with the possibility
% to describe the allowed changes in the real and auxiliary state of an
% object, this yields a technique to express and verify the diverse
% consistency conditions on histories such as CAL, QC and QQC.
% % \ab{Not sure about this last
% %   sentence. Need more
% %   ammo here. Do the examples only show extension with additional
% %   logical information? What are the novel reasoning patterns
% %   uncovered?}
% %
% % \is{By the way, should we emphasize somewhere around here that our
% %   histories are not of calls/returns. Even though it's somewhat said
% %   above, someone can get confused, as the word ``history'' is highly
% %   overloaded in the literature.}

% \paragraph{Why FCSL?} 

% The key ingredient of FCSL that captures the essence of most of these
% consistency conditions is
% \emph{subjectivity}~\cite{LeyWild-Nanevski:POPL13}. The notion
% facilitates local reasoning by differentiating between thread-local
% contributions and the contributions of the thread's concurrent
% environment. This split leads to a direct way of relating the
% functional behavior of a program to the interference of its
% environment. Subjectivity allows, in \emph{thread-local} assertions,
% quantification over the arbitrary effects produced by \emph{other}
% interfering threads, as well as the ``shape'' of the environment (\eg,
% the number of threads running concurrently with the one being
% specified). In addition to its native support of subjectivity, FCSL
% also possesses a uniform model of thread-local resources, based on
% PCMs. The PCMs can be instantiated to reason about arbitrary state,
% auxiliary or real, such as heaps, thread capabilities, and histories.
% Finally, FCSL has been implemented as a mechanized tool for concurrency
% verification~\cite{Sergey-al:PLDI15}, enabling provably sound
% computer-aided reasoning about concurrent objects, whose state
% invariants can be complex.
% \end{comment}


% \an{It sounds like we also need to work in the word
%   \emph{subjectivity}, in order to express what's new in our approach,
%   and why the previous work didn't quite succeed in capturing what we
%   propose. The point should be that quiescent consistency, and others,
%   just naturally want to have access to the contributions of others.}
%
% \is{okay, check the paragraphs above and below.}
%
%

%
% \an{Throw in the word "separation logic" somewhere here, to make the
%   point of our observation more precise. Strictly speaking, that you
%   can have histories as ghosts was an old observation (they even call
%   them "history variables" in the model checking works). Our
%   observation was that \emph{separation logic} can natually and
%   compositionally handle history variables, using the algebraic
%   structure of PCMs.}
%

% \an{This section reads meekly, but it's the
%   most important one.  It should be strenghtened by bringing up the
%   points about subjectivity further above, as I said.  Subjectivity as
%   a crucial and most important idea, which makes all the
%   difference. If we don't mention it, peple like Noam Rinetzky or
%   Cesar Sanches will not get the feel of what's different.
%   Subjectivity is the delta that makes us succeed where everyone else
%   went into wrong directions.  Also, we should be more proud of our
%   work. It doesn't matter if "other logics" could have been
%   used. Subjectivity has been invented by SCSL and in the fine-grained
%   setting by FCSL. Just because some copy-cats later decided to
%   implement it in their logics doesn't mean we should bow to them. So
%   I think we don't need to mention "other logics can do it" part. Just
%   say that we use FCSL to explain the ideas of this paper, because
%   FCSL is based on, and introduced the idea of subjectivity.  Of
%   course, this should be done after subjectivity has been promoted
%   further above as the main idea that makes everything fly.}

% \paragraph{Outline.}
% %\label{sec:contr-paper-outl}
% \an{The examples that we cover could be folded into the above
%   description of the different kind of histories we use. That would
%   probably make a separate outline section unnecessary.}
% Recent concurrent program logics such as
% HOCAP~\cite{Svendsen-al:ESOP13}, iCAP~\cite{Svendsen-Birkedal:ESOP14}
% have shown, using the parametrization technique (see
% Section~\ref{sec:related}), that when reasoning about concurrent
% objects, linearizability can be replaced by Hoare-style program
% specifications. It is an open question whether parametrization can
% scale to the alternative consistency criteria such as CAL, QC and QQC.
% The main contribution of this paper is to show that FCSL can be used
% to reason about highly parallel, non-linearizable concurrent objects
% as well, while incorporating CAL, QC and QQC.

% In the remainder of the paper, % we demonstrate the viability of the
% % logic-based approach for defining correctness conditions for highly
% % parallel concurrent objects by
% we formally specify and verify two
% concurrent data structures: an \emph{elimination-based exchange
%   channel}~\cite{Scherer-al:SCOOL05} and a simple \emph{counting
%   network}~\cite{Aspnes-al:JACM94}, whose behavior was previously
% described only in terms of dedicated
% criteria~\cite{Hemed-Rinetzky:PODC14,Derrick-al:FM14,Jagadeesan-Riely:ICALP14}.
% We then argue for the adequacy of the provided specs by modularly
% verifying a series of concurrent client programs, which employ these
% data structures. Specifically in this work we

% \vspace{2pt}

% \begin{itemize}

% % \item describe a series of novel reasoning patterns that unify
% %   state-based and history-based approaches for specification and
% %   verification of concurrent objects. \is{With that part of the intro removed,
% %     it's not clear how these two approaches are different.}

% \item provide the first formal logic-based spec of an
%   \emph{elimination-based concurrent
%     exchanger}~\cite{Scherer-al:SCOOL05} in the spirit of
%   CAL~\cite{Hemed-Rinetzky:PODC14} (Section~\ref{sec:overview});

% \item specify and formally verify a realistic client of the
%   exchanger adapted directly from the \code{java.util.concurrent}
%   library documentation~\cite{ExchangerClass} (Section~\ref{sec:cal}).

% \item give the first logical spec to a simple \emph{counting
%     network}~\cite{Aspnes-al:JACM94} (Section~\ref{sec:counting}) and
%   verify two its clients, making use of the proved specification in
%   the spirit of QC~\cite{Derrick-al:FM14} and
%   QL/QQC~\cite{Afek-al:OPODIS10,Jagadeesan-Riely:ICALP14}
%   (Section~\ref{sec:qclients}).

% \item supply all examples from the paper with proof scripts that
%   were mechanically checked in the Coq proof
%   assistant~\cite{Coq-manual,Bertot-Casteran:BOOK,Sergey-al:PLDI15}.

% \end{itemize}

% \vspace{2pt}


% \an{We should have in the intro with a paragraph like follows: That
%   linearizability can adequatly be replaced by Hoare-style reasoning
%   has been already argued by the previous work on HOCAP and ICAP,
%   which employ the method of parametrization (to be skethed in the
%   related work section). In this paper, we argue that similar
%   replacement can be carried out for three other alternative
%   consistency criteria such as Concurrency-aware linearizability
%   (CAL), quiescent consistency, and quantitative quiescent
%   consistency. Moreover, the last two consistency criteria seem
%   impossible to address by a method of parametrization, at least
%   without some significant and complicated meta-theoretic additions
%   (e.g., prophecy variables), whereas here we show how they can be
%   easily supplanted using reasoning based on subjectivity combined
%   with histories.}

% \is{I disagree with the remark by Aleks almost entirely. First, I'm
%   sick and tired of giving credit to CAP-like approaches for something
%   they don't have enev a slight idea how to do. While I'll most
%   certainly put something on them into the related work, I don't want
%   to have enything on them in the intro, otherwise there will be
%   another round of what we've seen already a year ago.
% %  
%   Second, I don't think that remark on prophecies is sound: the
%   counting network example doesn't have anything reminiscent to
%   prophecy-requiring linearization points. That is, any comparison to
%   LP-based methods, as they're done in CAP, is dangerous and
%   misleading, as (a) they might be able to do it via some other
%   callback-related mumbo-jumbo (or they might not, but we'll get a
%   strong reject on that grounds anyway, just like the last tim), or
%   (b) they indeed cannot do it, but this fact has nothing to with with
%   their prophecies-related troubles. 
% %
%   Finally, the only people who will be able to understand this
%   comments are those who will give us strong reject because of it
%   (just like the last time). That is why I suggest us to focus bringin
%   Henzinger and Rinetzky-like crown on our side by speaking the
%   language they understand.
% %
%   So, the bottom line: I'm strongly against any specific on
%   CAP-related logics or any of this stuff in the intro.
% }

% \an{Gee, that's a strong sentiment. I will just reiterate that HOCAP
%   and ICAP have suggested that Hoare logic should replace
%   linearizability. It seems prudent to be generous and give them
%   credit for that, especially as it doesn't cost us anything
%   (basically, just one line of space), and doesn't diminish from our
%   contribution at all. A year ago, we got rejected form POPL precise
%   because we didn't have a forward-pointer in the intro to the related
%   work section, where the comparison was done. They accused us in the
%   after-rebutal comment of setting '`wrong expectations'' in our
%   intro. After that, in the ESOP version, we did put the forward
%   pointer. So, I really don't think a forward pointer would hurt us
%   here, and can only help. As for prophecies, I just put them as a
%   side comment. I don't think we really have to mention them. We could
%   simply say that its unclear that parametrization can be used to
%   handle quiescense and quantitativeness, wihtout speculiating what's
%   needed to fix that. I do personally think that prophecies will
%   probably be OK for that, if they can get them. But then, I think
%   they'll never get them, precisely because subjectivity and histories
%   is what you need here :-). As for Henzinger and Rinetzky crowd:
%   well, Philippa will be reviewing this, so expect to get someone from
%   the CAP crowd too.}
% \an{I removed the discussion on the forward pointer. The new intro
%   quite directly side-steps linearizability in the third paragraph, so
%   discussing other Hoare logics for linearizability seems
%   unnecessary.}



%\lipsum[1-4]


%% \gad Macros to refer to snapshot's pointers and line-numbers are
%% defined together with the Figure in

%\newcommand{\fx}{\text{fx}}
%\newcommand{\fy}{\text{fy}}
%\newcommand{\x}{\text{x}}
%\newcommand{\y}{\text{y}}
%\newcommand{\s}{\text{S}}

\newcommand{\fx}{\mathit{fx}}
\newcommand{\fy}{\mathit{fy}}
\newcommand{\x}{x}
\newcommand{\y}{y}
\newcommand{\s}{S}

%%\begin{wrapfigure}[9]{r}[0pt]{0.4\textwidth} 
%% \begin{figure}
%% %
%% \centering
%% \begin{tabular}{l l l}
%% %
%% %  
%% \begin{minipage}[l]{.30\textwidth}
%% \begin{alltt}
%% \num{1}  write (p, v): () \{
%% \num{2}    \act{write} (p, v);
%% \num{3}    b <- \act{read} (S);
%% \num{4}    \textbf{if} b 
%% \num{5}    \textbfthen \act{transfer} (p, v);
%% \num{6}    \textbf{else skip};\}
%% \end{alltt} 
%% \end{minipage}
%% %
%% & \hfill
%% %
%% \begin{minipage}[l]{.6\textwidth}
%% \begin{alltt}
%% \num{1}  scan (): \(A {\times} A\)  \{
%% \num{2}    \act{write} (S, true);
%% \num{3}    \act{write} (fx,\( \bot\));
%% \num{4}    \act{write} (fy,\( \bot\));
%% \num{5}    vx <- \act{read} (x);
%% \num{6}    vy <- \act{read} (y);
%% \num{7}    \act{write}(S, false);
%% \num{8}    ox <- \act{read} (fx);
%% \num{9}    oy <- \act{read} (fy);
%% \num{10}   \textbf{let} rx = \textbf{if} ox \(\neq \bot\) \textbfthen ox \textbf{else} vx;  
%% \num{11}   \textbf{let} ry = \textbf{if} oy \(\neq \bot\) \textbfthen oy \textbf{else} vy;  
%% \num{11}   \act{relink}(rx, ry);
%% \num{12}   \textbf{return} (rx, ry);
%% \end{alltt} 
%% \end{minipage}
%% %
%% \end{tabular}
%% %
%% \caption{Jayanti's single-scanner, single-writer snapshot algorithm}
%% \label{fig:jayanti}
%% \end{figure}
%\end{wrapfigure}

\newcommand{\actwrite}[2]{{#1}\,{:=}\,{#2}}

% The following version saves a little more space
\begin{figure}
%
\centering
\begin{tabular}{c@{\ \ \ \ \ }c}
%  
\begin{minipage}[t][3.7cm][t]{.5\textwidth}
\small
\begin{alltt}
\num{1} write (p : ptr, v : \(A\)) \{
\num{2}  \actwrite{p}{v};
\num{3}  b \tbnd \act{read}(S);
\num{4}  if b 
\num{5}  then \actwrite{(f_of p)}{v}
\num{6}  {else return} \}

  f_of (p : ptr) \{
   return p = x ? fx : fy \}
\end{alltt}
\end{minipage}
%
&
\begin{minipage}[t][3.7cm][t]{.5\textwidth}
\small
\begin{alltt}
\num{ 7} scan (): \(A {\times} A\)  \{
\num{ 8}  \actwrite{S}{true};
\num{ 9}  \actwrite{fx}{\(\bot\)}; \actwrite{fy}{\(\bot\)};
\num{10}  vx \tbnd \act{read}(x); vy \tbnd \act{read}(y);
\num{11}  \actwrite{S}{false};
\num{12}  ox \tbnd \act{read}(fx); oy \tbnd \act{read}(fy);
\num{13}  rx \tbnd if (ox \(\neq\bot\)) then ox {else} vx;  
\num{14}  ry \tbnd if (oy \(\neq\bot\)) then oy {else} vy;  
\num{15}  return (rx, ry) \}
\end{alltt} 
\end{minipage}
%
\end{tabular}
%
\caption{Jayanti's single-scanner/single-writer snapshot algorithm.}
\label{fig:jayanti-snapshot}
\end{figure}



\newcommand{\jywrite}{\texttt{write}\xspace}
\newcommand{\jyscan}{\texttt{scan}\xspace}

\section{Verification challenge and main ideas}
\label{sc:overview}


Jayanti's snapshot algorithm~\cite{Jayanti+STOC05} provides the
functionality of a shared array of size $m$, operated on by two
procedures: \jywrite, which stores a given value into an element, and
\jyscan, which returns the array's contents. We use the
\emph{single-writer}/\emph{single-scanner} version of the algorithm.
which assumes that at most one thread writes into an element, and at
most one thread invokes the scanner, at any given time. In other
words, there is a scanner lock and $m$ per-element locks. A thread
that wants to scan, has to acquire the scanner lock first, and a
thread that wants to write into element $i$ has to acquire the $i$-th
element lock. However, scanning and writing into different elements
can proceed concurrently.
% 
%where a thread acquires a writer lock for a particular element before
%writing into it, and a scanner before scanning. A scanner lock does
%not preclude writing, and a writer lock for an element does not
%preclude scanning, or writing into other elements. 
This is the simplest of Jayanti's algorithms, but it already exhibits
linearization points of dynamic nature. We also restrict the array
size to $m\,{=}\,2$ (\ie, we consider two pointers $\x$ and $\y$,
instead of an array). This removes some tedium from verification, but
exhibits the same conceptual challenges.
 
The difficulty in this snapshot algorithm is ensuring that the scanner
returns the most recent snapshot of the memory. A na\"{i}ve scanner, which
simply reads $\x$ and $\y$ in succession, is unsound. To see why,
consider the following scenario, starting with $\x=5$, $\y=0$. The
scanner reads $\x$, but before it reads $\y$, another thread preempts
it, and changes $\x$ to $2$ and, subsequently, $\y$ to $1$. The
scanner continues to read $\y$, and returns $\x=5, \y=1$, which was
never the contents of the memory. Moreover, $(\x, \y)$, changed from
$(5,0)$ to $(2, 0)$ to $(2, 1)$ as a result of distinct
non-overlapping writes; thus, it is impossible to find a linearization
point for the scan because linearizability only permits reordering of
overlapping operations.

%\ab{Remove rest?} by dynamically reordering non-overlapping
%operations, as permitted by linearizability (though we show further
%below a scenario when {\jyscan} is justified in returning a pair that
%was not the contents of the memory).

%\gad{Do we make the latter example a graph/ figure somehow? We have
%  done so for the slides}

To ensure a sound snapshot, Jayanti's algorithm internally keeps
additional \emph{forwarding pointers} $\fx$ and $\fy$, and a boolean
\emph{scanner bit} $\s$. The implementation is given in
Figure~\ref{fig:jayanti-snapshot}.\footnote{Following Jayanti, we
  simplify the presentation and omit the locking code that ensures the
  single-writer/single-scanner setup. Of course, in our Coq
  development~\cite{CoqFiles}, we make the locking explicit.}
%
The intuition is as follows. A writer storing $v$ into $p$
(line~\lineWrtWrt), will additionally store $v$ into the forwarding
pointer for $p$ (line~\lineWrtFwd), provided $S$ is set. If the
scanner missed the write and instead read the old value of $p$
(lines~\lineScanReadsX--\lineScanReadsY), it will have a chance to
catch $v$ via the forwarding pointer
(lines~\lineScanReadsFX--\lineScanReadsFY). The scanner bit $S$ is
used by writers (line~\lineWrtChk) to detect a scan in progress, and
forward $v$.

{
%\setlength{\belowcaptionskip}{-5pt} 
\begin{figure}[t]
%
\captionsetup[subfigure]{justification=centering}
\centering  
\begin{subfigure}[t]{1\textwidth}
\centering
\begin{tabular}{l || l || l}
  \texttt{l: }\texttt{write (x,2);}\quad &
   \multirow{2}{*}{\texttt{c: scan ()}}\quad & 
    \multirow{2}{*}{\texttt{r: write (x,3)}}  \\
  \phantom{\texttt{l: }}\texttt{write (y,1)} & &   
\end{tabular}
\caption{\label{fig:weird:code}Parallel composition of three threads \texttt{l, c, r}.}
\end{subfigure}\\

\begin{subfigure}[b]{1\textwidth}
\begin{tabular}{l@{\hfill} l@{\hfil}}
\begin{minipage}[t]{0.5\textwidth}
\begin{alltt}
 \num{1}  c: \actwrite{S}{true}
 \num{2}  c: \actwrite{fx}{\(\bot\)}
 \num{3}  c: \actwrite{fy}{\(\bot\)}
 \num{4}  c: \act{read}(x)  // vx <- 5
 \num{5}  c: \act{read}(y)  // vy <- 0
 \num{6}  l: \actwrite{x}{2}
 \num{7}  l: \act{read}(S)  // b <- true
 \num{8}  l: \actwrite{fx}{2} 
 \num{9}  l: return ()
\num{10}  r: \actwrite{x}{3}
\end{alltt}
\end{minipage}
&
\begin{minipage}[t]{0.33\textwidth}
\begin{alltt}
\num{11} l: \actwrite{y}{1}
\num{12} l: \act{read}(S)  // b <- true
\num{13} l: \actwrite{fy}{1}
\num{14} l: return ()
\num{15} c: \actwrite{S}{false}
\num{16} r: \act{read}(S)  // b <- false
\num{17} r: return ()
\num{18} c: \act{read}(fx) // ox <- 2
\num{19} c: \act{read}(fy) // oy <- 1
\num{20} c: return (2,1)
\end{alltt} 
\end{minipage}
%
\end{tabular}
\caption{\label{fig:weird:exec} A possible interleaving of the threads
  in~(\subref{fig:weird:code}).}
\end{subfigure}
\caption{\label{fig:weird} An example leading to a scanner miss.%
}
\end{figure}
}

 
As Jayanti proves, this implementation \emph{is} linearizable. Informally,
every overlapping calls to \jywrite~and \jyscan~can be rearranged to
appear as if they occurred sequentially.  To illustrate, consider the
program in Figure~\ref{fig:weird:code}, and one possible interleaving
of its primitive memory operations in Figure~\ref{fig:weird:exec}. The
threads {\tt l}, {\tt c}, and {\tt r}, start with $\x = 5, \y = 0$.
%
The thread {\tt c} is scheduled first, and through lines~1--5 sets the
scanner bit, clears the forwarding pointers, and reads $\x = 5, \y =
0$. Then {\tt l} intervenes, and in lines~6--9, overwrites
$\x$ with $2$, and seeing $\s$ set, forwards $2$ to $\fx$. Next, {\tt
  r} and {\tt l} overlap, writing $3$ into $\x$ and $1$ into
$\y$. However, while $1$ gets forwarded to $\fy$ (line 13), $3$ is not
forwarded to $\fx$, because $\s$ was turned off in line 15 (\ie, the
scan is no longer in progress). Hence, when {\tt c} reads the
forwarded values (lines 18, 19), it returns $\x = 2, \y = 1$.

While $\x\,{=}\,2, \y\,{=}\,1$ was never the contents of the memory,
returning this snapshot is nevertheless justified because we can
\emph{pretend} that the scanner \emph{missed} {\tt r}'s write of
$3$. Specifically, the events in Figure~\ref{fig:weird:exec} can be
\emph{reordered} to represent the following sequential execution:
%
\begin{equation}
\hfill \mathtt{write\, (x, 2);\ write\, (y,1);\ scan\, ();\ write\, (x,
  3)}\hfill \label{eq:lin}
\end{equation}
%
Importantly, the client programs have no means to discover that a
different scheduling actually took place in real time, because they
can access the internal state of the algorithm only via interface
methods, \jywrite~and \jyscan.

This kind of temporal reordering is the most characteristic aspect of
linearizability proofs, which typically describe the reordering by
listing the linearization points of each procedure. At a linearization
point, the procedure's operations can be spliced into the execution
history as an uninterrupted chunk. For example, in Jayanti's proof,
the linearization point of \jyscan~is at line~\lineScanUnsetsS\ in
Figure~\ref{fig:jayanti-snapshot}, where the scanner bit is unset. The
linearization point of \jywrite, however, may vary. If
\jywrite~starts before an overlapping \jyscan's line~\lineScanUnsetsS,
and moreover, the \jyscan~misses the \jywrite---note the dynamic and
future-dependent nature of this property---, then \jywrite~should
appear after {\tt scan}; that is, the \jywrite's linearization point
is right after \jyscan's linearization point at line~\lineScanUnsetsS.
%
Otherwise, \jywrite's linearization point is at line~\lineWrtWrt.
%
In the former case, \jywrite~exactly has a non-local and
future-dependent linearization point, because the decision on the
logical order of this \jywrite~depends on the execution of \jyscan~in
a different thread. This decision takes effect on
lines~\lineScanReadsFX--\lineScanReadsFY, which can take place
\emph{after} the execution of \jywrite~has terminated.
%
For instance, in Figure~\ref{fig:weird:exec} the execution
of \jywrite~in \texttt{r} terminates at step 17, yet, in Jayanti's
proof, the decision to linearize this \jywrite\ after the
overlapping \jyscan\ is taken at line~18, when the \jyscan\ reads the
value from the previous \jywrite.

%%\gad{Rephrased the paragraph above to answer {\sf R1.Q5}}

%% \gad{Well, the non-regional argument is subtle and here is used with
%%   the wrong example: In this particular case, although the scanner bit
%%   is unset later at 15, the LP' of {\tt l} is fixed at line~11
%%   regardless of the future -- witnessed by the fact that it finishes
%%   green. The non-regionality argument has to be made about the
%%   position of the write to \x done by {\tt r}, which is the write that
%%   is relinked.}

%% \gad{When {\tt r} finishes in line~18, it's position in the final
%%   order is not settled as it depends on the scanners future: this
%%   write is missed by the scanner, and has to be relinked. This
%%   example, though showcases why relink is needed and how it works it
%%   does not showcase non-regionality: when 18 finishes, you have the
%%   information in lines 1-18 to determine that his position will be
%%   changed by scan before the end, so it can be linearized in line 18.}

\begin{figure}[t]
%\captionsetup[subfigure]{justification=centering}
\begin{subfigure}[t]{0.49\textwidth}
\includegraphics[width=6.1cm]{relink-before3.pdf}
\caption{\label{fig:reorder:before}} % Logical $=$ Real Time order, not a snapshot}
\end{subfigure} \hfill
\begin{subfigure}[t]{0.49\textwidth}
\includegraphics[width=6.1cm]{relink-after3.pdf}
\caption{\label{fig:reorder:after}} % Logical $\neq$ Real Time order, snapshot OK}
\end{subfigure}%
%
\caption{\label{fig:reorder} Changing the logical ordering (solid line
  $\ordlist$) of write events from (5, 0, 2, 3, 1) in
  (\subref{fig:reorder:before}) to (5, 0, 2, 1, 3) in
  (\subref{fig:reorder:after}), to reconcile with {\tt scan} returning
  the snapshot $\x=2, \y=1$, upon missing the write of $3$. Dashed
  lines $\hist$ represent real-time ordering.}
\end{figure}


Obviously, the high-level pattern of the proof requires tracking the
\emph{logical ordering} of the \jywrite\ and \jyscan\ events, which
differs from their \emph{real-time ordering}. As the logical ordering
is inherently dynamic, depending on properties such as
\jyscan\ missing a \jywrite, we formalize it in Hoare logic, by
keeping it as a list of events in auxiliary state that can be
dynamically reordered as needed. For example, Figure~\ref{fig:reorder}
shows the situation in the execution of \jyscan~that we reviewed
above. We start with the (initializing) writes of $5$ and $0$ already
executed, and our program performs the writes of $2$, $3$ and $1$ in
the real time order shown by the position of the events on the dashed
lines. In Figure~\ref{fig:reorder:before}, the logical order
$\ordlist$ coincides with real-time order, but is unsound for the
snapshot $\x=2, \y=1$ that \jyscan~wants to return. In that case, the
auxiliary code with which we annotate \jyscan, will change the
sequence $\ordlist$ in-place, as shown in
Figure~\ref{fig:reorder:after}.

Our specification and verification challenge then lies in reconciling
the following requirements. First, we have to posit specs that
say that \jywrite\ performs a write, and \jyscan\ performs a scan of
the memory, with the operations executing in a single logical
moment. Second, we need to implement the event reordering discipline
so that a method call only reorders events that overlap with it; the
logical order of the past events should be preserved. This will be
accomplished by introducing yet further structures into the auxiliary
state and code. Finally, the specs must hide the specifics of
the reordering discipline, which should be internal to the snapshot
object. Different snapshot implementations should be free to implement
different reorderings, without changing the method specs.


%Our challenge then lies in reconciling the following two conflicting
%requirements. First, we need to implement the reordering discipline so
%that the subsequent calls to \jywrite~and \jyscan~preserve the
%established logical order of the past events. This will be
%accomplished by introducing yet further structures into the auxiliary
%state and code. Second, we have to engineer Hoare triples for
%\jywrite~and \jyscan~to be \emph{intuitive} and \emph{helpful} to
%clients, but also to \emph{not expose} the specifics of the reordering
%discipline, which is internal to the snapshot object\footnotemark.
%%We discuss these issues next.
%\footnotetext{\ie we want to give the methods {\it principal} specifications}



\section{Background on FCSL}
\label{sec:background}

A Hoare specification in FCSL has the form
$\spec{P}\ e\ \spec{Q} @ \rcon$. $P$ and $Q$ are the precondition and
postcondition for $e$, and $\rcon$ defines the \emph{resource} on
which $e$ operates. We have elided $\rcon$ from the specs in
Section~\ref{sec:overview}, but explain it now. A resource is a state
transition system describing the state (real and auxiliary) and atomic
operations that the threads that want to simultaneously operate on
that state have to respect. For example, the state space
(aka.~resource invariants) of the resource $\cal E$ for the exchanger
is described by the predicates~\eqref{tag:exchanging} and
\ref{exP}--\ref{ex:gapless}. On the other hand, the transitions of
$\cal E$ can be read off from our discussion of the exchanger proof
outline, where at each atomic operation of \code{CAS}, we explicitly
described how the operation modifies the auxiliary state. These
modifications are the transitions that one has to declare in $\rcon$,
and then prove, when establishing $\spec{P}\ e\ \spec{Q} @ \rcon$,
that $e$ only makes declared transitions.

But a resource also has an important secondary role. Its definition
provides the variables that $P$ and $Q$ may scope over. For example,
in the case of exchanger, we used the variables $\heaps, \perms,
\hists$, $\heapo, \permo, \histo$, and $\heapj, \pending$.  Of course,
different resources will provide different state spaces, transitions
and variables. For example, a commonly used resource is $\cal P$ for
\emph{private state}. $\cal P$ allows only variables $\heaps$ and
$\heapo$ of type heap, denoting the private heap of a thread, and the
private heap of other threads. The transitions of $\cal P$ allow for
reading, writing, allocating and deallocating pointers from $\heaps$.

The mechanism by which a resource defines the allowed variables is as
follows. Underneath, a resource comes with only three variables:
$a_\lcl$, $a_\env$ and $a_\joint$ standing for abstract self state,
other state, and shared (joint) state, but the user can pick their
types depending on the application. In the case of $\cal E$, $a_\lcl$
and $a_\env$ are triples containing a heap, an offer-set and a
history. The variables we used in Section~\ref{sec:overview} are then
merely projections out of such triples:
$a_\lcl\,{=}\,(\heaps, \perms, \hists)$, and
$a_\env\,{=}\,(\heapo, \permo, \histo)$. Similarly,
$a_\joint\,{=}\,(\heapj, \pending)$.

It is essential that $a_\lcl$ and $a_\env$ have a common type, which
moreover, exhibits the algebraic structure of a \emph{partial
  commutative monoid} (PCM). A PCM requires a partial binary operation
$\bullet$ which is commutative and associative, and has a unit. In the
case of $\cal E$, each of the three components of $a_\lcl$ and
$a_\env$---heaps, offer-sets and histories---form a PCM, where
$\bullet$ is disjoint union $\hunion$, and $\emptyset$ is the
unit. Hence, the product of the three is a PCM as well, with $\bullet$
and unit defined point-wise. PCMs are important, as they give a way,
generic in $\rcon$, to define the inference rule for parallel composition.
%
\[
\tag{\normalsize \arabic{tags}}\refstepcounter{tags}\label{eq:parrule}
{\small{
\begin{array}{c}
\specK{\{P_1\}}\ e_1\ \specK{\{Q_1\}} @ \rcon \quad \specK{\{P_2\}}\ e_2\ \specK{\{Q_2\}} @ \rcon\\[2pt]
\hline\\[-7pt]
\specK{\{P_1 \circledast P_2\}}\ e_1 \parallel e_2\ \specK{\{[\res.1/\res]Q_1 \circledast [\res.2/\res]Q_2\}} @ \rcon
\end{array}
}}
\]
%
Here, $\circledast$ is defined as follows.
\[
\tag{\normalsize \arabic{tags}}\refstepcounter{tags}\label{eq:ssep}
\begin{array}{c}
(P_1 \circledast P_2)(a_\lcl, a_\joint, a_\env) \iff \exists x_1~x_2\ldot a_\lcl = x_1 \bullet x_2, \hbox{}\\
 P_1 (x_1, a_\joint, x_2 \bullet a_\env), P_2 (x_2, a_\joint, x_1 \bullet a_\env)
\end{array}
\]
%
%
The inference rule, and the definition of $\circledast$, formalize the
intuition that when a parent thread forks $e_1$ and $e_2$, then $e_1$
is part of the environment for $e_2$ and vice-versa. This is so
because the \emph{self} component $a_\lcl$ of the parent thread is
split into $x_1$ and $x_2$; $x_1$ and $x_2$ become the \emph{self}
parts of $e_1$, and $e_2$ respectively, but $x_2$ is also added to the
\emph{other} component $a_\env$ of $e_1$, and dually, $x_1$ is added
to the \emph{other} component of $e_2$.
%
Also note that parallel composition returns a pair of the outputs
produced by $e_1$ and $e_2$. Thus, the variable $\res$ in $Q_1$ and
$Q_2$ has to be appropriately renamed by the projections $\res.1$ and
$\res.2$ in the postcondition of the parallel composition.

The rule of frame of FCSL is a special case of parallel composition,
when $e_2$ is the idle thread.
%
\[
\tag{\normalsize \arabic{tags}}\refstepcounter{tags}\label{eq:frame}
{\small{
\begin{array}{c}
\specK{\{P_1\}}\ e\ \specK{\{P_2\}} @ \rcon\\[2pt]
\hline\\[-7pt]
\specK{\{P_1 \circledast Q\}}\ e\ \specK{\{P_1 \circledast Q\}} @ \rcon
\end{array}\qquad 
\begin{array}{c}
\mbox{$Q$ stable under}\\
\mbox{$\rcon$'s transitions}
\end{array}
}}
\]
A notable difference from the frame rules of other separation logics
is that FCSL's definition of $\circledast$ forces that the value being
framed onto \emph{self} component is \emph{subtracted} from the
\emph{other} component, whereas in other separation logic, the frame
value materializes out of nowhere. To illustrate, we can frame
$\gists$ onto the history $\hists$ in the the
spec~(\ref{tag:exchangespec}), by taking
$\rcon\,{\eqdef}\,a_\lcl\,{=}\,(\heaps, \hists,
\perms)\,{=}\,(\emptyset, \gists, \emptyset)$.
We obtain, after some simplification:
%
\[
{\small{
\begin{array}{c}
\specK{\{\heaps = \emptyset, \perms = \emptyset, \hists = \gists, \gist \subseteq \gists \hunion \histo \hunion \mygather{\pending}\}}\\[2pt]
\mathtt{exchange}\ v \\[2pt]
\spec{\!\!
  \begin{array}{c}
    \heaps = \emptyset, \perms = \emptyset, \gist \subseteq \gists \hunion \histo \hunion \mygather{\pending}, \hbox{}\\[1pt]
    \mathsf{if}\ \res\ \mathsf{is}\ \mathsf{Some}\ w\ \mathsf{then}\\[1pt]
    \exists t\ldot \hists = t \mapsto (v, w) \hunion \gists, 
    \mathsf{last} (\gist) < t, \twin{t}~\mathsf{else}\ \hists = \gists    
  \end{array}
\!\!}@\cal E
\end{array}
}}
\]
But notice how the spec now says that $\gist \subseteq \gists \hunion
\histo \hunion \mygather{\pending}$, whereas
in~(\ref{tag:exchangespec}) it said $\gist \subseteq \histo \hunion
\mygather{\pending}$. The addition of $\gists$ compensates for
$\gists$ having been subtracted out of $\histo$, to be moved to
$\hists$.

Finally, we will use one more constructor of FCSL, and its associated
inference rule: \emph{hiding}. The program $\mathsf{hide}\ e$
operationally just executes $e$, but logically allows installing a
resource within the scope of $e$. For example, a program can start
only with private heap variables $\heaps$ and $\heapo$ (\ie, using
only the resource $\cal P$). However, it can then take a chunk of heap
out of $\heaps$ and ``install'', for instance, an exchanger resource
$\cal E$ in it, thereby giving the threads in $e$ the ability to
exchange values. The spec of $e$ is then expressed using the variables
that $\cal E$ adds to $\cal P$: $\perms, \hists$, $\permo, \histo$,
and $\heapj, \pending$. Upon termination, the extra variables are
removed from the scope. We elide here the general form of the hiding
rule (it is given in~\cite{Nanevski-al:ESOP14}), and just provide the
special case involving $\cal E$ and $\cal P$.
\[
\tag{\normalsize \arabic{tags}}\refstepcounter{tags}\label{eq:ehide}
{\small{
\begin{array}{c}
\specK{\{P\}}\ e\ \specK{\{Q\}} @ \cal E\\[2pt]
\hline\\[-7pt]
\specK{\{\heaps = \Phi_1(\heapj), \Phi_1(P)\}}\ \mathsf{hide}_{\Phi_1}~e\ \specK{\{\exists \Phi_2\ldot \heaps = \Phi_2(\heapj), \Phi_2(Q)\}} @ \cal P
\end{array}
}}
\]
When read bottom up, the rule says that we can install the resource
$\cal E$ in the scope of a thread that works with $\cal P$, but then
we need substitutions $\Phi_1$ and $\Phi_2$, to map variables of $\cal
E$ ($\heaps, \perms, \hists$, \etc) to values expressed with variables
from $\cal P$ ($\heaps$, and $\heapo$). Here $\Phi_1$ is an initial
such substitution (user provided), and the rule guarantees the
existence of an ending substitution $\Phi_2$. The substitutions have
to satisfy a number of side conditions, which we elide here for
brevity. The most important one is that $\Phi_1$, $\Phi_2$, fix the
\emph{other} variable $a_\env = (\heapo, \permo, \histo)$ to be the
PCM unit (\ie,~a triple of empty sets). Fixing $a_\env$ to unit
formalizes the intuition that within the scope of $\mathsf{hide}$,
the outside threads cannot see $\cal E$, and thus cannot interfere
with the exchanges performed by $e$. They start with empty heap,
offer-set and history, and terminate with empty ones too.

At the beginning of $\mathsf{hide}~e$, the private heap equals the
value that $\Phi_1$ gives to $\heapj$ ($\heaps = \Phi_1(\heapj)$). In
other words, the $\mathsf{hide}$ rule takes the private heap of a
thread, and makes it shared, \ie, gives it to the $\heapj$ component
of $\cal E$. Upon finishing, $\mathsf{hide}~e$ makes $\heapj$ private
again.
%
%($\heaps = \Phi_2(\heapj)$).

%\an{Should I say something about compositionality? Why is FCSL
%  compositional? Maybe say, soundness of FCSL has been established by
%  shallow embeding in Coq. Thus, the logic immediately inherits the
%  substitution principle, thereby allowing that clients can reason
%  only out of the Hoare spec of an object.}

%\subsection{FCSL basics}
%\label{sec:fcsl-basics}
%
%\todo{A short overview of FCSL: mostly, concerning subjectivity and hiding}
%
%\subsection{Histories as auxiliary state}
%\label{sec:hist-state} 


\section{Verifying Exchanger's Client}
\label{sec:cal}
\newcommand{\ts}{{ts}}
\newcommand{\vvs}{{vs}}
\newcommand{\acc}{{ac}}
\newcommand{\ws}{{ws}}
\newcommand{\sorted}[1]{\mathsf{sorted}\ #1}

%\paragraph{Client definition.}
We next illustrate how the verified exchanger from
Section~\ref{sec:overview} can be used by client programs, and how the
\emph{other} component, asserted by the spec to satisfy $\gist
\subseteq \histo \hunion \mygather{\pending}$, is crucial in this
process.
%
We emphasize that proof of the client does not see the fine-grained
implementation details of the exchanger, which are hidden by
spec~\eqref{tag:exchangespec}.
%
A version of the client we consider is actually used in
\code{java.util.concurrent}~\cite{ExchangerClass}, and is defined as
follows. First, we make the exchanger loop until it succeeds in
exchanging the value.
%
\vspace{-5pt}
\[
\vspace{-5pt}
{\small{
\begin{array}{rl}
& \esc{exchange'}~(v : A) : A = \{\\[1pt]
&  ~~~~ w' \Asgn \esc{exchange}~v;\\[1pt]
&  ~~~~
  \kw{if}~~w'~~\kw{is}~~\esc{Some}~w~~\kw{then}~~\kw{return}~w~~\kw{else}~~\esc{exchange'}~v~\}
\end{array}
}}
\]
%
Next, $\esc{exchange'}$ is iterated over a sequence, exchanging each
element in order, appending the received matches to an accumulator. 
%
\vspace{-5pt}
\[
\vspace{-5pt}
{\small{
\begin{array}{rl}
& \esc{ex\_seq}~(\vvs, \acc : \esc{seq}~A) : \esc{seq}~A = \{\\[1pt]
& ~~~~ \kw{if}~~\vvs~~\kw{is}~~v{::}\vvs'~~\kw{then}\\[1pt]
& ~~~~ ~~~~ w \Asgn \esc{exchange'}~v;~~\esc{ex\_seq}~(\vvs', \esc{snoc}~\acc~w)\\[1pt]
& ~~~~ \kw{else}~~\kw{return}~\acc~\}
\end{array}
}}
\]
%
Our goal is to show
% compositionally, \ie~reasoning only out of the
%spec of $\mathtt{exchange}$, 
that the parallel composition
%
\[
e = \esc{ex\_seq}~(\vvs_1, \esc{nil}) \parallel \esc{ex\_seq}~(\vvs_2, \esc{nil})
\]
%
exchanges $\vvs_1$ and $\vvs_2$, \ie,~returns the pair $(\vvs_2,
\vvs_1)$. This is a valid property, assuming that $e$ runs without
interference, so that the two threads in $e$ have no choice but to
exchange the values between themselves. We make the assumption
explicit by using the $\hide$ constructor. Thus, the Hoare
triple we will prove is:
%
\[
\tag{\arabic{tags}}\refstepcounter{tags}\label{tag:hidespec} 
{\small{
\!\!\!
\begin{array}{c}
\specK{\{\heaps = g \mapsto\mathsf{null}\}}~~\hide~~e~~\specK{\{g \in
  \mathsf{dom}~\heaps, \res = (\vvs_2, \vvs_1)\}} @ \cal P
\end{array}
}}
\]
%
It says that we start with a heap where $g$ stores $\mathsf{null}$,
and end with a possibly larger heap (due to the memory leak of the
exchanger), but with the result $\res = (\vvs_2, \vvs_1)$.

%$vs$ and $ws$ are exchanged. This is a valid property under the
%assumptions that the two $\esc{ex\_seq}$ threads run in isolation,
%\ie, without interfernce from any other exchanging threads. In that
%case, the two threads have no other options but to exchange values
%between themselves.

\paragraph{Explaining the verification.}
%
The proof of $\hide~~e$ involves several stages: verification of
$\esc{exchange'}$, $\esc{ex\_seq}$, $e$ and finally $\hide~e$. We only
list the specs involved in the first two stages, omitting the
associated proof outlines. These are straightforward, as the programs
contain only sequential composition, and can be found in our Coq
scripts. The last two stages employ parallel composition and $\hide$,
and are thus more involved, so we present them in detail.

The following are the specs for $\esc{exchange'}$ and $\esc{ex\_seq}$.
%
\[
{\small{
\begin{array}{c}
\specK{\{\heaps = \emptyset, \perms = \emptyset, \hists = \gists, \gist \subseteq \gists \hunion \histo \hunion \mygather{\pending}\}}\\[2pt]
\esc{exchange'}\ v\\[2pt]
\spec{\!\!
\begin{array}{c}
\heaps = \emptyset, \perms = \emptyset, \gist \subseteq \gists \hunion
  \histo \hunion \mygather{\pending}, \\[1pt]    
\exists t\ldot \hists = t \mapsto (v, \res) \hunion \gists, \mathsf{last} (\gist) < t, \twin{t}
\end{array}
\!\!}@\cal E
% \specK{\{\heaps = \emptyset, \perms = \emptyset, \gist \subseteq \gists \hunion \histo \hunion \mygather{\pending}, \hbox{}}\\
% \specK{\exists t\ldot \hists = t \mapsto (v, \res) \hunion \gists, \mathsf{last} (\gist) < t, \twin{t}\}} @ 
\end{array}
}}
\]
%
%\vspace{-10pt}
%
\[
{\small{
\!\!\!
\begin{array}{c}
\spec{\!\!
\begin{array}{c}
\heaps = \emptyset, \perms = \emptyset, \hists = \mathsf{zip}~\ts~\vvs~\acc, \hbox{}\\[1pt]
\sorted{\ts}, \mathsf{zip}~\overline{\ts}~\acc~\vvs \subseteq \histo \hunion \mygather{\pending}
\end{array}
\!\!}
\\[2pt]
\mathtt{ex\_seq}~\vvs'~\acc\\[2pt]
\spec{\!\!\!\!
\begin{array}{c}
\exists \ts'~\acc'\ldot\heaps = \emptyset, \perms = \emptyset, \res =
  \acc~\esc{++}~\acc',\\[1pt] 
\hists = \mathsf{zip}~(\ts~\esc{++}~\ts')~(\vvs~\esc{++}~\vvs')~(\acc~\esc{++}~\acc'), 
 \sorted{(\ts~\esc{++}~\ts')},\\[1pt]
\mathsf{zip}~(\overline{\ts}~\esc{++}~\overline{\ts'})~(\acc~\esc{++}~\acc')~(\vvs~\esc{++}~\vvs') \subseteq 
      \histo  \hunion \mygather{\pending}
\end{array}
\!\!\!\!}@\cal E
\end{array}
}}
\]

The spec for $\esc{exchange'}$ is immediately derived
from~(\ref{tag:exchangespec}) by removing the now-impossible case of
exchange failing, and then framing the history components by $\gists$,
as explained in Section~\ref{sec:background}.
%
The spec for $\esc{ex\_seq}$ is more complicated.  It starts with two
logical variables $\ts$ and $\vvs$. The conjunct
$\hists = \mathsf{zip}~\ts~\vvs~\acc$ in the precondition says that
$\ts$ is the list of time-stamps currently generated by our thread,
and $\vvs$ is the list of values exchanged for those in $\acc$. Here
$\mathsf{zip}$ creates a history out of a list of time-stamps and
values:
%
\[
{\small{
\mathsf{zip}~ts~vs~ws = \left\{%
\begin{array}{l}
t \mapsto (v, w) \hunion \mathsf{zip}~\ts'~\vvs'~\ws', \\
\hphantom{\emptyset,}\ \mbox{if $\ts=t\,{::}\,\ts', \vvs=v\,{::}\,\vvs', \ws=w\,{::}\,\ws'$}\\
\emptyset, \mbox{if $\ts = \vvs = \ws = \mathsf{nil}$}\\
\mbox{undefined}, \mbox{otherwise}
\end{array}\right.
}}
\]
%
%\[
%\begin{array}{c}
%\mathsf{zip}~(t\,{::}\,ts)~(v\,{::}\,vs)~(w\,{::}\,ws) = 
%  t \mapsto (v, w) \hunion \mathsf{zip}~ts~vs~ws\\
%\mathsf{zip}~\mathsf{nil}~\mathsf{nil}~\mathsf{nil} = \emptyset\\
%\mathsf{zip}~ts~vs~ws = \mbox{undefined otherwise}
%\end{array}
%\]
The precondition further assumes that the time-stamps in $\ts$, when
considered as natural numbers, strictly grow ($\sorted{\ts}$), and
that the \emph{other} history contains the ``twin'' history of
$\hists$: $\mathsf{zip}~\overline{\ts}~\acc~\vvs \subseteq \histo
\hunion \mygather{\pending}$. Here $\overline{\ts}$ denotes the list
of twin time-stamps of $\ts$, and $\acc$ and $\vvs$ appear in the
reversed order compared to $\hists$.

The postcondition says that $\esc{ex\_seq}$ adds a list of time-stamps
$\ts'$, which are all larger than time-stamps in $\ts$ and also
strictly grow. Intuitively, this holds because the postcondition of
$\esc{exchange'}$ ensures the time-stamp $t$ is larger than any
time-stamps used in $\hists$, $\histo$ or $\pending$ (by choosing the
logical variable $\gist$ in the spec of $\esc{exchange'}$ to be the
union of $\hists$, $\histo$ and $\mygather{\pending}$). The self
history $\hists$ is extended with the new time-stamps and values:
$\hists =
\mathsf{zip}~(\ts~\esc{++}~\ts')~(\vvs~\esc{++}~\vvs')~(\acc~\esc{++}~\acc')$. Similarly
for the environment history, which must be a twin of $\hists$ by the
resource invariant~(\ref{tag:exchanging}). The result $\res$ appends
an unknown list of values $\acc'$, supplied by interfering threads, to
the starting list $\acc$.

The above spec for $\esc{ex\_seq}$ is also used as its loop
invariant. For use in clients, we restrict it to
$\acc = \ts = \vvs = \mathsf{nil}$, and derive:
%
\[
{\small{
\begin{array}{c}
\specK{\{\heaps = \emptyset, \perms = \emptyset, \hists = \emptyset\}}\\[2pt]
\mathtt{ex\_seq}~vs~\mathsf{nil}\\[2pt]
\spec{\!\!\!
\begin{array}{c}
\exists \ts\ldot \heaps = \emptyset, \perms = \emptyset, 
\mathsf{sorted}~\ts, \hists = \mathsf{zip}~\ts~\vvs~\res,
\\[1pt]
\mathsf{zip}~\overline{\ts}~\res~\vvs \subseteq \histo  \hunion \mygather{\pending}  
\end{array}
\!\!\!}@ \cal E
\end{array}
}}
\]
%
Naming the postcondition above as $Q(\vvs)$, we can now verify the
parallel composition $e$.
%
\[
\tag{\arabic{tags}}\refstepcounter{tags}\label{tag:e}\\
{\small{
\begin{array}{c}
\specK{\{\heaps = \emptyset, \perms = \emptyset, \hists = \emptyset\}}\\[2pt]
%\specK{\{\heaps = \emptyset\hunion\emptyset, \perms = \emptyset\hunion\emptyset, \hists = \emptyset\hunion\emptyset\}}\\
\specK{\{(\heaps = \emptyset, \perms = \emptyset, \hists = \emptyset) \circledast 
  (\heaps = \emptyset, \perms = \emptyset, \hists = \emptyset)\}}\\[2pt]
\begin{array}{c}
\specK{\{\heaps = \emptyset, \perms = \emptyset, \hists = \emptyset\}}\\[1pt]
\mathsf{ex\_seq}~\vvs_1~\mathsf{nil}\\[1pt]
\specK{\{Q(\vvs_1)\}}
\end{array} \parallel
\begin{array}{c}
\specK{\{\heaps = \emptyset, \perms = \emptyset, \hists = \emptyset\}}\\[1pt]
\mathsf{ex\_seq}~\vvs_2~\mathsf{nil}\\[1pt]
\specK{\{Q(\vvs_2)\}}
\end{array}\\[2pt]
\specK{\{Q(\vvs_1) \circledast Q(\vvs_2)\}}\makebox[0pt]{\quad $@ \cal E$}
\end{array}
}}
\]
Unfolding the definition of $\circledast$, the postcondition obtains:
%
%{\small{
\begin{align*}
\exists \ts_1~& \ts_2\ldot \heaps = \emptyset, \perms = \emptyset, \mathsf{sorted}~{\ts_1}, \mathsf{sorted}~{\ts_2},\\
& \hists = \mathsf{zip}~\ts_1~\vvs_1~\res.1 \hunion \mathsf{zip}~\ts_2~\vvs_2~\res.2,\\
& \mathsf{zip}~\overline{\ts_1}~\res.1~\vvs_1 \subseteq \mathsf{zip}~\ts_2~\vvs_2~\res.2 \hunion \histo \hunion\mygather{\pending}, \tag{\arabic{tags}}\refstepcounter{tags}\label{tag:x}\\
& \mathsf{zip}~\overline{\ts_2}~\res.2~\vvs_2 \subseteq \mathsf{zip}~\ts_1~\vvs_1~\res.1 \hunion \histo \hunion\mygather{\pending}. \tag{\arabic{tags}}\refstepcounter{tags}\label{tag:y}
\end{align*}
%}}
%
Intuitively, the values of each \emph{self} component $\heaps$,
$\perms$, $\hists$ from $Q(\vvs_1)$ and $Q(\vvs_2)$ are joined into
the self component of the joined thread. At the same time, the
\emph{other} component $\histo$ of the left thread equals the sum of
the $\hists$ of the right thread, and the $\histo$ of the joining
thread. Thus, the predicates $\mathsf{zip}~\overline{\ts}~\res~\vvs
\subseteq \histo \hunion \mygather{\pending}$ from $Q_1$ and $Q_2$
become $(\ref{tag:x})$ and $(\ref{tag:y})$, respectively.
%\[
%\begin{array}{c}
%\{\heaps = \emptyset, \perms = \emptyset, \chi_s = \emptyset\}\\
%\{\heaps = \emptyset\hunion\emptyset, \perms = \emptyset\hunion\emptyset, \chi_s = \emptyset\hunion\emptyset\}\\
%\begin{array}{c}
%\{\heaps = \emptyset, \perms = \emptyset, \chi_s = \emptyset\}\\
%\mathsf{ex\_seq}~vs_1~\mathsf{nil}\\
%\{\heaps = \emptyset, \perms = \emptyset, \\
%\exists ts_1\ldot \mathsf{sorted}~ts_1, \\
%  \chi_s = \mathsf{zip}~ts_1~vs_1~\res, \\
% \mathsf{zip}~\overline{ts_1}~\res~vs_1 \subseteq \chi_o \hunion \mygather{\pending}\}
%\end{array} \parallel
%\begin{array}{c}
%\{\heaps = \emptyset, \perms = \emptyset, \chi_s = \emptyset\}\\
%\mathsf{ex\_seq}~vs_2~\mathsf{nil}\\
%\{\heaps = \emptyset, \perms = \emptyset, \\
%\exists ts_2\ldot \mathsf{sorted}~ts_2, \\
%  \chi_s = \mathsf{zip}~ts_2~vs_2~\res, \\
%\mathsf{zip}~\overline{ts_2}~\res~vs_2 \subseteq \chi_o \hunion \mygather{\pending}\}
%\end{array}\\
%\{\heaps = \emptyset, \perms = \emptyset, \exists ts_1~ts_2\ldot \mathsf{sorted}~ts_1, \mathsf{sorted}~ts_2,\\
%\chi_s = \mathsf{zip}~ts_1~vs_1~\res.1 \hunion \mathsf{zip}~ts_2~vs_2~\res.2\\
%\mathsf{zip}~\overline{ts_1}~\res.1~vs_1 \subseteq \mathsf{zip}~ts_2~vs_2~\res.2 \hunion \chi_o \hunion\mygather{\pending}\\
%\mathsf{zip}~\overline{ts_2}~\res.2~vs_2 \subseteq \mathsf{zip}~ts_1~vs_1~\res.1 \hunion \chi_o \hunion\mygather{\pending}\}
%\end{array}
%\]

What does this postcondition say? First, the self history of $e$
contains both $\mathsf{zip}~\ts_1~\vvs_1~\res.1$ and
$\mathsf{zip}~\ts_2~\vvs_2~\res.2$. Thus, $\vvs_1$ is exchanged for
$\res.1$, and $\vvs_2$ for $\res.2$. But we additionally want to
conclude that $\res.1 = \vvs_2$ and $\res.2 = \vvs_1$, \ie, the lists
are exchanged for each other, in the absence of interference.

To derive this desired property, we will use the inequalities
$(\ref{tag:x})$ and $(\ref{tag:y})$. Notice that $(\ref{tag:x})$ and
$(\ref{tag:y})$ are ultimately instances of the conjunct
$\gist \subseteq \histo \hunion \mygather{\pending}$ that was part of
the specification~(\ref{tag:exchangespec}). Hence, this example
justifies the use of subjectivity in specs.

First, we apply $\hide$ to the spec~$(\ref{tag:e})$, to limit the
interference on $e$. We pick substitution
$\Phi_1 = [\emptyset/\heaps, \emptyset/\perms, \emptyset/\hists,
g\,{\mapsto}\,\mathsf{null}/\heapj$,
$\emptyset/\pending, \emptyset/\heapo, \emptyset/\permo,
\emptyset/\histo]$,
and obtain (via the $\hide$ rule~\eqref{eq:ehide}):
%
\[
{\small{
\begin{array}{c}
\specK{\{\heaps = g \mapsto \mathsf{null}\}}\\[2pt]
\specK{\{\heaps = g \mapsto \mathsf{null}, \Phi_1 (\heaps = \emptyset, \perms = \emptyset, \hists = \emptyset)\}}\\[2pt]
\hide_{\Phi_1}~~e \\[2pt]
\specK{\{\exists \Phi_2\ldot \heaps = \Phi_2(\heapj), \Phi_2(Q (\vvs_1) \circledast Q(\vvs_2))\}} @ \cal P
\end{array}
}}
\]
%
From the unfolding of $Q(\vvs_1) \circledast Q(\vvs_2)$, we obtain
that $\Phi_2$ must be such that $\heaps$ and $\perms$ map to
$\emptyset$, and $\hists$ maps to $\mathsf{zip}~\ts_1~\vvs_1~\res.1
\hunion \mathsf{zip}~\ts_2~\vvs_2~\res.2$. Moreover, by the general
conditions on $\Phi$, $\permo$ and $\histo$ also map to
$\emptyset$. The variable $\heapj$ maps to some heap which, by
resource invariant~\ref{exP}, must contain the pointer $g$. Hence,
from $\heaps = \Phi_2(\heapj)$, we derive $g \in
\mathsf{dom}~\heaps$. Also by resource invariant~\ref{matched},
$\mathsf{dom}\ \pending = \perms \hunion \permo$, and thus $\pending$
too must map to $\emptyset$. Hence, we can weaken and simplify the
postcondition into:
\begin{align*}
\exists \ts_1~\ts_2\ldot & g \in \mathsf{dom}\ \heaps, \sorted{\ts_1}, \sorted{\ts_2}\\
& \mathsf{zip}~\overline{\ts_1}~\res.1~\vvs_1 \subseteq \mathsf{zip}~\ts_2~\vvs_2~\res.2 \tag{\ref{tag:x}'}\label{tag:x'}\\
& \mathsf{zip}~\overline{\ts_2}~\res.2~\vvs_2 \subseteq \mathsf{zip}~\ts_1~\vvs_1~\res.1 \tag{\ref{tag:y}'}\label{tag:y'}
\end{align*}
%
Now, histories are finite maps from time-stamps to value pairs. Hence
$(\ref{tag:x'})$ implies that the time-stamps from $\overline{\ts_1}$
are included in the time-stamps of $\ts_2$. Similarly from
$(\ref{tag:y'})$, $\overline{\ts_2}$ is included in $\ts_1$. Thus,
$\ts_1$ and $\ts_2$ are of same size, and moreover, since they both
are strictly increasing, it must be that $\ts_2 =
\overline{\ts_1}$. Therefore, $(\ref{tag:x'})$ can be strengthened
into an equality: 
%
\[
\mathsf{zip}~\overline{\ts_1}~\res.1~\vvs_1 = \mathsf{zip}~\ts_2~\vvs_2~\res.2
\]
%
As the time-stamps in $\overline{\ts_1}$ and $\ts_2$ appear
in the same order, it must be $\res.1 = \vvs_2$ and $\vvs_1 = \res.2$,
leading to the desired (\ref{tag:hidespec}).  
%
% \an{I omitted here some steps. In particular, that $\sorted{\ts_1}$
%   implies $\sorted{\overline{\ts_1}}$ is not straightforward, and
%   requires also deriving that there are no twins in $\ts_1$. That can
%   all been done (and has been done), but should we present it?}




\section{Specifying Counting Networks in Hoare Style}
\label{sec:counting}

We now show how one can use histories and subjectivity to specify
another class of non-linearizable objects---\emph{counting networks}.

Counting networks are a special case of \emph{balancing networks}
introduced by Aspnes \etal~\cite{Aspnes-al:JACM94}, themselves
building on sorting networks~\cite{Ajtai-al:STOC83}. Counting networks
implement concurrent counters in a way free from synchronization
bottlenecks.
%
The key idea of counting networks is to decompose the workload between
\emph{several} counters, so that each of them is responsible for a
disjoint set of values. A thread trying to perform an incrementation
first approaches the \emph{balancer}, which is a logical ``switch''
that ``directs'' the thread, \ie, provides it with the address of the
counter to increment.
%
The balancers make counting networks' operations
\emph{non-linearizable}, as in the presence of interference the
results of increments might be observed out of order.
%
% \wrt~a sequential specification.

\begin{figure}%[18]{r}{4cm} 
\begin{tabular}{c@{\ \ \ \ \ \ }c}
\begin{minipage}[c]{2.5cm}
\includegraphics[width=2.1cm]{counter.pdf} 
\end{minipage}
&
\begin{minipage}[l]{4.9cm}
\centering
{\small{
\[
\begin{array}{rl}
\Num{1} & \esc{getAndInc()} : \esc{nat}~=~\esc{\{}  \\[2pt] 
\Num{2} & ~~~~ b \Asgn \esc{flip(}\bal\esc{)};\\[2pt]
\Num{3} & ~~~~ \res \Asgn \esc{fetchAndAdd2(}c_b\esc{)};\\[2pt]
\Num{4} & ~~~~ \kw{return}~\res~\esc{\}}
\end{array}
\]
}}
\end{minipage} 
\\
(a) & (b)
%
\end{tabular}
%
\caption{Simple counting network: intuition and pseudo-code.}
\label{fig:counter-fig} 
\end{figure}

Figure~\ref{fig:counter-fig} presents a schematic outline (a) and a
pseudo-code implementation (b) of a counting network with a single
balancer.
%
The implementation contains three pointers: the balancer $\bal$, which
stores either 0 or 1, thus directing threads to the shared pointers
$c_0$ or $c_1$, which count the even and odd values,
respectively. Threads increment by calling \code{getAndInc}, which
works as follows. It first atomically changes the bit value of the
balancer via a call to atomic operation \code{flip} (line 2). The
\code{flip} operation returns the \emph{previous} value $b$ of the
balancer as a result, thus determining which of the counters, $c_0$ or
$c_1$, should be incremented. The thread proceeds to atomically add 2
to the value of $c_b$ via \code{fetchAndAdd2} (line 3). The old value
of $c_b$ is returned as the result of the procedure.\footnote{In the
  counting network from Figure~\ref{fig:counter-fig}, the balancer
  itself might seem like a contention point. However, the \code{flip}
  operation is much less expensive than \code{CAS} as a
  synchronization mechanism. The performance can be further improved
  by constructing a \emph{diffracting tree} of several
  balancers~\cite[\S 12.6]{Herlihy-Shavit:08}, but we do not consider
  diffracting trees here.}

Assuming that $c_0$ and $c_1$ are initialized with $0$ and $1$, it is
easy to see that in a single-threaded program, the network will behave
as a conventional counter; that is, consecutive invocations of
\code{getAndInc} return consecutive nats.
%
However, in the concurrent setting, \code{getAndInc} may return
results out of order, as follows. 
%
% which historically led to the definition of quiescent
% consistency~\cite[\S 3.3]{Herlihy-Shavit:08} in order to specify the
% network's concurrent behavior.

\vspace{3pt}
\begin{example}
\label{ex:t1t2}
%
Consider two threads, $T_1$ and $T_2$ operating on the network
initialized with $\bal\,{\mapsto}\,0$, $c_b\,{\mapsto}\,b$. $T_1$
calls \code{getAndInc} and executes its line~2 to set $\bal$ to 1. It
gets suspended, so $T_2$ proceeds to execute lines~2 and~3, therefore
setting $\bal$ back to $0$ and returning $1$. While $T_1$ is still
suspended, $T_2$ calls \code{getAndInc} again, gets directed to $c_0$,
and returns 0, after it has just returned 1.
%
\end{example}
\vspace{3pt}

\noindent

This out-of-order behavior, however, is not random, and can be
precisely characterized as a function of the number of threads
operating on the
network~\cite{Afek-al:OPODIS10,Jagadeesan-Riely:ICALP14}. In the rest
of this section and in Section~\ref{sec:qc-client}, we show how to
capture such bounds precisely using auxiliary state of (subjective)
histories in a client-sensitive manner. As a form of road map, we
first list the desired requirements for the spec of \code{getAndInc},
%
adapting the design goals of the criteria, such as QC, QQC and
QL~\cite{Aspnes-al:JACM94,Afek-al:OPODIS10,Jagadeesan-Riely:ICALP14},
which we will proceed to verify formally, and then employ in
client-side reasoning.
%
\vspace{2pt}
\begin{itemize}

\item \textbf{R1:} Two different calls to \code{getAndInc}
  should return distinct results (\emph{strong concurrent
    counter semantics}).

\item \textbf{R2:} The results of calls to \code{getAndInc},
  separated by a period of quiescence (\ie, absence of interference),
  should appear in their sequential order (\emph{quiescent
    consistency}).

\item \textbf{R3:} The results of two sequential calls $C_1$ and
  $C_2$, in a single thread should be out of order by no more than
  $2\ N$, where $N$ is the number of interfering calls that overlap
  with $C_1$ and $C_2$ (\emph{quantitative quiescent
    consistency/quasi-linearizability}).
%\an{Can we chose one of the two here: either qqc or ql?}

\end{itemize}

\begin{comment}
\noindent 
In the rest of this and in the next section, we will illustrate how to
achieve all these goals by employing (a) \emph{subjective auxiliary
  state} as a mechanism for \emph{{explicitly referring to and
    quantifying over}} the effects of currently interfering threads
(via its \emph{other}-component) in combination with (b)
\emph{histories}, providing a way to \emph{{logically record relevant
    pieces of state information}} (including witnessed interference)
in Hoare-style program specifications.
\end{comment}

%\vspace{2pt}
%\lipsum[1]

\subsection{Formalizing the counting network}
\label{sec:counting-intuition}

To formalize the necessary invariants, we elaborate the counting
network with auxiliary state: \emph{tokens} (isomorphic to nats) and
\emph{interference-capturing histories}.

A \emph{token} provides a thread that owns it with the right to
increment an appropriate counter~\cite{Aspnes-al:JACM94}. In our
example, a thread that performs the \code{flip} in line 2 of
\code{getAndInc} will be awarded a token which it can then spend to
execute \code{fetchAndAdd2}.
%
Thus, any individual token represents a ``pending'' call to
\code{getAndInc}, and the set of unspent tokens serves as a bound on
the out-of-order behavior that the network exhibits. We will have four
different auxiliary variables tracking the different classes of
tokens: $\tkns^0$ and $\tkns^1$ keep the tokens owned by the
\emph{self} thread, administering access to $c_0$ and $c_1$,
respectively. Similarly, $\tkno^0$ and $\tkno^1$ keep the tokens owned
by the \emph{other} thread. We abbreviate $\tkn^i = \tkns^i \hunion
\tkno^i$, $i=0,1$, $\tkns = \tkns^0 \hunion \tkns^1$, $\tkno = \tkno^0
\hunion \tkno^1$.

Figure~\ref{fig:chist} illustrates a network with three \emph{even}
tokens: $x^0, y^0, z^0 \in \tkn^0$, held by threads that will
increment $c_0$, and one \emph{odd} token $u^1 \in \tkn^1$, whose
owner will increment $c_1$.
%
%\an{Removed: We also point out here that token names (and their
%  uniqueness) will be of critical importance for the specifications we
%  give further. This point was never emphasized later on, so why
%  bother drawing attention to it.}

\begin{figure}
\centering
\includegraphics[width=8.2cm]{chist.pdf}      
\caption{Tokens and histories of the simple counting network.}
\label{fig:chist}
\end{figure}

A \emph{history} of the counting network is an auxiliary finite map,
consisting of entries of the form
\[
\tag{\normalsize{\arabic{tags}}}\refstepcounter{tags}\label{eq:cn-entry} 
%
t \mapsto (\tknh^0, \tknh^1, z)
\]
Such an entry records that the value $t$ has been written into an
appropriate counter ($c_0$ or $c_1$, depending on the parity of $t$),
at the moment when $\tkn^0$ and $\tkn^1$ held values $\tknh^0$ and
$\tknh^1$, respectively. Moreover, in order to write $t$ into a
counter, the token $z$ was spent by the thread invoking
\code{getAndInc}. We will refer to $z$ as the \emph{spent}
token. Notice that the entries in the history contain tokens held by
both \emph{self} and \emph{other} threads. Thus, a history captures
the behavior of a thread subjectively, \ie, as a function of the
behavior of the threads interfering with it.

Similarly to tokens, network histories are split into four different
auxiliary variables, which are manipulated differently by our
proofs. Thus, we have $\hists^0$ and $\hists^1$ to track the even and
odd counter updates performed by the \emph{self} thread, and dually
$\histo^0$ and $\histo^1$ for the \emph{other} thread. We abbreviate
$\hist^i = \hists^i \hunion \histo^i$, $i = 0,1$, and $\hists =
\hists^0 \hunion \hists^1$, and $\histo = \histo^0 \hunion \histo^1$.

Figure~\ref{fig:chist} illustrates a moment in network's history and
how it relates to the state of the counters. Only $0$ has been written
to $c_0$ so far (upon initialization), hence $\hist^0$ only contains
an entry for $t = 0$ (we ignore at the moment the \emph{contents} of
the history entries). On the other hand, $\hist^1$ has entries for $1$
and $3$, because after initialization, one thread has increased $c_1$.
%
The gray boxes indicate that $0$ and $3$ are the current values of
$c_0$ and $c_1$, and thus also the latest entries in $\hist^0$ and
$\hist^1$, respectively. In particular, these values will be returned
by the next invocations of \code{fetchAndAdd2}. The dashed boxes
correspond to the entries to be added to the history by the currently
running threads holding the tokens $x^0$, $y^0$, $z^0$,
$u^1$. However, as thread scheduling is non-deterministic, we cannot
predict which of the tokens will be spent to, say, write 2 into $c_0$
(it may be any of the even tokens). On the other hand, in the absence
of other threads joining the network, we know that token $u^1$ will be
spent to write $5$ into $c_1$.

\subsubsection{Resource invariants of the counting network}
\label{sec:count-netw-invar}
We next formally list the invariants that describe the interdependence
between the various components of the real and auxiliary state. In
addition to $\tkn$ and $\hist$ which come in flavors private to
\emph{self} and \emph{other} threads, we require the following joint
variables: (1) $\heapj$ which stands for the joint heap of the
network, and (2) $b_\joint$, $n^0_\joint$ and $n^1_\joint$ which stand
for the contents of $\bal$, $c_o$ and $c_1$, respectively. 

The main invariant of the network relates the number of tokens, the
size of histories and the value of the balancer as follows.
%
\[
\tag{\normalsize{\arabic{tags}}}\refstepcounter{tags}\label{cn:si} 
%
|\hist^0| + |\tkn^0| =
|\hist^1| + |\tkn^1| + b_\joint
\]
%
The equation directly motivates our design of the auxiliary state, and
leads to establishing the requirements \textbf{R1}--\textbf{R3}. It
formalizes the intuition that out-of-order anomalies of the counting
network appear if one of the two counters is too far ahead of the
other one.
%
The invariant~(\ref{cn:si}) provides a bound on such a situation. One
counter can get ahead temporarily, but then the other counter must
have a number of threads waiting to spend their tokens, and increment
it. Thus, the other counter will eventually catch up.

The approaches such as quiescent and quantitative quiescent
consistency describe this situation by referring to the number of
\emph{unmatched} call events in an event
history~\cite{Derrick-al:FM14,Jagadeesan-Riely:ICALP14}. In contrast,
we formalize this property by relying on auxiliary state. In our case,
the sets of tokens $\tknh^0$ and $\tknh^1$ recorded in the entry for
the number $t$ determine the environment's capability to add new
history entries, and thus ``run ahead'' or ``catch up'' after $t$ has
been returned. Such auxiliary state will let us directly specify the
network's behavior in the moments of quiescence (\ie,~when $\tkno$ is
empty), but also \emph{quantitatively} bound the out-of-orderness as a
function of $\tkno$.

We next list the other resource invariants.
\vspace{2pt}
\begin{enumerate}[label=(\roman*)]

% %
% \[
% \tag{\normalsize{\arabic{tags}}}\refstepcounter{tags}\label{eq:cn-states} 
% {\small
% \begin{array}{r@{\ }c@{\ }l} 
% {\!\!\!\!\!\!\!\!}W_{\ccon} & \!\eqdef & \exists \tkns~\tkno~ \hists~\histo~b~n_0~n_1\ldot 
% %  
% \qcl \spts (\tkns, \hists)\aand \qcl \opts (\tkno, \histo) 
%   \\[4pt] 
% &\aand & \qcl \jpts \bal \hpts b \hunion c_0 \hpts n_0 \hunion c_1 \hpts n_1   
%  \aand \hvalid~(\hists \hunion \histo)         \\[3pt] 
% &\aand & \SI~\tkn^0~\tkn^1~\hist^0~\hist^1~b ~~~~\aand
%          \CI~\hist^0~\hist^1~n_0~n_1 \\[3pt]  
% &\aand & \TI~\tkn^0~\tkn^1~(\hist^0 \hunion \hist^1)\aand \AI~\hist^0~\hist^1~\tkn^0~\tkn^1~n_0~n_1.
% \end{array}
% }
% \]
%

% where $\hist^i = (\hists \hunion \histo)^i$ and
% $\tkn^i = (\tkns \hunion \tkno)^i$ for $i \in \set{0,1}$.

\item\label{cn:state} $\heapj = \bal \mapsto b_\joint \hunion c_0 \mapsto n^0_\joint
  \hunion c_1 \mapsto n^1_\joint$.

\item\label{cn:hvalid} The histories contain disjoint time-stamps. % (\ie $\hists \hunion \histo$ is always defined);
 
% The state-space invariant $W_{\ccon}$ fixes the auxiliary self/other
% components to be pairs of tokens and histories $(\tkns, \hists)$ and
% $(\tkno, \histo)$, which are held/contributed by the thread and its
% environment, correspondingly. The invarian $\hvalid~(\hists \hunion
% \histo)$ ensures that at any moment 
% %
% The joint part of the state contains the pointers $\bal$, $c_0$ and
% $c_1$, and the relations between all these components are specified by
% the invariants $\SI$, $\CI$, $\TI$ and $\AI$.

\item\label{cn:ci} 
%
  The history $\hist^0$ (resp. $\hist^1$) contains \emph{all} even
  (resp. odd) values in the interval $[0, n^0_\joint]$ (resp. $[1,
    n^1_\joint]$). In other words, the network does not ``skip''
  values. This further automatically ensures that $n^0_\joint$ and
  $n^1_\joint$ are the last time-stamps in $\hist^0$ and $\hist^1$,
  respectively.

\item\label{cn:ti}  
%
  $\tkn^0$, $\tkn^1$ and $\Tomb~(\hists \hunion \histo)$ contain
  mutually disjoint tokens, where $\Tomb~(t \mapsto (\tknh^0,\tknh^1,
  z) \hunion \hist') = \{z\} \hunion \Tomb~\hist'$, and
  $\Tomb~\emptyset = \emptyset$. In other words, a spent token
  never appears among the ``alive'' ones (\ie, in $\tkn^0 \hunion
  \tkn^1$).

%As a consequence, $\Tomb~(\hists \hunion \histo)$ is always defined.

\item\label{cn:ti1}
%
  $t \mapsto (\tknh^0, \tknh^1, z) \subseteq \hists \hunion \histo
  \implies z \in \tknh^0 \hunion \tknh^1$. \\[-7pt]

\item\label{cn:ai} 
%
For any $t$, $\tknh^0$, $\tknh^1$, $z$: \\[-7pt]
% 
  \begin{itemize}
  \item   $t \hpts (\tknh^0, \tknh^1, z) \subseteq \hist^0 ~\implies$\\[2pt]
    $t + 2\ |\tkn^0 \cap \tknh^0| < n^1_\joint + 2\ |\tkn^1 \cap
    \tknh^1| + 2$, and \\[-7pt]
  \item
    $t \hpts (\tknh^0, \tknh^1, z) \subseteq \hist^1 ~\implies$\\[2pt]
    $t + 2\ |\tkn^1 \cap \tknh^1| < n^0_\joint + 2\ |\tkn^0 \cap
    \tknh^0| + 2$.
  \end{itemize}
%
\end{enumerate}
\vspace{2pt}

We comment on the invariant~\ref{cn:ai}, as it is the only non-trivial
one. This invariant provides quantitative information about the
network history by relating the actual ($n^0_\joint$, $n^1_\joint$)
and the past ($t$) counter values, via the current amount of
interference ($\tkn^0$, $\tkn^1$) and the snapshot interference
($\tknh^0$, $\tknh^1$). 
%
%Whereas~(\ref{cn:si}) provides bounds for the network's present,
%\ref{cn:ai} does so for the counter values \emph{from the past}. 
%
To explain~\ref{cn:ai}, we resort to the intuition provided by the
following equality, which, however, being \emph{not quite valid},
cannot be used as an invariant, as we shall see. Focusing on the
first clause in~\ref{cn:ai}, if
$t \mapsto (\tknh^0, \tknh^1, z) \subseteq \hist^0$, then,
intuitively:
%
\[
{\small{
t + 2\ |\tknh^0 \setminus \tkn^0 | + 2\ |\tkn^0 \cap \tknh^0| =
n^1_\joint + 2\ |\tkn^1 \cap \tknh^1| + (2 b_\joint - 1)
}}
\]
%
The equality says the following. When $t$ is snapshot from $c_0$ and
placed into the history $\hist^0$, the set of outstanding even tokens
was $\tknh^0$. By the present time, $c_0$ has been increased
$|\tknh^0 \setminus \tkn^0|$ times, each time by $2$, thus
$n^0_\joint = t + 2\ |\tknh^0 \setminus \tkn^0|$. What is left to add
to $c_0$ to reach the \emph{period of quiescence}, when no threads
interfere with us, is $2\ |\tknh^0 \cap \tkn^0|$. Similar reasoning
applies to $c_1$. It is easy to see at the period of quiescence, $c_0$
and $c_1$ differ by $2 b_\joint - 1$; that is, the counter pointed to
by $\bal$ is behind by $1$. However, the equality is invalid, as
$b_\joint$ can be read off only in the present, whereas the
``intuitive'' reasoning behind the equality requires a value of
$b_\joint$ from a quiescent period \emph{in the future}. Hence, in
order to get a valid property, we bound $2 b_\joint - 1$ by 2. For
simplicity, we even further weaken the bounds by dropping
$|\tknh^0 \setminus \tkn^0|$ to obtain~\ref{cn:ai}; as it will turn
out, even such a simpler bound will suffice for proving
\textbf{R1}--\textbf{R3}.

% in Section~\ref{sec:qc-client}.

%As already aparent from our explanation, the invariant gives us a way
%to formally model when the network is in the period of quiescence, as
%required in \textbf{R2}, which we verify in
%Section~\ref{sec:qc-client}.
%%

\subsubsection{Transitions of the counting network}
\label{sec:count-netw-prot}
The STS $\cal C$ for the counting network admits the following two
transitions between states. They lift the atomic operations
\code{flip} and \code{fetchAndAdd2} from
Figure~\ref{fig:counter-fig}~(b), to work with auxiliary state.

%\noindent
\paragraph{Flipping transition}
%
changes the bit value $b_\joint$ of $\bal$ to the complementary one,
$1 - b_\joint$.
%
It also generates a token $z$ (of parity $b_\joint$) and stores it
into $\tkns$. The token is fresh, \ie, distinct from all alive and
spent tokens in $\tkns \hunion \tkno \hunion{\Tomb~(\hists
  \hunion \histo)}$.
%
% \[
% \small{
% \!\!\!\!
% %
% \begin{array}{r@{\ }c@{\ }l@{\ \ }c}
%   \tauflip & \eqdef &\qcl \jpts (\bal \hpts b \hunion c_0 \hpts n_0
%                       \hunion c_1 \hpts n_1)\aand \qcl \spts (\tkns, \hists) & 
%   \\[3pt]
%            &\rightsquigarrow &
%               \qcl \jpts (\bal \hpts (1-{b}) \hunion c_0 \hpts n_0
%               \hunion c_1 \hpts n_1) \aand \\[2pt]
%            && \qcl \spts (\tkns \hunion \set{\tfr{\tkns \hunion \tkno
%               \hunion{\Tomb~(\hists \hunion \histo)}}{b}}, \hists)
% \end{array}
% }
% \]
%\noindent
\paragraph{Incrementing transition}
%
takes a token $z$ as input. Depending on the parity $i$ of $z$, it
atomically increases the value $n^i_\joint$ of $c_i$ by two, while
simultaneously removing $z$ from $\tkns$ (thus, the precondition is
that $z \in \tkns$). The transition adds the entry $(n^i_\joint + 2)
\hpts (\tkn^0, \tkn^1, z^i)$ to $\hists$, thus snapshoting the values
of $\tkn^0$ and $\tkn^1$ from the pre-state.
%
% \[
% \small{
% \!\!\!\!
% %
% \begin{array}{r@{\ }c@{\ }l@{\ \ }c}
%   \tauadd({z^i}) & \eqdef &\qcl \jpts (\bal \hpts b \hunion c_i \hpts n_i
%                       \hunion c_{1-{i}} \hpts n_{1-i}) \aand \\[2pt]
% & &\qcl \spts (\tkns \hunion \set{z^i}, \hists) & 
%   \\[3pt]
%   & \rightsquigarrow &\qcl \jpts (\bal \hpts b \hunion c_i \hpts
%                             (n_i + 2)\hunion c_{1-{i}} \hpts n_{1-i}) \aand \\[2pt]
% & &\qcl \spts (\tkns, \hists \hunion (n_i + 2) \hpts (\tkns, \tkno, z^i))
% \end{array}
% }
% \]

%As required by FCSL metatheory, neither of the transitions modifies
%the other-state.
%
It is easy to check that both transitions preserve the state-space
invariants~(\ref{cn:si}), \ref{cn:state}--\ref{cn:ai}, and that their
effect on real state (with auxiliary state erased) are those of
\code{flip} and \code{fetchAndAdd2}.

\subsection{A Hoare spec for \texttt{getAndInc}}
\label{sec:spec-gaa}

We next provide a Hoare-style spec for \code{getAndInc}, and show how
it can be inferred from the specs of \code{flip} and
\code{fetchAndAdd2}, defined further in this section.  We use the
logical variable $\ikn$ and its variants to range over token sets, and
$\gist$ to range over histories.
%
\begin{comment}
In order to formulate these specs in an intuitive way, we will first
introduce some new notational conventions.

% \paragraph{Local naming conventions} 
% \label{sec:notat-state-proj}

In our Hoare-style assertions about network states, we will keep
referring to components and values of the \emph{current} state, which
is being constrained, as $\tkns$, $\hists$, \etc.
%
We will refer to a state's sets of even and odd tokens as $\tkn^0$ and
$\tkn^1$ and to actual values of its $c_0$ and $c_1$ as $n_0$
and~$n_1$.
%
For the ghost (logical) variables, appearing in Hoare triples, $\gist$
will range over histories, and $\ikn$ will range over sets of tokens.
\end{comment}
%
% We will use the logical assertion $\This{s}$ to capture the current
% state via a logical variable $s$. For stability, $\This{s}$ will hold
% on \emph{any} state, reachable from the \emph{fixed} state $s$ by
% taking an arbitrary number of transitions $\tauflip$ and $\tauadd$,
% executed by the environment (\ie, modifying \emph{other} and
% \emph{joint} parts of the state).
% %
% For such fixed ``snapshot'' states $s$, we will use the dot-notation,
% reminiscent to addressing object fields in Java, when referring to
% their components.
% %
% For instance, we will refer to the sets of tokens in self/other
% auxiliary state of $s$ as $s.\tkns$ and $s.\tkno$.
% %
% The introduction of $\Thisz$ is done any by weakening any state
% assertion $P$ to to $\exists s \ldot (\This{s}, P)$, where $s$ is
% fresh in $P$.
% %
% Indeed, all assertions about \emph{self}-components of the current
% state can be rephrased in terms of~$s$.
%
%\subsubsection{The spec of \code{getAndInc} and its components}
%\label{sec:spec-codeg-its}
%
%We ascribe the following Hoare-style spec to \code{getAndInc}:
%
\[
%
\tag{\normalsize \arabic{tags}}\refstepcounter{tags}\label{eq:qc-spec}
{\small
\!\!\!\!\!\!\!\! 
\begin{array}{c}
  \spec{\!\!
  \begin{array}{c}
    \tkns = \emptyset,
    \hists = \gists,
    \gisto \subseteq \histo,\\[2pt]
    \ikno \subseteq \tkno \hunion (\Tomb~\histo \setminus \Tomb~\gisto)
  \end{array}
  \!\!}
  \\\\[-6pt]
  \texttt{getAndInc()}
  % 
  \\[3pt]
  \spec{\!\!\!
  \begin{array}{c}
    \exists \iknh^0~\iknh^1~z \ldot \tkns = \emptyset, 
    \hists = \gists \hunion (\res + 2) \hpts (\iknh^0, \iknh^1,z), 
    \\[2pt]
    \gisto \subseteq \histo, \ikno \subseteq \tkno \hunion (\Tomb~\histo \setminus \Tomb~\gisto), 
    \\[2pt]
    \last~(\gists \hunion \gisto) < 
    \res + 2 + 2\ (|\iknh^0 \cap \ikno^0| + |\iknh^1 \cap
    \ikno^1|), 
    \\[2pt]
    \happrox~(\gists \hunion \gisto)~\res~\iknh^0~\iknh^1~z
  \end{array} 
  \!\!\!}@\ccon
%
\end{array}
}
\]
%
The precondition starts with an empty token set ($\tkns = \emptyset$),
and hence by framing, any set of tokens. The initial self-history
$\hists$ is set to an arbitrary $\gists$.\footnote{Alternatively, we
  could have also taken $\hists = \emptyset$, but the clients will
  require generalizing to $\hists = \gists$ by the FCSL's frame
  rule~\eqref{eq:frame}. To save space and simplify the discussion, we
  immediately frame \wrt the auxiliary $\hists$. Our examples do not
  require such client-side framing \wrt~$\tkns$.} The precondition
records the \emph{other} components of the initial state as
follows. First, $\gisto$ names (a subset of) $\histo$, to make it
stable under interference, as in Section~\ref{sec:overview}. Next, we
use $\ikno$ to name the (subset of) initially live tokens
$\tkno$. However, as $\tkno$ may shrink due to other threads spending
tokens, simply writing $\ikno \subseteq \tkno$ is unstable. Instead,
we write $\ikno \subseteq \tkno \hunion (\Tomb~\histo \setminus
\Tomb~\gisto)$ to account for the tokens spent by other threads as
well. The set $\tkno \hunion (\Tomb~\histo \setminus \Tomb~\gisto)$
only grows under interference, as new live tokens are generated, or
old live tokens are spent, making the inclusion of $\ikno$
stable.

%

The postcondition asserts that the final token set $\tkns$ is also
empty (\ie, the token that \code{getAndInc} generates by \code{flip},
is spent by the end). The history $\hists$ is increased by an entry
$(\res + 2) \hpts (\iknh^0, \iknh^1, z)$, corresponding to writing the
value of the result (plus two) into one of the network's counters,
snapshoting the tokens of that moment into $\iknh^0$ and $\iknh^1$,
and spending the token $z$ on the write. $\gisto$ is a subset of the
new value of $\histo$, and $\ikno$ is a subset of the new value of
$\tkno \hunion (\Tomb~\histo \setminus \Tomb~\gisto)$, by the already
discussed stability.

The next inequality describes where the history entry for $\res + 2$
is placed \wrt~the pre-state history $\gist = \gists \hunion
\gisto$. $\gist$ may have gaps arising due to out-of-order behavior of
the network, and $\res + 2$ may fill one such gap. However, there is a
bound on how far $\res$ (and hence $\res+2$) may be from the tail of
$\gist$, which we express as a function of the token-sets $\ikno$,
$\iknh^0$, $\iknh^1$. The intuition is similar to that of the bounds
in~\ref{cn:ai}. To explain it, let us assume that $\res$ is written
into $c_1$ and the last entry of $\hist$ (let us call it $t$) was
written into $c_0$. Then, we have a similar ``equation'' as in the
explanation in~\ref{cn:ai}:
\[
t + 2\ |\ikno^0 \setminus \iknh^0| + 2\ |\iknh^0 \cap \ikno^0| = \res
+ |\iknh^1 \cap \ikno^1| + (2 b_\joint -1)
\]
To advance $c_0$ to the moment when $\iknh^0$ and $\iknh^1$ were
recorded, we need to increase $t$ by $2\ |\ikno^0 \setminus
\iknh^0|$. After that, to advance both $c_0$ and $c_1$ to a quiescent
period, we have to spend the tokens in $|\iknh^0 \cap \ikno^0|$ (for
$c_0$) and $|\iknh^1 \cap \ikno^1|$ (for $c_1$). In the quiescent
period, $c_0$ and $c_1$ differ by $2 b_\joint - 1$. Moving $|\iknh^0
\cap \ikno^0|$ to the other side of the equation (while not changing
the sign), omitting $|\ikno^0 \setminus \iknh^0|$, and bounding the
value of $2 b_\joint - 1$ from above by $2$, we get:
\[
t < \res + 2\ |\iknh^0 \cap \ikno^0| + 2 \ |\iknh^1 \cap \ikno^1| + 2
\]
which we use in~(\ref{eq:qc-spec}). Being symmetric in $\ikn^0$ and
$\ikn^1$, the inequality has the pleasant property that it also holds
in the other three cases: when $\res$ is written into $c_0$ and $t$
into $c_1$, and when both $\res$ and $t$ are written into the same
counter, $c_0$ or $c_1$.

% The predicate $\strapprox$, stated next, summarizes several properties
% of the result and the newly introduced history entry, which will be
% crucial for reasoning about clients in Section~\ref{sec:qclients}:
% %
% \[
% \tag{\normalsize \arabic{tags}}\refstepcounter{tags}\label{eq:strapprox}
% %
% \!\!\!\!
% {\small{
% \begin{array}{l}
% \strapprox~\gists~\gisto~\ikno~m_0~m_1~\res~\iknh^0~\iknh^1~z
% ~\eqdef
% \\[2pt]
% ~~~~~~~~~~~~~~~~~~ \sapprox~\ikno~m_0~m_1~\res~\iknh^0~\iknh^1, \\[2pt]
% ~~~~~~~~~~~~~~~~~~ \happrox~(\gists \hunion \gisto)~\res~\iknh^0~\iknh^1~z,\\[2pt]
% ~~~~~~~~~~~~~~~~~~ \tapprox~(\gists \hunion \gisto)~\iknh^0~\iknh^1~z
% \end{array}
% }}
% \]
% %
% \[
% %
% \tag{\normalsize \arabic{tags}}\refstepcounter{tags}\label{eq:sapprox}
% %
% {\small{
% \begin{array}{l}
% \sapprox~\ikn~m_0~m_1~\res~\iknh^0~\iknh^1 ~\eqdef \\[2pt]
% %
% ~~~~  m_0 < \res + 2 + 2 \times (|\iknh^0 \cap \ikn^0| + |\iknh^1 \cap
%   \ikn^1|), \\[2pt]
% ~~~~ m_1 < \res + 2 + 2 \times (|\iknh^0 \cap \ikn^0| + |\iknh^1 \cap
%   \ikn^1|)
% \end{array}
% \hfill
% }}
% \]
% 

%\noindent
Finally, the predicate $\happrox$ provides some further bounds that we
will need in the proofs of the client code's properties.
%
\[ 
%
\tag{\normalsize \arabic{tags}}\refstepcounter{tags}\label{eq:happrox}
%
\!\!\!\!\!
{\small{
\begin{array}{l}
\happrox~\gist~\res~\iknh^0~\iknh^1~z \eqdef \hbox{}\\
% (\iknh^0 \hunion \iknh^1) \cap \Tomb~\gist = \emptyset,\\[2pt]
~~~~~~ \iknh^0 \hunion \iknh^1 \subseteq \tkno \hunion (\Tomb~\histo) \hunion
  \set{z},\\[2pt]
~~~~~~ \forall t~\ikn^0~\ikn^1 \ldot t \hpts (\ikn^0, \ikn^1, -) \subseteq \gist ~\implies\\[2pt]
~~~~~~~~~~~~
  z \notin \ikn^0 \hunion \ikn^1,~~ 
  t < \res + 2 + 2 \ (|\iknh^0 \cap \ikn^0| + |\iknh^1
  \cap \ikn^1|)
\end{array}
}}
\]
%
When instantiated with $\gist = \gists \hunion \gisto$, $\happrox$
says the following. The token set $\iknh^0 \hunion \iknh^1$ snapshot
when $\res+2$ was committed to history, is a subset of all the tokens
in post-state, including the live ones ($\tkno$), and
spent ones ($\Tomb~\histo \hunion \{z\}$).
%
Moreover, if $t$ is an entry in $\gist$, with contents $(\ikn^0,
\ikn^1, -)$, then: (1) $z \notin \ikn^0 \hunion \ikn^1$, because $z$
is a token generated when \code{getAndInc} executed
\code{flip}. Hence, $z$ must be fresh \wrt~any token sets used in the
pre-state history $\gist$; and (2) $t$, $\ikn^0$ and $\ikn^1$ must
satisfy the same bounds \wrt~$\res+2$, as those described for the last
history entry and sets $\ikno^0$ and $\ikno^1$.


%\noindent

%
% \begin{comment}
% \paragraph{Why the spec~\eqref{eq:qc-spec} is stable?}
% \label{sec:why-spec-eqrefeq:qc}

% The stability of the spec we ascribed to \code{getAndInc} follows from
% the following observations. First, all clauses in the pre- and
% postconditions that contsrain only \emph{self}-components of the state
% (\eg, $s.\hists = \gists$ or $\tkns = \emptyset$) are stable, since
% they cannot be affected by interference (which might change only
% \emph{other} and \emph{joint} components), as ensured by FCSL's
% meta-theory.
% %
% Second, the stability of all other clauses that also mention the
% \emph{other} component, follows from their \emph{monotonicity} with
% respect to interference. In particular, the union
% $\tkno \hunion \Tomb~\histo$, appearing also in the definition of
% $\happrox$, can only grow, while the union
% $\ikno \hunion \Tomb~\gisto$ is fixed.
% %
% Finally, the rest of the clauses mentions only values that are not
% components of the state being constrained (\eg, $\ikno$, $\gisto$,
% \etc) and, hence, are also unaffected by interference.
% %
% All these stability arguments are carried out as formal proofs in our
% Coq development, accompanying the paper.
% \end{comment}

\paragraph{How will the spec~\eqref{eq:qc-spec} be used?}

The clause $\hists\,{=}\,\gists \hunion (\res+2)\,{\mapsto}\,-$ of
\eqref{eq:qc-spec}, in conjunction with the resource
invariant~\ref{cn:hvalid}, ensures that any two calls to
\code{getAndInc}, sequential or concurrent, yield different history
entries, and hence a different result. This establishes the
requirement~\textbf{R1}, which we will not discuss further.

The inequality $\mathsf{last}(\gists \hunion \gisto)$ will provide
for~\textbf{R2} in client reasoning. To see how, consider the
particular case when $\ikno$ is empty, \ie, the pre-state is
quiescent. In that case, all the intersection in the inequality are
empty, and we can infer that the result (more precisely, $\res + 2$),
is larger than either counter's value in the pre-state. As we shall
see in Section~\ref{sec:qclients}, this captures the essence of QC.

Finally, the predicate $\happrox$~\eqref{eq:happrox} establishes a
bound for the ``out-of-order'' discrepancy between $\res$ and any
value $t$ committed to the history in the past, via the summand
$2\ (|\iknh^0 \cap \ikno^0| + |\iknh^1 \cap \ikno^1|)$. We will further
bound this summand using the expression $|\iknh^0 \hunion \iknh^1|$,
and the inclusion $\iknh^0 \hunion \iknh^1 \subseteq \tkno \hunion
\Tomb~\histo$ from~\eqref{eq:happrox}. All these bounds will
ultimately enable us to derive the requirement \textbf{R3}.
%in the clients.
%
% The approximation of the size of interference can be obtained via the
% relation, established by $\tapprox$~\eqref{eq:tapprox}, as we will
% soon show.

\subsubsection{Specifications of {\code{flip}} and
  {\code{fetchAndAdd2}}}
\label{sec:qacts}

The formal verification of the spec~\eqref{eq:qc-spec} follows by
sequential composition of its operations, \code{flip} and
\code{fetchAndAdd2}, to which we ascribe the following specs.
%
Both specs are obtained by relaxing the definitions of the transitions
from Section~\ref{sec:count-netw-prot}, \wrt~stability.
%
%
\[
%
%\tag{\normalsize \arabic{tags}}\refstepcounter{tags}\label{eq:flip-spec}
{\small
%\!\!\!\!\!\!\!\! 
\begin{array}{c}
  \spec{\!\!
  \begin{array}{c}
    \tkns = \emptyset,
    \hists = \gists,
    \gisto \subseteq \histo,\\[2pt]
    \ikno  \subseteq \tkno \hunion (\Tomb~\histo \setminus \Tomb~\gisto)
  \end{array}
  \!\!}
  \\\\[-6pt]
  \texttt{flip(}\bal\texttt{)}
  %  
  \\[3pt]
  \spec{\!\!
  \begin{array}{c}
    \exists b~z^b \ldot \res = (b, z^b)\aand
    \tkns = \set{z^b}\aand \hists = \gists \aand   \\[2pt]
    \gisto \subseteq \histo, \ikno \subseteq \tkno \hunion (\Tomb~\histo \setminus \Tomb~\gisto),\\[2pt]    
    \forall t~\ikn^0~\ikn^1 \ldot
    t \hpts (\ikn^0, \ikn^1, -) \subseteq (\gists \hunion \gisto) \Rightarrow z^b \notin \ikn^0 \hunion \ikn^1,
    \\[2pt]     
    \bapprox~(\last~(\gists \hunion \gisto)^0)~(\last~(\gists \hunion \gisto)^1)~\ikno 
  \end{array}
  \!\!}@\ccon
%
\end{array}
}
\]

\noindent
The precondition of \code{flip}'s matches the one of
\code{getAndInc}. The postcondition contains a clause with a new
predicate $\bapprox$, relating the last entries $m_0$ and $m_1$ of
either parity of the initial history $\gist = \gists \hunion \gisto$,
to the current values $n_\joint^0$ and $n_\joint^1$ of $c_0$ and
$c_1$.
%
\[
%
\!\!\!\!
{\small{
\begin{array}{l}
\bapprox~m_0~m_1~\ikno \eqdef \\[2pt]
%
  \begin{array}{l}
   m_0 \le n_\joint^0\aand
    m_1 + 2 \times |\ikno^1 \cap \tkn^1| < n_\joint^0 + 2 \times
  |\ikn^0 \cap  \tkn^0| + 2, \\[2pt]
   m_1 \le n_\joint^1\aand m_0 + 2 \times |\ikno^0 \cap \tkn^0| < n_\joint^1 + 2 \times
  |\ikn^1 \cap  \tkn^1| + 2 
  \end{array}
\end{array}
\hfill
}}
\]
%
The predicate says that the contents of $c_0$ and $c_1$ increases,
hence $m_0$ and $m_1$ are smaller or equal to the current values
$n_\joint^0$ and $n_\joint^1$, respectively. Moreover, when comparing
values of different parities (\ie, $m_1$ with $n_\joint^0$ and $m_0$
with $n_\joint^1$), we require bounds similar to the ones already
discussed in~\ref{cn:ai} and~\eqref{eq:qc-spec}, and expressed in
terms of token set $\ikno$ and $\tkn = \tkns \hunion \tkno$, that
capture the interference in the pre-state and post-state,
respectively. The predicate is internal to \esc{getAndInc}, and is not
used by, or even visible to, the clients.

The precondition of \code{fetchAndAdd2} is the same as \code{flip}'s
postcondition, and \code{fetchAndAdd2}'s post is the one of
\code{getAndInc}, so verifying the sequential composition is
straightforward.
%
% We note that the $\bapprox$ property is essential for deriving the
% inequalities from the postcondition~\eqref{eq:qc-spec}.
%
\[
%
%\tag{\normalsize \arabic{tags}}\refstepcounter{tags}\label{eq:add-spec}
{\small
%\!\!\!\!\!\!\!\!\!\! 
\begin{array}{c}
  \spec{\!\!
  \begin{array}{c}
   \tkns = \set{z^b}\aand \hists = \gists \aand   \\[2pt]
    \gisto \subseteq \histo, \ikno  \subseteq \tkno \hunion (\Tomb~\histo \setminus \Tomb~\gisto),\\[2pt]
    \forall t~\ikn^0~\ikn^1 \ldot
    t \hpts (\ikn^0, \ikn^1, -) \subseteq (\gists \hunion \gisto) \Rightarrow z^b \notin \ikn^0 \hunion \ikn^1,
    \\[2pt]    
    \bapprox~(\last~(\gists \hunion \gisto)^0)~(\last~(\gists \hunion \gisto)^1)~\ikno 
  \end{array}
  \!\!}
  \\\\[-5pt]
  \texttt{fetchAndAdd2($c_b, \specK{z^b}$)} 
  % 
  \\[3pt]
  {\normalsize{ 
  \specK{\{}~ {\small\texttt{getAndInc}}\specK{\text{'s post~\eqref{eq:qc-spec},
  instantiated with}~\gists, \ikno, \gisto\}}@\ccon
  }}
%
\end{array}
}
\]

\noindent
We note one peculiarity, however. In order to provide a provable spec
for \code{fetchAndAdd2}, we had to augment its signature with a
\emph{logical} parameter $\specK{z^b}$, representing the token,
obtained by executing \code{flip}, to be spent in incrementation of
$c_b$. While in most Hoare-style specs, logical variables scope over
the precondition and the postcondition, but do not appear in the code,
here we had to pass $z^b$ as a function argument.
%
This logical parameter serves purely for verification purposes, and
does not affect the result of the execution. Hence, in principle, it
can be safely erased, though our current formalization of FCSL in Coq
does not support such erasure.




\section{Verifying Counting Network's Clients}
\label{sec:qclients}

% We next demonstrate how the spec~\eqref{eq:qc-spec} can be used in
% clients, to establish properties \textbf{R2} and \textbf{R3}. These
% properties have been addressed in the previous work using dedicated
% consistency criteria of quiescent
% consistency~\cite{Aspnes-al:JACM94,Derrick-al:FM14} and quantitative
% quiescent consistency and
% quasi-linearizability~\cite{Afek-al:OPODIS10,Jagadeesan-Riely:ICALP14},
% but here we derive them compositionally, \ie, out of the spec of
% \code{getAndInc}. This will demonstrate the usefulness of Hoare logic
% in deriving properties related to the various flavors of quiescent
% consistency.

Following \textbf{\emph{Step 3}} of our verification method, we now
illustrate requirements \textbf{R2} and \textbf{R3} from the previous
section via two different clients which execute two sequential calls
to \code{getAndInc}. Both clients are higher-order, \ie, they are
parametrized by subprograms, which can be ``plugged in''.
%
The first client will exhibit a quiescence between the two calls, and
we will prove that the call results appear in order, as required by
\textbf{R2}. The second client will experience interference of a
program with a $N$ concurrent calls to \code{getAndInc}, and we will
derive a bound on the results in terms of $N$, as required by
\textbf{R3}.

Both our examples will rely on the general mechanism of hiding,
presented in Section~\ref{sec:background}, as a way to logically restrict the
interference on a concurrent object, in this case, a counting network,
in a lexically-scoped way.
%
To ``initialize'' the counting network data structure, we provide the
starting values for the shared heap ($h_0$) and for the history
($\gist_0$), assuming that the initial set of tokens is empty:
%
% Specifically, we will use the following derived rule:
% %
% {\small{
% \[
% \begin{array}{c}
% \spec{P}~e~\spec{Q} @ \ccon\\[2pt]
% \hline\\[-7pt]
% \!\!\!
% \specK{\{\heaps = \Phi_1(\heapj), \Phi_1(P)\}} \hide_{\Phi_1}~e \specK{\{\exists \Phi_2\ldot \heaps = \Phi_2(\heapj), \Phi_2(Q)\}} @ \cal P
% \end{array}
% \]
% }}
%
\[
\tag{\normalsize \arabic{tags}}\refstepcounter{tags}\label{eq:hide2}
{\small{
\begin{array}{r@{\ }c@{\ }l}
% \text{\normalsize{where}} &
% \Phi_1 & \eqdef & [\emptyset/\tkns, \gist_0/\hists, \heap_0/\heapj,
% \emptyset/\tkno, \histo/\hists] 
% \\[2pt]
\heap_0 & \eqdef & \bal \hpts 0 \hunion c_0 \hpts 0 \hunion c_1 \hpts 1     
\\[2pt]
\gist_0 & \eqdef & \set{0 \hpts (\set{0}, 0), 1 \hpts (\set{1}, 1)}
\end{array}
}}
\]
%
That is, $\gist_0$ provides the ``default'' history for the initial
values 0 and 1 of $c_0$ and $c_1$, with the corresponding tokens
represented by numbers 0 and 1.  As always with hiding, the
postcondition of the hidden program will imply that $\tkno$ and
$\histo$ are both empty, as there is no interference at the end.

% \gad{I think that here, again, the quotation marks around ``plugged in'',
%   ``initialize'', and ``default'' are unnecessary.}

\subsection{Exercising Quiescent Consistency}
\label{sec:qc-client}

\begin{figure}
\centering
\[
{\small{
\!\!\!\!\!\!\!\!
\begin{array}{c}
  \spec{\!\!
  \begin{array}{c}
    \tkns = \emptyset,
    \hists = \gists,
    \gisto \subseteq \histo, \Ic{\gisto}{\ikno}, \\[2pt]
    \ikno \subseteq \tkno \hunion (\Tomb~\histo \setminus \Tomb~\gisto)
  \end{array}
  \!\!}
\\\\[-5pt]
  \begin{tabular}{c || c}
   $\esc{getAndInc()}$ & ${\small{e_i}}$ 
\end{tabular}
\\\\[-5pt]
~~~~\spec{\!\!
\begin{array}{c}
  \exists \iknh~\gist_i \ldot  
  \tkns = \emptyset\aand \hists = \gbm{\gists \hunion \gist_i \hunion (\res.1 + 2) \hpts (\iknh, -)},\\[1pt]
  \gisto \subseteq \histo\aand \ikno \subseteq \tkno \hunion
  (\Tomb~\histo \setminus \Tomb~\gisto), \Ic{\gisto}{\ikno},\\[1pt]
  \last~(\gists \hunion \gisto)  < \gbm{\res.1} + 2 +
  2~|\iknh \cap \ikno|
%
\end{array}
\!\!} %@\ccon
%
\end{array}
}}  
\]
%
\caption{Parallel composition of \code{getAndInc} and~$e_i$ in~\eqref{eq:eqc}.}
  \label{fig:example1} 
\end{figure}
%



Our first client is the following program~$\eqc$:
%
\[
\tag{\normalsize \arabic{tags}}\refstepcounter{tags}\label{eq:eqc}
{\small{
\begin{array}{ll} 
\Num{1} & (\res_1, -) \Asgn (\esc{getAndInc()} ~||~ e_1) \esc{;} \\[1pt]
\Num{2} & (\res_2, -) \Asgn (\esc{getAndInc()} ~||~ e_2) \esc{;} \\[1pt]
\Num{3} &  \kw{return}~(\res_1, \res_2) 
\end{array}
}}
\]
%
Each of the calls to \esc{getAndInc} interferes with either $e_1$ or
$e_2$, but in the absence of \emph{external} interference, the
quiescent state is reached between the lines 1 and 2. Hence, after
executing $\hide~\eqc$, it should be $\res_1 < \res_2$, following
\textbf{R2}.

The programs $e_1$ and $e_2$ can invoke \code{getAndInc} and modify
the counters concurrently with the two calls of $\eqc$, which we
capture by giving both the following generic spec:
%
\[
%
\tag{\normalsize \arabic{tags}}\refstepcounter{tags}\label{eq:eispec}
{\small
\!\!\!\!\!\!\!\! 
\begin{array}{c}
  \spec{~
  \hists = \emptyset\aand
  \tkns = \emptyset\aand
   \ikn \subseteq \tkno \hunion \Tomb~\histo
  ~}
  \\[1pt]
  e_i
  % 
  \\[1pt]
  \spec{\!\!\!
  \begin{array}{c}
    \exists \gist_i \ldot \hists = \gist_i\aand 
    \tkns = \emptyset\aand 
    \ikn \subseteq \tkno \hunion \Tomb~\histo
  \end{array} 
  \!\!\!} %@\ccon
%
\end{array}
}
\]
%
The postcondition allows for a number of increments via calls to
\esc{getAndInc}, which is reflected in the addition $\gist_i$ to
$\hists$. However, all such calls are required to be \emph{finished}
by the end of $e_i$ ($\tkns = \emptyset$). As customary by now, we use
the logical variable $\ikn$ to name the initial set of \emph{other}
tokens.

Figure~\ref{fig:example1} provides a spec for each of the parallel
compositions in the program~\eqref{eq:eqc}, proved via the
corresponding FCSL inference rule for parallel
composition~\eqref{eq:parrule}.
%
The spec is very similar to~\eqref{eq:qc-spec} with the differences
highlighted via gray boxes: (a) the self-history $\hists$ is increased
by $e_i$'s contribution $\gist_i$ in addition to the entry, introduced
by \code{getAndInc}, (b) the result of the parallel composition is a
pair, but we only constrain its first component $\res.1$, resulting
from the left subprogram. We also drop the last conjunct with
$\happrox$ from~\eqref{eq:qc-spec}, which we won't require for this
example.

%



Next, we use the spec from Figure~\ref{fig:example1} to specify and
verify the program $\eqc$, so far \emph{assuming} external
interference.
%
\[
\!\!\!
{\small{
\begin{array}{c}
\!\!\!\!\!
\spec{\!\!
  \begin{array}{c}
    % \tkns = \emptyset,
    % \gisto \subseteq \histo, \ikno \subseteq \tkno \hunion (\Tomb~\histo \setminus \Tomb~\gisto)\\[2pt]
    % \hists = \gist_0,
    \mbox{Fig.~\ref{fig:example1}'s precondition with $\gists := \gist_0$, $\gisto :=
      \histo$, and $\ikno := \tkno$}
  \end{array}
  \!\!}
  ~\comm{P}
  \\\\[-6pt]
  (\res_1, -) \Asgn (\esc{getAndInc()} ~||~ e_1) \esc{;}
  \\[3pt]
\!\!\!\!{{
\spec{\!\!\!\!
\begin{array}{c}
 \exists \gist_1\ldot \tkns = \emptyset\aand \hists = \gists',~\ldots
% \ikno \subseteq \tkno \hunion (\Tomb~\histo \setminus \Tomb~\gisto),
\\[2pt]
\mbox{where 
 $\gbm{\gists' = \gist_0 \hunion
\gist_1 \hunion (\res_1 + 2)\mapsto -}$, $\gisto := \histo$ and $\ikno :=
\tkno$} 
% \\[2pt]
% \mbox{hence $\Ic{\gisto}{\ikno}$}
  \end{array}
\!\!\!\!}
}}
%
\\\\[-5pt]
(\res_2, -) \Asgn (\esc{getAndInc()} ~||~ e_2) \esc{;}      
\\[3pt]
\spec{\!\!\!\!
\begin{array}{c}
\exists \gist_1~\gist_2~\iknh \ldot     
%
\tkns = \emptyset \aand 
%%\hists = \gists' \hunion \gist_2 \hunion (\res_2 + 2)\mapsto (\iknh, -)\aand\hbox{}\\[2pt] 
\gbm{\ikno \subseteq \tkno \hunion (\Tomb~\histo \setminus \Tomb~\gisto)},\\[1pt]
\gbm{\last~(\gists' \hunion \gisto) < \res_2 + 2 + 2~|\iknh \cap \ikno|},~\ldots 
\end{array}
\!\!\!\!\!}~\comm{Q}
\\\\[-7pt]
~~~~~~~~~~~\kw{return}~(\res_1, \res_2); ~\comm{=: \res} 
\\[2pt]
\spec{~Q(\res.1/\res_1, \res.2/\res_2)~} %@\ccon
\end{array}
}} 
\]
%
We start by instantiating the logical variables $\gists$, $\gisto$ and
$\ikno$ from Figure~\ref{fig:example1} with $\gist_0$, \emph{current}
$\histo$ and $\tkno$, respectively, naming the obtained precondition
$P$.
%
In the following assertion we focus on the clauses constraining
$\tkns$ and $\hists$. To verify the second call, we instantiate
$\gists$, $\gisto$ and $\ikno$ from Figure~\ref{fig:example1} with
$\gists' = \gist_0 \hunion \gist_1 \hunion (\res_1 + 2)\mapsto -$,
\emph{current} $\histo$ and $\tkno$, correspondingly, obtaining the
postcondition, which we name~$Q$.

The inequality in the postcondition $Q$ gives the boundary on the
out-of-order position of $\res_2$ with respect to the \emph{last}
value in the history captured in between the two parallel
compositions. The boundary is given via the size of intersection of
the two sets of tokens: snapshot ($\iknh$) and ``alive'' between the
calls ($\ikno$).
%
Now, to ensure the absence of external interference, we consider the
program $(\hide~\eqc)$.
%
By the general property of hiding (Section~\ref{sec:background}), we
know that at the final state there is no interference, hence $\tkno =
\emptyset$ and $\histo = \emptyset$ in $Q$.
%
Therefore, from the set inclusion on $\ikno$ in $Q$ (the grayed part),
we deduce that $\ikno = \emptyset$.
%
As a consequence, the intersection $\iknh \cap \ikno = \emptyset$, so
from the inequality we obtain
%
\[
%
\tag{\normalsize \arabic{tags}}\refstepcounter{tags}\label{eq:tada1}
%
%{\small{
\begin{array}{c}
 \last~(\gists' \hunion \gisto) < \res.2 + 2
\end{array}
\hfill
%}}
\]
%
%\noindent
%
But $\gists'$ is defined as $(\res.1+2)\mapsto -~\hunion\ldots$,
hence, $\res.1 + 2 \in \mathsf{dom}\ \gists'$, and thus $\res.1 + 2
\le \mathsf{last}\ \gists'$. Even more:
%
\[
%
\tag{\normalsize \arabic{tags}}\refstepcounter{tags}\label{eq:tada2}
%
%{\small{
\begin{array}{c}
\res.1 + 2 \le \last~(\gists' \hunion \gisto).
\end{array}
\hfill
%}}
\]
%
From~\eqref{eq:tada1} and~\eqref{eq:tada2} follows the result
\textbf{R2}: $\res.1 < \res.2$.

\subsection{Proving Quantitative Bounds}
\label{sec:qqc-client}

We next show how the spec~\eqref{eq:qc-spec} also obtains quantitative
bounds on the out-of-order anomalies in terms of a number of running
threads in the following program $\eqqc$:
%
\[
\tag{\normalsize \arabic{tags}}\refstepcounter{tags}\label{eq:eqqc}
{\small{
\begin{tabular}{l || l}
$
\begin{array}{ll} 
\Num{1} & \res_1 \Asgn \esc{getAndInc();} \\[1pt]
\Num{2} & \res_2 \Asgn \esc{getAndInc();}  \\[1pt]
\Num{3} & \kw{return}~(\res_1, \res_2)
\end{array}
$  
&
$~~~e$
\end{tabular} 
}}
\]
%
The $e$'s spec says that the \emph{number} of calls to \esc{getAndInc}
in~$e$ (\ie, the size of interference $e$ exhibits) is some fixed $N$:
%
\[
%
\tag{\normalsize \arabic{tags}}\refstepcounter{tags}\label{eq:espec}
{\small
\!\!\!\!\!\!\!\!\!  
\begin{array}{c}
  \spec{
  \tkns = \emptyset,
  \hists = \gists }
~  e
~  \spec{\!\!\!
  \begin{array}{c}
    \exists \gist \ldot 
    \tkns = \emptyset,
    \hists = \gists \hunion \gist,
    |\gist| = N
  \end{array} 
  \!\!\!} %@\ccon
%
\end{array}
}
\]
%
Our goal is to prove that in the absence of external interference for
$\eqqc$, $\res_1 < \res_2 + 2 \ N$ (requirement \textbf{R3}).

\begin{figure}
\centering
%    
\[
\!\!\!
{\small{
\begin{array}{c}
  \spec{~
    \mbox{{\normalsize{\eqref{eq:qc-spec}'}}s precondition with $\gists := \gist_0$, $\gisto :=
      \histo$, and $\ikno := \tkno$}~}
% \\\\[-6pt]
% \spec{\!\!
%   \begin{array}{c}
%     \tkns = \emptyset,
%     \hists = \gists,
%     \gisto \subseteq \histo,
%     \ikno \subseteq \tkno \hunion (\Tomb~\histo \setminus \Tomb~\gisto)\\[2pt]
%     \mbox{where $\gisto = \histo$ and $\ikno = \tkno$, hence $\Ic{\gisto}{\ikno}$}
% %    \histo \subseteq \histo,\\[2pt]
% %    \tkno \hunion \Tomb~\histo \subseteq \tkno \hunion \Tomb~\histo
%   \end{array}
%   \!\!}
\\\\[-6pt]
\res_1 \Asgn \esc{getAndInc();}
\\[3pt]
\spec{\!\!
\begin{array}{c}
   \exists \ikn \ldot 
   \tkns = \emptyset, 
   \hists = \gists',
   % \gisto \subseteq \histo, \ikno \subseteq \tkno \hunion
   % (\Tomb~\histo \setminus \Tomb~\gisto)
   \ldots
   \\[2pt]
   \mbox{where $\gbm{\gists' = \gist_0 \hunion (\res_1 + 2) \hpts
       (\ikn, -)}$} 
   % , $\gisto = \histo$ and $\ikno = \tkno$, \ldots}
% \\[2pt]
%    \mbox{hence $\Ic{\gisto}{\ikno}$}
  \end{array}
  \!\!}%
%\\\\[-6pt]
%\spec{\!\!\!
%  \begin{array}{c}
%  \tkns = \emptyset,  
%  \hists = \gists \hunion (\res_1 \!+\! 2) \!\hpts\! (\ikn^0, \ikn^1, -), 
%  \gists' := \hists,\\[2pt] 
%  \gisto := \histo, 
%  \ikno := \tkno,
%  \ikno \hunion \Tomb~\gisto \subseteq \tkno \hunion \Tomb~\histo
%  \end{array}
%  \!\!\!}%
\\\\[-5pt] 
\res_2 \Asgn \esc{getAndInc();}
\\[3pt]
\spec{\!\!\!
\begin{array}{c}
  \exists \iknh~z \ldot     
  % \tkns = \emptyset, \hists = \gists' \hunion (\res_2+2) \mapsto (\iknh, z),\\[2pt]
  \happrox (\gists' \hunion \gisto)~\res_2~\iknh~z, \ldots
\end{array}
\!\!\!}
\\\\[-5pt]
\spec{\!\!\!
\begin{array}{c}
  \exists \iknh~z \ldot     
  % \tkns = \emptyset, \hists = \gists' \hunion (\res_2+2) \mapsto (\iknh^0, \iknh^1, z),\\[2pt]
  \gbm{\iknh \subseteq \tkno \hunion (\Tomb~\histo) \hunion \set{z}},
  z \notin \ikn,\\[2pt] 
  \res_1 + 2 < \res_2 + 2 + 2~|\iknh \cap \ikn|
\end{array}
\!\!\!}
\\\\[-5pt]  
~~~~~~~~~~~~~~~
\kw{return}~(\res_1, \res_2) ~\comm{=: \res}
\\\\[-5pt]
\spec{\!\!\!
\begin{array}{c}
  \res.1 < \res.2 + 2 \ |\tkno \hunion \Tomb~\histo| 
  %
  %\aand\\[1pt] 
  % \tkns = \emptyset\aand
  % \hists = \gists \hunion (\res_1 + 2) \hpts -
  % \hunion (\res_2 + 2) \hpts -
  \end{array}
  \!\!\!} %@\ccon
\end{array}
}} 
\]
%
%
\caption{Proof outline of sequential composition in~\eqref{eq:eqqc}.}
\label{fig:proof2}
\end{figure}

We first verify the sequential composition of the two calls
in~\eqref{eq:eqqc}; the proof outline is in
Figure~\ref{fig:proof2}. 
%
% \ab{Can this proof outline be cut significantly/removed? Can we just
%   give the overall pre-post specs and go directly to the verification
%   of $\eqqc$?}
%
As previously, we start by instantiating the logical variables
$\gists$, $\gisto$ and $\ikno$ from spec~\eqref{eq:qc-spec} with
$\gists$, $\histo$ and $\tkno$, respectively. In the assertion,
resulting by of the first \esc{getAndInc}, we keep only the clauses
involving $\tkns$ and $\hists$, dropping the rest.
%
To verify the second \esc{getAndInc} call, we instantiate $\gists$,
$\gisto$ and $\ikno$ with $\gists' = \gists \hunion (\res_1+2) \mapsto
(\ikn, -)$, current $\histo$ and $\tkno$.

In the postcondition of the second call to \esc{getAndInc}, we focus
on the $\happrox~(\gists' \hunion \gisto)~\res_2~\iknh~z$ clause,
where $\iknh$ is the set of tokens snapshot when contributing
$\res_2+2$.
%
Unfolding the definition of $\happrox$ from~\eqref{eq:happrox}, we
obtain $\iknh \subseteq \tkno \hunion \Tomb~\histo
\hunion\{z\}$. Also, using $(\res_1 +2)\mapsto (\ikn, -)$ in the
implication that the unfolding obtains, we get $z \notin \ikn$ and
\[
\tag{\normalsize \arabic{tags}}\refstepcounter{tags}\label{eq:le0}
{\small{\res_1 + 2 < \res_2 + 2 + 2~|\iknh \cap \ikn|
}}
\]
%
Now we use the following trivial fact to simplify.
%
\vspace{8pt}
%
\begin{lemma}
\label{lm:intersect2}
If $z \in \iknh$ and $z \notin \ikn$, then $|\iknh \cap \ikn| \le
|\iknh| - 1$.
\end{lemma}
%
%
\vspace{8pt}
%
\noindent
Using the invariant~\ref{cn:ti1}, Lemma~\ref{lm:intersect2}
derives
$|\iknh \cap \ikn| \le |\iknh| - 1$
%
after which, the inclusion $\iknh \subseteq \tkno
\hunion \Tomb~\histo \hunion \set{z}$ leads to
%
\[
\tag{\normalsize \arabic{tags}}\refstepcounter{tags}\label{eq:le5}
{\small{
|\iknh \cap \ikn| \le |\tkno \hunion \Tomb~\histo|}}
\]
%
Combined with~\eqref{eq:le0}, this gives us $\res_1 < \res_2 + 2 \
|\tkno \hunion \Tomb~\histo|$, as shown in Figure~\ref{fig:proof2}'s
postcondition. In words, it asserts that the discrepancy between
$\res.1$ and $\res.2$ is bounded by the size of the tokens, which are
either held by the interfering threads at the end or are spent.
%
% \gad{low-priority right now, but still: the inequality above spreads
%   across a linebreak, and worse, in the middle of the arguments for an
%   operator.}
%
% \is{won't fix}

\begin{figure}[t]
  \centering
\[
{\small{
\!\!\!\!\!\!\!\!
\begin{array}{c}
  ~~~~~~~~\spec{~
  \tkns = \emptyset,
  \hists = \gist_0, \ldots
~} ~\comm{P}
\\[2pt]
  \begin{tabular}{c || c}
$
\spec{\!\!\!
    \begin{array}{c}
    \tkns = \emptyset,
    \hists = \gist_0
  \end{array}\!\!\!}
$
&
$
\spec{\!\!\!\begin{array}{c}
    \tkns = \emptyset,
    \hists = \emptyset
  \end{array}\!\!\!}
$
\\[3pt]
   $\begin{array}{l}
      \res_1 \Asgn \esc{getAndInc();}\\[1pt]
      \res_2 \Asgn \esc{getAndInc();}\\[1pt]
      \kw{return}~(\res_1, \res_2) ~\comm{ =: \res}\!\!\!
    \end{array}$
\!\!\!\!\!\!
& ${\small{e}}$ 
\\\\[-5pt] 
$
\spec{\!\!\!
{\small{
  \begin{array}{c}
    \res.1 < \res.2 + 2 \ |\tkno \hunion \Tomb~\histo|
  \end{array}
}}
  \!\!\!}\!\!$
\!\!
&
\!\!$\spec{\!\!\!
{{
  \begin{array}{c}
    \exists \gist \ldot 
    \hists = \gist, 
    |\gist| = N,
    \ldots
  \end{array}
}}
\!\!\!}$
\end{tabular}
\\\\[-5pt]
\comm{\res_1 := \res.1.1, \res_2 := \res.1.2}
\\\\[-6pt]
\spec{
\res_1 < \res_2 + 2 \ |\tkno \hunion \Tomb~(\histo \hunion \gist)|
}
\\[3pt]
~~~~~~~~\spec{
\res_1 < \res_2 + 2 \ |\tkno \hunion \Tomb~\histo| + 2 \ N
} ~\comm{Q}
%
\end{array}
}}  
\]
  \caption{Proof outline for the $\eqqc$ program.}
  \label{fig:eqqcproof}
\end{figure}

Figure~\ref{fig:eqqcproof} shows the proof outline for $\eqqc$ via the
spec from Figure~\ref{fig:proof2}.
%
By the parallel composition rule~\eqref{eq:parrule}, the precondition
splits into two subjective views, where we send the initial history
$\gist_0$ to the left thread, and the empty history to the right
thread. The proof from Figure~\ref{fig:proof2} then applies to the
left thread, and the spec~\eqref{eq:espec} applies to the right
one. Final $\histo$ of the left thread is the union of $\histo$ from
the joined thread with $\gist$, since the environment of the left
thread includes the right thread and of the join. Rewriting by this
property in the postcondition of the left thread gives us the post of
the joint thread: $\res_1 < \res_2 +2\ |\tkno \hunion \Tomb~(\histo
\hunion \gist)|$, which we can next simplify into
\[
\res_1 < \res_2 + 2\ |\tkno \hunion \Tomb~\histo| + 2\ N
\]
because $\Tomb$ distributes over $\hunion$, and $|\Tomb~\gist| =
|\gist| = N$. Finally, we restrict the external interference by
considering $(\hide~\eqqc)$. From the properties of hiding,
we deduce that $\tkno$ and $\histo$ in $Q$ are empty, hence we can
simplify into $\res_1 < \res_2 + 2 \ N$, which is the desired
result~\textbf{R3}.
%
%\gad{The same happens here, just before the highlighted math.}

\section{Discussion}
%\section{Client-side reasoning}
\label{sc:discussion}

\begin{comment}
%\gad{Section 4 should emphasize that we are hiding colors,
%  end-points, and whatever `` internal'' aspects of the concurroid in
%  client reasoning. This might require to connet with the proof
%  outline from Figure~\ref{fig:weird:code} in the appendix. We should
%  say that this internal only ``leak'' through the
%  definition/instantiations of $\stableorder$.}

%% \gad{This section now needs two parts: (i) a first part about the
%%   clients and (ii) a second part discussing the \emph{principality} of
%%   scan's specs}

%% We already argued in Section~\ref{sc:formal} that the Hoare triples
%% from Figure~\ref{fig:specs} capture precisely what linearizability
%% would be used for, namely that the operations of the snapshot object
%% can be sequenced. The triples do so without the intermediary
%% sequential specifications for {\tt write} and {\tt scan}, and the
%% attending simulation proofs.  In this section, we argue that our
%% approach via Hoare logic can still do a bit more, specifically in the
%% cases of client side reasoning, and reasoning about parallel
%% composition.

%% By relying on FCSL, we immediately inherit the separation logic
%% mechanism for reasoning about programs with nested parallel
%% composition (i.e., dynamic, well-bracketed, forking and joining). We
%% refer the interested reader to the Appendix~\ref{sc:background} for a
%% brief background on FCSL, and the separation-logic style inference
%% rule for parallel composition that requires that program state, real
%% and auxiliary, satisfies the properties of a Partial Commutative
%% Monoid, or PCM (and indeed, our histories form a PCM under the
%% operation of disjoint union, with the empty history as a unit). This
%% gave us a thread-local way to reason about programs, via local
%% variables such as $\histS$ and $\histO$, rather than relying on global
%% abstractions, such as \emph{thread-id}s, to specify the behavior of
%% each thread.  This is in contrast to linearizability, whose very
%% definition requires identifying threads by thread-ids, thus focusing
%% on programs that are a top-level parallel composition of a fixed
%% number of sequential threads, but not providing convenient
%% abstractions for reasoning about nested parallel compositions.

%As a first point, we argue that the pre- and postcondition ascribed
%to {\tt write} and {\tt scan} in Section~\ref{sc:formal} are
%\emph{principal}, in that they can be used in larger program contexts
%without modification. This is not the case in most other verification
%methods, where programs, once verified, typically still have to be
%refactored wrt.~auxiliary state and code \emph{on a per-context
%basis}~\cite{Owicki-Gries:CACM76,Jacobs-Piessens:POPL11}. By using
%\emph{local} variables such as $\histS$ and $\histO$ to name the
%per-thread history, we avoid the need for refactoring. Such locality
%is supported by the FCSL inference rule for parallel
%composition~\cite{LeyWild-Nanevski:POPL13,Nanevski-al:ESOP14}. The
%rule is somewhat unusual, but please see Appendix~\ref{sc:background}
%for a brief description. What matters, however, is that reasoning in
%clients is done strictly out of the specs of {\tt write} and {\tt
%scan}. Notice that, while these expose the existence of the logical
%ordering $\stableorder$ and how some individual events may be related
%in it, they \emph{do not} expose the actual definition of
%$\stableorder$ in terms of colors, or any other internal of the
%auxiliary state and code. Other snapshot implementations can provide
%their own definition of $\stableorder$.

\gad{This part was not useful in Section~\ref{sc:clients}.}

As a brief example of reasoning about parallel composition, it is
possible to prove that the program
%
$ 
P_1 = {\mathtt{write}}\ (x, 1) \parallel {\mathtt{write}}\ (y, 2)
$
%
satisfies the spec which says that two events are executed, in an
unspecified order. Then we can reason about a larger program, say
$
  P_2 = P_1; {\mathtt{write}}\ (x, 3)
$
out of that spec, to prove that $P_2$ executes three events, with the
write of $3$ appearing last. 
%
Moreover, the \emph{substitution principle} holds; that is, the proof
remains valid, even if we replace $P_1$ by another program, say
%
$
P'_1 = {\mathtt{write}}\ (x, 1); {\mathtt{write}}\ (y, 2)
$
which satisfies a stronger spec (explicitly ordering the two events),
but which \emph{can be weakened} to the spec of $P_1$ by forgetting the
ordering by means of strengthening the precondition and weakening the
postcondition, as customary in Hoare logic. Thus, our client proofs
\emph{can ignore} the internal thread-structure of component programs.

\gad{The rest of this section will have to be merged either with the
  Related Work, or the conclusions}

\end{comment}

\begin{wrapfigure}[11]{t}[15pt]{0.5\textwidth} 
\vspace{-15pt}
%% \begin{figure}
%
%\centering  
%\begin{tabular}{l}
% 
\begin{minipage}[t]{.5\textwidth}
\[
\begin{array}{rl}
\num{1} & \esc{scan}\ () : ( A \times A )~ \{ \\ 
\num{2} & ~~~ (\var{cx}, \var{vx}) \tbnd \act{read}(\x);\\
\num{3} & ~~~ (\var{cy}, \_ ) \tbnd \act{read}(\y);\\
\num{5} & ~~~ (\_ , \var{tx}) \tbnd \act{read}(\x);\\
\num{5} & ~~~ \kw{if}\ vx = tx \\
\num{6} & ~~~ \kw{then}\ \kw{return}\ (\var{cx},\var{cy})\\
\num{7} & ~~~ \kw{else}\ \esc{scan}\ (); \}
\end{array}
\]
\end{minipage}
%
%\end{tabular}
%
\caption{{\tt Scan} using versions.}
\label{fig:readpair}
%\end{figure}
\end{wrapfigure}

% \gad{\textbf seems to have no efect whatsoever on alttt}
% \gad{unified action-names not to leave spaces before arguments}





The substitution principle can be pushed further. In particular, we
can use a different snapshot algorithm, without modifying the proofs
of $P_1$, $P_2$, $P'_1$ or any other client, as long as \jywrite\ and
\jyscan\ satisfy the expected specs.
%
We have confirmed this property on the toy example given in
Figure~\ref{fig:readpair} (we present only \jyscan, as \jywrite\ is
trivial)~\cite{SergeyNB+ESOP15}. In this example, the snapshot
structure consists of pointers $x$ and $y$ storing tuples $(c_x, v_x)$
and $(c_y, v_y)$, respectively. $c_x$ and $c_y$ are the payload of $x$
and $y$, whereas $v_x$ and $v_y$ are version numbers, internal to the
structure. Writes to $x$ and $y$ increment the version number. {\tt
  Scan} reads $x$, $y$ and $x$ again, in succession, but avoids
snapshot inconsistency by restarting if the two version numbers of $x$
differ. The definition of $\stableorder$ used to satisfy the specs
equals the real time one, as no dynamic reordering is needed.

%\gad{Why are we not calling read-pair by it's name?}
%%
%\an{Well, because we're trying to make the parallel with the Jayanti's
%  code. By calling it {\tt scan}, the connection is immediately
%  made. Speaking of names, I'm more concerned that we name with {\tt
%    write} two different things. The primitive memory operation should
%  probably be renamed into $x := v$ or {\tt assign}.}
%%
%\gad{I understand. But I was refering to the algorithm and not the
%  methods.}

%% \gad{Add HERE!: Clients}

\gad{Add HERE!: Readpair's proof outline w/ this paper's spec and
  instance of getters/ stable order etc.}

\gad{Reviewers --and Ralf-- complain about its not quite obvious that
  our spec is useful. We need to make an effort here to convice them
  that our assertions are useful for clients.}

  %% \gad{The simple sequential {\tt two\_scan} example will help here,
  %%   as it showcases the need for ``sequential`` and the stable ideal
  %%   inclusion on histories. }

\gad{Aleks' comments on 04/10 wrt. why our spec is better than what
  linearizability provides, even if the spec looks a priory more
  complicated. \\ We could reduce the spec by (rule of consequence)
  weakening, to get the assertion of the sequential spec $r
  =\mathsf{eval}\ \hist $, but that would not be useful for client
  reasoning, as you would be losing the information needed to combine
  the client in parallel with others: take the example in Figure~2 --
  the client, we've (almost) proved in Coq-- and a linearization of
  the client on the left and the center {\tt l: write x 2; c: scan; l:
    write y 1 } and do a proof using the sequential spec. Now if you
  are given a different linearization, {\tt l: write x 2; l: write y
    1; c: scan () }, you need to re do the proof. Moreover, there is
  no way to scale up adding more threads, if you add the client on the
  right, r, none of the proofs youy have can be reused. \\ In our
  setting, and with our specs, client reasoning depends solely on the
  API for scan and write, regardless of the different linearizations
  of a program. We use this to our advantage while building up the
  proof for the client in Figure~2.  }

%% \gad{Structure the proof outlines for the clients in the same why we
%%   have done in coq}
%%
%% \paragraph{Clients} \gad{Let's see what we have right now:}  

%% \begin{enumerate}
%%   \item[i] two sequential scans, and a proof that whatever the second
%%     scanner returned is at least as new as the second, and moreover
%%     the two snapshots are consistent, ie, the snapshot in the first
%%     call is still a snapshot with the histories and orderings defined
%%     by the second call. Morever, the stable order 

%%   \item[ii] The composite client building up to Figure~2.
    
%%   \item[iii] The latter with hide? \gad{We might not have time to hack
%%     it in in Coq, but we could do it by hand, and then upload it}.
 
%% \end{enumerate}

\section{Related work}
\label{sc:related}

% Early times

The proof method for establishing linearizability of concurrent
objects based on the notion of \emph{linearization points} has been
presented in the original paper by Herlihy and
Wing~\cite{Herlihy-Wing:TOPLAS90}. The first Hoare-style logic,
employing this method for compositional proofs of linearizability was
introduced in Vafeiadis' PhD thesis~\cite{Vafeiadis:PhD}. However,
that logic, while being inspired by the
combination~\cite{Vafeiadis-Parkinson:CONCUR07} of Rely-Guarantee
reasoning~\cite{Jones:TOPLAS83} and Concurrent Separation
logic~\cite{OHearn:TCS07} with syntactic treatment of linearization
points~\cite{Vafeiadis-al:PPoPP06}, did not have a soundness proof
with respect to any program semantics. Furthermore, the
work~\cite{Vafeiadis:PhD} did not connect reasoning about
linearizability to the verification of client programs, which make use
of linearizable objects in a concurrent environment (\cf
Section~\ref{sec:clients}).

% Modern logics for linearizability

Both these shortcomings were addressed in more recent works on program
logics for establishing linearizability~\cite{Liang-Feng:PLDI13}, or,
equivalently~\cite{Filipovic-al:TCS10}, \emph{observational
  refinement}~\cite{Turon-al:ICFP13}, which provided semantically
sound methodologies for (a) verifying linearizability/refinement of
concurrent objects \emph{as well as} for (b) giving the objects
Hoare-style specifications, useful for the clients.
%
However, in the both
approaches~\cite{Liang-Feng:PLDI13,Turon-al:ICFP13} establishing (a)
and (b) essentially requires one to prove \emph{two different} facts
about a program, and, if one is interested only in the Hoare-style
reasoning by means of composing program specifications, verifying
linearizability (a) is a detour, which might be avoided.

% No-linearizability

This observation has been recognized in a series of more recent works
on program logics for concurrency that all focused on establishing
Hoare-style specifications for concurrent objects (b) without
resorting to
linearizability~\cite{Sergey-al:ESOP15,Svendsen-Birkedal:ESOP14,ArrozPincho-al:ECOOP14,Jung-al:POPL15}.
%
In this paper, we are following the same way of thinking, building on
the ideas from the prior work~\cite{Sergey-al:ESOP15}, which explored
some patterns of assigning \emph{subjective} Hoare-style concurrent
specifications with auxiliary histories to concurrent objects
(including \emph{higher-order} ones, such as {flat
  combiner}~\cite{Hendler-al:SPAA10}) in
FCSL~\cite{Nanevski-al:ESOP14}. The work~\cite{Sergey-al:ESOP15} has
generalized earlier results on history-based Hoare-style
logics~\cite{Fu-al:CONCUR10, Gotsman-al:ESOP13,Bell-al:SAS10}, yet it
has not provided a way to reason about concurrent objects, featuring
future-dependent linearization points.
%

The key novelty of this work with respect to previous results
involving Hoare-style reasoning about histories~\cite{Fu-al:CONCUR10,
  Gotsman-al:ESOP13,Bell-al:SAS10,Sergey-al:ESOP15,Hemed-al:DISC15} is
the idea of dynamically \emph{re-linking} the auxiliary histories,
enabling efficient constructive reasoning about non-local and
future-dependent linearization points.
%
Since re-linking as we presented it is just manipulation with
otherwise standard auxiliary state, we did not have to extend the
metatheory of FCSL, and were able to use it
\emph{off-the-shelf}. Furthermore, relying on the auxiliary state
makes it possible to extend our verification method for reasoning
about higher-order (\ie, parametrized by another data structure)
snapshot-based concurrent constructions~\cite{Petrank-Timnat:DISC13},
which is our immediate future work.
%
In contrast, alternative modern programming
logics~\cite{ArrozPincho-al:ECOOP14,Jung-al:POPL15,Svendsen-Birkedal:ESOP14}
would require introduction of prophecy variables in order to verify
Jayanti's snapshot construction, and, to the best of our knowledge,
none of these extensions has been implemented yet.

Related to our result, O'Hearn \etal have demonstrated how to employ
history-based reasoning and Hoare-style logic for proving
\emph{non-constructively} existence of linearization points for
concurrent objects out of the data structure
invariants~\cite{OHearn-al:PODC10}---the result is known as \emph{the
  Hindsight Lemma}. The reasoning principle presented in this paper
generalizes that idea, since the Hindsight Lemma is only applicable to
``pure'' concurrent methods (\eg, concurrent set's
\texttt{contains}~\cite{Heller-al:OPODIS05}), which do not determine
position of other threads' linearization points. In contrast, our idea
of re-linking histories also handles the structures, where a
linearization point of a method call (e.g., \texttt{write}) might
depend on the (future) outcome of another operation (e.g.,
\texttt{scan}), as was showcased by Jayanti's construction.



% Most recent related work relies on parametrization to avoid reasoning
% about linearizability. But, that has its drawbacks. In particular,
% while it can handle situations in which linearization points are
% placed in different places, depending on the run-time infomration
% (speculiation), it is not currently strong enough to formalize
% examples where linearizatin points appear in different
% proceedures.\an{Hmm, are we super sure of this?} Thus, we don't
% believe they can handle Jayanti's algorithm.  \is{How about this time
%   we just mention these people briefly and instead do a comparison to
%   the PODC crowd and their reasoning methods, which are all about
%   harnessing the vanialla definition of liearizability. This way, the
%   whole discussion will be more relevant to the audience, as nobody
%   knows the concurrency logics anyway, so we'll just waste valuable
%   space, talking about them in detail.}  \an{I agree that we should
%   just mention them briefly.}

% Independently of us, Kyzha et al. have developed an a method whereby
% linearizability is proved by reordering time-stamped histories,
% similar to the basis of our approach. However, there are many
% differences. 

% \begin{enumerate}
% \item While linearizability does not say anything about clioent side
%   proofs, beyond the ability to replace the two programs in it, our
%   method also gives a way to reason about clients, as we illustrated
%   in Section 4.

% \item While they present a new logic, for us, it is all a mode of use
%   of auxiliary state.

% \item Our PCM of histories let us reuse separation logic in
%   infrastructure (e.g., frame rule) to reason about histories locally,
%   whereas Kyzha et al. use global histories only. Our setting also
%   immediatley lends itself to higher-order programming. This is
%   particularly important for snaphsots, as one of their major
%   application is in iterators -- a prototypical higher-order
%   program~\cite{PetrankT+disc13}. However, we don't explore iterators
%   in this particular paper.

% \item Kyzha et al. track the ordering of timestamps quite differnetly
%   from us. Where we keep an ``existential'' witness for the total
%   ordering of timestamps, at all stages of evaluation, they do so
%   ``universially''. Thus, they require proving that all possible
%   completions of a partial order into a total order are valid for
%   establishing the relation with the linearization program. We believe
%   this leads to larger proofs and more complicated proofs than
%   necessary.
% \end{enumerate}

% Liang et al~\cite{LiangF+pldi13} present a dedicated meta-theory to
% reason with future-dependent linearization points based on
% speculations. \gad{Are we sure they can't do Jayanti here? Need to
%   think what to say about their works as they do have a program logic
%   as we do}
  

%\vspace{-4pt}

\section{Conclusion and Future Work}
\label{sec:conclusion}

%\vspace{-2pt}

We have presented a number of formalization techniques, enabling
specification and verification of highly scalable non-linearizable
concurrent objects and their clients in Hoare-style program logics.
%
In particular, we have explored several reasoning patterns, all
involving the idea of formulating execution histories as auxiliary
state, capturing the expected concurrent object behavior.
%
We have discovered that quantitative logic-based reasoning about
concurrent behaviors can be done by storing relevant information about
interference directly into the entries of a logical history.

We believe that our results help to bring the Hoare-style reasoning
into the area of non-linearizable concurrent objects and open a number
of exciting opportunities for the field of mechanized logic-based
concurrency verification.

For instance, in this paper we have deliberately chosen to focus on
simple client programs to showcase the specs we gave to concurrent
libraries. However, any larger program incorporating these examples
can be verified compositionally in FCSL, out of \emph{these clients'
  specs}, via the substitution principles of
FCSL~\cite{Nanevski-al:ESOP14,Sergey-al:PLDI15}, without the need to
deal with concepts such as histories and tokens that are specific to
particular libraries. Given the bounds, which we formally proved in
Section~\ref{sec:qclients}, we believe that the reasoning patterns we
have described will be useful for mechanical verification of larger
weakly-synchronized approximate parallel
computations~\cite{Rinard:RACES}, exploiting the QC and QQC-like
behavior.

Furthermore, by ascribing interference-sensitive quantitative specs in
the spirit of~\eqref{eq:qc-spec} to relaxed concurrent
libraries~\cite{Henzinger-al:POPL13}, one can assess the applicability
of a library implementation for its clients: the clients should
tolerate the anomalies caused by interference, as long as they can
logically infer the desired safety assertions from a library spec,
which is fine-tuned for particular usage scenarios.


% Since logical approaches enable reasoning about higher-order
% concurrent data
% structures~\cite{Svendsen-al:ESOP13,Turon-al:ICFP13,Sergey-al:ESOP15},
% we envision the possibility of giving parametric logical specs to such
% generic relaxed constructions as diffracting/elimination
% trees~\cite{Shavit-Touitou:TCS97,Shavit:CACM11} that, once
% instantiated with suitably specified stacks or pools on the leaves,
% would yield a provably correct, highly scalable concurrent container
% implementation.


% Acknowledgements:

% Michael Emmi
% Pierre Ganty
% Andrea Cerone
% Anton Podkopaev

% \todo{Generalizing the construction of the counting network to
%   arbitrary diffracting trees}

% \todo{Elimination and diffracting trees~\cite{Shavit-Touitou:TCS97}.}
\paragraph{Acknowledgements}
We thank the anonymous reviewers from OOPSLA'16 PC and AEC for their feedback. We are also grateful to Sophia Drossopoulou for shepherding our paper. This research was partially supported by \ab{Please fill in} and the US National Science Foundation (NSF). Any opinion, findings, and conclusions or recommendations expressed in the material are those of the authors and do not necessarily reflect the views of NSF.


% \acks
% \todo{Acknowledgments, if needed.}

%\newpage

\setlength{\bibsep}{2.5pt}
\bibliographystyle{abbrv}
\softraggedright 
\bibliography{bibmacros,references,proceedings}


\end{document}


