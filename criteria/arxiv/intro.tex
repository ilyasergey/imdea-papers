\section{Introduction}
\label{sec:introduction}

Formally defining the behavior of highly parallel concurrent objects
is a fundamental challenge, which requires the program designer to
find a trade-off between the desired performance on a multicore
processor, possibly enabled by reduced contention, and the safety
guarantees, implied by the chosen correctness conditions.
 
Traditionally, the correctness of concurrent objects is defined using
\emph{event histories},\footnote{Here, \emph{event} stands for a call
  to or a return from an object's method.}  by (\emph{i}) providing a
\emph{specification set}, describing all ``basic'' behaviors that the
object's client might observe when using it, and (\emph{ii}) fixing a
\emph{consistency condition} that relates the concurrently observable
behaviors to the ones in the specification set.
% 
In the majority of the cases, the specification set is taken to be the
histories of the concurrent object's \emph{sequential} behaviors, in
which the calls are immediately followed by the corresponding returns.
 
The most well-known concurrent correctness condition is
\emph{linearizability}~\cite{Herlihy-Wing:TOPLAS90}, which requires
for each concurrent history of an object to exist a mapping to a
sequential history, such that the ordering of two operations, defined
as matching call/return pairs, is preserved if they are performed by
the same thread, or if they do not overlap.
% 
In the twenty five years since it invention, linearizability has been
shown to be remarkably scalable as a correctness condition, widely
applicable to capture the behavior of implementations of multiple
concurrent objects with intuitive sequential specifications, such as
stacks, queues, sets, locks, snapshots and their combinations, and
even suitable for automatic synthesis of some concurrent
objects~\cite{Vechev-Yahav:PLDI08}. Thanks to the compositional proof
method, based on \emph{linearization points}, proofs of
linearizability in most of the cases are amenable for practical
computer-aided
verification~\cite{Burckhardt-al:PLDI10,Derrick-al:TOPLAS11,Vafeiadis:CAV10,Amit-al:CAV07,Shacham-al:OOPSLA11}.
%
Furthermore, linearizability with respect to a sequential
specification implies observational
refinement~\cite{Filipovic-al:TCS10,Emmi-al:PLDI15}, which allows one
to verify the client code of linearizable concurrent objects
efficiently by replacing the actual object implementation with its
\emph{atomic} counterpart.

However, in order to be linearizable with respect to a sequential
specification, an implementation of a concurrent object should
inherently incorporate certain costly synchronization
primitives~\cite{Attiya-al:POPL11}, which then become points of high
contention and prevent efficient
parallelization~\cite{Shavit:CACM11}. The way to remedy this situation
for enabling better scalability is to change the semantics of a
concurrent object, \ie, the correctness criterion, redefining its
admissible behaviors.

The outlined \emph{history-based} approach for concurrency
specification allows one to relax the semantics of a concurrent object
by either changing the specification set to include other histories
than just the sequential ones, or by changing the consistency
condition, typically quantifying over the possible interference
patterns, which can be observed while using the object. The first
option is taken, for example, in the
works~\cite{Hemed-Rinetzky:PODC14,Henzinger-al:POPL13}, resulting in
such correctness-defining frameworks as \emph{concurrency-aware
  linearizability} and \emph{quantitative relaxation},
correspondingly. The second option is exemplified by such concurrency
correctness criteria as \emph{quiescent
  consistency}~\cite{Aspnes-al:JACM94,Derrick-al:FM14},
\emph{quantitative quiescent
  consistency}~\cite{Jagadeesan-Riely:ICALP14},
\emph{quasi-linearizability}~\cite{Afek-al:OPODIS10}, and \emph{local
  linearizability}~\cite{Haas-al-local15}.

Adopting a new correctness condition for concurrent objects comes with
a price. First, in order to make the newly developed criterion
\textbf{(a)} \emph{scalable}, one has to prove it to be compositional
with respect to combination of multiple concurrent data
structures. Second, in order to make it \textbf{(b)} \emph{practical},
one needs to supply it with a principled proof technique, enabling
efficient correctness checking. Finally, to make it \textbf{(c)}
\emph{relevant for the program verification at large}, one has to
devise a method for exploiting the provided safety guarantees for the
sake of reasoning about client code that uses the concurrent object.
%
While compositionality was formally established for most of the listed
above
criteria~\cite{Aspnes-al:JACM94,Jagadeesan-Riely:ICALP14,Afek-al:OPODIS10,Haas-al-local15},
only few of them were equipped with proof
techniques~\cite{Derrick-al:FM14,Zhang-al:ASE13}, and we are not aware
of any correctness conditions other than linearizability being
employed for client-side reasoning. This state of affairs makes us
pose the following research question:

\vspace{4pt}
\noindent
\textbf{Q:} \emph{Is there a generic specification approach that
  allows one to formally define adequate correctness conditions for
  arbitrary concurrent objects, immediately enjoying the properties
  \emph{\textbf{(a)}--\textbf{(c)}}}?
%
\vspace{4pt}

\noindent
We seek an answer to this question by exploring the opportunities,
opened by the recent advances in the \emph{state-based} approach for
concurrency verification using \emph{Hoare-style program logics}.

In modern program logics for
concurrency~\cite{Feng-al:ESOP07,Vafeiadis-Parkinson:CONCUR07,Feng:POPL09,DinsdaleYoung-al:ECOOP10,Nanevski-al:ESOP14,Svendsen-Birkedal:ESOP14,ArrozPincho-al:ECOOP14,Jung-al:POPL15,Raad-al:ESOP15,Fu-al:CONCUR10},
specifications (or \emph{specs}) for methods of concurrent objects
(and programs in general) are represented by \emph{Hoare tuples}
$\spec{P}~e~\spec{Q}@\ucon$, where $e$ is a program, $P$ is a
precondition that constrains a state in which the program is safe to
run, and $Q$ is a postcondition, describing a state upon the program's
termination. The last component of the tuple, $\ucon$, whose exact
syntax is different for various program logics, is typically referred
to as a \emph{concurrent protocol} or \emph{concurrent resource}.
%
% \an{Why not call $U$ a \emph{resource}, instead of concurrent
%   protocol.  Protocol is the name that has been pushed by CAP, but
%   resource is what everyone \emph{before} such as Brookes and O'Hearn,
%   have used for the concept.}
% 
% \is{Okay, I added the concurrent resource as an alternative. I don't
%   think it's important which name has appeared earlier, but in my
%   opion the world ``protocol" better explains the purpose of the
%   concept, whereas ``resource'' can be confused with the actual state,
%   e.g., heap.}
%
It defines the invariants of the shared state, that are respected by
all threads working with it
concurrently~\cite{Nanevski-al:ESOP14,OHearn:TCS07} and/or the allowed
state changes that the threads can make depending on the roles they
take in the protocol~\cite{Jones:TOPLAS83}.
%
To ensure thread-locality of the concurrent Hoare-style specs, the
assertions $P$ and $Q$, should be proven \emph{stable} with respect to
the concurrent protocol, \ie, they should be invariant under possible
changes that interfering environment threads can make to the state
according to $\ucon$.

Program logics provide a naturally compositional way to specify
concurrency: once a concurrent object is verified against a suitable
spec, its code is not required to be re-examined ever
again. Therefore, specification and verification of the client
programs (including other data structures that use the object) can be
performed out of the object's spec.
%
Program logics were used with a great success to specify and verify
such concurrent data structures and algorithms as
barriers~\cite{Dodds-al:POPL11,Hobor-Gherghina:ESOP11},
indices~\cite{ArrozPincho-al:OOPSLA11}, flat
combiner~\cite{Turon-al:ICFP13,Sergey-al:ESOP15}, shared graph
manipulation~\cite{Raad-al:ESOP15,Sergey-al:PLDI15}, as well as their
multiple client programs.
%
Program verification in such logics is done structurally, \ie, by
systematically applying syntax-directed inference rules, until the
spec is proven, and by now several mechanized tools for logic-based
reasoning about concurrency has been
released~\cite{Sergey-al:PLDI15,Appel-al:BOOK14}.

That is, the program logics-based approach seems like an answer to the
question we posed above, as logical specs satisfy the properties
\textbf{(a)}--\textbf{(c)}. But in order to fully adopt it, we first
need to figure out how to express the existing patterns of
\emph{history-based} reasoning in terms of \emph{state-based} logical
specs.
%
Several attempts to do so have been taken
recently~\cite{Fu-al:CONCUR10,Gotsman-al:ESOP13,Sergey-al:ESOP15,OHearn-al:PODC10},
however, all these works were focusing exclusively on specifying
\emph{linearizable} concurrent objects, thus, making us to rephrase
the question \textbf{Q} as follows:

\vspace{4pt}
\noindent
\textbf{Q$'$:} \emph{Can we employ existing program logics for
  specifying \textbf{non-linearizable} concurrent objects and
  reasoning about their clients?}
% 
\vspace{4pt}

\vspace{-8pt}
\noindent
In this paper we answer to this question affirmatively.

\subsection{Logic-based specifications for concurrent objects}
\label{sec:logic-based-spec}

We build our solution on the recently made observation that histories,
describing relevant atomic changes in the logical state of a
concurrent object, can be expressed as an instance of \emph{auxiliary
  state}~\cite{Sergey-al:ESOP15} (a generalization of auxiliary, or
\emph{ghost}, variables~\cite{Owicki-Gries:CACM76}, customary in
logic-based concurrency verification).
%
That is, reasoning about histories follows exactly the same patterns
that reasoning about \emph{heaps} follows in separation
logic~\cite{Reynolds:LICS02}.
%
We then use the expressive power of concurrent protocols to leverage
this observation for specifying and verifying non-linearizable
concurrent objects.  
%
% \an{Throw in the word "separation logic" somewhere here, to make the
%   point of our observation more precise. Strictly speaking, that you
%   can have histories as ghosts was an old observation (they even call
%   them "history variables" in the model checking works). Our
%   observation was that \emph{separation logic} can natually and
%   compositionally handle history variables, using the algebraic
%   structure of PCMs.}
%
% \is{okay, done}
%

In particular, we show how to define invariants of a concurrent object
in a way that they constrain the real and auxiliary state, capturing a
suitable specification set of histories (\cf (\emph{i})), \eg,
\emph{concurrency-aware} one.
%
The freedom to define histories in any way we need (as they are just
an instance of auxiliary state) allows us to record additional logical
information into them, capturing quantitative aspects of the expected
interference. In combination with a possibility to describe the
allowed changes in the real and auxiliary state of an object, it
provides us with a technique to express and verify diverse consistency
conditions on the histories (\cf (\emph{ii})).  

What is crucial for capturing the essence of most of these conditions
is the ability to formally quantify in \emph{thread-local} assertions
over the arbitrary effects produced by \emph{other} interfering
threads and the ``shape'' of the environment (\eg, a number of threads
running concurrently with the one being specified).
%
% \an{It sounds like we also need to work in the word
%   \emph{subjectivity}, in order to express what's new in our approach,
%   and why the previous work didn't quite succeed in capturing what we
%   propose. The point should be that quiescent consistency, and others,
%   just naturally want to have access to the contributions of others.}
%
% \is{okay, check the paragraphs above and below.}
%
Such ability, dubbed \emph{subjectivity}, has been introduced in the
recently developed \emph{Subjective} and \emph{Fine-grained Concurrent
  Separation Logics}, SCSL~\cite{LeyWild-Nanevski:POPL13} and
FCSL~\cite{Nanevski-al:ESOP14}. This is the reason why we have chosen
to use FCSL as a specification and verification framework.
%
% While some other logics could have been employed for this role (see
% Section~\ref{sec:related} for discussion on alternatives),

In addition to the native support of subjectivity, FCSL was appealing
to us because of its uniform model of thread-local resources, which is
based on \emph{partial commutative monoids} (PCMs) and can be
instantiated to reason about arbitrary state, auxiliary or real, such
as heaps, thread capabilities, and, indeed, histories.
%
Finally, FCSL has been implemented as a mechanized tool for concurency
verification~\cite{Sergey-al:PLDI15}, enabling provably sound
computer-aided reasoning about highly optimized concurrent objects,
whose state invariants tend to be quite complicated.

% \an{This section reads meekly, but it's the
%   most important one.  It should be strenghtened by bringing up the
%   points about subjectivity further above, as I said.  Subjectivity as
%   a crucial and most important idea, which makes all the
%   difference. If we don't mention it, peple like Noam Rinetzky or
%   Cesar Sanches will not get the feel of what's different.
%   Subjectivity is the delta that makes us succeed where everyone else
%   went into wrong directions.  Also, we should be more proud of our
%   work. It doesn't matter if "other logics" could have been
%   used. Subjectivity has been invented by SCSL and in the fine-grained
%   setting by FCSL. Just because some copy-cats later decided to
%   implement it in their logics doesn't mean we should bow to them. So
%   I think we don't need to mention "other logics can do it" part. Just
%   say that we use FCSL to explain the ideas of this paper, because
%   FCSL is based on, and introduced the idea of subjectivity.  Of
%   course, this should be done after subjectivity has been promoted
%   further above as the main idea that makes everything fly.}

\subsection{Contributions and paper outline}
\label{sec:contr-paper-outl}

In the remainder of the paper, we demonstrate the viability of the
logic-based approach for defining correctness conditions for highly
parallel concurrent objects by formally specifying and verifying two
concurrent data structures: an \emph{elimination-based exchange
  channel}~\cite{Scherer-al:SCOOL05} (or simply \emph{exchanger}) and
a simple \emph{counting network}~\cite{Aspnes-al:JACM94}, whose
behavior was previously described only in terms of dedicated
criteria~\cite{Hemed-Rinetzky:PODC14,Derrick-al:FM14,Jagadeesan-Riely:ICALP14}. We
then argue for the adequacy of the provided specs by modularly
verifying a series of concurrent client programs, which employ these
data structures.

\noindent
Specifically, in this work we make the following contributions:

\vspace{2pt}

\begin{itemize}

\item We describe a series of novel reasoning patterns that unify
  state-based and history-based approaches for specification and
  verification of concurrent objects.

\item We provide the first formal logic-based specification of an
  \emph{elimination-based concurrent
    exchanger}~\cite{Scherer-al:SCOOL05} in the spirit of
  concurrency-aware linearizability~\cite{Hemed-Rinetzky:PODC14}
  (Section~\ref{sec:overview});

\item We specify and formally verify of a realistic client of the
  exchanger adapted directly from the \code{java.util.concurrent}
  library documentation~\cite{ExchangerClass} (Section~\ref{sec:cal}).

\item We give the first logical specification to a simple
  \emph{counting network}~\cite{Aspnes-al:JACM94} and verify two its
  clients, exploiting the implications of the proved spec in the
  spirit of \emph{quiescent consistency}~\cite{Derrick-al:FM14} and
  \emph{quantitative quiescent
    consistency}~\cite{Jagadeesan-Riely:ICALP14}
  (Section~\ref{sec:counting}).

\item We supply all examples from the paper with proof scripts that
  were mechanically checked in the Coq proof
  assistant~\cite{Coq-manual,Bertot-Casteran:BOOK,Sergey-al:PLDI15}.

\end{itemize}

\vspace{2pt}

\noindent
Section~\ref{sec:background} provides necessary minimal background on
program logics for concurrency and key concepts of
FCSL~\cite{Nanevski-al:ESOP14}. In Section~\ref{sec:discussion} we
discuss other possible applications of the proposed approach for
reasoning about concurrent objects. We cover the relevant related work
in Section~\ref{sec:related} and conclude in
Section~\ref{sec:conclusion}.

\an{We should have in the intro with a paragraph like follows: That
  linearizability can adequatly be replaced by Hoare-style reasoning
  has been already argued by the previous work on HOCAP and ICAP,
  which employ the method of parametrization (to be skethed in the
  related work section). In this paper, we argue that similar
  replacement can be carried out for three other alternative
  consistency criteria such as Concurrency-aware linearizability
  (CAL), quiescent consistency, and quantitative quiescent
  consistency. Moreover, the last two consistency criteria seem
  impossible to address by a method of parametrization, at least
  without some significant and complicated meta-theoretic additions
  (e.g., prophecy variables), whereas here we show how they can be
  easily supplanted using reasoning based on subjectivity combined
  with histories.}

\is{I disagree with the remark by Aleks almost entirely. First, I'm
  sick and tired of giving credit to CAP-like approaches for something
  they don't have enev a slight idea how to do. While I'll most
  certainly put something on them into the related work, I don't want
  to have enything on them in the intro, otherwise there will be
  another round of what we've seen already a year ago.
%  
  Second, I don't think that remark on prophecies is sound: the
  counting network example doesn't have anything reminiscent to
  prophecy-requiring linearization points. That is, any comparison to
  LP-based methods, as they're done in CAP, is dangerous and
  misleading, as (a) they might be able to do it via some other
  callback-related mumbo-jumbo (or they might not, but we'll get a
  strong reject on that grounds anyway, just like the last tim), or
  (b) they indeed cannot do it, but this fact has nothing to with with
  their prophecies-related troubles. 
%
  Finally, the only people who will be able to understand this
  comments are those who will give us strong reject because of it
  (just like the last time). That is why I suggest us to focus bringin
  Henzinger and Rinetzky-like crown on our side by speaking the
  language they understand.
%
  So, the bottom line: I'm strongly against any specific on
  CAP-related logics or any of this stuff in the intro.
}

\an{Gee, that's a strong sentiment. I will just reiterate that HOCAP
  and ICAP have suggested that Hoare logic should replace
  linearizability. It seems prudent to be generous and give them
  credit for that, especially as it doesn't cost us anything
  (basically, just one line of space), and doesn't diminish from our
  contribution at all. A year ago, we got rejected form POPL precise
  because we didn't have a forward-pointer in the intro to the related
  work section, where the comparison was done. They accused us in the
  after-rebutal comment of setting '`wrong expectations'' in our
  intro. After that, in the ESOP version, we did put the forward
  pointer. So, I really don't think a forward pointer would hurt us
  here, and can only help. As for prophecies, I just put them as a
  side comment. I don't think we really have to mention them. We could
  simply say that its unclear that parametrization can be used to
  handle quiescense and quantitativeness, wihtout speculiating what's
  needed to fix that. I do personally think that prophecies will
  probably be OK for that, if they can get them. But then, I think
  they'll never get them, precisely because subjectivity and histories
  is what you need here :-). As for Henzinger and Rinetzky crowd:
  well, Philippa will be reviewing this, so expect to get someone from
  the CAP crowd too.}


%\lipsum[1-4]
