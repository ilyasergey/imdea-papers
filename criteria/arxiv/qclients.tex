
\section{Reasoning about Counting Network's Clients}
\label{sec:qclients}

We next demonstrate how the spec~\eqref{eq:qc-spec} can be used in
clients, to establish properties \textbf{R2} and \textbf{R3}. These
properties have been addressed in the previous work using dedicated
consistency criteria of quiescent
consistency~\cite{Aspnes-al:JACM94,Derrick-al:FM14} and quantitative
quiescent consistency and
quasi-linearizability~\cite{Afek-al:OPODIS10,Jagadeesan-Riely:ICALP14},
but here we derive them compositionally, \ie, out of the spec of
\code{getAndInc}. This will demonstrate the usefulness of Hoare logic
in deriving properties related to the various flavors of quiescent
consistency.

We illustrate \textbf{R2} and \textbf{R3} by means of two different
clients which execute two sequential calls to \code{getAndInc}. The
first client will exhibit a quiescence between the two calls, and we
will prove that the call results appear in order, as required by
\textbf{R2}. The second client will exhibit an interference by a
program with a number $N$ of calls to \code{getAndInc}, and we will
derive a bound on the results in terms of $N$, as required by
\textbf{R3}.

%Even though we don't have a formal proof of a correspondence between
%our spec of \code{getAndInc} and these two consistency conditions,
%stated in term of call/return histories, we nevertheless show that the
%spec we provided can be immediately used to infer the requirements
%\textbf{R2} and \textbf{R3} in the situations, described by these
%criteria.

Both examples will rely on the general mechanism of hiding, as a way
to logically restrict the interference on a resource, in this case, a
counting network~$\ccon$. Specifically, we will use the following
derived rule for hiding, similar to the one in
Section~\ref{sec:background}.
%
\[
{\small{
\begin{array}{c}
\spec{P}~e~\spec{Q} @ \ccon\\[2pt]
\hline\\[-7pt]
\specK{\{\heaps = \Phi_1(\heapj), \Phi_1(P)\}}\ \hide_{\Phi_1}~e\ \specK{\{\exists \Phi_2\ldot \heaps = \Phi_2(\heapj), \Phi_2(Q)\}} @ \cal P
\end{array}
}}
\]
%
%
\[
\tag{\normalsize \arabic{tags}}\refstepcounter{tags}\label{eq:hide2}
{\small{
\begin{array}{rr@{\ }c@{\ }l}
\text{\normalsize{where}} &
\Phi_1 & \eqdef & [\emptyset/\tkns, \gist_0/\hists, \heap_0/\heapj,
\emptyset/\tkno, \histo/\hists] 
\\[2pt]
&\heap_0 & \eqdef & \bal \hpts 0 \hunion c_0 \hpts 0 \hunion c_1 \hpts 1     
\\[2pt]
&\gist_0 & \eqdef & (0 \hpts (\set{0}, \emptyset, 0), 1 \hpts (\emptyset,
\set{1}, 1))
\end{array}
}}
\]
%
That is, $\heap_0$ defines the heap storing the initial state of the
counting network and $\gist_0$ provides the ``default'' history for
the initial values 0 and 1 of $c_0$ and $c_1$, with the corresponding
tokens represented by natural numbers 0 and 1.  As always with hiding,
any $\Phi_2$ from the postcondition will map $\tkno$ and $\histo$ to
$\emptyset$.

\subsection{Exercising quiescent consistency}
\label{sec:qc-client}

\begin{figure}
\centering
\[
{\small{
\!\!\!\!\!\!\!\!
\begin{array}{c}
  \spec{\!\!
  \begin{array}{c}
    \tkns = \emptyset,
    \hists = \gists,
    \gisto \subseteq \histo, \Ic{\gisto}{\ikno}, \\[2pt]
    \ikno \subseteq \tkno \hunion (\Tomb~\histo \setminus \Tomb~\gisto)
  \end{array}
  \!\!}
\\\\[-5pt]
%
%~~~\spec{\!\!
%{\scriptsize{  \left(\!\!\!
%  \begin{array}{c}
%    \tkns = \emptyset,
%    \hists = \gists,
%    \gisto \subseteq \histo,
%    \\[1pt]
%    \ikno \subseteq \tkno \hunion (\Tomb~\histo \setminus \Tomb~\gisto)
%  \end{array}
%\!\!\!\right)
%}}
%\!\circledast\!
%{\scriptsize{
%  \left(\!\!\!\!\begin{array}{c}
%    \hists = \emptyset\aand 
%    \tkns = \emptyset\aand \\[1pt]
%    \ikno \hunion \Tomb~\gisto  \subseteq \tkno \hunion \Tomb~\histo
%  \end{array}\!\!\!\!\right) 
%}}
%\!\!}
%\\\\[-5pt]
% 
  \begin{tabular}{c || c}
$
{\scriptsize{
  \spec{\!\!
  \begin{array}{c}
    \tkns = \emptyset,
    \hists = \gists,
    \gisto \subseteq \histo, \Ic{\gisto}{\ikno},
    \\[1pt]
    \ikno \subseteq \tkno \hunion (\Tomb~\histo \setminus \Tomb~\gisto)
  \end{array}
  \!\!}
}}$
&
$
\!\!{\scriptsize{
\spec{\!\!\!\begin{array}{c}
    \hists = \emptyset\aand
    \tkns = \emptyset\aand \\[1pt]
    \ikno  \hunion \Tomb~\gisto ~\subseteq \\[1pt]
    ~~~~~~~~~\tkno \hunion \Tomb~\histo 
  \end{array}\!\!\!}
}}$
\\[3pt]
   $\esc{getAndInc()}$ & ${\small{e_i}}$ 
\\[3pt] 
$
\spec{\!\!\!
{\scriptsize{
  \begin{array}{c}
    \exists \iknh^0~\iknh^1~z \ldot     
    \tkns = \emptyset \aand \gisto \subseteq \histo, \Ic{\gisto}{\ikno},\\[1pt]    
    \hists = \gists \hunion (\res\!+\!2) \hpts (\iknh^0, \iknh^1,z)\aand \\[1pt]
    \ikno \subseteq \tkno \hunion (\Tomb~\histo \setminus \Tomb~\gisto) \aand \\[1pt]
    \last~(\gists \hunion \gisto)  < \res + 2 ~+~~~~~~~~\\[1pt]
    ~~~~~~~~~~~~ 2 \ (|\iknh^0 \cap \ikno^0|\!+\!|\iknh^1 \cap \ikno^1|)
  \end{array}
}}
  \!\!\!}$
&
$\spec{\!\!\!
{\scriptsize{
  \begin{array}{c}
    \exists \gist_i \ldot \hists = \gist_i\aand 
    \tkns = \emptyset\aand \\[1pt]
    \ikno  \hunion \Tomb~\gisto ~\subseteq \\[1pt]
    ~~~~~~~~~\tkno \hunion \Tomb~\histo
  \end{array}
}}
\!\!\!}$
\end{tabular}
\\\\[-5pt]
%\spec{\!\!
%{\scriptsize{
%  \left(\!\!\!\!\begin{array}{c}
%    \exists \iknh^0~\iknh^1~z \ldot     
%    \tkns = \emptyset \aand \\[1pt]    
%    \hists = \gists \hunion (\res.1 \!+\! 2) \!\hpts\! (\iknh^0, \iknh^1,z)\aand \\[1pt]
%    \ikno \subseteq \tkno \hunion (\Tomb~\histo \setminus \Tomb~\gisto) \aand \\[1pt]
%    \last~(\gists \hunion \gisto)  < \res.1 + 2 ~+~~~~~~~~\\[1pt]
%    ~~~~~~~~~~~~ 2 \ (|\iknh^0 \cap \ikno^0|\!+\!|\iknh^1 \cap
%                  \ikno^1|)
%  \end{array}\!\!\!\!\right)
%}}
%%
%\!\!\circledast \!\!
%%
%{\scriptsize{
%  \left(\!\!\!\begin{array}{c}
%    \exists \gist_i \ldot \hists = \gist_i\aand
%        \tkns = \emptyset\aand \\[1pt]
%    \ikno  \hunion \Tomb~\gisto ~\subseteq \\[1pt]
%    ~~~~~~~~~~\tkno \hunion \Tomb~\histo
%  \end{array}\!\!\!\!\right)
%}}
%\!\!}
%\\\\[-4pt]
~~~~\spec{\!\!
\begin{array}{c}
\exists \iknh^0~\iknh^1~\gist_i \ldot  
\tkns = \emptyset\aand \hists = \gists \hunion \gist_i \hunion (\res.1 + 2) \hpts (\iknh^0,
  \iknh^1, -),\\[1pt]
\gisto \subseteq \histo\aand \ikno \subseteq \tkno \hunion
(\Tomb~\histo \setminus \Tomb~\gisto), \Ic{\gisto}{\ikno},\\[1pt]
    \last~(\gists \hunion \gisto)  < \res.1 + 2 +
    2 \ (|\iknh^0 \cap \ikno^0|\!+\!|\iknh^1 \cap \ikno^1|)
%
\end{array}
\!\!}@\ccon
%
\end{array}
}}  
\]
%
\caption{Parallel composition of \code{getAndInc} and~$e_i$ in~\eqref{eq:eqc}.}
  \label{fig:example1} 
\end{figure}
%
%


Our first client is the following program~$\eqc$:
%
\[
\tag{\normalsize \arabic{tags}}\refstepcounter{tags}\label{eq:eqc}
{\small{
\begin{array}{ll} 
\Num{1} & (\res_1, -) \Asgn (\esc{getAndInc()} ~||~ e_1) \esc{;} \\[1pt]
\Num{2} & (\res_2, -) \Asgn (\esc{getAndInc()} ~||~ e_2) \esc{;} \\[1pt]
\Num{3} &  \kw{return}~(\res_1, \res_2) 
\end{array}
}}
\]
%
Each of the calls to \esc{getAndInc} interferes with either $e_1$ or
$e_2$, but in the absence of \emph{external} interference, the
quiescent state is reached between the lines 1 and 2. Hence, after
executing $\esc{hide}~\eqc$, it should be $\res_1 < \res_2$.

The programs $e_1$ and $e_2$ are arbitrary. They can invoke
\code{getAndInc} and modify the counters concurrently with the two
calls of $\eqc$. We capture this property by giving both the following
generic spec.
%
\[
%
\tag{\normalsize \arabic{tags}}\refstepcounter{tags}\label{eq:eispec}
{\small
\!\!\!\!\!\!\!\! 
\begin{array}{c}
  \spec{~
  \hists = \emptyset\aand
  \tkns = \emptyset\aand
   \ikn \subseteq \tkno \hunion \Tomb~\histo
  ~}
  \\[1pt]
  e_i
  % 
  \\[1pt]
  \spec{\!\!\!
  \begin{array}{c}
    \exists \gist_i \ldot \hists = \gist_i\aand 
    \tkns = \emptyset\aand 
    \ikn \subseteq \tkno \hunion \Tomb~\histo
  \end{array} 
  \!\!\!}@\ccon
%
\end{array}
}
\]
%
The postcondition allows for a number of increments via calls to
\esc{getAndInc}, which is reflected in the addition $\gist_i$ to
$\hists$. However, all such calls are required to be \emph{finished}
by the end of $e_i$ ($\tkns = \emptyset$). As customary by now, we use
the logical variable $\ikn$ to name the initial set of \emph{other}
tokens in a stable manner.

Figure~\ref{fig:example1} provides a spec and proof for each of
the parallel composition in the program~\eqref{eq:eqc}.
%
%
We start by splitting the initial state into two subjective views via
$\circledast$~\eqref{eq:ssep} and apply the rule of parallel
composition~\eqref{eq:parrule}. The two threads trivially split the
initial empty set of tokens $\tkns$. On the other hand, the left
thread receives the initial history $\gists$, while the initial
history for the right thread is empty. Thus, the precondition for the
left thread is a copy of the global precondition for the parallel
composition, but it is a bit more complicated for the right thread.
%
In particular, we notice that the global precondition implies
$\ikno \subseteq \tkno \hunion \Tomb~((\histo \hunion \gists)
\setminus \Tomb~\gisto)$,
simply by weakening the set inclusion. But then, the subjective
variable $\histo$ in the right sub-thread is the union of $\gists$
with $\histo$ from the global precondition, reflecting that the left
sub-thread---with \emph{self}-history $\gists$---is part of the
environment for the right sub-thread. Thus, if we interpret $\histo$
in the right sub-thread:
$\ikno \subseteq \tkno \hunion (\Tomb~\histo \setminus \Tomb~\gisto)$,
after which we can move $\Tomb~\gisto$ to the other side of inclusion,
to obtain
$\ikno \hunion \Tomb~\gisto \subseteq \tkno \hunion \Tomb~\histo$. 
%
% as shown in Figure~\ref{fig:example1}.

%
% which holds, since both $\ikno$ and $\gisto$ are fixed, and the union
% $ \tkno \hunion \Tomb~(\gists \hunion \histo)$ may only grow.

%
We next employ the specifications~\eqref{eq:qc-spec}
and~\eqref{eq:eispec} (instantiating $\ikn$ with $\ikno \hunion
\Tomb~\gisto$), and join the views to the resulting state via
$\circledast$, keeping only relevant parts of~\eqref{eq:qc-spec}'s
postcondition (\ie, omitting the $\happrox$ part, as it is not needed
for this example). We also substitute $\res$ by $\res.1$, as required
by~\eqref{eq:parrule}.
%
The transition to the last assertion requires us to reconcile the
views to the \emph{other}-components $\tkno \hunion \Tomb~\histo$,
which are different for the parallel threads. To do so, we use the
following simple set-theoretic lemma:

\vspace{3pt}
\begin{lemma}
\label{lm:cheb}
For any sets $v$, $u$, $u_1$, and $u_2$, if
$u_1 \cap u_2 = \emptyset$, $v \subseteq u_1 \hunion u$, and
$v \subseteq u_2 \hunion u$, then $v \subseteq u$.
\end{lemma}
\vspace{3pt}
%
\noindent
%

We apply this lemma to obtain $\ikno \hunion \Tomb~\gisto \subseteq
\tkno \hunion \Tomb~\histo$ in the end, which is equivalent to the
desired $\ikno \subseteq \tkno \hunion (\Tomb~\histo \setminus
\Tomb~\gisto)$. The lemma is applied taking $u = \tkno \hunion
\Tomb~\histo$, $v = \ikno \hunion \Tomb~\gisto$, $u_1 =
\Tomb~\gist_i$, and $u_2= (\Tomb~\gists) \hunion \set{z}$. $u_1$ and
$u_2$ are disjoint, thanks to the state invariants~\ref{cn:hvalid}
and~\ref{cn:ti}, and the inclusions derive from the postconditions of
the parallel sub-threads in Figure~\ref{fig:example1}.

Next, we use the spec from Figure~\ref{fig:example1} to specify and
verify the program $\eqc$, assuming external interference.
%
\[
\!\!\!
{\small{
\begin{array}{c}
  \spec{\!\!
  \begin{array}{c}
    \tkns = \emptyset,
    \hists = \gists,
    \gisto \subseteq \histo, \ikno \subseteq \tkno \hunion (\Tomb~\histo \setminus \Tomb~\gisto)\\[2pt]
\mbox{where $\gisto = \histo$ and $\ikno = \tkno$, hence $\Ic{\gisto}{\ikno}$}
  \end{array}
  \!\!}
  ~\comm{P}
  \\\\[-6pt]
  (\res_1, -) \Asgn (\esc{getAndInc()} ~||~ e_1) \esc{;}
  \\[3pt]
%  \spec{\!\!\!
%\begin{array}{c}
%\exists \gist_1 \ldot     
%\tkns = \emptyset\aand
%\hists = \gists \hunion \gist_1 \hunion (\res_1 + 2) \hpts -
%\end{array}
%\!\!\!}
%
%\\\\[-5pt]
%
{{
\spec{\!\!
\begin{array}{c}
 \exists \gist_1\ldot \tkns = \emptyset\aand \hists = \gists', \gisto \subseteq \histo \aand \ikno \subseteq \tkno \hunion (\Tomb~\histo \setminus \Tomb~\gisto)\\[2pt]
\mbox{where 
 $\gists' = \gists \hunion
\gist_1 \hunion (\res_1 + 2)\mapsto -$, $\gisto = \histo$ and $\ikno =
\tkno$,} \\[2pt]
\mbox{hence $\Ic{\gisto}{\ikno}$}
%  \gists' := \hists,\\[2pt] 
%  \gisto := \histo, 
%  \ikno := \tkno,
  \end{array}
\!\!\!}
}}
%
\\\\[-5pt]
(\res_2, -) \Asgn (\esc{getAndInc()} ~||~ e_2) \esc{;}      
\\[3pt]
\spec{\!\!\!\!
\begin{array}{c}
\exists \gist_1~\gist_2~\iknh^0~\iknh^1 \ldot     
%
\tkns = \emptyset \aand 
\hists = \gists' \hunion \gist_2 \hunion (\res_2 + 2)\mapsto (\iknh^0, \iknh^1, -)\aand\hbox{}\\[2pt] 
  \ikno \subseteq \tkno \hunion (\Tomb~\histo \setminus \Tomb~\gisto),  \Ic{\gisto}{\ikno},\\[1pt]
%   \gists' = \gists \hunion \gist_1 \hunion (\res_1 + 2) \hpts -,
%  \\[1pt]
    \last~(\gists' \hunion \gisto) < \res_2 + 2 + 2 \ (|\iknh^0 \cap \ikno^0|\!+\!|\iknh^1 \cap
    \ikno^1|) 
\end{array}
\!\!\!\!\!}~\comm{Q}
\\\\[-7pt]
~~~~~~~~~~~~~~~~~~~~~~~~~~\kw{return}~(\res_1, \res_2); ~\comm{\res = (\res_1, \res_2)}
\\[2pt]
\spec{~Q(\res.1/\res_1, \res.2/\res_2)~}@\ccon
\end{array}
}} 
\]
%
We start by instantiating the logical variables $\gists$, $\gisto$ and
$\ikno$ from Figure~\ref{fig:example1} with $\gists$, $\histo$ and
$\tkno$, respectively. We name the obtained precondition $P$.
%
From the postcondition for $\res_1$ from Figure~\ref{fig:example1}, we
keep only the clauses constraining $\tkns$, $\tkno$, $\hists$ and
$\histo$. To verify the second call, we instantiate $\gists$,
$\gisto$ and $\ikno$ from Figure~\ref{fig:example1} with $\gists' =
\gists \hunion \gist_1 \hunion (\res_1 + 2)\mapsto -$, $\gisto$, and
$\ikno = \tkno$, and we obtain the postcondition, which we name $Q$.

%We also ``snapshot'' the components post-state of the first parallel
%composition, which is our ``candidate'' for the role of quiescent
%state, with $\gists'$, $\ikno$, and $\gisto$. We use these values to
%instantiate the spec from Figure~\ref{fig:example1} for the second
%call.
%%
%In the resulting postcondition, we keep the equality involving the
%self-history of the pre-state $\gists'$.

Now, to ensure the absence of external interference, we consider the
program $\hide_{\Phi_1}~\eqc$, and apply the hiding rule, taking $P$
and $Q$ to be the assertions from the proof outline above,
and~$\gists$ in $P$ is instantiated with~$\gist_0$~\eqref{eq:hide2}.
%
By the general condition on $\Phi$, we know that in that case $\histo
= \emptyset$ and $\tkno = \emptyset$ in $Q$.
%
Therefore, from the set inclusion on $\ikno$ in $Q$, we deduce that
$\ikno = \emptyset$ and, hence, $\ikno^0 = \emptyset$ and $\ikno^1 =
\emptyset$.
%
As a consequence, all intersections in $Q$ are \emph{empty}, so from
the inequalities we obtain
\[
%
\tag{\normalsize \arabic{tags}}\refstepcounter{tags}\label{eq:sapprox1}
%
%{\small{
\begin{array}{c}
 \last~(\gists' \hunion \gisto) < \res.2 + 2
\end{array}
\hfill
%}}
\]
%
%\noindent
%
But $\gists'$ is defined as $\gists \hunion \gist_1 \hunion
(\res.1+2)\mapsto -$. Hence, $\res.1 + 2 \in \mathsf{dom}\ \gists'$,
and thus $\res.1 + 2 \le \mathsf{last}\ \gists'$. Even more:
%
\[
%
\tag{\normalsize \arabic{tags}}\refstepcounter{tags}\label{eq:inv1}
%
%{\small{
\begin{array}{c}
\res.1 + 2 \le \last~(\gists' \hunion \gisto).
\end{array}
\hfill
%}}
\]
%
The last two inequalities now directly imply the desired quiescence
result \textbf{R2}: $\res.1 < \res.2$.
%under the precondition $P$ that $\hists$ is arbitrary $\gists$ and
%$\tkns = \emptyset$ (hence, by framing, also arbitrary).

\subsection{Proving quantitative bounds}
\label{sec:qqc-client}

We next show how the very same spec~\eqref{eq:qc-spec} also obtains
quantitative bounds on the out-of-order effects of the network,
expressed in terms of a number of running threads in the following
program $\eqqc$:
%
\[
\tag{\normalsize \arabic{tags}}\refstepcounter{tags}\label{eq:eqqc}
{\small{
\begin{tabular}{l || l}
$
\begin{array}{ll} 
\Num{1} & \res_1 \Asgn \esc{getAndInc();} \\[1pt]
\Num{2} & \res_2 \Asgn \esc{getAndInc();}  \\[1pt]
\Num{3} & \kw{return}~(\res_1, \res_2)
\end{array}
$  
&
$~~~e$
\end{tabular} 
}}
\]
%
Here, $e$ has a spec saying that the \emph{number} of calls to
\esc{getAndInc} in~$e$ (\ie, the size of interference $e$ exhibits) is
some fixed $N$:
%
\[
%
\tag{\normalsize \arabic{tags}}\refstepcounter{tags}\label{eq:espec}
{\small
\!\!\!\!\!\!\!\! 
\begin{array}{c}
  \spec{~
  \tkns = \emptyset,
  \hists = \gists ~}
~  e
~  \spec{\!\!\!
  \begin{array}{c}
    \exists \gist \ldot 
    \tkns = \emptyset,
    \hists = \gists \hunion \gist,
    |\gist| = N
  \end{array} 
  \!\!\!}@\ccon
%
\end{array}
}
\]
%
Our goal is to prove that in the absence of external interference for
$\eqqc$, $\res_1 < \res_2 + 2 \ N$ (requirement \textbf{R3}).

\begin{figure}
\centering
%    
\[
\!\!\!
{\small{
\begin{array}{c}
  \spec{\!\!
  \begin{array}{c}
    \tkns = \emptyset,
    \hists = \gists
  \end{array}
  \!\!}
\\\\[-6pt]
\spec{\!\!
  \begin{array}{c}
    \tkns = \emptyset,
    \hists = \gists,
    \gisto \subseteq \histo,
    \ikno \subseteq \tkno \hunion (\Tomb~\histo \setminus \Tomb~\gisto)\\[2pt]
    \mbox{where $\gisto = \histo$ and $\ikno = \tkno$, hence $\Ic{\gisto}{\ikno}$}
%    \histo \subseteq \histo,\\[2pt]
%    \tkno \hunion \Tomb~\histo \subseteq \tkno \hunion \Tomb~\histo
  \end{array}
  \!\!}
\\\\[-6pt]
\res_1 \Asgn \esc{getAndInc();}
\\[3pt]
\spec{\!\!
\begin{array}{c}
   \exists \ikn^0~\ikn^1 \ldot 
   \tkns = \emptyset, \hists = \gists', \gisto \subseteq \histo, \ikno \subseteq \tkno \hunion (\Tomb~\histo \setminus \Tomb~\gisto)\\[2pt]
   \mbox{where $\gists' = \gists \hunion (\res_1 + 2) \hpts (\ikn^0,
     \ikn^1, -)$, $\gisto = \histo$ and $\ikno = \tkno$,}\\[2pt]
   \mbox{hence $\Ic{\gisto}{\ikno}$}
  \end{array}
  \!\!}%
%\\\\[-6pt]
%\spec{\!\!\!
%  \begin{array}{c}
%  \tkns = \emptyset,  
%  \hists = \gists \hunion (\res_1 \!+\! 2) \!\hpts\! (\ikn^0, \ikn^1, -), 
%  \gists' := \hists,\\[2pt] 
%  \gisto := \histo, 
%  \ikno := \tkno,
%  \ikno \hunion \Tomb~\gisto \subseteq \tkno \hunion \Tomb~\histo
%  \end{array}
%  \!\!\!}%
\\\\[-5pt] 
\res_2 \Asgn \esc{getAndInc();}
\\[3pt]
\spec{\!\!\!
\begin{array}{c}
  \exists \ikn^0~\ikn^1~\iknh^0~\iknh^1~z \ldot     
  \tkns = \emptyset, \hists = \gists' \hunion (\res_2+2) \mapsto (\iknh^0, \iknh^1, z),\\[2pt]
  \happrox (\gists' \hunion \gisto)~\res_2~\iknh^0~\iknh^1~z
\end{array}
\!\!\!}
\\\\[-5pt]
\spec{\!\!\!
\begin{array}{c}
  \exists \ikn^0~\ikn^1~\iknh^0~\iknh^1~z \ldot     
  \tkns = \emptyset, \hists = \gists' \hunion (\res_2+2) \mapsto (\iknh^0, \iknh^1, z),\\[2pt]
  \iknh^0 \hunion \iknh^1 \subseteq \tkno \hunion (\Tomb~\histo) \hunion \set{z}\aand z \notin \ikn^0 \hunion \ikn^1,\\[2pt] 
  \res_1 + 2 < \res_2 + 2 + 2 \ (|\iknh^0 \cap \ikn^0| + |\iknh^1 \cap \ikn^1|)
\end{array}
\!\!\!}
\\\\[-5pt]  
~~~~~~~~~~~~~~~~~~~~~~~~~~~~~~~~
\kw{return}~(\res_1, \res_2) ~\comm{\res = (\res_1, \res_2)}
\\\\[-5pt]
\spec{\!\!\!
\begin{array}{c}
  \res.1 < \res.2 + 2 \ |\tkno \hunion \Tomb~\histo| 
  %
  %\aand\\[1pt] 
  % \tkns = \emptyset\aand
  % \hists = \gists \hunion (\res_1 + 2) \hpts -
  % \hunion (\res_2 + 2) \hpts -
  \end{array}
  \!\!\!}@\ccon
\end{array}
}} 
\]
%
%
\caption{Proof of sequential composition in~\eqref{eq:eqqc}.}
\label{fig:proof2}
\end{figure}

We first verify the sequential composition in lines 1-3
in~\eqref{eq:eqqc}; the proof outline is in Figure~\ref{fig:proof2}.
%
We start by instantiating the logical variables $\gists$, $\gisto$ and
$\ikno$ from spec~\eqref{eq:qc-spec} with $\gists$, $\histo$ and
$\tkno$, respectively.  In the postcondition of the first
\esc{getAndInc}, we keep only the clause involving $\tkns$, $\tkno$,
$\hists$ and $\histo$, and drop the rest. To verify the second
\esc{getAndInc}, we instantiate $\gists$, $\gisto$ and $\ikno$ with
$\gists' = \gists \hunion (res_1+2) \mapsto (\ikn^0, \ikn^1, -)$,
$\histo$ and $\tkno$.
%
In the postcondition of the second \esc{getAndInc}, we focus on the
$\happrox~(\gists' \hunion \gisto)~\res_2~\iknh^0~\iknh^1~z$ clause,
where $\iknh^0$, $\iknh^1$ are the tokens snapshot with $\res_2+2$.
Unfolding the definition of $\happrox$ from~\eqref{eq:happrox}, we
obtain $\iknh^0 \hunion \iknh^1 \subseteq \tkno \hunion \Tomb~\histo
\hunion\{z\}$. Also, using $(\res_1 +2)\mapsto (\ikn^0, \ikn^1, -)$ in
the implication that the unfolding obtains, we get $z \notin
\ikn^0 \hunion \ikn^1$ and 
\[
\tag{\normalsize \arabic{tags}}\refstepcounter{tags}\label{eq:le0}
{\small{\res_1 + 2 < \res_2 + 2 + 2 \ (|\iknh^0
\cap \ikn^0| + |\iknh^1 \cap \ikn^1|)
}}
\]
%
Now we use the following trivial lemma to simplify.
%
\vspace{2pt}
%
\begin{lemma}
\label{lm:intersect2}
For any sets $\hat{v}_0$, $\hat{v}_1$, $v_0$, $v_1$, and element $z$,
if $\hat{v}_0 \cap \hat{v}_1 = \emptyset $, $v_0 \cap v_1 =
\emptyset$, $z \in \hat{v}_0 \hunion \hat{v}_1$ and $z \notin {v}_0
\hunion {v}_1$, then $|\hat{v}_0 \cap v_0| + |\hat{v}_1 \cap v_1| \le
|\hat{v}_0\hunion \hat{v}_1| - 1$.
\end{lemma}
%
%
\vspace{2pt}
%
\noindent

Applying Lemma~\ref{lm:intersect2} to $\iknh^0$, $\iknh^1$, $\ikn^0$,
$\ikn^1$ and $z$, via~\ref{cn:ti1} derives:
%
\[
%\tag{\normalsize \arabic{tags}}\refstepcounter{tags}\label{eq:le1}
{\small{
|\iknh^0 \cap \ikn^0| + |\iknh^1 \cap \ikn^1| \le |\iknh^0 \hunion
\iknh^1| - 1
}}
\]
%
after which, the inclusion $\iknh^0 \hunion \iknh^1 \subseteq \tkno
\hunion \Tomb~\histo \hunion \set{z}$ leads to
%
\[
%\tag{\normalsize \arabic{tags}}\refstepcounter{tags}\label{eq:le2}
{\small{
|\iknh^0 \cap \ikn^0| + |\iknh^1 \cap \ikn^1| \le |\tkno \hunion
\Tomb~\histo|}}
\]
%
Combined with~\eqref{eq:le0}, this gives us $\res_1 < \res_2 + 2
\ |\tkno \hunion \Tomb~\histo|$, as shown in Figure~\ref{fig:proof2}.

Now we use spec from Figure~\ref{fig:proof2} to verify $\eqqc$.
%
\[
{\small{
\!\!\!\!\!\!\!\!
\begin{array}{c}
  ~~~~~~~~\spec{~
  \tkns = \emptyset,
  \hists = \gists
~} ~\comm{P}
\\[2pt]
%
%\spec{~
%(~  \tkns = \emptyset,
%  \hists = \gists~)
%\circledast
%  (\tkns = \emptyset, \hists = \emptyset)
%~}
%\\[2pt]
% 
  \begin{tabular}{c || c}
$
\spec{\!\!\!
    \begin{array}{c}
    \tkns = \emptyset,
    \hists = \gists
  \end{array}\!\!\!}
$
&
$
\spec{\!\!\!\begin{array}{c}
    \tkns = \emptyset,
    \hists = \emptyset
  \end{array}\!\!\!}
$
\\[3pt]
   $\begin{array}{l}
      \res_1 \Asgn \esc{getAndInc();}\\[1pt]
      \res_2 \Asgn \esc{getAndInc();}\\[1pt]
      \kw{return}~(\res_1, \res_2) ~\comm{\res = (\res_1, \res_2)}\!\!\!
    \end{array}$
\!\!\!\!\!\!
& ${\small{e}}$ 
\\\\[-5pt] 
$
\spec{\!\!\!
{\scriptsize{
  \begin{array}{c}
    \res.1 < \res.2 + 2 \ |\tkno \hunion \Tomb~\histo|
  \end{array}
}}
  \!\!\!}$
\!\!
&
\!\!$\spec{\!\!\!
{\scriptsize{
  \begin{array}{c}
    \exists \gist \ldot 
    \tkns = \emptyset,
    \hists = \gist, 
    |\gist| = N
  \end{array}
}}
\!\!\!}$
\end{tabular}
\\\\[-5pt]
%{\scriptsize{
%\begin{tabular}{c}
%$
%\spec{
%  \left( 
%  \res.1.1 < \res.1.2 \!+\! 2 \!\ \! |\tkno \!\hunion\! (\Tomb~\histo)|
%  \right)
%%
%\circledast
%%
%  \left(
%  \exists \gist \ldot 
%  \tkns = \emptyset,
%  \hists = \gist, 
%  |\gist| = N
%  \right)
%}$
%\end{tabular}
%}}
%\\[2pt]
\comm{\res_1 := \res.1.1, \res_2 := \res.1.2}
\\\\[-6pt]
\spec{
\res_1 < \res_2 + 2 \ |\tkno \hunion \Tomb~(\histo \hunion \gist)|
}
\\[3pt]
~~~~~~~~\spec{
\res_1 < \res_2 + 2 \ |\tkno \hunion \Tomb~\histo| + 2 \ N
}@\ccon ~\comm{Q}
%
\end{array}
}}  
\]
%
By the parallel composition rule~\eqref{eq:parrule}, the precondition
splits into two subjective views, where we send the history $\gists$
to the left thread, and the empty history to the right thread. The
proof from Figure~\ref{fig:proof2} then applies to the left thread, and
the spec~\eqref{eq:espec} applies to the right thread. We rename the
projections of $\res$ to join the subjective views, as required by
parallel composition. Finally, $\histo$ is the union of $\histo$ from
the joined thread with $\gist$, since the environment of the left
thread includes the right thread and the environment of the
join. Rewriting by this property in the postcondition of the left
thread gives us the postcondition of the joint thread: $\res_1 < \res_2
+2\ |\tkno \hunion \Tomb~(\histo \hunion \gist)|$, which then further simplifies
into
\[
\res_1 < \res_2 + 2\ |\tkno \hunion \Tomb~\histo| + 2\ N
\]
because $\Tomb$ distributes over $\hunion$, and $|\Tomb~\gist| =
|\gist| = N$. As the final step, we restrict the external interference
by considering $\hide_{\Phi_1}~\eqqc$. From the properties of hiding,
we deduce that both $\tkno$ and $\histo$ in $Q$ are empty, hence we
can further simplify into $\res_1 < \res_2 + 2 \ N$, which is the
desired result \textbf{R3}.
