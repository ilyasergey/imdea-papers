\documentclass[blockstyle,preprint]{sigplanconf}
%\documentclass[blockstyle,preprint,nocopyrightspace]{sigplanconf}

\usepackage{amssymb,amsmath,amsthm}
\usepackage{latexsym}
\usepackage{graphicx}
\usepackage{xcolor}
\usepackage{listings}
\usepackage{float}
\usepackage[scaled=.90]{helvet}
\usepackage[noend]{algorithmic}
\usepackage{mathrsfs}
\usepackage{mathpartir}
\usepackage{dsfont}
\usepackage{stmaryrd}
\usepackage{url}
\usepackage{textcomp} 
\usepackage{hyperref}
\usepackage{titlesec}
\usepackage{parskip}
\usepackage{alltt}
\usepackage{xspace}
\usepackage{verbatim}

% remarks
\newcommand{\todo}[1]{\textcolor{red}{({#1})}}
\newcommand{\is}[1]{\textcolor{blue}{(Ilya: {#1})}}
\newcommand{\an}[1]{\textcolor{red}{(Aleks: {#1})}}

% useful macros
\newcommand{\asgn}{\leftarrow}
\newcommand{\code}[1]{\lstinline{#1}}
\newcommand{\ccode}[1]{\code{#1}}
\newcommand{\var}[1]{\({#1}\)} 
\newcommand{\num}[1]{\({\text{{\scriptsize{#1}}}}\)}
\newcommand{\etc}{\emph{etc}}
\newcommand{\ie}{\emph{i.e.}\xspace}
\newcommand{\Ie}{\emph{I.e.}\xspace}
\newcommand{\eg}{\emph{e.g.}\xspace}
\newcommand{\Eg}{\emph{E.g.}\xspace}
\newcommand{\vs}{\emph{vs.}\xspace}
\newcommand{\etal}{\emph{et~al.}\xspace}
\newcommand{\adhoc}{\emph{ad hoc}\xspace}
\newcommand{\viz}{\emph{viz.}\xspace}
\newcommand{\dom}[1]{\mathsf{dom}(#1)}
\newcommand{\aka}{\textit{a.k.a.}\xspace}
\newcommand{\cf}{\textit{cf.}\xspace}
\newcommand{\wrt}{\emph{wrt.}\xspace}
\newcommand{\loef}{L\"{o}f}
\newcommand{\specK}[1]{\ensuremath{\textcolor{blue}{#1}}}
\newcommand{\spec}[1]{\specK{\left\{{#1}\right\}}}
\newcommand{\sep}{\textasteriskcentered}
\newcommand{\res}{\mathsf{res}} 
\newcommand{\ret}{\mathsf{ret}} 
\newcommand{\fix}{\mathsf{fix}} 



% Keep footnotes on one page
\interfootnotelinepenalty=10000 

\setlength{\parindent}{0.15in}
\setlength{\topsep}{0cm}
\setlength{\parskip}{0pt}
\titlespacing*{\section}{0pt}{*1}{*1}
\titlespacing*{\subsection}{0pt}{*1.0}{*0.5}
\titlespacing*{\subsubsection}{0pt}{*1.0}{*0.5}
\titlespacing*{\paragraph}{0pt}{*0.5}{*0.5}

% SSReflect listings 
% colors
\definecolor{shadecolor}{gray}{1.00}
\definecolor{darkgray}{gray}{0.30}
\definecolor{violet}{rgb}{0.56, 0.0, 1.0}
\definecolor{forestgreen}{rgb}{0.13, 0.55, 0.13}

% Col language definition
\lstdefinelanguage{Coq} {
mathescape=true,						
texcl=false,
morekeywords=[1]{
  Add,
  All,
  Arguments,
  Axiom,
  Bind,
  Canonical,
  Check,
  Close,
  CoFixpoint,
  CoInductive,
  Coercion,
  Contextual,
  Corollary,
  Defined,
  Definition,
  Delimit,
  End,
  Example,
  Export,
  Fact,
  Fixpoint,
  Goal,
  Graph,
  Hint,
  Hypotheses,
  Hypothesis,
  Implicit,
  Implicits,
  Import,
  Inductive,
  Lemma,
  Let,
  Local,
  Locate,
  Ltac,
  Maximal
  Module,
  Morphism,
  Next,
  Notation,
  Obligation,
  Open,
  Parameter,
  Parameters,
  Prenex,
  Print,
  Printing,
  Program,
  Projections,
  Proof,
  Proposition,
  Qed,
  Record,
  Relation,
  Remark,
  Require,
  Reserved,
  Resolve,
  Rewrite,
  Save,
  Scope,
  Search,
  Section,
  Show,
  Strict,
  Structure,
  Tactic,
  Theorem,
  Unset,
  Variable,
  Variables,
  View,
  inside,
  outside
},
morekeywords=[2]{
  as,
  cofix,
  else,
  end,
  exists,
  exists2,
  fix,
  for,
  forall,
  fun,
  if,
  in,
  is,
  let,
  match,
  nosimpl,
  of,
  return,
  struct,
  then,
  vfun,
  with
},
morekeywords=[3]{Type, Prop, Set, True, False},
morekeywords=[4]{
  apply,
  assert,
  auto,
  bool_congr,
  case,
  change,
  clear,
  compute,
  congr,
  cut,
  cutrewrite,
  destruct,
  elim,
  field,
  fold,
  generalize,
  have,
  heval, 
  hnf,
  induction,
  injection,
  intro,
  intros,
  intuition,
  inversion,
  left,
  loss,
  move,
  nat_congr,
  nat_norm,
  pattern,
  pose,
  refine,
  rename,
  replace,
  revert,
  rewrite,
  right,
  ring,
  set,
  simpl,
  split,
  suff,
  suffices,
  symmetry,
  transitivity,
  trivial,
  unfold,
  using,
  without,
  wlog,
  autorewrite
},        
morekeywords=[5]{
  assumption,
  by,
  contradiction,
  done,
  exact,
  lia,
  gappa,
  omega,
  reflexivity,
  romega,
  solve,
  tauto,
  discriminate,
  unsat
},
morecomment=[s]{(*}{*)},
morekeywords=[6]{do, last, first, try, idtac, repeat},
showstringspaces=false,
morestring=[b]",
% Size of tabulations
tabsize=3,							
% Enables ASCII chars 128 to 255
extendedchars=true,  		 		
% Case sensitivity
sensitive=true, 
% Automatic breaking of long lines
breaklines=false,
% Default style fors listings
basicstyle=\small\ttfamily,
% Position of captions is bottom
captionpos=b,							
% Full flexible columns 
columns=[l]fullflexible,
% Style for (listings') identifiers
identifierstyle={\color{black}},
% Style for declaration keywords
keywordstyle=[1]{\color{violet}},
% Style for gallina keywords
keywordstyle=[2]{\color{forestgreen}},
% Style for sorts keywords
keywordstyle=[3]{\color{forestgreen}},
% Style for tactics keywords
keywordstyle=[4]{\color{blue}},
% Style for terminators keywords
keywordstyle=[5]{\color{red}},
%Style for iterators
keywordstyle=[6]{\color{violet}},
% Style for strings
stringstyle=,
% Style for comments
commentstyle=\it\ttfamily\color{Bittersweet},
% Style for lines numbering
numberstyle=\tiny,
}

\lstdefinestyle{Coq}{language=Coq}
\lstset{style=Coq}


\begin{document}

\special{papersize=8.5in,11in}
\setlength{\pdfpageheight}{\paperheight}
\setlength{\pdfpagewidth}{\paperwidth}


\authorinfo{Ilya Sergey}
           {IMDEA Software Institute}
           {ilya.sergey@imdea.org}
%
\authorinfo{Aleksandar Nanevski}
           {IMDEA Software Institute}
           {aleks.nanevski@imdea.org}

\title{
  Introducing Functional Programmers\\
  to Interactive Theorem Proving and Program Verification
}

\subtitle{Teaching Experience Report\thanks{A part of this work has
    been carried out while the first author was giving lectures in a
    summer school on interactive theorem proving, which was sponsored
    by JetBrains and took place in August 2014 at Saint Petersburg
    State University.}}
\maketitle

\begin{abstract}

  We report on the design and preliminary evaluation of a short
  introductory course on interactive theorem proving and program
  verification using the Coq proof assistant, targeted at students
  with background in functional programming and software engineering.

  The course builds on concepts familiar from functional programming
  to develop understanding of logic and mechanized proving by means of
  the Curry-Howard isomorphism. A particular emphasis is made of the
  \emph{computational} nature of \emph{decidable} properties of
  various data structures. This approach is of practical importance,
  as Coq's normalization can automatically simplify or discharge such
  properties, thus reducing the burden of constructing the proofs by
  hand. As a basis for teaching this style of mechanization, we use
  Gonthier \etal's Ssreflect extension of Coq and its associated
  libraries.
  
  In the course, we minimize the exposure to ad-hoc proof automation
  via tactics, and request that students develop proofs using only a
  small set of proof-building primitives that they should clearly
  understand. In addition to introducing logic as an application of
  functional programming, the topics covered by the course include:
  implementation of custom rewriting principles as instances of
  indexed type families, boolean reflection, implementation of
  algebraic structures and inheritance between them, and verification
  of imperative programs in separation logic and Hoare Type Theory.

\end{abstract}

\section{Introduction}
\label{sec:intro}

The concept of a rigorous mathematical proof has a long story as an
educational topic. In our experience, however, a typical undergraduate
curriculum in computer science teaches proofs from the point of view
of logic, and ignores the fundamental connection between logic and
programming. Thus, most undergraduates in computer science receive a
strong background in programming, imperative as well as functional,
but will seldom relate their programming intuition to logic and formal
proofs.

This situation is unfortunate, as recent years have seen the
completion of a number of impressive and ground-breaking project in
the field of formal mathematics and program verification. Examples
include the four color theorem~\cite{Gonthier:AMS08}, the odd order
theorem~\cite{Gonthier-al:ITP13}, seL4 kernel~\cite{Klein-al:TCS14}
and CompCert compiler infrastructure~\cite{Leroy:POPL06}. These
advances attest that proof assistants have matured to the level at
which they are ready for use by the wider community in mathematics and
computer science.

Mechanized formalization of a large mathematical theory significantly
resembles development of a large programming project with a rich
infrastructure and a set of libraries. In this light, it is
unsurprising that an effective use of a proof assistant requires
remarkable programming skills. This insight is especially true of
proof assistants such as Coq~\cite{Coq-manual} or
Agda~\cite{Norell:PhD}, which include a powerful dependently typed
lambda calculus, in which one can write both programs as well as
proofs, via the Curry-Howard isomorphism.  However, the connection
between formal mathematics and programming goes much deeper than the
mere fact that proofs and programs can both be written in the lambda
calculus.
%
Indeed, even the way one defines the basic predicates of interest,
implements algebraic objects and structures such as groups or rings,
or organizes formal theories, bears a lot of similarity with
programming in the large. In particular, programming concepts such as
modules, functors, objects, abstract types and predicates,
higher-order functions, inductive and coinductive types and families,
overloading and inheritance, very quickly arise as natural mechanisms
for structuring formal mathematical developments.

Most fascinatingly---and, perhaps, confusingly to a newcomer with pure
mathematical background---there may be many different ways, employing
many different programming mechanisms, to formulate and mechanize the
same mathematical problem. Each choice may present a number of
tradeoffs that are not typically encountered in mere programming. For
instance, should one encode a particular property as a function or as
an inductive datatype? It is important to know the tradeoffs well;
good choices lead to reusable, maintainable and short proofs, just
like they would in good programs. But, conversely, suboptimal or
``hacky'' choices typically lead to an explosion in complexity, much
more quickly and severely than they do in programming.
%
In the light of this observation, one may say that effective
construction of formal proofs is not just a programming challenge, but
a programming challenge of the highest order, in which one's coding
skills, discipline and inventiveness can really shine, much more so
than in ordinary programming (\ie, in which proofs are not required).

This article describes an introductory course on interactive theorem
proving and verification using Coq, which we designed to embrace the
uniformity of the tradeoffs, that pertain to both programming and
proving. The considered issues appear frequently in our everyday
formalization work, but because they have not been collected and
emphasized in the existing introductory literature (rather, they are
interspersed across research papers of many
authors~\cite{Garillot:PhD,Mahboubi-Tassi:ITP13,Garillot-al:TPHOL09,Gonthier-al:JFP13,Gonthier-al:TR,Paulin-Mohring:TLCA93}),
we have frequently needed to devote time to introduce them to new
students and collaborators (all with significant experience with ML
and Haskell), individually. The idea for this course and its selection
of topics arose out of such informal teaching experience. In a more
formal setting, we report on the proof-of-concept evaluation of the
course, which was delivered by the first author in a five-days summer
school for a small group of students of Saint Petersburg State
University (SPbSU), majoring in mathematics and software engineering
and having necessary background in Haskell, thanks to the standard
curriculum courses.

By necessity imposed by space limits, our presentation in this report
assumes basic familiarity with Coq, though, of course, such
familiarity is not assumed in the course itself, the associated
lecture notes, slides, and code templates for hands-on
classes~\cite{Sergey:PnP}. However, we do not assume that the reader
will be able to follow the ocasional Coq proofs we present. In such
cases, we discuss the main ideas of the proof in prose. In the rest of
the article, we present the broader outline of the course, focusing in
more depth on a few specific topics that are not usually covered by
the existing literature and courses on Coq.

\section{Overview}
\label{sec:overview}

Currently, there are several books by different authors on interactive
theorem proving in
Coq~\cite{Bertot-Casteran:BOOK,Pierce-al:SF,Chlipala:BOOK} (see
Section~\ref{sec:related} for the discussion).
%
Although all these books have been successfully used in numerous
introductory courses on Coq, we thought that there are still some
topics, stemming from the proving-as-programming intuition and
essential for effective and boilerplate-free mathematical reasoning
via a proof assistant, that are left underrepresented. Our course is
targeted to fill these gaps while still covering the common teaching
material required to get the students off the ground.
%
In particular, in our course we emphasized the following aspects of
proof engineering, most of which are enabled or empowered by Gonthier
\etal's \emph{small-scale reflection} extension (Ssreflect) to
Coq~\cite{Gonthier-al:TR}:

\vspace{5pt}

\begin{itemize}

\item Special treatment is given to the \emph{computational} nature of
  reasoning about \emph{decidable} propositions. In other words, many
  results about decidable properties can be \emph{computed} as boolean
  values, rather than \emph{proven} interactively. But to do so, one
  has to formulate the properties as computable recursive Coq
  functions with \code{bool} result, rather than as inductive
  predicates. The latter is more in the spirit of the traditional
  introductions to Coq.

\item Instead of supplying the students with a large vocabulary of
  automated tactics necessary for everyday Coq hacking, we focus on a
  small but expressive set of primitives (about six in total), offered
  by Ssreflect or inherited from the vanilla Coq with notable
  enhancements.

\item Reasoning by rewriting is presented from the perspective of
  Coq's definition of the propositional equality and followed by
  elaboration on the idea of using \emph{index type families} as a
  tool to define client-specific conditional rewrite rules. The usual
  way indexed type families are presented in the related work is for
  use in programming and pattern matching, rather than rewriting.
  Conversely, we advocate parametrized (but not indexed) types, for
  programming and pattern matching.

\item We provide a detailed explanation of the essentials of
  Ssreflect's \emph{boolean reflection} between the sort \code{Prop}
  and the datatype \code{bool} as a particular case of conditional
  rewriting, following the computational approach to proving decidable
  properties.

\item Formal encoding of familiar mathematical structures (\eg,
  monoids, partial orders, \etc) is carried out by means of
  \emph{dependent records}; mathematical operations are overloaded
  using the mechanism of \emph{canonical instances}, similar to
  Haskell type classes. This analogy illustrates how the programming
  ideas of overloading and inheritance come into formal
  proving~\cite{Garillot-al:TPHOL09}.
%
  % \an{maybe we should cite some people here to give them credit for
  %   the ideas.}

\item A novel (from a teaching perspective) case study is considered,
  introducing the students to the concepts of stateful program
  verification using separation logic and Hoare Type
  Theory~\cite{Nanevski-al:JFP08}.

\end{itemize}

\vspace{5pt}

\subsection{Why teach with Ssreflect?}
\label{sec:why-ssreflect}

A significant part of our material is presented using the Ssreflect
extension of Coq~\cite{Gonthier-al:TR}. Ssreflect is developed as a
part of the Mathematical Components
project,\footnote{\scriptsize{\url{http://www.msr-inria.fr/projects/mathematical-components-2/}}}
to facilitate automated reasoning and simplification in very large
mathematical developments, in particular, the formalizations of the
four color theorem~\cite{Gonthier:AMS08} and the Feit-Thompson (odd
order) theorem~\cite{Gonthier-al:ITP13}.

Ssreflect includes a small but expressive (and comfortable for
practical work) set of primitives for proof construction, related to
but different from the traditional set provided by Coq. It also comes
with a large library of algebraic structures, ranging from natural
numbers to graphs, finite sets and algebras, formalized and shipped
with exhaustive toolkits of lemmas and facts about them. Finally,
Ssreflect introduces some mild modifications to Coq's native syntax
and the semantics of the proof script interpreter, which makes the
proofs very concise.

Using Ssreflect for our development was not the goal by itself: a
large part of the course could be presented using traditional Coq
without any loss in the insights but, perhaps, some loss in brevity.
%
However, having been developed as part of a very large formalization
effort, Ssreflect's libraries are well-tested in the wild, and are an
excellent example of good mechanization practice. In particular, they
make it very easy for us to emphasize the above-listed mechanization
aspects of the course. In fact, to the best of our knowledge, it was
the work on Ssreflect and the Four color
theorem~\cite{Gonthier:AMS08,Garillot-al:TPHOL09,Garillot:PhD} that
first applied boolean reflection and canonical structures in Coq to a
larger-scale formalization effort.

Last, but not least, Ssreflect comes with a much improved
\code{Search} tool, compared to standard Coq. The \code{Search} tool
is invaluable, given that a fair part of the time spent on
mechanization is typically devoted to reading third-party code and
lemma libraries.

\section{Structure of the course}

% In a nutshell, we begin by explaining how simple functional programs
% over basic datatypes can be defined and executed in the programming
% environment of Coq. We then proceed to the definition of propositional
% logic connectives and elements of interactive proof construction, by
% insisting on the Curry-Howard analogy with functional programming.
% %
% Building further on the common intuition about ``truth
% tables'', 
% the course introduces propositional equality and the way to encode
% custom rewrite rules by means of indexed type families. The idea
% naturally extends to the discussion of boolean reflection and
% reasoning by computation.  We proceed to introduce 
% important principles of proof by induction, and provide a brief
% overview of standard Ssreflect libraries about basic datatypes such as
% natural numbers, booleans, equality types, \etc. These datatypes are
% very basic and broadly used, so the familiarity with the main
% libraries of lemmas about them is essential. Finally, all the
% introduced concepts and reasoning principles are put together and
% employed in a larger hands-on study of verification of imperative
% programs in Hoare Type Theory~\cite{Nanevski-al:POPL10}.

In this section we provide a detailed description of the structure of
the course and refer the interested reader to the full syllabus and
the accompanying exercises available online~\cite{Sergey:PnP}.

\subsection{From functional programs to propositional logic}

Assuming that the students are knowledgeable about the basic concepts
of functional programming, such as algebraic datatypes, pattern
matching and possibly higher-order functions, we begin the course by
demonstrating the syntax for the same concepts in Coq, focusing solely
on Coq's programming language component.

\subsubsection{Functional programming in Coq}
\label{sec:programming-coq}

We start by presenting simple datatypes, such as \code{unit},
\code{bool}, \code{nat} and \code{empty} (\ie, the type without
constructors), proceeding shortly after to the definitions of
functions on these datatypes. For example, addition on \code{nat} is
defined by the program below, which introduces the students to Coq's
syntax for pattern matching and its abbreviated Ssreflect variant
\code{if-is}. The latter is useful for cases with only one non-default
branch. \code{_.+1} is Ssreflect notation for successor.

\begin{lstlisting}
Fixpoint my_plus n m := 
  if n is n'.+1 then (my_plus n' m).+1 else m.
\end{lstlisting}

Next, we focus on the induction and recursion principles, generated by
Coq automatically for each inductive definition. We explain these as
higher-order functions, whose return types can depend on the value of
the argument; hence, this gives us a way to introduce \emph{dependent
  function types} as well. We show how the recursive function
\code{my_plus} can be rewritten explicitly in terms of the recursion
principles, and suggest a number of exercises intended to expose the
students to working with higher-order dependently-typed
functions. This lecture culminates with a short demonstration of
custom dependently-typed functions, \eg, a function returning
\code{unit} on some inputs and \code{nat} on some others.

Elaborating futher on the types of generated recursion and induction
principles, we draw the students' attention to the recursion principle
generated for the datatype \code{empty}:

\begin{lstlisting}
empty_rect : forall (P : empty -> Type) (e : empty), P e
\end{lstlisting}

A function with such a type allows one to obtain instance of
\emph{any} type, given an argument \code{e} of type \code{empty},
thanks to the argument \code{P}, which can be instantiated by a
constant function returning a necessary type. A closer examination of
the body of \code{empty_rect} by means of \code{Print} machinery
leaves no doubts that it is well-typed. Therefore, an important
conclusion follows: \emph{assuming existence of a value that cannot be
  constructed, we are able to construct anything}. We next show that
\code{empty} is not the only datatype that can have this ``magical''
property of allowing to construct a value of any type from it. In
particular, we refer to the example with a type \code{strange}, taken
from~\cite{Bertot-Casteran:BOOK}, which is also \emph{non-inhabited},
even though it has a constructor:
%
\begin{lstlisting}
Inductive strange := cs of strange.
\end{lstlisting}
%
%
% \an{What's the definition of \code{strange}? Maybe show it here? I'm
% kinda
% curious.}
%
While the students don't have all that much experience with the
\code{empty} type, as it is not frequently encountered in programming,
we point out that it will correspond to the empty set in mathematics,
and to the false proposition in logic, and will thus have a
significant role to play very soon in the course.

We continue by showing a number of familiar algebraic datatypes, such
as pairs, sums, lists and trees as well as functions operating on
these, suggesting a number of simple programming exercises. After
successfully accomplishing a few of them, the students already feel
relatively comfortable with Coq as a programming language and realize
its main ``limitation''. In particular, they notice they can't define
generally-recursive functions in the usual programming style, and are
forced to rely on primitive recursion in order to convince Coq's type
checker that their functions terminate.

\subsubsection{Searching and structuring libraries}
\label{sec:search-defin}

In practical programming, it is important to be able to search for
appropriate procedures in someone else's libraries. The same holds for
interactive proof development, where one needs to search for
definitions and lemmas. When programming in Haskell, one may use
search engines such as
\emph{Hoogle}\footnote{\footnotesize{\url{http://www.haskell.org/hoogle/}}}
and
\emph{Hayoo!}.\footnote{\footnotesize{\url{http://hayoo.fh-wedel.de/}}}
This functionality is provided in Coq by the \code{Search} command,
which is further enhanced by Ssreflect (see Chapter~10
of~\cite{Gonthier-al:TR}). We decided to introduce the students to
\code{Search} very early on. Even though by this moment in the course,
they are not familiar with formal logic and proofs, they can already
use \code{Search} to look for definitions and functions from the
programming side, based on their types. For example, they can execute
queries like the following one
%
%\vspace{-5pt}
%
\begin{lstlisting}
Search _ ((?X -> ?Y) -> _ ?X -> _ ?Y).
\end{lstlisting}
%
%\vspace{-5pt}
%
which returns all \code{map}-like functions currently available in
the imported libraries.  

Coq's extensible parser is a powerful tool when defining custom
notations for mathematical theories. From our experience it is also a
source of constant frustration for the Coq newcomers, who struggle to
figure out whether some notation is defined in a library, or is Coq's
native syntax. To avoid the pain of working with custom third-party
notations, we teach them to use commands like \code{Locate} and
\code{Print}. We also show how to ``switch off'' all the syntactic
sugar from custom notations with the \code{Set Printing All}
command. We also show how to define one's own custom notation, with a
warning that the feature should be used carefully, as it is easy to
abuse and make one's code unreadable.

At this point we also introduce the students to basic program
structuring primitives of Coq, such as sections and modules. We
explain the difference between the two (\eg, modules provide
hierarchical name spaces, while sections declare local variables that
generalize over the section body), and illustrate how to combine the
two to achieve a degree of ``locality'' for definitions and hiding of
the internal details.
%
% \an{Hmm, this is not a very clear paragraph, as it's not clear what is
%   ``desired level of encapsulation''. Why not revise as follows. After
%   ``we explain the difference between the two'', just say that we
%   illustrate by simple examples what can be done by one feature, and
%   what by the other. Do we want to go any deeper into this, and
%   explain in more detail what the differences are? I can imagine that
%   the Pierce-style courses probably mess this point up, just like they
%   mess many other things.}
% %
% \is{I tried to clarify what I meant by encapsulation. How does it read
%   now?}

\subsubsection{Introducing logical connectives}
\label{sec:intr-logic-conn}

By this moment, the students are familiar with the basic principles of
programming in Coq. In particular, they understand that values of
inductive datatypes are constructed by applying constructors to the
arguments and ``destructed'' by pattern matching. They know how to
define functions of appropriate types, and that a function should be
applied to get the value of its result type. They have even seen
dependently-typed functions whose return type can vary depending on
the value of the arguments. In other words, everything is set up for
the introduction of constructive logic by means of the Curry-Howard
analogy.

First of all, we define when a logical statement will be \emph{true}
for us: when its proof \emph{can be built} from hypotheses and rules
by referring to the hypotheses and applying the rules,
correspondingly. This definition has two important implications, which
we elaborate upon: (1) this notion of truth is \emph{constructive};
something is only true if its proof can be constructed in a finite
number of steps, and (2) the truth is \emph{relative} with respect to
the initial hypotheses and the rules of the logic. In our experience,
this definition most often matches very well the intuition that the
students have from their classical mathematical education. Even if
they haven't been exposed to extensive study of mathematical logic
and, hence, have only a very informal idea of what a formal proof is,
they know how to write informal ones, and how to find flaws in
them. The definition also makes it easy to introduce \emph{falsehood}
as a proposition whose proof cannot be constructed.

%\an{Hehe, this paragraph sounds way too deep for its own
%  good. I mean, when I hear someone talking about ``the notion of
%  truth'', I pull out the gun. Most of the time in such cases I can't
%  quite figure out what the interlocutor means. More constructively, I
%  think that the above paragraph should be made much more concrete and
%  down-to-earth. Why not say the following. (1) We introduce basic
%  logical propisitions as simple datatypes as in functional
%  programming. (2) Then we show conjunction corresponds to pairing,
%  disjunction to union, falsehood to empty type, truth to
%  unit. Finally, (3) implication corresponds to function types, and
%  proofs of implications to functions. Aside: how did the students
%  react when you introduced them to this? Did they not know this
%  already?}

Next, we appeal to the analogy between the falsehood \code{False} and
the datatype \code{empty}, defined previously, followed by the
introduction of standard logical connectives by the Curry-Howard
correspondence: conjunction corresponds to the product datatype
\code{prod}, disjunction to \code{sum}, truth to a type with a single
element (in Coq, the element is named \code{I}), implication to
function type \code{A -> B}, universal quantification to dependent
function types, and existential quantification to a specific dependent
record type. This intuition gradually leads to the idea that proving a
logical proposition corresponds to building a program that has the
proposition as its type. In other words, we're trying to
\emph{inhabit} a type with a \emph{proof term} element.\footnote{While
  drawing the analogy of types/propositions as sets of proofs is
  useful for building intuition, we find it important to emphasize
  that, unlike sets in set theory, types in Coq are always
  \emph{disjoint}; that is, an element cannot belong to more than one
  type.}

Given this relation, it is natural to explain the usual inference rules
in terms of functional programming.
%
In particular, a function can be introduced by \emph{assuming} its
argument and constructing its body, in which the argument is used,
which corresponds to the rules of implication introduction. The
similar intuition holds for universal quantification. The introduction
rules for other logical connectives correspond to application of a
\emph{constructor} of the corresponding logical connective (\eg,
\code{conj} in the case of conjunction, and similarly for existential
quantification). As for elimination rules, most of the connectives
(conjunction, disjunction, existentials) clearly correspond to
case-analysis by means of Coq's \code{match-with} expression
construct. 
%
% \an{conjunction elimination is really by means of fst and snd
%   projections.}
% %
% \is{... which are implemented using pattern matching, so the point
%   stays.}
%
 The notable exceptions are the \emph{modus ponens}
rule for implication, and the \code{forall}-specialization rule, which
both correspond to function application.

\subsubsection{Interactive proof construction: first encounter}
\label{sec:inter-proof-constr}

% Once we have presented the analogy, which relates introduction rules
% (\ie, the rules that allow one to construct a logical connective,
% which is a \emph{goal} to be proved) and constructors/assumptions as
% well as elimination rules (\ie, the rules allowing one to \emph{make
% use} of the assumptions by means of case-analysing on them or
% applying them) and pattern matching/function applications,
%
After concepts related to proofs have been introduced, we can proceed
to interactive proof development. We point out that even though proofs
are the same as programs, it's frequently easier to develop them not
as explicit lambda-terms, but in a step-by-step manner using proof
scripting with a small (but expressive) set of Ssreflect's primitives.
%
% \an{I added the point about doing scriptintg. I expected
%   that students would have been puzzled otherwise as to why not just
%   write programs? I.e., why are we moving to scripting instead of
%   sticking with functional programming? Should we elaborate even
%   more?}

First, we show the simplest possible proof script, which just appeals
to a known fact whose proof is constructed explicitly as a
lambda-term.  This construction is done by the \code{exact} primitive,
as in the following proof of \code{True}, whose explicit proof is
called \code{I}:
%
\begin{lstlisting}
Theorem true_is_true : True.
Proof. exact: I. Qed.
\end{lstlisting}
%
Next, we demonstrate the machinery enabled by Ssreflect's \code{move}
tactic placeholder and two \emph{bookkeeping} tacticals, \code{=>} and
\code{:}. These allow one to work with \emph{assumptions}, moving them
``bottom-up" (\ie, from the goal to the context) and vice versa,
respectively:
%
\begin{lstlisting}
Theorem imp_trans (P Q R: Prop) : 
         (P -> Q) -> (Q -> R) -> P -> R.
Proof. move => H1 H2 p; exact: (H2 (H1 p)). Qed.
\end{lstlisting}
%
We emphasize that the proof by \code{exact} is and instance of
so-called \emph{forward} reasoning style, in which assumptions are
combined to eventually obtain the proof of the goal. In the above
proof, \code{H1}, \code{H2}, \code{p} are proofs of \code{P->Q},
\code{Q->R}, and \code{p}, respectively, and the proof of \code{R} is
obtained as an application \code{H2 (H1 p)}. We also exhibit the
opposite---\emph{backwards}---reasoning style, which replaces the goal
by a number of obligations arising from the types of the applied
function/hypothesis' arguments. In Ssreflect, backwards reasoning is
implemented by the \code{apply:} tactic,\footnote{Which is somewhat
  more powerful than Coq's native \code{apply} (without ``:'').} the
use of which we immediately demonstrate on a number of simple
examples:

%\newpage

\begin{lstlisting}
Theorem all_imp_dist A (P Q: A -> Prop) : 
  (forall x: A, P x -> Q x) -> (forall y, P y) -> 
  forall z, Q z. 
Proof. move => H1 H2 z; apply: H1; apply: H2. Qed.
\end{lstlisting}
%
We elaborate that constructing proofs by means of \code{apply:}
essentially corresponds to reading logical introduction rules
``bottom-up''. In this way, constructors of logical connectives are
essentially equated with hypotheses in context.

At this point, students are equipped with three main proof building
primitives, and can already construct simple proofs. The main missing
component is the primitive for case analysis (\ie, pattern-matching),
which we introduce next. As most of the Ssreflect tactics, \code{case}
can be effectively combined with bookkeeping tacticals, which we make
use of immediately.\footnote{The necessary syntax for working with
  multiple subgoals, \eg, the \code{[||]} tactical, is introduced on
  the way by gradually ``compressing'' the proofs in front of the
  students during the hands-on recitations.}
%
\begin{lstlisting}
Theorem conj_comm P Q : P /\ Q -> Q /\ P.
Proof. case=> p q; apply: conj; [exact: q|exact: p]. Qed.
\end{lstlisting}
%
If no specific argument is provided, the tactics \code{move} and
\code{case} by default always work with the \emph{leftmost} assumption
of the goal.

The four primitives introduced by now: \code{exact}, \code{move},
\code{case} and \code{apply}, together with Ssreflect's bookkeeping
machinery, are sufficient for building proofs of arbitrary logical
propositions.
%
% To provide students with the oportunity for practice that should
% familiarize them with the effects that each of the primitives has on
% the goal and the context, we suggested number increasingly complex
% exercises (\eg, proofs of statements involving the falsehood
% \code{False} and reasoning out of contradiction).  
%
% \an{The above paragraph seems unfocused. What's the point of saying
%   that we let them reason out of False. It's not the most intriguing
%   kind of assignments. I tried to make the paragraph more concrete,
%   but I didn't succeed. Let's just erase it.}
%
%\is{Ok, let's remove it.}
%
We next explain a number of ``shortcut'' tactics that simplify the
work with particular logical connectives and predicates. Examples
include: \code{split}, \code{exists}, \code{left}, \code{right}, \etc.
While explaining the shortcut tactics, we ask the students to express
them through the four basic ones. We also explain Ssreflect's
\emph{terminators} \code{by} and \code{done}, which raise an error
message if they fail to discharge the goal. Finally, we illustrate the
tactic \code{intuition} on a few examples. This tactic attempts to
prove propositions in intuitionistic logic automatically. In this case
too, we ask the students to prove the same examples without
automation, using only the four basic primitives.

We conclude this part of the course by providing a brief overview of
non-constructive axioms from classical propositional logic, such as
the \emph{law of excluded middle}, \emph{Peirce's law}, the \emph{law
  of double negation}. Students are asked to to prove that these are
all equivalent, following an excercise by Bertot and
Cast\'{e}ran~\cite{Bertot-Casteran:BOOK}. We also briefly elaborate on
the impredicativity of the sort \code{Prop} of propositions in Coq,
and provide a short explanation of the basics of Coq's hierarchy of
universes.

\subsection{Equality and rewriting}
\label{sec:rewriting}

After explaining the basics of propositional logic, we proceed to
address equational reasoning and the corresponding proof technique of
\emph{rewriting}. Propositional equality is defined in Coq by the
following \emph{indexed type family} (in this case, the index is the
element of type \code{A} in the type signature \code{A -> Prop}):
%
\begin{lstlisting}
Inductive eq (A : Type) (x : A) : A -> Prop :=  
   eq_refl : eq x x.
Notation "x = y" := (eq x y) (at level 70).
\end{lstlisting}
%
A common way of understanding indexed type families such as the one
above is to compare them to \emph{generalized algebraic datatypes}
(GADTs) in Haskell~\cite{PeytonJones-al:ICFP06,Xi-al:POPL03}. The
latter allow the programmer to refine and guide the process of
(dependent) pattern matching by the type index of the
scrutinee. Studying indexed type families from this point of view is a
very active research
area~\cite{Morris-al:FCS09,Dagand-McBride:ICFP12}.
%
However, not wanting to get bogged down with too many details of
dependent pattern matching, we decided to explain indexed type
families by another well-known analogy~\cite{Paulin-Mohring:TLCA93},
%
%\is{Where exactly does it occur in the litrerature?}
%
specifically useful in Coq: indexed type families correspond to
\emph{custom} (\ie, user-defined) \emph{simultaneous conditional
  rewrite rules}.

The intuition about this is best built by examples. For instance, the
proof of the equality's symmetry will look as follows:\footnote{In
  fact, the case-analysis on Coq's predicate family \code{eq} is
  overloaded by Ssreflect, so this example is presented in the course
  using an equivalent, user-defined, equality predicate family
  \code{my_eq}.}

%\newpage

\begin{lstlisting}
Lemma eq_sym A (x y: A) : x = y -> y = x.
Proof. by case. Qed.
\end{lstlisting}

The \code{case}-analysis on the assumption \code{x = y} essentially
forces the unification mechanism of Coq to exploit the ``implicit
equality'', encoded via the \code{eq}'s index (\ie, the second
argument), and \emph{substitute} all occurrences of the index's actual
value by the value declared in the constructor \code{eq_refl} first
argument. In other words, case-analysing the assumption rewrites the
occurrence of \code{y} by \code{x}, therefore leading to a subgoal
\code{x = x} which trivially discharged by the terminator
\code{by}. In a more pictorial analogy, one can think of the predicate
\code{eq x y} as of the following ``rewriting'' table:

\vspace{5pt}

\begin{center}
\hspace{-20pt}
  \begin{tabular}{r|r|c|}
    \cline{3-3}
    \multicolumn{2}{c|}{} & \code{y}
    \\
    \cline{2-3} 
    \multicolumn{1}{r}{\code{eq_refl}} &\multicolumn{1}{|c||}{\code{x}} & \code{x}
    \\    \cline{2-3}
  \end{tabular}
\end{center}

\vspace{5pt}

In English, for the fixed value \code{x} (which is bound by the
datatype's parameter), the instance of \code{eq x y} defines an
``rewriting table'', whose top-right cell specifies \emph{what} should
be replaced (in this case, this is \code{y}, which is captured by the
datatype's index). The bottom-right cell, thus, specifies what the
actual value of the index, \code{y}, should be substituted
\emph{with}. Hence, case-analysis on such a rewriting table will
replace all the occurrences of the columns' ``headers'' (values of
indices in the actual instance) by the corresponding values in the
``cells'' (values of indices in the constructors). In this analogy,
the ``rows'' of the table correspond to particular constructors and
their arguments, and in the case of \code{eq} there is just one,
\code{eq_refl}, with a single argument \code{x}.

This intuition is further strenghtened by more involved examples from
the Ssreflect's library of natural numbers (\code{ssrnat}). In
particular, we consider the following indexed family \code{nat_rels}
and the companion lemma \code{ltngtP}:

\begin{lstlisting}
Inductive nat_rels m n : bool -> bool -> bool -> Set :=
| CompareNatLt of m < n : nat_rels m n true false false
| CompareNatGt of m > n : nat_rels m n false true false
| CompareNatEq of m = n : nat_rels m n false false true.

Lemma ltngtP m n: nat_rels m n (m < n) (n < m) (m == n).
\end{lstlisting}

\noindent
The parameters of \code{nat_rels} are the nats \code{m} and
\code{n}. The definition also has three indexes, all boolean.
Together, \code{nat_rels} and \code{ltngtP} encode the following
rewriting table.

\vspace{5pt}

\begin{center}
\hspace{-20pt}
  \begin{tabular}{r|r|c|c|c|}
    \cline{3-5}
    \multicolumn{2}{c|}{} & \code{m < n} & \code{n < m} & \code{m == n}
    \\
    \cline{2-5}
    \multicolumn{1}{r}{\code{CompareNatLt}}& \multicolumn{1}{|c||}{\code{m < n}} & \code{true} & \code{false} & \code{false}
    \\
    \cline{2-5}
    \multicolumn{1}{r}{\code{CompareNatGt}}& \multicolumn{1}{|c||}{\code{m > n}} & \code{false} & \code{true} & \code{false}
    \\
    \cline{2-5}
    \multicolumn{1}{r}{\code{CompareNatEq}}& \multicolumn{1}{|c||}{\code{m = n}} & \code{false} & \code{false} & \code{true}
    \\
    \cline{2-5}
  \end{tabular}
\end{center}

\vspace{5pt}
\noindent
The column headers specify the subexpression in the goal to be
substituted by case analysing on a proof of \code{ltngtP} (\ie,
\emph{what} is substituted). In this particular case, we want to
simultanously substitute for three different boolean expressions:
\code{m < n}, \code{n < m} and \code{m == n}.\footnote{The ordering
  relations and the equality on natural numbers are implemented in
  Ssreflect as boolean functions, since they are decidable.}  The row
headers specify when (\ie, \emph{under which conditions}) the
substitution takes place. The value identified by a given row and
column says what we substitute \emph{with}.  More concretely, case
analysing on a proof of \code{ltngtP} will make Coq substitute the
occurrences of \code{m == n} in a goal as follows (and similarly for
\code{m < n} and \code{n < m}). It will generate three subgoals, in
which \code{m == n} is replaced with \code{false}, \code{false}, and
\code{true}, respectively. The subgoals will contain hypotheses as
read off from the row headers; the first subgoal will be conditioned
on \code{m < n}, the second on \code{m > n}, and the third on \code{m
  = n}. One can employ such simultaneous conditional rewriting to
carry out very short and effective proofs. For example, the following
proofs about \code{minn} and \code{maxn} functions are taken from the
\code{ssrnat} library:
%
\begin{lstlisting}
Definition maxn m n := if m < n then n else m.
Definition minn m n := if m < n then m else n.

Lemma addn_min_max m n : minn m n + maxn m n = m + n.
Proof.
by rewrite /minn /maxn; case: ltngtP=>//; rewrite addnC.
Qed.
\end{lstlisting}
%
In the proof of \code{addn_min_max}, right after \emph{unfolding} the
definitions of \code{minn} and \code{maxn} by the rewrite command, the
case-analysis on the table constructed by \code{ltngtP} with
\emph{simultaneous} rewritings and reducing \code{if}-expressions,
produces a number of subgoals, most of which are discharged by
Ssreflect's trailing tactical \code{//}, which can be expressed in
somewhat standard Coq as \code{try done} for each of the subgoals.
%
The observation about simultaneity of rewriting with respect to
\code{m < n}, \code{n < m} and \code{m == n} is of essence. Had we
tried instead to build the proof of \code{addn_min_max} based on the
case-analysis over a lemma stating the totality of
\code{<},\footnote{This is the lemma \code{forall m n, m < n \\/ m == n \\/ m > n}.} we would first have to prove a
series of auxiliary properties, such as that \code{n < m} is
\code{false} when \code{m < n} is \code{true}, and vice versa. The
latter increases the proof burden. The simultaneous rewriting via the
\code{ltngtP} lemma ``packages'' these facts together in the truth
table \code{nat_rels}.
%
% \an{This is a bit too fast. Should we show the proof state after the
%   case? Also, should we say that we asked students to carry out the
%   same proofs without using ltngtP, but with using a lemma about
%   totality, which should be more cumbersome. We may even show that
%   other proofs here in the paper, for a comparison. Does that make
%   sense?}
% %
% \is{I don't mind showing an alternative proof, but I'm not sure I
%   understand what kind of a proof do you want to demonstrate? The one
%   by induction on \code{m}/\code{n}?}

While case analysis on lemmas over indexed families logically
corresponds to rewriting, the above example readily illustrates that a
separate \code{rewrite} primitive is also very useful, in particular
with a number of enhancements provided by Ssreflect. The usage
scenarious of a separate \code{rewrite} primitive include
folding/unfolding definitions (by means of the \code{/} modality in
Ssreflect), rewriting in assumptions and sub-expressions, and using
occurrence switches to select specific subexpressions for rewriting
(see Chapter~7 of~\cite{Gonthier-al:TR}).
%

In the excercises for this module, we ask the students to practice
with rewrite tactics on a number of examples involving natural
numbers. The selection of exercises also familiarizes them with the
\code{ssrnat} library, where most of the lemmas take the form of
equality. One such lemma is \code{addnC} employed in the proof above,
which states commutativity of addition. The students also get to
practice with the \code{Search} tool to find lemmas that encode the
familiar properties of natural numbers such as associativity of
addition, neutrality of zero, \etc.

\subsection{Boolean reflection and views}
\label{sec:views}
%
% \an{But views are not really discussed}

One confusing point for students with programming background, when
they are exposed to Coq, is the difference between the two-constructor
enumeration type \code{bool} and Coq's sort of propositions
\code{Prop}. At this point in the course, we explain the difference,
but in a way that illustrates the trade-offs between the two concepts,
when it comes to building interactive proofs. Following Ssreflect, we
suggest that one should use boolean expressions to encode
\emph{decidable} properties, whereas the propositions from \code{Prop}
should be used for undecidable properties, or properties whose
decidability is not trivial to establish. Consequently, one obvious
difference that can be immediately pointed out is that \code{Prop}
allows quantification over infinite domains, while \code{bool}
doesn't. But more importantly, \code{bool} and \code{Prop} can be
related by an indexed family (i.e, a ``rewriting table'', as
introduced in the previous section). This idea is at the core of
Ssreflect and, if used properly, often leads to very effective and
compact proofs.

\subsubsection{Coercing \code{bool} into \code{Prop}}
\label{sec:coerc-codeb-into}
We start by pointing out that booleans can be naturally injected into
propositions by equating their value to \code{true}, using
propositional equality defined in the previous section. This injection
is done in Ssreflect by the implicit \emph{coercion}, defined in the
\code{ssrbool} library:
%
\begin{lstlisting}
Coercion is_true (b: bool) : Prop := b = true
\end{lstlisting}

We explain this coercion as an implicit type conversion, familiar to
students from languages like Scala or Haskell, which Coq inserts
automatically every time it ``expects to see'' a proposition, but
instead encounters a boolean. We next show that, given a suitably
formulated boolean expressions, reasoning with booleans can be
surprisingly nifty, as demonstrated by the following example,
involving Ssreflect's bool-returning function \code{prime}, which
implements Erathostenes' sieve:
%
\begin{lstlisting}
Goal prime (16 + 13). Proof. done. Qed.
\end{lstlisting}
%
Indeed, the proof would have been much longer had one used the usual
definition of prime numbers as those with only trivial
divisors.\footnote{In fact, Ssreflect's library \code{prime} provides
  both definitions and proves them equivalent, so that the user can
  employ either one as needed.}
%
This example, and a number of similar ones, convey the idea that
decidable properties, if implemented as computable functions returning
a boolean result, allow Coq to perform automatic simplifications, and
even obtain results by computation, without imposing any proving
burden on the user.

\subsubsection{Introducing the \code{reflect} datatype}
\label{sec:reflect}

Coercing booleans into propositions is easy enough by an implicit
coercion, but going in the opposite direction is a somewhat more
complicated. Ssreflect suggests a general methodology for implementing
the equivalence between particular propositions and boolean
expressions by means of the following \code{reflect} indexed family:
%
\begin{lstlisting}
Inductive reflect (P : Prop) : bool -> Set :=
  | ReflectT of   P : reflect P true
  | ReflectF of ~ P : reflect P false.
\end{lstlisting}
%
Indeed, following the ``rewriting table'' analogy (defined in
\S~\ref{sec:rewriting}), one can think of \code{reflect} as a generic
rewrite rule, parametrized with respect to ``logically equivalent''
proposition \code{P} and a boolean~$\mathtt{b_P}$.

\vspace{5pt}

\begin{center}
\hspace{-20pt}
  \begin{tabular}{r|r|c|}
    \cline{3-3}
    \multicolumn{2}{c|}{} & $\mathtt{b_P}$
    \\
    \cline{2-3}
    \multicolumn{1}{r}{\code{ReflectT}} & \multicolumn{1}{|c||}{\code{P}} & \code{true} 
    \\
    \cline{2-3}
\multicolumn{1}{r}{\code{ReflectF}} & \multicolumn{1}{|c||}{{\small\texttt\textasciitilde}~\code{P}}  & \code{false} 
\\\cline{2-3}
  \end{tabular}
\end{center}
%
\vspace{5pt}
%
We illustrate the use of the \code{reflect} datatype by examples
involving reflection of logic connectives, \eg, conjunction
%
\begin{lstlisting}
Lemma andP (b1 b2 : bool) : reflect (b1 /\ b2) (b1 && b2).
\end{lstlisting}
%
The students are asked to prove a number of similar lemmas, \eg, for
disjunction, in order to master the principles of providing custom
instances of \code{reflect}.

We next explain the use of \emph{views} and \emph{view hints}, which
are part of Ssreflect's bookkeeping machinery (see Chapter~9
of~\cite{Gonthier-al:TR}). We omit the discussion here, but briefly
note for the reader unfamiliar with Ssreflect, that views are simply
\code{reflect} lemmas which can be employed to on-the-fly convert a
boolean expression into the equivalent proposition. For example, the
above lemma \code{andP} can be used as a view to transform \code{b1 &&
  b2} into a conjunction \code{b1 /\\ b2}. The latter form may be
preferable at times; for example, the conjunction can be destructed
into component conjuncts by the usual case analysis, whereas the
similar destruction is much more cumbersome in the case of \code{b1 &&
  b2}. Of course, the advantage of the \code{b1 && b2} form is that it
may often simplify by computation, for special values of \code{b1} and
\code{b2}, whereas those simplification won't be carried out
automatically by Coq in the case of the \code{b1 /\\ b2} form.  A
particularly useful view is the lemma \code{eqP} which relates boolean
and propositional equality, in the special case of \code{eqType}s,
\ie, types with decidable equality. We refer to such types as
\emph{equality types}, but postpone the definition of \code{eqType}
until one of the future lectures.
\begin{lstlisting}
Lemma eqP (A : eqType) (x y : A): reflect (x = y)(x == y).
\end{lstlisting}
We use the \code{eqP} lemma to illustrate the interoperatbility
between the two different equalities. For example, the boolean
equality \code{x == y} can be used in the scrutinee of conditionals,
as the following example shows, whereas the propositional one
cannot. However, in proofs, we may need to use \code{eqP} to switch
between the two forms.
%
\begin{lstlisting}
Definition foo (x y : nat) := if x == y then 1 else 0.
Goal forall x y, x = y -> foo x y = 1.
Proof. by rewrite /foo => x y /eqP ->. Qed.
\end{lstlisting}
%
To prove the goal, one first has to unfold \code{foo}.
%
% \an{We presumably don't have to unfold here, as the view will apply
%   anyway. We should try to minimize the proofs whenever possible.}
% %
% \is{I tried it before and this proof doesn't go through without the unfolding.}
%
Then the view \code{/eqP} is used to covert the proposition \code{x =
  y} into the boolean \code{x == y}, or more precisely, into \code{(x
  == y) = true}.  The later is subsequently used (by \code{->}), to
rewrite the condition in \code{if-then-else} to \code{true}. By
computation, this reduces the goal into \code{1 = 1}, which is then
trivially discharged.
%

\subsection{Proofs about inductive predicates and recursive functions}
\label{sec:induction}
The next course module is dedicated to proofs by induction on custom
datatypes and inductively-defined predicates.

Proof by induction in Ssreflect are usually done using the \code{elim}
tactic. It has a number of enhancements over the standard Coq
\code{induction}, and in particular, it keeps with the semantic of
Ssreflect, already introduced with \code{move} and \code{case}, that
the proof-scripting primitives by default always apply to the leftmost
assumption in the goal, initiating the proof by induction on it.
%
%\an{Was this mentioned before in the paper? I don't recall.}
%
% \is{No, it wasn't, but I couldn't find a better place to put this
%   remark in.}

We begin by pointing out that \code{elim} can be seen as a special
case of \code{apply}, where the lemma chosen for application is the
default induction principle generated automatically by Coq for the
datatype in question. For example, inducting on natural numbers
corresponds to applying \code{nat_ind} lemma, which expresses the
well-known Peano induction schema. Similarly, eliminating falsehood
(or, equivalently, the empty set) applies the lemma \code{empty_ind}.
%
% \an{Maybe better to say, eliminating empty, will apply the lemma
%   empty-rect. The point here is that False-ind is not presented in the
%   paper, so the non-expert reader will wonder what that is. Or maybe
%   just omit the sentence.}
%
%   \is {Fixed by mentioning empty_ind instead}

We illustrate the usual patterns of inductive reasoning by working out
a number of examples related to decidable properties of natural
numbers. Here, we focus on evenness. We tie to the previous lecture by
expressing the property in two ways: as a boolean expressions
\code{evenb}, and as an inductive proposition \code{evenP} in
\code{Prop}.\footnote{In the definition of \code{evenP}, we make the
  equalities such as \code{n = m.+2} \emph{explicit} in the
  constructor \code{EvenSS}. This can be avoided by switching to
  indexed families, but we reserve indexed families for implementing
  ``rewriting tables'' only, as shown in \S~\ref{sec:rewriting}.
%
  With explicit equalities, we can avoid the \code{inversion} tactic,
  whose behavior is not trival to explain.}
%
\begin{lstlisting}
Inductive evenP n : Prop :=
  Even0 of n = 0 | EvenSS m of n = m.+2 & evenP m.

Fixpoint evenb n := if n is n'.+2 then evenb n' else n==0.
\end{lstlisting}
%
We consider the proofs of following two logically equivalent
statements formulated using the two definitions given above.
%
%\an{Is this taken from CPDT? Do we need a citation?}
%
%\is{No, it's not precise and not that big to cite Adam for this.}
%
\begin{lstlisting}
Lemma evenP_contra n : evenP (n + 1 + n) -> False.
Proof.
elim: n=>[|n IH]; first by rewrite addn0 add0n; case.
rewrite addn1 addnS addnC !addnS. 
rewrite addnC addn1 addnS in IH.
by case=>// m /eqP; rewrite !eqSS => /eqP <-.
Qed.

Lemma evenb_contra n: evenb (n + 1 + n) -> False.
Proof. by elim: n=>[//|n IH]; rewrite addSn addnS. Qed.
\end{lstlisting}
%
We do not expect the reader to understand the two proofs, but it can
be noticed that the first proof is significantly more verbose and
requires a number of rewritings. Pleasantly, in the second proof, the
burden of rewritings is much smaller, thanks to the boolean and
computational nature of \code{evenb}, which automatically performs
simplification and partial evaluation in both the base and the
inductive case. Intuitively, in the second proof, the boolean
computation automatically discharges the base case (\code{//}), and
produces a residual subgoal for the inductive case which is almost
identical to the induction hypothesis. We only need a few simple
commutations of ``1'', performed by the lemmas \code{addSn} and
\code{addnS} from the \code{ssrnat} library, to finish the proof.

Sometimes, it may happen that a boolean property is less convenient
than propositional ones. In the case of \code{evenb}, the problem
arises because the property has an ``orbit'', which differs from the
constructors of the underlying datatype. In the case of evenness, the
orbit is 2 (\ie, if a number $n$ is even, then so is $n+2$), but
natural numbers are constructed by incrementing by $1$. This makes the
following lemma a bit more cumbersome to prove in the boolean case
than in the propositional one.
%
\begin{lstlisting}
Lemma even_add n m : evenb n -> evenb m -> evenb (n + m).
\end{lstlisting}
%
In the propositional case, it turns out one can use induction either
on the first or the second \emph{assumption}. Contrarily, in the
boolean version, it is only natural to induct on \code{n} and
\code{m}. Either way, the inductive case is decremented by 1, but to
exploit the orbit, we need to decrement by 2. In the course, we
illustrate several ways in which this proof can be carried out. One is
by generalizing the induction hypothesis in a reasonably simple way to
quantify over all numbers smaller than $n$ (or $m$), not just over the
immediate predecessor. But we also illustrate that one can build and
then use \emph{custom induction principles} with \code{elim}. In this
particular case, the following custom induction principle suffices.
%
\begin{lstlisting}
Lemma nat2_ind (P: nat -> Prop) : P 0 -> P 1 -> 
  (forall n, P n -> P n.+2) -> forall n, P n.
\end{lstlisting}
%
With this lemma, the proof of \code{even_add} reduces to a half-liner:
\code{by elim/nat2_ind : n.}

The module concludes by summarizing the observed patterns of inductive
definitions using predicates and recursive boolean functions, and
enumerating the common practices of proving in both styles. A number
of follow-up exercises in the hands-on recitations motivates the
students to master both techniques, and familiarizes them with the
basic Ssreflect libraries about the datatypes \code{nat}, \code{bool}
and functional lists.
% \an{Is this paragraph important enough to leave? It seems
% uninformative and obvious.}
%
% \is{I think it's better to leave}

\subsection{Programming with abstract algebraic structures}
\label{sec:depstruct}
At this point in the course, the students are sufficiently proficient
in proving statements in equational logic over natural numbers. We
next introduce them to building their own theories of algebraic
structures, familiar from the university classes on abstract algebra.

One can initiate the discussion of abstract algebraic structure with
an intuition from programming. Indeed, an abstract algebraic structure
is similar to the notion of class from object-oriented programming,
module from Standard ML, or type class from
Haskell~\cite{Wadler-Blott:POPL89}. In simply-typed languages, these
mechanisms allow the programmer to aggregate, \ie, \emph{package}
together, a number of operations over some datatype, potentially
abstracting over the datatype implementation. In a dependently-typed
language such as Coq, we can do a bit more: we can package the
operations together with \emph{proofs} of their important properties,
such as commutativity or associativity, to obtain an abstract
structure.

\subsubsection{Defining abstract structures}
\label{sec:defin-abstr-struct}

%Drawing further from the intuition supplied by Haskell's type classes
%and record types from imperative programming languages, we provide a
%gentle introduction into a possible way to build algebraic theories in
%Coq. 
As a running example, we consider partial commutative monoids (PCMs);
an algebraic structure which recurrs in our current ongoing work on
the verification of stateful and concurrent
programs~\cite{Nanevski-al:ESOP14}. We implement PCMs using two of the
Coq's native constructs: \emph{dependent records} and \emph{canonical
  structures}.  We follow the estasblished Ssreflect design pattern of
defining algebraic data structures by means of \emph{mix-in}
composition~\cite{Garillot:PhD}, whereby different dependent records
formalize different algebraic properties, which can be combined using
\emph{packed classes} mechanism. The latter also defines the field
resolution strategy~\cite{Garillot-al:TPHOL09} in a case of
overlapping names.
%
% \an{Hmm, what is field-resolution strategy?}
%
%\is{I clarified}
%
For instance, the \emph{mix-in} defining PCMs is represented by the
following dependent record:
%
\begin{lstlisting}
Record mixin_of (T : Type) := Mixin {
  valid_op : T -> bool;
  join_op : T -> T -> T;
  unit_op : T;
  _ : commutative join_op;
  _ : associative join_op;
  _ : left_id unit_op join_op;
  _ : forall x y, valid_op (join_op x y) -> valid_op x; 
  _ : valid_op unit_op }.
\end{lstlisting}
%
The type \code{T} is the \emph{carrier type} of the structure. The
field \code{valid_op} selects a subset of \code{T}, standing for the
``defined'' elements. The invalid (or ``undefined'') elements help
model partiality: a partial function over \code{T} will return some
invalid element on an input on which it is mathematically undefined.
\code{join_op} is the binary operation of the PCM, and \code{unit_op}
is the unit element. The remaining five unnamed fields enumerate the
axioms that have to be satisfied by each PCM instance.

Next, the mix-in ``interface'' is packaged with a carrier type, into a
dependent record type, which represents PCMs. We also introduce a
coercion from the package to the underlying carrier type, so that the
two can be conflated. This coercion essentially accounts for the
delegation hierarchy from object-oriented languages.
%
%\an{Delegation?}
%
\begin{lstlisting}
Structure pcm : Type := 
    Pack {type : Type; _ : mixin_of type}.
Coercion type : pcm >-> Sortclass.
\end{lstlisting}
%
Next, we explain the mechanism of packaging all necessary definitions
along with lemmas about data structures (such as \emph{join}'s
commutativity and associativity in the case of PCMs) into a single
module that should be imported by the clients of the algebraic
structure.  For example, we introduce appropriate notation for the
join operation, and specifically name and prove the lemmas that
correspond to the PCM properties that we left unnamed in the mixin.
\begin{lstlisting}
Notation x \+ y := (join_op x y).
Lemma joinC (U : pcm) (x y : U) : x \+ y = y \+ x.
Lemma joinA (U : pcm) (x y z : U) : 
             x \+ y \+ z = x \+ (y \+ z).
\end{lstlisting}
The lemmas such as \code{joinC} and \code{joinA} are proved by
destructing the package \code{U}, but notice how the coercion allows
conflating \code{U} with its carrier type. Also notice how the
notation \code{x \\+ y} allows the PCM \code{U} to be ommitted from
the equations themselves, as the typechecker can infer it from the
context.

Algebraic structures can inherit the properties of other, more basic
structures. Thus, we also require an analogue of object-oriented
\emph{inheritance}. We illustrate how this can be done in Coq, by
defining an interface for a \emph{cancellative} PCM, which inherits
from an ordinary PCM. The cancellative PCM is defined as the following
mix-in record:
%
\begin{lstlisting}
Record mixin_of (U : pcm) := Mixin {
  _ : forall a b c: U, valid (a \+ b) -> 
                       a \+ b = a \+ c -> b = c }.
\end{lstlisting}
%
Notice that the dependent record \code{mixin_of} in this case is
parametrized via the carrier PCM \code{U}, which is used as a target
for a coercion whenever an instance of a plain PCM or a carrier type
\code{U} is required, since coercions a transitive.
%
% \an{Hmm, are you refering to $a\ b\ c : U$? Then, it's more
%   appropriate to say that the coercion allows to conflate $U$ and its
%   underlying carrier type}.
%
% \is{Ok, added some explanation}
%

\subsubsection{Canonical instances of abstract structures}
\label{sec:canonicals}
We proceed to show how to \emph{instantiate} the definition of
abstract structure with concrete datatypes. It turns out that it is
insufficient to merely prove that a datatype satisfies the PCM
axioms. To work comfortably with an algebraic structure in practice,
one has to explicitly ``register'' the structure with the type
inference engine.

We first show what goes wrong if one doesn't perform the
``registration''.  For instance, assume we first define an instance of
a PCM for \code{nat} with addition, by proving that \code{+} with
\code{0} satisfies the PCM axioms.  Then the following lemma which
uses the generic notation \code{\\+} for the PCM operation, is
considered ill-formed by Coq. The reason is that Coq cannot figure
that there is a PCM associated with \code{nat}, and that the generic
notation \code{\\+} should be resolved with addition. Indeed, we could
have defined the PCM for \code{nat} via multiplication $(\times)$ with
$1$, in which case \code{\\+} should be resolved by~$\times$.
%
\begin{lstlisting}
Lemma add_perm (a b c : nat) : 
       a \+ (b \+ c) = c \+ (b \+ a).
\end{lstlisting}
%
This is why for any given type such as \code{nat}, we need to register
which structure should be considered as its ``default'' PCM. We do so
using \emph{canonical instances} mechanism of
Coq~\cite{Mahboubi-Tassi:ITP13,Saibi:PhD}. In the above case, once a
structure is registered as the default PCM for \code{nat}, the
\code{add_perm} lemma can be proved by selective rewriting using the
standard PCM properties.\footnote{The \emph{rewriting selectors}, \eg,
  \code{[c \+ _]}, specify by means of regular expressions, in which
  subterms of the goal the rewriting should be performed.}
%
% \an{This proof can be simplified, presumably? We haven't discussed
%   subterm selection before such as \code{c \\+ _}, so it might be
%   better not to use them. Or if we want to use selectors, then have an
%   example with them in some of the previous sections.}
%
% \is{Well, I mention that we introduce a number of concepts on the fly,
%   so I don't introduce selectors specifically (although they are
%   mentioned in the section about rewriting).}
%
\begin{lstlisting}
Proof. by rewrite joinA [c \+ _]joinC [b \+ _]joinC. Qed.
\end{lstlisting}
%
%
%
% A similar effect to the above use of canonical instances can be
% obtained by Coq's type classes, which are based on type classes in
% Haskell~\cite{Sozeau-Oury:TPHOL08}. However, similarly to the
% Haskell's type classes, Coq's ones do not provide a way to program the
% resolution policy for particular lemma instances in different
% contexts~\cite[\S~5.2]{Kriener-al:PPDP13}.
% %
% \an{Really?  Is there a reference for this?}
% %
% \is{Yes, I've just added it.}
%
We conclude this module by presenting a few more examples of algebraic
structures. We recal the structure \code{eqType} of types with
decidable equality from (\S~\ref{sec:reflect}), and present
Ssreflect's definition of a mix-in for \code{eqType} (defined in the
\code{eqtype} library). The instances of this structure provide
definitions of the boolean equality operation \code{==}, and proofs
that \code{==} is a decidable version of propositional equality
\code{=}. We exhibit a number of instances for standard datatypes such
as \code{nat} and \code{bool}. As an exercise in the recitations, the
students have to implement an interface for the
\emph{partially-ordered set} structure, and to provide a number of
canonical instances for it.


\subsection{Case study: verifying imperative programs using separation
  logic in Hoare Type Theory}

The last module of the course presents a large case study, which
employs all of the Coq programming and proving skills acquired by now:
specification and verification of imperative programs in Hoare Type
Theory (HTT)~\cite{Nanevski-al:JFP08,Nanevski-al:POPL10}.
%
HTT is an implementation of separation logic, formalized as a shallow
embedding in Coq. In particular, types are used to implement
specifications in the style of Hoare triples. The implementation
provides a number of lemmas that correspond to the customary inference
rules of separation logic, and are used to interactively establish
that an imperative program satisfies a type that corresponds to a
Hoare triple. The soundness of HTT is proved with respect to standard
state-transformer denotational semantics, which is formalized in Coq.

\subsubsection{Introducing Hoare triples using the types analogy}
\label{sec:intr-hoare-logic}

We expected that the students of the course would all have been
exposed to Hoare logic in their university education. Nevertheless, to
set up the stage, we begin the module by revisiting the main concepts,
such as \emph{partial correctness}, notations and rules. In
particular, we stress a number of points to build an intuition that
Hoare triples are a kind of ``types in disguise''. This is not a
standard way of understanding Hoare logic, but it is reasonably
accurate, especially in the case of separation logic. Indeed, an
essential property of separation logic is \emph{fault avoidance}---a
property that verified programs are memory-safe. This insight can be
summarized by the slogan ``well-proved programs don't go wrong'',
which is a slight modification of Milner's motto about typability,
generalized to a dependently-typed setting.
%
% \an{This may annoy reviewers; it sounds more like ideology than
%   reality. Do we really want it in the paper?}

Another point relating Hoare triples to types is the rule of
consequence, which allows for strengthening the precondition and
weakening the postcondition.
%
\begin{mathpar}
\hspace{-30pt}
\small{
\inferrule*[Right={(Conseq)}]
 {P \implies P' \\
  \spec{P'}~ c~ \spec{Q'}\\
  Q' \implies Q}
 {\spec{P}~ c~\spec{Q}}
}
\end{mathpar}
%
The rule features a very similar variance policy as the rule of
subtyping for arrow types in the simply typed lambda-calculus with
records~\cite[Chapter~15]{Pierce:BOOK02}: 
%
%\vspace{-10pt}
%

\begin{mathpar}
\hspace{-30pt}
\small{
\inferrule*[Right={(S-Arrow)}]
 {P  <: P' \\
  Q' <: Q}
 {P' \rightarrow Q' <: P \rightarrow Q}
}
\end{mathpar}

%\vspace{-10pt}

\subsubsection{Basics of separation logic}
\label{sec:sep-log}
After a short tour of Hoare logic, we demonstrate a number of
well-known scalability problems that arise in the presence of pointers
and aliasing. We proceed to introduce the main ideas behind separation
logic~\cite{Ishtiaq-Ohearn:POPL01,Reynolds:LICS02}, focusing mainly on
definition of heaps, and explicit heap disjointness.
%
% \an{Hmm, why is ``explicit'' in emphasis? What would be implicit heap
%   disjointness?  Maybe just say: focusing mainly on definition of
%   heaps, and how the scalability lost in the demonstrated examples
%   could be recovered if the preconditions and postconditions could
%   explicitly state disjointness between various heap values and
%   variables.}

We expected that our students will not be familiar with separation
logic, as it is not typically a topic covered in standard university
curricula. This fact embolded us to take a very non-standard approach
in our presentation, and in particular, tailor the ideas behind
separation logic to Coq-supported mechanized reasoning.

%
First of all, we ommitted \emph{stack variables} from the
presentation. As in functional programming, all our variables will be
immutable. If mutation is required, it should be done via heap
references. A consequence is that the effectful commands of our
programs has to be able to return non-unit results, unlike in
separation logic, where no results are returned.  Second, we introduce
higher-order functions and the fixed-point combinator, therefore
removing \code{while}-loops (and loop invariants), as they can be
expressed through recursion.  Third, and perhaps most drastically, we
avoid using the separation logic's standard \emph{separating
  conjunction} connective $\sep$. In the way we will work with Coq,
$\sep$ introduces a level of indirection; the first move in almost all
the proofs is to unfold the definition of $\sep$, and reveal the
existential quantification over the disjoint heaps. Instead, we base
the specifications on the operation of disjoint heap union. The latter
is well-known as the semantic foundation
behind~$\sep$~\cite{Calcagno-al:LICS07}, but we found that it works
well in practical mechanization too. Finally, we freely use
heap-valued variables in our specs, which is not a particularly
accepted practice in the world of separation logic, but (1) it works
well in Coq, and (2) it also seems unavoidable in the absence of
native logical infrastructure to reason about
bunches~\cite{Nanevski-al:POPL10}.

%of mutable \emph{stack variables}, we agree to implement mutation only
%by means of changing the values of pointers in the heap, therefore
%posing all local variables to be immutable. We also assume that the
%procedures for allocation and deallocation of fresh pointers are
%available and implemented opaquely.
%%
%Second, we agree that now each command in the language, besides having
%a possible \emph{side-effect} (\eg, changing a value of a pointer in
%the heap) returns a result, which might very well be of the
%\code{unit} type. 
%%
%Third, we introduce higher-order functions and the fixed-point
%operator to the language, therefore removing the \code{while}-loops,
%as they can be now expressed through generally-recursive functions. 
%% 
%Finally, and the most important, we totally avoid the usage of the
%separation logic's standard \emph{separating conjunction} logical
%connective $\sep$, as it introduces the unnecessary level of
%indirection that would force us to introduce and explain the
%``entails'' relation between heaps and propositions as well as a
%number of distribution laws. 

Therefore, our separation logic specs have the following form,
featuring explicitly-quantified initial/final heaps $h$ and the result
$\res$, as illustrated by the rules for writing to a pointer $x$ (the
rule from reading $!x$ is similar)
%
\begin{mathpar}
\hspace{-30pt}
\small{
\inferrule*[Right={(Write)}]
 {}
 {\spec{h ~|~ h = x \mapsto -}~ x ::= e ~\spec{\res :
     \mathtt{unit}, h ~|~ h = x \mapsto e}}
}
\end{mathpar}
%
and for a pointer allocation
%
\begin{mathpar}
\hspace{-30pt}
\small{
\inferrule*[Right={(Alloc)}]
 {}
 {\spec{h ~|~ h = \mathsf{empty}}~ \mathtt{alloc}(v) ~\spec{\res, h ~|~ h = \res \mapsto v}}
}
\end{mathpar}
%
Defining an appropriate form of a frame rule is possible, so working
with explicit heap and result variables in assertions is pleasant,
perhaps surprisingly so.
% and, as we will soon show, has to be done in Coq anyway.
%
% \is{Should we say something more about the lack of BI in Coq and
%   give references?}  \an{I said something above. Now this sentence on
%   the ``pleasentless'' seems superfluous, and should probably be
%   erased.}
%

With only immutable variables allowed, sequential composition has to
explicitly account for the results of commands, binding the result to
a variable in the continuation.

\begin{mathpar}
\small{
\inferrule*[Right={(Bind)}]
 {\spec{h ~|~ P(h) }~c_1~\spec{\res, h ~|~ Q(\res, h)} \\
 \spec{h ~|~ Q(x, h) }~c_2~\spec{\res, h ~|~ R(\res, h)}}
 {\spec{h ~|~ P(h)}~ x \asgn c_1; c_2 ~\spec{\res, h ~|~ R(\res, h)}}
}
\end{mathpar}
%
One can also \emph{inject} a \emph{pure} expression into a command by
means of the $\ret$-statement. The corresponding rule is an equivalent
of Hoare-logic rule for variable assignment:
%
\begin{mathpar}
\hspace{-30pt}
\small{
\inferrule*[Right={(Return)}]
 {} 
 {\spec{h ~|~ P(h)}~ \ret~e ~\spec{\res, h ~|~ P(h) \wedge \res = e}}  
}
\end{mathpar}
%
%\an{Hmm, why not use small footprint spec?}  
%
% Since \code{while}-loops can be expressed by means of a fixed-point
% combinator, we add an inference rules for named functions (so the
% logic now requires a form of a variable context) and a rule the
% $\fix$-construct.  \an{The last sentence seems repetitive. The fix is
%   already mentioned above.}

\subsubsection{Effectful computations as monads}
\label{sec:hoare-types-as}
A functional programmer will imediately notice that the rules
\textsc{(Bind)} and \textsc{(Return)}, shown in \S~\ref{sec:sep-log},
bear a lot of similarity with the monadic operations \emph{bind} and
\emph{return}, familiar from the Haskell's parametrized type class
\texttt{Monad}:
%
\vspace{5pt}
%
{\small{
\begin{verbatim}
class Monad m where
  (>>=)            :: m a -> (a -> m b) -> m b
  return           :: a -> m a
\end{verbatim}
}}
%
\vspace{5pt}
%
More specifically, each command in HTT returns a result of some type,
similarly to \emph{monadic} programs. Furthermore, commands can be
bound by means of the $x \asgn c_1; c_2$ syntax, and the corresponding
logic rule checks that the pre/postconditions of $c_1$ and $c_2$ agree
with each other (modulo the rule of consequence), so they could be
chained. Finally, similarly to how pure expressions in Haskell can be
embedded into a monad using the \texttt{return} command, one can
construct commands from expressions using the $\ret~e$ syntax. All
these observations illustrate that there is a correspondence in the
style of the Curry-Howard isomorphism between monadic programs and
inference rules in Hoare/separation logic, thereby lifting the
Curry-Howard theme of the course to stateful programming as well.

\subsubsection{HTT essentials}
\label{sec:htt-essentials}
We can now present the main concepts of HTT, such as the notion of
monadic \emph{Hoare type} and the notation for writing effectful
programs and their specifications in Coq.
%
But before we do so, we remind the students that in general, one
cannot implement effectful and potentially non-terminating general
recursive computations as pure expressions in Coq. (They have already
faced this issue \wrt~recursion in one of the previous lectures). To
encode such computations, similar as in Haskell, one has to
encapsulate their side-effects by a monad. However, in the case of Coq
and HTT, the monad will be more general than in Haskell. First, it
will encapuslate recursion, in addition to the effects related to
mutable state, whereas Haskell doesn't consider recursion to be an
effect in this sense. Second, the type will include the precondition
and the postcondition of the specified computation; thus such monadic
types are called Hoare types.

A Hoare type is written using the notation of the form
\texttt{\small\{x1 x2 ...\}, STsep (p, q)}. Here \code{p} and \code{q}
are heap predicates, standing for the pre and postcondition. More
specifically, the type of \code{p} is \code{heap -> Prop} and the type
of \code{q} is \code{A -> heap -> Prop}, where \code{A} is the type of
the result of the command being specified. The identifiers \code{x1},
\code{x2} \etc.~bind the logical variables that scope over both
\code{p} and \code{q}, and are implicitly universally quantified, as
customary in Hoare logics. For example, the allocation procedure,
mentioned in \S~\ref{sec:sep-log} is given the following Hoare type,
which is a straightforward rephrasing of its Hoare-style specification
with a small (\ie, minimal) heap footprint:\footnote{We omit the
  discussion on the distinction between \code{fun} and \code{vfun}; it
  has to do with the treatment of exceptions in HTT, but we don't
  consider exceptions in the course.}

%\newpage

\begin{lstlisting}
alloc : forall (A : Type) (v : A),
           STsep (fun h => h = Unit,
                 [vfun (res : ptr) h => h = res :-> v])
\end{lstlisting}
%

\subsubsection{Verifying an imperative factorial implementation}
\label{sec:verify-an-imper}
We illustrate the HTT verification machinery on an imperative
implementation of the factorial procedure
(Figure~\ref{fig:fact}). This program will be our main running example
for the section. In the course, we first present a paper-and-pencil
proof outline for this program (ommitted here). The proof outline
lists the assertions that are valid at each program point, and how
they have been justified by the inference rules of separation
logic. This demonstration introduces the students to how separation
logic is used in practice.

\begin{figure}
{\small
\begin{alltt}
\num{ 1}  fun fact (\var{N} : nat) : nat = \{
\num{ 2}      n   \(\asgn\) alloc(\var{N});
\num{ 3}      acc \(\asgn\) alloc(1); 
\num{ 4}      res \(\asgn\) 
\num{ 5}        (\(\fix\) loop (_ : unit). 
\num{ 6}            a' \(\asgn\) !acc;
\num{ 7}            n' \(\asgn\) !n;
\num{ 8}            if n' is m' + 1 then
\num{ 9}               acc ::= a' * n';;
\num{10}               n   ::= m';;
\num{11}               loop(tt) 
\num{12}            else \(\ret\) a'
\num{13}         )(tt);
\num{14}      dealloc(n);;
\num{15}      dealloc(acc);;
\num{16}      \(\ret\) res
\num{17}  \}
\end{alltt}
}
\caption{A pseudocode implementation of an imperative factorial
  procedure with pointer allocation. The notation \texttt{\small{;;}}
  stands for sequential composition without result binding.}
\label{fig:fact}
\end{figure}

Next we show how the same reasoning is implemented in Coq. We start
with providing a declarative definition \code{fact_pure} of a
factorial by means of Coq's primitive recursion.

%\newpage

\begin{lstlisting}
Fixpoint fact_pure n := 
 if n is n'.+1 then n * (fact_pure n') else 1.
\end{lstlisting}
%
Our goal will be to demonstrate that the procedure \code{fact} from
Figure~\ref{fig:fact} computes \code{fact_pure} of its input value. We
will also prove that \code{fact} does not \emph{leak} memory.

To stress the compositionality of the verification, we will break the
proof into two parts. We will first verify the recursive loop function
on lines 5-13. After that, we refactor the factorial program so that
it invokes the recursive function, rather than inline it. Then we
verify that program, reusing the already developed proof for the
recursive loop function.


The ``loop invariant'' of the recursive function is defined as a type
\code{fact_tp} below. It constrains the heap inside the loop to
consist of the two pointers: the counter \code{n} and the accumulator
\code{acc}, storing values \code{n'} and \code{a'}, respectively.
These facts are stated by using heap values \code{n :-> n'} and
\code{acc :-> a'}. Heaps form a PCM under the operation of
\emph{disjoint union}, so we use the familiar operator \code{\\+} from
the lecture on PCMs (\S~\ref{sec:depstruct}) to combine the two
single-pointer heaps into a disjoint union. Thus, the verification in
HTT in general, and of the factorial example in particular, will rely
heavily on the PCM library that the students practiced with
previously. Once the loop terminates, the postcondition of the type
\code{fact_tp} says that the counter \code{n} is decremented to
\code{0}, and the accumulator stores the return result, which equals
\code{a' * fact_pure n'}.

%\vspace{30pt}

\begin{lstlisting}
Notation fact_tp n acc := 
 {n' a'}, 
    STsep (fun h => h = n :-> n' \+ acc :-> a',
           [vfun res h => h = n :-> 0 \+ acc :-> res /\ 
                            res = a' * fact_pure n']).
\end{lstlisting}

\begin{lstlisting}
Program Definition fact_acc (n acc : ptr): 
  fact_tp n acc := 
  Fix (fun (loop : unit -> fact_tp n acc) (_ : unit) => 
    Do (a' <-- read nat acc;
        n' <-- read nat n;
        if n' is m'.+1 then 
          acc ::= a' * n';; 
          n ::= m';;
          loop tt
        else ret a')) tt.
\end{lstlisting}

%
One can see that the implementation of \code{fact_acc} almost exactly
matches the pseudocode of the lines 5--13 from
Figure~\ref{fig:fact}. The notations \code{Fix} and \code{Do} for
effectful and generally-recursive computations are provided by HTT.
%
The Coq's command \code{Program Definition} is similar to the standard
definition, except that it allows the expression being defined to have
uninstantiated components, which are left as obligations to be filled
later~\cite{Sozeau:TYPES06}. In this particular case, what is omitted
is the proof that \code{fact_acc} meets the type. 
%

We next show how this proof is built. It essentially amounts to the
application of the rule \textsc{(Conseq)}
(\S~\ref{sec:intr-hoare-logic}) to demonstrate that the type
automatically computed by Coq for the loop body (and containing the
\emph{weakest precondition} and \emph{strongest postcondition} of the
loop body) can be weakened to the explicitly provided type
\code{fact_tp n acc}. Application of \textsc{(Conseq)} issues a pair
of implications, which can be discharged together by applying a number
of helper lemmas that correspond to structural rules in separation
logic. All in all, the proof script looks as follows.
\begin{lstlisting}
Next Obligation.
apply: ghR=>_ [n' a'] /= -> _; heval.
case: n'=>[|m'] /=; rewrite ?muln1 ?mulnA; heval=>//.
by do 2![apply: (gh_ex _)]; apply: val_doR. 
Qed.
\end{lstlisting}
%

While the script is cryptic, it is conceptually straightforward.  The
first line applies HTT's lemma \code{ghR}, which applies the rule of
consequence, and ``pulls out'' all logical variables occurring in the
specification (\ie, \code{n'} and \code{a'} in this case). In Hoare
logic parlance, this amounts to an iterated application of the rule of
universal quantification (infinitary version of the rule of
conjunction). The application of this lemma is followed by some
bookkeeping of assumptions and use of HTT's \code{heval} tactic to
symbolically evaluate the program; that is, automatically apply the
rules for writing and reading from the pointers. The symbolic
evaluation stops at the conditional. Next, \code{case}-analysis on
\code{n'} considers the two branches of the conditionals:
\code{then}-branch is when the counter \code{n'} has not reached zero;
\code{else}-branch is when zero has been reached. Both branches are
first simplified by rewriting with \code{mulnA} and \code{muln1}
lemmas. These two lemmas are taken from the standard \code{ssrnat}
library, and express the associativity of multiplication, and
neutrality of \code{1}. After the simplification, the
\code{else}-branch is trivially discharged by symbolic evaluation. The
\code{then}-branch requires application of a procedure call, in this
case a recursive call to \code{loop}.  The fixed-point combinator
tells us that \code{loop} has the type \code{fact_tp n acc}, but we
need to instruct the system as to the values of logical variables
\code{n'} and \code{a'} to be used in the recursive call.  We do so by
using the lemma \code{gh_ex} twice; once for \code{n'} and again for
\code{a'}. In the particular case of \code{fact_acc}, we don't
actually have to provide the values for \code{n'} and \code{acc'}
explicitly, as the unification mechanism of Coq's will be able to
infer them from the equations in the residual subgoals. Thus, we
provide underscore \code{\_} as an argument to \code{gh_ex}. Finally,
we apply the lemma \code{val_doR}, which allows the system to verify
that by making the function call to \code{loop}, under the assumption
that \code{loop} has the type \code{fact_tp n acc}, as specified by
the fixed-point combinator, we can reach the specification for
\code{fact_acc}.

One can see that most of the proof script has to do with bookkeeping
technicalities, such as starting symbolic evaluation and naming the
variables that arise from case analysis. The only non-trivial insight
was that we need to rewrite by \code{muln1} and \code{mulnA} at an
appropriate place.
%
% \an{Hmm, I don't know if this
%   writeup is good. But I have to run now.}\is{Are you planning to
%   revise it?}

We can now specify and verify the full procedure \code{fact}, which
wraps the allocation and deallocation of the around \code{fact_acc}. 

%\vspace{42pt}

\begin{lstlisting}
Program Definition fact (N : nat) : 
  STsep (fun h => h = Unit, 
         [vfun res h => res = fact_pure N /\ h = Unit]) := 
  Do (n   <-- alloc N;
      acc <-- alloc 1;
      res <-- fact_acc n acc;
      dealloc n;;
      dealloc acc;;
      ret res).
\end{lstlisting}
%
The equalities \code{h = Unit} in \code{fact}'s pre- and
postconditions ensure that the procedure does not leak any memory. The
proof of the specification is straightforward and mostly carried out
by symbolic evaluation via the \code{heval} tactic. We omit it here,
but it can be found in the lecture notes~\cite{Sergey:PnP} and the
code accompanying them.

\subsubsection{Proving specifications with and without automation}
\label{sec:proving-with-without}

Too much reliance on proof automation implemented by unspecified
third-party tactics can easily obscure the arguments behind the
proof. To avoid such situations in the case of HTT, this module asks
the students to perform a few program verifications, without relying
on the automation provided by \code{heval}. The students are supposed
to figure out the lemmas applied by \code{heval} in the course of its
run, and apply these lemmas by hand as appropriate.

As an example, we consider a simple program that swaps the natural
values of two pointers, \code{x} and \code{y}.
%
\begin{lstlisting}
Program Definition swap (x y : ptr):
 {a b : nat},
  STsep (fun h => h = x :-> a \+ y :-> b,
        [vfun _ h => h = x :-> b \+ y :-> a]) :=
    Do (vx <-- read nat x;
        vy <-- read nat y;
        x ::= vy;;
        y ::= vx).
\end{lstlisting}
%
The proof can be carried out by first ``pulling out'' the logical
variables and applying automation by \code{heval}.
%
\begin{lstlisting}
Next Obligation. by apply: ghR=> _ [a b]-> _; heval. Qed.
\end{lstlisting}
%
However, in a paper-and-pencil proof one would consistently apply a
series of the rules, such as \textsc{(Bind)}, \textsc{(Write)}
\etc. Therefore, we suggest the students to use the \code{Search}
machinery to find the appropriate lemmas and apply them, which leads
to the following proof of \code{swap}'s specification, exactly in the
spirit of Hoare-style reasoning.
%
\begin{lstlisting}
Next Obligation.
apply: ghR => _ [a b]-> _.
apply: bnd_seq; apply: val_read => _.
apply: bnd_seq; apply: val_readR => _.
apply: bnd_seq; apply: val_write => _.
by apply val_writeR.
Qed.
\end{lstlisting}
%
Indeed, each line of the proof decomposes a sequential composition by
means of \textsc{(Bind)}-like lemma \code{bnd_seq} and then applies a
necessary lemma for writing or reading a pointer, and the variant
\code{val_readR} allows one to avoid making rewritings in the heap to
bring it to the ``right'' shape.\footnote{This problem can be avoided
  by means of lemma overloading, as demonstrated
  in~\cite{Gonthier-al:JFP13}, but we do not discuss this technique
  in the course.}  
%
% \an{Hmm, the part of the paragraph starting with
%   ``and the variant'' is too mysterious at this point. It's probably
%   best to omit it.}

Our introduction to the imperative program verification and HTT is
concluded by a series of exercises, including a proof of an imperative
Fibonacci procedure and a verification of number of procedures
operating on singly-linked lists.

%\newpage

\section{Evaluation and experience}
\label{sec:eval-exper}

A preliminary proof-of-concept evaluation of the described course has
been conducted by the first author in a form of a five-days summer
school, which took place in Saint Petersburg State University (SPbSU)
in August 2014. The school was advertised amongst the senior-years
students specializing in mathematics and software
engineering. Overall, five students enrolled for the course.

As one of the main prerequisites for the school, we listed good
knowledge of functional programming (preferably, based on a
statically-typed language, such as ML, Haskell or Scala), which
the students of the software engineering chair obtain from the
standard curriculum that includes, in particular, a course on Haskell.
%
Passing familiarity with classical propositional logic was desired,
and students at SPbSU take such a course during their second or third
year. We didn't assume any knowledge of intuitionistic logic or Hoare
logic, although simple Hoare logic is usually briefly mentioned in one
of the introductory computer science courses at SPbSU.

Every day of the school featured four hours of lectures, which were
mostly delivered in a hands-on mode, with the instructor presenting
the material through interaction with Coq. The exceptions were the
introductory lecture and parts of the last lecture on separation logic
and HTT, which required some amount of theoretical background to be
presented beforehand using slides. In the afternoon, four hours were
dedicated to solving homework assignments under supervision of the
lecturer. Every lecture started from the discussion on the solutions
of the exercises from the previous day. The exercises on HTT on the
very last day of the lectures were discussed immediately after the lab
session.

\subsection{Observations}
\label{sec:common-observations}
%In our experience based on the summer school, 
Students had little trouble learning Coq's programming syntax. They
quickly mastered writing simple recursive programs, and satisfying the
requirements imposed by Coq's termination checker. Some of the
students expressed a lot of eagerness to \emph{test} their code.
Before being introduced to formal proofs, they started writing
``proofs'' that correspond to unit tests, such as the one below.
%
\begin{lstlisting}
Theorem test_plus_3_8 : my_plus 3 8 = 11.
Proof. reflexivity. Qed.
\end{lstlisting}
%
Surprisingly, such clever hackers experienced most troubles later on,
when making transition from programming to proving, until they finally
understood the semantics of proof primitives such as \code{move} and
\code{case}.

In the process of explaining the basic concepts of the interactive
proof (\S~\ref{sec:inter-proof-constr}), we noticed that, when it
comes to applying a tactic, the students often confuse \emph{goals}
and \emph{assumptions}. Specifically, some of the student had trouble
deciding whether they should use \code{split} or \code{case} when a
conjunction is encountered in a goal. It helped when we repeated the
explanation of the four basic primitives (\code{exact:}, \code{move},
\code{case} and \code{apply:}), and emphasize that all other proof
constructions can be expressed through them.

% Equalities are usually confused with equivalences by the
% students. Fortunately, equivalences are rare in Ssreflect libraries,
% and most of the reasoning they encountered in this course can be
% conducted by simple rewriting (\ie, without setoids). However, as in
% many other cases, instead of giving the students a direct hint on how
% to prove/discharge an equivalence, it was much more beneficial (in a
% long run) to reming them of the
% \code{Search}/\code{Locate}/\code{Check}/\code{Print} utilities, so
% they would figure out themselves on how to deal with a particular
% logical connective.

Being dedicated hackers, about a half of the students often found
themselves in a situation that, after some progress with the proof,
they no longer understand the proof state. This situation typically
occurs when doing proofs by induction (\S~\ref{sec:induction}). In
such cases, the students often went to the Coq manual in search of a
powerful tactic that would solve the problem for them. We tried to
prevent these situations by suggesting them to reflect on a problem
and construct a paper-and-pencil proof first. For those who insisted
on using some standard library tactics (\ie, \code{inversion}, which
usually resulted from the definitions following the traditional Coq
style~\cite{Chlipala:BOOK,Pierce-al:SF} of using indexed type families
as GADTs), we allowed them to do so once they were able to explain the
outcome through the use of only the basic primitives.

The students found the way we introduced Hoare-style reasoning about
imperative programs out of the type analogy
(\S~\ref{sec:intr-hoare-logic}) to be quite natural. Moreover, right
after the scalability problems of the traditional Hoare logic were
listed (\ie, the proof burden appearing when reasoning about
aliasing), and before the introduction of separation logic, one of the
students has immediately suggested to make the fact of heap
disjointness to be \emph{explicit} in the assertions, which is the
crucial idea of the separation logic.

We were pleasantly surprised to find some of the students' proofs of
the exercises (\S~\ref{sec:rewriting}, \S~\ref{sec:induction}) were
shorter and conceptually simpler than those we developed ourselves as
model solutions.

\subsection{Feedback from the participants}
\label{sec:feedback}

In the post-school anonymous evaluation, we received largely positive
feedback. The course has been praised for a selection of topics with
an emphasis on verification of imperative programs, which some of the
students found to be an ``impressive piece of real-world research
project that can be taught in a comprehensible way''.

One major criticism we got had to do with a way the proofs were
presented in the files accompanying the lectures and used for the
hands-on sessions. In the later lectures we have ``compressed'' some
of the proof steps into non-atomic lines (\eg, several rewritings
followed by the bookkeeping machinery). Although not complex
conceptually, some students found such proofs hard to follow without
breaking the lines into more atomic steps. And while doing so, they
would lose the pace of the lecture, and had to ask the lecturer to
start the example over. We intend to revise the code supporting the
course to avoid this problem in the future.

\section{Related courses and future work}
\label{sec:related}

The Coq proof assistant has been in development since 1983, and by now
there are a number of courses that provide introductions to
Coq-powered proving and programming.
%
The book \emph{Interactive Theorem Proving and Program
  Development. Coq'Art: The Calculus of Inductive Constructions} by
Yves Bertot and Pierre Cast\'{e}ran~\cite{Bertot-Casteran:BOOK} is an
exhaustive overview of Coq as a formal system and tool, covering both
logical foundations, reasoning methodologies and automation, and
offering large number of examples and exercises.
%
Benjamin Pierce \etal's \emph{Software Foundations} electronic
book~\cite{Pierce-al:SF} introduces Coq development from an angle of
the basic research in programming languages, focusing primarily on
formalization of language semantics and type systems, which serve both
as main motivating examples of Coq usage and as a source of intuition
for explaining Coq's logical foundations.
%
The most recently published book, \emph{Certified Programming with
  Dependent Types} by Adam Chlipala~\cite{Chlipala:BOOK} provides an
introduction to Coq from the perspective of writing programs that
manipulate \emph{certificates}, \ie, first-class proofs of the
program's correctness. The idea of certified programming is a natural
fit for a programming language with dependent types, which Coq offers,
and the book is structured as a series of examples that make the
dependently-typed aspect of Coq shine, along with the intuition behind
these examples and a detailed overview of state-of-the-art \emph{proof
  automation} techniques.

While designing the course, we have drawn a lot of inspiration from
these books, from which we also borrowed a number of examples and
exercises. However, we often had to redesign these examples and
exercises in order to keep to the explicit but minimalistic Ssreflect
style of proof, based on the small set of core tactics, and in order
to emphasize the computational nature of properties being defined and
verified.

Initially, we intended to adopt some of the teaching insights from
Henz and Hobor's (H\&H) experience
report~\cite{Henz-Hobor:CPP11}. Alas, a majority of the practices
described there are quite specific to the teaching program of National
University of Singapore's School of Computing (in particular, some
background on modal logic was assumed). The introduction into formal
proofs in H\&H's course is made through the Aristotle’s term logic,
whereas we deliberately strived to employ the students' intuition
acquired from the courses on functional programming. Both approaches
have their strengths. In particular H\&H's seems to be more
approachable by students with no background in formal methods. On the
other hand, our course is at freedom to employ a richer vocabulary of
programming analogies, making it easier to explain the notions such as
inductive proof, dependent records and Hoare types. Noteworthy, H\&H
notice the same anti-patterns, exhibited by the students during the
learning process, in particular, the undue ``hacking'' with tactics.

In the future, we plan to extend the course with a discussion on
co-fixpoints and proofs by coinduction, supported by recent advances
on the mechanized proof construction in this
area~\cite{Hur-al:POPL13}. We also intend to enhance the course with a
survey of proof automation techniques by means of of tactic
engineering~\cite{Stampoulis-Shao:ICFP10,Ziliani-al:ICFP13} or lemma
overloading via canonical
structures~\cite{Gonthier-al:JFP13,Mahboubi-Tassi:ITP13}.

\section{Conclusion}
\label{sec:conclusion}

Programming and proving are the two sides of the same coin:
construction of scalable and reusable formal proofs can be effectively
taught basing on the knowledge of constructing scalable software
systems.
%
In this report, we have described the design of an experimental course
on mechanized reasoning in the Coq proof assistant, which embraces the
proving-as-programming insight.
%
The particular cases of this duality, which we have instantiated as
teaching principles in a class, are the computational approach to
decidable properties, boolean reflection, implementation of algebraic
structures as dependent records with inheritance, and reasoning about
effectful programs in a Hoare-style program logic.

It is our belief that the proposed approach to teaching mechanized
theorem proving and formal reasoning can be adopted in educational
programs for students, who already have background in software
engineering and program design, and we support this claim by reporting
on our preliminary teaching experience, which is positive.

All the materials of the course, including the lecture notes,
exercises, source files for the hands-on lectures, and necessary
libraries, are available online~\cite{Sergey:PnP}.

\acks
% \todo{Acknowledgments, if needed.}

We are grateful to Dmitri Boulytchev for his initiative to organize a
summer school on dependent types at SPbSU. We also thank Anindya
Banerjee, Olivier Danvy, Michael D. Ernst, R\'{e}my Haemmerl\'{e},
Jos\'{e} Francisco Morales, Pablo Nogueira and \'{E}ric Tanter for
their comments on the lecture notes and on earlier drafts of this
article.

\bibliographystyle{abbrvnat}
\bibliography{references,proceedings}

\end{document} 


