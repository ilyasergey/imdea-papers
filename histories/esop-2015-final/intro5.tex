\section{Introduction}
\label{sec:intro}

For sequential programs and data structures, Hoare-style
specifications (or specs) in the form of pre- and postconditions are %uniformly
%accepted as 
a declarative way to express a program's behavior. For example, an
abstract specification of stack operations can be given as follows:
%
\[
\tag{\arabic{tags}}\refstepcounter{tags}\label{eq:seqstack-spec} 
{\small
\hspace{-5pt}
\begin{array}{r@{\ }c@{\ }l}
\spec{~s \hpts \mathit{xs}~} &\Cmd{push}{x} & \spec{~ s \hpts
  x::\mathit{xs}~}
\\[3pt]
\spec{~s \hpts \mathit{xs}~} & \Cmd{pop}{} &\spec{
\begin{array}{l}
  \res = \None \aand \mathit{xs} = \nil \aand s \hpts \nil \oor \hbox{}\\
  \res = \Some x \aand \exists \mathit{xs'},~\mathit{xs} = x ::\mathit{xs'} \aand
  s \hpts \mathit{xs'}
\end{array}
}
\end{array}
}
\]
% 
where $s$ is an ``abstract pointer'' to the data structure's logical
contents, and the logical variable $\mathit{xs}$ is universally
quantified over the spec.
%
The result $\res$ of $\cmd{pop}$ is either $\Some x$, if $x$ was on
the top of the stack, or $\None$ if the stack was empty.
%
The spec (\ref{eq:seqstack-spec}) is usually accepted as canonical for
stacks: it hides the details of method implementation, but exposes
what is important about the method behavior, so that a verification of
a stack \emph{client} does not need to explore the implementations of
$\cmd{push}$ and $\cmd{pop}$.

The situation is much more complicated in the case of concurrent data
structures. In the concurrent setting, the
spec~\eqref{eq:seqstack-spec} is of little use for implementations
with server-side locking, as the interference of the threads executing
concurrently may invalidate the assertions about the stack. For
example, a call to $\cmd{pop}$ may encounter an empty stack, and
decide to return $\None$, but by the time it returns, the stack may be
filled by the other threads, thus invalidating the postcondition of
$\cmd{pop}$ in~\eqref{eq:seqstack-spec}. To soundly reason about
concurrent data structures, one has to devise specs that are
\emph{stable} (\ie, invariant under interference), but this may
require trade-offs with respect to the specifications' expressivity
and precision for the client's needs.
% 

Reasoning about concurrent data structures is further complicated by
the fact that their implementations are often \emph{fine-grained}.
Striving for better performance, they avoid explicit locking, and
implement sophisticated synchronization patterns that deliberately
rely on interference.
% 
For reasoning purposes, however, it is desirable that the clients can
perceive such fine-grained implementations as if they were
\emph{coarse-grained}; that is, as if the effects of their methods
take place \emph{atomically}, at singular points in time.  The
standard correctness criteria of
\emph{linearizability}~\cite{Herlihy-WingTOPLAS90} establishes that a
fine-grained data structure implementation \emph{contextually refines}
a coarse-grained one~\cite{Filipovic-al:TCS10}. One can make use of a
refined, fine-grained, implementation for efficiency in programming,
but then soundly replace it with a more abstract coarse-grained
implementation to simplify the reasoning about clients.

% , 
% therefore
%taking advantage of a \emph{granularity abstraction}.

Semantically, one program linearizes to another if the
\emph{histories} of the first program (\ie, the sequence of actions it
executed) can be transformed, in a suitable sense, into the histories
of the second. Thus, histories are an essential ingredient in
specifying fine-grained concurrent data structures. However, while a
number of logical methods exist for establishing the linearizability
relation between two programs, for a class of data
structures~\cite{OHearn-al:PODC10,Elmas-al:TACAS10,Vafeiadis:PhD,Liang-Feng:PLDI13},
in general, it is a non-trivial property to prove and use. First, in a
setting that employs Hoare-style reasoning, showing that a
fine-grained structure refines a coarse-grained one is not an end in
itself. One still needs to ascribe a stable spec to the coarse-grained
version~\cite{Turon-al:ICFP13,Liang-Feng:PLDI13}.
%
Second, the standard notion of linearizability does not directly
account for modern programming features, such as ownership transfer of
state between threads, pointer aliasing, and higher-order
procedures. Theoretical extensions required to support these features
are a subject of active ongoing
research~\cite{Cerone-al:ICALP14,Gotsman-Yang:CONCUR12}.
%
Finally, being a relation on \emph{two} programs, deriving
linearizability by means of logical inference inherently requires a
\emph{relational program
  logic}~\cite{Turon-al:ICFP13,Liang-Feng:PLDI13}, even though the
spec one is ultimately interested in 
%(\eg,~\eqref{eq:stack-weak} for a concurrent stack) 
may be expressed using a Hoare triple that operates over a
\emph{single} program.

In this paper, we propose a novel method to specify and verify
fine-grained programs by directly reasoning about histories in the
specs of an elementary Hoare logic. We propose using
\emph{timestamped} histories, which carry information about the atomic
changes in the abstract state of the program, indexed by discrete
timestamps, and tracking the history of a program as a form of
auxiliary state.

While using histories in Hoare-style specs is a simple and natural
idea, and has been used
before~\cite{Bell-al:SAS10,Fu-al:CONCUR10,Gotsman-al:ESOP13}, in our
paper it comes with two additional novel observations.

First, timestamped histories are technically very similar to heaps, as
both satisfy the algebraic properties of a \emph{partial commutative
  monoid} (PCM). A PCM is a set $\pcmS$ with an associative and
commutative \emph{join} operation $\pcmF$ and unit element
$\pcmU$. Both heaps and histories (considered as sets of actions with
distinct timestamps, correspondingly) form a PCM with disjoint union
and empty heap/history as the unit. Also, a singleton history $t \hpts
a$ is similar to the singleton heap $x \hpts v$ containing only the
pointer $x$ with value $v$.
%
We emphasize the connection by using the same notation for~both.
%
The common PCM structure makes it possible to reuse for histories the
ideas and results developed for heaps in the work on separation
logic~\cite{Calcagno-al:LICS07}. In particular, in this paper, we make
both heaps and histories subject to the same assertion logic, the same
rules of inference (\eg, the frame rule), and thus the same style of
\emph{local reasoning}. Moreover, concepts such as ownership transfer,
well-studied for heaps, apply to histories as well. For example, in
Section~\ref{sec:flatco}, we use ownership transfer on histories to
formalize the important design pattern of
\emph{helping}~\cite{Hendler-al:SPAA10}, whereby a concurrent thread
may execute a task on behalf of other threads. That helping
corresponds to a kind of ownership transfer (though not on histories,
but on auxiliary commands) has been noticed
before~\cite{Turon-al:POPL13,Liang-Feng:PLDI13}.  However, commands
do not form a PCM, while histories do---a fact that makes our
development simple and uniform.

Second, we argue that precise history-based specs have to
differentiate between the actions that have been performed by the
specified thread, from the actions that have been performed by the
thread's concurrent environment. Thus, our specs will range over
\emph{two} different history-typed variables, capturing the
timestamped actions of the specified thread (\emph{self}) and its
environment (\emph{other}), respectively. This split between self and
other will provide us with a novel and very direct way of relating the
functional behavior of a program to the interference of its
environment, leading to specs that have a similar canonical ``feel''
in the concurrent setting, as the specs (\ref{eq:seqstack-spec}) have
in the sequential~one.

The self/other dichotomy required of histories is a special case of
the more general specification pattern of \emph{subjectivity},
observed in the recent related work on Subjective and Fine-grained
Concurrent Separation Logic
(FCSL)~\cite{LeyWild-Nanevski:POPL13,Nanevski-al:ESOP14}. That work
generalized Concurrent Separation Logic (CSL)~\cite{OHearn:TCS07}
%,and its modular rules for framing and parallel composition, 
to apply not only to heaps, but to any abstract notion of state (real
or auxiliary) satisfying the PCM properties.
%
We thus reuse FCSL~\cite{Nanevski-al:ESOP14} \emph{off-the-shelf}, and
instantiate it with histories, \emph{without any additions to the
  logic or its meta-theory}. Surprisingly, the FCSL style of auxiliary
state is sufficient to enable expressive history-based proofs of
realistic fine-grained algorithms, including those with helping.


Specifications with histories also allow the clients of a fine-grained
data structure to pretend, for the sake of simplifying their own
reasoning, that they are using a coarse-grained version of the data
structure.
%
%Specifications with histories also provide a \emph{form} of so-called
%\emph{granularity abstraction}, in the sense that clients of a
%fine-grained data structure can pretend, for the sake of simplifying
%their own verification, that they are using a coarse-grained
%version~\cite{Turon-al:ICFP13}.
%
In this sense, we consider a program \emph{logically} atomic
(irrespective of the physical granularity of its implementation), if
its specification is a singleton history $t \hpts a$, containing only
an abstract action $a$ time-stamped with $t$. This spec provides an
abstraction that the effect $a$ of the program takes place at a
singular point in time $t$, as if the program were coarse-grained,
thus providing a uniform way to reason about coarse- and fine-grained
programs.\footnote{An orthogonal aspect of granularity abstraction is
  the ability of a logical framework to express synchronized changes
  to auxiliary state that is spread across several shared data
  structures. We do not consider such abstraction in this paper, but
  elaborate in Section~\ref{sec:related} on how to extend FCSL to
  support it, as well as on related
  approaches~\cite{Svendsen-Birkedal:ESOP14,Svendsen-al:ESOP13,ArrozPincho-al:ECOOP14,Jacobs-Piessens:POPL11}.}
%
%Another
%  treatment of the notion of granularity abstraction is the ability of
%  a logical framework to express \emph{synchronized changes} of
%  auxiliary state, which is logically spread across \emph{several}
%  shared data
%  structures~\cite{Svendsen-Birkedal:ESOP14,Svendsen-al:ESOP13,ArrozPincho-al:ECOOP14,Jacobs-Piessens:POPL11}. Although
%  ortogonal to this work's idea of using histories for abstraction of
%  granularity, such ability is currently not supported by FCSL, and we
%  elaborate in Section~\ref{sec:related} on the ways to address this
%  shortcoming, as well as on related approaches.}
%

We show how a number of well-known algorithms can be proved logically
atomic, and illustrate how the specs with histories facilitate
client-side reasoning. We consider an atomic pair snapshot data
structure~\cite{Qadeer-al:TR09,Liang-Feng:PLDI13}
(Section~\ref{sec:overview}), a Treiber stack~\cite{Treiber:TR} along
with its clients (Section~\ref{sec:examples}), and Hendler \etal's
flat combining algorithm~\cite{Hendler-al:SPAA10}, a non-trivial
example employing first-class functions and helping
(Section~\ref{sec:flatco}). All our proofs, including the theory of
histories, have been checked mechanically in Coq, and the sources are
available online~\cite{Sergey-al:ESOP15ext}.
%

