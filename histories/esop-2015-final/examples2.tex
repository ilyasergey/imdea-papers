\section{Treiber stack and its client}
\label{sec:examples}

In this section we illustrate how histories can be used to specify and
verify the fine-grained data structure of Treiber
stack~\cite{Treiber:TR}. We also show how the specs can be used by
clients, where they provide an abstraction that facilitates client
reasoning as if the structure were coarse-grained.

%%\vspace{-3pt}

\begin{wrapfigure}{r}{4.5cm}
\centering
\scriptsize
%
\begin{tabular}{l}
%
\begin{minipage}[l]{4.5cm}
~\begin{alltt}
\num{1} push(e : \(A\)): \Unit \{
\num{2}  p <- \act{alloc}();
\num{3}  \textbf{fix} loop() \{
\num{4}   p1 <- \act{readSentinel}();
\num{5}   \act{write}(p, (e, p1));
\num{6}   ok <- \act{tryPush}(p1, p);
\num{7}   \textbf{if} ok \textbf{then return} ();
\num{8}   \textbf{else} loop();\}();
\num{9} \}
\end{alltt} 
\end{minipage}
\\\\\hline \\
\begin{minipage}[l]{4.5cm}
\begin{alltt}
\num{1} pop(): \act{option} \(A\) \{
\num{2}  p <- \act{readSentinel}();
\num{3}  \textbf{if} p == null 
\num{4}  \textbf{then return} \act{None};
\num{5}  \textbf{else} \{
\num{6}   (e,p1) <- \act{readNode}(p);
\num{7}   ok     <- \act{tryPop}(p,p1);
\num{8}   \textbf{if} ok \textbf{then return} \act{Some} e;
\num{9}   \textbf{else} pop();\}\}
\end{alltt}
\end{minipage}
%
\end{tabular} 
%
\caption{Treiber stack methods.}
\label{fig:treiber-code}
\end{wrapfigure}
%
The Treiber stack works as follows. Physically, the stack is kept as a
singly-linked list in the heap, with a sentinel pointer $\sent$
pointing to the stack top $\eesc{p1}$. The call to \code{push(e)}
allocates a node $\eesc{p}$ that is supposed to go to the top of
stack, and attempts to link the node into the stack, by changing the
sentinel to $\eesc{p}$. Clearly, this operation should not succeed if
some interfering thread has in the meantime changed the top by pushing
or popping elements. Thus \code{push} applies a CAS read-modify-write
operation~\cite{Herlihy-Shavit:08}, which atomically reads $\sent$,
compares its contents with $\eesc{p1}$, and if the two are equal (\ie,
if the stack's top has not changed), writes $\eesc{p}$ into $\sent$,
thus en-linking the new top. Otherwise, \code{push} is restarted.
%
$\code{pop()}$ behaves similarly. It reads the first node $\code{p}$,
pointed to by $\sent$, and obtains its value $\eesc{e}$ and pointer
$\eesc{p1}$ to the next node. Then it tries to de-link $\code{p}$, by
changing the sentinel to $\eesc{p1}$ using a CAS to identify
interference. Note that \code{pop} does not deallocate the de-linked
node $\eesc{p}$ (this is enforced by the design of the appropriate
concurroid as we will soon see), which thus remains in the data
structure as garbage. This is by design, to prevent the ABA
problem~\cite[\S10]{Herlihy-Shavit:08}: if $\eesc{p}$ is deallocated,
then some other \code{push} may allocate it again, and place it back
on top of the stack. A procedure that observed $\eesc{p}$ on top of
the stack, but has not performed its CAS yet may thus be fooled as
follows. Its CAS may encounter $\eesc{p}$ on top of the stack, and
proceed as if the stack had not changed, producing invalid results.

The described code of the Treiber stack operations is given in
Figure~\ref{fig:treiber-code}, where we used descriptive names for the
atomic operations. Instead of CAS, we used \aact{tryPush} and
\aact{tryPop}, and instead of pointer \aact{read}, we used
\aact{readSentinel} and \aact{readNode}. The reason for the
descriptive names is that the atomic operations in FCSL operate not
only on concrete heap pointers, but on auxiliary state as well. In the
particular case of Treiber, the auxiliary state will be histories,
which \aact{tryPush} and \aact{tryPop} change in different ways, even
though they both operationally perform a CAS. Similarly,
\aact{readSentinel} and \aact{readNode} deduce different facts about
the histories, even though they both simply read from a pointer.
%
We elide here any further discussion on how the atomic operations are
specified and verified in FCSL (it can be found
in~\cite{Nanevski-al:ESOP14}
and~\cite[Appendix~C]{Sergey-al:ESOP15ext}). Instead, whenever needed,
we simply state the Hoare specs for the atomics and proceed to use
them in proof outlines, as if the atomics were ordinary procedures. Of
course, our Coq files contain proofs that all such Hoare triples are
valid.

\newcommand{\tbinv}{I}

\paragraph{Treiber concurroid.}
Given a label $\htb$, the sentinel pointer $\sent$, and the type $A$
of the stack elements, the state space of the Treiber concurroid
$\tbcon$ is described as follows. Its auxiliary self/other components
are histories $\histS$ and $\histO$ that store mathematical sequences
$l$ corresponding to the logical contents of the stack at various
timestamps. The joint component contains a heap $h_s$ storing a
sentinel $\sent$ pointing to a linked list, a heap~$h$ implementing
the list, and a garbage section $\garb$ of de-linked nodes.
%
% \footnote{We will be using explicit heap (or general PCM)
%   variables, such as $h_s$, in predicates and specs, as it has to be
%   done in Coq anyway, since Coq lacks support for contexts of
%   bunches~\cite{Nanevski-al:POPL10}.}
%
%\footnote{Here and further, we will be freely using explicit
%  heap (or general PCM) variables, such as $h_s$, in predicates and
%  specs, because it allows us to avoid introducing state modlities in
%  the presence of several concurroids and self/joint/other state as
%  well as stating a number of laws relating such modalities with other
%  assertion connectives. Moreover, working with heap variables has to
%  be done in Coq anyway, since Coq lacks support for BI contexts, as
%  argued in~\cite{Nanevski-al:POPL10}.}
%

\[
\tag{\normalsize{\arabic{tags}}}\refstepcounter{tags}\label{eq:tb-states} 
{\small
\begin{array}{rcl} 
W_{\tbcon} & \eqdef & \exists \histS~\histO~h_s\ldot \htb \spts \histS \aand
\htb \opts \histO \aand \htb \jpts h_s \aand \tbinv~(\histS \hunion
\histO)~h_s\\[3pt]
\tbinv~\hist~h_s & \eqdef &
\exists p~h~\garb~l\ldot h_s = (\sent \hpts p) \hunion h \hunion \garb
\aand \llist(p, l, h) \aand\hbox{}\\
&& \complete(\hist) \aand \mathsf{continuous}(\hist) \aand \cont(\hist) \aand \inat{l}{\hist}{\last{\hist}}
\end{array}
}
\]
%
The auxiliary predicates are:
%
%%\vspace{-10pt}
%
\[
%\tag{\arabic{tags}}\refstepcounter{tags}\label{eq:tb-preds}
\hspace{-5pt}
{\small
\begin{array}{r@{\ }c@{\ }l}
\llist(p, l, h) & \eqdef & 
p = \Null \aand l = \nil \aand h = \hempty ~\oor \\
&& \exists e\ p'\ l'\ h'\ldot l = e::l' \aand h = p\hpts(e, p') \hunion h' \aand
\llist(p',l',h')\\[2pt]
\complete(\hist) & \eqdef &
\exists l_0\ldot \hist(0) = (l_0, l_0) \aand
\forall t\ldot t < |\dom{\hist}| \Rightarrow t \in \dom{\hist}
\\[2pt]
\cont(\hist) & \eqdef &
\forall t \in \dom{\hist}\ldot t > 0 ~\Rightarrow \exists l~e\ldot \hist(t) = (l,e::l) \oor \hist(t)
= (e::l,e)
\end{array}
}
\]
%
%
In particular: (1) the overall history $\histS \hunion \histO$ is
complete, \ie no gaps exist between timestamps (this property was
irrelevant for the pair snapshot structure, but essential for stacks
to ensure the absence of the ABA-problem); (2) aside from the
initialization in timestamp $0$, the history only stores events
corresponding to pushing or popping, and (3) the last recorded state
in the history captures the current contents of the stack. For
simplicity, we disable reasoning about the structure's inherent memory
leak by not relating histories to $\garb$ in \eqref{eq:tb-states}.

The transitions of $\tbcon$ allow for popping and pushing only. 
%
\[ 
{ 
\small
\hspace{-3pt}
  \begin{array}{r@{\ }c@{\ }l@{\ \ }c}
    \taupop & \eqdef &\htb \jpts \sent \hpts p \hunion h \hunion
    \garb \aand \htb \spts \histS  \aand h = (p \hpts (e, p')\hunion h') \aand \llist(p, (e::l), h) &
    \rightsquigarrow 
    \\
    & &\htb \jpts \sent \hpts p' \hunion h' \hunion (p \hpts (e,
    p') \hunion \garb) \aand \htb \spts \histS \hunion \tfr{\histS \hunion \histO} \hpts
    (e::l, l) 
    \\[3pt]
    \alphapush_{p', e, p} & \eqdef &\htb \jpts \sent \hpts p \hunion h \hunion
    \garb \aand \htb \spts \histS \aand \llist(p, l, h) &  \rightsquigarrow 
    \\
    & & \htb \jpts \sent \hpts p' \hunion (p' \hpts (e, p) \hunion h) \hunion
    \garb \aand \htb \spts \histS \hunion \tfr{\histS \hunion \histO} \hpts
    (l, e::l) 
    \end{array}
}\]
%
In $\taupop$, the sentinel pointer is swapped from used-to-be head $p$
to its next one, $p'$, whereas $(p \hpts -)$ logically joins the
garbage. The transition $\alphapush$ describes how a heap of the shape
$p' \hpts (e, p)$, describing the node to be pushed, is acquired and
placed at the top of the stack. It is an external transition, which
means it only fires when entangled with a concurroid from which the
heap $p' \hpts (e, p)$ can be taken away. In our case, that will be
the concurroid $\privcon$ for private state. 
%
Indeed, both transitions preserve the state invariant
$I$~\eqref{eq:tb-states}.
%
Importantly, $\tbcon$ does not have a release transition; once a memory
chunk is in the joint state, it never leaves, capturing that $\tbcon$
does not allow deallocation.
%
%Another important stability property, derivable from the above
%transitions, is that no pointer except $\sent$ in the joint part of
%$\tbcon$ ever changes its value.


\paragraph{Method specs.}
We give the following history-based specs.
%
\[
\tag{\normalsize \arabic{tags}}\refstepcounter{tags}\label{eq:stack-spec}
{\small
\begin{array}{c}
\hspace{-5pt}
\begin{array}{r@{\ }c@{\ }l}
\spec{
  \begin{array}{c}
    \hpriv \spts \hempty ~\sep \\
    \histso{\htb}{\hempty, -, \hist}
  \end{array}
} &\eesc{push}(e)& 
\spec{
  \begin{array}{c}
    \exists t~l\ldot \hpriv \spts \hempty ~\sep\\
   \histso{\htb}{t \hpts  (l,e::l), -, \hist} \aand \hist < t
  \end{array}
}@\privcon \entangle \tbcon
\end{array}
\\\\ 
\begin{array}{c}
\spec{
  \begin{array}{l}
    \histso{\htb}{\hempty, -, \hist}
  \end{array}
} \\
\eesc{pop}()
\\
\spec{
  \begin{array}{c}
    \exists e~t~l\ldot \res = \Some e \aand 
    \histso{\htb}{t \hpts (e::l,l), -, \hist} \aand
    \hist < t \oor \hbox{}\\
    \exists \histO~t\ldot \res = \None \aand \histso{\htb}{\hempty, \histO, \hist} \aand \tani{\nil}{\histO}{t}\\
  \end{array}
}@\tbcon
\end{array}
\end{array}
}
\]
%
%\hspace{-5pt}
%
\begin{wrapfigure}[17]{r}{6.8cm} 
%\vspace{-8pt}
\centering
\begin{minipage}[l]{6.8cm}
\[
{\scriptsize
\hspace{-5pt}
~~\begin{array}{r@{\ \ }ll}
  \Num{1}& \sspec{~~\hpriv \spts \hempty ~\sep~ \histso{\htb}{\hempty, -, \hist}~~} & 
  \\
  \Num{2} &  ~\esc{p <- [\act{alloc}()];}~ & 
  \\
  \Num{3} & \sspec{~~\hpriv \spts \esc{p} \hpts - ~\sep~ \histso{\htb}{\hempty, -, \hist}~~}
  \\
  \Num{4} &  ~\esc{\textbf{fix} loop() \{}~ & 
  \\
  \Num{5}& \sspec{~~\hpriv \spts \esc{p} \hpts - ~\sep~ \histso{\htb}{\hempty, -, \hist}~~}
  \\
  \Num{6} &  ~\esc{p1 <- [\act{readSentinel}()];}~ & 
  \\
  \Num{7}& \sspec{~~
     \hpriv \spts \esc{p} \hpts - ~\sep~
     \histso{\htb}{\hempty, -, \hist}
  ~~} 
  \\
  \Num{8} &  ~\esc{[\act{write}(p, (e, p1))];}~ & 
  \\
  \Num{9}& \sspec{~~
     \hpriv \spts \esc{p} \hpts (\esc{e}, \esc{p1}) ~\sep~
     \histso{\htb}{\hempty, -, \hist} 
  ~~} 
  \\
  \Num{10} &  ~\esc{ok <- \act{tryPush}(p1, p);}~ & 
  \\[1pt]
  \Num{11}& \spec{
    \begin{array}{l}
     \esc{ok} \aand \exists t\ l\ldot \hpriv \spts \hempty ~\sep~
     \histso{\htb}{t \hpts (l,e::l), -, \hist}\aand 
     \hist < t~\oor\\
     \neg\esc{ok} \aand 
     \hpriv \spts \esc{p} \hpts (\esc{e}, \esc{p1}) ~\sep~ \histso{\htb}{\hempty, -, \hist}
    \end{array}
  } 
  \\
  \Num{12} &  ~\esc{\textbf{if} ok \textbf{then return} ();}~ & 
  \\
  \Num{13}&\sspec{~~\exists t\ l\ldot
     \hpriv \spts \hempty  ~\sep~\histso{\htb}{t \hpts (l,e::l), -, \hist} \aand \hist < t ~~}
  \\
  \Num{14} &  ~\esc{\textbf{else}}~ & 
  \\
  \Num{15} &  
  \sspec{~~\hpriv \spts \esc{p} \hpts - ~\sep~ \histso{\htb}{\hempty, -, \hist}~~}& 
  \\
  \Num{16} &  ~\esc{loop();\}();}~ & 
  \\
  \Num{17}&\sspec{~~\exists t\ l\ldot
     \hpriv \spts \hempty  ~\sep~ \histso{\htb}{t \hpts (l,e::l), -, \hist} \aand \hist < t ~~}
\end{array}
}
\]
\end{minipage}
\caption{A proof outline of Treiber's \code{push} method.}
\label{fig:push-proof}
\end{wrapfigure}
%
A call to \code{push} runs with empty private heap and history, thus
by framing, it can run with any private heap and history. After
termination, the self history is incremented by a singleton exposing
that a push event has been executed at a time stamp $t$; $\hist < t$
indicates that the push event appeared strictly after the events
preceding the call. The spec for \code{pop} is slightly more
complicated as \code{pop} checks for stack emptiness, but ultimately
proceeds in the similar manner. \code{push} works over the entangled
concurroid $\privcon \entangle\tbcon$, as it needs to allocate memory;
\code{pop} works over $\tbcon$ only, as it does not deallocate.

Verification of \code{push} and \code{pop} implementations relies on
the specifications of the atomic actions \aact{alloc} and
\aact{write}, which are specific to the $\privcon$ concurroid.
%
\[
{\small
\tag{\arabic{tags}}\refstepcounter{tags}\label{eq:alloc-spec}
\begin{array}{r@{\ }c@{\ }l}
\sspec{~
    \hpriv \spts \hempty
~} &\aact{alloc}()& 
\sspec{~
    \hpriv \spts  \res \hpts -
~}@\privcon
\\[2pt]
\sspec{~\hpriv \spts x \hpts - ~}
&
\aact{write}(x, e)
&
\sspec{~\hpriv \spts x \hpts e~}@\privcon
\end{array}
}
\]
%
In Figure~\ref{fig:push-proof}, we present the proof outline for
\code{push} (the proof for \code{pop} can be found in the Coq
files). It is mostly self-explanatory, so we only point out a few
technicalities.  The actions \aact{alloc} and \aact{write} have to be
explicitly injected into $\privcon \entangle \tbcon$, by means of the
coercion $[-]$, introduced in Section~\ref{sec:background}. Similarly
for \aact{readSentinel}, whose concurroid is $\tbcon$. Somewhat
surprisingly, the call to \aact{readSentinel} in line 6 is irrelevant
for the (partial) correctness of \aact{tryPush}; thus, line 7 does not
say anything about \eesc{p1}.\footnote{Though, taking a random
  \eesc{p1} here will affect liveness, as \code{push} will keep
  looping until it finds the chosen \eesc{p1} at the top of the
  stack.}
%
The proof rule for \code{fix} allows assuming the spec of a procedure
in the proof of the body, and is presented
in~\cite[Appendix~D]{Sergey-al:ESOP15ext}.
%
The \aact{tryPush} action appears in the proof outline with its
precise specification; that is, line 9 contains its precondition, and
11 contains the postcondition, describing that a successful outcome of
\aact{tryPush} removed a heap from $\privcon$, moved it to the joint
heap of $\tbcon$, and updated the history, following the $\alphapush$
transition.


\paragraph{Recovering sequential specifications.}\label{sec:rest-seq}
%
We next show that the subjective spec~\eqref{eq:stack-spec} is a
generalization of the canonical sequential
spec~\eqref{eq:seqstack-spec}. In particular, if there is no
interference from other threads, \eqref{eq:stack-spec} can be reduced
to~\eqref{eq:seqstack-spec}. The mechanism for achieving the reduction
relies on the self/other dichotomy, thus substantiating our point that
the dichotomy is important for precise reasoning with histories.

\begin{wrapfigure}{r}{6.0cm}
%\vspace{-10pt} 
\centering
\[
{\scriptsize
\hspace{-5pt}
\begin{array}{r@{\ \ }l}
  \Num{1}& \sspec{\exists p~h\ldot \hpriv \spts (\sent \hpts p \hunion h) \aand \llist(p, l, h)} 
  \\
  \Num{2} & \sspec{\Psi\ \hempty\ \hempty~\sep~\left(\Phi(\hempty)
      \wand \histso{\htb}{0 \hpts (l, l), -, -} \right) }

  \\[2.5pt]
  \Num{3} &  ~\mathsf{hide}_{\Phi, \hempty}~\esc{\{} ~  
  \\
  \Num{4} & 
  \sspec{~\hpriv \spts \hempty ~\sep~ 
    \histso{\htb}{0 \hpts (l, l), -, -}  ~} 
  \\[2.5pt]
  \Num{5} &  ~\esc{push}(e); ~  
  \\
  \Num{6} &
  \spec{\begin{array}{l}
      \exists t~l'\ldot \hpriv \spts \hempty ~\sep\\
    \histso{\htb}{0 \hpts (l, l) \hunion t \hpts (l', e::l'), -, -}
   \end{array}} ~~~~~\esc{\}} 
  \\[2.5pt]
  \Num{7} &
  \spec{\begin{array}{l}
  \exists \hist\ldot \Psi~\hist~\hempty~\sep \\
   (\Phi(\hist) \wand \exists t~l'\ldot \histso{\htb}{0 \hpts  (l, l)
     \hunion t \hpts  (l', e::l'), -, -})
 \end{array}} 
  \\[3.5pt]
  \Num{8} &
  \spec{\begin{array}{l}
          \exists t~l'~\hist\ldot\hist = 0 \hpts (l, l) \hunion t \hpts (l', e::l') \aand \hbox{}\\
          \mathsf{complete}(\hist) \aand \mathsf{continuous}(\hist) \aand \Psi~\hist~\hempty
         \end{array}}
       \\[2.5pt]
  \Num{9} &
  \sspec{~\exists \hist\ldot \hist = 0 \hpts (l, l) \hunion 1 \hpts (l, e::l) \aand \Psi~\hist~\hempty}\\
  \Num{10}& \sspec{~\exists p'~h\ldot \hpriv \spts (\sent \hpts p' \hunion h \hunion -) \aand \llist(p', e::l, h)~}
\end{array}
}
\]
%\vspace{-5pt}
\caption{Proof of sequential spec for \code{push}.}
\label{fig:seq-proof}
\end{wrapfigure}
%
To this end, we use the $\mathsf{hide}$ constructor from
Section~\ref{sec:background}. It introduces a concurroid in a
delimited scope, and prohibits the environment threads from
interfering on it. The heap for the introduced concurroid is
appropriated from the private heap. In the case of \code{push}, we
will appropriate a heap storing the sentinel and the linked list of
the stack, install the $\tbcon$ concurroid over this heap, perform
\code{push} with interference disabled, then return the heap back to
private heaps. We will derive the following specification, which is
essentially an elaborated version of~\eqref{eq:seqstack-spec}, modulo
the memory leak inherent to Treiber stack (hence $\garb$ in the
postcondition).
%
%%\vspace{-3pt}
%
\[
\tag{\arabic{tags}}\refstepcounter{tags}\label{eq:seq}
{\small
\begin{array}{c}
\sspec{~\exists p~h\ldot \hpriv \spts (\sent \hpts p \hunion h) \aand
  \llist(p, l, h)~} 
\\[3pt]
\mathsf{hide}_{\Phi, \hempty}\ \esc{\{}~\code{push}(e);~\esc{\}}
\\[3pt]
\sspec{~\exists p~h~\garb\ldot \hpriv \spts (\sent \hpts p \hunion h \hunion \garb) \aand \llist(p, e::l, h)~} @\privcon
\end{array}
}
\]
% 
The self/other dichotomy affords explicit access to other-owned
histories, so that we can define the following predicate $\Phi$
stating that other-histories remain empty within the scope of
$\mathsf{hide}$.
\[
\tag{\arabic{tags}}\refstepcounter{tags}\label{eq:phi}
{\small
\hspace{-13pt}
\begin{array}{c}
\Phi(\hist) \eqdef \exists l\ldot \htb \spts ((0 \hpts (l, l)) \hunion
\hist) \aand \htb \opts \hempty \aand W_{\tbcon}
\end{array}
}
\]
%
Inside hide, the stack is initialized (the history contains the
singleton $0 \hpts (l, l)$), there is no interference ($\htb \opts
\hempty$), and the state is a valid one for $\tbcon$ (\ie, it is
captured by the definition~\eqref{eq:tb-states}).

One can prove that if the histories are erased from any state in
$\Phi(\hist)$, the remaining concrete heap consists of $\sent$ and the
stack. Moreover, the contents of the stack is the last entry of
$\hist$ (or $l$ if $\hist$ is empty). In other words, using
$\Psi$~\eqref{eq:hide-rule}, defined in Section~\ref{sec:background}:
\[
\tag{\arabic{tags}}\refstepcounter{tags}\label{eq:phieq}
{\small
\begin{array}{c}
\Psi~\hist~\hempty \iff \exists p~h\ldot \hpriv \spts (\sent \hpts p \hunion h \hunion -) \aand \llist(p, l', h)  
\end{array}
}
\]
where $l' = \hist[\mathsf{last}(\hist)]$ (or $l' = l$ if $\hist$ is
empty).

The derivation is in Figure~\ref{fig:seq-proof}, and we comment on the
main points. In line~2, the right conjunct uses the property inherent
in $\Psi$, that $\Phi(\hempty)$ erases to the heap storing $l$. Thus,
this is the $l$ that appears in the consequent of $\wand$. In line~7,
the second conjunct implies that the history $\hist$, whose existence
obtains from the rule for hiding~\eqref{eq:hide-rule}, must be the
self-history returned by \code{push}. Hence, it is equal to $0 \hpts
(l, l) \hunion t \hpts (l', e::l')$ for some $t$ and $l'$. But, we
also know that $\hist$ must be complete (no gaps between timestamps)
and continuous. Hence $t = 1$ and $l' = l$ in line~9, which derives
the postcondition by~\eqref{eq:phieq}.

\paragraph{A stack client.}
%
We next illustrate how the specs (\ref{eq:stack-spec}) are exploited
by the \emph{concurrent} clients of Treiber stack to abstract from the
fine-grained nature of Treiber's implementation.
%
% and view each effect as if it appeared atomically.
%
The example code in Figure~\ref{fig:client} presents two procedures,
\code{produce} and \code{consume}, that communicate via a common
Treiber stack~$\htb$. \code{produce} pushes onto the stack the
elements of its array \code{ap} in order, whereas \code{consume} pops
from the stack, to fill its array \code{ac}. Both arrays are of equal
size $n$. The procedure \code{exchange} runs \code{produce} and
\code{consume} concurrently. We will prove that after \code{exchange}
terminates, \code{ap} has been copied to \code{ac}, modulo element
permutation. The inference will only use the specs
(\ref{eq:stack-spec}) but not the code of stack methods, thus
obtaining a coarse-grained view of effects provided by histories.

\begin{figure}[t!]
{\centering
%
\begin{tabular}{l@{\ \ }|l@{\ \ }|l}
%
\begin{minipage}[l]{3.4cm}
{\scriptsize
\begin{alltt}
\num{1} produce(n: \(\nat\), i: \(\nat\)) \{
\num{2}  \textbf{if} i == n
\num{3}  \textbf{then return} ();
\num{4}  \textbf{else} \{
\num{5}   e <- ap[i];
\num{6}   push\(\sb{\htb}\)(e);
\num{7}   produce(i + 1);
\num{8}  \}
\num{9} \}

\end{alltt} 
}
\end{minipage}
& 
%
\begin{minipage}[l]{3.5cm}
{\scriptsize
\begin{alltt}
\num{ 1} consume(n: \(\nat\), i: \(\nat\)) \{
\num{ 2}  \textbf{if} i == n
\num{ 3}  \textbf{then return} ();
\num{ 4}  \textbf{else} \{
\num{ 5}   r <- pop\(\sb{\htb}\)();
\num{ 6}   \textbf{if} r == \textsf{Some} e
\num{ 7}   \textbf{then} \{
\num{ 8}    ac[i] := e;
\num{ 9}    consume(i + 1);\}
\num{10}   \textbf{else} consume(i);\}\}
\end{alltt} 
}
\end{minipage}
&
\begin{minipage}[l]{3.7cm}
{\scriptsize
\begin{alltt}
\num{ 1} exchange(n: \(\nat\)): \Unit \{ 
\num{ 2}  \textsf{hide}\(\sb{\Phi,\hempty}\) \{ 
\num{ 3}    produce(n, 0); \(||\) consume(n, 0); 
\num{ 4}  \}
\num{ 5} \} 





\end{alltt}
}
\end{minipage}
%
\end{tabular}
}
\caption{A parallel stack-based producer/consumer program.}
\label{fig:client}
\end{figure}
%

We use several auxiliary predicates. First, $\Array{a}{n}{l}{h}$
defines an array of size $n$ as a sequence of consecutive pointers in
the heap $h$, starting from pointer $a$, and storing elements of the
list $l$:
%
\[
\tag{\arabic{tags}}\refstepcounter{tags}\label{eq:array}
{\small
\begin{array}{rcl}
\Array{a}{n}{l}{h} & \eqdef & \size{l} = n \aand h = \Hunion{i < n} (a + i)
\hpts l(i)
\end{array}
}
\]
%
Next, the predicates $\mathsf{Pushed}$ and $\mathsf{Popped}$ extract
the lists of pushed and popped elements from a stack history $\hist$.
%
\[
\tag{\arabic{tags}}\refstepcounter{tags}\label{eq:pp}
{\small
\begin{array}{rcl}
\Pushed{\hist}{l} & \eqdef & l \eqm \mset{e ~|~
  \exists t~l\ldot t \hpts (l, e::l) \in \hist 
  \oor 0\hpts (l, l) \in \hist \aand e \in l}
\\[3pt]
\Popped{\hist}{l} & \eqdef &  l \eqm \mset{e ~|~ \exists t~ l\ldot t \hpts
  (e::l, l) \in \hist}
\end{array}
}
\]
%
%\hspace{-4pt}
%
The notation $\mset{-}$ stands for multisets, and $\eqm$ is multiset
equality, which we conflate with list equality modulo permutation.
%
We can now ascribe the following specs to \code{produce} and
\code{consume}:
%
\[
\tag{\arabic{tags}}\refstepcounter{tags}\label{eq:pcspecs}
{\small
\begin{array}{c}
\hspace{-5pt}
\spec{
\begin{array}{c}
\Prod(h_p, l_{<i}) \aand 
  \Array{\esc{ap}}{n}{l}{h_p} 
\end{array}
} 
~~\eesc{produce}(n, i)~~
\spec{
\begin{array}{c}
\Prod(h_p, l) \aand 
  \Array{\esc{ap}}{n}{l}{h_p} 
\end{array}
}%@\privcon\entangle(\tbcon \apart \acon)
%
\\[7pt]
%
\spec{
\begin{array}{c}
\exists h_c~l\ldot \Cons(h_c, l_{<i}) \aand \Array{\esc{ac}}{n}{l}{h_c} 
\end{array}
} 
~~\eesc{consume}(n, i)~~
\spec{
\begin{array}{c}
\exists h_c~l\ldot \Cons(h_c, l) \aand \Array{\esc{ac}}{n}{l}{h_c} 
\end{array}
}%@\privcon\entangle\tbcon
\end{array}}
\]
%
%\hspace{-5pt}
%
both over the $\privcon \entangle \tbcon$ concurroid. $\Prod$ and
$\Cons$ are defined as follows:
%
%%\vspace{-3pt}
%
\[
%\tag{\arabic{tags}}\refstepcounter{tags}\label{eq:pcpreds}
{\small
\def\arraystretch{1.3}
\begin{array}{r@{\ }c@{\ }l}
\Prod(h_p, l)&\eqdef&
   \hpriv \spts h_p ~\sep~ \htb \spts \histS \aand 
   \Pushed{\histS}{l}  \aand  \Popped{\histS}{\nil}
\\
\Cons(h_c, l)&\eqdef&
   \hpriv \spts h_c ~\sep~ \htb \spts \histS \aand
   \Pushed{\histS}{\nil}  \aand  \Popped{\histS}{l},
\end{array}
}
%\vspace{-3pt}
\]
%
so they essentially describe the producer/consumer loop invariants;
%
$l_{<i}$ is a prefix of $l$ for elements with indices less than $i$.
%
The specs~\eqref{eq:pcspecs} show that \code{produce} pushes all the
elements from \code{ap}, and \code{consume} fills \code{ac} with
elements of some sequence of the length~$n$. The proofs of both specs
(available in our Coq development) derive easily from
(\ref{eq:stack-spec}) after these are framed to allow running in
arbitrary initial self heap and history.
%
% We omit the proofs here, but provide them in the Coq files.



The interesting part of the example is proving \code{exchange}, where
we compose \code{produce} and \code{consume} in parallel, and then use
hiding to infer that the \code{ap} and \code{ac} arrays in the end
contain the same elements, modulo permutation. The proof outline is in
Figure~\ref{fig:pc-proof}, and it relies on the following important
lemmas about histories.
%
%The following two lemmas hold for histories wrt.  $\mathsf{Pushed}$
%and $\mathsf{Popped}$.
%
%%\vspace{1mm}
%
\begin{lemma}
\label{lm:pushpop1}
%
$\Pushed{\hist_1}{l_1}$ $\aand$ $\Popped{\hist_1}{\nil}$ $\aand$
$\Popped{\hist_2}{l_2} \aand \Pushed{\hist_2}{\nil} \implies
\Pushed{\hist_1 \hunion \hist_2}{l_1} \aand \Popped{\hist_1 \hunion
  \hist_2}{l_2} $.
\end{lemma}
%
\begin{lemma}
\label{lm:pushpop2}
If $\complete(\hist)$ and $\cont(\hist)$ then $\Pushed{\hist}{l_1}
\aand \Popped{\hist}{l_2}$ $\aand$\\ $|l_1|~=~|l_2| \implies l_1 \Eqm
l_2$.
\end{lemma} 
%
%%\vspace{-1mm}

\begin{wrapfigure}{r}{6.9cm}
%\vspace{-10pt}
{\scriptsize
\begin{gather*}
\spec{
\begin{array}{c}
\hpriv \spts h_p \hunion h_c \hunion \sent \hpts \Null \aand 
  \Array{\esc{ap}}{n}{l}{h_p}  \aand
  \Array{\esc{ac}}{n}{-}{h_c}
   \end{array}
}
%@\privcon \entangle \acon
\\[-3pt]
\mathsf{hide}_{\Phi, \hempty}~\esc{\{} 
\\[-3pt] 
\spec{
\begin{array}{c}
\hpriv \spts h_p \hunion h_c \aand \Array{\esc{ap}}{n}{l}{h_p}  \aand
\Array{\esc{ac}}{n}{-}{h_c} ~\sep~\\
\htb \spts 0 \hpts (\nil, \nil) \aand \htb \opts \hempty
\end{array}}
%@\privcon \entangle \acon \entangle \tbcon
%& \text{by \textsc{Hide}}
\\
\!\!\!\!\!\!\!\!\!\!\!\!\!\!\!\spec{\left(
\begin{array}{c}
\hpriv \spts h_p \aand \Array{\esc{ap}}{n}{l}{h_p} \\
~\sep~ \htb \spts 0 \hpts (\nil, \nil)      
\end{array}
\right) \ssep 
\left(
\begin{array}{c}
  \hpriv \spts h_c \aand
  \Array{\esc{ac}}{n}{-}{h_c} \\
~\sep~ \htb \spts \hempty      
\end{array}
\right)}
% & \text{by def. } \ssep
\\
\begin{array}{r||l}
\spec{\Prod(h_p, l_{<0}) \aand
  \Array{\esc{ap}}{n}{l}{h_p}}
&
\spec{\exists l'\ldot~\Cons(h_c, l'_{<0}) \aand
  \Array{\esc{ac}}{n}{l'}{h_c}}
\\
\esc{produce}(n, 0); & \esc{consume}(n, 0); 
\\
\spec{\Prod(h_p, l) \aand
  \Array{\esc{ap}}{n}{l}{h_p}}
&
\spec{\exists h'_c~l'\ldot~\Cons(h_c, l') \aand
  \Array{\esc{ac}}{n}{l'}{h'_c}}
\end{array}
\\
~~~~\spec{\left(
\begin{array}{c}
\Prod(h_p, l) \aand \Array{\esc{ap}}{n}{l}{h_p}
\end{array}
\right) \ssep 
\left(
\begin{array}{c}
\exists h'_c~l'\ldot~\Cons(h_c, l') \aand \Array{\esc{ac}}{n}{l'}{h'_c}
\end{array}
\right)}
\\
\spec{
\begin{array}{c}
\exists h'_c~l'\ldot 
\hpriv \spts h_p \hunion h_c % \hunion \sent \hpts - \hunion -) \aand
\aand
\Array{\esc{ap}}{n}{l}{h_p}
\aand \Array{\esc{ac}}{n}{l'}{h'_c} \\
~\sep~\exists \histS, \htb \spts \histS \aand \Pushed{\histS}{l} \aand \Popped{\histS}{l'}   
\aand \htb \opts \hempty
\end{array}
}~~~\esc{\}}
\\[-4pt]
\spec{
\begin{array}{c}
\exists h'_c~l'\ldot
  \hpriv \spts h_p \hunion h'_c \hunion (\sent \hpts -) \hunion -
  \aand \hbox{}\\
  \Array{\esc{ap}}{n}{l}{h_p}
  \aand \Array{\esc{ac}}{n}{l'}{h'_c} \aand l \eqm l'
\end{array}
}
%@\privcon \entangle \acon
\end{gather*}}
%\vspace{-10pt}
\caption{Proof outline for producer/consumer.}
\label{fig:pc-proof}
\end{wrapfigure}
%
The proof outline in Figure~\ref{fig:pc-proof} starts in the
concurroid $\privcon$, which extends to $\privcon\entangle\tbcon$ in
the scope of $\mathsf{hide}$. The invariant $\Phi$ of $\mathsf{hide}$
is the one we already used, defined in~\eqref{eq:phi}. It introduces a
Treiber stack structure with an initial history $0 \hpts (\nil,
\nil)$. Also, the heaplet $\sent~\hpts~\code{null}$ with the sentinel
pointer has been donated to the state space of the Treiber stack, so
it is removed from the private heap.
%
Next, the self-heap and history are split via $\ssep$; the parts are
given to \code{produce} and \code{consume}, respectively, according to
the parallel composition rule~\eqref{eq:parcom}. Next, we reason out
of specifications~\eqref{eq:pcspecs} for producer/consumer and combine
the subjective views back via $\ssep$ upon joining of the parallel
threads: we thus derive that the contents of \code{ap} and
\code{ac}, are $l$ and $l'$ respectively. By unfolding the definitions of $\mathsf{Pr}$
and $\mathsf{Cn}$, and using Lemma~\ref{lm:pushpop1}, we derive
$\Pushed{\histS}{l} \aand \Popped{\histS}{l'}$, where $\histS$ is the
combined history of \code{produce} and \code{consume}.
%
Finally, $\histS$ is complete and stack-like (since other-history is
provably $\hempty$ thanks to hiding). Moreover, both $l$ and $l'$ have size 
$n$, as ensured by the assertion $\AAArray_n$ constraining both of them.
Thus, in the last assertion, we can use Lemma~\ref{lm:pushpop2} to obtain the 
desired equality of $l$ and $l'$ modulo permutation. 
%
Note also that the sentinel pointer is returned back to the private
heap, along with the garbage heap (existentially abstracted by the $-$
placeholder).


