\section{Conclusion}
\label{sec:conclusion}

In this work we proposed using specifications over auxiliary state in
the form of histories as means of providing general and expressive
specifications for fine-grained concurrent data structures in a
separation style logic.

% We relied on singleton time-stamped histories $t \hpts a$, to specify
% that a program at time $t$ performs an action $a$. The action is
% viewed as \emph{logically} atomic, even though the program may
% implement it in a fine-grained manner. Client programs that reason with
% this spec can treat the program as if it were coarse-grained. Thus, in
% the context of Hoare logic, history-based specs can achieve one of the
% main goals behind linearizability.

Histories satisfy the algebraic properties of PCMs, and thus can
directly reuse the underlying infrastructure from an employed
separation logic, such as its assertion logic and frame rule,
enabling a separation logic style of local reasoning about histories
that has usually been reserved for heaps.
%
Moreover, as we illustrated with the formalization of the flat
combiner Section~\ref{sec:flatco}, the concept of ownership transfer
from separation logic, when specialized to the PCM of histories,
captures the design pattern of helping.
%
%whereby a thread can execute code on behalf of another.
%

In addition to the flat combiner, we have verified a number of
benchmark fine-grained structures, such as the pair snapshot
structure, and the Treiber stack. The novelty of
the specs and the proofs is that they all rely in an essential
way on the subjective dichotomy between self and other auxiliary
state, in order to directly relate the result of a program execution
with the interference of other threads.
%
Such explicit dichotomy provides for what we consider very concise
proofs, as demonstrated by our implementation in Coq.

\paragraph{Acknowledgements.}
%
We thank the anonymous ESOP~2015 reviewers for their feedback. This
research was partially supported by Ramon y Cajal grant RYC-2010-0743.

% We substantiate this observation by both mechanizing metatheory
% and verifying all examples in Coq.

%
%
%\is{Points to stress:}
%
%\begin{itemize}
%
%\item Common approaches for FG specifications are ...
%
%\item We focused on a differen thing - histories
%
%\item By doing so, we got the following insights and benefits (e.g.,
%  the reasoning is similar in spirit to hindsight)
%
%\item The Coq proofs we got are proportional in the size to programs
%  and are conceptually simple
%
%\item Exciting future work is to formalise more patterns
%
%\item Therefore we contributed to general understanding to the ideas
%  behind design and use of concurrent algorithms
%
%\end{itemize}

% The main contribution of our work is an observation that the
% time-stamped histories provide a simple and expressive abstraction to
% capture the behavior of concurrently used data structures. The idea to
% treat such histories as ``strictly growing'' resources allowed us to
% make an effective use of FCSL in order to provide intuitive specs for
% a series of concurrent resources.
% %
% Moreover, the idea of a subjective split of resources to self/other,
% originating from the work on SCSL, made it possible to argue for the
% \emph{principality} of the specs we give as well as gave a very
% straightforward way to reason about other-interference, allowing one,
% in particular to derive sequential specifications out of concurrent
% ones.

% Importantly, we required \emph{no} extensions to the metatheory of
% FCSL in order to construct the reasoning with state histories.

