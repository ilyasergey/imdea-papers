\section{Related and future work}
\label{sec:related}
%% History-based semantics for concurrency
Histories are a recurring idea in the semantics of shared-memory
concurrency, in one form or another. For example, the classical
Brookes' semantics~\cite{Brookes:TCS07} uses \emph{traces} to give a
model for CSL.
%
%Arguing for grainless semantics, Reynolds~\cite{Reynolds:FSTTCS04}
%suggests traces with , similar
%to our self/other combining.
%
Traces are similar to histories, but don't contain time stamps. The
explicit time-stamping makes it straightforward to define a merge
(\ie, join) for histories, and endows them with PCM structure. While
Brookes uses traces in the semantics, we use histories in the specs.

%% Hindsight

Temporal reasoning about shared-memory concurrent programs has also
been employed before. For example, O'Hearn
\etal~\cite{OHearn-al:PODC10} advocate \emph{hindsight lemmas} to
directly and elegantly capture the intuition about linearizability of
a class of concurrent data structures. In this paper, we put histories
to use in ordinary Hoare-style specs. This avoids the relational
reasoning about permuting traces of \emph{two} programs, as required
by linearizability, but is strong enough to provide Hoare logic specs
that are expressive, and capable of abstracting granularity. In our
experience, deriving history-based specs very much resembles reasoning
by hindsight (\eg, verifying \code{locate}~\cite{OHearn-al:PODC10} and
\code{readPair}).

%% Logics with temporal reasoning

HLRG by Fu \etal is a Hoare logic for concurrency that admits
history-based assertions~\cite{Fu-al:CONCUR10}. However, their
histories are hard-coded into the logic. In contrast, our histories
are just a specific PCM, that one can use to instantiate the general
framework of FCSL. This affords greater flexibility: if history-based
specifications are not needed (\eg, the incrementation
example~\cite{Nanevski-al:ESOP14}), they don't have to be used.  HLRG
defines separating conjunction~$\sep$ over histories as follows:
conjoined histories must have equal length, and their corresponding
entry heaps are merged via disjoint union.  In contrast, our histories
are not required to have heaps in the codomain. One can choose an
arbitrary datatype to capture what is important for an example at
hand.

Bell \etal use a variant of concurrent separation logic augmented with
a monoid of \emph{sets of histories} to reason about programs with
asynchronous communication via channels~\cite{Bell-al:SAS10}. Their
logic is tailored for producer/consumer pattern (similar to the
example we have considered in Section~\ref{sec:examples}), and it
features dedicated produce/consume predicates PHist and CHist defined
for a particular channel and a set of histories. However, without
time-stamping, Bell \etal's sets of histories don't enjoy the
unifomity with heaps, hence, they are a subject of a series of
dedicated inference rules.

Gotsman \etal use temporal reasoning to verify several concurrent
memory reclamation algorithms using the notion of \emph{grace
  period}~\cite{Gotsman-al:ESOP13}. Their logic extends
RGSep~\cite{Vafeiadis-Parkinson:CONCUR07} with a very specific notion
of histories, which live in the shared state. In contrast, we use
histories not as shared, but as private auxiliary state, following the
self/other dichotomy. This enables us to directly reuse the frame
rule and other logical infrastructure from the separation logic FCSL,
without any extensions.

%with no possibility of framing with
%respect to them.
%%
%Even though we believe that the notion grace period can be captured
%using our formulation of the histories, in our experience it is not
%significantly more difficult to specify and verify correct memory
%reclamation in FCSL without relying on any sort of temporal reasoning.
%%%
%% \footnote{Our paper-and-pencil proof of the Michael's
%% stack~\cite{Michael:TPDS04} employs heap ownership transfer to
%% specify memory reclamation and can be carried out solely basing on a
%% formulation of the concurroid, and do not require to make statements
%% about histories. Since we did not implement this proof in Coq by the
%% time of submission, we do not present it here, leaving it for future
%% work.}
%%
%Unlike our host logic FCSL, none of the above mentioned
%logics~\cite{Fu-al:CONCUR10,Gotsman-al:ESOP13} has a notion of
%subjective contributions, which prevents them from giving principal
%elementary Hoare-style specifications as well as restricting
%other-interference by means of hiding.

Several recent approaches, such as Turon \etal's
CaReSL~\cite{Turon-al:ICFP13} (which also verifies the flat combiner),
and the logic of Liang and Feng (L\&F)~\cite{Liang-Feng:PLDI13}
support granularity abstraction by unifying Hoare-style reasoning with
linearizability and contextual refinement. 
%
%\ab{Drop LF. It has other connotations.}
%
In contrast, in this paper, we argue that a form of granularity
abstraction achieved by these works can already be obtained
\emph{without} relying on linearizability. Instead, by using
histories, one obtains Hoare-style specs which hide the fine-grained
nature of the underlying programs. This can be done in a simple Hoare
logic (and we reuse FCSL off-the-shelf), whereas CaReSL and L\&F
require significant additional logical
infrastructure~\cite{Liang-al:POPL12,Turon-al:POPL13}, as
linearizability is a stronger property than our specs. One example of
the additional infrastructure has to do with helping (\eg, in the flat
combiner), where these logics consider the refined effectful commands
as resources, and make them subject to ownership
transfer~\cite{Turon-al:ICFP13}.
% 
While on the surface there's a similarity between
commands-as-resources and histories-as-resources, there are also
significant differences. Commands-as-resources are about executing
specification-level programs (and an effectful abstract program, once
executed, can't be ``re-executed'', since it has reached a value),
while histories are about what has transpired. Unlike
commands-as-resources, histories also contain information about the
order in which something happened in the form of timestamps, thus
enabling temporal reasoning by
hindsight~\cite{OHearn-al:PODC10}. Histories have a PCM structure,
whereas commands-as-resources don't. Hence, histories in FCSL are
subject to the same set of inference rules as heaps, in contrast to
commands-as-resources which requires a number of dedicated inference
rules.

% Liang and Feng presented a methodology to verify linearizability by
% means of establishing linearization points in via Hoare-style
% reasoning~\cite{Liang-Feng:PLDI13} by extending a general-purpose
% concurrency logic LRG~\cite{Feng:POPL09}.

Many of our history-based proofs are very close in spirit to proofs of
linearizability (\eg, the proofs of Treiber stack in
Section~\ref{sec:examples} compared to the proofs in
L\&F~\cite{Liang-Feng:PLDI13}), since adding an entry to a self-history
can be seen as linearizing an effectful operation.
%
However, we obtain some simplification in the proofs of pure methods
such as \code{readPair}. In particular, L\&F and related logics require
\emph{prophecy variables}~\cite{Qadeer-al:TR09} (or, equivalently,
\emph{speculations}~\cite{Liang-Feng:PLDI13,Turon-al:POPL13}) in their
proofs of \code{readPair}, but we don't.
%
We do expect, however, that prophecy variables will be required in
examples where the shape of the event to be inserted into the history
can't be fully determined at the moment when it logically takes place
(\eg, Harris \etal's MCAS~\cite{Vafeiadis:PhD}). We plan to address
such examples in the future work, by choosing another history-based
PCM; that of branching-time histories, in contrast to the linear-time
ones used here.
 
In this work, we argued for the abstraction of granularity via the
singleton histories of the form $t \hpts (s_1, s_2)$, which describe
the atomic changes in the abstract state, although other ways are
possible to express what it means for a program to behave ``like an
atomic one'' in a setting of a Hoare-style logic.

In particular, a different approach to express atomicity abstraction
is suggested by da Rocha Pinto \etal's logic
TaDA~\cite{ArrozPincho-al:ECOOP14} (a successor of the Concurrent
Abstract Predicates framework (CAP)~\cite{DinsdaleYoung-al:ECOOP10})
using the notion of an ``atomic Hoare triple'' of the form
$\drspec{p}~c~\drspec{q}$, where the precondition $p$ is required to
be stable, whereas $q$ is not.
% 
TaDA proposes a \emph{make\_atomic} command and a number of related
inference rules, which allow one to specify \emph{synchronized
  changes} of auxiliary resources across \emph{several} shared
regions. The changes themselves don't have to be physically atomic;
it's sufficient that they appear atomic from the point of view of
specs. TaDA's assertions range over \emph{atomic tracking} resources,
similar to the operations-as-resources in the linearizability
proofs~\cite{Liang-Feng:PLDI13,Turon-al:ICFP13}. Unlike histories,
these resources don't have the PCM structure, and thus require special
treatment in TaDA's metatheory. The atomic tracking resources aren't
subject of ownership transfer, which is why TaDA currently doesn't
support reasoning about helping.

Yet another view of atomicity abstraction and canonical concurrent
specifications, which also bypasses linearizability, is advocated by
Svendsen~\etal in a series of papers on Higher-Order and Impredicative
Concurrent Abstract
Predicates~\cite{Svendsen-al:ESOP13,Svendsen-Birkedal:ESOP14}.
%
Both HOCAP and iCAP leverage the idea, originated by Jacobs and
Piessens~\cite{Jacobs-Piessens:POPL11}, of parametrizing specs of
concurrent data types by a user-provided auxiliary code.
%
Such auxiliary code can be seen as a callback, which, when invoked at
some point during the execution of a specified method, changes the
values of auxiliary resources in several regions simultaneously.
%
Thus, when proving a parametrized spec, one should locate a right
moment to invoke the provided auxiliary code, so its precondition
would be ensured and the postcondition handled properly, a reasoning
similar to locating a linearization point.
%
The use of the first-class auxiliary code can introduce circularity in
the domain underlying the logic---the issue tackled in HOCAP by means
of indirection via ``region types'' and resolved in iCAP by providing
a (non-elementary) model in the topos of trees.
%
One difference between iCAP and TaDA is that \emph{make\_atomic} in
TaDA presents a more \emph{localized} view of atomicity, whereas the
specs in iCAP have to predict the uses of the data structure, and
provide hooks for callbacks. The hooks lead to somewhat indirect
specs, and propagate client-side information into the reasoning about
the structure.

We haven't considered either of these two ways of exploiting abstract
atomicity in the current paper, 
%
%
%Had we done it, it would made it possible to replace the non-local
%reasoning based on execution traces in assertions, (\eg, in the proof
%of the stack client in Section~\ref{sec:examples} by means of
%Lemmas~\ref{lm:pushpop1} and~\ref{lm:pushpop2}), by ``more local''
%reasoning in Hoare logic rules themselves (such as TaDA's
%\emph{make\_atomic} or iCAP's \textsc{(Atomic)}). To implement the
%possibility of such a trade-off, we plan to add \emph{make\_atomic} to
%
but plan to add \emph{make\_atomic} to FCSL in the future work. The
challenge will be to generalize \emph{make\_atomic} to work with
different notions of histories (\eg, branching-time histories may be
useful, as mentioned above). We believe that the PCM approach
(together with subjectivity), neither of which is exploited by TaDA
and iCAP, will be beneficial in that respect. In particular, we plan
to use PCMs to generalize the notion of logical atomicity afforded by
histories, that we explored in this paper. Given a PCM $\pcmS$, the
element $x \in \pcmS$ is \emph{prime} if it can't be represented as $x
= x_1 \bullet x_2$, for non-unit $x_1$, $x_2$. For example, in the PCM
of heaps, the prime elements are the singleton heaps. In the PCM of
natural numbers with multiplication, the prime elements are the prime
numbers. In the PCM of histories, the prime elements are the singleton
histories $t \hpts a$. A program can be considered logically atomic if
it augments the self-owned portion of its state by a prime element, or
by a unit. According to this definition, all the examples presented in
this paper are atomic. We expect it should be possible to soundly
apply \emph{make\_atomic} to programs that are atomic in this logical
sense.

