\section{Overview: specifying snapshots with histories}
\label{sec:overview}


In this section, we illustrate history-based specifications by
applying them to the fine-grained \emph{atomic pair snapshot} data
structure~\cite{Qadeer-al:TR09,Liang-Feng:PLDI13}.
%
This data structure contains a pair of pointers, $x$ and $y$, pointing
to tuples $(c_x, v_x)$ and $(c_y, v_y)$, respectively. The components
$c_x$ and $c_y$ of type $A$ represent the accessible contents of $x$
and $y$, that may be read and updated by the client. The components
$v_x$ and $v_y$ are $\mathsf{nat}$s, encoding ``version numbers'' for
$x$ and $y$. They are internal to the structure and not directly
accessible by the client.

\begin{wrapfigure}[9]{r}[0pt]{0.4\textwidth} 
\centering 
%
\begin{tabular}{l@{\ \ \ }l}
% 
\begin{minipage}[l]{4.3cm}
\small
\begin{alltt}
\num{1}  readPair(): \(A {\times} A\)  \{
\num{2}    (cx, vx) <- \aact{readX}();
\num{3}    (cy, _)  <- \aact{readY}();
\num{4}    (_, tx)  <- \aact{readX}();
\num{5}    \textbf{if} vx == tx 
\num{6}    \textbf{then return} (cx, cy);
\num{7}    \textbf{else} readPair();\}
\end{alltt} 
\end{minipage}
&
%
\end{tabular}
%
\caption{Atomic pair snapshot}
\label{fig:readpair}
\end{wrapfigure}
%
The data structure interface exports three methods: \code{readPair},
\code{writeX}, and \code{writeY}. \code{readPair} is the main method,
and the focus of the section. It returns the \emph{snapshot} of the
data structure, \ie, the accessible contents of $x$ and $y$ as they
appear together at the moment of the call. However, while $x$ and $y$
are being read by \code{readPair}, other threads may change them, by
invoking \code{writeX} or \code{writeY}. Thus, a na\"{i}ve
implementation of \code{readPair} which first reads $x$, then $y$, and
returns the pair $(c_x, c_y)$ does not guarantee that $c_x$ and $c_y$
ever appeared together in the structure. One may have \code{readPair}
first lock $x$ and $y$ to ensure exclusive access, but here we
consider a fine-grained implementation which relies on the version
numbers to ensure that \code{readPair} returns a valid snapshot.

The idea is that $\eesc{writeX}(\eesc{cx})$ (and symmetrically,
$\eesc{writeY}(\eesc{cy})$), changes the logical contents of $x$ to
$\eesc{cx}$, while incrementing the internal version number,
\emph{simultaneously}. Since the operation involves changes to the
contents of a single pointer, in this paper we assume that it can be
performed atomically (\eg, by some kind of read-modify-write
operation~\cite[\S5.6]{Herlihy-Shavit:08}). We also assume atomic
operations \code{readX} and \code{readY} for reading from $x$ and $y$
respectively. Then the implementation of \code{readPair}
(Figure~\ref{fig:readpair}) reads from $x$ and $y$ in succession, but
makes a check (line~5) to compare the version numbers for $x$ obtained
before and after the read of $y$. If $x$'s version has changed, the
procedure~restarts.

We want to specify and prove that such an implementation of
\code{readPair} is correct; that is, if it returns a pair $(c_x,
c_y)$, then $c_x$ and $c_y$ occurred simultaneously in the structure.
To do so, we use histories as auxiliary state of every method of the
structure. Histories, ranged over by $\hist$, are finite maps from the
natural numbers to pairs of elements of some type $S$; \ie, $\Hist{S}\
{\eqdef}\ \mathsf{nat} \pfun S \times S$.\footnote{Other sets for
  time-stamping are possible besides $\nat$, as will be mentioned in
  Section~\ref{sec:related}.} The natural numbers represent the
moments in time, and the pairs represent the change of state. Thus, a
singleton history $t \hpts (s_1, s_2)$ encodes an atomic change from
abstract state $s_1$ to abstract state $s_2$ at the time moment
$t$. We will only consider \emph{continuous} histories, for which $t
\hpts (s_1, s_2)$ and $t+1 \hpts (s_3, s_4)$ implies $s_2 = s_3$. We
use the following abbreviations to work with histories:
%
%\vspace{-3pt}
%
\[
\tag{\arabic{tags}}\refstepcounter{tags}\label{eq:hist-not}
{\small
\begin{array}{lcl}
{\hist}[t] & \eqdef & s,~\text{such that}~\exists s',~\hist(t) = (s', s) \\ 
\hist \le t & \eqdef & \forall t' \in \dom{\hist}, t' \le t\\
\hist \pre \hist' & \eqdef & \mbox{$\hist$ is a subset of $\hist'$}
\end{array}
}
\]
%
Similarly to heaps, histories form a PCM under the operation $\hunion$
of disjoint union, with the $\hempty$ history as the unit. The type
$S$ can be chosen arbitrarily, depending on the application, to
capture whichever logical aspects of the actual physical state are of
interest.
%
For the snapshot structure, we take $S = A \times A \times \nat$. That
is, the entries in the histories for pair snapshot will be of the form
%
%\vspace{-3pt}
%
\[
\tag{\arabic{tags}}\refstepcounter{tags}\label{eq:entry}
\small{
t \hpts
(\angled{c_x, c_y, v_x}, \angled{c'_x, c'_y, v'_x}).
}
\] 
%
The entry encodes that at time moment $t$, the contents of $x$, $y$,
and the version of $x$ have changed from $(c_x, c_y, v_x)$ to $(c'_x,
c'_y, v'_x)$. We ignore $v_y$, as it doesn't factor in the
implementation of \code{readPair} (even though it is present for the
sake of symmetry).
%
%\an{I think we should just drop vy
%  from the story completely. That way we don't waste space explaining
%  it, and then saying it does not matter.}
%
%\is{We'd rather not, as it makes the program assymetric and quite
%  ugly, especially, taking into account that the original program
%  from~\cite{Liang-Feng:PLDI13} dealt with a pair snapshot over an
%  arbitrary number of cells (pick any two).}  \an{Well, the whole
%  thing is already assymetric since the histories don't keep track of
%  y, and thus the specs for readX and readY are completely
%  different. But ok, it is not a big deal.}
%

All the threads working over the pair snapshot structure respect a
protocol on histories consisting of the following three properties. We
explain in Section~\ref{sec:background} how these are formally
specified and enforced, but for now simply assume them. They will be
important in the proof outline for \code{readPair}.
%
\begin{itemize}
\item [$(i)$] Whenever a thread modifies $x$ or $y$ (\eg, by calling
  \code{writeX} or \code{writeY}), its history gets augmented by an
  entry such as (\ref{eq:entry}), where the timestamp $t$ is chosen
  afresh. Thus, histories only grow, and only by adding valid
  snapshots (\ie, snapshots corresponding to values of $x$ and $y$,
  \emph{simultaneously} present in the data structure).
\item [$(ii)$] Whenever the contents of $x$ is changed in a history,
  its version number changes too. In contrapositive form, if
  $\tau[t_1] = \angled{c_1, -, v}$ and $\tau[t_2] = \angled{c_2, -,
    v}$, then $c_1 = c_2$.
\item [$(iii)$] Version numbers in a history grow monotonically. That
  is, if $\tau[t_1] = \angled{-, -, v_1}$ and $\tau[t_2] = \angled{-,
    -, v_2}$ and $t_1 \le t_2$, then $v_1 \le v_2$.
\end{itemize}
%
\paragraph{Specification.}
We now describe an FCSL spec for \code{readPair} and explain how it
captures that its result is a valid snapshot of $x$ and $y$.
%
%\vspace{-5pt} 
%
\[
{\small
\begin{array}{c}
%\vspace{-3pt}
\spec{\exists \histO\ldot \hps \spts \hempty \aand \hps \opts \histO
  \aand \hist \pre \histO}
\\
\Cmd{readPair}{} 
\tag{\normalsize \arabic{tags}}\refstepcounter{tags}\label{eq:pair-spec-read}
\\
\spec{\begin{array}{l}
  \exists \histO\ t\ldot \hps \spts \hempty \aand \hps \opts \histO \aand \hist \pre \histO \aand \tau \le t \aand \inat{\angled{\res.1, \res.2, -}}{\histO}{t}
    \end{array}}
%\vspace{-3pt}
\end{array}
}
\]
%
%
First, note the label $\hps$, which serves as an ``abstract pointer''
that differentiates the instance of the pair snapshot structure from
any other structure that may exist in the program. In particular,
$\hps$ identifies the histories of concern to \code{readPair}. Each
thread keeps track of two such histories: the self-history, describing
the operations that the thread itself has executed, and the
other-history for the operations executed by all the other threads
combined. They are captured by the assertions $\hps \spts \tau$ and
$\hps \opts \tau$,~respectively.

Thus, the precondition in (\ref{eq:pair-spec-read}) requires that
\code{readPair} starts with the empty self-history, \ie, the calling
thread has not performed any updates to $x$ or $y$. We show in
Section~\ref{sec:background} that the frame rule can be used to relax
the requirement, so that \code{readPair} can be invoked by threads
with an arbitrary self history.
%
The precondition allows an arbitrary initial
other-history~$\histO$. As $\histO$ is bound locally in the
precondition, we use the logical variable $\hist$ and a conjunct
$\hist \pre \histO$ to propagate the information about $\histO$ into
the postcondition.
%
%and we need to relate to it in the postcondition, we use
%the logical variable $\hist$, and a conjunct $\hist \pre \histO$ to
%``name'' it. 
Because $\hist$ and $\histO$ are related by inclusion, the
precondition is stable under growth of $\histO$ due to interfering
threads, according to $(i)$.
%However, we can't achieve the naming in the usual way, by conjoining
%$\hist = \histO$ to the precondition. Such an equation is unstable,
%because the environment threads can augment the history
%$\histO$. Fortunately, conjoining $\hist \pre \histO$ suffices.

The postcondition states that \code{readPair} does not perform any
changes to $x$ and $y$; it's a \emph{pure} method, thus its
self-history remains empty.
%
%\an{Should I say that according to our criterion of atomicity from
%  Section~\ref{sec:intro}, this makes \code{readPair} an atomic
%  program.}
%
The main novelty of the specification is that the postcondition
directly relates the result of \code{readPair} to the interference of
the environment, \ie, to the value of $\histO$. This is in contrast to
the extant logics, which don't keep track of the \emph{other}
component, and hence can't specify \code{readPair} as directly. In
particular, the postcondition says that $\inat{\angled{\res.1, \res.2,
    -}}{\histO}{t}$, \ie, that the components of the returned pair
$\res$ appear in the environment history. Since according to the
property $(i)$ above, the histories only store valid snapshots, the
resulting pair must be a valid snapshot too. In other words,
\code{readPair} behaves as if it read $x$ and $y$ atomically, at time
$t$.
%
Moreover, $\tau \le t$, \ie, the read occurred after \code{readPair}
was invoked.

The specification pattern whereby a logical variable $\tau$ names the
initial history of the environment is very common, so we streamline it
by introducing the following notation.
\[
\tag{\arabic{tags}}\refstepcounter{tags}\label{eq:histso}
{\small
\begin{array}{r@{\ }c@{\ }l}
\histso{\ell}{\histS, \histO, \hist} \eqdef \ell \spts \histS \aand \ell
\opts \histO \aand \hist \pre \histS \hunion \histO
 \end{array}
}
\]

\paragraph{Proof outline.}
Figure~\ref{fig:pair-proof} contains the proof outline for
\code{readPair}, which we discuss next. The relation $\hist \pre
\histO$ is folded into the definition of $\histso{\hps}{\hempty,
  \histO, \hist}$. Lines 1 and 3 abbreviate the precondition in
(\ref{eq:pair-spec-read}). The \code{readX} method has the following
spec:
%
%
\[
%\tag{\arabic{tags}}\refstepcounter{tags}\label{eq:readx-spec}
{\small
\hspace{-3.5mm}
\begin{array}{r@{\ }c@{\ }l}
\spec{
  \begin{array}{l}
    \histso{\hps}{\hempty, -, \hist}
  \end{array}
} &\Cmd{readX}{} & 
\spec{ 
  \begin{array}{l}
    \exists \histO\ t\ldot \histso{\hps}{\hempty, \histO, \hist} \aand \hist \le t \aand  \inat{\angled{\res.1, -, \res.2}}{\histO}{t}
  \end{array}
}
\end{array}
}
\]
%
Since the ``initial'' other-history is bounded by $\hist$ in the
precondition, and the ``final'' $\histO$ may only grow, we require
$\hist \le t$ in the postcondition to ensure that we won't get a value
from the history, which has ``expired'' \emph{before} the call to
\code{readX}.
%
Thus in line~5 of the proof, we infer the existence of the
history $\hist_1$ and time stamp $t_1 \geq \hist$, such that the
\code{cx} and \code{vx} appear in $\hist_1$ at the time~$t_1$. 
%
Similarly, \code{readY} has the spec:
%\
\[
%\tag{\arabic{tags}}\refstepcounter{tags}\label{eq:ready-spec}
{\small
\begin{array}{r@{\ }c@{\ }l}
\spec{
  \begin{array}{l}
    \histso{\hps}{\hempty, -, \hist}
  \end{array}
} & \Cmd{readY}{} & 
\spec{ 
  \begin{array}{l}
    \exists \histO\ t\ldot \histso{\hps}{\hempty, \histO, \hist} \aand \hist \le t \aand \inat{\angled{-, \res.1, -}}{\histO}{t}
  \end{array}
}
\end{array}
}
\]
%

\begin{wrapfigure}{r}{0.5\textwidth}
%\vspace{-7pt}
\begin{align*}
{\scriptsize
\begin{array}{r@{\ \ \ }l@{\ }l}
 \Num{1} & \spec{~~\histso{\hps}{\hempty, -, \hist}~~} & 
  \\[1pt]
  \Num{2} &  ~\esc{readPair():}~ A \times A~\{ & 
  \\[1pt]
 \Num{3}  & \spec{~~\histso{\hps}{\hempty, -, \hist}~~} & 
  \\[1pt]
 \Num{4} &  ~\esc{(cx, vx) <- \act{readX}();}  & 
  \\[1pt]
 \Num{5} & \spec{
    \begin{array}{l@{\ }l}
     \histso{\hps}{\hempty, \hist_1, \hist} \aand \hist \le t_1 \aand 
     \inat{\angled{\esc{cx}, -, \esc{vx}}}{\hist_1}{t_1}
    \end{array}
  } &
  \\[1pt]
  \Num{6} &  ~\esc{(cy, \_) <- \act{readY}();} & 
  \\[1.5pt]
  \Num{7} & \spec{
    \def\arraystretch{1.2}
    \begin{array}{l}
    \histso{\hps}{\hempty, \hist_2, \hist} \aand \hist \le t_1 \le t_2 \aand \esc{vx} \le v \aand \hbox{} \\
    \inat{\angled{\esc{cx}, -, \esc{vx}}}{\hist_2}{t_1} \aand
    \inat{\angled{c, \esc{cy}, v}}{\hist_2}{t_2}   
    \end{array}
  } & 
  \\[1.5pt]
\Num{8} &  ~\esc{(\_, tx) <- \act{readX}();} & 
  \\[1.5pt]
  %\vspace{1mm}
 \Num{9} & \spec{
    \def\arraystretch{1.2}
    \begin{array}{l@{\ }l}
      \histso{\hps}{\hempty, \hist_3, \hist} \aand \hist \le t_1 \le t_2 \le t_3 \aand \esc{vx}\le v \le \esc{tx} \aand \hbox{}\\
      \inat{\angled{\esc{cx}, -, \esc{vx}}}{\hist_3}{t_1}  \aand
      \inat{\angled{c, \esc{cy}, v}}{\hist_3}{t_2} \aand \inat{\angled{-, -, \esc{tx}}}{\hist_3}{t_3} \\
    \end{array}
  } & 
  \\[1.5pt]
  \Num{10} &  ~\esc{\textbf{if} vx == tx} \\
 \Num{11} & \quad\spec{
    \begin{array}{l@{\ }l}
      \histso{\hps}{\hempty, \hist_3, \hist} \aand \hist \le t_2
      \aand \esc{cx} = c \aand
      \inat{\angled{\esc{cx}, \esc{cy}, v}}{\hist_3}{t_2}
    \end{array}
  } & 
  \\[1.5pt]
  \Num{12} &  \quad~\esc{\textbf{then return}~(cx, cy);} 
  \\[1.5pt]
  \Num{13} & 
  \quad \spec{~~ 
  \exists \histO\ t\ldot \histso{\hps}{\hempty, \histO, \hist} \aand \hist \le t \aand \inat{\angled{\res.1, \res.2, -}}{\histO}{t} ~}
  &
 \\[1.5pt]
  \Num{14} &  ~\esc{\textbf{else} readPair();\}}
  \\[1.5pt]
\Num{15} & \spec{~~
   \exists \histO\ t\ldot \histso{\hps}{\hempty, \histO, \hist} \aand \hist \le t \aand \inat{\angled{\res.1, \res.2, -}}{\histO}{t} ~}
\end{array}
}
\end{align*}
%\vspace{-7pt}
\caption{Proof outline for \code{readPair}.}
\label{fig:pair-proof}
\end{wrapfigure}
%
To obtain line~7, instantiate $\hist$ with $\hist_1$ in the spec of
\code{readY}. This derives the existence of $\tau_2$, $t_2$, $c$ and
$v$, such that $\histso{\hps}{\hempty, \hist_2, \hist_1}$, $\hist_1
\le t_2$, and $\inat{\angled{c, \eesc{cy}, v}}{\hist_2}{t_2}$. Because
$t_1 \in \mathsf{dom}(\hist_1)$, it must be that $t_1 \le t_2$. Moreover,
because $\tau \pre \tau_1 \pre \tau_2$, we further obtain
$\histso{\hps}{\hempty, \hist_2, \hist}$, and $\hist \le t_2$, and
lifting from line 5, $\inat{\angled{\eesc{cx}, -,
    \eesc{vx}}}{\hist_2}{t_1}$. Because $t_1, t_2$ appear in the same
history $\hist_2$, with versions $\eesc{vx}$ and $v$, respectively, by
property $(iii)$, $\eesc{vx} \le v$. Similarly, instantiating $\hist$
in the spec of \code{readX} with $\hist_2$, and invoking $(iii)$,
derives line~9 of the proof outline, and in particular $\eesc{vx} \le v
\le \eesc{tx}$.

From this property, if $\eesc{vx} = \eesc{tx}$ in the conditional on
line~10, it must be that $\eesc{vx} = v$, and thus by $(ii)$, $\eesc{cx} =
c$. Substituting $c$ by $\eesc{cx}$ in line~9 gives us
$\inat{\angled{\eesc{cx}, \eesc{cy}, v}}{\hist_3}{t_2}$, which, after
$(\eesc{cx}, \eesc{cy})$ are returned in $\res$, obtains the
postcondition of \code{readPair}. Otherwise, if $\eesc{vx} \neq
\eesc{tx}$ in the conditional 10, we perform the recursive call to
\code{readPair}. The precondition for the call is
$\histso{\hps}{\hempty, -, \hist}$, which is clearly met in line~9, so
the postcondition immediately follows.

\paragraph{Monolithic histories.}
We compare the spec (\ref{eq:pair-spec-read}) with an alternative spec
where the history is not split into self/other portions, but is kept
monolithically as a \emph{joint} (or shared) state. We use the
predicate $\hps \jpts \tau$ to specify such state:
%
\[
\tag{\arabic{tags}}\refstepcounter{tags}\label{eq:readpair-spec2}
{\small
\begin{array}{c}
\spec{\exists \histO\ldot \hps \jpts \histO \aand \hist \pre \histO}~
\Cmd{readPair}{} 
~\spec{
  \begin{array}{c}
    \exists \histO\ t\ldot \hps \jpts \histO \aand \hist \pre \histO~\aand \hbox{}\\
     \tau \le t \aand \inat{\angled{\res.1, \res.2,
         -}}{\histO}{t} 
  \end{array}
}
\end{array}
}
\]
%
Note that the spec (\ref{eq:readpair-spec2}) imposes no restrictions
on the growth of $\histO$ (unlike (\ref{eq:pair-spec-read}) which
keeps the self history $\hempty$). Thus, (\ref{eq:readpair-spec2}) is
weaker than (\ref{eq:pair-spec-read}), as it allows more behaviors. In
particular, it can be ascribed to any program which, in addition to
calling \code{readPair}, also modifies $x$ and $y$. This substantiates
our claim from Section~\ref{sec:intro} that the self/other dichotomy
is required to prevent history-based specs from losing precision. We
provide further evidence for this claim in Section~\ref{sec:examples},
where we show that subjective specs for \emph{stacks} generalize the
sequential canonical ones~\eqref{eq:seqstack-spec}. The latter can be
derived from the former by restricting $\histO$ to be the empty
history. Such a restriction isn't possible if the history is kept
monolithic.



