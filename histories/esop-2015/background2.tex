\section{Background: a review of FCSL}
\label{sec:background}
In this section we review the relevant aspects of the previous work on
Fine-grained Concurrent Separation Logic
(FCSL)~\cite{Nanevski-al:ESOP14}. We explain FCSL by showing how it
can be specialized to our novel contribution of specifying concurrent
objects by means of histories. FCSL has been previously implemented as
a shallow embedding in Coq; thus our assertions will freely use Coq's
higher-order logic and datatype definition mechanism.
%
%whenever required.

%
%We will focus on several main aspects of FCSL. First, we illustrate
%how FCSL allows encoding and enforcing invariant properties of a
%stateful structure (such as the properties $(i)-(iii)$ in
%Section~\ref{sec:overview}). Second, we illustrate how specifications
%in terms of histories are generalized in FCSL to working with
%arbitrary PCMs. The generality will allow us to compose histories with
%other specification-level structures in the future examples. Third, we
%illustrate how structures in FCSL can be composed into larger ones,
%and the inference rules that support reasoning about composed
%structures.

%\subsection{Basic connectives}
%\label{sec:basic-definitions}
FCSL is a Hoare logic, generalizing CSL, hence its assertions are
predicates on state. But unlike in CSL where state is a heap, in FCSL
state may consist of a number of labeled components (sometimes dubbed
as ``regions'' or ``islands'' in the
literature~\cite{DinsdaleYoung-al:ECOOP10,Svendsen-Birkedal:ESOP14,Turon-al:ICFP13}),
each of which may represent state by a different type. If the type
used by some label is non-heap, then that label encodes auxiliary
state, used for logical specification, but erased at run time. For
example, histories are an auxiliary state identified by the label
$\hps$ in the atomic snapshot example. If we had a program which used
two different atomic snapshot structures, we may label these by
$\hps_1$ and $\hps_2$, \etc.

\subsection{Subjectivity}
The state recorded in labels is further divided across another
orthogonal axis -- ownership. Each label identifies three different
chunks of state: self, joint and other portion. The self portion is
private to the specified thread, and can't be accessed by the other
threads. Dually, other is private to the environment threads, and
can't be accessed by the one being specified. Finally, the joint
section is shared and can be accessed by everyone. The self and other
portions of any given label have to belong to a common PCM (the joint
portion, though, is not required to be a PCM element, as it's not a
subject of a split between threads, as we will see below), and are
often combined together by means of the $\join$ operation of that
PCM. Of course, different labels can use different PCMs, and,
therefore, the points-to assertions are implicitly parametrized with a
PCM type.

The FCSL assertions reflect the division across these axes. We have
already illustrated the assertions $\hlabel\,{\spts}\,v$,
$\hlabel\,{\jpts}\,v$ and $\hlabel\,{\opts}\,v$, which identify the
self/joint/other component stored in the label $\hps$ of the
state. These three basic assertions, constraining only one state
component correspondingly (and leaving the two other unconstrained),
can be, therefore, combined by the usual propositional connectives,
such as $\aand$ and $\oor$, as we have already shown in
Section~\ref{sec:overview}. FCSL further provides two connectives that
generalize the \emph{separating conjunction} $\lsep$ from separation
logic, along the two axes of state splitting. We next illustrate the
\emph{subjective separating conjunction} $\ssep$, and defer the
discussion of the \emph{resource separating conjunction} $\lsep$ until
additional technical material has been introduced. The formal
definitions of all the connectives can be found in
Appendix~\ref{sec:broccoli}.
%
The subjective conjunction $\sstar$ models the division of
state between concurrent threads upon forking and joining. In
particular, the parallel composition rule of FCSL is:
\[
\tag{\arabic{tags}}\refstepcounter{tags}\label{eq:parcom}
{\small
\begin{array}{c}
\stconc{p_1}{c_1}{q_1}{\ucon} \qquad \stconc{p_2}{c_2}{q_2}{\ucon}\\\hline
\stconc{p_1 \ssep p_2}{c_1 \parallel c_2}{q_1 \ssep q_2}{\ucon}
\end{array}
}
\]
Ignoring $\ucon$ and the result types of $c_1$ and $c_2$ for now, we
describe how $\ssep$ works. In this rule, it splits the pre-state of
$c_1 \parallel c_2$ into two parts, satisfying $p_1$ and $p_2$
respectively. The parts contain the same labels, and equal joint
portions, but the self and other portions are recombined to match the
thread-relative views of $c_1$ and $c_2$. 
%
%
Concretely, in the case of
%assertions with 
one label $\hlabel$, with a PCM $\mathbb U$ and values $a, b, c \in
\mathbb U$, we have the following implication.
%
\[
\tag{\arabic{tags}}\refstepcounter{tags}\label{sep-star-impl}
\hspace{-2mm}
{\small
\begin{array}{l}
\hlabel \spts a \join b \aand %\hlabel \jpts h \aand 
  \hlabel \opts c \implies 
  (\hlabel \spts a \aand %\hlabel \jpts h \aand 
   \hlabel \opts b \join c) \ssep 
  (\hlabel \spts b \aand %\hlabel \jpts h \aand 
   \hlabel \opts a \join c)
\end{array}}
\]
%
Thus, if before the fork, the self-state of the parent thread
contained $a \join b$, and the other-state contained $c$, then after
the fork, the children will have self-states $a$ and $b$, and the
other-states $b \join c$ and $a \join c$, respectively.
%
In the opposite direction:
\[\tag{\arabic{tags}}\refstepcounter{tags}\label{sep-star-inverse}
{\small
\begin{array}{l}
(\hlabel \spts a \aand \hlabel \opts c_1) \ssep
(\hlabel \spts b \aand \hlabel \opts c_2) \implies \hbox{}\\
\quad \exists c\ldot c_1 = b \join c \aand c_2 = a \join c \aand
\hlabel \spts a \join b \aand \hlabel \opts c
\end{array}
}\] 
%
That is, if the state can be subjectively split between two child
threads so that their other-views are $c_1$, $c_2$ (with self-views
$a$, $b$), then there exists a common $c$---the other-view of the
parent thread---such that $c_1 = b \join c$ and $c_2 = a \join c$.  In
this sense, the rule for parallel composition models the important
effect that upon a split, $c_1$ becomes an environment thread for
$c_2$, and vice-versa.

There are a few further equations that illustrate the interaction
between the different assertions. First, every label contains all
three of the self/joint/other components. Thus:
%
\[\tag{\arabic{tags}}\refstepcounter{tags}\label{spts-opts}
{\small
 \hlabel \spts a \iff \hlabel \spts a \aand \hlabel \jpts - \aand \hlabel \opts -
}\]
%
and similarly for $\hlabel \jpts a$ and $\hlabel \opts a$. Also:
%
%We commonly encounter assertions of the form $\hlabel \spts a \aand
%\hlabel \opts -$, where the other-value is existentially
%abstracted. These can be simplified into $\hlabel \spts a$, and in
%those cases, (\ref{sep-star-impl} and \ref{sep-star-inverse}) reduce
%into a bi-implication:
%We commonly encounter cases when $c$ is existentially abstracted (and
%hence the conjunct $\hlabel \opts c$ is omitted), in which case, we
%use the form:
\[
\tag{\arabic{tags}}\refstepcounter{tags}\label{biimpl}
{\small
%\begin{array}{r@{\ }c@{\ }l}
%\hlabel \spts a \join b \aand \hlabel \opts c & \implies & (\hlabel \spts a
%\aand \hlabel \opts c \join b) \ssep (\hlabel \spts b \aand \hlabel \opts c \join a)\\
\hlabel \spts a \join b \iff \hlabel \spts a \ssep \hlabel \spts b
%\end{array}
}\]
which is provable from (\ref{sep-star-impl}), (\ref{sep-star-inverse})
and (\ref{spts-opts}).


%Both (\ref{sep-star-impl}) and (\ref{biimpl}) generalize in the
%obvious way to $\sstar$-separated assertions with more than one label.

FCSL also provides a \emph{frame rule}, obtained as a special case of
parallel composition when $c_2$ is the idle thread, and $p_2 = q_2 =
r$ is a stable predicate, as usual in fine-grained
logics~\cite{Feng:POPL09,Vafeiadis:PhD,DinsdaleYoung-al:ECOOP10}.
%
\[
\tag{\arabic{tags}}\refstepcounter{tags}\label{eq:frame}
\begin{array}[m]{c}
  \stconc{p}{c}{q}{\ucon}\\\hline
  \stconc{p \ssep r}{c}{q \ssep r}{\ucon}
\end{array} \quad \mbox{$r$ stable under $\ucon$}
\]

We illustrate the frame rule by deriving from the \code{readPair}
spec~(\ref{eq:pair-spec-read}) a relaxed spec which allows
\code{readPair} to apply when the calling thread has non-trivial self
history $\histS$:
\[
\tag{\arabic{tags}}\refstepcounter{tags}\label{eq:pair-spec-read-framed}
{\small
\hspace{-3mm}
\begin{array}{r@{\ }c@{\ }l}
\spec{~ \histso{\hps}{\histS, -, \hist} ~} 
&\Cmd{readPair}{} & 
\spec{\begin{array}{l}
  \exists \histO\ t\ldot
    \histso{\hps}{\histS, \histO, \hist} \aand \tau \le t \aand \hbox{}\\
    \quad \inat{\angled{\res.1, \res.2, -}}{(\histS \hunion \histO)}{t}
  \end{array}
}
\end{array}
}
\]
Note that (\ref{eq:pair-spec-read-framed}), when compared to
(\ref{eq:pair-spec-read}), changes the self component from $\hempty$
to $\histS$, but also ${\histO}[t]$ changes into ${(\histS \hunion
  \histO)}[t]$. The latter accounts for the possibility that the
returned snapshot may have been recorded in $\histS$ as a consequence
of the thread itself changing $x$ or $y$, immediately before invoking
\code{readPair}.

The spec (\ref{eq:pair-spec-read-framed}) derives from
(\ref{eq:pair-spec-read}) by framing with the predicate $r = \hps
\spts \histS$. $r$ is trivially stable, as it describes self-state,
which is inaccessible to the interfering threads. We only show how to
weaken the framed postcondition of (\ref{eq:pair-spec-read}) to the
postcondition in (\ref{eq:pair-spec-read-framed}); the preconditions
can be strengthened similarly. Abbreviating $\hist \pre \histO \aand
\hist \le t \aand \inat{\angled{\res.1, \res.2, -}}{\histO}{t}$ by
$P(\histO)$, which is a label-free (\ie, pure) assertion, and thus
commutes with $\ssep$, we get:
%\[
%{\small
%\begin{array}{l}
%(\hps \spts \hempty \aand \hps \opts \histO \aand \hist \pre \histO \aand \hist \le t \aand 
% \inat{X}{\histO}{t}) \ssep (\hps \spts \histS) \implies \hbox{}\\
%\quad\mbox{by (\ref{spts-opts}) and commuting out label-free assertions}\\
%(\hps \spts \hempty \aand \hps \opts \histO) \ssep (\hps \spts \histS \aand \hps \opts -) \aand \hbox{}\\
%\hspace{3cm}\hist \pre \histO \aand \hist \le t \aand  \inat{X}{\histO}{t} \implies \mbox{by (\ref{sep-star-inverse})}\\
%\exists \histO'\ldot \histO = \histS \hunion \histO' \aand 
%\hps \spts \histS \aand \hps \opts \histO' \aand \hist \pre \histO \aand \hist \le t \aand \inat{X}{\histO}{t} \implies \hbox{}\\
%\quad\mbox{by substituting $\histO$}\\
%\exists \histO'\ldot \histso{\hps}{\histS, \histO', \hist} \aand \hist \le t \aand \inat{X}{(\histS \hunion \histO')}{t}.
%\end{array}
%}
%\]
\[
{\small
\begin{array}{l}
(\hps \spts \hempty \aand \hps \opts \histO \aand P(\histO)) \ssep (\hps \spts \histS) \implies \mbox{by (\ref{spts-opts}) and $P$-pure}\\
(\hps \spts \hempty \aand \hps \opts \histO) \ssep (\hps \spts \histS \aand \hps \opts -) \aand P(\histO) \implies \mbox{by (\ref{sep-star-inverse})}\\
\exists \histO'\ldot \histO = \histS \hunion \histO' \aand 
\hps \spts \histS \aand \hps \opts \histO' \aand P(\histO) \implies \mbox{by substituting $\histO$}\\
\exists \histO'\ldot \histso{\hps}{\histS, \histO', \hist} \aand \hist \le t \aand \inat{\angled{\res.1, \res.2, -}}{(\histS \hunion \histO')}{t}.
\end{array}}
\]
Intuitively, in~\eqref{eq:pair-spec-read-framed} the frame history
$\histS$ is ``subtracted'' from the other-history $\histO$ of
(\ref{eq:pair-spec-read}), and moved to the self-history, illustrating
one important difference between the frame rule of FCSL and that of
CSL. In FCSL, the frame is always subtracted from the other component,
whereas in CSL it simply materializes out of nowhere. On the flip
side, CSL doesn't consider the other component, and can't easily
express a spec such as~\eqref{eq:pair-spec-read}.
%At the same time, the result $\histO'$ of the subtraction is
%existentially quantified.

\subsection{Concurroids}
\label{sec:concurroids}

We now turn to the component $\ucon$ of the FCSL specs, which is
called \emph{concurroid}. Concurroids are responsible for enforcing
the invariants on the evolution of the state. For example, the
properties $(i)${--}$(iii)$ in Section~\ref{sec:overview} will be
enforced by defining an appropriate concurroid to govern the
pair-snapshot structure. Thus, concurroids formally represent
concurrent data structures, over which the programs operate.

A concurroid is (a form of) a state transition system (STS). It's a
quadruple $\ucon = ({L}, W, I, {E})$ where: (1) $L$ is a set of
labels, identifying different data structures; (2) $W$ is a set of
admissible states (alternatively, an FCSL assertion); (3) $I$ is the
set of \emph{internal transitions} on $W$; (4) $E$ is a set of pairs
$(\alpha, \rho)$, where $\alpha$ is a \emph{heap-acquiring} and $\rho$
is a \emph{heap-releasing} transition, collectively called
\emph{external} transitions. The internal transitions are relations on
states, describing how a state of the STS evolves in one atomic
step. The external transitions serve for transfer of state
ownership. The concurroids thus bound the moves of the concurrent
programs that operate on a data structure, and therefore represent a
structured form of rely/guarantee transitions from Rely/Guarantee
logics~\cite{Feng-al:ESOP07,Vafeiadis:PhD,Jones:IFIP83,Feng:POPL09,Vafeiadis-Parkinson:CONCUR07}. We
next illustrate concurroids by example.

\paragraph{Pair-snapshot concurroid.}
Given a label $\hps$, pointers $x$, $y$, and the type $A$ of the
accessible contents of $x$ and $y$, the concurroid for the
pair-snapshot structure is ${\pscon} = (\{\hps\}, W_{\pscon}, \{wr_x,
wr_y, \id\}, \emptyset)$.
%
The set of states $W_{\pscon}$ is described below. We assume that
$\histS, \histO$ are histories, $c_x, c_y\,{:}\,A$ and $v_x,
v_y\,{:}\,\mathsf{nat}$, and are implicitly existentially quantified.
%
%\vspace{-5pt}
%
\[
{\small
\begin{array}{l@{\ }c@{\ }l}
W_{\pscon} & \eqdef & \hps \spts \histS \aand \hps \jpts (x \hpts (c_x, v_x) \hunion y \hpts (c_y, v_y)) \aand \hps \opts \histO \aand \hbox{}\\
    & & \mbox{$\histS$, $\histO$ satisfy $(ii)-(iii)$}, \mbox{$\histS \hunion \histO$ is continuous, and}\\
    & & \mbox{if $t = \last{\histS \hunion \histO}$, then $(\histS \hunion \histO)[t] = (c_x, c_y, v_x)$}
\end{array}
}
\]
%
A state in $W_{\pscon}$ consists of the auxiliary part, which are
histories in the self and other components, and concrete part, which
is a joint heap, storing pointers $x$ and $y$, with accessible
contents $c_x, c_y$, and version numbers $v_x, v_y$,
respectively.\footnote{Notice the overloading of the $\hpts$ notation
  for singleton heaps and histories.}
%
It requires several additional properties of the auxiliary
histories. First, the combined history $\histS \hunion \histO$ is
continuous; that is, adjacent timestamps have matching states. Second,
the last timestamp in $\histS \hunion \histO$ correctly reflects
what's stored in $x$ and $y$. Finally, $W_{\pscon}$ also bakes in the
properties $(ii)-(iii)$ required in the proof outline of
\code{readPair}, so the specification~\eqref{eq:pair-spec-read} and
its proof were, in fact, carried out in the concurroid context
$@\pscon$, which was omitted.

The internal transitions $wr_x$ and $wr_y$ synchronize the changes to
$x$ and $y$ with histories. The transitions operate only on self and
joint portions of the state, and the other-portion, $\histO$, is fixed
(\cf\ notation (\ref{spts-opts})). That is, the transitions
essentially define the concurroid's Guarantee. In both transitions,
$\tfr{\histS \hunion \histO}$ is the smallest timestamp unused by
$\histS$ and~$\histO$.
%
%\vspace{-5pt}
%
\[ 
{\small 
  \begin{array}{l@{\ }c@{\ }l@{\ }c}
    wr_x & \eqdef &\hps \jpts (x \hpts (c_x, v_x) \hunion y
    \hpts (c_y, v_y))  \aand 
    \hps \spts \histL ~~~~~\rightsquigarrow \\
    & &
    \hps \jpts (x \hpts (c'_x, v_x + 1) \hunion y \hpts (c_y, v_y) \aand \hps \spts \histL \hunion \tfr{\histS \hunion \histO} \hpts (\angled{c_x, c_y, v_x},\angled{c'_x, c_y, v_x+1})
    \\
    wr_y & \eqdef & \hps \jpts (x \hpts (c_x, v_x) \hunion y
    \hpts (c_y, v_y)) \aand \hps \spts \histL ~~~~~\rightsquigarrow \\
    && \hps \jpts (x \hpts (c_x, v_x) \hunion y \hpts (c'_y, v_y
    +1) \aand \hps \spts \histL \hunion \tfr{\histS \hunion \histO} \hpts (\angled{c_x, c_y, v_x},\angled{c_x, c'_y, v_x})
    \end{array}
}\] 
%
The first conjunct after $\rightsquigarrow$ in $wr_x$ (and $wr_y$ is
similar) allows that the version number of $x$ can only increase by 1
in an atomic step. The second conjunct shows that simultaneously with
the change of $x$, the snapshot of the changed state is committed to
the self-history of the invoking thread.
%
Together, $wr_x$ and $wr_y$ ensure that histories only grow, and only
by adding valid snapshots; \ie, precisely the property $(i)$ from
Section~\ref{sec:overview}.
%

$\ucon$ also contains the identity transition $\id$, whose presence
enables programs that don't modify the state at all. In the
pair-snapshot example, these are the \code{readX} and \code{readY}
actions, and the \code{readPair} method.  The pair-snapshot example
doesn't involve ownership transfer, so $\pscon$ has no external
transitions, but these will be important in the forthcoming examples.

\paragraph{Entanglement and private heaps.} 
Larger concurroids may be constructed out of smaller ones. A
particularly common construction is
\emph{entanglement}~\cite{Nanevski-al:ESOP14}. Given concurroids
$\ucon$ and $\vcon$, the entanglement $\ucon \entangle \vcon$ is a
concurroid whose state space is the Cartesian product $W_{\ucon}
\times W_{\vcon}$, and the transitions allow the $\ucon$ portion to
perform a $\ucon$ transition, while the $\vcon$ portion remains idle,
and vice-versa. Additionally, $\ucon$ and $\vcon$ portions can
communicate to \emph{transfer a heap} between themselves, by having
one take a heap-acquiring, and the other \emph{simultaneously} taking
a heap-releasing transition.

The most common is the entanglement with the concurroid $\privcon$ of
\emph{private heaps} (see
Appendix~\ref{sec:conc-priv-heaps}). Entangling with $\privcon$ lets
the concurroids temporarily move heaps to a private section, via the
communication discussed above, where threads may then perform the
customary operations of reading, writing, allocating, and deallocating
pointers, without interference.\footnote{Our Coq proofs actually use
  two different concurroids, one for reading/writing, another for
  allocation/deallocation, which we entangle to provide all four
  operations. For simplicity, here we assume a monolithic
  implementation.}
%
$\privcon$ comes with a dedicated label $\hpriv$. As an illustration,
the following assertion may describe one possible state in the state
space of the entanglement $\privcon \entangle \pscon$ with the
snapshot concurroid.
%
%\vspace{-8pt}
%
\[
{\small
\begin{array}{c}
\hpriv \spts (z \hpts 0) \lsep \hps \jpts (x \hpts (c_x, v_x) \hunion
y \hpts (c_y, v_y))
%\vspace{-7pt}
\end{array}
}
\]
%
%%\vspace{-5pt}
%
The $\hps \jpts -$ portion describes the part of the state coming from
$\pscon$, which is joint, containing pointers $x$ and $y$, as
explained before. The $\hpriv \spts (z \hpts 0)$ describes the part of
the state coming from $\privcon$. In this case, it contains a heap
with a single pointer $z$. The heap is private, \ie, owned by the self
thread, so $z$ can't be modified by other threads.
%
Notice that the assertions about $\hpriv$ and $\hps$ are separated by
the resource separating conjunction $\lsep$, which splits the state
into portions with disjoint labels and heaps. In this particular case,
it signifies that the labels $\hpriv$ and $\hps$ are distinct, as are
the pointers $z$, $x$~and~$y$.

%
%Moreover, heaps can be transfered to $\privcon$ from the other
%components of the entanglement, thus becoming private, or out of
%$\privcon$, when the private status of the heap is not desired
%anymore.

\newcommand{\inject}[1]{[#1]}

\subsection{Extending and hiding concurroids}
Concurroids represent concurrent data structures; thus it's important
to be able to introduce and eliminate them. FCSL provides two
programming constructors (both no-ops operationally), and
corresponding inference rules for that purpose. For completeness, we
introduce them here, but postpone the illustration until
Section~\ref{sec:examples}.

The injection rule shows that if a program is proved correct with
respect to a smaller concurroid $\ucon$, then it can be extended to
$\ucon \entangle \vcon$, without invalidating the proof.
%
\[
\tag{\arabic{tags}}\refstepcounter{tags}\label{eq:inject}
\begin{array}{c}
 \stconc{p}{c}{q}{\ucon}\\\hline
 \stconc{p \lsep r}{\inject c}{q \lsep r}{{\ucon} \entangle {\vcon}}
\end{array} \quad \mbox{$r \subseteq W_{\vcon}$ stable under $\vcon$}
\]
%
This is a form of framing rule, along the axis of adding new
resources. The operator $\lsep$ splits the state into portions with
disjoint labels, and the side-condition that $r \subseteq W_{\vcon}$
forces $r$ to remove the labels of the concurroid $\vcon$, so that $c$
is verified \wrt~the labels of $\ucon$. The program constructor
$\inject -$ is a coercion from $\ucon$ to ${\ucon} \entangle {\vcon}$.

Hiding is the ability to introduce a concurroid $\vcon$, \ie, install
it in a private heap, for the scope of a thread $c$. The children
forked by $c$ can interfere on $\vcon$'s state, respecting $\vcon$'s
transitions, but $\vcon$ is hidden from the environment of $c$, To the
environment, $\vcon$'s state changes look like changes of the private
heap of $c$. Upon termination of $c$, $\vcon$ is deinstalled.
%
\begin{gather*}
{\small
\begin{array}{c}
{\stconc{\hpriv\spts h \lsep p}{c}{\hpriv\spts h' \lsep q}{(\privcon
    \entangle {\ucon}) \entangle {\vcon}}} 
\\ \hline
{\stconc{\Psi\ g\ h \lsep (\Phi\,(g) \wand p)}{\mathsf{hide}_{\Phi, g}\ c}{\exists g'. \Psi\ g'\ h' \lsep (\Phi\,(g') \wand q)}{\privcon \entangle \ucon}}
\end{array}
}
\\
\tag{\normalsize\arabic{tags}}\refstepcounter{tags}\label{eq:hide-rule}
{\small
\mbox{where}\ \Psi\ g\ h = \exists k{:}\mathsf{heap}.\, \hpriv \spts {h \hunion k} \aand \Phi\,(g)\ \mbox{erases to}\ k
}
\end{gather*}

%\[
%\begin{array}[t]{c}
%  {\stconc{\hpriv\spts h_1 * r * \Phi\ g_1}{C}{\hpriv\spts h_2 * q * \Phi\ g_2}{P \entangle (U \apart V)}}\\\hline
%  {\stconc{\hpriv \spts {h_1 \hunion x_1} * r \aand \Phi g_1 \downarrow x_1}{\mathsf{hide}_{\Phi} C}{\hpriv \spts {h_2 \hunion x_2} * q \aand \Phi g_2 \downarrow x_2}{P \entangle U}}
%\end{array}
%\]
%\an{Here's where the point-free notation may look attractive. We can
%  define $\Psi$ simply as $\mathsf{this}\ h \lsep_\pi \Phi\
%  g\downarrow$ and save one existential.}  
%
%
Since installing $\vcon$ consumes a chunk of private heap, the rule
requires the overall concurroid to support private heaps, \ie, to be
an entanglement of $\privcon$ with an arbitrary $\ucon$. In programs,
we use the coercion $\mathsf{hide}\ c$ to indicate the change from
$(\privcon \entangle {\ucon}) \entangle {\vcon}$ to $\privcon
\entangle \ucon$. If $\ucon$ is of no interest, one can take it to be
the empty concurroid $\econ$, which is a right unit for $\entangle$
(see Appendix~\ref{sec:empty-concurroid}).

The annotation $\Phi$ is a predicate; it describes an invariant that
holds within the scope of $\mathsf{hide}$, parametrized by an
argument. It's subject to a number of conditions (see
Appendix~\ref{sec:phi-properties}). $g$ is the initial argument, so
$\Phi(g)$ holds in the initial state into which $\vcon$ is placed upon
installation. The rule guarantees that the ending state of $c$
satisfies $\exists g'\ldot \Phi(g')$. The surrounding connectives
$\lsep$ and $\wand$ merely mediate between $\ucon$, $\vcon$, and the
erasure of $\vcon$ to heaps. We explain the precondition, and the
postcondition is similar.

In the precondition, $*$ separates private heaps from $\ucon$, and
$\Psi$ requires that every state in $\Phi(g)$ obtains the same private
heap when the auxiliary fields are erased. $\wand$ is inherited from
separation logic. $\Phi(g) \wand p$ says that if the initial state
(which is in $W_{\ucon}$) is extended with a state from $\Phi(g)$
(which is in $W_{\vcon}$), then the result is a state satisfying
$p$. In other words, if a state satisfying $\Phi(g)$ is installed in
the initial state of $c$, while its heap footprint is removed from the
private heaps, then $c$'s precondition is satisfied.

% In the formal FCSL reasoning, all ``relevant'' concurroids, whose
% state is being affected or examined by the program being verified,
% should be mentioned explicitly and concretely, similarly to
% Rely/Guarantee-style reasoning. The ``irrelevant'' concurroids can be
% (and usually are) "framed away" by means of the
% rule~\eqref{eq:inject}.
