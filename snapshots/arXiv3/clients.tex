\section{Client reasoning}
\label{sc:clients}

\paragraph*{Comparison with linearizability specifications.}

In linearizability one would specify \jywrite\ and \jyscan\ by
relating them, via a simulation argument, to sequential programs for
writing and scanning, respectively. On the face of it, such specs are
simpler than ours above, as they merely state that \jywrite\ writes
and \jyscan\ scans. Our specs capture this property with one conjunct
in each postcondition. The remainders of the postconditions describe
the relative order of the atomic events, including explicit
prohibition on reordering non-overlapping events, which is itself
inherent in the definition of linearizability.

However, the additional specifications are not pointless, and they
become useful when it comes to reasoning about
clients. Linearizability tells us that we can simplify a fine-grained
client program by replacing the occurrences of \jywrite\ and
\jyscan\ with the atomic and sequential equivalents, thus turning the
client into an equivalent coarse-grained concurrent program. However,
linearizability is not directly concerned with verifying that
coarse-grained equivalent itself. One has to develop separate methods
for such client reasoning, which may very well re-introduce histories
and orderings into the logic.

On the other hand, if one wants to reason about clients using Hoare
logic, then our specs are directly useful. For example, in the sequel,
we illustrate how to derive interesting client timing properties out
of the specs of \jywrite\ and \jyscan.

Moreover, because we use separation logic, our approach easily
supports reasoning about programs with a dynamic number of threads,
and about programs that transfer state ownership. In fact, as we
already commented in Section~\ref{sc:formal}, our proofs rely on
transferring write events from $\histJ$ (joint ownership) to $\histS$
(private ownership), upon the write's termination. In contrast,
linearizability is usually considered for a fixed number of threads,
and has not yet been fully reconciled with programs that perform
ownership transfer~\cite{Cerone-Gotsman-Yang+ICALP14}.

An additional benefit of specifying the event orders by Hoare triples
at the user level, is that one can freely combine methods with
different event-ordering properties, that need not respect the
constraints of linearizability~\cite{SergeyNBD+OOPSLA16}.

\paragraph*{Example clients.}

We first consider the client $e$, defined as follows:
%
\[
\begin{array}{l||l||l}
        \begin{array}{c}
         \esc{write}~(x,2);\\ 
         \esc{write}~(y,1)~ 
        \end{array}~
& ~\esc{scan}~()~ &~\esc{write}~(x,3)  
\end{array}
\]
%
It is our running example from Figure~\ref{fig:weird:code}. We will
show that it satisfies the spec below (omitting the STS $C$ from now
on, as it never changes).
%
%\begin{figure}[t]
{\small
\[
\!\!\!\begin{array}{l}
e : 
  \tsPre{\{ \histS = \hempty \}} \
\begin{array}[t]{l}
  \tsPos{\{ r\ldot\, \exists t_1\, t_2\, t_3\, t_s \ldot\, 
    \histS' = t_1 \hpts (y,1) \hunion t_2 
    \hpts (x,2) \hunion t_3 \hpts (x,3) \wedge \hbox{}}\\
%    \hist \subseteq \histP \wedge 
%    \stableorder \subseteq \stableorderP \wedge}\\
  \tsPos{\hphantom{\{ r\ldot\,\ \ }%\exists t_1\, t_2\, t_3\, t_s \ldot}\, 
  \!\!\! \dom{\hist} \subseteq \ideal{\stableorderP}{t_s} \wedge
         \dom{\histO} \subseteq \sideal{\stableorderP}{t_2}, \sideal{\stableorderP}{t_3} \wedge 
%
%    \histO \subseteq \sideal{\stableorderP}{t_3} \wedge}\\
%  \tsPos{\hphantom{\{ r\ldot\, \exists t_1\, t_2\, t_3\, t_s \ldot}\, 
    t_2\ {\stableorderP}\ t_1 \wedge \hbox{}}\\
   \tsPos{\hphantom{\{ r\ldot\,\ \ }
%    \hist \subseteq \ideal{\stableorderP}{t_s} \wedge 
   \!\!\! r = \eval\ t_s\ \stableorderP\ \histP \}}\\
\end{array}
\end{array}
\]}


The spec says that (1) $\esc{write}~(x,2)$, timestamped $t_2$, occurs
sequentially before $\esc{write}~(y,1)$ which is timestamped $t_1$,
(2) the remaining write, timestamped $t_3$, and the scan, timestamped
$t_s$, are not temporally constrained, and (3) the writes that
terminated before the client started are ordered before $t_2$ (and
thus before $t_1$), $t_3$ and $t_s$. The example illustrates how
to track timestamps and their order, but does not utilize
$\scanned{\stableorder}$. We illustrate the latter in another example.

We verify the subprograms $\esc{scan}() \parallel \esc{write}~(x,3)$
and $\esc{write}~(x,2);\ \esc{write}~(y,1)$ separately, and then
combine them into the full proof. As proof outlines show intermediate,
in addition to pre- and post-state, we cannot quite utilize VDM
notation in them. As a workaround, we explicitly introduce logical
variables $h$ and $h_o$ to name (subsets of) the initial global and
other history.
%
%
{\small
\[
\!\!\!\begin{array}{l l} & \num{1} \hfill
 {\specK{\{ \histS = \hempty \wedge h \subseteq \hist \wedge h_o
                           \subseteq \histO \}}} \hfill \\
% \num{2} & \specK{\{ ( \histS = \hempty) \mathrel{\circledast} ( \histS = \hempty \wedge h = \hist \wedge w = \stableorder)\}}\\ 
& \!\!\!\! \begin{tabular}{l || l}
     \begin{array}[t]{l l}
       \num{2a} &
       \hfil \specK{\{ \histS = \hempty \wedge h \subseteq \hist \wedge h_o \subseteq \histO \}}\hfil\\
       \num{3a} & \multicolumn{1}{c}{\esc{scan}~()} \\
       \num{4a} &
       \specK{\{ r\ldot\, \exists\, t_s\ldot
         \histS = \hempty \wedge  \dom{h} \subseteq \ideal{\stableorder}{t_s} \wedge \hbox{}}\\
       &
       \specK{\phantom{\{r\ldot\, \exists\, t_s\ldot}\,
         r = \eval\, t_s\, \stableorder\, \hist\}}
     \end{array}~&~
     \!\!\!\begin{array}[t]{r l}
       \num{2b} &
       \specK{\{ \histS = \hempty \wedge h \subseteq \hist \wedge h_o \subseteq \histO \}} \\       
       \num{3b} & \multicolumn{1}{c}{\esc{write}~(x,3)}\\
       \num{4b} &
       \specK{\{ \exists\, t_3\ldot
         \histS = t_3 \hpts (x,3) \wedge \hbox{}}\\
       & \specK{\phantom{\{\exists\, t_3\ldot}\,
         \dom{h_o} \subseteq \sideal{\stableorder}{t_3}\}}
     \end{array}
   \end{tabular}\\
%  \num{6} & \specK{\{ r\ldot}\\
%          & \specK{\phantom{\{}
%          (\exists t_s\ldot
%         \histS = \hempty\! \wedge\! h \subseteq \hist \wedge\!
%         w \subseteq\! \stableorder \wedge\!
%         h \subseteq\! \ideal{\stableorder}{t_s} \wedge\!
%         t_s \in \scanned\, \stableorder \wedge\!
%         r = \eval\, t_s\, \stableorder\, \hist)}\\
%   & \specK{\phantom{\{}~~~
%     \mathrel{\circledast}
%      (\exists\, t_3\ldot
%         \histS = t_3 \hpts (x,3) \wedge
%         h \subseteq \hist \wedge
%         w \subseteq \stableorder \wedge
%         \dom{h_{\othersub}} \subseteq \sideal{\stableorder}{t_x})\}}\\
  &  \num{5} \hfill {\specK{\{ r\ldot\, \exists t_3\, t_s\ldot\,
    \histS = t_3 \hpts (x,3) \wedge
%    \hist \subseteq \hist \wedge
%    \stableorder \subseteq \stableorder \wedge
    \dom{h_o} \subseteq \sideal{\stableorder}{t_3} \wedge \hbox{}}}\hfill\\
&   \hfill \specK{\phantom{\{ r\ldot\, \exists t_3\, t_s\ldot}\,
    \dom{h} \subseteq \ideal{\stableorder}{t_s} \wedge r = \eval\ t_s\ \stableorder\ \hist\}}\hfill 
\end{array}
\]}

The proof applies the rule for parallel composition of FCSL. This rule
is described in
%
\ifdefined\extflag{Appendix~\ref{sc:background}.}
\else {Appendix~A~\cite{CoqFiles}.}
\fi
%
Here, we just mention that, upon forking, the rule distributes the
value of $\histS$ of the parent thread, to the $\histS$ values of its
children; in this case, all these are $\hempty$. Dually, upon joining,
the $\histS$ values of the children in lines 4a and 4b, are collected,
in line 5, into that of the parent. The other assertions in 4a and 4b
directly follow from the specs of \jyscan\ and \jywrite\ and the
Invariant~~\ref{inv:monotonicity}.\ref{inv:mono:hist}, and directly
transfer to line 5.
%
While the proof outline does not establish how
\jyscan\ and \jywrite\ interleaved, it establishes that $t_3$ and $t_s$
both appear after the writes that are prior to the client's call.
{\small
\[
\begin{array}{r l}
  \num{1} & \specK{\{ \histS = \hempty \wedge h \subseteq \hist \wedge
    h_o \subseteq \histO \}} \\ %\wedge w \subseteq \stableorder \}} \\
%  \num{2} & \specK{\{ \histS = \hempty \wedge h = \histP \wedge w = \stableorder \}}\\
  \num{2} & \esc{write}~(x,2);\\
  \num{3} & \specK{\{ \exists\, t_2 \ldot\, \histS = t_2 \hpts (x,2) \wedge \dom{h_{\othersub}} \subseteq \sideal{\stableorder}{t_2} \}} \\
%  \num{5} & \specK{\{ \exists\, t_2 \ldot\,
%    \histS = t_2 \hpts (x,2) \wedge j = \hist \wedge
%    h \subseteq j \wedge  s = \stableorder \wedge w \subseteq s \wedge
%    \dom{h_{\othersub}} \subseteq \sideal{s}{t_2} \}} \\
%  \num{6} & \specK{\{ \exists\, t_2 \ldot\,
%    ( \histS = \hempty \wedge
%    h \subseteq j \wedge   w \subseteq s)\mathrel{\circledast}}\\
%    & \specK{\phantom{\{ \exists\, t_2 \ldot}\quad
%    (\histS = k = t_2 \hpts (x,2) \wedge
%      h \subseteq j \wedge   w \subseteq s \wedge
%    \dom{h_{\othersub}} \subseteq \sideal{s}{t_2}) \}} \\
  \num{4} & \esc{write}~(y,1)\\
%  \num{8} & \specK{\{ \exists\, t_1\, t_2\ldot\, 
%    (\histS = t_1 \hpts (y,1) \wedge
%    h \subseteq j \subseteq \hist \wedge
%    w \subseteq s  \subseteq \stableorder \wedge
%    \dom{j_{\othersub} \hunion k} \subseteq \sideal{\stableorder}{t_1})
%       \mathrel{\circledast}}\\
%   & \specK{\phantom{\{\exists\, t_1\, t_2\ldot}\,
%    (\histS = t_2 \hpts (x,2) \wedge
%    h \subseteq j \subseteq \hist \wedge
%    w \subseteq s \subseteq \stableorder \wedge
%    \dom{h_{\othersub}} \subseteq \sideal{\stableorder}{t_2}) \}} \\
  \num{5} & \specK{\{ \exists\, t_1\, t_2\ldot\, 
    \histS = t_1 \hpts (y,1) \hunion t_2 \hpts (x,2) \wedge %\hbox{}}\\
%    & \specK{\hphantom{\{ \exists\, t_1\, t_2\ldot}\, 
    \dom{h_{\othersub}} \subseteq \sideal{\stableorder}{t_2} \wedge 
    t_2\ {\stableorder}\ t_1\}}
\end{array}
\]}
%%

The second proof outline starts with the same precondition. Then line
$3$ directly follows from the spec of \jywrite, using $h_o
\subseteq \histO$. To proceed, we need to apply FCSL
\emph{framing}: the precondition of \jywrite\ requires $\histS =
\emptyset$, but we have $\histS = t_2 \mapsto (x, 2)$.
%%
The frame rule is explained in
%
%
\ifdefined\extflag{Appendix~\ref{sc:background}.}
\else {Appendix~A~\cite{CoqFiles}.}
\fi
%
Here we just mention that framing modifies the spec of \jywrite\ by
joining $t_2 \mapsto (x, 2)$ to $\histS$, $\histSP$ \emph{and}
$\histO$ as follows.  {\small
\[
\mathtt{write}\ (p, v) : 
\begin{array}[t]{l}
\tsPre{\{\histS = t_2 \mapsto (x, 2)\}}\
\tsPos{\{\exists t\ldot \histS' = t \mapsto (p, v) \hunion t_2 \mapsto (x, 2) \wedge \hbox{}}\\
\tsPos{\hphantom{\{\histS = t_2 \mapsto (x, 2)\}\ \{\exists t\ldot}
    \dom {\histO \hunion t_2 \mapsto (x, 2)} \cup \scanned\, \stableorder
       \subseteq \sideal{\stableorderP}{t}\}}
\end{array}
\]}

Such a framed spec for \jywrite\ gives us that after line 4: (1)
$\histS = t_1 \hpts (y,1) \hunion t_2 \hpts (x,2)$, and (2)
$\dom{h_{\othersub} \hunion t_2 \mapsto (x, 2)} \subseteq
\sideal{\stableorder}{t_1}$. From Invariants~\ref{inv:monotonicity},
we also obtain that (3) $\dom{h_{\othersub}} \subseteq
\sideal{\stableorder}{t_2}$, which simply transfers from line 3. Now,
in the presence of (2), we can simplify (3) into
$t_2\ {\stableorder}\ t_1$, thus obtaining the postcondition in line
5.

The final step applies the rule for parallel composition to the two
derivations, splitting $\histS$ upon forking, and collecting it upon
joining:
%
{\small
\[
e : \begin{array}[t]{l}
    \specK{\{\histS = \hempty \wedge h \subseteq \hist \wedge h_o \subseteq \histO\}}\\
 \specK{\{r\ldot \exists\, t_1\, t_2\, t_3\, t_s \ldot 
    \histS = t_1 \hpts (y,1) \hunion t_2 \hpts (x,2) \hunion t_3 \hpts
        (x, 3) \wedge  \dom{h} \subseteq \ideal{\stableorder}{t_s} \wedge \hbox{}}\\
\specK{\hphantom{\{r\ldot \exists\, t_1\, t_2\, t_3\, t_s \ldot }
\dom{h_{\othersub}} \subseteq \sideal{\stableorder}{t_2}, \sideal{\stableorder}{t_3} \wedge 
 t_2\ {\stableorder}\ t_1 \wedge 
%   t_s \in \scanned\, \stableorder \wedge 
     r = \eval\, t_s\, \stableorder\, \hist\}}
\end{array}
\]}

From here, the VDM spec of $e$ is derived by priming the Greek letters
in the postcondition, and choosing $h = \hist$ and $h_o = \histO$.

The spec of $e$ can be further used in various contexts. For example,
to recover the context from Section~\ref{sc:overview}, where $e$ is
invoked with $x = 5$, $y = 0$, we can frame $e$ wrt.~$\histS = t_5
\hpts (x, 5) \hunion t_0 \hpts (y, 0)$ to make explicit the events
that initialize $x$ and $y$. Then, 
%
%by enumerating all the
%possibilities for $\stableorder$ that satisfy the constraints of the
%postcondition, 
%
it is possible to derive in FCSL that if $e$ executes without
interference (\ie, if $\hist = \histO = \histP = \histOP = \hempty$),
then the result at the end must be $r \in \{(5,0), (2,0), (3,0),
(2,1), (3,1)\}$. As expected, $r \neq (5, 1)$, because the write of
$2$ sequentially precedes the write of $1$.

We next illustrate the use of Invariants~\ref{lem:scanned}, which are
required for clients that use $\esc{scan}$ in \emph{sequential
  composition}. We consider the program
%
\[
e' = r \tbnd \esc{scan} ;\, \esc{write}\, (x,v);\, {\small{\kw{return}}}\, r 
\]
%
and prove the following spec:
{\small
\[
e' : \tsPre{\{ \histS = \hempty \}} \
\begin{array}[t]{l}
     \tsPos{\{ \exists\, t_s\ t_x\ldot\,
               \histSP = t_x \hpts (x, v) \wedge t_s \in \sideal{\stableorderP}{t_x} \wedge %\hbox{}}\\
%     \tsPos{\hphantom{\{\exists\, t_s\ t_x\ldot\,} 
           r = \eval\ t_s\ \stableorderP\ \histP \}}
\end{array}
\]}

The spec says that the write event ($t_x$) is subsequent to the scan
($t_s$), as one would expect. In particular, the snapshot $r$ remains
valid, \ie, the write does not change the order $\stableorder$ and
history $\hist$ in a way that makes $r$ cease to be a valid snapshot
in $\stableorderP$ and $\histP$. The proof outline follows, with the
explanation of the critical steps.
%
{\small
\[
\!\begin{array}{r l}
% \num{1} & \specK{\{ \histS = \hempty \}} \\
 \num{1} & \specK{\{ \histS = \hempty \}}\\ %\wedge
%                        h \subseteq \hist \wedge
%                        w \subseteq \stableorder \}} \\
 \num{2} & r \tbnd \esc{scan};\\
 \num{3} & \specK{\{ \exists\, t_s, w'(=\stableorder), h'(=\hist)\ldot 
       \histS = \hempty \wedge 
       %w \subseteq \stableorder \wedge
%      \dom{h} \subseteq \ideal{\stableorder}{t} \wedge}\\
        t_s \in \scanned{w'}  \wedge r = \eval\ t_s\ w'\ h'\}}\\
%\num{5} & \specK{\{ \exists\, t_s\ldot\,
%       \histS = \hempty \wedge j = \hist \wedge h \subseteq j \wedge
%       s = \stableorder \wedge w \subseteq s \wedge
%      \dom{h} \subseteq \ideal{s}{t} \wedge }\\
%          & \specK{\hphantom{\{\ \exists\, t_s\ldot}\,
%        t \in \scanned s \wedge r = \eval\, t_s\, s\, j\}}\\
 \num{4} & \esc{write}\, (x,v);\\
% \num{7} & \specK{\{ \exists\, t_s\, t_x \ldot\,
%           \histS = t_x \hpts (x, v) \wedge \hbox{}}\\
%           h \subseteq j \subseteq \hist \wedge
%           w \subseteq s \subseteq \stableorder \wedge}\\
%          & \specK{\hphantom{\{\ \exists\, t_s\, t_x\ldot}\,
%           \dom{h} \subseteq \ideal{s}{t} \wedge
%           \scanned\, s \subseteq \sideal{\stableorder}{t_x} \wedge
%           t_s \in \scanned s \wedge}\\
%         & \specK{\hphantom{\{\ \exists\, t_s\, t_x\ldot}\,
%             \ideal{s}{t_s} = \ideal{\stableorder}{t_s} \wedge\
%             r = \eval\, t_s\, s\, j\}}\\
 \num{5} & \specK{\{ \exists\, t_s\, t_x\ldot\,
               \histS = t_x \hpts (x, v) \wedge t_s \in \sideal{\stableorder}{t_x} \wedge t_s \in \scanned{\stableorder} \wedge
%               \dom{h} \subseteq \sideal{\stableorder}{t_x} \wedge
               r = \eval\ t_s\ \stableorder\ \hist\}}\\
 \num{6} & \kw{return}\ r
% \num{10} & \specK{\{ \exists\, t_s\, t_x\ldot\,
%               \histS = t_x \hpts (x, v) \wedge
%               w \subseteq \stableorder \wedge
%               h \subseteq \hist \wedge
%}\\
%          & \specK{\hphantom{\{ \exists\, t_s\, t_x\ldot}\,
%               t_s \in \scanned\, \stableorder \wedge
%               r = \eval\, t_s\, \stableorder\, \hist \wedge
%               t_s \in \sideal{\stableorder}{t_x}\}}
 %% \num{11} & \specK{\{ \exists\, t_s\ t_x\ldot\,
 %%               \histSP = t_x \hpts (x, v) \wedge
 %%               \hist \subseteq \histP \wedge
 %%               \stableorder \subseteq \stableorderP \wedge
 %%               \dom{\hist} \subseteq \sideal{\stableorderP}{t_x} \wedge}\\
 %%          & \specK{\hphantom{\{ \exists\, t_s\ t_x\ldot}\,
 %%               t_s \in \scanned\, \stableorderP \wedge
 %%               r = \eval\, t_s\, \stableorderP\, \histP \wedge
 %%               t_s \in \sideal{\stableorderP}{t_x}\}}
\end{array}
\]}

Line 3 is a direct consequence of the spec of \jyscan, where we
omitted the conjunct $\dom{\hist} \subseteq
\ideal{\stableorderP}{t_s}$, as we do not need it for the subsequent
derivation. We also introduce explicit names $w'$ and $h'$ for the
current values of $\stableorder$ and $\hist$.
%
Now, to derive line 5, by the spec of \jywrite, we know there exists a
timestamp $t_x$ corresponding to the write, such that (1) $\histS =
t_x \hpts (x, v)$, which is a conjunct in line 5, and also (2)
$\dom{\histO} \cup \scanned{w'} \subseteq
\sideal{\stableorder}{t_x}$. Furthermore, (3) $t_s \in \scanned{w'}$,
and (4) $r = \eval\ t_s\ w'\ h'$, simply transfer from line 3. From
(2) and (3), we infer that $t_s \in \sideal{\stableorder}{t_x}$. To
complete the derivation of line 5, it remains to show that $t_s \in
\scanned{\stableorder}$ and $r = \eval\ t_s\ \stableorder\ \hist$. For
this, we use (3), (4) and the Invariants~\ref{inv:monotonicity}
and~\ref{lem:scanned}, as follows.  First, by
Invariant~\ref{inv:monotonicity}.\ref{inv:mono:ideal}, and because
$t_s \in \scanned{w'}$, we get $\ideal{w'}{t_s} =
\ideal{\stableorder}{t_s}$. By
Invariant~\ref{lem:scanned}.\ref{lem:scan:prefix}, this gives us $t_s
\in \scanned{\stableorder}$ as well.  By
invariant~\ref{inv:monotonicity}.\ref{inv:mono:hist}, $h' \subseteq
\hist$, and then by Invariant~\ref{lem:scanned}.\ref{lem:scan:eval},
$r = \eval\ t_s\ w'\ h' = \eval\ t_s\ \stableorder\ \hist$, completing
the deduction of line 5.  

%Finally, the desired postcondition immediately follows after the
%return, by priming $\stableorder$ and $\hist$ to switch to the VDM
%notation.

Observe that the main role of $\scanned$ in
proofs is to enable showing stability of values obtained by $\eval$,
using Invariants~\ref{lem:scanned}.\ref{lem:scan:prefix}
and~\ref{lem:scanned}.\ref{lem:scan:eval}. The remaining
Invariants~\ref{lem:scanned}.\ref{lem:scan:total}
and~\ref{lem:scanned}.\ref{lem:scan:wkn} allow us to replace a number
of conjuncts about $\scanned$ by a single one that expresses the membership
of the largest timestamp in the current $\scanned$-set.

\begin{comment}

\begin{figure}
\[
\begin{array}{r l}
  \num{1} & \specK{\{\ \histS = \hempty \}} \\
  \num{2} & \specK{\{\ \histS = \hempty \wedge
    w = \stableorder \wedge h = \hist \}} \\
  \num{3} & r_1 \tbnd \esc{scan};\\
   \num{4} & \specK{\{ \exists\, t_1\ldot\,
       \histS = \hempty \wedge h \subseteq \hist \wedge
       w \subseteq \stableorder \wedge
      \dom{h} \subseteq \ideal{\stableorder}{t_1} \wedge}\\
          & \specK{\hphantom{\{ \exists\, t_1\ldot}\,
        t_1 \in \scanned{\stableorder}  \wedge
        r_1 = \eval\, t_s\, \stableorder\, \hist\}}\\
   \num{5} & \specK{\{ \exists\, t_1\ldot\,
       \histS = \hempty \wedge j = \hist \wedge h \subseteq j \wedge
       s = \stableorder \wedge w \subseteq s \wedge
      \dom{h} \subseteq \ideal{s}{t} \wedge }\\
          & \specK{\hphantom{\{\ \exists\, t_1\ldot}\,
        t_1 \in \scanned s \wedge r_1 = \eval\, t_1\, s\, j\}}\\
  \num{6} & r_2 \tbnd \esc{scan};\\
  \num{7} & \specK{\{ \exists\, t_1\, t_2\ldot\,
     \histS = \hempty \wedge
     h \subseteq j \subseteq \hist \wedge
      w \subseteq s  \subseteq \stableorder \wedge
      \dom{h} \subseteq \ideal{s}{t_1} \wedge}\\
      & \specK{\hphantom{\{ \exists\, t_1\, t_2\ldot}\,
      t_1 \in \scanned s \wedge r_1 = \eval\, t_1\, s\, j \wedge
      \wedge \ideal{s}{t_1} = \ideal{\stableorder}{t_1}\wedge}\\
      & \specK{\hphantom{\{ \exists\, t_1\, t_2\ldot}\,
      \dom{j} \subseteq \ideal{\stableorder}{t_2} \wedge
      t_2 \in \scanned \stableorder \wedge
      r_2 = \eval\, t_2\, \stableorder\, \hist \}}\\
  \num{8} & \specK{\{ \exists\, t_1\, t_2\ldot\,
     \histS = \hempty \wedge h \subseteq \hist \wedge
      w \subseteq \stableorder \wedge
      \dom{h} \subseteq \ideal{\stableorder}{t_2} \wedge
      t_1 \in \scanned \stableorder \wedge}\\
      & \specK{\hphantom{\{ \exists\, t_1\, t_2\ldot}\,
      r_1 = \eval\, t_1\, \stableorder\, \hist \wedge
      t_1 \mathrel{\stableorder} t_2 \wedge
      t_2 \in \scanned \stableorder \wedge
      r_2 = \eval\, t_2\, \stableorder\, \hist \}}\\
  \num{9} & \kw{return}\, (r_1, r_2) \\
 \num{10} & \specK{\{ \exists\, t_1\, t_2\ldot\,
     \histS = \hempty \wedge h \subseteq \hist \wedge
      w \subseteq \stableorder \wedge
      \dom{h}\! \subseteq \ideal{\stableorder}{t_2} \wedge
      t_1 \in \scanned \stableorder \wedge}\\
      & \specK{\hphantom{\{ \exists\, t_1\, t_2\ldot}\,
      r_1 = \eval\, t_1\, \stableorder\, \hist \wedge
      t_1 \mathrel{\stableorder} t_2 \wedge
      t_2 \in \scanned \stableorder \wedge
      r_2 = \eval\, t_2\, \stableorder\, \hist \}}
  %% \num{11} & \specK{\{ \exists\, t_1\ t_2\ldot\,
  %%              \histSP = \hempty \wedge\
  %%              \hist \subseteq \histP \wedge\
  %%              \stableorder \subseteq \stableorderP \wedge
  %%             \dom{\hist} \subseteq \ideal{\stableorderP}{t_2} \wedge
  %%                t_2 \in \scanned\, \stableorderP \wedge}\\
  %%         & \specK{\phantom{\{ \exists\, t_1\ t_2 \ldot}\,  
  %%              r_1 = \eval\, t_1\, \stableorderP\, \histP \wedge
  %%              t_1 \mathrel{\stableorderP} t_2 \wedge
  %%              t_2 \in \scanned\, \stableorderP \wedge
  %%             r_2 = \eval\, t_2\, \stableorderP\, \histP \}}
\end{array}
\]
\caption{\label{proof:seq-clientSS}A proof outline for a client with two
  sequential calls to \jyscan}
  
\end{figure}

Figure~\ref{proof:seq-clientSS} presents the proof outline for the
second sequential client, which also starts with an empty self
history. In line~2, $h$ and $w$ bind the current values of $\hist$ and
$\stableorder$ before the first call to \jyscan\ in line~3. After the
first call to \jyscan, we have in line~4 that $t_1$ is the scan event
that determines the snapshot $r_1$ in the current state, and that this
captures the complete history before the call to scan--- $\dom{h}
\subseteq \ideal{\stableorder}{t_1}$. Before the second call to
$\jyscan$ in line~6, we bind the current $\hist$ and $\stableorder$ by
$j$ and, respectively, $s$. Now, in line~7, we have that $t_2$ is the
scan even that captures the new snapshot $r_2$, and it is such that
has witnessed all events in $j$, $\dom{j} \subseteq
\ideal{\stableorder}{t_1}$. We also have that prefix from the previous
scan call has been preserved---\ie ${s}{t_1} =
\ideal{\stableorder}{t_1}$. By Lemma~\ref{lem:scan:prefix}, we assert
in line~8 that $t_1 \in \scanned\, \stableorder$, \ie that it is still
a valid scan event. This latter fact, together with the snapshot
preservation lemma~(Lemma~\ref{lem:scan:eval}), allows us to infer
that the snapshot $r_1$ is still valid, \ie $r_1 = \eval\, t_1\,
\stableorder\, \hist$. Finally, by transitivity and subset inclusion
we infer that $\dom{h}\! \subseteq \ideal{\stableorder}{t_2}$ and $
t_1 \mathrel{\stableorder} t_2$. Again, the last step is trivial as it
just returns the collected snapshots.

\gad{ I'm not sure we should keep both clients, they are very
  similar. In any case, we might also want to consider switching the
  order between them.}

\paragraph*{Sequential Clients}%
In order to illustrate the points made above, we consider are toy
sequential clients which illustrate how to work with our specs,
specially the particulars about how to reason out of $\scanned$ and
$\eval$ assertions:
%
\[
\begin{array}[t]{l l}
\begin{array}[t]{l}
  \tsPre{\{ \histS = \hempty \}} \\
  r \tbnd \esc{scan} ;\, \esc{write}\, (x,v);\, \kw{return}\, r\\
  \tsPos{\{ \exists\, t_s\ t_x\ldot\,
               \histSP = t_x \hpts (x, v) \wedge
               \hist \subseteq \histP \wedge}\\
  ~~~~\tsPos{
             \stableorder \subseteq \stableorderP \wedge
             t_s \in \scanned\, \stableorderP \wedge}\\
  ~~~~\tsPos{
    \dom{\hist} \subseteq \sideal{\stableorderP}{t_x} \wedge
     t_s \in \sideal{\stableorderP}{t_x} \wedge}\\
  ~~~~\tsPos{ r = \eval\, t_s\, \stableorderP\, \histP \}}
\end{array}
&
\begin{array}[t]{l}
  \tsPre{\{ \histS = \hempty \}} \\
  r_1 \tbnd \esc{scan} ;\ r_1 \tbnd \esc{scan};\, \kw{return}\, (r_1,r_2)\\
  \tsPos{\{ \exists\, t_1\ t_2\ldot\,
               \histSP = \hempty \wedge
               \hist \subseteq \histP \wedge
               \stableorder \subseteq \stableorderP \wedge}\\
  ~~~~\tsPos{ \dom{\hist} \subseteq \ideal{\stableorderP}{t_2}\wedge
               t_1 \mathrel{\stableorderP} t_2 \wedge}\\
  ~~~~\tsPos{t_1 \in \scanned\, \stableorderP \wedge\!
             r_1 = \eval\, t_1\, \stableorderP\, \histP \wedge}\\
  ~~~~\tsPos{ t_2 \in \scanned\, \stableorderP \wedge\!
              r_2 = \eval\, t_2\, \stableorderP\, \histP \}}
\end{array}
\end{array}
\]
%
The first client performs a call to \jyscan\ followed by a call to
\jywrite. We are interested in proving that the snapshot returned by
\jyscan\ is still valid further down and, moreover, that the write
event $t_x$ that has been produced is fresh with regard to $t_s$, that
is that $ t_s \in \sideal{\stableorder}{t_x}$.
%
The second client illustrates the interaction between subsequent calls
to \jyscan. The purpose of this exercise is to show that the second
call to \jyscan\ returns a newer snapshot--- \ie $ t_1
\mathrel{\stableorderP} t_2$--- and that both are valid snapshots of
the memory.
%
Since we intend to keep the definition of $\stableorder$, $\eval$ and
$\scanned$ opaque to clients, we first need to provide an interface to
reason with them. The latter consist of a sequence of lemmas regarding
those assertion. All of this lemmas have been proved in our Coq
development~\cite{CoqFiles}.

%The first thing we need to stablish is the stability of all the
%assertion which appears in Figure~\ref{fig:specs}:

%% \an{Hmm, don't understand why we need to talk about stability here?
%%   That should not be used in these simple clients. We need other
%%   lemmas here, namely those that show how $\scanned{\stableorder}$
%%   relate to $\sideal{\stableorder}{t}$.} \gad{We do need the
%%   stability when the two clients return the scanned values, right?
%%   In any case, it would be weird to first present the rules, then
%%   clients and leave the stability for the end.} \gad{I know I'm
%%   missing some other lemmas.}

%% \begin{lemma}[Stability of Assertions I]\label{lem:menvs}
%% If by environment stepping the auxiliary state projections
%% $\ordlist, \stableorder, \chi, \kappa, \ldots,$ evolve to
%% $\ordlist', \stableorder', \chi', \kappa', \ldots$, then:
%% %
%% \begin{enumerate}
%%  \item\label{lem:menvs:jleq}
%%  $ \stableorder \subseteq \stableorder'$ \ie the logical ordering is
%%  monotone. The same holds for the suborders imposed by the ideals
%%  $\sideal{\stableorder}{t}$ and $\ideal{\stableorder}{t}$.

%%  \item \label{lem:menvs:hist} $ \hist \subseteq \hist'$
%%   \ie histories grow monotonically.

%%  \item \label{lem:menvs:scann} If $ t \in \scanned\ \stableorder$ then
%%    $ t \in \scanned\ \stableorderP$ and $\ideal{\stableorder}{t} =
%%    \ideal{\stableorderP}{t}$ \ie, environment steps preserve scanned
%%    prefixes.
   
%%  \item \label{lem:menvs:eval} If $t \in \scanned\ \stableorder$ then
%%    $\eval\ t\, \stableorder\, \hist = \eval\ t\, \stableorderP\,
%%    \histP$ \ie the result of a previous scan is invariant under
%%    environment interference.
%% \end{enumerate}
%% \end{lemma}
%% \begin{proof}
%% We have proven the correctness of these lemmas in our Coq
%% development. They are in most cases carried out by induction on the
%% number of steps the environment performs, and the base cases are
%% provided by the action of the auxiliary code described before in
%% Section~\ref{sc:implementation}, on the transposition of the state
%% (see~\cite{NanevskiLSD+ESOP14} for further details). In the online
%% appendices~\cite{CoqFiles}, Appendix~\ref{sc:coq-code} gives some
%% insight into the most interesting details of such proofs. \qed
%% \end{proof}



\gad{I'm not sure if these are good names at all. Perhaps I should
  better enumerate them.}


%% The first client performs a call to \jyscan\ followed by a call to
%% \jywrite. We are interested in proving that the snapshot returned by
%% \jyscan\ is still valid further down and, moreover, that the write
%% event $t_x$ that has been produced is fresh with regard to $t_s$, that
%% is that $ t_s \in \sideal{\stableorder}{t_x}$. We present the proof
%% in a traditional Hoare logic proof outline:

\begin{figure}
\[
\!\begin{array}{r l}
 \num{1} & \specK{\{ \histS = \hempty \}} \\
 \num{2} & \specK{\{ \histS = \hempty \wedge
                        h = \hist \wedge
                        w = \stableorder \}} \\
 \num{3} & r \tbnd \esc{scan};\\
 \num{4} & \specK{\{ \exists\, t_s\ldot\,
       \histS = \hempty \wedge h \subseteq \hist \wedge
       w \subseteq \stableorder \wedge
      \dom{h} \subseteq \ideal{\stableorder}{t} \wedge}\\
          & \specK{\hphantom{\{ \exists\, t_s\ldot}\,
        t \in \scanned{\stableorder}  \wedge
        r = \eval\, t_s\, \stableorder\, \hist\}}\\
\num{5} & \specK{\{ \exists\, t_s\ldot\,
       \histS = \hempty \wedge j = \hist \wedge h \subseteq j \wedge
       s = \stableorder \wedge w \subseteq s \wedge
      \dom{h} \subseteq \ideal{s}{t} \wedge }\\
          & \specK{\hphantom{\{\ \exists\, t_s\ldot}\,
        t \in \scanned s \wedge r = \eval\, t_s\, s\, j\}}\\
 \num{6} & \esc{write}\, (x,v);\\
 \num{7} & \specK{\{ \exists\, t_s\, t_x \ldot\,
           \histS = t_x \hpts (x, v) \wedge
           h \subseteq j \subseteq \hist \wedge
           w \subseteq s \subseteq \stableorder \wedge}\\
          & \specK{\hphantom{\{\ \exists\, t_s\, t_x\ldot}\,
           \dom{h} \subseteq \ideal{s}{t} \wedge
           \scanned\, s \subseteq \sideal{\stableorder}{t_x} \wedge
           t_s \in \scanned s \wedge}\\
         & \specK{\hphantom{\{\ \exists\, t_s\, t_x\ldot}\,
             \ideal{s}{t_s} = \ideal{\stableorder}{t_s} \wedge\
             r = \eval\, t_s\, s\, j\}}\\
 \num{8} & \specK{\{ \exists\, t_s\, t_x\ldot\,
               \histS = t_x \hpts (x, v) \wedge
               h \subseteq \hist \wedge
               \dom{h} \subseteq \sideal{\stableorder}{t_x} \wedge
               w \subseteq \stableorder \wedge}\\
          & \specK{\hphantom{\{ \exists\, h\, w\, t_s\, t_x\ldot}\,
               t_s \in \scanned\, \stableorder \wedge
               r = \eval\, t_s\, \stableorder\, \hist \wedge
               t_s \in \sideal{\stableorder}{t_x}\}}\\
 \num{9} & \kw{return}\ r \\
 \num{10} & \specK{\{ \exists\, t_s\, t_x\ldot\,
               \histS = t_x \hpts (x, v) \wedge
               w \subseteq \stableorder \wedge
               h \subseteq \hist \wedge
}\\
          & \specK{\hphantom{\{ \exists\, t_s\, t_x\ldot}\,
               t_s \in \scanned\, \stableorder \wedge
               r = \eval\, t_s\, \stableorder\, \hist \wedge
               t_s \in \sideal{\stableorder}{t_x}\}}
 %% \num{11} & \specK{\{ \exists\, t_s\ t_x\ldot\,
 %%               \histSP = t_x \hpts (x, v) \wedge
 %%               \hist \subseteq \histP \wedge
 %%               \stableorder \subseteq \stableorderP \wedge
 %%               \dom{\hist} \subseteq \sideal{\stableorderP}{t_x} \wedge}\\
 %%          & \specK{\hphantom{\{ \exists\, t_s\ t_x\ldot}\,
 %%               t_s \in \scanned\, \stableorderP \wedge
 %%               r = \eval\, t_s\, \stableorderP\, \histP \wedge
 %%               t_s \in \sideal{\stableorderP}{t_x}\}}
\end{array}
\]
\caption{\label{proof:seq-clientSW} A proof outline for a client with
  sequential calls to \jyscan\ and \jywrite.}
\end{figure}


\end{comment}

