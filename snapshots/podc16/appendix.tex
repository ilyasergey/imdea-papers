\section*{Optional appendices}

\section{Some background on FCSL}
\label{sc:background}

A state of a resource in FCSL, such as that of snapshot data structure
discussed in this paper, always consists of three distinct auxiliary
variables that we name $a_\lcl$, $a_\env$ and $a_\joint$. These stand
for the abstract self state, other state, and shared (joint) state.

However, the user can pick the types of these variables based on the
application. In this paper, we have chosen $a_\lcl$ and $a_\env$ to be
histories, and have correspondingly named them $\histS$ and $\histO$.
On the other hand, $a_\joint$ consists of all the other auxiliary
components that we discussed, such as the variables $\histJ$, $\E$,
$\C$, $\sx$, $\sy$, $\wx$ and $\wy$. These variables become merely
projections out of $a_\joint$.

It is essential that $a_\lcl$ and $a_\env$ have a common type, which
moreover, exhibits the algebraic structure of a \emph{partial
  commutative monoid} (PCM). A PCM requires a partial binary operation
$\bullet$ which is commutative and associative, and has a unit. In the
case of histories, the partial operation is disjoint union $\hunion$
with $\mathsf{empty}$ history as the unit. PCMs are important, as they
give a generic way to define the inference rule for parallel
composition.
%
\[
%\tag{\normalsize \arabic{tags}}\refstepcounter{tags}\label{eq:parrule}
{\small{
\begin{array}{c}
\specK{\{P_1\}}\ e_1\ \specK{\{Q_1\}} \quad \specK{\{P_2\}}\ e_2\ \specK{\{Q_2\}} \\[2pt]
\hline\\[-7pt]
\specK{\{P_1 \circledast P_2\}}\ e_1 \parallel e_2\ \specK{\{[r.1/r]Q_1 \circledast [r.2/r]Q_2\}} 
\end{array}
}}
\]
%
Here, $\circledast$ is defined as follows.
\[
%\tag{\normalsize \arabic{tags}}\refstepcounter{tags}\label{eq:ssep}
\begin{array}{c}
(P_1 \circledast P_2)(a_\lcl, a_\joint, a_\env) \iff \exists x_1~x_2\ldot a_\lcl = x_1 \bullet x_2, \hbox{}\\
 P_1 (x_1, a_\joint, x_2 \bullet a_\env), P_2 (x_2, a_\joint, x_1 \bullet a_\env)
\end{array}
\]
%
%
The inference rule, and the definition of $\circledast$, formalize the
intuition that when a parent thread forks $e_1$ and $e_2$, then $e_1$
is part of the environment for $e_2$ and vice-versa. This is so
because the \emph{self} component $a_\lcl$ of the parent thread is
split into $x_1$ and $x_2$; $x_1$ and $x_2$ become the \emph{self}
parts of $e_1$, and $e_2$ respectively, but $x_2$ is also added to the
\emph{other} component $a_\env$ of $e_1$, and dually, $x_1$ is added
to the \emph{other} component of $e_2$.

As discussed in Section~\ref{sec:clients}, this makes our specs of
{\tt scan} and {\tt write} \emph{principal}. Many other logics that
employ auxiliary state, going back to Owicki and
Gries~\cite{Owicki-Gries:CACM76}, require that their procedures are
annotated for a specific number and forking pattern of interfering
threads. When this pattern changes, the programs have to be
reannotated, and proofs redone. But this is not the case in FCSL.

