\section{Related work}
\label{sc:related}

% Early times

The proof method for establishing linearizability of concurrent
objects based on the notion of \emph{linearization points} has been
presented in the original paper by Herlihy and
Wing~\cite{Herlihy-Wing:TOPLAS90}. The first Hoare-style logic,
employing this method for compositional proofs of linearizability was
introduced in Vafeiadis' PhD thesis~\cite{Vafeiadis:PhD}. However,
that logic, while being inspired by the
combination~\cite{Vafeiadis-Parkinson:CONCUR07} of Rely-Guarantee
reasoning~\cite{Jones:TOPLAS83} and Concurrent Separation
logic~\cite{OHearn:TCS07} with syntactic treatment of linearization
points~\cite{Vafeiadis-al:PPoPP06}, did not have a soundness proof
with respect to any program semantics. Furthermore, the
work~\cite{Vafeiadis:PhD} did not connect reasoning about
linearizability to the verification of client programs, which make use
of linearizable objects in a concurrent environment (\cf
Section~\ref{sec:clients}).

% Modern logics for linearizability

Both these shortcomings were addressed in more recent works on program
logics for establishing linearizability~\cite{Liang-Feng:PLDI13}, or,
equivalently~\cite{Filipovic-al:TCS10}, \emph{observational
  refinement}~\cite{Turon-al:ICFP13}, which provided semantically
sound methodologies for (a) verifying linearizability/refinement of
concurrent objects \emph{as well as} for (b) giving the objects
Hoare-style specifications, useful for the clients.
%
However, in the both
approaches~\cite{Liang-Feng:PLDI13,Turon-al:ICFP13} establishing (a)
and (b) essentially requires one to prove \emph{two different} facts
about a program, and, if one is interested only in the Hoare-style
reasoning by means of composing program specifications, verifying
linearizability (a) is a detour, which might be avoided.

% No-linearizability

This observation has been recognized in a series of more recent works
on program logics for concurrency that all focused on establishing
Hoare-style specifications for concurrent objects (b) without
resorting to
linearizability~\cite{Sergey-al:ESOP15,Svendsen-Birkedal:ESOP14,ArrozPincho-al:ECOOP14,Jung-al:POPL15}.
%
In this paper, we are following the same way of thinking, building on
the ideas from the prior work~\cite{Sergey-al:ESOP15}, which explored
some patterns of assigning \emph{subjective} Hoare-style concurrent
specifications with auxiliary histories to concurrent objects
(including \emph{higher-order} ones, such as {flat
  combiner}~\cite{Hendler-al:SPAA10}) in
FCSL~\cite{Nanevski-al:ESOP14}. The work~\cite{Sergey-al:ESOP15} has
generalized earlier results on history-based Hoare-style
logics~\cite{Fu-al:CONCUR10, Gotsman-al:ESOP13,Bell-al:SAS10}, yet it
has not provided a way to reason about concurrent objects, featuring
future-dependent linearization points.
%

The key novelty of this work with respect to previous results
involving Hoare-style reasoning about histories~\cite{Fu-al:CONCUR10,
  Gotsman-al:ESOP13,Bell-al:SAS10,Sergey-al:ESOP15,Hemed-al:DISC15} is
the idea of dynamically \emph{re-linking} the auxiliary histories,
enabling efficient constructive reasoning about non-local and
future-dependent linearization points.
%
Since re-linking as we presented it is just manipulation with
otherwise standard auxiliary state, we did not have to extend the
metatheory of FCSL, and were able to use it
\emph{off-the-shelf}. Furthermore, relying on the auxiliary state
makes it possible to extend our verification method for reasoning
about higher-order (\ie, parametrized by another data structure)
snapshot-based concurrent constructions~\cite{Petrank-Timnat:DISC13},
which is our immediate future work.
%
In contrast, alternative modern programming
logics~\cite{ArrozPincho-al:ECOOP14,Jung-al:POPL15,Svendsen-Birkedal:ESOP14}
would require introduction of prophecy variables in order to verify
Jayanti's snapshot construction, and, to the best of our knowledge,
none of these extensions has been implemented yet.

Related to our result, O'Hearn \etal have demonstrated how to employ
history-based reasoning and Hoare-style logic for proving
\emph{non-constructively} existence of linearization points for
concurrent objects out of the data structure
invariants~\cite{OHearn-al:PODC10}---the result is known as \emph{the
  Hindsight Lemma}. The reasoning principle presented in this paper
generalizes that idea, since the Hindsight Lemma is only applicable to
``pure'' concurrent methods (\eg, concurrent set's
\texttt{contains}~\cite{Heller-al:OPODIS05}), which do not determine
position of other threads' linearization points. In contrast, our idea
of re-linking histories also handles the structures, where a
linearization point of a method call (e.g., \texttt{write}) might
depend on the (future) outcome of another operation (e.g.,
\texttt{scan}), as was showcased by Jayanti's construction.



% Most recent related work relies on parametrization to avoid reasoning
% about linearizability. But, that has its drawbacks. In particular,
% while it can handle situations in which linearization points are
% placed in different places, depending on the run-time infomration
% (speculiation), it is not currently strong enough to formalize
% examples where linearizatin points appear in different
% proceedures.\an{Hmm, are we super sure of this?} Thus, we don't
% believe they can handle Jayanti's algorithm.  \is{How about this time
%   we just mention these people briefly and instead do a comparison to
%   the PODC crowd and their reasoning methods, which are all about
%   harnessing the vanialla definition of liearizability. This way, the
%   whole discussion will be more relevant to the audience, as nobody
%   knows the concurrency logics anyway, so we'll just waste valuable
%   space, talking about them in detail.}  \an{I agree that we should
%   just mention them briefly.}

% Independently of us, Kyzha et al. have developed an a method whereby
% linearizability is proved by reordering time-stamped histories,
% similar to the basis of our approach. However, there are many
% differences. 

% \begin{enumerate}
% \item While linearizability does not say anything about clioent side
%   proofs, beyond the ability to replace the two programs in it, our
%   method also gives a way to reason about clients, as we illustrated
%   in Section 4.

% \item While they present a new logic, for us, it is all a mode of use
%   of auxiliary state.

% \item Our PCM of histories let us reuse separation logic in
%   infrastructure (e.g., frame rule) to reason about histories locally,
%   whereas Kyzha et al. use global histories only. Our setting also
%   immediatley lends itself to higher-order programming. This is
%   particularly important for snaphsots, as one of their major
%   application is in iterators -- a prototypical higher-order
%   program~\cite{PetrankT+disc13}. However, we don't explore iterators
%   in this particular paper.

% \item Kyzha et al. track the ordering of timestamps quite differnetly
%   from us. Where we keep an ``existential'' witness for the total
%   ordering of timestamps, at all stages of evaluation, they do so
%   ``universially''. Thus, they require proving that all possible
%   completions of a partial order into a total order are valid for
%   establishing the relation with the linearization program. We believe
%   this leads to larger proofs and more complicated proofs than
%   necessary.
% \end{enumerate}

% Liang et al~\cite{LiangF+pldi13} present a dedicated meta-theory to
% reason with future-dependent linearization points based on
% speculations. \gad{Are we sure they can't do Jayanti here? Need to
%   think what to say about their works as they do have a program logic
%   as we do}
  
