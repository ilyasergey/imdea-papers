%%%
%%% G: Don't go down here for the time being
%%%

We finish this section with a proof of correctness for the
specification of {\tt write} and {\tt scan} presented
in~\eqref{eq:specs}. Even if the formal mechanisation of these proof
in Coq is a work in progress, we can still provide a proof sketch
describing the standard approach to prove the correctness of a
specification in FCSL.

\paragraph{Correctness of write and scan}%
Before the main results we need an intermediate step towards that,
which is that the order $\jleq$, defined in~\eqref{def-jleq} above, is
{\it stable} under environment interference. In FCSL, this entails
considering all possible steps the methods can do, both ``real'' and
auxiliary.

\begin{lemma}[$\jleq{\_}$ is stable]\label{lem:jleq-stable}
If $\ordlist$ evolves to $\ordlist'$ by environment interference then
$ (\jleq{\ordlist}) \subseteq\ (\jleq{\ordlist'})$.
\end{lemma}
\proofsketch{
Since $\jleq$ is defined only by auxiliary state, only auxiliary code
can challenge its stability. Among the operations presented above,
only the relink transition poses a serious threat its stability. The
proof that $\act{relink}$ doesn't affect the chains in $\jleq$ relies,
ultimately on two observations: (i) that relink doesn't affect the
order of green entries in the history, which we mentioned above and
(ii), that the elements in the ``window'' determined by the elements
being pushed correspond to overlapping events. We reason with this
case as follow: Consider a $\act{relink}$ environment call that pushes
$t_y$ past $t_x$, with $t_y \in \histy$ and $t_x \in \histx$, then all
elements $t$ in the chain $t_y \tle t \tleq t_x$ belong to $\histx$,
by virtue of $t_y$ being the yellow timestamp of $t_y$ (and
Invariant~\ref{inv:redzone}). Then, we observe that en entry that is
yellow when relink occurs, commited to memory before {\tt scan} set
the bit to false and either is still active or finished after {\tt
scan} set the $S$ bit to false. If any $t$ in that chain began after
$t_y$ finished it cannot not possibly be green, as the spec for
register would have marked it red. \qed }

\begin{theorem}[Correctness of {\tt write} and {\tt scan}]
The implementations for {\it scan} and {\tt write} in
Figure~\ref{fig:fcsl-snapshot} can be ascribed the spec given
in~\eqref{eq:specs}.
\end{theorem}

\proofsketch{


}
%% \begin{itemize}
%% %% \item Maybe move the invariants here if they are not needed above
%% %% in Section~\ref{sc:formal}.
%%  \item Sketch an explanation of the proof.  \item{Maybe fork this part
%%  as a separate section?}
%% \end{itemize}

