\section{Verification challenge and main ideas}
\label{sc:overview}

%%\begin{wrapfigure}[9]{r}[0pt]{0.4\textwidth} 
%% \begin{figure}
%% %
%% \centering
%% \begin{tabular}{l l l}
%% %
%% %  
%% \begin{minipage}[l]{.30\textwidth}
%% \begin{alltt}
%% \num{1}  write (p, v): () \{
%% \num{2}    \act{write} (p, v);
%% \num{3}    b <- \act{read} (S);
%% \num{4}    \textbf{if} b 
%% \num{5}    \textbfthen \act{transfer} (p, v);
%% \num{6}    \textbf{else skip};\}
%% \end{alltt} 
%% \end{minipage}
%% %
%% & \hfill
%% %
%% \begin{minipage}[l]{.6\textwidth}
%% \begin{alltt}
%% \num{1}  scan (): \(A {\times} A\)  \{
%% \num{2}    \act{write} (S, true);
%% \num{3}    \act{write} (fx,\( \bot\));
%% \num{4}    \act{write} (fy,\( \bot\));
%% \num{5}    vx <- \act{read} (x);
%% \num{6}    vy <- \act{read} (y);
%% \num{7}    \act{write}(S, false);
%% \num{8}    ox <- \act{read} (fx);
%% \num{9}    oy <- \act{read} (fy);
%% \num{10}   \textbf{let} rx = \textbf{if} ox \(\neq \bot\) \textbfthen ox \textbf{else} vx;  
%% \num{11}   \textbf{let} ry = \textbf{if} oy \(\neq \bot\) \textbfthen oy \textbf{else} vy;  
%% \num{11}   \act{relink}(rx, ry);
%% \num{12}   \textbf{return} (rx, ry);
%% \end{alltt} 
%% \end{minipage}
%% %
%% \end{tabular}
%% %
%% \caption{Jayanti's single-scanner, single-writer snapshot algorithm}
%% \label{fig:jayanti}
%% \end{figure}
%\end{wrapfigure}

\newcommand{\actwrite}[2]{{#1}\,{:=}\,{#2}}

% The following version saves a little more space
\begin{figure}
%
\centering
\begin{tabular}{c@{\ \ \ \ \ }c}
%  
\begin{minipage}[t][3.7cm][t]{.5\textwidth}
\small
\begin{alltt}
\num{1} write (p : ptr, v : \(A\)) \{
\num{2}  \actwrite{p}{v};
\num{3}  b \tbnd \act{read}(S);
\num{4}  if b 
\num{5}  then \actwrite{(f_of p)}{v}
\num{6}  {else return} \}

  f_of (p : ptr) \{
   return p = x ? fx : fy \}
\end{alltt}
\end{minipage}
%
&
\begin{minipage}[t][3.7cm][t]{.5\textwidth}
\small
\begin{alltt}
\num{ 7} scan (): \(A {\times} A\)  \{
\num{ 8}  \actwrite{S}{true};
\num{ 9}  \actwrite{fx}{\(\bot\)}; \actwrite{fy}{\(\bot\)};
\num{10}  vx \tbnd \act{read}(x); vy \tbnd \act{read}(y);
\num{11}  \actwrite{S}{false};
\num{12}  ox \tbnd \act{read}(fx); oy \tbnd \act{read}(fy);
\num{13}  rx \tbnd if (ox \(\neq\bot\)) then ox {else} vx;  
\num{14}  ry \tbnd if (oy \(\neq\bot\)) then oy {else} vy;  
\num{15}  return (rx, ry) \}
\end{alltt} 
\end{minipage}
%
\end{tabular}
%
\caption{Jayanti's single-scanner/single-writer snapshot algorithm.}
\label{fig:jayanti-snapshot}
\end{figure}



%\newcommand{\fx}{\mathtt{fx}}
%\newcommand{\fy}{\mathtt{fy}}
%\newcommand{\x}{\mathtt{x}}
%\newcommand{\y}{\mathtt{y}}
%\newcommand{\s}{\mathtt{S}}

\newcommand{\fx}{fx}
\newcommand{\fy}{fy}
\newcommand{\x}{x}
\newcommand{\y}{y}
\newcommand{\s}{S}

Jayanti's snapshot algorithm~\cite{Jayanti:STOC05} provides the
functionality of a shared array of size $m$, operated on by two
procedures: {\tt write}, which stores a given value into an element,
and {\tt scan}, which returns the array's contents. We use the
\emph{single-writer}/\emph{single-scanner} version of the algorithm,
which assumes that at most one thread writes into an element, and at
most one thread invokes the scanner, at any given time.
% 
%where a thread acquires a writer lock for a particular element before
%writing into it, and a scanner before scanning. A scanner lock does
%not preclude writing, and a writer lock for an element does not
%preclude scanning, or writing into other elements. 
This is the simplest of Jayanti's algorithms, but it already exhibits
linearization points of dynamic nature, as we shall see. For
simplicity, we also further restrict the array size to $m=2$ (i.e., we
consider two distinct pointers $\x$ and $\y$, instead of a whole
array). This removes some tedium from verification, but conceptually
exhibits the same challenges.
 
The main difficulty in a snapshot algorithm is to ensure that the
scanner returns sound values of $\x$ and $\y$. A naive scanner
implementation which simply reads $\x$ and $\y$ in succession, is
unsound. To see why, consider the following scenario, starting with
$\x=5$ and $\y=0$. The scanner reads $\x$, but before it reads $\y$,
another thread sequentially changes $\x$ to $2$, and $\y$ to $1$. The
scanner continues to read $\y$, and returns $\x=5, \y=1$, but this was
never the contents of the memory.

To ensure a sound snapshot, Jayanti's algorithm internally keeps
additional \emph{forwarding pointers} $\fx$ and $\fy$, and a boolean
\emph{scanner bit} $\s$. The implementation is given in
Figure~\ref{fig:jayanti-snapshot}, and assumes the
single-writer/single-scanner setup (the assumption can be enforced
with wrapper locking code, which we omit).
%
The intuition is as follows. A writer storing $v$ into $p$ (line~2),
may additionally store $v$ into the forwarding pointer for $p$
(line~5). Thus, if the scanner missed the write and instead read the
old value of $p$ (line~10), it will have a chance to catch $v$ via the
forwarding pointer (line~12). The scanner bit $S$ is used by writers
(line~3) to detect a scan in progress, and forward $v$.

{
%\setlength{\belowcaptionskip}{-5pt} 
\begin{figure}[t]
%
\captionsetup[subfigure]{justification=centering}
\centering  
\begin{subfigure}[t]{1\textwidth}
\centering
\small  
\begin{tabular}{l || l || l}
  \texttt{l: }\texttt{write (x,2);} &
   \multirow{2}{*}{\texttt{c: scan ()}} & 
    \multirow{2}{*}{\texttt{r: write (x,3)}}  \\
  \phantom{\texttt{l: }}\texttt{write (y,1)} & &   
\end{tabular}
\caption{\label{fig:weird:code}Parallel composition of three threads \texttt{l, c, r}.}
\end{subfigure}\\

\begin{subfigure}[b]{1\textwidth}
\begin{tabular}{l@{\hfill} l@{\hfil} l}
\begin{minipage}[t]{0.33\textwidth}
\begin{alltt}
\num{1}  c: \actwrite{S}{true}
\num{2}  c: \actwrite{fx}{\(\bot\)}
\num{3}  c: \actwrite{fy}{\(\bot\)}
\num{4}  c: \act{read}(x)  // vx <- 5
\num{5}  c: \act{read}(y)  // vy <- 0
\num{6}  l: \actwrite{x}{2}
\num{7}  l: \act{read}(S)  // b <- true
\end{alltt}
%% \begin{alltt}
%% \num{1}  c: scan () 
%% \num{2}  c: \act{write}(S, true)
%% \num{3}  c: \act{write}(fx,\( \bot\))
%% \num{4}  c: \act{write}(fy,\( \bot\))
%% \num{5}  c: \act{read}(x)      // vy <- 5
%% \num{6}  c: \act{read}(y)      // vx <- 0
%% \num{7}  l: write (x, 2)
%% \num{8}  l: \act{write}(x, 2)
%% \num{9}  l: \act{read}(S)      // b <- true
%% \num{10} l: \act{write}(fx, 2) 
%% \num{11} l: ret ()
%% \num{12} r: write (x, 3)
%% \end{alltt}
%% \gad{will clean up after final version set}
\end{minipage}
&
\begin{minipage}[t]{0.33\textwidth}
\begin{alltt}
\num{8}  l: \actwrite{fx}{2} 
\num{9}  l: ret ()
\num{10} r: \actwrite{x}{3}
\num{11} l: \actwrite{y}{1}
\num{12} l: \act{read}(S)  // b <- true
\num{13} l: \actwrite{fy}{1}
\num{14} l: ret ()
\end{alltt}
\end{minipage}
&
\begin{minipage}[t]{0.33\textwidth}
\begin{alltt}
\num{15} c: \actwrite{S}{false}
\num{16} r: \act{read}(S)  // b <- false
\num{18} r: ret ()
\num{18} c: \act{read}(fx) // ox <- 2
\num{19} c: \act{read}(fy) // oy <- 1
\num{20} c: ret (2,1)
\end{alltt} 
%% \begin{alltt}
%% \num{13} r: \act{write}(x, 3)
%% \num{14} l: write (y, 1)
%% \num{15} l: \act{write}(y, 1)
%% \num{16} l: \act{read}(S)      // b <- true
%% \num{17} l: \act{write}(fy, 1)
%% \num{18} l: ret ()
%% \num{19} c: \act{write}(S, false)
%% \num{20} r: \act{read}(S)      // b <- false
%% \num{21} r: ret ()
%% \num{22} c: \act{read}(fx)     // ox <- 2
%% \num{23} c: \act{read}(fy)     // oy <- 1
%% \num{24} c: ret (2,1)
%% \end{alltt} 
\end{minipage}
%
\end{tabular}
\caption{\label{fig:weird:exec} A possible interleaving of the threads
  in~(\subref{fig:weird:code}).}
\end{subfigure}
\caption{\label{fig:weird} An example leading to a scanner miss.%
%\gad{TO DO: Add a proof outline for 2a}
}
\end{figure}
}
%\gad{TO DO: Add a proof outline for 2a}
% \gad{Remember that \small should go inside minifigures!}

 
As Jayanti proves, this implementation is
linearizable. Informally, every overlapping calls to {\tt write} and
{\tt scan} can be rearranged so that it appears as if the calls
occurred sequentially.  To illustrate, consider the program in
Figure~\ref{fig:weird:code}, and one possible interleaving of its
primitive memory operations in Figure~\ref{fig:weird:exec}. The three
threads {\tt l}, {\tt c}, and {\tt r}, start in the state
$\x = 5, \y = 0$.
%
The thread {\tt c} is scheduled first, and through lines~2-6 sets the
scanner bit, clears the forwarding pointers, and reads $\x = 5, \y =
0$. Then {\tt l} intervenes, and in lines~7-11, overwrites $\x$ with
$2$, and seeing $\s$ being set, forwards $2$ to $\fx$. Next, {\tt r}
and {\tt l} overlap, writing $3$ into $\x$ and $1$ into $\y$. However,
while $1$ gets forwarded to $\fy$ (line 13), $3$ is not forwarded to
$\fx$, because $\s$ was turned off in line 15. Thus, when {\tt c}
reads the forwarded values (lines 18, 19), it returns $\x = 2, \y =
1$.

While $\x=2, \y=1$ was never the contents of the memory, returning
this snapshot is nevertheless justified, because we can \emph{pretend}
that the scanner \emph{missed} {\tt r}'s write of $3$. Specifically,
the events in Figure~\ref{fig:weird:exec} can be \emph{reordered} in
time so as to represent the following sequential execution.
%
\begin{equation}
\mathtt{write\, (x, 2);\ write\, (y,1);\ scan\, ();\ write\, (x,
  3)} \label{eq:lin}
\end{equation}
The client programs cannot discover that a different scheduling
actually took place in real time, because they can access the internal
state of the algorithm only via {\tt write} and {\tt scan}.

This kind of temporal reordering is the most characteristic aspect of
linearizability proofs. When carried out on paper, the proofs
typically describe the reordering by listing the linearization points
of each procedure. At a linearization point, the procedure's
operations can be spliced into the execution trace as a homogeneous
chunk. For example, in Jayanti's proof, the linearization point of
{\tt scan} is at line~11 (Figure~\ref{fig:jayanti-snapshot}), where
the scanner bit is unset. The linearization point of {\tt write},
however, may vary. If {\tt write} starts before an overlapping {\tt
  scan}'s line 5, and moreover, the {\tt scan} misses the {\tt write}
(note the dynamic nature of this property), then {\tt write} should
appear after {\tt scan}; that is, the {\tt write}'s linearization
point is right after {\tt scan}'s linearization point at
line~11. Otherwise, {\tt write}'s linearization point is at line~2,
when it commits a value to memory.

Now, rather than formalizing the correctness proof of Jayanti's
algorithm via linearization points, we follow a different path.
Obviously, the high-level pattern of the proof requires tracking a
logical ordering of the events, which differs from the real time. As
the logical ordering is inherently dynamic, depending on properties
such as {\tt scan} missing a {\tt write}, we formalize it in Hoare
logic, by keeping it as a list of events in auxiliary state that can
be dynamically reordered as needed. For example,
Figure~\ref{fig:reorder:before} shows the situation in the execution
of {\tt scan} that we reviewed above, where the logical ordering
(solid line L) coincides with real-time ordering (dashed lines R), but
is unsound for the snapshot $\x=2, \y=1$ that {\tt scan} wants to
return. In that case, the auxiliary code that \emph{we will annotate
  {\tt scan} with}, will change the list in-place, as shown in
Figure~\ref{fig:reorder:after}.

Our challenge then lies in reconciling the following two conflicting
requirements. First, we need to implement the reordering discipline so
that the subsequent calls to {\tt write} and {\tt scan} would preserve
the new logical order encoded in the auxiliary list. This will be
accomplished by introducing yet further structures into the auxiliary
state and code. Second, we have to engineer Hoare triples for {\tt
  write} and {\tt scan} to \emph{not expose} the specifics of this
reordering discipline, which are internal to the snapshot object, but
must remain invisible to the code and proof of the clients. We discuss
these issues next.

\begin{figure}[t]
%\captionsetup[subfigure]{justification=centering}
\begin{subfigure}[t]{0.49\textwidth}
\includegraphics[width=6.1cm]{relink-before3.pdf}
\caption{\label{fig:reorder:before}} % Logical $=$ Real Time order, not a snapshot}
\end{subfigure} \hfill
\begin{subfigure}[t]{0.49\textwidth}
\includegraphics[width=6.1cm]{relink-after3.pdf}
\caption{\label{fig:reorder:after}} % Logical $\neq$ Real Time order, snapshot OK}
\end{subfigure}%
%
\caption{\label{fig:reorder} Changing the logical ordering (solid line
  $\ordlist$) of write events from (5, 0, 2, 3, 1) in
  (\subref{fig:reorder:before}) to (5, 0, 2, 1, 3) in
  (\subref{fig:reorder:after}), to reconcile with {\tt scan} returning
  the snapshot $\x=2, \y=1$, upon missing the write of $3$. Dashed
  lines $\hist$ represent real-time ordering.}
\end{figure}



