% Intro stub

\section{Introduction}
\label{sc:intro}


Formal verification of concurrent data structures commonly requires
reasoning about temporal position of the so called ``linearization
points'' of a program.

A linearization point is a concept related to linearizability, a
standard correctness criteria for concurrent data structures. It is a
point where a certain operation takes place, \emph{logically}.  In
other words, even if the operation physically executes across a whole
interval of moments in time, it is possible and useful to
\emph{pretend}, for reasoning purposes, that it occurred in one moment
--- its linearization point. This simplifies the reasoning about
client programs, which can treat the physical operation as if it were
indivisible, i.e.~\emph{atomic}.

But, reasoning about linearization point of an operation can be
tricky. First, one has to decide just how to shrink to a single point
the whole physical interval in which the operation executed. This
allows for some leeway. For example, operations that execute during
overlapping time intervals can be assigned linearization points that
are ordered arbitrarily. An operation A can be ``linearized'' before
B, despite B beginning before A physically begins, and ending before A
physically ends, as long as A and B overlapped.

Worse, a position of a linearization point for a certain procedure may
not be determined \emph{locally}, but may appear in another procedure
or thread. Moreover, this position may vary depending on past,
\emph{or even future}, run-time events. For instance, a linearization
point of a procedure A, may appear in the procedure B, or in procedure
C, depending on the value of the pointer $x$ in the third line of
procedure D.

\is{The above is a soooper-vague example. I'd recommend to show
  something concrete here, for instance the simple readPair program,
  outlining the difficulty in reasoning.}  
%
\an{Indeed, I'd like an example. But, readpair may be too complicated
  to put here, no?  Moreover, if I recall correctly, it's
  linearization points were all in the same procedure. Why not just
  put a forward pointer here, and say here that an example will be
  given in Section 2? Alternatively, we introduce readpair here, but
  say that in Section 2 we'll show an example where distribution of
  linearization points is even trickier. However, both algorithms will
  be assignable the same spec.}  
%
\an{Can somebody please type up readpair, so we can see how it looks
  in the intro? We need to make good progress on this paper before we
  skype next.} \gad{I'm on it}

The contribution of this paper is a novel method for specifying and
verifying concurrent data structures, which avoids the above problems.

\gad{We should say here that previous/most approaches to reason with
  non-fixed linearization point do so with a dedicated
  meta-theory~\cite{Liang-Fend:PLDI13}. More references needed here}

In particular, we circumvent linearizability, in favor of Hoare-style
specifications in a variant of concurrent separation logic,
FCSL~\cite{LeyWild-Nanevski:POPL13,Nanevski-al:ESOP14}. We will rely
on two key features of FCSL to solve the problems. First, FCSL's
notion of auxiliary state will allow us to track the temporal
positioning of operations (an equivalent of linearization points in
linearizability reasoning) in a \emph{local} way. We do so by
utilizing FCSL abstractions that specify not only what the \emph{self}
thread (i.e., the thread being specified) does, but also what all the
\emph{other} threads in the system do collectively.

Second, FCSL abstracts over the specifics of how the state, real or
auxiliary, of the data structure is implemented. Any structure is
admissible, as long as it satisfies the algebraic axioms of a
\emph{Partial Commutative Monoid} (PCM). Heaps with the operation of
disjoint union are a particular PCM, taken as the default notion of
state in separation logic. But, we can take other PCMs, such as that
of \emph{histories} to keep track of the position of linearization
points. Thus, where separation logic provides effective means for
reasoning about pointer linking (and re-linking) in \emph{space}, in
FCSL, exactly the same logical infrastructure will provide us with
equally effective means for reasoning about linking (and re-linking)
in \emph{time}, which is precisely what the ordering (and re-ordering)
of linearization points of a data structure represents.

Shirking from linearizability in favor of Hoare logic has a few
benefits, in addition of the two listed above\an{In case it's unclear
  which benefits I'm talking about, it's locality of specs, and
  effectiveness of reasoning about time.} First, where linearization
allows that one program is \emph{replaced} by another, which can serve
as the specification for the first, in FCSL we use Hoare-style specs
for specification. This leads to a kind of APIs and specifications for
abstract data types which is standard in sequential programming (e.g.,
a stack is typically specified by assertions such as that a push
followed by a pop does not change the stack, etc.), but so far has not
been used in concurrency. In particular, in this paper, we make the
first proposal of what we believe to be a canonical API for a
concurrent snapshot data structure.
%The API is similar in spirit to the canonical APIs
%one considers for sequential structures, such as stacks or
%queues. 
Second, while linearizability allows replacing a procedure by a
simpler one in a client program, one still needs to consider the
composition of these simpler procedures in the client. Linearizability
offers no help in this process\an{E.g., we may need to introduce
  auxiliary state into the simpler program, but that may destroy the
  connectons with the original program}, whereas our method is
designed precisely with client reasoning in mind.
%
Lastly, our approach immediately lends itself to reasoning about
higher-order programs, whereas scaling linearizability to higher-order
programs has been tricky, and has so far been completed only for very
specific scenarios (CITE some Gotsman stuff)\an{May need to temper
  this statement, to not incense CaReSL guys.}\is{No, we don't have to
  as they do the refinement reasoning, which is the subject of the
  listed above problem with giving meaningful specs. This can go to
  related work.}\an{This sentence would be stronger, if we actually
  considered some higher-order client. It shouldn't be hard to concoct
  one in the section on clients.}

We illustrate our method by verifying in detail a particularly
intricate snapshot algorithm due to Jayanti~\cite{Jayanti+stoc05},
presented in Section 2. We show how the FCSL auxiliary state should be
designed in order to verify this example, in Section 3. We make an
interesting parallel with linearizability, as this auxiliary state has
to keep track of both beginning and ending times of an operation, for
the purposes of the proof, although one of the endpoints can be
omitted in the specs, which we present in the abstract form in Section
4, along with the commentary of the formal proof. This is the first
\emph{formal} proof of Jayanti's algorithm. Moreover, the proof is
mechanized in in Coq. \gad{Actually.. we don't have one yet. This is
  what bothers me the most at this point.} In Section 4, we illustrate
that the same specs can be ascribed to at least one more snapshot
implementation. Following the old adage that a method is a trick which
worked at least twice, this lends credence to the claim that our
snapshot API is canonical. We illustrate how the specs works in a
several simple client program scenarios.





