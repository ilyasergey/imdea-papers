\begin{figure}
%
\centering
\begin{tabular}{c@{\hfill}c}
%  
\begin{minipage}[t][4cm][t]{.5\textwidth}
\begin{alltt}
\num{1}  write (p, v): () \{
\num{2}   \lat\,\actwrite{p}{v}; \Aux{register}(p,v)\rat;
\num{3}   \lat\,b \tbnd \act{read}(S); \Aux{check}(p,b)\rat;
\num{4}   if b
\num{5}   then \lat\,\actwrite{(f_of p)}{v}; \Aux{forward}(p,b))\rat
\num{6}   else  \lat\,\Aux{finalize}(); return ()\rat\}
\end{alltt}
\end{minipage}
%
&
%
\begin{minipage}[t][4cm][t]{.5\textwidth}
\begin{alltt}
\num{ 7}  scan (): \(A {\times} A\)  \{
\num{ 8}  \lat \actwrite{S}{true}; \Aux{setS}(true)\rat;
\num{ 9}  \lat\,\actwrite{fx}{\(\bot\)}; \Aux{clear}(x)\rat; \lat\,\actwrite{fy}{\(\bot\)}; \Aux{clear}(y)\rat;
\num{10}  vx \tbnd \act{read}(x); vy \tbnd \act{read}(y);
\num{11}  \lat \actwrite{S}{false}; \Aux{setS}(false)\rat;
\num{12}  ox \tbnd \act{read}(fx); oy \tbnd \act{read}(fy);
\num{13}  rx \tbnd if (ox \(\neq\bot\)) then ox else vx;  
\num{14}  ry \tbnd if (oy \(\neq\bot\)) then oy else vy;  
\num{15}  \lat\,\Aux{relink}(rx, ry); return (rx, ry)\,\rat\}
\end{alltt} 
\end{minipage}
%
\end{tabular}
%
\caption{Snapshot procedures annotated with auxiliary code.}
\label{fig:fcsl-snapshot}
\end{figure}


\begin{comment}

I actually want ot be able to implement the code like above, it would
allow us to express better and more precise information about
end-times, and decouple finalize from exit. It would take me a day or
so to do it, but I think it might be worht it.

The correct version should be

\begin{figure}
%
\centering
\begin{tabular}{c@{\hfill}c}
%  
\begin{minipage}[t][4cm][t]{.5\textwidth}
\small
\begin{alltt}
\num{1}  write (p, v): () \{
\num{2}   \lat\,\actwrite{p}{v}; \Aux{register}(p,v)\rat;
\num{3}   \lat\,b \tbnd \act{read}(S); \Aux{check}(p,b)\rat;
\num{4}   if b
\num{5}   then \lat\,\actwrite{(f_of p)}{v}; \Aux{forward}(p,b))\rat
\num{6}   else skip;
\num{7}   \lat\,\Aux{finalize}(); return ()\rat\}
\end{alltt}
\end{minipage}
%
&
%
\begin{minipage}[t][4cm][t]{.5\textwidth}
\small
\begin{alltt}
\num{ 8}  scan (): \(A {\times} A\)  \{
\num{ 9}  \lat \actwrite{S}{true}; \Aux{setS}(true)\rat;
\num{10}  \lat\,\actwrite{fx}{\(\bot\)}; \Aux{clear}(x)\rat; \lat\,\actwrite{fy}{\(\bot\)}; \Aux{clear}(y)\rat;
\num{11}  vx \tbnd \act{read}(x); vy \tbnd \act{read}(y);
\num{12}  \lat \actwrite{S}{false}; \Aux{setS}(false)\rat;
\num{13}  ox \tbnd \act{read}(fx); oy \tbnd \act{read}(fy);
\num{14}  rx \tbnd if (ox \(\neq\bot\)) then ox else vx;  
\num{15}  ry \tbnd if (oy \(\neq\bot\)) then oy else vy;  
\num{16}  \lat\,\Aux{relink}(rx, ry); return (rx, ry)\,\rat\}
\end{alltt} 
\end{minipage}
%
\end{tabular}
%
\caption{Snapshot procedures annotated with auxiliary code.}
\label{fig:fcsl-snapshot}

\end{figure}
\end{comment}


\section{Auxiliary code and sketch of the proof}
\label{sc:implementation}

%\subparagraph*{Implementation}%

Figure~\ref{fig:fcsl-snapshot} annotates Jayanti's procedures with
auxiliary code (typed in \emph{italic}), with
$\langle\mbox{angle braces}\rangle$ denoting that the enclosed real
and auxiliary code execute \emph{simultaneously}. The auxiliary code
builds the object histories and evolves the logical ordering in-place,
while respecting the invariants from Section~\ref{sc:formal}. Thus, it
is the \emph{constructive} component of our proofs.

%
%As customary, auxiliary code is \emph{erasable} at run time; it does
%not mutate the real state.
%
%% We name auxiliary code as one would name procedure calls, and proceed
%% to describe these procedures.
%

The auxiliary code is divided into several procedures, all of which
are sequences of state reads followed by updates to auxiliary
variables. We present them as Hoare triples in
Figure~\ref{fig:auxcode}, listing reads in the precondition, and
writes in the postcondition, with the unmentioned state considered
unchanged. Following are the characteristic cases.

%The main auxiliary is $\aux{relink}$ in {\tt scan} which decides the
%correctness of snapshot, and whether or not the underlying order
%$\tleq$ has to change. The rest of them are merely doing changes in
%the auxiliary variables in order to provide $\aux{relink}$ with enough
%information to operate. For completeness sake, we will introduce and
%briefly describe all of them, but the reader might consider skipping
%them and go straight down to focus on $\aux{relink}$. We start with
%the auxiliary code for the {\tt write} method, as listed on the first
%column in Figure~\ref{fig:fcsl-snapshot}:
%

%We present a Hoare-style specification of the auxiliary code so as to
%provide an intuition on how the system evolves. We present the
%auxiliary code transitions as if they act only on parts of the state
%-- e.g. $\wstate{x}$, $\hist$ -- but in fact, they act on the whole
%state of the resource and each of them is required to preserve the
%invariants of the joint state described above. We begin with {\tt
%write}'s auxiliary code:


%
%% \[
%% \begin{array}{r c l}
%% %% register
%%  \{\ \hist= h,\ \histJ= j,\ \wstate{p} = \wInit\}
%%   & \aux{register}(p,v) &
%%   \{ \!\!\!
%%   \begin{array}[t]{l}
%%    \mathit{let}\ t = \mathsf{fresh}(\hist),\ w = t \hpts (p,v) \\
%%    \mathit{in}\ \histJ= j \hunion w ,\ \wstate{p} = \wWrite\ t\\
%%    \phantom{\mathit{in}}\ \hist = h \hunion (w,
%%    \mathit{if}\, ({\mathtt S} \wedge S_p)\, \mathit{then}\
%%     \mathbf{yellow}\ \mathit{else}\ \mathbf{red})\}
%%   \end{array}\\
%% %% checkS
%%   \{\ \wstate{p} = \wWrite\, t\} & \aux{check}(p,b) &
%%   \{\ \wstate{p} = \mathit{if}\ b\
%%   \mathit{then}\ \wDirty\, t\ \mathit{else}\ \wClean\, t\}\\
%% %% transfer  
%%   \{\ \wstate{p} = \wDirty\, t,\ \hist = h\} & \aux{forward}(p) &
%%   \{\!\!\!
%%   \begin{array}[t]{l}
%%    \wstate{p} = \wClean\, t,\\
%%    \hist= \mathit{if}\ ({\mathtt S} \wedge S_p)\,
%%    \mathit{then}\ (\mathsf{paint}\ [t]\ \mathsf{green}\ h)\ \mathit{else}\ h\}
%%   \end{array}\\
%% %% exit  
%%   \{\!\!\! \begin{array}[t]{l}
%%     \wstate{p} = \wClean\, t,\ \histS = h, \\
%%     \histJ = j \hunion t \hpts (p,v), E = e\}
%%   \end{array}\! & \aux{exit}(p) &
%%    \{\!\!\! \begin{array}[t]{l}
%%      \wstate{p} = \wInit,\ \histS = h \hunion t \hpts (p,v),\\
%%      \histJ = j,\ E = e \hunion t \hpts \mathsf{max}\ \dom{\hist}\}
%%    \end{array}
%% \end{array}
%% \]

For example, $\aux{register}(p, v)$ creates the write event for the
assignment of $v$ to $p$ in line~2. It allocates a fresh timestamp
$t$, inserts the entry $t \mapsto (p, v)$ into $\histJ$, and adds $t$
to the end of $\ordlist$, thus making $t$ be the last write event,
until and unless $\ordlist$ is permuted later. The variable $\wpp$
records that the writer finished line~2 with a fresh $t$. The color of
$t$ is set to yellow (i.e., the rearrangement status of the event is
left undetermined), if $(\sss= \sOn) \& \spp$, i.e., an active scanner
is in line 10. Otherwise, $t$ is colored red, indicating that the
logical order of $t$ will be determined by a future active scan.
%
%In line~3, $\aux{check}(p,b)$, depending on $b$, marks the current
%write as $\wDirty$ (hence, a scan is in progress, and we should
%forward the write), or as $\wClean$, and ready to terminate.
%

In line~5, \aux{forward} paints the allocated timestamp $t$ green, if
an active scanner is in line~10, because such a scanner will
definitely see the write, either by reading the original value in
line~10, or by reading the forwarded value in line~12. Thus, the
logical order of $t$ becomes fixed. In fact, it is possible to derive
from the invariants in Section~\ref{sc:formal}, that this order is the
same one $t$ was assigned at registration, i.e., the linearization
point of this write is line~2.

%
%$\aux{Finalize}$ moves the current write event from the joint history
%$\histJ$ to the thread's self history $\histS$, thus finalizing
%it. The current maximum timestamp of $\hist$ is recorded in $\E$ as
%the ending time of the event. 
%

%$\aux{setS}$ toggles the scanner state $\sss$ on and off. When
%executed in line 11, it records the current maximum timestamp as
%$\toff$. Note that this does not create a fresh timestamp, but rather
%selects that of the last write event in $\hist$.
%

%The procedure $\aux{clear}(p)$ is executed in line~9 simultaneously
%with clearing the forward pointer for $p$. In addition to recording
%that the scanner passed line~9 by setting the $\spp$ bit, it paints
%the sub-history $\histp$ green. Thus, all the previous writes to $p$
%will be visible by the ongoing scan.
%%, and hence accounted for in the remaining machinery of
%%the scan, with their ordering being fixed forever.


{
\setlength{\belowcaptionskip}{-5pt}
\begin{figure}[t]
%
\centering
\begin{subfigure}[t]{1\textwidth}
\small
\[
\begin{array}{l@{\, :\ } l}
%% register
 \aux{register}(p,v) &
  \!\!\! \begin{array}[t]{l}
   \{\ordlist = l,\ \histJ = j,\ \wpp = \wInit,\, \C = c\}\\ 
   \{\ordlist = \mathsf{snoc}\ l\ t,
     \histJ = j \hunion t \hpts (p,v), \wpp = \wWrite\ t,\\
   \phantom{\{} C = \textrm{if}\ (\sss = \sOn) \& \spp\
                    \textrm{then}\ c[t \mapsto \mathsf{yellow}]\
                    \textrm{else}\ c[t \mapsto \mathsf{red})]\}
   \ \mbox{\small{where $t = \mathsf{fresh}\ \hist$}}
%c \hunion
%(t \mapsto \textrm{if}\ \s \& \spp\ \textrm{then}\ \mathsf{green}\
%\textrm{else}\ \mathsf{yellow})\}\quad\mbox{where $t = \mathsf{fresh}\ \hist$}
  \end{array} \\[2.5pt]
%% checkS
 \aux{check}(p,b) & \{\wpp = \wWrite\ t\}\quad
  \{\wpp = \textrm{if}\ b\
  \textrm{then}\ \wDirty\ t\ \textrm{else}\ \wClean\ t\}\\
%% transfer  
  \aux{forward}(p) &
  \!\!\! \begin{array}[t]{l}
   \{\wpp = \wDirty\ t,\, \C = c \}\\ 
   \{\wpp = \wClean\ t,\, %
   \C= \textrm{if}\ (\sss=\sOn) \& \spp\ \textrm{then}\ c[t \mapsto \mathsf{green}]\ \textrm{else}\ c\}
  \end{array}\\
%% exit  
  \aux{finalize}(p) &
  \!\!\! \begin{array}[t]{l}
  \{\wpp = \wClean\ t,\ \histS = h,\, %
  \histJ = j \hunion t \hpts (p,v), E = e \}\\
  \{\wpp = \wInit,\, \histS = h \hunion t \hpts (p,v),\, %
     \histJ = j,\ E = e \hunion t \hpts \mathsf{last}\ \hist \}
 \end{array}
\end{array}
\]\vspace{-8mm}
\caption{\label{fig:writeauxcode}}
\hrulefill
\end{subfigure}
\begin{subfigure}[b]{1\textwidth}
\small
\[\begin{array}{l@{\, :\ }l}
%% setbit
  \aux{setS}(b) &
  \!\!\! \begin{array}[t]{l}
            \{ \sss = \textrm{if}\ b\ \textrm{then}\ \sOff\ \_ \ 
            \textrm{else}\ \sOn,\ \sx = \neg\, b, \sy = \neg\, b \}\\
            \{ \sss = \textrm{if}\ b\ \textrm{then}\ \sOn\ %
            \textrm{else}\ \sOff\ (\mathit{max}\, (\dom{H})),\
             \sx = \neg\, b, \sy = \neg\, b \}

\end{array}\\[2.5pt]      
%% clear
   \aux{clear}(p) &
      \{\sss = \sOn,\ \spp = \FF,\ \C = c \}\quad
      \{\sss = \sOn,\ \spp = \TT,\
      \C = c[\histp \hpts \mathsf{green}] \}\\[2.5pt]
%% re-link
   \aux{relink}(r_x, r_y) &
  \!\!\! \begin{array}[t]{l}
    \{\sss = \sOff\ \toff, \C = c, \ordlist = l,\, %
%      t_x = \admissible\ \histx, t_y = \admissible\ \histy, rx = \histx(t_x), r_y = \histy(t_y) \}\\
        \hist = h \hunion t_x \hpts (x, r_x) \hunion t_y \hpts (y, r_y),\\
      %% \ (t_x = \aux{last\_green}\ \histx \vee
      %%     \C(t_x) = \mathsf{yellow}),
      %% \ (t_y = \aux{last\_green}\ \histy \vee
      %%     \C(t_y) = \mathsf{yellow}) \}\\
      \ p \in \{x,y\}:\ \spp =\TT, (t_p = \aux{last\_green}\ \histp \vee
                \C(t_p) = \mathsf{yellow})\}\\
     \{\sss = \sOff\ \toff, \sx=\FF, \sy=\FF,%
        \C = c[t_x, t_y \hpts \mathsf{green}],\\
      \ \ordlist = \textrm{if}\ (d = \mathsf{Yes}\ x\ t')\
                \textrm{then}\ \aux{push}\ t'\, t_y\, l\\
                 \phantom{\ \ordlist =\ } \textrm{else\ if}\
                 (d = \mathsf{Yes}\ y\ t')\ \textrm{then}\
                 \aux{push}\ t'\, t_x\, l\ \textrm{else}\ l\}\
  \quad \mbox{where $d = \aux{inspect}\ t_x\, t_y\, l\, \C$}
  \end{array}
\end{array}
\]\vspace{-8mm}
\caption{\label{fig:scanauxcode}}
\end{subfigure}
\caption{\label{fig:auxcode} Auxiliary procedures for (\subref{fig:writeauxcode}) {\tt write} and (\subref{fig:scanauxcode}) {\tt scan}. Small caps variables (e.g., $l$) record the pre-values of auxiliaries (e.g., $L$).}
\end{figure}
}


The key auxiliary procedure of our method is $\aux{relink}$. It is
executed just before the scanner returns the pair $(r_x, r_y)$. Its
task is to modify the logical order of the writes, to make $(r_x,
r_y)$ \emph{appear} as a valid snapshot. This will always be possible
under the precondition of $\aux{relink}$ that the timestamps $t_x$,
$t_y$ of the events that wrote $r_x$, $r_y$ respectively, are either
the last green or the yellow ones in the respective histories $\histx$
and $\histy$, and $\aux{relink}$ will consider all four cases. This
precondition holds after line~14 in Figure~\ref{fig:fcsl-snapshot}, as
one can prove from the algorithm invariants (e.g.,
Propositions~\ref{inv:color} and~\ref{inv:readFP}).
%
%Given $(r_x,r_y)$, $\aux{relink}$ does is to identify their timestamps
%$t_x$ and $t_y$. The precondition of $\aux{relink}$ ensures that this
%is either the yellow or--- if there is no such ---the last green
%timestamp in each of $\histx$ and $\histy$. This comes as a result of
%Propositions~\ref{inv:color} and~\ref{inv:readFP} above, amongst
%others.

%% It starts by finding the timestamps $t_x$ and $t_y$ that are
%% responsible for writing $r_x$ and $r_y$ into $\histx$ and $\histy$,
%% respectively. There may be many such timestamps, but we focus on the
%% ones that are yellow, or last green in the respective subhistories of
%% their pointer. It follows from invariants (\ref{inv:forward}) and
%% (\ref{inv:scanner}) that such must exist.
%

$\aux{Relink}$ uses two helper procedures $\aux{inspect}$ and
$\aux{push}$, to change the logical order. $\aux{inspect}$ decides if
the selected $t_x$ and $t_y$ determine a valid snapshot, and
$\aux{push}$ performs the actual reordering. The snapshot determined
by $t_x$ and $t_y$ is valid if there is no event $t'$ such that
$t_x \tle t' \tle t_y$ and $t'$ is a write to $x$ (or, symmetrically
$t_y \tle t' \tle t_x$, and $t'$ is a write to $y$). If such $t'$
exists, $\aux{inspect}$ returns $\mathsf{Yes}\ x\ t'$ (or
$\mathsf{Yes}\ y\ t'$ in the symmetric case). The reordering is
completed by $\aux{push}$, which moves $t'$ right after $t_y$ (after
$t_x$ in the symmetric case) in $\tleq$. Finally, $\aux{relink}$
paints $t_x$ and $t_y$ green, to fix them in $\jleq$. We can then
prove that $(r_x, r_y)$ is a valid snapshot wrt.~$\jleq$, and remains
so under interference. 
%
Notice that the timestamp $t'$ returned by $\aux{inspect}$ is always
uniquely determined, and yellow. Indeed, since $t_x$ and $t_y$ are not
red, no timestamp between them can be red either
(Proposition~\ref{inv:redzone}). If $t_x \tle t' \tle t_y$ and $t'$ is
a write to $x$ (and the other case is symmetric), then $t_x$ must be
the last green in $\histx$, forcing $t'$ to be the unique yellow
timestamp in $\histx$, by Proposition~\ref{inv:color}.

%
To illustrate, in Figure~\ref{fig:reorder:before} we have $r_x = 2$,
$r_y = 1$, $t_x$ and $t_y$ are both the last green timestamp of
$\histx$ and $\histy$, respectively, and $t_x \tle t_y$. However,
there is a yellow timestamp $t'$ in $\histx$ coming after $t_x$,
encoding a write of $3$. Because $t_x \tle t' \tle t_y$, the pair
$(r_x, r_y)$ is not a valid snapshot, thus $\aux{inspect}$ returns
$\mathsf{Yes}\ x\ t'$, after which $\aux{push}$ moves $3$ after $1$.

%Next, $\aux{relink}$ uses the helper procedure $\aux{inspect}$ to
%decide if selected $t_x$ and $t_y$ determine a valid snapshot (i.e.,
%there are no other events between $t_x$, $t_y$ that make $(r_x, r_y)$
%not be a snapshot).
%
%By way of example, in Figure~\ref{fig:reorder:before} we have $r_x =
%2$, $r_y = 1$, $t_x$ and $t_y$ are both the last green timestamp of
%$\histx$ and $\histy$, respectively, and $t_x \tle t_y$. However,
%there is a yellow timestamp $t'$ in $\histx$ coming after $t_x$,
%encoding a write of $3$. Because $t_x \tle t' \tle t_y$, the pair
%$(r_x, r_y)$ is not a valid snapshot. $\aux{inspect}$ will recognize
%the situation, and indicate to $\aux{relink}$ that $t'$ has to be
%reordered by returning $\mathsf{Yes}\ x\ t'$. The move is completed by
%the helper function $\aux{push}$, which reorders $t'$ right after
%$t_y$ in $\tleq$, as shown in Figure~\ref{fig:reorder:after}.
%Finally, $\aux{relink}$ paints $t_x$ and $t_y$ green, to fix them in
%$\jleq$. We can then prove that $(r_x, r_y)$ is a valid snapshot
%wrt.~$\jleq$, and remains so under interference.

% \gad{TO DO: The explanation for relink is still a bit too
% long.}\gad{Do Do I need to explain all auxiliaries? we might consider
% explaining only relink and pushing the rest to the appendices.}

\begin{comment}
%
$\aux{relink}$ works as follows: first, the auxiliary function 
$\mathsf{decide}$ checks $l$ to see if $t_x$ and $t_y$ form a valid 
snapshot, in which case it will return $\mathsf{No}$ and then 
$\ordlist = l$, else it will return $\mathsf{Yes}\ p\ t_r$, indicating 
that there is a miss and that $t_r \in \hist_{p}$--- and is such that 
$t_r \leq t_{\neq p}$ ---should be {\it pushed} in $\ordlist$ past 
$t_{\neq p}$, the latter being $t_y$ if $p=x$, and vice 
versa. Finally, the returned timestamps are painted green.



In Figure~\ref{fig:relink}, we revisit the example from
Section~\ref{sc:overview}, adding the colors to the time-stamps. There
we see in Figure~\ref{fig:relink:before} that $(r_x,ry)$ points to
$(2,1)$ and both timestamps are green. We notice that there is a
yellow in between, $\mathsf{yellow}\, \histx$.  In
Figure~\ref{fig:relink:after}, we see that it has been pushed after
$r_y$.

$\aux{Relink}$ works as follows: first, the auxiliary function 
$\mathsf{decide}$ checks $l$ to see if $t_x$ and $t_y$ form a valid 
snapshot, in which case it will return $\mathsf{No}$ and then 
$\ordlist = l$, else it will return $\mathsf{Yes}\ p\ t_r$, indicating 
that there is a miss and that $t_r \in \hist_{p}$--- and is such that 
$t_r \leq t_{\neq p}$ ---should be {\it pushed} in $\ordlist$ past 
$t_{\neq p}$, the latter being $t_y$ if $p=x$, and vice 
versa. Finally, the returned timestamps are painted green.

We clarify how $\mathsf{decide}$ works: without any loss of
generality, assume $t_x \tleq t_y$. The pre-condition of
$\aux{relink}$ says $t_x$ and $t_y$ are, respectively, the last green
or yellow timestamps of $\histx$ and respectively $\histy$. From the
latter facts facts and Invariant~\ref{inv:redzone}, we know that every
timestamp in the chain from $t_x \tleq t_y$ will be green or yellow
and there will be, at most, two yellow timestamps, one for each
$\histx$ and $\histy$. Then, if $t_x$ is yellow, $\mathsf{decide}$
will return $\mathsf{No}$-- if there are further elements in $\histx$,
they will be in the red tail, and outside the chain-- e.g. the token 7
in Figure~\ref{fig:relink}. Now, if $t_x$ is green and there is no
yellow key in $\histx$, $\mathsf{decide}$ will return
$\mathsf{No}$. If there is, we need to compare it with $t_y$: if
$\mathsf{yellow}\ \histx \tle t_y$, as in Figure~\ref{fig:relink},
$\mathsf{decide}$ will return $\mathsf{Yes}\ x\
(\mathsf{yellow}\, \histx)$ and the latter will be pushed right after
$t_y$ in \ordlist-- $\mathsf{push}$ implements the pointer-swing-like
manipulation on $l$. Last, if $ t_y \tle \mathsf{yellow}\ \histx$
there will be no $\mathsf{push}$ either.

\end{comment}

%% \gad{The figure might change to include the red-zone invariant as well,
%%   as the endpoints if we need to}. Here give a proof-sketch on how
%% re-link satisfies the invariants.
