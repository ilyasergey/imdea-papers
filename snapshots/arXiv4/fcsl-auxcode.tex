
{
%\setlength{\belowcaptionskip}{-5pt}
\begin{figure}[t]
%
\centering
%\begin{subfigure}[t]{1\textwidth}
\small
\[
\begin{array}{l@{\, :\ } l}
%% register
 \aux{register}(p,v) & 
  \begin{array}[t]{l}
   \langle \wpp = \wInit\rangle\\ 
   \langle\ordlistP = \mathsf{snoc}\ {\ordlist}\ t,\
     \histJP = \histJ \hunion t \hpts (p,v),\ \wppP = \wWrite\ t\, v,\\
   \phantom{\langle} \C' = \textrm{if}\ (\sss = \sOn) \& \spp\
                    \textrm{then}\ \C[t \mapsto \mathsf{yellow}]\
                    \textrm{else}\ \C[t \mapsto \mathsf{red}]\rangle \\
   \ \mbox{\small{where $t = \mathsf{fresh}\ \hist = \mathsf{max}\, (\dom{\hist})+1$}}
%c \hunion
%(t \mapsto \textrm{if}\ \s \& \spp\ \textrm{then}\ \mathsf{green}\
%\textrm{else}\ \mathsf{yellow})\}\quad\mbox{where $t = \mathsf{fresh}\ \hist$}
  \end{array} \\[2.5pt]
%% checkS
% Im placing it within an  array env so it aligns nicely with the rest
  \aux{check}\,(p,b) & [t, v]\ldot
  \!\begin{array}[t]{l}
  \langle\wpp = \wWrite\ t\, v\rangle\ 
  \langle\wppP = \textrm{if}\ b\
  \textrm{then}\ \wDirty\ t\, \,v\ \textrm{else}\ \wClean\ t\, v\rangle
 \end{array}\\[2.5pt]
%% transfer   
  \aux{forward}\,(p) & [t, v]\ldot
  \!\begin{array}[t]{l}
   \langle\wpp = \wDirty\ t\,v \rangle\\ 
   \langle\wppP = \wClean\ t\,v,\, %
   \C' = \textrm{if}\ (\sss=\sOn) \& \spp\ \textrm{then}\ \C[t \mapsto \mathsf{green}]\ \textrm{else}\ \C\rangle
  \end{array}\\[2.5 pt]
%% exit  
  \aux{finalize}(p) & [t, v]\ldot
  \!\begin{array}[t]{l}
  \langle\wpp = \wClean\ t\, v, %
  t \hpts (p, v) \in \histJ \rangle\\
  \langle\wppP = \wInit,\, \histSP = \histS \hunion t \hpts (p,v),\, %
  \histJP\! = \histJ\setminus\{t\},\,
  \E'\! = \E \hunion\! t \hpts \mathsf{max}\, (\dom{\hist}) \rangle
 \end{array}
\end{array}
\]
%\caption{\label{fig:writeauxcode}}
\hrulefill
%\end{subfigure}
%
%\begin{subfigure}[b]{1\textwidth}
\small
\[
\begin{array}{l@{\, :\ }l}
%% setbit
  \aux{set}(b) &
  \begin{array}[t]{l}
        \langle \sss = \textrm{if}\ b\ \textrm{then}\ \sOff\,(\_) \ 
        \textrm{else}\ \sOn,\ \sx = \neg\, b, \sy = \neg\, b \rangle\\
        \langle \sss' = \textrm{if}\ b\ \textrm{then}\ \sOn\ %
        \textrm{else}\ \sOff\,(\mathsf{last}\ \hist),\
         \sx' = \neg\, b, \sy' = \neg\, b \rangle
\end{array}\\[2.5pt] 
%% clear
   \aux{clear}(p) &
  \begin{array}[t]{l}
   \langle\sss = \sOn,\ \spp = \FF \rangle\\
   \langle\sss' = \sOn,\ \spp' = \TT,\
     \CP = \C[\histp \hpts \mathsf{green}] \rangle
  \end{array}\\[2.5pt]
%% re-link
   \aux{relink}(r_x, r_y) & [t_x, t_y]\ldot
    \begin{array}[t]{l}
    \langle\sss = \sOff(\_), %
      t_x \hpts (x, r_x), t_y \hpts (y, r_y) \in \hist, \sx = \sy = \TT, \\
%      \ p \in \{x,y\}:\ \spp =\TT, (t_p = \aux{last\_green}\ \histp \vee
%                \C(t_p) = \mathsf{yellow})\}\\
     \hphantom{\langle} \forall p \in \{x,y\}\ldot \lgVy\ p\, t_p \rangle\\
        \langle\sss' = \sss, \sx'= \sy'=\FF,%
        \CP = \C[t_x, t_y \hpts \mathsf{green}],\\
      \ \ordlist' = \textrm{if}\ (d = \mathsf{Yes}\ x\ s)\
                \textrm{then}\ \aux{push}\ s\ t_y\ \ordlist\\
                 \phantom{\ \ordlist =\ } \textrm{else\ if}\
                 (d = \mathsf{Yes}\ y\ s)\ \textrm{then}\
                 \aux{push}\ s\ t_x\ \ordlist\ \textrm{else}\ \ordlist\rangle\\
  \quad \mbox{where $d = \aux{inspect}\ t_x\, t_y\, \ordlist\, \C$}
  \end{array}
\end{array}
\]
%\caption{\label{fig:scanauxcode}}
%\end{subfigure}
\caption{\label{fig:auxcode} Auxiliary procedures for
%  (\subref{fig:writeauxcode}) 
\jywrite~and 
%(\subref{fig:scanauxcode})
  \jyscan. Bracketed variables (\eg, $[t, v]$) are logical variables
  that scope over precondition and postcondition.}
\end{figure}
}

%\gad{I'm not sure we want to keep the $(t_p =
%  \aux{last\_green}\ \histp \vee \C(t_p) = \mathsf{yellow})$ part, as
%  it is not the pre-variable relating to some auxiliary state
%  projection. This will appear later in the pre and post condition of
%  the atomic commands so it seems repetitive here. Note that the
%  ACTUAL implementation does take a proof that the values are pointed
%  by the last green or yellow timestamp, but there is no need to show
%  that here.}
%
%\gad{In the next section , I'm introducing this as an assertion. So I
%  guess that settles it.}
