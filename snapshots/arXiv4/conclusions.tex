\section{Conclusions}
\label{sec:conclusions}

The paper illustrates a new approach allowing one to specify that the
execution history of a concurrent data structure can be seen as a
\emph{sequence of atomic events}. The approach is thus similar in its
goals to linearizability, but is carried out exclusively using a
separation-style logic to uniformly represent the state and time
aspects of the data structure and its methods.

Reasoning about time using separation logic is very effective, as it
naturally supports \emph{dynamic and in-place updates} to the temporal
ordering of events, much as separation logic supports dynamic and
in-place updates of spatially linked lists. The need to modify the
ordering of events frequently appears in linearizability proofs, and
has been known to be tricky, especially when the order of a terminated
event depends on the future. In our approach, the modification becomes
a conceptually simple manipulation of auxiliary state of histories of
colored timestamps.

%The uniformity provided by separation logic is further important. It
%enables that a program, verified against a spec that shows the
%atomicity of its events, can easily be combined with other programs
%whose specs are unrelated to atomicity or event ordering.
%%
%Moreover, when compared to linearizability, which equates a method to
%an atomic program, our specification method is somewhat stronger, as
%it allows one to specify the discipline of temporal reordering of
%events. This is important, as the more expressive specs can directly
%be used in client reasoning.

%We have illustrated the approach on a snapshot algorithm of
%Jayanti~\cite{Jayanti+STOC05}, whose linearizability proof exhibits
%exactly such dependence on the future. This is apparent in our code
%and proofs, where a call to \jywrite\ method may terminate with its
%timestamps being colored yellow, \ie, available for reordering in the
%future by a concurrent \jyscan.

%We have carried out and mechanized the proof in an off-the-shelf logic
%FCSL, without needing any additions to the logic. To the best of our
%knowledge, this establishes FCSL as the first program logic capable of
%mechanically proving full functional correctness of data structures
%whose linearizability proofs exhibit non-locality \emph{and}
%dependence on future events.
%%
%% \an{Fix the above sentence, depending on what is said in the related
%%   work section. Also, consider putting this line in the intro too.}
%%

We have carried out and mechanized our proof of Jayanti's
algorithm~\cite{Jayanti+STOC05} in FCSL, without needing any additions
to the logic.
%
%% To the best of our knowledge, this establishes FCSL as the first
%% program logic capable of mechanically proving full functional
%% correctness of data structures whose linearizability proofs exhibit
%% non-locality \emph{and} dependence of terminated events on the
%% future.
%
Such development, together with the fact that FCSL has previously been
used to verify a number of non-trivial concurrent
structures~\cite{SergeyNB+ESOP15,SergeyNB+PLDI15,SergeyNBD+OOPSLA16},
gives us confidence that the approach will be applicable, with minor
modifications, to other structures whose linearizations exhibit
dynamic dependence on the
future~\cite{DoddsHK+POPL15,Morrison-al:PPoPP13,Hoffman-al:OPODIS07}.

% \an{There's some overlap with related work
%   here. Reconcile once related work is written.}

One modification that we envision will be in the design of the data
type of timestamped histories. In the current paper, a history of the
snapshot object needs to keep only the \jywrite\ events, but not the
\jyscan\ events. In contrast, in the case of stacks, a history would
need to keep both events for push and pop operations. But in FCSL,
histories are a \emph{user-defined} concept, which is not hardwired
into the semantics of the logic. Thus, the user can choose any
particular notion of history, as long as it satisfies the properties
of a Partial Commutative
Monoid~\cite{LeyWildN+POPL13,NanevskiLSD+ESOP14}. Such a history can
track pushes and pops, or any other auxiliary notion that may be
required, such as, \eg,~specific ordering constraints on the events.

% \gad{In the same spirit, we can also say again here that the
%   flexibility of not having histories or its invariant hardwired into
%   the model, has allowed us to implement non-linearizable behavior as
%   well \cite{SergeyNBD+OOPSLA16}. For instance, we could relax
%   Invariant~\ref{inv:overlap} in order to implement other criteria
%   which do not impose the preservation of real time order between
%   non-overlapping events.}

% \gad{I'm almost tempted to say we could do Local-Linearizability
%   easily =).}

\begin{comment}
%% \gad{which is true, we might finish
%% while they are reviewing and, if I want to have a thesis by the end of
%% the year, surely before the notification}.
\begin{itemize}
\item sum up our contributions
  
\item Reinforce the fact that we have the first formal proof of
  Jayanti's concurrent object.

\item Say something about the formalization experience. 

\item Claim we can cope with the elephants in the room:
  ~\cite{Dodds-al:POPL15 ,PetrankT+DISC13,
  HerlihyW+TOPLAS90}.
\end{itemize}

\paragraph{What's next}
Immediate task is to use the snapshots for something good,
e.g. Iterators \cite{PetrankT+DISC13}. We claim that our method could
scale up easily to tackle other usual suspects:
  
\begin{itemize}
\item  Dodds et.al.'s stack~\cite{Dodds-al:POPL15}

\item Petrank and Timnat's iterators based upon Jayanti's snapshot
  construction~\cite{PetrankT+DISC13}

\item Herlihy--Wing's queue~\cite{HerlihyW+TOPLAS90}.

\item \gad{Elimination Queues!! References!!}.

\item \gad{Conditional
  CAS??~\cite{HarrisFP+DISC02,FraserH+TOPLAS07}. Check Turon et al}.
\end{itemize}

\paragraph*{Mechanization} \gad{It might
  be honest to admit here that the resulting code effort is not a neat
  as it could be.} The process of carrying out the formalization of
Jayanti helped us realize that FCSL's metatheory only allows for
combining resources -- with their transitions and atomics -- from a
communication/ownership transfer
perspective~\cite{NanevskiLSD+ESOP14}. Whereas here, the transitions
and atomic commands hints there is a different structure going on, and
we look forward to acknowledge such structures in the next version of
FCSL. We hope that in the future, a lot of \emph{boilerplate proofs}
can be reduced further.
\end{comment}

