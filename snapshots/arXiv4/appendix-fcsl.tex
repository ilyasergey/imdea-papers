\section{A brief introduction to FCSL}
\label{sc:background}

A state of a resource in FCSL~\cite{NanevskiLSD+ESOP14}, such as that
of snapshot data structure discussed in this paper, always consists of
three distinct auxiliary variables that we name $a_\lcl$, $a_\env$ and
$a_\joint$. These stand for the abstract self state, other state, and
shared (joint) state.

However, the user can pick the types of these variables based on the
application. In this paper, we have chosen $a_\lcl$ and $a_\env$ to be
histories, and have correspondingly named them $\histS$ and $\histO$.
On the other hand, $a_\joint$ consists of all the other auxiliary
components that we discussed, such as the variables $\histJ$, $\E$,
$\C$, $\sx$, $\sy$, $\wx$ and $\wy$. These variables become merely
projections out of $a_\joint$.

It is essential that $a_\lcl$ and $a_\env$ have a common type, which
moreover, exhibits the algebraic structure of a \emph{partial
  commutative monoid} (PCM). A PCM requires a partial binary operation
$\bullet$ which is commutative and associative, and has a unit. 
%
%In the case of histories, the partial operation is disjoint union
%$\hunion$ with $\mathsf{empty}$ history as the unit.
%
PCMs are important, as they give a generic way to define the inference
rule for parallel composition.
%
\[
%\tag{\normalsize \arabic{tags}}\refstepcounter{tags}\label{eq:parrule}
\hfill\begin{array}{c}
  e_1 : \specK{\{P_1\}}\ A\ \specK{\{Q_1\}} @ C \quad
  e_2 : \specK{\{P_2\}}\ B\ \specK{\{Q_2\}} @ C \\[2pt]
\hline\\[-7pt]
e_1 \parallel e_2 : \specK{\{P_1 \circledast P_2\}}\ (A \times B)\
\specK{\{[r.1/r]Q_1 \circledast [r.2/r]Q_2\}} @ C
\end{array}\hfill
\]
Here, $\circledast$ is defined over state predicates $P_1$ and $P_2$
as follows.
\[
%\tag{\normalsize \arabic{tags}}\refstepcounter{tags}\label{eq:ssep}
\hfill\begin{array}{c}
  (P_1 \circledast P_2) (a_\lcl, a_\joint, a_\env) \iff
  \exists x_1~x_2\ldot\
  a_\lcl = x_1 \bullet x_2 \wedge
  P_1 (x_1, a_\joint, x_2 \bullet a_\env) \wedge
  P_2 (x_2, a_\joint, x_1 \bullet a_\env)
\end{array}\hfill
\]

The inference rule, and the definition of $\circledast$, formalize the
intuition that when a parent thread forks $e_1$ and $e_2$, then $e_1$
is part of the environment for $e_2$ and vice-versa. This is so
because the \emph{self} component $a_\lcl$ of the parent thread is
split into $x_1$ and $x_2$; $x_1$ and $x_2$ become the \emph{self}
parts of $e_1$, and $e_2$ respectively, but $x_2$ is also added to the
\emph{other} component $a_\env$ of $e_1$, and dually, $x_1$ is added
to the \emph{other} component of $e_2$.

In this paper, the PCM we chose is that of histories, which are a PCM
under the operation of disjoint union $\hunion$, with the $\hempty$
history as the unit. More common in separation logic is to use heaps,
which, similarly to histories, form PCM under disjoint (heap) union
and theempty heap, $\mathsf{empty}$. In FCSL, these can be combined
into a Cartesian product PCM, to enable reasoning about both space and
time in the same system.

%As discussed in Section~\ref{sc:clients}, this makes our specs of
%{\tt scan} and {\tt write} \emph{principal}. Many other logics that
%employ auxiliary state, going back to Owicki and
%Gries~\cite{Owicki-Gries:CACM76}, require that their procedures are
%annotated for a specific number and forking pattern of interfering
%threads. When this pattern changes, the programs have to be
%reannotated, and proofs redone. But this is not the case in FCSL.

The frame rule is a special case of the parallel composition rule,
obtained when $e_2$ is taken to be the idle thread.
\[
%\tag{\normalsize \arabic{tags}}\refstepcounter{tags}\label{eq:parrule}
\hfill \begin{array}{c}
e : \specK{\{P\}}\ A\ \specK{\{Q\}} @ C\\[2pt]
\hline\\[-7pt]
e : \specK{\{P \circledast R\}}\ A\ \specK{\{Q \circledast R\}} @ C 
\end{array}  \quad \mbox{$R$ is stable} \hfill
\]

For the purpose of this paper, the rule is important because it allows
us to generalize the specifications of {\tt write} and {\tt scan} from
Figure~\ref{fig:specs}. In that figure, both procedures start with the
precondition that $\histS = \hempty$. But what do we do if the
procedures are invoked by another one which has already completed a
number of writes, and thus its $\histS$ is non-empty. By
$\circledast$-ing with the frame predicate $R \eqdef (\histS = k)$,
the frame rule allows us to generalize these specs into ones where the
input history equals an arbitrary $k$:
%
\[
\begin{array}{l}
\mathtt{write}\ (p, v) : % : \mathtt{ptr}, n : \mathtt{int}) : %\hbox{}\\
%\quad
\begin{array}[t]{l}
\tsPre{\{\histS = k\}}\\
\tsPos{\{\exists t\ldot \histS' = h \hunion t \mapsto (p, v) \wedge
    \dom {\histO} \cup \scanned\, \stableorder
       \subseteq \sideal{\stableorderP}{t}\}} @ C % \wedge \hbox{}}\\
\end{array}\\[5pt]
%
\mathtt{scan} : 
\!\!\begin{array}[t]{l}
\tsPre{\{\histS = k\}}\\
\tsPos{\{r\ldot \exists t\ldot \histSP = k \wedge
   r =\! \eval\ t\, {\stableorderP}\, {\histP} \wedge
  \dom{\hist} \subseteq \ideal{\stableorderP}{t} \wedge
  t \in \scanned{\stableorderP}\}} @ C % \wedge \hbox{}}\\
\end{array}
\end{array}
\]
%
These two {\it large-footprint} instances of the rules for {\jyscan}
and {\jywrite} are those used in the proof of our clients in
Section~\ref{sc:clients}. For further details on FCSL, its semantics
and implementation, we refer the reader to~\cite{NanevskiLSD+ESOP14}.


%% \[
%% \begin{array}{l}
%% \mathtt{write}\ (p : \mathtt{ptr}, n : \mathtt{int}) : 
%% \!\!\!\begin{array}[t]{l}
%% \{\histS = k \wedge h \subseteq \histO
%%            \wedge \omega \subseteq \jleq{\ordlist}\}\\[3pt]
%% \{\exists t\ldot\, \histS = k \hunion t \mapsto (p, n) \wedge h \subseteq \histO \wedge
%%   \omega \subseteq \jleq{\ordlist}\,{\wedge}\, k \hunion h \subseteq H^{\hbox{}\sqsubsetneq_\ordlist t}\}
%% \end{array}\\
%% %
%% \mathtt{scan} : 
%% \!\!\!\begin{array}[t]{l}
%% \{\histS = k \wedge\, h \subseteq \histO \wedge\,
%%           \omega \subseteq \jleq{\ordlist}\}\\[3pt]
%%           \{\histS = k \wedge\, h \subseteq \histO \wedge\,
%%             \omega \subseteq \jleq{\ordlist} \wedge\, \exists\, t\ldot\, %
%%              k \hunion h \subseteq H^{\hbox{}\sqsubseteq_\ordlist t} \wedge\,
%%              \mathsf{chain}\ H^{\hbox{}\sqsubseteq_\ordlist t} \wedge\,
%%              r = \mathsf{eval}\ {H^{\hbox{}\sqsubseteq_\ordlist t}}\}
%%        \end{array}
%% \end{array}
%% \]


%% \[
%% \begin{array}{l}
%% \mathtt{write}\ (p : \mathtt{ptr}, n : \mathtt{int}) : \
%% \logvar{h,w}\
%% \!\!\begin{array}[t]{l}
%% \tsPre{\{\, \histS = k \wedge h \subseteq \histO
%%            \wedge w \subseteq \stableorder \}}\\[3pt]
%% \tsPos{\{\, \exists t\ldot\, \histS = k \hunion t \mapsto (p, n) \wedge
%%         h \subseteq \histO \hunion \histS \wedge}\\
%% \tsPos{\phantom{\{\, \exists t\ldot\,}
%%     w \subseteq \stableorder \wedge
%%     h \subseteq \sideal{\stableorder}{t}\}}
%% \end{array}\\[1em]
%% \mathtt{scan}\ () : \ \logvar{h,w}\ 
%% \!\!\begin{array}[t]{l}
%% \tsPre{\{\, \histS = k \wedge\, h \subseteq \hist \wedge
%%           w \subseteq \stableorder \}}\\[3pt]
%% \tsPos{\{\, r\ldot\, \exists t \ldot\, \histS = k \wedge\,
%%            h \subseteq \hist \wedge
%%            w \subseteq \stableorder \wedge
%%            h \subseteq \ideal{\stableorder}{t} \wedge}\\
%% \tsPos{\phantom{\{\, r\ldot\, \exists t \ldot\,}
%%            \static\ \prefix{t} \wedge
%%            \chain\ \prefix{t} \ideal{\stableorder}{t} \wedge
%%            r = \eval\ \ideal{\stableorder}{t} \chi\}}
%%        \end{array}
%% \end{array}
%% \]
