\section{Related work}
\label{sc:related}

% Early times

\subparagraph*{Program logics for linearizability.}

The proof method for establishing linearizability of concurrent
objects based on the notion of linearization points has been presented
in the original paper by Herlihy and
Wing~\cite{HerlihyW+TOPLAS90}. The first Hoare-style logic, employing
this method for compositional proofs of linearizability was introduced
in Vafeiadis' PhD thesis~\cite{VafeiadisHHS+PPoPP06, Vafeiadis+PhD07}.
However, that logic, while being inspired by the combination of
Rely-Guarantee reasoning and Concurrent Separation
logic~\cite{VafeiadisP+CONCUR07} with syntactic treatment of
linearization points~\cite{VafeiadisHHS+PPoPP06}, did not connect
reasoning about linearizability to the verification of client programs
that make use of linearizable objects in a concurrent environment.

% \todo{Check that Viktor's thesis does not verify clients in the same
%   way we do}

%\gad{Is this correct? Or are we still insulting Viktor?}
%
%% However, that logic, while being inspired by the combination of
%% Rely-Guarantee reasoning and Concurrent Separation
%% logic~\cite{VafeiadisP+CONCUR07} with syntactic treatment of
%% linearization points~\cite{VafeiadisHHS+PPoPP06}, did not have a
%% soundness proof with respect to any program semantics.
%% %
%% Furthermore, the work~\cite{Vafeiadis+PhD07} did not connect reasoning
%% about linearizability to the verification of client programs that make
%% use of linearizable objects in a concurrent environment (\cf
%% Section~\ref{sc:clients}).

% Modern logics for linearizability

Both these shortcomings were addressed in more recent works on program
logics for linearizability~\cite{LiangF+PLDI13,KhyzhaGP+FM16}, or,
equivalently, \emph{observational
  refinement}~\cite{FilipovicOHRW+TCS10,TuronDB+ICFP13}. These works
provided semantically sound methodologies for verifying refinement of
concurrent objects, by encoding atomic commands as resources
(sometimes encoded via a more general notion of
\emph{tokens}~\cite{KhyzhaGP+FM16}) directly into a Hoare
logic. Moreover, the logics~\cite{LiangF+PLDI13, TuronDB+ICFP13}
allowed one to give the objects standard Hoare-style specifications.
%
However, in the works~\cite{LiangF+PLDI13,TuronDB+ICFP13}, these two
properties (\ie,~linearizability of a data structure and validity of
its Hoare-style spec) are established separately, thus doubling the
proving effort.
%
That is, in those logics, provided a proof of linearizability for a
concurrent data structure, manifested by a spec that suitably handles
a \emph{command-as-resource}, one should then devise a declarative
specification that exhibits temporal and spatial aspects of executions
(akin to our history-based specs from Figure~\ref{fig:specs}),
required for verifying the client code.

Importantly, in those logics, determining the linearization order of a
procedure is tied with that procedure ``running'' the
command-as-resource within its execution span. This makes it difficult
to verify programs where the procedure terminates before the order is
decided on, such as \jywrite~operation in Jayanti's snapshot. The
problem may be overcome by extending the scope of \emph{prophecy
  variables}~\cite{AbadiL+lics88} or \emph{speculations} beyond the
body of the specified procedure. However, to the best of our
knowledge, this has not been done yet.

% \an{Jung told
%   German at ICFP that \cite{TuronTABD+POPL13} has an example of a
%   conditional CAS, which is non-regional. And indeed, they explicitly
%   say on page 2, first column, that they support reasoning about
%   temporally-non-local linearization points, which is their word for
%   non-regional. They say they do it via speculation!!! I was up to now
%   under the impression that speculations can handle future-dependent
%   linearization points, but only by the time the procedure
%   terminates. So, I'm puzzled; I guess different people mean different
%   things under ``speculation''. That said, this paper is the model
%   behind CaReSL, not the CaReSL logic itself. Nobody's using CaReSL
%   now, which says that something is off in it regarding the
%   interaction between logical refinement (\ie, linearization) and
%   Hoare stuff. This may explain why IRIS people gave talks at Dagstuhl
%   saying that speculations are a problem for them.  Also, did I
%   mischaracterize the prophecy variables business here? In the
%   previous version, these two logics were not tied to
%   prophecies. Should they have been?}  \an{At any rate, we may
%   consider weakening or removing the statement that these people can't
%   do non-regional. Instead, we may just say that they prove
%   refinement, but double the proving effort when it comes to combining
%   that with Hoare logic specs, whereas we just do it all at once,
%   which has its advantages, as we already discussed a bit in Section
%   3.}

%
% No-linearizability

\subparagraph*{Hoare-style specifications as an alternative to
  linearizability.} 

A series of recent Hoare logics focus on specifying concurrent
behavior \emph{without} resorting to
linearizability~\cite{SergeyNB+ESOP15, SergeyNBD+OOPSLA16, SvendsenB+ESOP14,
  PintoDYG+ECOOP14, JungSSSTBD+POPL15}.
%
This paper continues the same line of thinking, building
on~\cite{SergeyNB+ESOP15}, which explored patterns of assigning
Hoare-style specifications with self/other auxiliary histories to
concurrent objects, including \emph{higher-order} ones (e.g., {flat
  combiner}~\cite{HendlerHIST+SPAA10}), and \emph{non-linearizable}
ones~\cite{SergeyNBD+OOPSLA16} in FCSL~\cite{NanevskiLSD+ESOP14}, but
has not considered non-local, future-dependent linearization points,
as required by Jayanti's algorithm.
%

% Following the style of specifications, advocated by 
Alternative logics, such as
Iris~\cite{JungSSSTBD+POPL15,JungKBD+ICFP16} and
iCAP~\cite{SvendsenB+ESOP14}, employ the idea of ``ghost
callbacks''~\cite{Jacobs-Piessens+POPL11}, to identify precisely the
point in code when the callback should be invoked.  Such a program
point essentially corresponds to a local linearization
point. Similarly to the logical linearizability proofs, in the
presence of future-dependent LPs, this method would require
speculating about possible future execution of the callback, just as
commented above, but that requires changes to these logics'
metatheory, in order to support speculations, that have not been
carried out yet.

The specification style of TaDA logic~\cite{PintoDYG+ECOOP14} is
closer to ours in the sense that it employs \emph{atomic tracking
  resources}, that are reminiscent of our history entries. However,
the metatheory of TaDA does not support ownership transfer of the
atomic tracking resources, which is crucial for verifying algorithms
with non-local linearization points. As demonstrated by this paper and
also previous works~\cite{SergeyNB+ESOP15,SergeyNBD+OOPSLA16}, history
entries can be subject to ownership transfer, just like any other
resources.

%% \todo{Fix reference w.r.t TaDa. R.1: `` It might be better to refine
%%   this sentence to, e.g. TaDA does not support helping (ownership
%%   transfers of atomic tracking resources)".}

The key novelty of the current work with respect to previous results
on Hoare logics with histories~\cite{FuLFSZ+CONCUR10, LiangF+PLDI13,
  GotsmanRY+ESOP13, BellAW+SAS10, SergeyNB+ESOP15, HemedRV+DISC15} is
the idea of representing logical histories as auxiliary state, thus
enabling constructive reasoning, by \emph{relinking}, about
dynamically changing linearization points.
%
Since relinking is just a manipulation of otherwise standard auxiliary
state, we were able to use FCSL \emph{off the shelf}, with no
extensions to its metatheory. Furthermore, we expect to be able to use
FCSL's higher-order features to reason about higher-order (\ie,
parameterized by another data structure) snapshot-based
constructions~\cite{PetrankT+DISC13}.
%
Related to our result, O'Hearn \etal have shown how to employ
history-based reasoning and Hoare-style logic to
\emph{non-constructively} prove the existence of linearization points
for concurrent objects out of the data structure
invariants~\cite{OHearnRVYY+PODC10}; this result is known as \emph{the
  Hindsight Lemma}. The reasoning principle presented in this paper
generalizes that idea, since the Hindsight Lemma is only applicable to
``pure'' concurrent methods (\eg, a concurrent set's
\texttt{contains}~\cite{HellerHLMSS+OPODIS05}) that do not influence
the position of other threads' linearization points. In contrast, our
history relinking handles such cases, as showcased by Jayanti's
construction, where the linearization point of \texttt{write} depends
on the (future) outcome of \texttt{scan}.

%% Aspects oriented linearizability and other approaches to cope with
%% non-local linearization points.

\subparagraph*{Semantic proofs of linearizability.}
\label{sec:semant-proofs-line}

There has been a long line of research on establishing linearizability
using forward-backwards
simulations~\cite{SchellhornWD+CAV12,ColvinGLM06,ColvinDG05}. These
proofs usually require a complex simulation argument and are not
modular, because they require reasoning about the entire data
structure implementation, with all its methods, as a monolithic STS.
%state-transition system.

Recent works~\cite{HenzingerSV+CONCUR13, ChakrabortyHSV+LMCS15,
  DoddsHK+POPL15} describe methods for establishing linearizability of
sophisticated implementations (such as the Herlihy--Wing
queue~\cite{HerlihyW+TOPLAS90} or the time-stamped
stack~\cite{DoddsHK+POPL15}) in a modular way, via
\emph{aspect-oriented} proofs.
%
This methodology requires devising, for each class of objects (\eg,
queues or stacks), a set of specification-specific conditions, called
{\it aspects}, characterizing the observed executions, and then
showing that establishing such properties implies its linearizability.
%
This approach circumvents the challenge of reasoning about
future-dependent linearization points, at the expense of (a)
developing suitable aspects for each new data structure class and
proving the corresponding ``aspect theorem'', and (b) verifying the
aspects for a specific implementation. 
%
% \ab{Very long sentence. Can we break it up?}
%
Even though some of the aspects have been mechanized and proved
adequate~\cite{DoddsHK+POPL15}, 
%at the moment, aspect-oriented
%verification of implementations is still a manual effort.
%
currently, we are not aware of such aspects for snapshots.

%  establish the linearizability of Herlihy--Wing
% queue---another algorithm where the linearization order of an event
% may be determined only after the event terminated---but using
% automated verification methods. In contrast, we used a logic for
% modularly establishing full functional specifications, which, while
% not exactly linearizability, in the case of Jayanti's algorithm
% achieves the same goals. \an{Somebody please check that I didn't
%   mischaracterize these papers.}

%% Jayanti's ``Proof''

Our approach is based on program logics and the use of STSs to
describe the state-space of concurrent objects. Modular reasoning is
achieved by means of separately proving properties of specific STS
transitions, and then establishing specifications of programs,
composed out of well-defined atomic commands, following the
transitions, and respecting the STS invariants.

\subparagraph*{Proving linearizability using partial orders.}
Concurrently with us, Khyzha \etal~\cite{KhyzhaGP+ESOP17} have
developed a proof method for proving linearizability, which can handle
certain class of data structures with similar future dependent
behavior.
%
The method works by introducing a partial order of events for the data
structure as auxiliary state, which in turn defines the abstract
histories used for satisfying the sequential specification of the data
structure. Relations are added to this partial order at {\it
  commitment points} of the instrumented methods, which the verifier
has to identify.

The ultimate goal of this method is to assert the
linearizability of a concurrent data structure. As we have shown in
Section~\ref{sc:clients}, FCSL goes beyond as it provides a logical
framework to carry out formal proofs about the correctness of a
concurrent data structure and its clients.

%% However, these proof obligations have to be discharged by hand and,
%% unlike FCSL, it does not provide a verification framework in which
%% to reason about clients.

The proof technique also tracks the ordering of events 
differently from ours. Where we keep a single witness for
the current total ordering of events at all stages of execution,
their technique requires keeping many witnesses. Their main theorem requires a
proof that all linearizations of the abstract histories---\ie all
possible linear extensions of the partial order into a total
order---satisfy the sequential specification of the data structure.
%We believe this eventually leads to longer and more complicated
%proofs than necessary.

%% \gad{After talking to Artem, I decided to remove that part that was
%%   saying that: ``We believe this eventually leads to longer and more
%%   complicated proofs than necessary''. He said that he might
%%   understand what we mean by this, but that Mike and Matthew have {\bf
%%     strong opinions} about the {\it simplicity} of their
%%   method. Moreover, it is true that it would be a somewhat
%%   unsubstantiated claim: we have disjoint sets of examples, and they
%%   do things by hand. Since any formal evidence is hard to come up
%%   with, I rather play the political card here. We are already saying
%%   we believe we can do better than them---by their own admittance.}

Through personal communication we learned that the
technique cannot apply, for instance, to the verification of the {\it
  time-stamped} (TS) stack~\cite{DoddsHK+POPL15}. This is because a
partial order does not suffice to characterize the abstract histories
required to verify the data structure.
%
In contrast, given the flexibility of FCSL in designing and reasoning
with auxiliary state, we believe that our technique would not suffer
such shortcomings.

%% \gad{Artem gave the OK to the latter paragraph. He also suggested
%%  we can substantiate the claim by pointing out that Mike Dodds'
%%  original proof~\cite{DoddsHK+POPL15} does something akin to a
%%  posteriori relinking in the histories, or that at least it looks
%%  it could be implemented with a similar auxiliary code
%%  operation. I'm not sure about this, since it would sound that the
%%  relinking idea is not original.}

%% I'm moving this part to the discussion Section2.
\begin{comment}

\subparagraph*{Relation to Jayanti's original proof.}
\label{sec:relat-jayant-orig}

We note that our proof of Jayanti's algorithm seems very different
from Jayanti's original proof. Jayanti relies on so-called
\emph{forwarding principles}, as a key property of the proof. For
example, Jayanti's First Forwarding Principle says (in paraphrase)
that if {\tt scan} misses the value of a concurrent write through
lines~\lineScanReadsX--\lineScanReadsY\ of
Figure~\ref{fig:jayanti-snapshot}, but the write terminates before the
scanner goes through line~\lineScanUnsetsS\ (the linearization point
of \jyscan), then the scanner will catch the value in the forwarding
pointers through lines~\lineScanReadsFX--\lineScanReadsFY.
%
Instead of forwarding principles, we rely on colors to algorithmically
construct the status of each write event as it progresses through
time, and express our assertions using formal logic. For example,
though we did not use the First Forwarding Principle, we nevertheless
can express a similar property, whose proof follows from the auxiliary
state Invariants introduced in Section~\ref{sc:auxiliaries}:
%
\begin{myprop}\label{inv:fwd1}%
If $\sss = \sOff\ \toff$ and $\sx = \sy = \TT$---i.e., the scanner is
in lines~\lineScanReadsFX--\lineScanReadsFY\ and it has unset $\s$ in
line~\lineScanUnsetsS\ at time $\toff$---then:
%
\[\forall t \in
  \hist\ldot\ t \leq \E(t)< \toff \implies \C(t)= \mathsf{green}\]
\end{myprop}
%
%This proposition is still somewhat different from the forwarding
%principle, because it states that \emph{all} events, not just the
%missed ones, that terminated before the scanner goes through
%line~\lineScanUsetS, are colored green; hence, are ``observed'' by the
%scanner.
%
%Notice that our forwarding principle is a bit stronger than Jayanti's:
%it states that at the moment the scanner reads from the forwarding
%pointers on lines~\lineScanReadsFX--\lineScanReadsFY, all write events
%that finished before \jyscan's line~\lineScanUnsetsS---including those
%originally missed---are green and thus will not be missed by the
%current scan.
%
%\gad{Better names are welcome! Does this explanation helps? In fact,
%  the statement reads as if we are verifying that we comply with
%  Jayanti's FFP.}

\begin{wrapfigure}[10]{r}[0pt]{0.4\textwidth} 
%% \begin{figure}
%
\centering  
\begin{tabular}{l}
% 
\begin{minipage}[l]{5cm}
\small
\begin{alltt}
\num{1}  scan(): \(A {\times} A\)  \{
\num{2}    (cx, vx) \tbnd \act{read}(x);
\num{3}    (cy, _)  \tbnd \act{read}(y);
\num{4}    (_, tx)  \tbnd \act{read}(x);
\num{5}    if vx == tx 
\num{6}    then return (cx, cy)
\num{7}    else scan (); \}
\end{alltt} 
\end{minipage}
%
\end{tabular}
%
\caption{{\tt Scan} using version numbers.}
\label{fig:readpair}
%\end{figure}
\end{wrapfigure}

% \gad{\textbf seems to have no efect whatsoever on alttt}
% \gad{unified action-names not to leave spaces before arguments}


\subparagraph*{Alternative snapshot implementations and substitution principle.}

We close this section by revisiting the previous verification in FCSL
of another snapshot algorithm~\cite{SergeyNB+ESOP15}. We present only
\jyscan\ in Figure~\ref{fig:readpair}, as \jywrite\ is trivial. In
this example, the snapshot structure consists of pointers $x$ and $y$
storing tuples $(c_x, v_x)$ and $(c_y, v_y)$, respectively. $c_x$ and
$c_y$ are the payload of $x$ and $y$, whereas $v_x$ and $v_y$ are
version numbers, internal to the structure. Writes to $x$ and $y$
increment the version number, while {\tt scan} reads $x$, $y$ and $x$
again, in succession. Snapshot inconsistency is avoided by restarting
if the two version numbers of $x$ differ. In the notation of the
current paper, the spec we proved is:
\[
\esc{scan} : \tsPos{\{\histS = \hempty\}}\
\tsPos{\{\exists t\ldot \histS' = \hempty \wedge
  r = \eval\ t\ \histP \wedge \dom{\hist} \subseteq \ideal{\histP}{t}\}}
\]
Indeed the spec is very similar to the one from
Figure~\ref{fig:specs}, but exhibits that the algorithm does not
require dynamic modification to the event ordering. Thus, by defining
$\stableorder$ to be the natural numbers ordering on timestamps in the
global history $\hist$ (so that $\ideal{\stableorderP}{t} =
\ideal{\histP}{t}$), and taking $\scanned{\stableorder}$ to be the set
of all timestamps in $\hist$ (so that $t \in \scanned{\stableorder}$
is trivially true and can be added to the postcondition above), the
above spec directly weakens into that of Figure~\ref{fig:specs}. As
FCSL satisfies a substitution principle whereby client proofs are
developed out of the specs, and not the code of programs, we can
substitute different implementations of snapshot algorithms in
clients, without disturbing the client proof. This is similar in
spirit to the property that programs that linearize to the same
sequential code are interchangeable in clients.  
\end{comment}
