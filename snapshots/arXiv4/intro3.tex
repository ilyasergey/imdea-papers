\section{Introduction}
\label{sc:intro} 
   
Formal verification of concurrent objects commonly requires reasoning
about linearizability~\cite{HerlihyW+TOPLAS90}. This is a standard
correctness criterion whereby a concurrent execution of an object's
procedures is proved equivalent, via a simulation argument, to some
sequential execution. The clients of the object can be verified under
the sequentiality assumption, rather than by inlining the procedures
and considering their interleavings. Linearizability is often
established by describing the \emph{linearization points} (LP) of the
object, which are points in time where procedures take place,
\emph{logically}.  In other words, even if the procedure physically
executes across a time interval, exhibiting its linearization point
enables one to pretend, for reasoning purposes, that it occurred
instantaneously (\ie, atomically); hence, an interleaved execution of
a number of procedures can be reduced to a sequence of atomic events.

Reasoning about linearization points can be tricky. Many times, a
linearization point of a procedure is not \emph{local}, but may appear
in another procedure or thread. Equally bad, linearization points'
place in time may not be determined statically, but may vary based on
the past, and even future, \emph{run-time} information, thus
complicating the simulation arguments. A particularly troublesome case
is when run-time information influences the logical order of a
procedure that has already terminated.
%
This paper presents a novel approach to specification of concurrent
objects, in which the dynamic and non-local aspects inherent to
linearizability can be represented in a procedure-local and
thread-local manner. 

%\subparagraph{Reasoning about concurrent clients.}


The starting point of our idea is to realize what are the
shortcomings of linearizability as a canonical specification method
for concurrent objects.
%
Consider, for instance, the following two-threaded program manipulating
a correct implementation of stack by invoking its \texttt{push}
and \texttt{pop} methods, which are atomic, \ie, linearizable:
%
\begin{center}
\begin{tabular}{l || l}
\texttt{push(3);} & \texttt{push(4)}
\\
\texttt{t1 := pop(); } & \texttt{t2 := pop();}
\end{tabular} 
\end{center}
%
Assuming that the execution started in an empty stack, we would like
to derive that it returns an empty stack and \texttt{(t1, t2)} is
either \texttt{(3, 4)} or \texttt{(4, 3)}.
%
Linearizability of the stack guarantees that the overall trace of
\texttt{push}/\texttt{pop} calls is coherent with respect to a
sequential stack execution. However, it does not capture
\emph{client}-specific partial knowledge about the \emph{ordering} of
particular \texttt{push}/\texttt{pop} invocations in sub-threads,
which is what allows one to prove the desired result as a
composition of separately-derived partial specifications of the left and the right thread.

This thread-local information, necessary for compositional reasoning
about clients, can be captured in a form of \emph{auxiliary
  state}~\cite{OwickiG+CACM76} (a generalization of \emph{history
  variables}~\cite{AbadiL+lics88}), widely used in Hoare-style
specifications of concurrent
objects~\cite{SergeyNB+ESOP15,LeyWildN+POPL13,JungSSSTBD+POPL15,JungKBD+ICFP16}.
%
A testament of expressivity of Hoare-style logics for concurrency with
rich auxiliary state are the recent results in verification of
fine-grained data structures with helping~\cite{SergeyNB+ESOP15},
concurrent graph manipulations~\cite{SergeyNB+PLDI15},
barriers~\cite{JungKBD+ICFP16,DoddsJPSB+TOPLAS16}, and even
\emph{non-linearizable} concurrent objects~\cite{SergeyNBD+OOPSLA16}.

Although designed to capture information about events that happened
concurrently \emph{in the past} (hence the original name \emph{history
  variables}), auxiliary state is known to be of little use for
reasoning about data structures with \emph{speculative} executions, in
which the ordering of past events may depend on other events happening
in the \emph{future}. Handling such data structures requires
specialized metatheory~\cite{LiangF+PLDI13} that does not provide
convenient abstractions such as auxiliary state for client-side
proofs. This is one reason why the most expressive client-oriented
concurrency logics to date avoid reasoning about speculative data
structures altogether~\cite{JungKBD+ICFP16}.

\subsubsection*{Our contributions}

The surprising result we present in this paper is that by allowing
certain \emph{internal} (\ie,~not observable by clients) manipulations
with the auxiliary state, we can use an existing program logic for
concurrency, like, \eg,
FCSL~\cite{NanevskiLSD+ESOP14,SergeyNB+PLDI15}, to specify and verify
algorithms whose linearizability argument requires speculations, \ie,
depends on the \emph{dynamic reordering} of events based on run-time
information from the future.
%
% and especially, reordering of terminated events.
%
To showcase this idea, we provide a new specification (spec) and the
first formal proof of a very sophisticated snapshot algorithm due to
Jayanti~\cite{Jayanti+STOC05}, whose linearizability proof exhibits
precisely such kind of dependence.

While we specify Jayanti's algorithm by means of a separation-style
logic, the spec nevertheless achieves the same general goals as
linearizability, combined with the benefits of compositional
Hoare-style reasoning.
%
In particular, our Hoare triple specs expose the logical atomicity of
Jayanti's methods (Section~\ref{sc:formal}), while hiding their true
fine-grained and physically non-atomic nature.  The approach also
enables that the separation logic reasoning is naturally applied to
clients (Section~\ref{sc:clients}).
%
% \highlight{This is in contrast to linearizability, which allows
%   replacing, in a client, a sophisticated concurrent procedure with a
%   simpler sequential one, but does not offer any guidance in verifying
%   the client itself.}
%
Similarly to linearizability, our clients can reason out of
procedures' spec, not code. We can also ascribe the same spec to
different snapshot algorithms, without modifying client's code or
proof.

% \gad{Does the ``in contrast'' in the highlighted text above need some
%   weakening, now?}

In more detail, our approach works as follows. 
%
We use shared auxiliary state to record, as a list of timed events
(\eg, writes occurring at a given time), the logical order in which
the object's procedures are perceived to execute, each instantaneously
(Section~\ref{sc:auxiliaries}). Tracking this time-related information
through state enables us to specify its dynamic aspects. We can use
\emph{auxiliary code} to mutate the logical order \emph{in place},
thereby permuting the logical sequencing of the procedures, as may be
needed when some run-time event occurs
(Sections~\ref{sc:implementation} and~\ref{sc:proof}). This mutation
is similar to updating pointers to reorder a linked list, except that
it is executed over auxiliary state storing time-related data, rather
than over real state. This is why we refer to the idea as
\emph{linking-in-time}.

% Second, we specify our procedures \emph{in relation} to the behavior
% of the interfering threads. This facilitates verification of clients,
% and also enables simple spec of non-local, and even future-dependent,
% behavior as follows. Our Hoare triples scope over \emph{two}
% \emph{local} variables $\histS$ (aka.~\emph{self}-variable) and
% $\histO$ (aka.~\emph{other}-variable) that store the histories of the
% events attributed to the specified program, and to its interfering
% environment, respectively. The specs can relate the events in $\histS$
% and $\histO$ to each other, and to the shared auxiliary list of
% logical times described above. For example, an event with a timestamp
% $t$ appearing in $\histS$ of a procedure $A$, models that a call to
% $A$ was logically executed (\ie, ``linearized'') at time $t$. But this
% timestamp may also be seen as a pointer into the list of logical
% times. ``$A$'s linearization point appearing in procedure $B$'' will
% be manifested by the auxiliary code of $B$ rearranging this list, to
% permute the node pointed by $t$.
% %
% %(see Figure~\ref{fig:relink:intro}). 
% %
% However, the rearrangement does not change $A$'s ownership of the
% event occurring at $t$, as $t$ still appears in $\histS$ of $A$. The
% setup will enable us to specify $A$ \emph{locally}, in terms of
% auxiliary state that $A$ manipulates, rather than in terms of line
% numbers in the code of $B$, as done, for example, in Jayanti's
% original proof.

%\begin{figure}[t]
%\captionsetup[subfigure]{justification=centering}
\begin{subfigure}[t]{0.48\textwidth}
\includegraphics[height=4cm]{res/before-cloud.pdf}
\caption{\label{fig:relink:intro:before}}
\end{subfigure} \hfill
\begin{subfigure}[t]{0.48\textwidth}
\includegraphics[height=4cm]{res/before-cloud.pdf}
\caption{\label{fig:relink:intro:after}} % Logical $\neq$ Real Time order, snapshot OK}
\end{subfigure}%
%
\caption{\label{fig:relink:intro} Placeholder for a general
  picture. The list ordering the events $T_0-T_5$ is permutted from
  (a) to (b), while preserving the event ownership: $T_1$, $T_2$ and
  $T_4$ have been executed---and are thus owned---by the specified
  thread (aka.~self), while $T_0$, $T_3$ and $T_5$ are executed by the
  interfering threads (aka.~other).}
\end{figure}


% Treating time as space 
Encoding temporal information by way of representing it as mutable
state allows us to use FCSL off-the-shelf to verify example
programs. In particular, FCSL has been implemented in the proof
assistant Coq, and we have fully mechanized the proof of Jayanti's
algorithm~\cite{CoqFiles}.

% \gad{I don't like the sound of some clients. But, I don't want to say
%   we have mechanized all the clients-- because the one using {\sf
%     hide} is not implemented.}

%
%We have previously applied FCSL to non-trivial concurrent data
%structures~\cite{SergeyNB+ESOP15}, including
%graphs~\cite{SergeyNB+PLDI15} and non-linearizable
%structures~\cite{SergeyNBD+OOPSLA16}. However, it was surprising to
%discover that the same logic can be applied to an algorithm such as
%Jayanti's, whose correctness argument requires dynamic reordering of
%terminated events.

\begin{comment}
While several recent Hoare logics have targeted concurrent programs
with non-thread-local and future-dependent linearization
points~\cite{LiangF+PLDI13,TuronDB+ICFP13}, they only allowed to
establish a procedure's LP position based on the observations made
\emph{during} the procedure's execution, \ie, scoped within
its \emph{region}.
%
However, a number of modern concurrent data structures exhibit
executions whose linearization order can only be established
\emph{after} the involved overlapping procedure calls have
terminated~\cite{HerlihyW+TOPLAS90, Jayanti+STOC05,
  DoddsHK+POPL15}. We call the linearization points of such executions
\emph{non-regional}, and our method is the first that provides a logic
for reasoning about non-regional linearization points.

\gad{The last statement might not be \it{entirely} true, Turon \etal
  do verify {\sf
    conditional-CAS}~\cite{HarrisFP+DISC02,FraserH+TOPLAS07} in
  CARESL's POPL'13 paper~\cite{TuronTABD+POPL13}, which is {\it
    supposed} to be non-regional as well. I've got to check the paper
  again to see how they do it.}
%
%% This was other of Ralf's remarks after the talk}}

\gad{More non-regional references we might want to mention, Elimination
  Queues. Gotta fetch a reference!}

The rest of the paper is organized as
follows. Section~\ref{sc:overview} describes Jayanti's algorithm, and
Sections~\ref{sc:formal} and~\ref{sc:implementation} show how the
auxiliary state and code can be designed to provide local
specs for it. Section~\ref{sc:proof} discusses the important
aspects of the correctness proof, and Section~\ref{sc:clients} shows
how to reason about clients. Section~\ref{sc:related} discusses the
related work.

\end{comment}


%We make an
%interesting parallel with linearizability, as this auxiliary state has
%to keep track of both beginning and ending times of an operation, for
%the purposes of the proof, although one of the endpoints can be
%omitted in the specs, which we present in the abstract form in Section
%4, along with the commentary of the formal proof. 
%
%This is the first proof of Jayanti's algorithm in a formal program
%logic. 
%
%Moreover, the proof is mechanized in in Coq.
%
%\an{Maybe not in Coq.}
%
%% In Section 4, we illustrate that the same specs can be ascribed to at
%% least one more snapshot implementation.
%
%
%before concluding (Section~6).
%
%Following the old adage that a method is a trick which worked at least
%twice, this lends credence to the claim that our snapshot API is
%canonical. We illustrate how the specs works in a several simple
%client program scenarios.
%
%\gad{WRT the Coq mechanization: Note that PODC's submission page
%  expects a .pdf file. Then if we provide code, it will have to be
%  available on-line. That could save us some time to try to push it
%  after the deadline but, it would be a risky enterprise. I'd rather
%  go safe and claim the mechanization is a work in progress. There is
%  no rebuttal period to argue back that we have finished it either.}

%\gad{ We should remember to point out the fact that unlike
%  linearizability proofs, we do not track the timestamp of all events,
%  but rather a few selected events. Moreover, our specs only need to
%  expose the timestamps of atomic write events in the histories, as
%  the only scanner timestamp that we are interested in, the
%  linearization point of the last scanner, is stored only in the
%  internal auxiliary state.  We need to spin this fact in our favor
%  here, and stress it later in Section~\ref{sc:formal} in further
%  detail.}
