\section{On the Stability of Certain Assertions}
\label{sc:coq-code}

\newcommand\chainjleq[2]{\hist^{\hbox{}\sqsubseteq_{#2} {#1}}}
\newcommand\chainjltn[2]{\hist^{\hbox{}\sqsubsetneq_{#2} {#1}}}
\newcommand\chaintleq[2]{\hist^{\hbox{}\leq_{#2} {#1}}}
\def\cat{{\mathrel{+\!\!+}}}

%% Stablity proofs in general
In this appendix we discuss the stability of the assertions we used in
the specifications of {\tt write} and {\tt scan}. A proposition $P$
over state is stable if it is invariant under environment
interference. That is, if the state $s$ evolves to $s'$ by environment
steps, then $P\ s$ implies $P\ s'$. By the nature of our
setting~\cite{Nanevski-al:ESOP14}, such environment steps are
restricted to the auxiliary code used by the concurrent object; in
this case, the auxiliary procedures from Figure~\ref{fig:auxcode}. In
particular, this appendix develops the stability proofs for the
properties from Lemma~\ref{lem:menvs}. All the proofs of the lemmas
listed here have being mechanizaed in our FCSL developement~\cite{CoqFiles}.

%%% PUSH MONO
We begin with a property of the logical order $\tle$. This order is
not stable, as it can be changed, for example, by the helper procedure
$\aux{push}\ i\ j\ \ordlist$, used in $\aux{relink}$
(Figure~\ref{fig:auxcode}). However, we consider the special cases of
elements for which the ordering is preserved, as this will become
important for the subsequent lemmas.



\begin{lemma}[Push Mono]\label{lem:push-mono}
Given elements $a, b, i, j$, all in $\ordlist$:
\begin{enumerate}
\item\label{lem:push:left} If $a \tle i$ then $ a \tle b \implies a
  <_{\aux{push}\, i\, j\, \ordlist} b$. 
\item\label{lem:push:right} If $j \tle b$ then $ a \tle b \implies a
  <_{\aux{push}\, i\, j\, \ordlist} b $.
\item\label{lem:push:window} If $a \neq i$ then $ a \tle b \implies a
  <_{\aux{push}\, i\, j\, \ordlist} b $
\end{enumerate}
\end{lemma}

\begin{proof}
First, as $\ordlist$ is used to order timestamps, it contains only
distinct elements. Second, $\aux{push}$ is a surgery operation on
$\ordlist$ defined as follows.  Let $\ordlist = \ordlist_{<_i}\ \cat
\ i\ \cat\ \ordlist_{i..j}\ \cat\ j\ \cat\ \ordlist_{>_j}$, then
%
$$\aux{push}\ i\ j\ \ordlist = \ordlist_{<_i} \cat\  \ordlist_{i..j}\
  \cat\ j\ \cat\ i\ \cat \ordlist_{>_j}$$
%
%% \begin{equation}
%%   \aux{push}\ i\ j\ \ordlist =\
%%    \aux{take}\ \iota_{i}\ \cat\
%%     \aux{rot}\ 1\ (\aux{drop}\ \iota_i\ (\aux{take}\ (\iota_j + 1))\ \ordlist) \cat\
%%     \aux{drop}\ (\iota_j+1)\ \ordlist
%% \end{equation}
%
The changes in the order $\tle$ can therefore occur only in the
segment $ i\ \cat\ \ordlist_{i..j}\ \cat\ j$ of $\ordlist$.
%
Thus, (\ref{lem:push:left}) and~(\ref{lem:push:right}) are trivial, as
changing the position of $i$ does not affect $a \tle b$. For
(\ref{lem:push:window}), observe that the only relations invalidated
by $\aux{push}$ are those $ i \tle b$, where $b \in
\ordlist_{i..j}\ \cat\ j$. Since $a \neq i$, it follows that $ a
<_{\aux{push}\, i\, j\, \ordlist} b$.
\end{proof}

We now consider the statements of Lemma~\ref{lem:menvs}, proving each
one in order, as a separate lemma. First is the lemma on monotonicity
of the full history $\histx$ wrt.~environment steps.

%will build incrementally from the statements and proofs for
%items~\ref{lem:menvs:jleq} and~\ref{lem:menvs:drs}, which we have
%carried out completely in Coq, in order to finish with the more {\it
%  delicate} assertions about the stability of $\mathsf{chain}$ and
%$\mathsf{eval}$, resp. items~\ref{lem:menvs:chain}
%and~\ref{lem:menvs:eval}. The latter lemmas require further
%consideration and, moreover, have been axiomatized in Coq but we have
%not (as of the time of this submission) finished discharging
%them. First, we consider the behaviour of the full history $\histx$
%\wrt environmment steps:

%%% HISTORIES

\begin{lemma}[Monotonic Histories]\label{lem:hist-mono}
If $\ordlist$ steps to $\ordlist'$ by environment steps, then $\hist
\subseteq \hist'$ (i.e., histories grow monotonically).
\end{lemma}

\begin{proof}
The proof follows straightforwardly from the definitions of the
auxiliary code (Figure~\ref{fig:auxcode}). In particular, the
evolution of the internal state does not remove entries from
histories, but either $(i)$ adds new entries via $\aux{register}$, or
$(ii)$ shuffles entries from $\histJ$ to $\histS$ via
$\aux{finalize}$. In either case, the total history grows. We have
mechanized this proof in Coq~\cite{CoqFiles}.
\end{proof}

%%% Stable order


\begin{lemma}[$\mathrel{\sqsubseteq}$ is stable]\label{lem:jleq-stable}
If $\ordlist$ steps to $\ordlist'$ by environment steps, then
$\stableorder\,{\subseteq}\,\stableorder'$.
\end{lemma}

\begin{proof}
This proof has also been mechanized, so we present here only the
interesting case of $\aux{relink}$, which is the only auxiliary
procedure that can invalidate existing relations in $\tle$. We assume
that $a \stableorder b$, and prove that $a \stableorder' b$. From the definition of
$a \stableorder b$, the cases of this proof are $(i)$ $a = b$, $(ii)$ $E(a) <
b$, and $(iii)$ $a \tle b \wedge C(a)=\mathsf{green}$. The first two
follow immediately, by definition of $a \tleqP b$.

In the proof of $a \stableorder' b$ in the case of $(iii)$, the interesting
situation is when in the call to $\act{relink}$, the environment
obtains $\aux{inspect}\ t_x\ t_y\ \ordlist\ \C =
\mathsf{Yes}\ x\ t'$. By Lemma~\ref{lem:inspect}, $\C(t') =
\mathsf{yellow}$, and thus $a \neq t'$, since
$\C(a)=\mathsf{green}$. But then by
Lemma~\ref{lem:push-mono}.\ref{lem:push:window}, we immediately get $a
\tle_{\aux{push}\ t'\ t_y\ \ordlist} b$.  Since $\ordlist'$ is
precisely equal to $\aux{push}\ t'\ t_y\ \ordlist$, we obtain the
result.
\end{proof}

%In this case, the proof relies on two observations: $(i)$ that
%$\aux{relink}$ does not affect the order of green entries, and $(ii)$,
%that in $\aux{push}\ i\ j\ \ordlist$, given $ \ordlist =
%\ordlist_{<_i}\ \cat
%\ i\ \cat\ \ordlist_{i..j}\ \cat\ j\ \cat\ \ordlist_{>_j}$, all the
%write events in $\ordlist_{i..j}\ \cat\ j$ overlap with $i$.
%
%Consider an environment call to $\act{relink}$, where
%$\aux{inspect}\ t_x\ t_y\ \ordlist\ \C = \mathsf{Yes}\ x\ t'$. By
%Lemma~\ref{lem:inspect}, $\C(t') = \mathsf{yellow}$, and by
%Lemma~\ref{lem:push-mono}, all relations $a \tle b$ where $a$ is green
%will be preserved in $a \tle_{\aux{push}\ t'\ t_y \ordlist} b$. This
%proves $(i)$. For $(ii)$, observe that all $t$'s in the chain $t' \tle
%\ldots \tle t \tle \ldots \tleq t_y$ belong to $\histy$, by virtue of
%$t_y$ being the yellow timestamp of $t_y$ (and
%Proposition~\ref{inv:redzone}). Moreover, observe that $t'$ has
%necesarily been commited to memory before {\tt scan} set the bit $S$
%to $\FF$ and either the write event $t'$ is still active (and thus
%overlapping with all writes after in the chain) or finished after
%line~11, and hence $ t_{off} \leq E(t')$. All elements $t_z$
%non-overlapping with $t'$, and thus $t' \tle t_z$, will be painted
%{\sf red} and hence not considered by $\stableorder$.
%\end{proof}

%%% Downward restriction

This gives us the first statement of Lemma~\ref{lem:menvs}. The second
statement of Lemma~\ref{lem:menvs} also follows easily from the
previous two lemmas, by mere unfolding of the definition of
$\chainjleq{t}{\ordlist} =\ \{ r \hpts (p,v) \in \hist \mid r
\stableorder t \}$.

\begin{lemma}[$\chainjleq{t}{}$ is stable]\label{lem:ideal-stable}
If $\ordlist$ steps to $\ordlist'$ by environment interference then
forall $t$, $\chainjleq{t}{\ordlist} \subseteq
\chainjleq{t}{\ordlist'}$.
\end{lemma}

%% \begin{proof}
%% The proof of this lemma follows from Lemmas~\ref{lem:hist-mono}
%% and~\ref{lem:jleq-stable}.
%% \end{proof}

%% \begin{proof}
%% The proof of item~\ref{lem:push:left} has been completely mechanized
%% in {\sf Coq}. The Proof of items~\ref{lem:push:left}
%% and~\ref{lem:push:window} have been admitted as axioms. And we proceed
%% to prove them here:
%% \end{proof}

Now, to consider the stability of {\sf chain} and {\sf eval}, which
are the third and fourth statements of Lemma~\ref{lem:menvs}, we first
introduce some intermediate concepts and results.  

%The first observation relies on proving that, as a result of a call to
%{\tt scan}, the system defines a prefix $ \tle t $ in the logical
%order $\tle$ --- where $t$ comes from the postcondition of {\tt scan}
%in Figure~\ref{fig:specs}---and that is all {\sf green}.  Such a
%prefix makes $\chainjleq{t}{\ordlist}$ {\bf complete}: no new items
%can be inserted into $\chainjleq{t}{\ordlist}$ by the evolution of the
%auxiliary state. In the following, we formalize these concepts and
%subsequently show how this builds up to the stability of {\sf chain}
%and {\sf eval}.


% First definition, history prefix.
\begin{definition}[$\chaintleq{t}{\ordlist}$]\label{def:chaintleq}%
The downward restriction on $\tle$ is defined as $
\chaintleq{t}{\ordlist} = \{ r \hpts (p,v) \in \hist | r \tleq t \}$,
analogously to $\chainjleq{t}{\ordlist}$.
\end{definition}

\begin{lemma}\label{lem:chaintleq-dom}
  $ \dom{ \_ \tleq t} = \dom{\chaintleq{t}{\ordlist}}$.
\end{lemma}

\begin{proof}
The statement of this lemma is a direct consequence of one of the
invariants we omitted in Section~\ref{sc:formal}, which states that $
\dom{\ordlist} = \dom{\hist}$.
\end{proof}

We now prove that, if $\chaintleq{t}{\ordlist}$ is a {\sf all green},
which we denote as $\C(\chaintleq{t}{\ordlist}) = {\sf green}$, then
$\chaintleq{t}{\ordlist}$ is invariant under environment interference.

% Green prefix STABLE

\begin{lemma}[{\sf all green} stable]%
\label{lem:green-stable}%
If $\ordlist$ steps to $\ordlist'$ by environment interference, then
$\C(\chaintleq{t}{\ordlist}) = {\sf green} \implies
\chaintleq{t}{\ordlist} = \chaintleq{t}{\ordlist'}$.
\end{lemma}
%
\begin{proof}
The proof is by induction on environment steps. The inductive case is
trivial, and of the base cases, the only interesting one is stepping
by $\aux{register}$. Other auxiliary procedures do not extend
$\ordlist$, and do not affect the position in $\ordlist$ of green
timestamps.

By definition, $\aux{register}$ adds a fresh timestamp $t' =
\mathsf{fresh}\ \ordlist$ to the tail of $\ordlist$, to obtain
$\ordlist'$. Hence $t \tleP t'$, and $t'$ does not belong to
$\dom{\chaintleq{t}{\ordlist'}}$. By Lemma~\ref{lem:hist-mono}, we
know that $\dom{\chaintleq{t}{\ordlist}} \subseteq
\dom{\chaintleq{t}{\ordlist'}}$. The difference between the two sets
can be at most $t'$, but since $t'$ does not belong to
$\dom{\chaintleq{t}{\ordlist'}}$, we obtain the desired equality.
\end{proof}

%order $\tleP$ are not in the domain of
%$\dom{\chaintleq{t}{\ordlist'}}$, and hence $
%\dom{\chaintleq{t}{\ordlist}} = \dom{\chaintleq{t}{\ordlist'}} $ and
%thus we conclude.  Now, we infer from Lemma~\ref{lem:hist-mono} than $
%\dom{\chaintleq{t}{\ordlist}} \subseteq
%\dom{\chaintleq{t}{\ordlist'}}$.
%\end{proof}

%% From the monotonicity of histories,
%% Lemma~\ref{lem:hist-mono}, we infer than $
%% \dom{\chaintleq{t}{\ordlist}} \subseteq
%% \dom{\chaintleq{t}{\ordlist'}}$. To deduce the equality, it sufices to
%% show that $ \forall t' \in (\ordlist' \backslash \ordlist)\ldot\ t
%% \tleP t' $. The latter entails that all new entries in the underlying
%% order $\tleP$ are not in the domain of
%% $\dom{\chaintleq{t}{\ordlist'}}$, and hence $
%% \dom{\chaintleq{t}{\ordlist}} = \dom{\chaintleq{t}{\ordlist'}} $ and
%% thus we conclude.

%% Now, it suffices to only consider $\aux{register}$, which is the only
%% auxiliary operation that adds entries to $\ordlist$. By definition,
%% this will added a new timestamp $t' = {\sf fresh}\ \ordlist$ in the
%% tail of the $\ordlist$ and hence, $ t \tleP t'$. 

%% COMPLETE
\begin{definition}[Complete]\label{def:complete}
Downward restriction $\chainjleq{t}{\ordlist}$ is {\sf complete} \Iff
$\chainjleq{t}{\ordlist} = \chaintleq{t}{\ordlist}$.
\end{definition}

%% 
\begin{lemma}[Complete Greens]\label{lem:green-complete}
  If $\C(\chaintleq{t}{\ordlist}) = {\sf green}$ then
  $\chainjleq{t}{\ordlist}$ is {\sf complete},
  \ie $\chainjleq{t}{\ordlist} = \chaintleq{t}{\ordlist}$.
\end{lemma}

\begin{proof}
From the definition of $\chaintleq{t}{\ordlist}$, and of the stable
order $\stableorder$ (Definition~\ref{def-jleq}), we infer that
$\dom{\chainjleq{t}{\ordlist}} \subseteq
\dom{\chaintleq{t}{\ordlist}}$. Then, from
$\C(\chaintleq{t}{\ordlist}) = {\sf green}$ we know that for all $a
\in \chaintleq{t}{\ordlist}$, $\C(a) = {\sf green} \wedge a \tleq t$,
and therefore $ a \stableorder t$. It follows that
$\dom{\chaintleq{t}{\ordlist}} \subseteq
\dom{\chainjleq{t}{\ordlist}}$, and thus we conclude, since
$\dom{\chaintleq{t}{\ordlist}} = \dom{\chainjleq{t}{\ordlist}}$.
\end{proof}

%%% CHAIN STABLE

\begin{lemma}[$\mathsf{chain}\ H^{\hbox{}\sqsubseteq_\ordlist t}$ is
    stable]%
\label{lem:chain-stable}%
If $\ordlist$ steps to $\ordlist'$ by environment interference then
if $\mathsf{chain}\ \chainjleq{t}{\ordlist}$ and
$\C(\chaintleq{t}{\ordlist}) = {\sf green}$ then
$\mathsf{chain}\ \chainjleq{t}{\ordlist'}$.
\end{lemma}

\begin{proof}
The proof follows up from Lemmas~\ref{lem:green-complete}
and~\ref{lem:green-stable} above: by Lemma~\ref{lem:green-complete} we
infer that $\chainjleq{t}{\ordlist} = \chaintleq{t}{\ordlist}$. Then
by Lemma~\ref{lem:green-stable}, $\chaintleq{t}{\ordlist'} =
\chaintleq{t}{\ordlist}$ and, moreover, $\C(\chaintleq{t}{\ordlist'})
= {\sf green}$. From the latter fact we apply
Lemma~\ref{lem:green-complete} again to get $\chainjleq{t}{\ordlist'}
= \chaintleq{t}{\ordlist'}$. Then, by transitivity
$\chainjleq{t}{\ordlist} = \chainjleq{t}{\ordlist'}$ and we conclude.
\end{proof}

%%% EVAL STABLE

\begin{lemma}[$\mathsf{eval}\ {H^{\hbox{}\sqsubseteq_\ordlist t}}$ is
    stable]%
\label{lem:eval-stable}%
If $\ordlist$ steps to $\ordlist'$ by environment interference then
if $\mathsf{eval}\ \chainjleq{t}{\ordlist}$ and
$\C(\chaintleq{t}{\ordlist}) = {\sf green}$ then
$\mathsf{eval}\ \chainjleq{t}{\ordlist'}$.
\end{lemma}

\begin{proof}
The proof is analogous to that of Lemma~\ref{lem:chain-stable} above.
\end{proof}
%% \gad{Here, we come clean about which lemmas are admitted, and try to
%%   give an informal proof sketch/intuition on why they should hold. The
%%   idea is to keep updating this section on the on-line draft until we
%%   finish the proofs.}

%% \gad{Come clean about these lemmas being whats missing in the mechanization}

\begin{comment}
%%%% Final Theorem
To sum up, we revisit Theorem~\ref{thm:specs} from
Section~\ref{sc:proof}. As for the {\tt write } method, we have
mechanized its proof of correctness in FCSL:
%
\begin{theorem}[{\tt write}]\label{thm:write}
The implementation for the {\tt write} method in
Figure~\ref{fig:fcsl-snapshot} can be ascribed the specification given
in Figure~\ref{fig:specs}.
\end{theorem}
%
As for scan, we haven't finished all the details of the mechanization,
but we present instead \gad{blah...}

\begin{lemma}[{\sf all green} {\tt scan}]\label{lem:green-scan}
After the execution of {\tt scan}, given $t_x$ and $t_y$, the
timestamps of the resulting snapshot $(r_X,r_y)$. If $t =
\mathsf{max}_{\tle} (t_X,t_y)$ then $ \C(\chaintleq{t}{\ordlist}$.
\end{lemma}
\begin{proof}
The proof unfolds as follows,
\end{proof}

\begin{theorem}[{\tt scan}]\label{thm:scan}
The implementation for the {\tt write} method in
Figure~\ref{fig:fcsl-snapshot} can be ascribed the specification given
in Figure~\ref{fig:specs}.
\end{theorem}

\end{comment}


%%% Commented out proofs.
\begin{comment}

We need to tweak the definition of {\sf chain}, else we will be able
to prove it stable, but only allowed to use the stability inside an
ongoing scan, and not after the scanner finishes in a client. We need
to capture the idea that the chain $H^{\hbox{}\sqsubseteq_\ordlist t}$
is not only a {\sf chain} as we have defined it in
Section~\ref{sc:formal}, but also is {\sf maximal} or {\sf complete}:

This means that $H^{\hbox{}\sqsubseteq_\ordlist t}$ defines a {\sf
  prefix} of $\ordlist$, \ie a chain that is completely determined
from a time onward.  In our development, we are trying to prove the
stability of {\sf chain}, which is weaker, but the hypothesis in the
lemma are stronger: we are requiring that

 $$ \forall\ t' \in \hist^{\hbox{}\sqsubseteq_\ordlist t}\ldot\quad
  \C(t')= {\sf green} $$

This will hold because at the atomic moment relink finishes, it
defines a {\sf green} prefix from the first timestamp in the history
$t_0$ up to $t$, $ t_o \tle \ldots \tle t$, which not only is totally
ordered but also {\emph maximal} any new write event $t'$, either mine
or the environments' will be positioned after $t$ in the chain, $ t_0
\tle \ldots \tle t \tle t'$.

However, if we don't include this information, either a a part of the
definition of ``chain'' or as an extra predicate, we will loose this
information, to clients, and {\sf eval} becomes uninformative.


\begin{lemma}[$\mathrel{\sqsubseteq{\_}}$ is stable]\label{lem:jleq-stable}
If $\ordlist$ steps to $\ordlist'$ by environment interference, then
$\stableorder \subseteq\ \stableorder'$.
\end{lemma}

\begin{proof}
The proof goes by structural induction on environment steps. Amongst
them, we highlight only relink relink, as the others are carried out
without much effort. In this case, the proof that $\aux{relink}$
preserves the chains in $\stableorder$ relies, ultimately on two
observations: (i) that the atomic opeation does not affect the order
of green entries in the history, which we mentioned above and (ii),
that the elements in the ``window'' determined by the elements being
pushed correspond to overlapping events. We reason with this case as
follow: Consider a $\act{relink}$ environment call that pushes $t_y$
past $t_x$, with $t_y \in \histy$ and $t_x \in \histx$, then all
elements $t$ in the chain $t_y \tle t \tleq t_x$ belong to $\histx$,
by virtue of $t_y$ being the yellow timestamp of $t_y$ (and
Invariant~\ref{inv:redzone}). Then, we observe that en entry that is
yellow when relink occurs, committed to memory before {\tt scan} set
the bit to false and either is still active or finished after {\tt
  scan} set the $S$ bit to false. If any $t$ in that chain began after
$t_y$ finished it cannot not possibly be green, as the spec for
register would have marked it red.
\end{proof}


\end{comment}
