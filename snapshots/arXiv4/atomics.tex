\subsection{Specification of Atomic Commands}
\label{sc:atoms-pecs}

We present here the stable specifications of the atomic commands
featuring in Figure~\ref{fig:fcsl-snapshot}, we don't give a full
detailed explanation of their proofs, which have being carried out in
Coq~\cite{CoqFiles}, as their a mandatory part of a FCSL
implementation. We will, however, give a general intuition on why
these assertions hold after the atomic actions and also why there are
stable under environment intereference. We start with \jywrite~atomic
commands:

\gad{TO DO: Update with new specs, complete. Change to proper VDM
  stylle}.

% Writer atomic commands
% Register

\[
\begin{array}{l}
\lat\,\actwrite{p}{v}; \aux{register}(v)\rat: \\[5pt]
\quad \quad \logvar{h,w}
\begin{array}[t]{l}
{\tsPre{\{\, \histS = \hempty \wedge h \subseteq \histO \wedge
             w \subseteq \stableorder \wedge \wpp = \wInit\}}} \\[5pt]
\tsPos{\{\, \exists t\ldot\, \histS = \hempty \wedge
            h \subseteq \histO \wedge
            w \subseteq \stableorder \wedge
  \wpp = \wWrite\, t\, v \wedge h \subseteq \sideal{\stableorder}{t}\}}
\end{array}
\end{array}
\]

%% \[
%% \begin{array}{r l}
%% \lat\,\actwrite{p}{v}; \aux{register}(v)\rat :&
%% \logvar{h,w}\
%% \!\!\begin{array}[t]{l}
%% {\tsPre{\{\, \histS = \hempty \wedge h \subseteq \histO \wedge
%%              w \subseteq \stableorder \wedge \wpp = \wInit\}}} \\[5pt]
%% \tsPos{\{\, \exists t\ldot\, \histS = \hempty \wedge
%%             h \subseteq \histO \wedge
%%             j \hpts (p,v) \in \histJ \wedge
%%             w \subseteq \stableorder \wedge}\\
%% \tsPos{\phantom{\{\, \exists t\ldot\,}
%%   \wpp = \wWrite\, t\, v \wedge h \subseteq \sideal{\stableorder}{t}\}}
%% \end{array}
%% \end{array}
%% \]

On line~\lineWrtWrt~in Figure~\ref{fig:fcsl-snapshot},
$\lat\,\actwrite{p}{v}; \aux{register}(v)\rat$ atomically commits the
write event and executes the $\aux{register}$ auxiliary method
described in Section~\ref{sc:implementation}, and summarized in
Figure~\ref{fig:writeauxcode}. The precondition states that
$\aux{register}$ is safe to run in the state of a recently initiated
writer thread. The postcondition states that the new write event has
being processed, and thus the writer state is now $\wWrite\, t\, v$
where, from the defintion of $\aux{register}$ in
Figure~\ref{fig:writeauxcode}, we know $t$ is a fresh timestamp in the
history $\hist$. By Proposition~\ref{inv:joint-hist}, we also know
that $t$ is in $\histJ$.

%%  \gad{We should have said by now that $ t \hpts
%%  (p,v) \in \histJ \iff \wpp = \wWrite\ p\, v \vee \wpp = \wDirty\
%%  p\, v \vee \wClean\ p\, v$ somehwere, else we need to make this as
%%  an invariant or as the definition of $\histJ$. Technically
%%  speaking, $\histJ$ is a function of $\wx$ and $\wy$.}

The stability of the writer state assertions $\wpp = \wInit$, $\wpp
= \wWrite\, t\, v$, and so on, as well as any scanner state assertions
for \jyscan's atomics, are guaranteed by the fact we enforce there is
at most one thread executing \jyscan, and at most one thread executing
a call to \jywrite~on each of the pointers.
%
%% The stability of $j \hpts (p,v) \in \histJ $ follows immediately
%% from this fact.

The assertions $\histS = \hempty$ and $h \subseteq \histO$ hold
trivially, all is left to show is that the $ w \subseteq \stableorder$
and that $h \subseteq \sideal{\stableorder}{t}$. The former holds by
the fact that we are snoc-ing a fresh timestamp on the tail of
$\ordlist$, and thus, it will be on the right of every existing write
event $t_0 \in \hist$ that has either finished---$\E(t_o) < t$---, and
on the right of any event $t_1 \in \hist$ such that $\C(t_1)
= \mathsf{green}$. As for the latter, we notice that $h$ binds
finished events---$\dom{\E} = \dom{\histS \hunion \histO}$---, and as
we stablished before they all are going to be sorted in $\stableorder$
wirth regard to $t$, and by its definition~\eqref{eq:subset-sideal},
included in the strict ideal $\sideal{\stableorder}{t}$.

Finally, it should be noticed that the post-condition does not make
any mention to the write event being commited to memory, other than
having recorded the write event in the writer state $\wpp$, which is
the last of $\histp$ in $\ordlist$. Then, by
Proposition~\ref{inv:last-key}, we know the value must have being
commited to memory--- else the invariant could not have being
satisfied by the $\aux{register}$ auxiliary code.

%Checkbit

\[
\begin{array}{l}
\lat\, b \tbnd \act{read}(S);\ \aux{checkS}(p,b) \rat :\\[5pt]
\quad\quad \logvar{h,w,t,v}
\begin{array}[t]{l}
\tsPre{\{\, \histS = \hempty \wedge h \subseteq \histO \wedge
             w \subseteq \stableorder \wedge
             \wpp = \wWrite\, t\, v \wedge
         h \subseteq \sideal{\stableorder}{t} \}}\ \esc{bool} \\[5pt]
\tsPos{\{\, r\ldot\, \exists t\ldot\, \histS = \hempty \wedge
            h \subseteq \histO \wedge  w \subseteq \stableorder \wedge}\\
\tsPos{\phantom{\{\, r\ldot\, \exists t\ldot\,}
   h \subseteq \sideal{\stableorder}{t} \wedge 
    \wpp =  \textrm{if}\ r\
  \textrm{then}\ \wDirty\ t\, \,v\ \textrm{else}\ \wClean\ t\, v\}}
\end{array}
\end{array}
\]

 %% \gad{Do I really need to put $j \hpts (p,v) \in \histJ$ here? It's
 %% not in the files---as it is infered from $\wpp = \_ $ (see above). I
 %% just put it in in order to reflect the narrative that the write event
 %% is placed in the joint history and, to be consistent with
 %% Figure~\ref{fig:auxcode}. I think that if we remove it I can make the
 %% pre here fit one line. Also if we remove the assertion , and place
 %% the definition/invariant of $\histJ$ somewhere, either at the
 %% beginning of this section, or in Section~\ref{sc:formal}}

%% \[
%% \begin{array}{r l}
%% \lat\, b \tbnd \act{read}(S);\ \aux{checkS}(p,b) \rat :&
%% \logvar{h,w,t,v}\
%% \!\!\begin{array}[t]{l}
%% \tsPre{\{\, \histS = \hempty \wedge h \subseteq \histO \wedge
%%              w \subseteq \stableorder \wedge}\\
%% \tsPre{\phantom{\{\,}            
%%              \wpp = \wWrite\, t\, v \wedge }\\
%% \tsPre{\phantom{\{\,}          
%%          j \hpts (p,v) \in \histJ \wedge
%%          \wedge h \subseteq \sideal{\stableorder}{t} \}}\ \esc{bool} \\[5pt]
%% \tsPos{\{\, r\ldot\, \exists t\ldot\, \histS = \hempty \wedge
%%             h \subseteq \histO \wedge  w \subseteq \stableorder \wedge }\\
%% \tsPos{\phantom{\{\, r\ldot\, \exists t\ldot\,}
%%   j \hpts (p,v) \in \histJ \wedge h \subseteq \sideal{\stableorder}{t} \wedge}\\
%% \tsPos{\phantom{\{\, r\ldot\, \exists t\ldot\,} 
%%     \wpp =  \textrm{if}\ r\
%%   \textrm{then}\ \wDirty\ t\, \,v}\\
%% \tsPos{\phantom{\{\, r\ldot\, \exists t\ldot\,
%%     \wpp =  \textrm{if}\ r\ }
%%   \textrm{else}\ \wClean\ t\, v\}}
%% \end{array}
%% \end{array}
%% \]

In line~\lineWrtChk~(Figure~\ref{fig:fcsl-snapshot}), $ \lat\,
b \tbnd \act{read}(S);\ \aux{checkS}(p,b) \rat$ reads the value of the
scanner bit $\s$ and depending on it being true or false, \aux{checkS}
changes the writer state accordingly
(Figure\ref{fig:writeauxcode}). The value of the scanner bit $\s$ is
returned as a result of the computation. The precondition
of \esc{checkS} states that we have an active ongoing write event,
which has already been registered--- $ \wpp = \wWrite\, t\, v $ ---
and that, moreover, its timestamp $t$ is such that every write event
prior to the call to the \jywrite~method is included in the ideal
$\sideal{\stableorder}{t}$. Ths postcondition here just reflects that
depending on the returned value, the writer states changes to indicate
the value has to be been forwarded, or alternativeley, the writer will
finish without writing to its forwarding pointer. Since $\aux{checkS}$
does not affect $\E$, $\C$ or $\ordlist$, $\stableorder$ can only
change by environment interference before or after executing
$\aux{checkS}$ (see~\cite{NanevskiLSD+ESOP14}) and then, by
Lemma~\ref{lem:menvs} and transitivity, we assert the validity of the
rest of the assertions which are carried down from the precondition.

% Forward

\[
\begin{array}{l}
\lat\, \actwrite{\fwdp{p}}{v};\ \aux{forward}(i,v)\rat\ :\\[5pt]
\quad\quad\logvar{h,w,t,v}
\begin{array}[t]{l}
\tsPre{\{\, \histS = \hempty \wedge h \subseteq \histO \wedge
             w \subseteq \stableorder \wedge
             \wpp = \wDirty\, t\, v \wedge
             h \subseteq \sideal{\stableorder}{t}\}}\ \\[5pt]
\tsPos{\{\, \exists t\ldot\, \histS = \hempty \wedge
            h \subseteq \histO \wedge  w \subseteq \stableorder \wedge
  h \subseteq \sideal{\stableorder}{t} \wedge
    \wpp = \wClean\ t\, v\}}
\end{array}
\end{array}
\]

%% \[
%% \begin{array}{r l}
%% \lat\, \actwrite{\fwdp{p}}{v};\ \aux{forward}(i,v)\rat\ :&
%% \logvar{h,w,t,v}\
%% \!\!\begin{array}[t]{l}
%% \tsPre{\{\, \histS = \hempty \wedge h \subseteq \histO \wedge
%%              w \subseteq \stableorder \wedge}\\
%% \tsPre{\phantom{\{\,}          
%%          \wpp = \wDirty\, t\, v \wedge j \hpts (p,v) \in \histJ \wedge}\\
%% \tsPre{\phantom{\{\,} h \subseteq \sideal{\stableorder}{t}\}}\ \\[5pt]
%% \tsPos{\{\, \exists t\ldot\, \histS = \hempty \wedge
%%             h \subseteq \histO \wedge  w \subseteq \stableorder \wedge }\\
%% \tsPos{\phantom{\{\, \exists t\ldot\,}
%%   j \hpts (p,v) \in \histJ \wedge
%%   h \subseteq \sideal{\stableorder}{t} \wedge}\\
%% \tsPos{\phantom{\{\,\exists t\ldot\,} 
%%     \wpp = \wClean\ t\, v\}}
%% \end{array}
%% \end{array}
%% \]

In line~\lineWrtFwd~(Figure~\ref{fig:fcsl-snapshot}), provided the
previous command returned {\tt true}--- \ie $b = true$ ---
$ \lat\, \actwrite{\fwdp{p}}{v};\ \aux{forward}(i,v)\rat $ forwards
the new value to the forwarding pointer $\fwdp{p}$ and executes the
auxiliary method $\aux{forward}$ which, if the scanner bit $\s$ is
still on, will paint the entry $\mathsf{greeen}$ and mark the write
event as {\it done} and thus ready to be transfered to the self
history by $\aux{finalize}$. Again, here the specs do not need to
mention the effect on the heap memory, as, this is captured by the
invariants of the auxiliary state, in particular, the ``Green/Yellow
Forwarded Values'' Invariant from Proposition~\ref{inv:readFP}.

In the case that $\aux{forward}$ paints $t$ green in $\C$,
$\stableorder$ grows adding relations $ t \stableorder t'$ for those
$t'$ such that $ t \tleq t'$. Since this were not included in $w$ and,
by definition, $ t' \notin \sideal{\stableorder}{t}$, the validity of
assertions $ w \subseteq \stableorder $ and $
h \subseteq \sideal{\stableorder}{t}$ follows immeadiately.

% Finalize

\[
\begin{array}{l}
\lat\,\aux{finalize}(i,v)\rat :\\
~\logvar{h,w,t,v}\
\!\!\!\!\begin{array}[t]{l}
\tsPre{\{\, \histS = \hempty \wedge h \subseteq \histO \wedge
             w \subseteq \stableorder \wedge \wpp = \wClean\, t\, v \wedge
             h \subseteq \sideal{\stableorder}{t}\}}\ \\[5pt]
 \tsPos{\{\, \exists t\ldot\, \histS = j \hpts (p,v) \wedge
            h \subseteq \histO \hunion \histS \wedge
            w \subseteq \stableorder \wedge
            h \subseteq \sideal{\stableorder}{t} \wedge \wpp = \wInit\}}
\end{array}
\end{array}
\]

 \gad{I'll look into making this one fit the linewidth later. It'll be
 a shame to have to break the line for 2 characters worth of space.}

%% \[
%% \begin{array}{r l}
%% \lat\,\aux{finalize}(i,v)\rat :&
%% \logvar{h,j,w}\
%% \!\!\begin{array}[t]{l}
%% \tsPre{\{\, \histS = \hempty \wedge h \subseteq \histO \wedge
%%              w \subseteq \stableorder \wedge \wpp = \wClean\, t\, v \wedge }\\
%% \tsPre{\phantom{\{\,}
%%          j \hpts (p,v) \in \histJ \wedge
%%          h \subseteq \sideal{\stableorder}{t}\}}\ \\[5pt]
%%  \tsPos{\{\, \exists t\ldot\, \histS = j \hpts (p,v) \in \histJ \wedge
%%             h \subseteq \histO \hunion \histS \wedge  w
%%             xs\subseteq \stableorder \wedge }\\
%% \tsPos{\phantom{\{\, \exists t\ldot\,}
%%   h \subseteq \sideal{\stableorder}{t} \wedge \wpp = \wInit\}}

%% \end{array}
%% \end{array}
%% \]

In line~\lineWrtFnz~(Figure~\ref{fig:fcsl-snapshot}),
$ \lat\,\aux{finalize}(i,v)\rat $ finalizes the write event in
process, tranfering $t \hpts (p,v)$ from $\histJ$ to $\histS$ and
recording the ending time of the event in $\E$, as per the definition
of $\aux{finalize}$ in Fifgure~\ref{fig:writeauxcode}. By doing so,
new relations are added to the stableorder $\stableorder$. The key is
to prove that by doing so, we don't contradict the hypothesis $
w \subseteq \stableorder $. Again, we succed by the fact that we add
at this step relations $t \stableorder t'$ such that $t \tle t'$ and,
$\aux{finalize}$ does not make any changes to $\ordlist$. Again, any
$t'$ such that $ t \stableorder t'$ is not in the ideal
$\sideal{\stableorder}{t}$ by definition, and thus $
h \subseteq \sideal{\stableorder}{t} $.

% Scanner!
We focus now on the atomic commands of the \jyscan~method in
Figure~\ref{fig:fcsl-snapshot}. But, before that, we introduce a
couple of auxiliary assertions and establish their stability. Again,
the proof of the stability statements has being carried out in our Coq
source files~\cite{CoqFiles}.

\begin{definition}[Last Green / Yellow Timestamp]\label{def:lgVy}%
Forall $p \in \{\x, \y\}$, the assertion $ \lgVy\ p\, t$ holds iff the
timestamp $t$ is the last green timestamp of $\histp$ in $\ordlist$ or
the yellow one:
%
\begin{align*}
\lgVy\ p\, t\, v\, \eqdef & t = \mathsf{last\_green}_{\ordlist}\,
\histp \vee \C(t) = \mathsf{yellow}
\end{align*}
\end{definition}

\begin{lemma}[Stability of Assertions II]\label{lem:menvs-scan}
The following assertions are stable under interference from
environment steps:

\begin{enumerate}
 \item \label{lem:menvs:green-phist}

 \item\label{lem:menvs:green-subhist} Forall history $h$ such that $ h
   \subseteq \hist$, $\C(\{ \ordlist \cap \dom{h}\})$ is stable.

 \item \label{lem:menvs:none-lgVy} Forall $p \in \{x, y\}$ the
   assertion $\fwdp{p} \hpts \bot \implies \lgVy\ p\, t\, v$ is stable.

 \item \label{lem:menvs:some-lgVy} Forall $p \in \{x, y\}$, if $\sss =
   \sOn$ and $\spp = \TT$ then the assertion $ \lgVy\ p\, t\, v$ is
   stable.
\end{enumerate}
\end{lemma}

\gad{It should! be $\implies$, else we wouldn't be able to assert the
  property, or its stability). More to follow.}

%% For clarity sake, we further abbreviate the assertion in
%% Lemma~\ref{lem:menvs-scan}, item~\ref{lem:menvs:none-lgVy} with the
%% following notation:
%% %
%% \begin{equation} \label{def:none-lgVy}
%%  \bot{-}\aux{last}_{G \vee Y}\ p\ t\ v \eqdef \fwdp{p}
%%  = \bot \implies \lgVy\ p\, t\, v\, \histp\, \ordlist
%% \end{equation}

%
% setBit
\[
\begin{array}{l}
\lat\,\actwrite{\s}{b};\ \aux{setS}(b) \rat :\\
~~ \logvar{h,w,t_x,t_y,v_x,v_y}\\
~~~
   \tsPre{\{\, \histS = \hempty \wedge h \subseteq \hist \wedge
                    w \subseteq \stableorder \wedge}\\
~~~
   \tsPre{\phantom{\{\,}
   \textrm{if}\ b~
   \textrm{then}\ (\sss, \sx, \sy) = (\sOff\,\_, \neg b, \neg b)} \\
~~~
\tsPre{\phantom{\{,\textrm{if}\ b~}
   \textrm{else}\ (\sss, \sx, \sy) = (\sOn, \neg b, \neg b) \wedge
  \C(\{ t \in \ordlist | t \in \dom{h}\}) = \mathsf{green} \wedge}\\
~~~
%% \tsPre{\phantom{\{,\textrm{if}\ b\ \textrm{else}}\
%%        \bot{-}\aux{last}_{G \vee Y}\ \x\, t_x\, v_x \wedge
%%        \bot{-}\aux{last}_{G \vee Y}\ \y\, t_y\, v_y \}}\\[5pt]
\tsPre{\phantom{\{,\textrm{if}\ b\ \textrm{else}}\
       (\fwdp{x} \hpts \bot \implies \lgVy\ x\, t_x\, v_x) \wedge
       (\fwdp{y} \hpts \bot \implies \lgVy\ y\, t_y\, v_y)\}}\\[5pt]
~~~
   \tsPos{\{\ \histS = \hempty \wedge h \subseteq \hist \wedge
                    w \subseteq \stableorder \wedge}\\
~~~
   \tsPos{\phantom{\{\,}
   \textrm{if}\ b~
   \textrm{then}\ (\sss, \sx, \sy) = (\sOn, \neg b, \neg b)} \\
~~~
\tsPos{\phantom{\{,\textrm{if}\ b~}
   \textrm{else}\ \exists\, t\ldot\,
         (\sss, \sx, \sy) = (\sOff\ t, \neg b, \neg b) \wedge
  \C(\{ t \in \ordlist | t \in \dom{h}\}) = \mathsf{green} \wedge}\\
~~~
\tsPre{\phantom{\{,\textrm{if}\ b~\textrm{else}\ \exists\, t\ldot\,}
       \bot{-}\aux{last}_{G \vee Y}\ \x\, t_x\, v_x \wedge
       \bot{-}\aux{last}_{G \vee Y}\ \y\, t_y\, v_y \}}
\end{array}
\]

\gad{ToDo:Explain!}

%% % clear
%% \[
%% \begin{array}{l}
%% \lat\,\actwrite{\fwdp{p}}{\bot};\ \aux{clear}(p)\rat :\\
%% ~\logvar{h,w}\
%% \!\!\!\!\begin{array}[t]{l}
%% \tsPre{\{\, \histS = \hempty \wedge h \subseteq \histO \wedge
%%              w \subseteq \stableorder \wedge \wpp = \wClean\, t\, v \wedge
%%              h \subseteq \sideal{\stableorder}{t}\}}\ \\[5pt]
%%  \tsPos{\{\, \exists t\ldot\, \histS = j \hpts (p,v) \wedge
%%             h \subseteq \histO \hunion \histS \wedge
%%             w \subseteq \stableorder \wedge
%%             h \subseteq \sideal{\stableorder}{t} \wedge \wpp = \wInit\}}
%% \end{array}
%% \end{array}
%% \]

%% \gad{ToDo:Explain!}

%% % read p
%% \[
%% \begin{array}{l}
%% \lat \act{read}(p) \rat :\\
%% ~\logvar{h,w}\
%% \!\!\!\!\begin{array}[t]{l}
%% \tsPre{\{\, \histS = \hempty \wedge h \subseteq \histO \wedge
%%              w \subseteq \stableorder \wedge \wpp = \wClean\, t\, v \wedge
%%              h \subseteq \sideal{\stableorder}{t}\}}\ \\[5pt]
%%  \tsPos{\{\, \exists t\ldot\, \histS = j \hpts (p,v) \wedge
%%             h \subseteq \histO \hunion \histS \wedge
%%             w \subseteq \stableorder \wedge
%%             h \subseteq \sideal{\stableorder}{t} \wedge \wpp = \wInit\}}
%% \end{array}
%% \end{array}
%% \]

%% % read fp
%% \[
%% \begin{array}{l}
%% \lat \act{read}(\fwdp{p}) \rat :\\
%% ~\logvar{h,w,t_p,v_p}\
%% \!\!\!\!\begin{array}[t]{l}
%% \tsPre{\{\, \histS = \hempty \wedge h \subseteq \histO \wedge
%%              w \subseteq \stableorder \wedge \wpp = \wClean\, t\, v \wedge
%%              h \subseteq \sideal{\stableorder}{t}\}}\ \\[5pt]
%%  \tsPos{\{\, \exists t\ldot\, \histS = j \hpts (p,v) \wedge
%%             h \subseteq \histO \hunion \histS \wedge
%%             w \subseteq \stableorder \wedge
%%             h \subseteq \sideal{\stableorder}{t} \wedge \wpp = \wInit\}}
%% \end{array}
%% \end{array}
%% \]

%% % relink
%% \[
%% \begin{array}{l}
%% \lat\,\aux{relink}(\var{rx}, \var{ry})\rat :\\
%% ~\logvar{h,w,t_x,t_y}\
%% \!\!\!\!\begin{array}[t]{l}
%% \tsPre{\{\, \histS = \hempty \wedge h \subseteq \histO \wedge
%%              w \subseteq \stableorder \wedge \wpp = \wClean\, t\, v \wedge
%%              h \subseteq \sideal{\stableorder}{t}\}}\ \\[5pt]
%%  \tsPos{\{\, \exists t\ldot\, \histS = j \hpts (p,v) \wedge
%%             h \subseteq \histO \hunion \histS \wedge
%%             w \subseteq \stableorder \wedge
%%             h \subseteq \sideal{\stableorder}{t} \wedge \wpp = \wInit\}}
%% \end{array}
%% \end{array}
%% \]



\gad{Complete with scanner methods!}

