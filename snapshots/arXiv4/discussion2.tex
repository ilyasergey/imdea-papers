\section{Discussion}
%\section{Client-side reasoning}
\label{sc:discussion}

\begin{comment} 
We put a discussion section back in, to discuss/address the complaints
from the ETAPS reviewers: the comparison with linearizability, the
unclear generalization principle, and the complexity of the
specs. Moreover, I think that the comparison with Jayanti's proof and
the read-pair example belong here rather than in the related work
section.
\end{comment}

%% In this section, we try to anticipate the questions that a reader
%% might have as to the general approach to concurrent data structures
%% verification advocated in this paper. Some of these concerns have been
%% raised by our fellow colleagues while discussing these work, and some
%% of them have already been discussed elsewhere.\todo{[Citation-Needed]}

%In this section we discuss different aspects of the general approach
%to concurrent data structures verification advocated in this paper.
%
%% , its connection to linearizability, and, in particular how our
%% proof of correctness for Jayanti's snapshot construction relates to
%% its original proof.

\begin{comment}
Aleks' comments on 04/10 w.r.t. why our spec is better than what
linearizability provides, even if the spec looks a priory more
complicated:

We could reduce the spec by (rule of consequence) weakening, to get
the assertion of the sequential spec $r =\mathsf{eval}\ \hist $, but
that would not be useful for client reasoning, as you would be losing
the information needed to combine the client in parallel with others:
take the example in Figure~2---the client, we've (almost) proved in
Coq---and a linearization of the client on the left and the center
{\tt l: write x 2; c: scan; l: write y 1 } and do a proof using the
sequential spec. Now if you are given a different linearization, {\tt
  l: write x 2; l: write y 1; c: scan () }, you need to re do the
proof. Moreover, there is no way to scale up adding more threads, if
you add the client on the right, r, none of the proofs you have can be
reused.

In our setting, and with our specs, client reasoning depends solely on
the API for scan and write, regardless of the different linearizations
of a program. We use this to our advantage while building up the proof
for the client in Figure~2.

\gad{ After talking with Aleks' on 10/01, it became clear that
  we should tone down certain comparisons with linearizability:
  whenever we were thinking of sequential specs clients, I at least,
  was thinking that in terms of {\it sequential} Hoare-logic specs
  which don't feature any auxiliary state. According to Reviewer 1
  and, also to Artem et. al. a sequential spec can be forced by
  histories as well, tough sequential. \\ In 1.(1) Reviewer 1 states
  that%
  \begin{quotation}
    (...) In client reasoning, method calls are replaced by atomic
    commands (e.g. a call to the method write will be replaced by an
    atomic write), thus existing concurrent program logics should
    still apply. (...)
  \end{quotation}%
  A similar observation was made to me by Artem in one of my meetings
  last week: for them, specs are predicates over abstract histories
  and their interpretation of a ``sequential spec'' is not necessarily
  a predicate over state but rather one which is satisfied by a
  sequential history $$ H \models \{P_{\textrm{seq}}\} c
  \{Q_{\textrm{seq}}\} \iff H \models \{P\} c \{Q\} \wedge
  H\ \textrm{is sequential} $$ 
\end{comment}

\subparagraph*{Comparison with linearizability, revisited}
  
As we argued in Section~\ref{sc:formal}, our specifications for the
snapshot methods directly capture that the method calls can be placed
in a linear sequence, in a way that preserves the order of
non-overlapping calls. This is precisely what linearizability achieves
as well, but by technically different means. We here discuss some
similarities and differences between our method and linearizability.

The first distinction is that linearizability is a property of a
concurrent object, whereas our specifications are ascribed to
individual methods, as customary in Hoare logic. This immediately
enables us to use an of-the-shelf Hoare logic, such as FCSL, for
specification. 

Second, linearizability draws its power from the connection to
contextual refinement~\cite{FilipovicOHRW+TCS10}: one can substitute a
potentially complex method $A$ in a larger context, by a simpler
method $B$, to which $A$ linearizes. In our setting, such a property
is enabled by a general substitution principle, which says that
programs with the same spec can be interchanged in a larger context,
without affecting the larger context's proof. Moreover, contextual
refinement (and thus linearizability) is defined for general programs,
without regard to their preconditions and postconditions. However, it
is often the case that the refinement only holds if the substituted
programs satisfy some Hoare logic spec. In this sense, our setting is
more expressive, since the substitution principle is given relative to
a Hoare logic spec.

Finally, while our specification of the snapshot methods are motivated
by linearizability, there is no requirement---and hence no
proof---that an FCSL specification implies linearizability. But this
is a feature, rather than a bug. It enables us to specify and combine,
in one and the same logic, programs that are linearizable, with those
that are not. We refer to~\cite{SergeyNBD+OOPSLA16} for examples of
how to specify and verify non-linearizable programs in FCSL.


\begin{comment}
First of all, we do not see at this moment how to formally connect the
specifications in FCSL in general with linearizability for the
following reasons. Linearizability is a property of a concurrent
objects, whereas FCSL specifications are ascribed to individual
methods. Moreover, it is not the case that if methods of some object
are ascribed FCSL specs that the linearizability of the structure
follows. In fact, as we have argued in the previous
work~\cite{SergeyNBD+OOPSLA16}, FCSL can be used to specify
\emph{non-linearizable} objects. But this is a strength, not a
weakness: one can use one and the same logic to reason about and
combine programs that are specified by different criteria. 
%\end{draft}

Since we do not propose a new correctness criteria, we do not see the
need to prove {\it formally} a connection between FCSL specifications
and linearizability. Moreover, it is not clear how to formally
establish such a connection: linearizability is a property of a
concurrent data structure, whereas FCSL, being a concurrent separation
logic, is concerned with providing specifications to individual
methods. Moreover, FCSL has been used to prove coarse-grained
concurrent data structures~\cite{LeyWildN+POPL13}, different
fine-grained linearizable data structures~\cite{SergeyNB+ESOP15}, some
of them supporting {\it helping}~\cite{SergeyNB+PLDI15}, and even
non-linearizable ones~\cite{SergeyNBD+OOPSLA16}. Then, it is quite
likely that if such a connection does indeed exist, it has to be
proven for each concurrent data structure.

In this paper's case, we have also taken the fact that Jayanti's data
structure is linearizable---something that Jayanti already
did~\cite{Jayanti+STOC05}---as a starting point to the design of the
auxiliary state and its invariants. This is witnessed by the fact the
two non-overlapping events are always sorted by $\stableorder$
(Lemma~\ref{def-jleq}); and by Invariant~\ref{inv:overlap} which
explicitly forbids reordering such non-overlapping events.

An intermediate step could be to encode linearizability as an FCSL
assertion, and include the latter as part of the specification of each
of the data structure's methods. We argue that such approach would be
too expensive in the mechanization effort and there would yield little
or no benefit to either proving a correct specification of a method
and/or the verification of the data structure's clients.

If we were to follow this path, we would definitely have to extend the
auxiliary state space with extra structure: e.g. moving away from FCSL
style of histories---\ie one history entry per method---to
linearizability histories where every line of code is instrumented
and/or timestamped. This would result in several more invariants added
to the definition of the concurrent resource.

In our experience with the mechanized verification of fine-grained
concurrent programs, we have learned that it is important to avoid
adding unnecessary structure to the auxiliary state and its
invariants. Even when proving some book-keeping property might be
trivial, if it has to be done inductively---e.g. for each of the
auxiliary code methods in Figure~\ref{fig:auxcode}---you would better
be sure that the set of invariants is as optimal as
possible.
%\footnotemark.

%% %% This example was too much detail

%% \footnotetext{An example of such optimization is the following:
%% consider Invariant~\ref{inv:readFP} and
%% Lemma~\ref{lemma:first-read}. Both establish similar properties,
%% that at a certain point in the execution of the scanner a timestamp
%% points to the last green or yellow write event in the history
%% $\histp$. They were both originally invariants of the snapshot's
%% concurrent resource $C$, but then we proved the latter follows from
%% the the rest of the invariants. If it remained an invariant, we
%% should have proved that it was preserved by each of the auxiliary
%% code methods presented in Figure~\ref{fig:auxcode}. Consequently,
%% we remove it of the invariants and prove it only once as a derived
%% lemma.}

Even so, we argue that our specs already capture some essence of
linerizability: the set $\scanned{\stableorder}$ is a chain
(Lemma~\ref{lemma:complete-green}), which totally orders the complete
history of the data structure
(Invariants~~\ref{lem:scanned}.\ref{lem:scanned:total}
and~~\ref{lem:scanned}.\ref{lem:scanned:wkn}) before and after the
execution of each command. Then, each call to \jyscan\ possibly
extends the current linearization of the data-structure, resolving the
order of events that have being left out of the last
linearization. The monotonicity of $\scanned{\stableorder}$, by
Invariant~\ref{inv:mono}.\ref{inv:mono:stable}, acts then as a safety
property stating that the linearizations of previous existing events
is respected on each step.

Finally, we argue that not proving linearizability is a feature, not a
bug. As we stated in the Introduction, and illustrated in
Section~\ref{sc:clients}, proving linearizability is orthogonal to
performing client reasoning. And if our specs can already establish
the relevant ordering information of events required to proving
interesting properties of client programs, proving linearizability
would become a detour.

\gad{I had Juanma (Crespo) comment on this paragraph in particular---I
  wanted an outsider opinion with a verification background and he had
  the free time. Moreover my feeling was that this paragraph was
  coming across as smug, and he being my friend, he knows when I'm
  being a bit pedantic and enjoys calling me out. Surprisingly, he
  told me that he found it {\bf quite defensive}. Which, after a third
  look I agree. \\ He suggested whether it wouldn't be a better idea
  to establish the comparison with linearizability here as a trade-off
  in between having a unique ``simple'' spec that requires a
  specialized, more complex logic to having a tailored spec in a more
  general purpose logic which doesn't require extra metatheory. Then,
  from a theorem proving perspective it's better to commit to a
  simpler logic at the cost of having to tailor the specs, than giving
  a general spec which is harder to implement/ proof properties
  in. \\ Its an interesting opinion, and I'm trying to see whether I
  can use his input somehow. At least the part about being too
  defensive.}

%% Reinforce the argument we made in the client's section, even if we
%% have to repeat it again. Linearizability alone doesn't give you
%% enough to prove clients.

%% \gad{Say that in the end linearizability is a strait-jacket, and as
%%   Section~4 shows, even a detour if your goal is to prove clients.}

\end{comment}


%% We believe that the quality of any Hoare-style specification is not
%% determined only by its apparent simplicity but that other factors
%% should be considered, the foremost being their applicability to client
%% reasoning.

%%\subparagraph*{Are these \emph{good} specs?}
%% We claim the specification given to \jyscan\ and \jywrite\ is sensible
%% because, as we shown in Section~\ref{sc:clients}, it enables reasoning
%% about clients, and, moreover, it does so in cases where the obvious
%% sequential Hoare logic specification which just asserts the contents
%% of memory would work for an atomic method call, but no for a client
%% involving a parallel composition.

%% It is debatable whether there exists a \emph{simpler} spec than ours
%% which allows for the same client reasoning. We argue that our spec is
%% nonetheless a good spec, and propose two orthogonal experiments to
%% justify that claim.

%% First, we claim that our spec does not introduce any further
%% complexity from the obvious sequential spec mentioned above, which is
%% not used for dealing with the concurrent nature of the data-structure.
%% We argue that, by restricting syntactically the environment
%% interference as we did in Section~\ref{sc:clients} we can prove that
%% this coarse-grained version of our methods satisfy the sequential,
%% memory only specification:
%% %
%% %
%% %% We omit here the formal details about its implementation using this
%% %% construct, called $\mathbf{hide}$ , and refer the reader
%% %% to~\cite{NanevskiLSD+ESOP14}.  Using this construct, it is possible to
%% %% give the obvious sequential specs to this coarse-grained version of
%% %% our methods:
%% %
%% \[
%% \begin{array}{l l}
%% \mathtt{writeX_{\text{seq}}}\ n &:
%% % \logvar{v,w}\,
%%        {\tsPre{\{\, \heapS = x \hpts v \hunion y \hpts w \}}}
%%        {\tsPos{\{\, \heapSP = x \hpts n \hunion y \hpts w]\}}}\\
%% \mathtt{scan_{\text{seq}}}\ &:
%% % \logvar{v,w}\,
%%        {\tsPre{\{\, \heapS = x \hpts v \hunion y \hpts w \}}}
%%        {\tsPos{\{\, r\ldot\, \heapSP = \heapS \wedge
%%                     r = (v,w)\}}}
%% \end{array}
%% \]
%% %%
%% %%
%% %% \[
%% %% \begin{array}{l l}
%% %% \mathtt{write_{\text{seq}}}\ (p, n) &\eqdef \mathbf{hide}\ (\mathtt{write}\ (p, n)):\
%% %% \logvar{v,w}\,
%% %%        {\tsPre{\{\, \heapS = x \hpts v \hunion y \hpts w \}}}
%% %%        {\tsPos{\{\, \heapSP = \heapS[p \hpts n]\}}}\\
%% %% \mathtt{scan_{\text{seq}}}\ & \eqdef \mathbf{hide}\ \mathtt{scan}\ :
%% %% \logvar{v,w}\,
%% %%        {\tsPre{\{\, \heapS = x \hpts v \hunion y \hpts w \}}}
%% %%        {\tsPos{\{\, r\ldot\, \heapSP = \heapS \wedge
%% %%                     r = (v,w)\}}}
%% %% \end{array}
%% %% \]
%% %
%% Here, $\heapS$ denotes the contents of the underlying heap. The
%% derived specs for $\mathtt{writeX_{\text{seq}}}$ and
%% $\mathtt{scan_{\text{seq}}}$ assert then the obvious sequential
%% behavior of a snapshot: the precondition of
%% $\mathtt{writeX_{\text{seq}}}$, and dually for $y$, asserts that the
%% heap that contains the snapshot is well defined and points to two
%% values, $v$ and $w$. The postcondition just states that the pointer
%% $x$ stores $n$ and the contents of $y$ have not changed. Similarly,
%% the precondition of $\mathtt{scan_{\text{seq}}}$ captures the initial
%% contents of the memory, while the postcondition states that the
%% contents of did not change and that the snapshot is constructed from
%% the values present in memory when $\mathtt{scan_{\text{seq}}}$ was
%% called.

%% \gad{Somebody check this part.}

%In a way, this exercise becomes a lightweight proof of the
%\highlight{{\it adequacy}} of the specs given in
%Figure~\ref{fig:specs}.

%\gad{Is {\bf adequacy} the right word here? I'm trying to say a proof
%  it implements its sequential specification on the heap.}

% \gad{Aleks was not sure about presenting hide here since (i) we are
%   not giving details about it and (ii) I haven't finished
%   implementing the decorator/hiding infrastructure for it. We could
%   add something in the Appendix about hide and also, nobody
%   complained with the sequential version in Section~\ref{sc:clients}
%   and I am confident we can finish this proof, so I'm inclined to
%   leave it. After talking with Anindya, I'm going for presenting the
%   sequential version without presenting the implementation using
%   {\bf hide}, as we did in Section~\ref{sc:clients}.}

%\todo{Incorporate Anindya's comments on the first half}

%\gad{Here we show that the specs aren't complex beyond what's needed
%  for client reasoning.}  \gad{We should consider moving the
%  discussion regarding readpair here.}

\begin{wrapfigure}[10]{r}[0pt]{0.4\textwidth} 
%% \begin{figure}
%
\centering  
\begin{tabular}{l}
% 
\begin{minipage}[l]{5cm}
\small
\begin{alltt}
\num{1}  scan(): \(A {\times} A\)  \{
\num{2}    (cx, vx) \tbnd \act{read}(x);
\num{3}    (cy, _)  \tbnd \act{read}(y);
\num{4}    (_, tx)  \tbnd \act{read}(x);
\num{5}    if vx == tx 
\num{6}    then return (cx, cy)
\num{7}    else scan (); \}
\end{alltt} 
\end{minipage}
%
\end{tabular}
%
\caption{{\tt Scan} using version numbers.}
\label{fig:readpair}
%\end{figure}
\end{wrapfigure}

% \gad{\textbf seems to have no efect whatsoever on alttt}
% \gad{unified action-names not to leave spaces before arguments}


%\subparagraph*{Alternative snapshot implementations and substitution
%principle.}

\subparagraph*{Alternative snapshot implementations.}
FCSL's substitution principle can be exploited further in an
orthogonal way: it allows us to re-use the specs for \jywrite\ and
\jyscan\ in Figure~\ref{fig:specs}, ascribing them to a different
concurrent snapshot algorithm. For that matter, we re-visit the
previous verification in FCSL of the pair-snapshot
algorithm~\cite{SergeyNB+ESOP15}. We present only \jyscan\ in
Figure~\ref{fig:readpair}, as \jywrite\ is trivial.

In this example, the snapshot structure consists of pointers $x$ and
$y$ storing tuples $(c_x, v_x)$ and $(c_y, v_y)$, respectively. $c_x$
and $c_y$ are the payload of $x$ and $y$, whereas $v_x$ and $v_y$ are
version numbers, internal to the structure. Writes to $x$ and $y$
increment the version number, while {\tt scan} reads $x$, $y$ and $x$
again, in succession. Snapshot inconsistency is avoided by restarting
if the two version numbers of $x$ differ. In this paper's notation,
the specification proved for \jyscan in~\cite{SergeyNB+ESOP15} reads:
\[\hfill
\esc{scan} : \tsPos{\{\histS = \hempty\}}\
\tsPos{\{\exists t\ldot \histS' = \hempty \wedge
  r = \eval\ t\ \histP \wedge \dom{\hist}
  \subseteq \ideal{\histP}{t}\}}\hfill
\]
This spec is indeed very similar to the one of \jyscan in
Figure~\ref{fig:specs}, but exhibits that the algorithm does not
require dynamic modification to the event ordering. Thus, by defining
$\stableorder$ to be the natural ordering on timestamps in the global
history $\hist$ (so that $\ideal{\stableorderP}{t} =
\ideal{\histP}{t}$), and taking $\scanned{\stableorder}$ to be the set
of all timestamps in $\hist$ (so that $t \in \scanned{\stableorder}$
is trivially true and can be added to the postcondition above), the
above spec directly weakens into that of Figure~\ref{fig:specs}.
%
Since client proofs are developed in FCSL out of the specs, and not
the code of programs, we can substitute different implementations of
snapshot algorithms in clients, without disturbing the clients'
proofs.
%
This is akin to the property that programs that linearize to the same
sequential code are interchangeable in clients.

\begin{comment}
% I'm removing this section after my conversation with Anindya. The
% point there it is that it's an informal comment that might be
% interesting but at the same time ``increasing the attack surface of
% the paper'' from a reviewers perspective.
 
\subparagraph*{Scalability of our approach.}
%
Even when we have, in practice, only formalized one case study, we can
distill how the proposed verification technique of {\it relinking in
  time} would scale to other more complex data structures. As we have
mentioned throughout our paper, the crux of our technique is the
ability to change atomically the apparent order of past events as long
as they satisfy the invariants of the data-structure and the {\it
  real-time} order of non-overlapping events inherited from {\it
  linearizability}.

An informal taxonomy of the auxiliary code methods can be given,
dividing them into three families according on how they relate with
the logical ordering of events:

\begin{description}
\item[register] Auxiliary code methods whose role is to witness the
  existence of a new atomic event. In Jayanti's snapshot case, we have
  only one: $\aux{register}$. This needs not be the start of the
  method, but rather the atomic moment when it becomes visible to the
  environment: the pointer is written, the value is enqueued,
  dequeued, pushed, popped, \etc.

\item[forward] This class of auxiliary code methods update the
  auxiliary state in order to reflect certain dynamic events. Their
  purpose is to contribute/refine the evidence the $\aux{relink}$
  method needs to assess whether the order of events is correct, and
  if it is not the case, how to fix it. In our case study:
  $\aux{checkS}$, $\aux{forward}$, $\aux{finalize}$, $\aux{set}$, and
  $\aux{clear}$.

\item[relink] auxiliary code methods which change the order to justify
  the correctness of a result. In our case study $\aux{relink}$
  implements such behavior.
\end{description}

The concurrent resource we presented in this paper for Jayanti's
snapshot algorithm considers only {\it write} events in the history,
and computes snapshots by evaluating $\mathsf{scanned}$ segments of
the history. Thus, there could only be one \aux{register}-class
auxiliary method. If we had encoded scan events as well---a needless
and painful exercise, as per our argument above---there would be a
another method of this kind. In a similar way, it is easy to envision
that data-structures with more than one {\it effectful} method like
queues or stacks, will define two auxiliary state code
\aux{register}-like operations of this class, one per thread. Or at
least, two instances of such.

As for \aux{relink}, in the case of a snapshot data structure we argue
that it is optimal to have only one of such methods, and place it in
the \jyscan\ method before returning the snapshot, just as we have
done in this paper.
%
Checking the correctness of this auxiliary code
and, the stability of its specification and proving it satisfies the
2-step invariants defined in Invariant~\ref{inv:mono} constitutes a
significant part of the proof burden in our
mechanization\footnotemark.
%
If we had gone for a design in which a \aux{relink}-like auxiliary
code operation was happening, for instance, together with each of the
reading loops in \jyscan, we would have increased significantly the
size and complexity of our mechanization.\todo{I should try tomorrow
  to crunch the numbers to illustrate the claim about relink being the
  most heavy of the auxiliary code operations.}

We have identified a potential candidates for applying our technique
in the future, such as the TS stack~\cite{DoddsHK+POPL15} and the ever
ubiquitous Herlihy--Wing~\cite{HerlihyW+TOPLAS90,SchellhornWD+CAV12,
  HenzingerSV+CONCUR13, KhyzhaGP+ESOP17} queue. Our preliminary
understanding is that our approach would be to add one \aux{relink}
auxiliary code operation in those cases as well: respectively at the
end of the the \esc{pop} and \esc{dequeue} methods. Of course, we do
not have any evidence to claim this hints to a general rule. Nor we
believe it should be the case. A tailored suit fits better than a
\emph{one-size-fits-all} jacket.

\gad{This last paragraph is very speculative, and we might as well
  scrap it altogether. Also, sorry about the flare at the closing,
  but, I needed to complain.  In our last meeting on Tuesday Aleks
  kept repeating ``Linearizability is a strait-jacket'', and I felt
  compelled to use his wisdom somehow.}

\footnotetext{Some numbers be crunched and placed here.}
\end{comment}

%% This follows the philosophy of previous history-based concurrent
%% resource designs in
%% FCSL~\cite{SergeyNB+ESOP15,SergeyNB+PLDI15,SergeyNBD+OOPSLA16},
%% where histories entries denote the effect

%% We could have gone for an alternative design where the both
%% \jyscan~and \jywrite~methods are recorded in the histories. Again,
%% that would have been an unnecessary complication: introducing {\it
%%   scan} events into $\hist$, would require extra book-keeping
%% invariants with an increased proof burden. The resulting spec is
%% likely to be similar to the one given in Section~\ref{sc:spec}.


%% This has been done in FCSL before--- \eg\, for the {\it Treiber} stack
%% in~\cite{SergeyNB+ESOP15}---.

%\gad{Also, we could move here the comparison with Jayanti's proof.}

\subparagraph*{Relation to Jayanti's original proof.}
\label{sec:relat-jayant-orig}
%
Finally, we close this section by noting that our proof of Jayanti's
algorithm seems very different from Jayanti's original proof. Jayanti
relies on so-called \emph{forwarding principles}, as a key property of
the proof. For example, Jayanti's First Forwarding Principle says (in
paraphrase) that if {\tt scan} misses the value of a concurrent write
through lines~\lineScanReadsX--\lineScanReadsY\ of
Figure~\ref{fig:jayanti-snapshot}, but the write terminates before the
scanner goes through line~\lineScanUnsetsS\ (the linearization point
of \jyscan), then the scanner will catch the value in the forwarding
pointers through lines~\lineScanReadsFX--\lineScanReadsFY.
%
Instead of forwarding principles, we rely on colors to algorithmically
construct the status of each write event as it progresses through
time, and express our assertions using formal logic. For example,
though we did not use the First Forwarding Principle, we nevertheless
can express a similar property, whose proof follows from
% the auxiliary state
Invariants introduced in Section~\ref{sc:auxiliaries}:
%
\begin{proposition}\label{inv:fwd1}%
If $\sss = \sOff\ \toff$ and $\sx = \sy = \TT$---i.e., the scanner is
in lines~\lineScanReadsFX--\lineScanChoosesRY\ and it has unset $\s$
in line~\lineScanUnsetsS\ at time $\toff$---then: $ \forall t \in
\hist\ldot\ t \leq \E(t)< \toff \implies \C(t)= \mathsf{green}$.
%% %
%% \[\forall t \in
%%   \hist\ldot\ t \leq \E(t)< \toff \implies \C(t)= \mathsf{green}\]
\end{proposition}
% This proposition is still somewhat different from the forwarding
%principle, because it states that \emph{all} events, not just the
%missed ones, that terminated before the scanner goes through
%line~\lineScanUsetS, are colored green; hence, are ``observed'' by
%the scanner.  Notice that our forwarding principle is a bit stronger
%than Jayanti's: it states that at the moment the scanner reads from
%the forwarding pointers on lines~\lineScanReadsFX--\lineScanReadsFY,
%all write events that finished before \jyscan's
%line~\lineScanUnsetsS---including those originally missed---are green
%and thus will not be missed by the current scan.  \gad{Better names
%are welcome! Does this explanation helps? In fact, the statement
%reads as if we are verifying that we comply with Jayanti's First FP.}

%%\gad{Something for the second?}
