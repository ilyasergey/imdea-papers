\section{Internal auxiliary state}
\label{sc:auxiliaries}

In order to verify the \emph{implementations} of \jywrite~and \jyscan,
we require further auxiliary state that does \emph{not} feature in the
specifications, and is thus hidden from the clients.

First, we track the point of execution in which \jywrite~and
\jyscan~are, but instead of line numbers, we use datatypes to encode
extra information in the constructors.
%
For example, the scanner's state is a triple $(\sss, \sx,
\sy)$. $\sss$ is drawn from $\{ \sOn, \sOff\, t\}$. If $\sOn$, then
the scanner is in lines~\lineScanSetsS--\lineScanReadsY~in
Figure~\ref{fig:jayanti-snapshot}. If $\sOff\ t$, the the scanner
reached line~\lineScanUnsetsS~at ``time'' $t$, and is now in
\lineScanReadsFX--\lineScanRelinks. $\sx$ is a boolean bit, set when
the scanner clears $\fx$ in line~\lineScanClearsX, and reset upon
scanner's termination (dually for $\sy$ and $\fy$).
%
Writers' state for $x$ is tracked by the auxiliary $\wx$ (dually,
$\wy$). These are drawn from $\{\wInit, \wWrite\ t\, v, \wDirty\ t\,
v, \wClean\ t\, v \}$, where $t$ marks the beginning of the write and
$v$ is the value written to pointer $p$. If $\wInit$, then no write is
in progress. If $\wWrite\ t\, v$, then the writer is in
line~\lineWrtWrt. If $\wDirty\ t\, v$, then $b$ has been set in
line~\lineWrtChk, triggering forwarding. If $\wClean\ t\, v$, the
writer is free to exit.

%% \gad{Should we place a pointer forward to the next section, saying
%%   that these states are the states of the STS that describe
%%   writer/scanner tarnsitions?}

%\gad{Update writers and scanners states and names!}  \gad{Relate to
%the transiton STS Diagram?}\gad{Not sure, cause they are happening in
%the next section?}

Second, like in linearizability, we record the ending times of
terminated events, using an auxiliary variable $\E$. $\E$ is a
function that takes a timestamp identifying the beginning of some
event, and returns the ending time of that event, and is undefined if
the event has not terminated. However, we do not \emph{generate} fresh
timestamps to mark event ending times. Instead, at the end of
\jywrite, we simply read off the last used timestamp in $\hist$, and
use it as the ending time of \jywrite. This is a somewhat non-standard
way of keeping time, but it suffices to prove that events $t_1$ and
$t_2$ which are non-overlapping (\ie, $\E(t_1) < t_2$ or $\E(t_2) <
t_1$) are never reordered. The latter is required by the
postconditions of \jywrite\ and \jyscan, as we discussed in
Section~\ref{sc:formal}. 
%
%Note that wheareas this invariant is a fundamental feature of
%linearizability reasoning, in our setting it is \emph{just} another
%invariant. We will shows later how it is necessary to prove that
%relink behaves corectly but it's not tied to the semantics of our
%method. Implementing other correctness criteria in FCSL---\cf our work
%embedding \textsf{CAL} and \textsf{QQC} reasoning in FCSL
%~\cite{SergeyNBD+OOPSLA16}---, might not necessarily need to enforce
%such strict invariants. \gad{Expand!}
%
%
Formally, the following is an \emph{invariant} of the snapshot object;
\ie, a property of the state space of STS $C$ from
Figure~\ref{fig:specs}, preserved by $C$'s transition.

\begin{invariant}\label{inv:overlap}%
The logical order $\tle$ preserves the real time order of
non-over\-lap\-ping events: $\forall t_1 \in \dom{\E}, t_2 \in
\dom{\hist}$, if $\E(t_1) < t_2$ then $t_1 \tle t_2$.
\end{invariant}

Third, we track the rearrangement status of write events wrt.~an
ongoing \emph{active} scan, by \emph{colors}. A scan is \emph{active}
if it has cleared the forwarding pointers in
lines~\lineScanClearsX\ and\ \lineScanClearsY, and is ready to read
$x$ and $y$. We keep the auxiliary variable $\C$, which is a function
mapping each timestamp in $\hist$ to a color, as follows.
%
%% It is always relative to the ongoing scan, or the last finished one,
%% if no scans are active.
%
\begin{itemize}
\item {\sf Green} timestamps identify write events whose position in
  the logical order is fixed in the following sense: if $\C(t_1) =
  \mathsf{green}$ and $t_1 \tle t_2$, then $t_1 <_{\ordlist'} t_2$ for
  every $\ordlist'$ to which $\ordlist$ may step by auxiliary code
  execution (Section~\ref{sc:implementation}). For example, since we
  only reorder overlapping events, and only the scanner reorders
  events, every event that finished before the active scan started
  will be green. Also, a green timestamp never changes its color.

\item {\sf Red} timestamps identify events whose order is not fixed,
  but which will \emph{not} be manipulated by the active scan, and are
  left for the next scan.

\item {\sf Yellow} timestamps identify events whose order is not fixed
  yet, but which \emph{may} be manipulated by the ongoing active scan,
  as follows.  The scan can \emph{push} a yellow timestamp in logical
  time, \emph{past} another green or yellow timestamp, but not past a
  red one. \emph{This is the only way the logical ordering can be
    modified.}

%
%% They
%% will be put after the green ones.
\end{itemize}
%

There are a number of invariants that relate colors and timestamps. We
next list the ones that are most important for understanding our
proof. We use $\histp$ to denote the set of writes into the pointer
$p$ that appear in the history $\hist$.
% Color invariant

\begin{invariant}[Colors]\label{inv:color}%
The colors of $\histp$ are described by the regular expression $\GYR$:
there is a non-empty prefix of green timestamps, followed by \emph{at
  most} one yellow, and arbitrary number of reds.
\end{invariant}

By the above invariant, the yellow color identifies the write event
into the pointer $p$, that is the \emph{unique} candidate for
reordering by the ongoing active scan. Moreover, all the writes into
$p$ prior to the yellow write, will have already been colored green
(and thus, fixed in time), whether they overlapped with the scanner or
not.

\gad{{\sf R1.Q7}: Do the color patterns follow $\ordlist$? A: Yes they
  do. TO DO: Clarify in the prose above!}

\begin{comment}
\begin{proposition}[First Forwarding Principle]\label{inv:fwd1}%
If the scanner state is $\sss = \sOff\ \toff, \spp = \TT$, then
$\forall\ t \in \histp\ldot\ t \leq \E(t)< \toff \implies \C(t)=
\mathsf{green}$.
\end{proposition}
%
The above is our mathematical formulation approximating the
equally-named, but only informally stated property of
Jayanti~\cite{Jayanti:STOC05}, which says that if \jyscan~misses the
value of a concurrent write in lines~lineScanReadsY
and/or~\lineScanReadsY of Figure~\ref{fig:jayanti-snapshot}, but the
write finishes before the scanner goes through line~\lineScanUnsetsS
(the scan's linearization point), then the scanner will catch the
value in by reading from the forwarding pointer in
lines~\lineScanReadsFX and/or \lineScanReadsFY. In our setting, this
is captured by saying that a write event that finished before $\toff$,
is green. The write event may have been yellow in the past, but the
act of forwarding will paint it green. We will see in
Section~\ref{sc:implementation} that auxiliary code for forwarding
will do just that.

%It is easy to see how this proposition corresponds to Jayanti's First
%Forwarding Priciple: in our seeting a write event which is visible,
%ergo never going to be missed is {\sf green}. Then the Proposition
%above says that any write event that finished strictly before $\toff$
%will be green. This means that in the case that a write event was
%committed to memory after the scanner read in line 10-- in which case
%it was originally painted {\sf yellow} ---it must have been forwarded,
%as forwarding paints entries from yellow to green. As for the Second
%Forwarding Principle, we capture it through the two following
%propositions:

Conversely, Jayanti's Second Principle states that any non-$\bot$
value read in lines~\lineScanReadsX and/or~\lineScanReadsY comes from
a write event that is concurrent with the scan. Moreover, any later
write to the same pointer will finish after the linearization point of
\jyscan~in line~\lineScanUnsetsS, and hence will be missed by the
scanner. While we do not formally capture exactly this property, we
approximate it with the following two invariants, which are
sufficient.

\end{comment}

%\an{Put labels instead of line numbers 12-14.}
%% %% Key Inv for fp
\begin{invariant}[Color of forwarded values]\label{inv:readFP}%
Let $\sss= \sOff\ \toff$, and $p \in \{x, y\}$, and $\spp = \TT$ (\ie,
scanner is in lines~\lineScanReadsFX--\lineScanReadsFY), and $v \neq \bot$ has been forwarded to
$p$; \ie, $\aleksfwdp{p} \hpts v$. Then the event of writing $v$ into
$p$ is in the history, \ie, there exists $t$ such that $t \hpts (p, v)
\in \histp$. Moreover, $t$ is the last green, or the yellow timestamp
in $\histp$.
%(It can then be
%derived that $t$ must be overlapping with the scan).
\end{invariant}

The above invariant restricts the set of events that could have
forwarded a value to the scanner, to only two: the event with the
(unique) yellow timestamp, or the one corresponding to the last green
timestamp. By Invariant~\ref{inv:color}, these two timestamps are
consecutive in $\histp$.

\vspace{10pt}

\begin{invariant}[Red zone]\label{inv:redzone}%
If $\sss = \sOff\ \toff, \sx = \TT, \sy = \TT$, then $\hist$ satisfies
the $\RZ$ pattern. Moreover, for every $t \in \dom{\hist}$:
%
\begin{itemize}
\item $\C(t) = \mathsf{green} \implies t \leq \toff$
\item $\C(t) = \mathsf{yellow} \implies t \leq \toff \leq \E (t)$
\item $\C(t) = \mathsf{red} \implies \toff < t$  
\end{itemize}
\end{invariant}

This invariant restricts the global history $\hist$ (not the
pointer-wise projections $\histp$). First, the red events in $\hist$
are consecutive, and cannot be interspersed among green and yellow
events. Thus, when a scanner pushes a yellow event past a green event,
or past another yellow event, it will not ``jump over'' any
reds. Second, the invariant relates the colors to the time $\toff$ at
which the scanner was turned off (in line~\lineScanUnsetsS,
Figure~\ref{fig:jayanti-snapshot}). This moment is important for the
algorithm; \eg, it is the linearization point for \jyscan~in
Jayanti's proof~\cite{Jayanti+STOC05}.
%
We will use the above inequalities wrt.~$\toff$ in our proofs, to
establish that the events reordered by the scanner \emph{do} overlap,
as per Invariant~\ref{inv:overlap}.

\begin{comment}
Proposition~\ref{inv:readFP} states that any non-$\bot$ value read
from the forwarding pointer $\mathit{fp}$ at line 12 might have been
written to $p$ after the scanner cleared the forwarding pointers, and
and hence, be newer than the values read in line 10. This captures the
essence of the first part of Jayanti's Second forwarding principle.
%
Proposition~\ref{inv:redzone} states that when the scanner is in
between lines~\lineScanReadsFX--\lineScanRelinks\ in
Figure~\ref{fig:jayanti-snapshot}, then $\hist$ can be partitioned
into a non-empty prefix with yellow and/or green timestamps, and a
reds-only suffix. Moreover, if a new write event $t$ occurs at this
point, with $ \toff < t <= \E (t)$, it will be colored red and hence
ignored by the current scanner. This later fact captures the last part
of the second Forwarding Principle. Later on, this proposition will be
crucial into proving that relink does not affect the order of write
events that are meant to missed, by keeping them in the red suffix,
and also that those being reordered are overlapping.

\gad{Explain better? The second part of this invariant might also be
  connected to Jayanti's ``Correctness'' Theorem, Lemma 5 in his
  Appendix. I'm not sure however if I need to express its connection
  here.}
\end{comment}

We can now define the stable logical order $\stableorder$, and the set
$\scanned{\stableorder}$, using the internal auxiliary state of colors
and ending times.
%
\begin{definition}[Logical order $\stableorder$ and
    $\scanned{\stableorder}$]\label{def-jleq}
%\begin{equation*}
\begin{enumerate}
\item $t_1\,{\stableorder}\,t_2 \eqdef (t_1 = t_2) \vee (\E(t_1) <
  t_2) \vee (t_1 \tle t_2 \wedge \C(t_1) = \mathsf{green})$
\item  $\scanned{\stableorder} = \{t \mid \ideal{\stableorder}{t} = \ideal{{\tleq}}{t} \wedge \forall s\in \ideal{\stableorder}{t}\ldot \C(s) = \mathsf{green}\}$.
\end{enumerate}
%\end{equation*}
\end{definition}

From the definition of $\stableorder$, notice that
$t_1\,{\stableorder}\,t_2$ is stable (\ie, invariant under
interference), since threads do not change the ending times $\E$, the
color of green events, or the order of green events in $\tle$, as we
already discussed. 
%
From the definition of $\scanned{\stableorder}$, notice that for every
$t \in \scanned{\stableorder}$, it must be that
$\ideal{\stableorder}{t}$ is a linearly-ordered set
wrt.~$\stableorder$, because it equals a prefix of the \emph{sequence}
$\ordlist$.

%Stability holds because the first two disjuncts do not relate mutable
%values, and the third is stable because green timestamps are not
%reordered nor recolored.
%\gad{Explain stability here}
%
%\gad{Add some comment about Forwarding Principles back in, either here
%  or in Related Work.}
%
%\gad{Add instantiation of the ideal $H^{\hbox{}\sqsubseteq_\ordlist t}
%  $ w.r.t Figures~\ref{fig:weird:exec} and~\ref{fig:reorder}}

%% %
%% \gad{I've changed the definition so that it is closer to its
%%   implementation in the Coq development.}

%% \begin{equation}
%%  t_1 \stableorder t_2  = (t_1 = t_2) \vee (E\,(t_1) < t_2) \vee
%%  (t_1\tle t_2 \wedge \C(t_1) = \mathsf{green}) \label{def-jleq}
%% \end{equation}
%

%% \gad{Introduce Jayanti's forwarding principles and how we implement
%%   them with the colors.  Use this to motivate the invariants in
%%   particular~\ref{inv:wrtP},~\ref{inv:readFP}, and~\ref{inv:redzone},
%%   which are the ones that correspond closely to the forwarding
%%   principles.}

%% \paragraph*{Forwarding Principles}
%% Jayanti introduces two {\it Forwarding
%%   Principles}~\cite{Jayanti:STOC05} that guide the design of his
%% snapshot algorithm and provide an intuition for the correctness of the
%% proof of their linearizability. In essence, the First Principle says
%% that, in Figure~\ref{fig:jayanti-snapshot} if \jyscan~misses the
%% value of a concurrent write when reads in line~10, if the said writer
%% finishes before line~11 (the scanner method's linearization point),
%% the scanner will catch it by reading the forwarding pointer in
%% line~12.


%% The Second Principle then dually states that any non-$\bot$
%% value read in the latter line comes from a write event's timestimap
%% $t$ concurrent with \jyscan~and that for any other write event such
%% that $ t < t'$, $t'$ finishes after line 11 in \jyscan, and thus
%% has to be missed. We capture the essence of these forwarding
%% principles through a series of propositions and invariants expressed
%% in terms of the auxiliary state described above:

%% %% Formalizing the Forwarding Principles in full detail requires keeping
%% %% track of otherwise unnecesary scanner events in the history. Instead,
%% %% we capture their essence with the following three propositions, based
%% %% on the colors of the write events and the scanner state.

%% %
%% \gad{We actually don't have this statement in the development, but we
%%   can prove it out of the refined colorinvariant color\_inv and
%%   Proposition~\ref{inv:redzone} below. I will add it to the Coq files
%%   to-do list.}

%% It is easy to see how this proposition corresponds to Jayanti's
%% First Forwarding Priciple: in our seeting a write event which is
%% visible, ergo never going to be missed is {\sf green}. Then the
%% Proposition above says that any write event that finished strictly
%% before $\toff$ will be green. This means that in the case that a
%% write event was committed to memory after the scanner read in line
%% 10---in which case it was originally painted {\sf yellow}---it must
%% have been forwarded, as forwarding paints entries from yellow to
%% green. As for the Second Forwarding Principle, we capture it
%% through the two following propositions:

%% \begin{comment}
%% %%%%%%%%%  
%%  Jayanti's Forwarding Principles: in quote:
%%  '' The first principle ensures that if the scanner misses a Write
%%  at A[i], then it will catch that Write at B[i]. More precisely:

%%   Forwarding Principle 1:Suppose that a Scan operation S misses
%%   aWrite(i, v)operation W because S reads A[i] before W writes in
%%   A[i]. If W completes before S performs Line 7, then W will have
%%   surely informed S of v by writing v in B[i].

%%   The second principle ensures that any non-\bot value that the
%%   scanner may find in B[i]is current. More precisely:

%%   Forwarding Principle 2:Suppose that a Scan opera- tion S reads at
%%   Line 9 a non-\bot value written in B[i]by some Write(i, v)operation
%%   W.Then, (1) W is concurrent with S, and (2) If W′ is any Write(i,
%%   ∗)operation thatis executed after W,then W′ completes only after S
%%   performs Line 7. (...)''
%% %%%%%%%%
%% \end{comment}

%% %% Key Inv for P
%% \begin{proposition}[Green/Yellow Writes]\item\label{inv:wrtP}%
%% For each of $\histx, \histy$, if the scanner state is $\sss= \sOn,
%% \spp = \TT$, \ie, it is in lines 9--11 in
%% Figure~\ref{fig:jayanti-snapshot}, and $ \mathit{p} = v$ in the
%% physical state, then this is recorded by $\histp$: $\exists t\ldot t
%% \mapsto (p, v) \in \histp$. Moreover, $t$ will be considered by the
%% scanner, \ie, $t$ is the last green or the yellow timestamp in
%% $\histp$.
%% \end{proposition}

%% \gad{I'm not sure Proposition~\ref{inv:wrtP} is helpful here}
  

%% \gad{We might need to explain the new part of the red-zone invariant
%%   \ie the one that explains the relation between the timestamps of
%%   the writes and $t_{\mathsf{off}}$. This is important for
%%   establishing that re-link only changes the order of overlapping
%%   events.}

%% In order to prove the correctness of these specs, we introduce the
%% invariants imposed on our axiliary state, as well as the
%% implementation in FCSL of the algorithm, decorated with auxiliary
%% code:

%% \paragraph*{Invariants}%
%% The following are (selected) properties relevant to the proof of the
%% specs, that the state (real and auxiliary) of the snapshot algorithm
%% satisfies throughout execution.
%% \an{Some are mostly obvious
%%   book-keeping properties, but some are essential. We should present
%%   only the important ones.}

%% % Self/other histories record finish
%% writes.  \item\label{inv:gapless} $\hist$ is gapless, \ie,
%% contains all the timestamps between $0$ and the maximal
%% one. Moreover, each timestamp is associated with either the pointer
%% $x$ or the pointer $y$; if we denote by $\histx$ and $\histy$ the
%% entries in $\hist$ writing into $x$ and $y$ respectively, then
%% $\hist = \histx \hunion \histy$. \an{removable?} \gad{yes}

%% % Self/other histories record finish
%% writes.  \item\label{inv:finished} $\histS$ and $\histO$ record
%% finished writes: $\dom{E} = \dom{\histS \hunion
%% \histO}$. \an{removable?} \gad{yes}

% Color invariant
%% \item\label{inv:color} The colors of the timestamps in each of
%%   $\histx$ and $\histy$ can be described by the regular expression
%%   $\GYR$: They have a non-empty prefix consisting of green timestamps,
%%   followed by \emph{at most} one yellow timestamp, and zero or more
%%   red timestamps in the tail.
%%   %
%%   %% Importantly, there is always a (green)
%%   %%  entry describing the initial write into each $x$ and $y$.
%
%  \ie the coloring of each of the pointer-view histories satisfies
%  the pattern: many (at least one) green, at most one yellow, zero or
%  more red. \gad{Will I need the refinements?}. This pattern entails
%  that there is an initial value in each of $\histx$ and $\histy$ and,
%  moreover, that initial write is {\bf green}. One of those initial
%  timestamps will be $0$, \ie $C(0)= green$. In addition, each
%  pointer-view history is sorted in real time -- naturally sorted--
%  and its elements don't overlap.

%% Non-Overlapping are sorted
%% \item\label{inv:overlap} The ending real time of a finished event
%%   appears after the initial real time: $\forall\, t \in
%%   \dom{E}\ldot\ t \leq E(t)$. \an{removable?}  Moreover, events that
%%   do not overlap in real time are ordered in logical time: $\forall
%%   t_1 \in \dom{E}, t_2 \in \dom{\hist}$, if $E(t_1) < t_2$ then $t_1
%%   \tle t_2$.
  
%% %% Last keys
%% \item\label{inv:lastkey} The entry of the last timestamp in $\histx$
%%   contains the physical value of $\x$: $\x = \histx (\mathsf{last}\,
%%   \histx)$. Similarly for $y$. \an{removable?} \gad{yes}.

%% %% Key Inv for fp
%% \item\label{inv:forward} If the scanner has cleared $\mathit{fx}$
%%   ($\sx = \TT$) but a non-$\bot$ value $v$ has been forwarded since
%%   ($\mathit{fx} = v$), then this is recorded by $\histx$:
%%   $\exists t\ldot t \mapsto (x, v) \in \histx$. Moreover, $t$ will be
%%   considered by the scanner, \ie, $t$ is the last green or the yellow
%%   timestamp in $\histx$. Dually for $\y$.

%% %  For each $\mathtt{p} \in \{\mathtt{x},\mathtt{y}\}$, if $(S_p \wedge
%% %  \mathtt{f\_of\ p}= v)$ then exists $t_p$ such that $t_p \hpts (p,v)
%% %  \in \hist$ and $t_p$ is the last green or yellow timestamp of
%% %  $\hist_{\mathtt{p}}$ \ie $t_x = \mathsf{last\_green}\ \histx \vee
%% %  t_p = \mathsf{yellow\_ts}\ \hist_{\mathtt{p}}$. This property
%% %  entails that after the scanner has cleared the forwarding pointers,
%% %  if some new value has been forwarded, its timestamp is the last
%% %  green or the yellow of the pointer's view-history.

%% %% Key Inv for p
%% \item\label{inv:scanner} If the scanner has cleared $\mathit{fx}$
%%   ($\sx = \TT$), but still has not turned off $\s$ (\ie, it is in
%%   lines 9-11 in Figure~\ref{fig:jayanti-snapshot}), and $\x = v$ in
%%   physical state, then this is recorded by $\histx$:
%%   $\exists t\ldot t \mapsto (x, v) \in \histx$. Moreover, $t$ will be
%%   considered by the scanner, \ie, $t$ is the last green or the yellow
%%   timestamp in $\histx$. Dually for $\y$.

%  For each $\mathtt{p} \in \{\mathtt{x},\mathtt{y}\}$, if $(S_p
%  \wedge \mathtt{S})$ then exists $t_p$ such that $t_p \hpts (p,v) \in
%  \hist$ and $t_p$ is, again, the last green or yellow timestamp of
%  $\hist_{\mathtt{p}}$ \ie $t_x = \mathsf{last\_green}\ \histx \vee
%  t_p = \mathsf{yellow\_ts}\ \hist_{\mathtt{p}}$. This property
%  entails that after the in the scanner has cleared the forwarding
%  pointers and before it unsets the scanner bit {\tt S}, the current
%  value of the pointer was committed by the last green or the yellow
%  timestamp of the pointer's view-history.

%% %% RedZone invariant
%% \item\label{inv:redzone} 
%% %
%% If $\s = \FF, \sx = \TT, \sy = \TT$, \ie, the scanner is in lines
%% 12-15 in Figure~\ref{fig:jayanti-snapshot}, then $\hist$ satisfies the
%% $\RZ$ pattern. In other words, if a new write events occur at this
%% point, it will be colored red, and hence ignored by the scanner.

%If $\wstate{S} = (\FF,\TT,\TT)$ then $\hist$ satisfies the $\RZ$
%coloring pattern: if there is a red timestamp in $\hist$, then then
%the {\sf first} red timestamp partitions $\hist$ so that everything
%before it is yellow or green and everything starting from it is red.
  
%\end{enumerate}
%\gad{to do: I'm not sure how deep I should get into the invariants?
%  Won't know for sure until I finish with the proof sketch. We might
%  cut down on them later.}

%\gad{Also: I'm thinking of pushing the invariants down together with
%  the proof sketch, or even fork them together as a separate section
%  sketching the ``correctness'' of our approach}

%% Jayanti's Forwarding Principles are guidelines for the design of the
%% algorithm and also the definition of the Linearization
%% Points. Instead, our implementation of them, as provided by
%% Propositions~\ref{inv:fwd1},~\ref{inv:readFP}, and~\ref{inv:redzone},
%% are formally implemented as invariants of the auxiliary state of the
%% data structure---or {\it concurrent resource} in FCSL's jargon. In
%% the following we present the atomic auxiliary code operations that
%% implement \jywrite~and \jyscan~in FCSL. These will be required
%% to satisfy the propositions introduced above, amongst other
%% requirements.

%% \an{Should we make the point that the auxiliary code constructivelly
%%   builds and evolves the various total and partial orderings that are
%%   abstracted by the specifications. It is thus an inherent component
%%   of the proof, in the same way that in type theory, existentials
%%   require a witness. There's probably no space to bring type theory
%%   into the mix.}


We close this section with a few technical invariants that we use in
the sequel.

%% Last Key Inv:
%% memory p points to latest entry in Hp according to L

\begin{invariant}[Last write]\label{inv:last-key}%
Let pointer $p \in \{\x,\, y\}$, and $\mathsf{last}_\ordlist\ \histp$
be the timestamp in $\histp$ that is largest wrt.~the logical order
$\tleq$. Then the contents of $p$ equals the value written by the
event associated with $\mathsf{last}_\ordlist\ \histp$. That is, $p
\mapsto \histp (\mathsf{last}_\ordlist \histp)$.
\end{invariant}

%% Joint History invariant The history \histJ are is function of \wx
%% and \wy and not a getter per se. Since we present it as if it were,
%% indeed, a getter, we tie them thorugh the following invariant.

\begin{invariant}[Joint history]\label{inv:joint-hist}%
Let pointer $p \in \{\x,\, y\}$. If the writer for $p$ is active \ie
$\wpp \neq \wInit$, then the write event that it is performing is
timestamped and placed into joint history $\histJ$. Dually, if $t \in
\dom{\histJ}$, then the event $t$ is performed by the active writer
for $p$:
$$t \hpts (p,v) \in \histJ \iff \wpp = \wWrite\ t\, v \vee
\wpp = \wDirty\ t\, v \vee \wpp = \wClean\ t\, v $$
\end{invariant}

%% Domain Bureaucracy
%%

\begin{invariant}[Terminated events]\label{inv:dom-tau}%
Histories $\histO$ and $\histS$ store only terminated events, \ie,
events whose ending times are recorded in $\E$. Moreover, the codomain
of $\E$ is bounded by the maximal timestamp, in real time, in
$\dom{\hist}$:
\begin{enumerate}
\item $\dom{\E} = \dom{\histS} \hunion \dom{\histO}$.
\item $\forall a \in \dom{\E}\ldot\ \E(a) \leq
  \mathsf{last}(\hist)$.
\end{enumerate}
\end{invariant}

%% Scanner State Not used right now, will be needed in the appendix later.
%%
%% \begin{invariant}[Scan state]\label{inv:scan-state}
%% The contents of the scanner bit $S$ and the values of the scanner
%% state triple $(\sss, \sx, \sy)$ are related as follows: if $S \hpts
%% true$ then $\sss = \sOn$ else exists $\toff \in \hist$ such that $\sss
%% = \sOff\, \toff$.
%% \end{invariant}

%% Key Inv for P
\begin{lemma}[Green/yellow read values]\label{lemma:first-read}%
Let $p \in \{x, y\}$. If the scanner state is $\sss= \sOn, \spp =
\TT$, \ie, the scanner is between
lines~\lineScanReadsX--\lineScanReadsY\ in
Figure~\ref{fig:jayanti-snapshot}, and $ p \hpts v$ in the physical
heap, then exists $t$ such that $ t \hpts (p, v) \in
\histp$. Moreover, $t$
%will be considered by the scanner, \ie, $t$ 
is the last green or the yellow timestamp in $\histp$.
\end{lemma}

%%% Proof in prose in the appendix, as well as in coq
\begin{lemma}[Chain]\label{lemma:complete-green}%
If $t\,{\in}\,\dom{\hist}$ and $\C(\ideal{{\tleq}}{t})\,{=}\,\mathsf{green}$, 
%$\forall s \tleq t\ldot \C(s) = \mathsf{green}$ 
then
$\ideal{\stableorder}{t}\ =\ \ideal{{\tleq}}{t}$.
\end{lemma}


