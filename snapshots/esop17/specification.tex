\section{Specification}
\label{sc:formal}

%% Auxiliary state defs

\def\histx{\hist_\x}
\def\histy{\hist_\y}
\def\histp{\hist_p}

\newcommand{\sx}{S_\x}
\newcommand{\sy}{S_\y}
\newcommand{\spp}{S_p}
\newcommand{\sss}{S_s}
\newcommand{\wx}{W_\x}
\newcommand{\wy}{W_\y}
\newcommand{\wpp}{W_p}

%\newcommand{\admissible}{\mathsf{fine}}

% Names for writer/scanner-states
\def\toff{t_{\mathsf{off}}}

%% \newcommand{\wInit}{\mathsf{W_{Off}}}
%% \newcommand{\wWrite}{\mathsf{W_{New}}}
%% \newcommand{\wDirty}{\mathsf{W_{Fwd}}}
%% \newcommand{\wClean}{\mathsf{W_{Done}}}

%% \newcommand{\sOn}{\mathsf{S_{On}}}
%% \newcommand{\sOff}{\mathsf{S_{Off}}}


\paragraph*{General considerations and background.}
%
The main property of our specs is that the state of each data
structure is logically divided into three components: self, joint and
other~\cite{LeyWildN+POPL13,NanevskiLSD+ESOP14,OPLSS:Notes}. These
components may include the real state (i.e., the heap), but also
arbitrary user-defined auxiliary state. Each thread can refer to these
three components, but the self and other components are local, i.e.,
can have different values in different threads.  The self component is
private to the specified thread, and can be modified only by it. The
other component is private to the concurrent environment of the
specified thread (i.e.~the agglomeration of all the interfering
threads), and can be modified only by it. The joint component is
shared, and can be modified by all threads.

It is important that self and other components satisfy the properties
of a Partial Commutative Monoid (PCM), i.e., there exist a partial
commutative and associative operation $\hunion$ which can combine
self- and other- state components, which, moreover, has a unit element
$\hempty$. In this paper, the most important PCM will be that of
\emph{histories} that we use as auxiliary state, to store the the
events executed by various threads~\cite{SergeyNB+ESOP15}. For
example, if the self history of a certain thread executes an event, then the event is
timestamped and stored in the self history of that thread. The local
values of self and other histories must be coherent across
threads. Thus, if the self history of one thread records an event,
then that event is also placed in the other history of every other
thread.

%
%The fact that both heaps and histories satisfy the properties of PCMs
%motived our idea to organize the temporal reasoning about events in
%histories, as needed for Jayanti's algorithm, similarly to spatial
%reasoning in separation logic about linked pointer structures in the
%heap.
%

Of course, histories are common in concurrency, where they are used to
track the various operations of the data structures of interest, such
as, e.g., pushes and pops in the case of stacks, or writes and scans
in the case of snapshot algorithms. In the case of snapshots, however,
our approach affords a bit more economy: histories can track only the
write events, and can ignore the scan events. The latter do not modify
the state in a way that is observable by clients who can access
the shared pointers only via interface methods \jywrite\ and\ \jyscan.

Hence, we define history to be a finite set of entries of the form $t
\mapsto (p, n)$. The entry means that at timestamp $t$ (a natural
number), the value $n$ was written by a call to \jywrite, into the
pointer $p$. A domain of a history is the set of timestamps of its
events. Histories are PCMs under the operation of disjoint union,
which is undefined if operand histories have non-disjoint domains. The
unit element is the empty set. Of course, if one wants to specify a
different concurrent structure---say a stack---then histories can be
defined differently, e.g., as finite sets of entries of the form $t
\mapsto \mathsf{push}\ v$ or $t \mapsto \mathsf{pop}$. Our approach
provides the required expressiveness by being generic wrt.~the PCM of
timestamped histories.

Once the definition of histories is chosen, we specify the methods of
the data structure in terms of three different histories: self, joint,
and other history, as explained above. The self history $\histS$
records the events performed by the thread that invokes the method,
and the other history $\histO$ records the events performed by
others. It is important that these two variables record the write
events that have \emph{terminated}. The joint history $\histJ$ is used
to record \emph{selected} events that have started, but have not yet
terminated. When an event in $\histJ$ finishes, its entry will be
moved from $\histJ$ into the executing thread's $\histS$. In general,
we name by $\hist$ the disjoint union $\histS \hunion \histO \hunion
\histJ$, which is the global history of the data structure. The
disjoint union enforces that $\histS$, $\histO$ and $\histJ$ do not
contain duplicate timestamps.

The approach leaves some freedom in selecting which events to track in
$\histJ$, and when to place them there; the choices will depend on the
program at hand. For example, in our proof of Jayanti's algorithm, the
auxiliary code with which we will annotate \jywrite, will place a new
write event with a fresh timestamp in $\histJ$, but only after the
physical write in line 2 of Figure~\ref{fig:jayanti-snapshot} has
executed. Hence, for this algorithm, $\histJ$ tracks the ongoing write
events whose executions have sufficiently advanced, so that a future
scanner must consider their effects.

\gad{Ilya suggested dropping this paragraph in favour of a one liner
  non-specification, to describe some aspects of the spec in advance
  of what's formally presented.}

\gad{He also complaint about the verbosity of Section 3 in general.}

\an{Would it work to move Fig 4 to the beginning of the section, so
  that the reader sees where we're going, and then explain the rest?}

The timestamps in the global history $\hist$ determine the real-time
ordering of the tracked events. To track the logical ordering of
events, we need further auxiliary notions.
%
The first is the auxiliary variable $\ordlist$, whose type is a
mathematical sequence. The sequence $\ordlist$ is a permutation of
timestamps from $\hist$: this permutation reflects the \emph{logical}
ordering of the events in $\hist$. We write $t_1 \tleq t_2$, and say
that $t_1$ is logically ordered before $t_2$, if $t_1$ appears before
$t_2$ in $\ordlist$. The sequence $\ordlist$ resides in joint state,
and can be dynamically modified by any thread. For example, the
execution of the scanner may reorder $\ordlist$, as shown in
Figure~\ref{fig:reorder:after}. Because $\ordlist$ is a sequence, the
order $\tleq$ is total.

%
%In this sense, $\ordlist$ is the linkage in time for the snapshot
%algorithm, and it will be dynamically modified. 
%

The second auxiliary notion is the \emph{partial} order
$\stableorder$, a suborder of $\tleq$ that is \emph{stable} in the
following sense. Because sequence $\ordlist$ changes dynamically under
interference, it is not appropriate for specifications. In specs
we use $\stableorder$ to relate the timestamps of events whose logical
order has been determined, \emph{and will not change in the
  future}. Thus $\stableorder$ can grow over time, to add new
relations between previously unrelated timestamps, but cannot change
the old relations.

%we require further a subset of ${\tleq}$, that is \emph{stable} in the
%following sense. This order relates the timestamps of events whose
%logical order has already been fixed, and will not change in the
%future. One can say that such events have been ``linearized''.  Being
%stable, $\stableorder$ can only grow over time to add new relations
%between previously unrelated timestamps, but cannot change the old
%relations.
%

To illustrate the distinction between the two orders, we refer to
Figure~\ref{fig:reorder:before}. There, $\ordlist$ represents the
total order $5{-}0{-}2{-}3{-}1$, which changes in
Figure~\ref{fig:reorder:after} to $5{-}0{-}2{-}1{-}3$.
%
Since $1$ and $3$ exchange places, the stable order $\stableorder$
cannot initially relate the two. Thus, one possible order that
$\stableorder$ represents in Figure~\ref{fig:reorder:before} can be
written in lattice notation as
$5{-}0{-}2{<}\!\begin{array}[c]{c}1\\ 3\end{array}$. In
Figure~\ref{fig:reorder:after}, the relation $1{-}3$ is added to this
lattice, making it the total order $5{-}0{-}2{-}1{-}3$. Note how the
previous relations remain unchanged.

\newcommand{\scanned}[1]{\mathsf{scanned}\,#1}

The third auxiliary notion is the set $\scanned\stableorder$ of
timestamps. A write's timestamp is in $\scanned\stableorder$, if that
write has been observed by some scanner; that is, the written value is
returned in some snapshot, or is superseded by another written value
that is returned in a snapshot. All the events logically preceding $t$
also satisfy this property, thus $\scanned\stableorder$ is a totally
ordered subset of $\stableorder$.\footnote{In terminology of
  linearizability, one may say that $t \in \scanned\stableorder$ is
  ``linearized''.}  The set $\scanned\stableorder$ can thus also be
seen as \emph{representing all the scans that already have been
  executed}. This representation of scans allows us to avoid tracking
scan events directly in the history.

In the sequel, we will concretize $\stableorder$ and
$\scanned\stableorder$ in terms of $\ordlist$ and other auxiliary
state. However, we keep the notions abstract in the method specs and
in client reasoning. This will enable us to use other snapshot
algorithms, with the same specs, without invalidating the client
proofs. We also mention that $\ordlist$, $\stableorder$ and
$\scanned\stableorder$ can be encoded as user-level concepts in FCSL,
and require no new logic to be developed.

%%
%% \begin{equation}
%% \begin{array}{l}
%% \mathtt{write}\ (p : \mathtt{ptr}, n : \mathtt{int}) : 
%% \!\!\!\begin{array}[t]{l}
%% \{\hists = \hempty \wedge h \subseteq \histO
%%            \wedge \omega \subseteq {\stableorder}\}\\[3pt]
%% \{\exists t\ldot\, \hists = t \mapsto (p, n) \wedge h \subseteq \histO \wedge
%%   \omega \subseteq {\stableorder}\,{\wedge}\, h \subseteq H^{\hbox{}\sqsubsetneq_\ordlist t}\}
%% \end{array}\\
%% %
%% \mathtt{scan} : 
%% \!\!\!\begin{array}[t]{l}
%% \{\hists = \hempty \wedge\, h \subseteq \histO \wedge\,
%%           \omega \subseteq {\stableorder}\}\\[3pt]
%% \{\hists = \hempty \wedge\, h \subseteq \histO \wedge\, \omega \subseteq {\stableorder} \wedge\, \exists\, t\ldot\, %
%%            h \subseteq H^{\hbox{}\sqsubseteq_\ordlist t} \wedge\,
%%            \mathsf{chain}\ H^{\hbox{}\sqsubseteq_\ordlist t} \wedge\,
%%            r = \mathsf{eval}\ {H^{\hbox{}\sqsubseteq_\ordlist t}}\}
%% \end{array}
%% \end{array}
%% \label{eq:specs}
%% \end{equation}
%

\paragraph*{Snapshot specification.}
% There following version saves a little more space
\begin{figure}
%
\centering
\[
\begin{array}{l}
\mathtt{write}\ (p : \mathtt{ptr}, n : \mathtt{int}) : 
\!\!\!\begin{array}[t]{l}
\{\histS = \hempty \wedge h \subseteq \histO
           \wedge \omega \subseteq {\jleq}\}\\[3pt]
\{\exists t\ldot\, \histS = t \mapsto (p, n) \wedge h \subseteq \histO \wedge
  \omega \subseteq {\jleq}\,{\wedge}\, h \subseteq \chainjltn{\ordlist}{t}\}
\end{array}\\[2em]
%
\mathtt{scan} : 
\!\!\!\begin{array}[t]{l}
\{\histS = \hempty \wedge\, h \subseteq \histO \wedge\,
          \omega \subseteq {\jleq}\}\\[3pt]
          \{\histS = \hempty \wedge\, h \subseteq \histO \wedge\,
            \omega \subseteq {\jleq} \wedge\, \exists\, t\ldot\, %
             h \subseteq \chainjleq{\ordlist}{t} \wedge\,
             \mathsf{chain}\ \chainjleq{\ordlist}{t} \wedge\,
             r = \mathsf{eval}\ \chainjleq{\ordlist}{t}\}
       \end{array}
\end{array}
\]
\caption{\label{fig:specs} Specification of {\tt write} and {\tt scan}
  methods.}
\end{figure}
 
%
Figure~\ref{fig:specs} presents our specs for \jyscan~and
\jywrite. These are partial correctness Hoare triples, describing how
the methods change the state from the precondition (first braces) to
the postcondition (second braces), possibly influencing the value $r$
that the procedure returns. We use VDM-style notation with unprimed
variables for the state before a method executed, and primed ones for
the state afterwards. We use Greek letters to denote values that are
state-dependent and can be mutated by the methods, and Latin letters
for immutable variables. 
%
The component $C$ is a state transition system (STS) describing the
state space of the algorithm, i.e, the relationship between the
auxiliary variables that must always hold, as well as the atomic
transition that the programs are allowed to make when modifying such
state. For now, we keep $C$ abstract, but will define it in
Sections~\ref{sc:auxiliaries} and~\ref{sc:implementation}.
%
We denote by $\ideal{\stableorder}{t}$ the downward-closed set of
timestamps
$\ideal{\stableorder}{t} = \{ s \mid s\ {\stableorder}\ t \}$. Let
$\sideal{\stableorder}{t} = (\ideal{\stableorder}{t})\setminus\{t\}$.

%downward-closed wrt.~$\stableorder$:
%\begin{align*}
%%\label{eq:ideal-order} 
%\alekideal{\stableorder}{t} \eqdef \{ s \mid s\ {\stableorder}\ t \}\qquad\mbox{and}\qquad
%%\label{eq:sideal-order} 
%\aleksideal{\stableorder}{t} \eqdef
%  \{ s \mid s\ {\stableorder}\ t \wedge s \neq te\}
%\end{align*}

The specs for \jywrite\ says the following. The precondition
starts with the empty self history $\histS$, indicating that the
procedure has not made any writes. In the postcondition, a new write
event $t \mapsto (p, n)$ has been placed into $\histS'$. Thus, a call
to \jywrite\ performed a write of $n$ into pointer $p$. The timestamp
$t$ is fresh, because $\histP$ does not contain duplicate timestamps. 
%
%\histS$ and $\hist$ \ab{?} are always disjoint (by the
%semantics of FCSL triples). 
%
Moreover, the write appears as if it occurred atomically at time $t$,
thus capturing in a Hoare-style spec, the atomicity aspect of
linearizability of \jywrite.

The next conjunct, $\dom{\histO} \cup \scanned\stableorder \subseteq
\sideal{\stableorderP}{t}$, positions the write $t$ into the context
of other events. In particular, it says that a write event $s \in
\dom{\histO}$ that finished prior to invoking \jywrite\ is logically
ordered strictly before $t$. Notice that such $s$ does not overlap in
real time with the call (at time $t$) to \jywrite, and it is an
important property of linearizability that events that do not overlap
in real time are not reordered logically. Thus, the above conjunct
encodes this important property of linearizability, but in a
Hoare-style spec. For similar reasons, we require that
$\scanned\stableorder \subseteq \sideal{\stableorderP}{t}$. As
mentioned before, $\scanned\stableorder$ represents all the scans that
finished prior to the call to \jywrite. Consequently, they do not
overlap with \jywrite\ in real time, and have to be logically ordered
before $t$.

Notice what the spec of \jywrite\ \emph{does not prevent}. It
is possible that some event, say with a timestamp $s$, finishes in
real time before the call of \jywrite~at time $t$. Events $s$ and $t$
do not overlap, and hence cannot be reordered; thus
$s\,{\stableorder}\,t$ always. However, the relationship of $s$ with
other events need not be fixed immediately.  For example, in
Figure~\ref{fig:reorder:before}, the event $3$ is immediately ordered
after $5$, $0$, and $2$, but its order wrt.~$1$ is determined only
later.

%For example, if $s$ overlapped with some other event $s'$, it is
%possible for $s$ and $s'$ to not be immediately related in
%$\stableorder$. The relative ordering of $s$ and $s'$ may be
%determined after either of the events has terminated.  \ab{Can we give
%  an example here?}

In the case of \jyscan, we start and terminate with an empty $\histS$,
because \jyscan\ does not create any write events, and we do not
timestamp and track scan events. However, when \jyscan\ returns the
pair $r = (r_\x, r_\y)$, we know that there exists a timestamp $t$
that can describe when the scan took place; this $t$ is the one that
corresponds to the last write preceding the call to \jyscan.

The postcondition relates such $t$ to the other writes by the conjunct
$r = \eval\ t\ \stableorderP\ \histP$. In this expression, $\histP$ is
the entire real time post-history of write events. The function
$\eval\ t\ \stableorderP\ \histP$ works as follows. First, it reorders
$\histP$ according to $\stableorderP$. Next, it computes and returns
the values of $x$ and $y$ that would result from executing the write
events of such reordered history up to the timestamp $t$. Thus, the
conjunct says that \jyscan\ performed a scan of $x$ and $y$,
consistent with the ordering $\stableorderP$, and returned the read
values into $r$. This scan appears as if it occurred atomically,
immediately after time $t$, thus capturing in a Hoare-style spec, the
atomicity aspect of linearizability of \jyscan.  For example, if $t$
is the timestamp of event $1$ in Figure~\ref{fig:reorder:after}, then
$\eval\ t\ \stableorderP\ \histP$ would return $(2, 1)$.
%\ab{Can we connect with Figure 3b here?}

The next conjunct, $\dom{\hist} \subseteq \ideal{\stableorderP}{t}$,
says that the scanner returned a pair of values that is current,
rather than corresponding to an outdated scan. For example,
wrt.~Figure~\ref{fig:reorder}, if \jyscan\ is invoked after the events
$2$ and $1$ have already executed, then it should not return the pair
$(5, 0)$ and have $t$ be the timestamp of the event $0$. Specifically,
the conjunct says that the write events from $\hist$ are ordered no
later than $t$, similar to the postcondition of \jywrite. However,
while in \jywrite\ we used $\dom{\histO} \cup \scanned\stableorder$,
here we use the domain of the full global history $\hist = \histO
\hunion \histJ$. The use of $\histJ$ shows that the scanner will
``see'' and order all of the write events that have been timestamped
and recorded in $\histJ$ (and thus, that have written their value to
memory), prior to the invocation of \jyscan.

Lastly, the conjunct $t \in \scanned\stableorderP$ explicitly says
that $t$ has been ``seen'' by the just finished call to scan. 

Again, it is important what the spec does not prevent. It is possible
that the timestamp $t$ identified by scanner as the moment of the
scan, corresponds to a write that has been initiated, but has not
terminated yet. Despite the ongoing write, $t$ is placed into
$\scanned\stableorderP$, and its order in $\stableorderP$ is fixed
(i.e., $t$ is ``linearized''). Also, notice that the postcondition of
\jyscan\ actually specifies the ``linearization'' order of events that
are initiated by another method, namely \jywrite, thus enabling
implementations of ``non-local'' nature, such as Jayanti's.


We close the section with a brief discussion of how the specs are
used. Because $C$, $\stableorder$ and $\scanned$ are kept abstract to
the clients, we need to provide an interface to work with them. The
latter consists of a number of properties showing how various
assertions interact, summarizeed in the statements below.

The first statement presents the invariants on the transitions of STS
$C$ (often refered to as 2-state invariants). Another way of working
with such invariants is to include them in the postcondition of every
method spec.\footnote{In fact, this is what we currently do in our Coq
  files.} For simplicity, here we agglomerate the properties, and use
them implicitly in proofs as needed.

\begin{invariant}[Transition invariants]\label{inv:monotonicity}
In any program respecting the transitions of $C$: 
\begin{enumerate}
\item\label{inv:mono:hist} $\hist \subseteq \histP$, $\histS \subseteq
  \histSP$ and $\histO \subseteq \histOP$, and also $\stableorder
  \subseteq \stableorderP$.
\item\label{inv:mono:ideal} For every $s \in \scanned{\stableorder}$,
  $\ideal{\stableorder}{s} = \ideal{\stableorderP}{s}$.
\end{enumerate}
\end{invariant}
%
Invariant~\ref{inv:monotonicity}.\ref{inv:mono:hist} says that
histories and $\stableorder$ only grow, but does not insist that
$\histJ \subseteq \histJP$, as timestamps can be removed from $\histJ$
and transfered to $\histS$.
Invariant~\ref{inv:monotonicity}.\ref{inv:mono:ideal} says that if a
new event is added to increase $\stableorder$ to $\stableorderP$, that
event appears logically later than any $s \in
\scanned{\stableorder}$. In other words, once events are observed by a
scanner (i.e., placed into $\scanned{\stableorder}$) in a certain
order, we cannot insert new events among them to modify the past
observation.

The second statement exposes the properties of abstract notions
$\stableorder$ and $\scanned$ that are used for client reasoning.
%
\begin{invariant}[Relating $\stableorder$ and $\scanned$]\label{lem:scanned}%
%The set $\scanned\, \stableorder$ satisfies the following properties:
%
\begin{enumerate}
 \item\label{lem:scan:total} if $ t_1 \in
   \scanned{\stableorder}$ and $t_2 \in \scanned{\stableorder}$, then
   $ t_1 \mathrel{\stableorder} t_2 \vee t_2 \mathrel{\stableorder}
   t_1 $ (totality).
   
 \item \label{lem:scan:wkn} if $ t_2 \in
   \scanned{\stableorder}$ and $ t_1 \mathrel{\stableorder} t_2$, then
   $t_1 \in \scanned{\stableorder}$ (downward closure).

%\item[\textit{(principality???)}] \label{lem:scan:subset} if $ t \in
%  \scanned{\stableorder}$ then $\ideal{\stableorder}{t} \subseteq
%  \scanned{\stableorder}$.

\item \label{lem:scan:prefix} if $t \in
  \scanned{\stableorder}$ and $ \ideal{\stableorder}{t} =
  \ideal{\stableorderP}{t}$, then $t \in \scanned{\stableorderP}$ (principality).
  
\item \label{lem:scan:eval} if $t \in
  \scanned{\stableorder}$, $\ \hist \subseteq \histP$ and $
  \ideal{\stableorder}{t} = \ideal{\stableorderP}{t}$ then $ \eval\,
  t\, \stableorder\, \hist = \eval\, t\, \stableorderP\, \histP$. % (eval preservation).
\end{enumerate}
\end{invariant}


