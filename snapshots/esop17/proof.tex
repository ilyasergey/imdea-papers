\section{Correctness}
\label{sc:proof}

We can now prove that \jywrite\ and \jyscan\ satisfy the
specifications from Figure~\ref{fig:specs}.



This section illustrates some aspects of our proofs of \jywrite\ and
\jyscan. The full proofs can be found in our Coq files.
% including some notewhorthy aspects of their mechanization in
%FCSL/Coq. For lack of space, we cannot present the full theory, but
%just focus on the structure and a few characteristic properties.
The first stage of the proof involves defining the state space of the
algorithm; that is, the definitions of the auxiliaries, and the
invariants that relate them. We introduced the auxiliaries and
selected invariants in Section~\ref{sc:auxiliaries}. 
% There is a number of properties that need to be established under
%the state space, e.g., that it is closed under exchanging history
%entries between $\histS$ and $\histO$. The latter facilitates dynamic
%thread creation, because it allows the reasoning to focus on
%individual threads in a large parallel composition.
The second stage involves defining the auxiliary code, as in
Figure~\ref{fig:auxcode}, and proving a number of its properties. In
addition to correctness, the properties involve \emph{erasure} (i.e.,
auxiliary code only mutates auxiliary state, but not the real one),
\emph{framing} (i.e., if auxiliary code is executable in some input
heap and input history, then executing it in an extended heap and
history leads to the same results, and the extensions remain
invariant), preservation of invariants from
Section~\ref{sc:auxiliaries}, and \emph{termination} of each auxiliary
procedure. The correctness aspect of auxiliary code involves proving
that the code preserves the state-space invariants, and that the
precondition and postcondition hold.

\gad{Refer to a previous lemma that should be added at the
  end of Section~\ref{sc:implementation}, saying that the auxiliary
  code satisfies the state space invariants.}

The third stage consists of giving stable FCSL specifications to the
atomic commands in Figure~\ref{fig:fcsl-snapshot} that combine state
mutation with auxiliary procedures. A specification is stable if it
holds under the interference of threads that can execute the same set
of atomic commands.

\begin{lemma}[Ghost State Monotonicity]\label{lem:ghosts-mono}%
Any transformation of the auxiliary state variable, $\hist$,
$\stableorder$, $\ordlist$, $\E$ into, $\histP$, $\stableorderP$,
$\ordlistP$, $\EP$, be it by environment or local steps is monotonic:
%
\begin{enumerate}
 \item\label{lem:stableorder-mono} $\stableorder \subseteq
   \stableorderP$. It also holds that for all $t$,
   $\ideal{\stableorder}{t} \subseteq \ideal{\stableorderP}{t}$ and
   $\sideal{\stableorder}{t} \subseteq \sideal{\stableorderP}{t}$.

 \item\label{lem:hist-mono} $\hist \subseteq \histP$. Moreover,
   $\histS \subseteq \histSP$ and $\histO \subseteq \histOP$.

 \item\label{lem:scanned-mono} $\scanned\, \stableorder \subseteq
   \scanned\, \stableorderP$ and $\forall\, t \in \scanned\,
   \stableorder\ldot\, \ideal{\stableorder}{t} =
   \ideal{\stableorderP}{t}$.

 \item\label{lem:events-mono} $\dom{\ordlist} \subset \dom{\ordlistP}$.

 \item\label{lem:ends-mono} $\E \subseteq \EP $.
\end{enumerate}
\end{lemma}

\gad{This would probably have to be split into two lemmas, the first
  half going into Section~\ref{sc:clients} and the second one should
  be kept either here, or pulled to the end of
  Section~\ref{sc:auxiliaries}} \gad{TODO:Replace or rephrase to say
  we prove the statments of invariant~\ref{inv:monotonicity}}

% Main correctness proof of the specification
Only then we may have the main correctness result.
%%
%% Final Correctness of the methods
\begin{theorem}[Correctness of {\tt write} and {\tt scan}]\label{thm:specs}
The implementations for {\tt scan} and {\tt write} in
Figure~\ref{fig:fcsl-snapshot} can be ascribed the specification given
in Figure~\ref{fig:specs}.
\end{theorem}
%
The theorem is proved following a customary style of Floyd-Hoare
symbolic evaluation using the inference rules of
FCSL~\cite{NanevskiLSD+ESOP14}, a variant of Hoare logic. We have
mechanized this proof in our FCSL/Coq development~\cite{CoqFiles},
including all the properties from the previous stages described above.

In the sequel, we present the specs for the atomic commands in
Figure~\ref{fig:fcsl-snapshot}, together with the proof outlines for
each of the methods. We do not give a full detailed explanation of
their proofs but we will rather provide a general intuition on why
these assertions hold after each of the atomic actions and also why
there are stable under environment intereference.

\input{atomics2}


%% Moved to appendix-auxiliaries.tex
%%  For example, the correctness
%% proof of $\aux{relink}$, relies on the following two helper lemmas.
%The first lemma asserts that $\aux{inspect}$ correctly determines the
%``offending'' timestamp; the second lemma asserts that $\aux{push}$
%modifies the logical ordering in a way that makes the pair $(r_x,
%r_y)$ a valid snapshot.

%% \begin{lemma}[Correctness of $\aux{inspect}$]\label{lem:inspect}
%% If $t_x$, $t_y$ are timestamps for write events of $r_x$, $r_y$, then
%% $\aux{inspect}\ t_x\ t_y\ \ordlist\ \C$ correctly determines that
%% $(r_x, r_y)$ is a valid snapshot under ordering $\tleq$ and colors
%% $\C$, or otherwise returns the ``offending'' timestamp. More formally,
%% if $\sss = \sOff\ \toff, \sx =\TT, \sy =\TT$, and for each $p \in
%% \{x,y\}$, $ t_p \hpts (p, r_p) \in \hist$ and $(t_p =
%% \aux{last\_green}\ \histp \vee \C(t_p) = \mathsf{yellow})$, the
%% following are exhaustive possibilities.
%% \an{Here we're not abbreviating lastYG...}
%% % if $\aux{inspect}\ t_x\ t_y\ \ordlist\ \C = \mathsf{No}$ then
%% %$(r_x,r_y)$ is a valid snapshot. Else, if
%% %$\aux{inspect}\ t_x\ t_y\ \ordlist\ \C = \mathsf{Yes}\ p\ t'$ then
%% %$(r_x,r_y)$ is not a valid snapshot and moreover $t'$ is the
%% %offending timestamp in $\histp$. Furthermore, by exhausting the
%% %hypotheses about $t_x$ and $t_y$ we can infer that:
%% \begin{enumerate}
%%  \item If $t_x \tle t_y$ and $ \C(t_x) = {\sf yellow}$, then
%%    $\aux{inspect}\ t_x\ t_y\ \ordlist\ \C =
%%    \mathsf{No}$. Symmetrically for $t_y \tle t_x$.

%%  \item If $ t_x \tle t_y $, $ t_x = \aux{last\_green}\ \histx$, and
%%        $\forall s \in \histx\ldot\ t_x \tle s\,{\implies}\,t_y \tle
%%        s$, then \\ $\aux{inspect}\ t_x\, t_y\, \ordlist\, \C
%%        = \mathsf{No}$. Symmetrically for $t_y \tle t_x$.

%%  \item If $ t_x <_{\ordlist} t_y $, $ t_x = \aux{last\_green}\ \histx
%%    $, $s \in \histx$, and $t_x \tle s \tle t_y$, it follows that
%%    $\aux{inspect}\ t_x\, t_y \ordlist \C = \mathsf{Yes}\, x\, s$ and
%%    $\C(s) = {\sf yellow}$. Symmetrically for $t_y \tle t_x$.
%% \end{enumerate}
%% \end{lemma}

%% \begin{lemma}[Correctness of $\aux{push}$]\label{lem:push}
%% If $\sss = \sOff\ \toff, \sx =\TT, \sy =\TT$, and for $p \in \{x,y\}$,
%% we have $t_p \hpts (p, r_p) \in \hist$ and $(t_p =
%% \aux{last\_green}\ \histp \vee \C(t_p) = \mathsf{yellow})$, and
%%     {$\aux{inspect}\ t_x\ t_y\ \ordlist\ \C = {\sf Yes}\ p\ s$},
%%     then, given: $\ordlist' = \aux{push}\ s\ t_y\ \ordlist$:
%% \an{Here we're not abbreviating lastGY...}
%% \begin{enumerate}
%%  \item If $p = x$, then $(r_x, r_y)$ is a valid snapshot under
%%    $\leq_{\ordlist'}$. Symmetrically for $p = y$.
%% \item If $p = x$, then $\ordlist'$ and $\C' = \C[t_x, t_y \hpts
%%   \mathsf{green}]$ satisfy
%%   Invariants~\ref{inv:overlap}--\ref{inv:redzone}. Symmetrically for
%%   $p = y$.
%% \end{enumerate}
%% \end{lemma}

% Stability.


%% \gad{The Stability lemmas is redundant now. I will fix these later}

%% \begin{lemma}[Stability I]\label{lem:menvs}
%% Let unprimed variables $\ordlist$, $\stableorder$, $\chi$, $\kappa$,
%% etc., denote the state before inteference, and the primed versions
%% denote the state after interference. Then:
%% \begin{enumerate}
%%  \item \label{lem:menvs:hist} $ \hist \subseteq \hist'$ and
%%    $\dom{\ordlist} \subseteq \dom{\ordlist'}$, i.e., histories and
%%    write events are monotone.

%%  \item\label{lem:menvs:jleq} $ \stableorder \subseteq \stableorder'$
%%    \ie stable logical order is monotone. Similarly for
%%    $\ideal{\stableorder}{t}$ and $\sideal{\stableorder}{t}$.

%%  \item \label{lem:menvs:static} $\scanned{\stableorder} \subseteq
%%    \scanned{\stableorderP}$
 
%% % \item \label{lem:menvs:chain} If $\static\, \prefix{t}$ and
%% %   $\mathsf{chain}\, \prefix{t}\, \ideal{\stableorder}{t} $ then $
%% %   \chain\, \prefixP{t}\, \ideal{\stableorder}{t} $ \ie once a scan
%% %   event happended, the suborder remains total.

%%  \item \label{lem:menvs:eval} If $t \in \scanned\, \stableorder $ and
%%    $ \ideal{\stableorderP}{t} = \ideal{\stableorder}{t}$ then $\eval\,
%%    t\, \stableorderP \histP = \eval\, t\, \stableorder \hist$, \ie the
%%    result of a previous scan is invariant under the environment.
%% \end{enumerate}
%% \end{lemma}
%% \begin{proof}
%% By induction on the number of environment steps. The base cases are
%% the actions of the auxiliary code described before in
%% Section~\ref{sc:implementation}, on the \emph{transposition} of the
%% state, where the self and other components are swapped
%% (see~\cite{NanevskiLSD+ESOP14} for further details).
%
%In the online appendices~\cite{CoqFiles}, Appendix~\ref{sc:coq-code}
%gives some insight into the most interesting details of such
%proofs. \qed

