\section{Correctness}
\label{sc:proof}

% Abrevs.
\def\botLGY{{\ensuremath{\mathsf{fwdLastGY}}}}
\def\histLGY{{\ensuremath{\mathsf{lastGYHist}}}}
\def\greenH{{\ensuremath{\mathsf{green\_prefix}}}}
\newcommand{\spz}{S_z}

\begin{figure}[t]
\[
\begin{array}[t]{r l}
  \num{1} & \specK{\{ \histS = \hempty  \wedge
                      w \subseteq \stableorder \wedge h \subseteq \hist \wedge h_o \subseteq \histO \}} \\
  \num{2} & \specK{\{ \histS = \hempty \wedge \wpp =\wInit \wedge
                      w \subseteq \stableorder \wedge h \subseteq \hist \wedge h_o \subseteq \histO \}}\\
  \num{3} &\lat\,\actwrite{p}{v}; \aux{register}(v) \rat; \\
  \num{4} & \specK{\{\exists t\, \ldot\,
                      \histS = \hempty \wedge
                      \wpp = \wWrite\ t\ v \wedge t\hpts(p,v) \in \histJ \wedge
                      \dom{h_{\othersub}} \cup \scanned{w}
                      \subseteq \sideal{\stableorder}{t}\}}\\
  \num{5} & \lat\, b \tbnd \act{read}(S);\ \aux{check}(p,b) \rat;\\
  \num{6} & \specK{\{\exists t\ldot \histS = \hempty \wedge\!
               \wpp =  \textrm{if}\ b\
               \textrm{then}\ \wDirty\ t\ v\ \textrm{else}\ \wClean\ t\ v \wedge\ t\hpts(p,v) \in \histJ \wedge \hbox{}}\\
          & \specK{\hphantom{\{\exists t\ldot \hbox{}}
              \dom{h_{\othersub}} \cup \scanned{w}\!
                   \subseteq \sideal{\stableorder}{t} 
               \}}\\
  \num{7} & \kw{if}\ b\ \kw{then}\ \lat\, \actwrite{\aleksfwdp{p}}{v};\ \aux{forward}(p,v)\rat;\\
  \num{8} & \specK{\{\exists t\ldot \histS = \hempty \wedge
                \wpp = \wClean\ t\ v \wedge t\hpts(p, v) \in \histJ \wedge
                \dom{h_{\othersub}} \cup \scanned{w} \subseteq \sideal{\stableorder}{t} \}}\\
  \num{9} & \lat\,\aux{finalize}(i,v)\rat \\
  \num{10} & \specK{\{\exists t\ldot \histS = t \hpts (p,v) \wedge
              \dom{h_{\othersub}} \cup \scanned{w} \subseteq \sideal{\stableorder}{t} \}}
\end{array}
\]
  \caption{\label{proof:write} Proof outline for $\jywrite$.}
\end{figure}


We can now show that \jywrite\ and \jyscan\ satisfy the specifications
from Figure~\ref{fig:specs}.  As before, we avoid VDM notation in
proof outlines by using logical variables.
%
%Both proofs rely on the following two lemmas.
%
%\begin{lemma}\label{lem:invariants}
%The atomic commands from Figure~\ref{fig:auxcode}) preserve the state
%space Invariants~\ref{inv:overlap}-\ref{inv:scan-state}.
%\end{lemma}
%
%\begin{lemma}\label{lem:ghost-mono}
%The atomic commands from Figure~\ref{fig:auxcode}) satisfy the
%two-step properties from Invariant~\ref{inv:monotonicity}. We
%additionally have the following two-step invariants for the internal
%auxiliary state: $\dom{\ordlist} \subset \dom{\ordlistP}$, and $\E
%\subseteq \EP$.
%\end{lemma}
%
%
%


\subparagraph*{Proof outline for \jywrite}
%
The proof outline for \jywrite is presented in
Figure~\ref{proof:write}.
%
Line 1 introduces logical variables $w$, $h$ and $h_o$, which name the
initial values of $\stableorder$, $\hist$, and $\histO$.
%
Line 2 adds the knowledge that the writer for the pointer $p$ is
turned off ($\wpp = \wInit$). This follows from our implicit
assumption that there is only one writer in the system, which, in the
Coq code, we enforce by locks.

Line 3 is the first command of the program, and the most important
step of the proof. Here $\aux{register}$ allocates a fresh timestamp
$t$ for the write event, puts $t$ into $\histJ$, coloring it yellow or
red, and changes $\wpp$ to $\wWrite\ t\ v$, simultaneously with the
physical update of $p$ with $v$ (see Figure~\ref{fig:auxcode}). The
importance of the step shows in line 4, where we need to establish
that $t$ is placed into the logical order after all the other finished
or scanned events (\ie, $ \dom{h_\othersub} \cup \scanned\,
\stableorder \subseteq \sideal{\stableorder}{t}$). This information is
the most difficult part of the proof, but once established, it merely
propagates through the proof outline.

Why does this inclusion hold? From the definition, we know that
$\aux{register}$ appends $t$ to the end of the list $\ordlist$ (the
clause $\ordlistP = \mathsf{snoc}\ {\ordlist}\ t$ in the definition of
$\aux{register}$ in Figure~\ref{fig:auxcode}). Thus, after the
execution of line 3, we know that for every other timestamp $s$, $s
\tle t$. In particular, $s \neq t$, so it suffices to prove
$s\ {\stableorder}\ t$.
%
We consider two cases: $s \in \dom{h_o}$ and $s \in
\scanned{\stableorder}$.  In the first case, by
Invariant~\ref{inv:dom-tau}, $s \in \dom{\E}$. By freshness of $t$
wrt.~global history $h$ (which includes $h_o$), we get $\E(s) < t$,
and then the desired $s\ {\stableorder}\ t$ follows from the
definition of $\stableorder$.  In the second case, by definition of
$\scanned$, $\C(s) = \mathsf{green}$. Since $s \tle t$, the result
again follows by definition of $\stableorder$.

Still regarding line 4, we note that $t \in \dom{\histJ}$ holds
despite the interference of other threads. This is ensured by the
Invariant~\ref{inv:joint-hist}, because no other thread but the writer
for $p$, can modify $\wpp$. Thus, this property will continue to hold
in lines 6 and 8.

In line 6, the writer state $\wpp$ is updated following the definition
of the auxiliary procedure $\aux{check}$. The conjunct on $\dom{h_o}
\cup \scanned{w} \subseteq \sideal{\stableorder}{t}$ propagates from
line 4, by monotonicity of $\stableorder$ (Invariant~\ref{inv:mono}).
%
Similarly, in line 8, $\wpp$ is changed following the definition of
$\aux{forward}$, and the the other conjunct
propagates. $\aux{Forward}$ further colors a number of timestamps
green, but this is done in order to satisfy the state space invariants
from Section~\ref{sc:formal}, and is not exposed in the proof of
\jywrite.
%
Finally, in line 10, $\aux{finalize}$ moves $t\mapsto(p, v)$ from
$\histJ$ to $\histS$, thus completing the proof. 


%Internally, $\aux{finalize}$ also changes $\wpp$, as otherwise, moving
%$t$ will break Invariant~\ref{inv:joint-hist}, but this is not
%relevant for \jywrite.

%The proof is completed with a switch to the VDM notation, \ie, by
%priming the Greek symbols in the postconditions, and instantiating
%$h$, $h_o$ and $\stableorder$ with their initial values from line 1.

%%
%% Scanner Atomics!
%%
%\subsection{Atomic Commands and Proof Outline for \jywrite}
%\label{sc:atoms-write}

\begin{figure}[t]
\[
\begin{array}[t]{r l}
  \num{1} & \specK{\{ \histS = \hempty \wedge h \subseteq \hist \}} \\
  \num{2} & \specK{\{ \histS = \hempty \wedge
                \sss = \sOff\, \_ \wedge \sx = \sy = \FF
                \wedge h \subseteq \hist\}}\\
  \num{3} & \lat\,\actwrite{\s}{\esc{true}};\ \aux{set}(\esc{true}) \rat;\\
  \num{4} & \specK{\{\, \histS = \hempty \wedge
              \sss = \sOn \wedge \sx = \sy = \FF \wedge
              h \subseteq \hist \}}\\
  \num{5} & \lat\,\actwrite{\var{\fx}}{\bot};\ \aux{clear}(x)\rat;\\
  \num{6} & \specK{\{
                 \histS = \hempty \wedge \sss= \sOn \wedge \sx = \TT
                 \wedge \sy = \FF \wedge
%          & \specK{\hphantom{\{}\,
               h \subseteq \hist \wedge
               \C(\dom{h_{x}}) = \mathsf{green}\}}\\
  \num{7} & \lat\,\actwrite{\var{\fy}}{\bot};\ \aux{clear}(y)\rat;\\
  \num{8} & \specK{\{\histS = \hempty \wedge \sss= \sOn \wedge
             \sx = \sy = \TT \wedge
             h \subseteq \hist \wedge \C(\dom{h}) = \mathsf{green}\}}\\
  \num{9} & \var{vx} \tbnd \lat \act{read}(x) \rat;\\
  \num{10} & \specK{\{ \exists\, t_x\ldot\, \histS = \hempty \wedge
                  \sss = \sOn \wedge \sx = \sy = \TT  \wedge}\\
           & \specK{\hphantom{\{ \exists\, t_x\ldot}
                h \subseteq \hist \wedge \C(\dom{h}) = \mathsf{green} \wedge    
                \botLGY\ x\, t_x\, \var{vx} \}} \\
  \num{11} & \var{vy} \tbnd \lat \act{read}(y) \rat;\\
  \num{12} & \specK{\{ \exists\, t_x\, t_y\ldot\, \histS = \hempty \wedge
              \sss = \sOn \wedge \sx = \sy = \TT  \wedge
              h \subseteq \hist \wedge}\\
           & \specK{\hphantom{\{ \exists\, t_x \, t_y\ldot}
             \C(\dom{h}) = \mathsf{green} \wedge    
            \botLGY\ x\, t_x\, \var{vx} \wedge \botLGY\ x\, t_x\, \var{vy}\}} \\
  \num{13} &\lat\,\actwrite{\s}{\esc{false}};\ \aux{set}(\esc{false}) \rat;\\\
  \num{14} & \specK{\{ \exists\, t_x\, t_y\, \toff \ldot\, \histS = \hempty \wedge
             \sss = \sOff\, \toff \wedge \sx = \sy = \TT  \wedge
             h \subseteq \hist \wedge}\\
           & \specK{\hphantom{\{ \exists\, t_x \, t_y\, \toff\ldot}
             \C(\dom{h}) = \mathsf{green} \wedge    
            \botLGY\ x\, t_x\, \var{vx} \wedge \botLGY\ y\, t_y\, \var{vy}\}} \\
  \num{15} & \var{ox} \tbnd \lat \act{read}(\var{\fx}) \rat;\\
  \num{16} & \specK{\{\, \exists\, t_y\, t'_x\, \toff \ldot\,
              \histS = \hempty \wedge
              \sss = \sOff\ \toff \wedge \sx = \sy = \TT \wedge
              h \subseteq \hist \wedge}\\
           & \specK{\hphantom{\{\, \exists\, t_y\, t'_x\, \toff \ldot}\,
              \C(\dom{h}) = \mathsf{green} \wedge
              \botLGY\ y\, t_y\, \var{vy} \wedge}\\
           & \specK{\hphantom{\{\, \exists\, t_y\, t'_x\, \toff \ldot}\,
              \histLGY\ x\ t'_x\ (\textrm{if}\ r = \bot\
                   \textrm{then}\ \var{vx}\ \textrm{else}\, r)\}}\\
  \num{17} & \var{oy} \tbnd \lat \act{read}(\var{\fy}) \rat;\\
  \num{18} & \specK{\{\, \exists\, t'_x\, t'_y\, \toff \ldot\,
              \histS = \hempty \wedge
              \sss = \sOff\ \toff \wedge \sx = \sy = \TT \wedge
              h \subseteq \hist \wedge}\\
             & \specK{\hphantom{\{\,
                  \exists\, t'_x\, t'_y\, \toff \ldot}\,
              \C(\dom{h}) = \mathsf{green} \wedge
              \histLGY\ x\ t'_x\
                      (\textrm{if}\ \var{ox} = \bot\
                       \textrm{then}\ \var{vx}\
                       \textrm{else}\, \var{ox}) \wedge}\\
             & \specK{\hphantom{\{\,
                  \exists\, t'_x\, t'_y\, \toff \ldot}\,
                 \histLGY\ y\ t'_y\
                       (\textrm{if}\ \var{oy} = \bot\
                        \textrm{then}\ \var{vy}\ \textrm{else}\, \var{oy})\}}\\
  \num{19} & \var{rx} \tbnd \kw{if}\ (\var{ox} \neq\bot)\
                \kw{then}\ \var{ox}\ \kw{else}\ \var{vx};\\
  \num{20} & \var{ry} \tbnd \kw{if}\ (\var{oy} \neq\bot)\
                 \kw{then}\ \var{oy}\ \kw{else}\ \var{vy};\\
   \num{21} & \specK{\{\, \exists\, t'_x\, t'_y\, \toff \ldot\,
              \histS = \hempty \wedge
              \sss = \sOff\ \toff \wedge \sx = \sy = \TT \wedge
              h \subseteq \hist \wedge}\\
             & \specK{\hphantom{\{\,
                  \exists\, t'_x\, t'_y\, \toff \ldot}\,
              \C(\dom{h}) = \mathsf{green} \wedge
              \histLGY\ x\ t'_x\ \var{rx} \wedge
              \histLGY\ y\ t'_y\ \var{ry}\}}\\
  \num{22} & \lat\,\aux{relink}(\var{rx}, \var{ry});\
                \kw{return}\ (\var{rx}, \var{ry})\,\rat \\
  \num{23} & \specK{\{\, r\ldot \exists t\ldot \histS = \hempty \wedge
    r = \eval\ t\ {\stableorder}\ {\hist} \wedge
    \dom{h} \subseteq \ideal{\stableorder}{t} \wedge
    t \in \scanned{\stableorder}\}}
\end{array}
\]
  \caption{\label{proof:scan} Proof outline for $\jyscan$.}
\end{figure}


\subparagraph{Proof outline for \jyscan.}
%
Finally, the proof outline for is given in
Figure~\ref{proof:scan}. Line 1 introduces the logical variable $h$ to
name the initial $\hist$. Line 2 adds the knowledge that $\sss =
\sOff\, \_$ and $\sx = \sy = \FF$, \ie, that there are no other
scanners around, which is enforced by locking in our Coq files.

Line 3 is the first line of the code; it simply sets the scanner bit
$S$, and the auxiliaries $\sx$ and $\sy$, following the definition of
$\aux{set}$. The conjunct $h \subseteq \hist$ follows from
monotonicity by Invariant~\ref{inv:mono}.
%
The first important property comes from the lines 5 and 7. In these
lines, $\aux{clear}$ sets the values of $\sx$ and $\sy$, but,
importantly, also colors the events from $h$ green, first coloring
$x$-events, and then $y$-events. This will be important at the end of
the proof, where the fact that $h$ is all green will enable inferring
the postcondition. Moreover, because green events are never
re-colored, we propagate this property to subsequent lines without
commentary.

The read from $\x$ in line 9, and from $\y$ in line 11, must return
the last green, or the yellow event of their pointer, if no values are
forwarded in $\fx$ and $\fy$, respectively. This holds by
Lemma~\ref{lemma:first-read}, and is reflected by the conjuncts
$\botLGY\ x\ t_x\ \var{vx}$ and $\botLGY\ x\ t_x\ \var{vy}$ in line
12, where:
%
\[
 \hfill \botLGY\ p\ t\ v \eqdef \aleksfwdp{p} \hpts \bot
 \implies \lgVy\ p\ t \wedge t \hpts (p, v) \in \hist \hfill
\]
%
The implication guard $\aleksfwdp{p} \hpts \bot$ will be stripped away
in the future, if and when the reads of forwarding pointers in lines
15 and 17 observe that no forwarding values exist.

In line~13, the scanner unsets the bit $\s$ and records the ending
time of the scanner into the variable $\toff$ in line 14. The
conjuncts $\botLGY\ x\ t_x\ vx$ and $\botLGY\ y\ t_y\ vy$ from line 12
transfer to line 14 directly. This is so because $\aux{set}$ does not
change any colors. Moreover, any writes that may run concurrently with
this scan cannot invalidate the conjuncts. To see this, assume that we
had a concurrent \jywrite\ to $\x$ (reasoning is symmetric for $y$).
Such a \jywrite\ may add a new yellow timestamp $s$, but only if $t_x$
itself is the last green, in accord with Invariant~\ref{inv:color}. In
that case, $t_x$ remains the last green timestamp, and
$\botLGY\ x\ t_x\ vx$ remains valid. The concurrent \jywrite\ may
change the color of $s$ to green, by invoking $\aux{forward}$
(Figure~\ref{fig:fcsl-snapshot}, line 5), but then $\fx$ becomes
non-$\bot$, thus making $\botLGY\ x\ t_x\ vx$ hold trivially.

In lines~15 and~17, \jyscan\ reads from the forwarding pointers $\fx$
and $\fy$ and stores the obtained values into $ox$ and $oy$,
respectively.  By Invariant~\ref{inv:readFP}, we know that if $ox \neq
\bot$, there exists $t'_x$ s.t. $t'_x \hpts (x,ox) \in \hist$, and
$t'_x$ is the last green or yellow write event of $\hist_x$.  In case
$ox = \bot$, we know from the $\botLGY$ conjunct preceding the read
from $\fx$, that such last green or yellow event is exactly $t_x$.
%
The consideration for $\fy$ is symmetric, giving us the assertion in
line 18, where:
%
\[
\hfill\histLGY\ p\ t\ v \eqdef
\lgVy\ p\ t \wedge t \hpts (p, v) \in \hist\hfill
\]

Next, line~19 merely names by $rx$ the value of $vx$, if $ox$ equals
$\bot$, and similarly for $ry$ in line 20, leading to
line~21. Finally, on line~22, the method finishes by invoking
$\lat\,\aux{relink}(\var{rx}, \var{ry});\ \kw{return}\ (\var{rx},
\var{ry})\,\rat$. Thus, it returns the selected snapshot $(r_x,r_y)$
and relinks the events so that the $\stableorder$ justifies the choice
of snapshots.

%% Line 19 merely names by $rx$ the value of $vx$, if $ox$ equals $\bot$,
%% and similarly for $ry$ in line 20, leading to line 21.

%% Finally, line 22 invokes $\lat\,\aux{relink}(\var{rx},
%% \var{ry});\ \kw{return}\ (\var{rx}, \var{ry})\,\rat$, returning the
%% selected snapshot $(r_x,r_y)$ and relinking the events so that the
%% $\stableorder$ justifies the choice of snapshots.

We prove that the final state satisfies the postcondition in line 23,
by using the main property of $\aux{relink}$
(Lemma~\ref{lem:relink-prefix}).
%
First, we pick $t = \mathsf{max}_{\ordlist}(t'_x, t'_y)$. Then $r =
\eval\ t\ \stableorder\ \hist$ holds, by the following argument. By
Lemma~\ref{lem:relink-prefix}.\ref{lem:relink-lgVy}, $rx$ is the value
of the last green timestamp in $\histx$. By
Lemma~\ref{lem:relink-prefix}.\ref{lem:relink-green}, all the
timestamps below $t$ are green, thus $rx$ is the value of the
\emph{last} timestamp in $\histx$ that is smaller or equal to $t$.  By
a symmetric argument, the same holds of $ry$. But then, the pair $r =
(rx, ry)$ is the snapshot at $t$, \ie, equals
$\eval\ t\ \stableorder\ \hist$.

The conjunct $t \in \scanned{\stableorder}$ is proved as
follows. Unfolding the definition of $\scanned$, we need to show
$\ideal{\stableorder}{t} = \ideal{{\tleq}}{t}$, and $\forall s\in
\ideal{\stableorder}{t}\ldot \C(s) = \mathsf{green}$.  The first
conjunct follows from Lemma~\ref{lemma:complete-green}. The second
immediately follows from the first by
Lemma~\ref{lem:relink-prefix}.\ref{lem:relink-green}.

To establish $\dom{h} \subseteq \ideal{\stableorder}{t}$, we proceed
as follows. Let $s \in \dom{h}$. From line 21, we know $\C(s) =
\mathsf{green}$. Because $t'_x$ and $t'_y$ are last green (by
$\ordlist$) or yellow events, by Invariant~\ref{inv:color} it must be
$s \tleq t'_x, t'_y$, and thus $s \tleq t$. However, we already showed
that $\ideal{\stableorder}{t} = \ideal{{\tleq}}{t}$. Thus,
$s\ {\stableorder}\ t$, finally establishing the postcondition.
