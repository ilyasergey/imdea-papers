% \begin{comment}
% Linearizability is the de facto correctness criterion for reasoning
% with concurrent fine-grained data-structures. It works by relating the
% concurrent history of a program with its sequential behaviour. More
% precisely, for each concurrent history of an object, linearizability
% requires that there exists a mapping to a sequential history, such
% that the ordering of matching call/return pairs is preserved either if
% they are performed by the same thread, or if they do not overlap. To
% prove linearizability, one usually has to identify linearization
% points in programs or object methods, showing that this particular
% point is the single, atomic, point where the effect of the operation
% occurs.

% However, for certain complex concurrent objects, proving them to be
% linearizable is not a straightforwards task: the linearization points
% of their methods are not fixed by the structure of the programs
% themselves, but rather depend on intricate interactions with the
% environment. Traditionally, verifying such objects requires a
% dedicated metatheory, e.g. supporting prophecy variables, capable of
% reasoning about their highly speculative nature.

% In contrast, in this paper we will present our on-going effort to
% verify such objects in an existing concurrent Hoare-style logic, FCSL,
% giving them (and proving) intuitive specifications that hide the
% speculative reasoning about the environment. We build on the
% philosophy of FCSL, where by relying upon simple, but powerful,
% abstractions like partial commutative monoids (PCMs), auxiliary
% histories, and state transition systems it is possible prove tight
% specifications for concurrent fine-grained data-structures that
% resemble those of its sequential counterparts.

% To that end, we introduce re-sortable histories as an abstraction that
% (i) can be implemented in FCSL off-the-shelf, without changes to the
% underlying logical framework, and (ii) in combination with other FCSL~
% features, such as history PCMs, it can internalize speculative
% reasoning about the concurrent environment, hiding it from the library
% clients.

% We will illustrate our technique by presenting the mechanization in
% FCSL of an optimal concurrent-snapshot algorithm originally introduced
% by Jayanti, whose correctness itself has a highly-speculative nature,
% relying on non-trivial arguments about its non-fixed linearization
% points. Furthermore, we will comment on a few interesting details
% about its mechanization in the Coq proof assistant and discuss its
% pitfalls and the challenges ahead.
% \end{comment}

% \begin{comment}
% Arguments about linearizability of a concurrent data structure are
% typically carried out by specifying the linearization points of the
% structure's procedures.
% %
% Carrying proofs out of such specification is often cumbersome, as the
% linearization points' position in time can be \emph{dynamic} and
% \emph{non-local}: it can depend on the interference, run-time values
% and events from the past, or even future, and appear in procedures
% other than the one considered.

% In this paper we propose a new method, based on a separation logic,
% for reasoning about concurrent objects with such linearization
% points. The method embraces the dynamic nature of linearization
% points, and encodes it as part of the structure's \emph{auxiliary
%   state}, so that it can be dynamically changed by \emph{auxiliary
%   code}, \emph{in place}.

% We name the idea \emph{linking in time}, because we reason about
% temporal ordering of events using the same logical features that one
% would use to reason about spatial linking via pointers in separation
% logic. Moreover, relying on separation logic to reason about time
% enables that linearization points of a procedure can be specified
% locally, even when they may be positioned in other procedures in the
% classical approach.

% We illustrate the method by verifying an intricate optimal snapshot
% algorithm due to Jayanti.
% \end{comment}

Arguments about correctness of a concurrent data structure are
typically carried out by using the notion of \emph{linearizability}
and specifying the linearization points of the data structure's
procedures.
%
Such arguments are often cumbersome as the linearization points'
position in time can be \emph{dynamic} (depend on the interference,
run-time values and events from the past, or even future),
\emph{non-local} (appear in procedures other than the one considered),
and whose position in the execution trace may only be determined after
the considered procedure has already terminated.

In this paper we propose a new method, based on a separation-style
logic, for reasoning about concurrent objects with such linearization
points. We embrace the dynamic nature of linearization points, and
encode it as part of the data structure's \emph{auxiliary state}, so
that it can be dynamically modified in place by auxiliary code, as
needed when some appropriate run-time event occurs.
%
We name the idea \emph{linking-in-time}, because it reduces temporal
reasoning to spatial reasoning. For example, modifying a temporal
position of a linearization point can be modeled similarly to a
pointer update in separation logic.
%
Furthermore, the auxiliary state provides a convenient way to
concisely express the properties essential for reasoning about clients
of such concurrent objects.
%
%Or, if a procedure $A$ has a linearization point in procedure B, that
%can be modeled by $B$ rearranging a list of temporal nodes, without
%invalidating $A$'s temporal pointers into the list.
%
%Or, if a procedure $A$ has a linearization point in procedure B, that
%can be modeled similarly to $B$ allocating a pointer and then passing
%it on to $A$.% and can be specified locally.
%
% in separation
%logic. Moreover, relying on separation logic to reason about time
%enables that linearization points of a procedure can be specified
%locally, even when they may be positioned in other procedures in the
%classical approach.
%
We illustrate the method by verifying (mechanically in Coq) an
intricate optimal snapshot algorithm due to Jayanti, as well as some
clients.


