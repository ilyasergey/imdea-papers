\section{Implementation and Correctness of $\aux{relink}$}
\label{sc:relink-lemmas}
\def\cat{{\mathbin{+\!\!+}}}

In Section~\ref{sc:implementation}, we described briefly the
implementation of $\aux{relink}$, without giving much details on the
auxiliary helper functions $\aux{inspect}$ and $\aux{push}$. We give
here their definitions, together with some associated properties:

\begin{definition}[inspect]\label{def:inspect}%
Given two timestamps $t_x$, $t_y$ then $\aux{inspect}\ t_x\, t_y\,
\ordlist\, \C$ is defined as follows:
\begin{equation*}
\aux{inspect}\ t_x\, t_y\ \C\ \eqdef
\begin{cases}
  \mathsf{Yes}\ x\ t_z\ \qquad \qquad
      \begin{array}[t]{l}
      \textrm{if}\ t_x \tle t_y,\ t_x = \mathsf{last\_green}_\ordlist\ \histx, \\
       t_z = \mathsf{yellow\_timestamp}_\ordlist\ \histx,\,
       \textrm{and} \ t_z \tle t_y
      \end{array} \\[5pt]
  \mathsf{Yes}\ y\ t_z\ \qquad \qquad
      \begin{array}[t]{l}
      \textrm{if}\ t_y \tle t_x,\ t_y = \mathsf{last\_green}_\ordlist\ \histy, \\
       t_z = \mathsf{yellow\_timestamp}_\ordlist\ \histy,\,
       \textrm{and} \ t_z \tle t_x
      \end{array} \\[5pt]
  \mathsf{No}\hphantom{\mathsf{s}\ x\ t_z}\ \qquad \qquad \textrm{otherwise}
\end{cases}
\end{equation*}
\end{definition}

\begin{definition}[push]\label{def:push}%
 $\aux{push}$ is a surgery operation defined on $\ordlist$ as
  follows:
\[\hfill
  \textrm{Let}\ \ordlist = \ordlist_{<_i}\ \cat
  \ i\ \cat\ \ordlist_{i..j}\ \cat\ j\ \cat\ \ordlist_{>_j},\
  \textrm{then}\
  \aux{push}\ i\ j\ \ordlist =
  \ordlist_{<_i} \cat\  \ordlist_{i..j}\ \cat\ j\ \cat\ i\ \cat \ordlist_{>_j}
  \hfill
\]
\end{definition}

The definition of $\aux{inspect}$ works under the assumption that
$t_x$ and $t_y$ are, respectively, the last green or yellow timestamp
in $\histx$ and $\histy$. This latter fact is recovered in the
definition of $\aux{relink}$ in Figure~\ref{fig:auxcode} and
reinforced in line~21 in the proof of {\jyscan} in
Figure~\ref{proof:scan}. When $\aux{inspect}$ returns
$\mathsf{Yes}\ p\, t_z$, $\ordlistP$ is computed by pushing some $i$
timestamp past another timestamp $j$ in $\ordlist$. The definition of
$\aux{push}$ above shows that this operation is an algebraic
manipulation on sequences. In fact, we implement it using standard
{\it surgery} operations on lists: $\cat$, $\mathsf{take}$, \etc.

In Section~\ref{sc:implementation}, we have mentioned that the
correctness aspect of auxiliary code involves proving that the code
preserves the auxiliary state invariants from
Section~\ref{sc:auxiliaries}. For example, the correctness proof of
$\aux{relink}$, relies on the following helper lemmas. The first lemma
asserts that $\aux{inspect}$ correctly determines the ``offending''
timestamp; the second and the third lemma assert that $\aux{push}$
modifies $\ordlist$ in a way that allows us to prove (in
Section~\ref{sc:proof}), that the pair $(r_x, r_y)$ a valid snapshot.

\begin{lemma}[Correctness of $\aux{inspect}$]\label{lem:inspect}
If $t_x$, $t_y$ are timestamps for write events of $r_x$, $r_y$, then
$\aux{inspect}\ t_x\ t_y\ \ordlist\ \C$ correctly determines that
$(r_x, r_y)$ is a valid snapshot under ordering $\tle$ and colors
$\C$, or otherwise returns the ``offending'' timestamp. More formally,
if $\sss = \sOff\ \toff, \sx =\TT, \sy =\TT$, and for each $p \in
\{x,y\}$, $ t_p \hpts (p, r_p) \in \hist$ and $\lgVy\ p\, t_p $, the
following are exhaustive possibilities.

\begin{enumerate}
 \item If $t_x \tle t_y$ and $ \C(t_x) = {\sf yellow}$, then
   $\aux{inspect}\ t_x\ t_y\ \ordlist\ \C =
   \mathsf{No}$. Symmetrically for $t_y \tle t_x$.

 \item If $ t_x \tle t_y $, $ t_x = \aux{last\_green}\ \histx$, and
       $\forall s \in \histx\ldot\ t_x \tle s\,{\implies}\,t_y \tle
       s$, then \\ $\aux{inspect}\ t_x\, t_y\, \ordlist\, \C
       = \mathsf{No}$. Symmetrically for $t_y \tle t_x$.

 \item If $ t_x <_{\ordlist} t_y $, $ t_x = \aux{last\_green}\ \histx
   $, $s \in \histx$, and $t_x \tle s \tle t_y$, it follows that
   $\aux{inspect}\ t_x\, t_y \ordlist \C = \mathsf{Yes}\, x\, s$ and
   $\C(s) = {\sf yellow}$. Symmetrically for $t_y \tle t_x$.
\end{enumerate}
\end{lemma}

\begin{lemma}[Push Mono]\label{lem:push-mono}
Given elements $a, b, i, j$, all in $\ordlist$, and $\ordlistP =
\aux{push}\ i\, j\, \ordlist$, then:
\begin{enumerate}
\item\label{lem:push:left} If $a \tle i$ then $ a \tle b \implies a
  \tleP b$.
\item\label{lem:push:right} If $j \tle b$ then $ a \tle b \implies a
  \tleP b $.
\item\label{lem:push:window} If $a \neq i$ then $ a \tle b \implies a
  \tleP b $
\end{enumerate}
\end{lemma}

%% \begin{proof}
%% First, as $\ordlist$ is used to order timestamps, it contains only
%% distinct elements. Second, by the definition of $\aux{push}$ above,
%% the changes in the order $\tle$ can therefore occur only in the
%% segment $ i\ \cat\ \ordlist_{i..j}\ \cat\ j$ of $\ordlist$.
%% %
%% Thus, (\ref{lem:push:left}) and~(\ref{lem:push:right}) are trivial, as
%% changing the position of $i$ does not affect $a \tle b$. For
%% (\ref{lem:push:window}), observe that the only relations invalidated
%% by $\aux{push}$ are those $ i \tle b$, where $b \in
%% \ordlist_{i..j}\ \cat\ j$. Since $a \neq i$, it follows that $ a
%% <_{\aux{push}\, i\, j\, \ordlist} b$. \qed
%% \end{proof}


\begin{lemma}[Correctness of $\aux{push}$]\label{lem:push}
Given $\sss = \sOff\ \toff, \sx = \sy =\TT$, and for $p \in \{x,y\}$,
we have $t_p \hpts (p, r_p) \in \hist$, $ \lgVy\ p\, t_p$, and
{$\aux{inspect}\ t_x\ t_y\ \ordlist\ \C = {\sf Yes}\ p\ t_s$}. If we
name $t_z \in \{ t_x, t_y\}$, with $p \neq z$, and $\ordlist' =
\aux{push}\ t_s\ t_z\ \ordlist$, then:
\begin{enumerate}
 \item $\aux{relink}$ satisfies the 2-state invariants from
   Invariant~\ref{inv:mono}.
\item $\histP, \ordlistP, \EP, \CP$ satisfies all the resource
  invariants from Section~\ref{sc:auxiliaries}, \ie
  Invariants~\ref{inv:overlap}--\ref{inv:dom-tau}.
\end{enumerate}
\end{lemma}

In our mechanization, these three lemmas allow us to prove
Lemma~\ref{lem:relink-prefix}, $\aux{relink}$'s main property.
