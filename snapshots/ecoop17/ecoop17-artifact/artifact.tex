% config from instructions webpage
\documentclass[a4paper,USenglish]{lipics-v2016}

\usepackage{microtype}%if unwanted, comment out or use option "draft"

\usepackage{authblk}
\usepackage{fancyhdr}
\usepackage{url}
\usepackage{lipsum}

\title{Concurrent Data Structures Linked in Time}

\author{Germ\'{a}n Andr\'{e}s Delbianco}
\author{Ilya Sergey}
\author{Aleksandar Nanevski}
\author{Anindya Banerjee}

% \affil[1]{IMDEA Software Institute,  Madrid, Spain\\
%           {\texttt{\{german.delbianco,aleks.nanevski,anindya.banerjee\}@imdea.org}}}

% \affil[2]{University College London, UK\\
%   {\texttt{i.sergey@ucl.ac.uk}}}

% \affil[3]{Universidad Polit\'{e}cnica de Madrid, Spain}

\begin{document}
\maketitle
 
\thispagestyle{empty}


\section*{\LARGE{Artifact Description}}
\label{sec:note-reviewers}

Thank you for agreeing to review the artifact for our ECOOP 2017
submission entitled \textit{``Concurrent Data Structures Linked in
  Time''}. The artifact is represented by the Coq sources of the
implementation of Fine-grained Separation Logic (FCSL) in the Coq
proof assistant, as well as the case studies conducted for the
ECOOP'17 submission.

There are two possible ways to test the development: compiling FCSL
from the source files or running the provided virtual machine. We
provide further instructions in the sequel. A brief commentary with
respect to this paper's specific case studies is also provided
below. Moreover, the \texttt{README.md} file also provides information
about previous FCSL developments and case studies.

Finally, if you are interested in finding out more details about the
implementation of FCSL and several case studies, we encourage you to
check the FCSL reference manual:

\begin{centering}
  \url{http://software.imdea.org/fcsl/papers/fcsl-manual.pdf}
\end{centering}

\paragraph*{Building FCSL from source}
Build FCSL from the Coq sources files packed in the ZIP archive
(\texttt{fcsl-ecoop17.zip}). In order to compile and check the sources
and case studies from the papers, you will need to have Coq 8.5pl3 and
Ssreflect 1.6 installed. They can be downloaded from the following
sites:

\begin{itemize}
\item \url{https://coq.inria.fr/coq-85}
\item \url{https://math-comp.github.io/math-comp/}
\end{itemize}

However, we recommend installing both Coq and Ssreflect using the
OPAM package manager. With OPAM installed, copy and paste the
following commands to build Coq 8.5.3 and Ssreflect, together with
their dependencies.

\begin{verbatim}
  opam repo add coq-released https://coq.inria.fr/opam/released

  opam pin add coq 8.5.3

  opam install coq-mathcomp-ssreflect
\end{verbatim}

We also recommend to use Emacs with installed Proof General
mode\footnote{\url{https://github.com/emacsattic/proofgeneral}} to
browse and step through the sources.

Once you have Coq and Ssreflect installed, follow the instructions in
the \texttt{README.md} file of the \texttt{fcsl-ecoop2017.zip} archive
to build the project. It takes around 100 minutes to build the project
and all the case studies, so we thank you in advance for your
patience.

The \texttt{README.md} file also provides commentary on the case
studies from this paper, which are located in the folder
\texttt{Examples/Relink}, as well as a description of previous FCSL
case studies.

\paragraph*{Using the VirtualBox VM image}
%
You can also use the VirtualBox VM image with
Coq/Ssreflect/Emacs/Proof General installed. The VM image is available
by the link in the artifact description. It requires VirtualBox
5.0\footnote{\url{https://www.virtualbox.org/wiki/Downloads}} to
run. Whenever necessary, use \texttt{fcsl} as both the username and
password.

Once successfully launched, navigate to the
\texttt{\textasciitilde/fcsl-ecoop17} folder to browse the project
sources. The project is already compiled, but you can follow the
instructions from \texttt{README.md} file in order to re-build the
project. This will automatically check all the proofs.

Use Emacs with Proof General mode to step through *.v files from the
\texttt{Examples/Relink} folder.

\paragraph*{A brief summary of this paper's Coq development}

\vspace{.5cm}

{\bf \large Folder:} \texttt{Examples/Relink}\\

The folder \texttt{Examples/Relink} folder contains the main case
study of the {\it Linking in Time} technique for verifying concurrent
data structures with non-fixed linearization points introduced in the
paper. It consists of several files implementing the verification in
FCSL of Jayanti's single writer/single scanner snapshot construction:

\begin{itemize}

 \item \texttt{jayanti\_snapshot.v} --- Preliminaries. Lemmas for
       reasoning about coloring patterns of histories, and other
       history invariants.

 \item \texttt{jayanti\_ghoststate.v} --- Implementation of the
 auxiliary state for the snapshot object, its transitions, invariants
 and associated invariant preservation proofs.

 \item \texttt{jayanti\_concurroid.v} --- Implementation of the FCSL
 concurrent resource ---a.k.a. {\it concurroid} for the snapshot.

 \item \texttt{jayanti\_actions.v} --- FCSL atomic actions for the
 snapshot object concurrent resource.

 \item \texttt{jayanti\_stability.v} --- Assertions used in the proof
 outline of the main methods and proofs of their stability under FCSL
 transitions.

 \item \texttt{jayanti\_library.v} --- Implementation and verification
 of the atomic commands (auxiliary code) and the snapshot library
 (i.e. scan and write procedures).

 \item \texttt{jayanti\_clients.v} --- Verification of the clients of
   the snapshot data structure, as presented in Section 4 of the
   paper.
\end{itemize}

In the sequel, we relate the different parts of the development with
their corresponding presentation in the paper:

The invariants described in Section~5, together with the
implementation of the concurrent resource for the snapshot object are
defined in the \texttt{jayanti\_ghoststate.v} and
\texttt{jayanti\_concurroid.v} files. In the later file, there is also
the implementation of the auxiliary code functions (a.k.a transitions
in FCSL jargon) from Section~6. The proof of the 2-state invariants
from Invariant~1, and those from Invariant~2, together with the
definitions of the main assertions (e.g.\ the {\it stable order}
$\Omega$) are carried out in \texttt{jayanti\_stability.v}. The
verification of the \texttt{write} and \texttt{scan} files, as
presented in Section~7, are carried our in the file
\texttt{jayanti\_library.v}.

Finally, the clients in Section~4 are implemented and verified in
\texttt{jayanti\_clients.v}. We refer the reader to the descriptions
in the \texttt{README.md} file of the project and/or the accompanying
manuscript for further detail.

\vspace{.5cm}

{\bf \large Folder:} \texttt{Core}\\

Two new libraries were added to the Core of existing FCSL libraries: a
new library for {\it relinkable} histories, added to
\texttt{histories.v} and an extended theory of algebraic surgery
operations for lists, which were necessary to prove the properties of
the {\it relink} operation. The latter is implemented in
\texttt{seq\_extras.v}.

\end{document}
