\section{Discussion}
%\section{Client-side reasoning}
\label{sc:discussion}

\begin{comment}
%\gad{Section 4 should emphasize that we are hiding colors,
%  end-points, and whatever `` internal'' aspects of the concurroid in
%  client reasoning. This might require to connet with the proof
%  outline from Figure~\ref{fig:weird:code} in the appendix. We should
%  say that this internal only ``leak'' through the
%  definition/instantiations of $\stableorder$.}

%% \gad{This section now needs two parts: (i) a first part about the
%%   clients and (ii) a second part discussing the \emph{principality} of
%%   scan's specs}

%% We already argued in Section~\ref{sc:formal} that the Hoare triples
%% from Figure~\ref{fig:specs} capture precisely what linearizability
%% would be used for, namely that the operations of the snapshot object
%% can be sequenced. The triples do so without the intermediary
%% sequential specifications for {\tt write} and {\tt scan}, and the
%% attending simulation proofs.  In this section, we argue that our
%% approach via Hoare logic can still do a bit more, specifically in the
%% cases of client side reasoning, and reasoning about parallel
%% composition.

%% By relying on FCSL, we immediately inherit the separation logic
%% mechanism for reasoning about programs with nested parallel
%% composition (i.e., dynamic, well-bracketed, forking and joining). We
%% refer the interested reader to the Appendix~\ref{sc:background} for a
%% brief background on FCSL, and the separation-logic style inference
%% rule for parallel composition that requires that program state, real
%% and auxiliary, satisfies the properties of a Partial Commutative
%% Monoid, or PCM (and indeed, our histories form a PCM under the
%% operation of disjoint union, with the empty history as a unit). This
%% gave us a thread-local way to reason about programs, via local
%% variables such as $\histS$ and $\histO$, rather than relying on global
%% abstractions, such as \emph{thread-id}s, to specify the behavior of
%% each thread.  This is in contrast to linearizability, whose very
%% definition requires identifying threads by thread-ids, thus focusing
%% on programs that are a top-level parallel composition of a fixed
%% number of sequential threads, but not providing convenient
%% abstractions for reasoning about nested parallel compositions.

%As a first point, we argue that the pre- and postcondition ascribed
%to {\tt write} and {\tt scan} in Section~\ref{sc:formal} are
%\emph{principal}, in that they can be used in larger program contexts
%without modification. This is not the case in most other verification
%methods, where programs, once verified, typically still have to be
%refactored wrt.~auxiliary state and code \emph{on a per-context
%basis}~\cite{Owicki-Gries:CACM76,Jacobs-Piessens:POPL11}. By using
%\emph{local} variables such as $\histS$ and $\histO$ to name the
%per-thread history, we avoid the need for refactoring. Such locality
%is supported by the FCSL inference rule for parallel
%composition~\cite{LeyWild-Nanevski:POPL13,Nanevski-al:ESOP14}. The
%rule is somewhat unusual, but please see Appendix~\ref{sc:background}
%for a brief description. What matters, however, is that reasoning in
%clients is done strictly out of the specs of {\tt write} and {\tt
%scan}. Notice that, while these expose the existence of the logical
%ordering $\stableorder$ and how some individual events may be related
%in it, they \emph{do not} expose the actual definition of
%$\stableorder$ in terms of colors, or any other internal of the
%auxiliary state and code. Other snapshot implementations can provide
%their own definition of $\stableorder$.

\gad{This part was not useful in Section~\ref{sc:clients}.}

As a brief example of reasoning about parallel composition, it is
possible to prove that the program
%
$ 
P_1 = {\mathtt{write}}\ (x, 1) \parallel {\mathtt{write}}\ (y, 2)
$
%
satisfies the spec which says that two events are executed, in an
unspecified order. Then we can reason about a larger program, say
$
  P_2 = P_1; {\mathtt{write}}\ (x, 3)
$
out of that spec, to prove that $P_2$ executes three events, with the
write of $3$ appearing last. 
%
Moreover, the \emph{substitution principle} holds; that is, the proof
remains valid, even if we replace $P_1$ by another program, say
%
$
P'_1 = {\mathtt{write}}\ (x, 1); {\mathtt{write}}\ (y, 2)
$
which satisfies a stronger spec (explicitly ordering the two events),
but which \emph{can be weakened} to the spec of $P_1$ by forgetting the
ordering by means of strengthening the precondition and weakening the
postcondition, as customary in Hoare logic. Thus, our client proofs
\emph{can ignore} the internal thread-structure of component programs.

\gad{The rest of this section will have to be merged either with the
  Related Work, or the conclusions}

\end{comment}

\begin{wrapfigure}[10]{r}[0pt]{0.4\textwidth} 
%% \begin{figure}
%
\centering  
\begin{tabular}{l}
% 
\begin{minipage}[l]{5cm}
\small
\begin{alltt}
\num{1}  scan(): \(A {\times} A\)  \{
\num{2}    (cx, vx) \tbnd \act{read}(x);
\num{3}    (cy, _)  \tbnd \act{read}(y);
\num{4}    (_, tx)  \tbnd \act{read}(x);
\num{5}    if vx == tx 
\num{6}    then return (cx, cy)
\num{7}    else scan (); \}
\end{alltt} 
\end{minipage}
%
\end{tabular}
%
\caption{{\tt Scan} using version numbers.}
\label{fig:readpair}
%\end{figure}
\end{wrapfigure}

% \gad{\textbf seems to have no efect whatsoever on alttt}
% \gad{unified action-names not to leave spaces before arguments}





The substitution principle can be pushed further. In particular, we
can use a different snapshot algorithm, without modifying the proofs
of $P_1$, $P_2$, $P'_1$ or any other client, as long as \jywrite\ and
\jyscan\ satisfy the expected specs.
%
We have confirmed this property on the toy example given in
Figure~\ref{fig:readpair} (we present only \jyscan, as \jywrite\ is
trivial)~\cite{SergeyNB+ESOP15}. In this example, the snapshot
structure consists of pointers $x$ and $y$ storing tuples $(c_x, v_x)$
and $(c_y, v_y)$, respectively. $c_x$ and $c_y$ are the payload of $x$
and $y$, whereas $v_x$ and $v_y$ are version numbers, internal to the
structure. Writes to $x$ and $y$ increment the version number. {\tt
  Scan} reads $x$, $y$ and $x$ again, in succession, but avoids
snapshot inconsistency by restarting if the two version numbers of $x$
differ. The definition of $\stableorder$ used to satisfy the specs
equals the real time one, as no dynamic reordering is needed.

%\gad{Why are we not calling read-pair by it's name?}
%%
%\an{Well, because we're trying to make the parallel with the Jayanti's
%  code. By calling it {\tt scan}, the connection is immediately
%  made. Speaking of names, I'm more concerned that we name with {\tt
%    write} two different things. The primitive memory operation should
%  probably be renamed into $x := v$ or {\tt assign}.}
%%
%\gad{I understand. But I was refering to the algorithm and not the
%  methods.}

%% \gad{Add HERE!: Clients}

\gad{Add HERE!: Readpair's proof outline w/ this paper's spec and
  instance of getters/ stable order etc.}

\gad{Reviewers --and Ralf-- complain about its not quite obvious that
  our spec is useful. We need to make an effort here to convice them
  that our assertions are useful for clients.}

  %% \gad{The simple sequential {\tt two\_scan} example will help here,
  %%   as it showcases the need for ``sequential`` and the stable ideal
  %%   inclusion on histories. }

\gad{Aleks' comments on 04/10 wrt. why our spec is better than what
  linearizability provides, even if the spec looks a priory more
  complicated. \\ We could reduce the spec by (rule of consequence)
  weakening, to get the assertion of the sequential spec $r
  =\mathsf{eval}\ \hist $, but that would not be useful for client
  reasoning, as you would be losing the information needed to combine
  the client in parallel with others: take the example in Figure~2 --
  the client, we've (almost) proved in Coq-- and a linearization of
  the client on the left and the center {\tt l: write x 2; c: scan; l:
    write y 1 } and do a proof using the sequential spec. Now if you
  are given a different linearization, {\tt l: write x 2; l: write y
    1; c: scan () }, you need to re do the proof. Moreover, there is
  no way to scale up adding more threads, if you add the client on the
  right, r, none of the proofs youy have can be reused. \\ In our
  setting, and with our specs, client reasoning depends solely on the
  API for scan and write, regardless of the different linearizations
  of a program. We use this to our advantage while building up the
  proof for the client in Figure~2.  }

%% \gad{Structure the proof outlines for the clients in the same why we
%%   have done in coq}
%%
%% \paragraph{Clients} \gad{Let's see what we have right now:}  

%% \begin{enumerate}
%%   \item[i] two sequential scans, and a proof that whatever the second
%%     scanner returned is at least as new as the second, and moreover
%%     the two snapshots are consistent, ie, the snapshot in the first
%%     call is still a snapshot with the histories and orderings defined
%%     by the second call. Morever, the stable order 

%%   \item[ii] The composite client building up to Figure~2.
    
%%   \item[iii] The latter with hide? \gad{We might not have time to hack
%%     it in in Coq, but we could do it by hand, and then upload it}.
 
%% \end{enumerate}
