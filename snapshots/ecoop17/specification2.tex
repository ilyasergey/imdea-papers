\section{Specification}
\label{sc:formal}

%% Auxiliary state defs

\def\histx{\hist_\x}
\def\histy{\hist_\y}
\def\histp{\hist_p}

\newcommand{\sx}{S_\x}
\newcommand{\sy}{S_\y}
\newcommand{\spp}{S_p}
\newcommand{\sss}{S_s}
\newcommand{\wx}{W_\x}
\newcommand{\wy}{W_\y}
\newcommand{\wpp}{W_p}

%\newcommand{\admissible}{\mathsf{fine}}

% Names for writer/scanner-states
\def\toff{t_{\mathsf{off}}}

%% \newcommand{\wInit}{\mathsf{W_{Off}}}
%% \newcommand{\wWrite}{\mathsf{W_{New}}}
%% \newcommand{\wDirty}{\mathsf{W_{Fwd}}}
%% \newcommand{\wClean}{\mathsf{W_{Done}}}

%% \newcommand{\sOn}{\mathsf{S_{On}}}
%% \newcommand{\sOff}{\mathsf{S_{Off}}}


\subparagraph*{General considerations.}
%
For the purposes of specification and proof, we record a history of
the snapshot object as a set of entries of the form $t \mapsto (p,
v)$. The entry says that at time $t$ (a natural number), the value $v$
was written into the pointer $p$. We thus identify a write event with
a \emph{single} moment in time $t$, enabling the specs of
\jywrite\ and \jyscan\ to present the view that write events are
logically atomic.
%
Moreover, in the case of snapshots, we can ignore the scan events in
the histories. The latter do not modify the state in a way observable
by clients who can access the shared pointers only via interface
methods \jywrite\ and\ \jyscan.

We keep three auxiliary history variables. The history variables
$\histS$ and $\histO$ are local to the specified thread, and record
the \emph{terminated} write events carried out by the specified
thread, and that thread's interfering environment, respectively. We
refer to $\histS$ as the \emph{self}-history, and to $\histO$ as the \emph{other}-history~\cite{LeyWildN+POPL13,NanevskiLSD+ESOP14,OPLSS:Notes,SergeyNB+ESOP15}. The
role of $\histO$ is to enable the spec of \jywrite\ to situate the
performed write event within the larger context of past and ongoing
writes, and the spec of \jyscan\ to describe how it logically
reordered the writes that overlapped with it.
%
The third history variable $\histJ$ records the set of write events
that are in progress. These are events that have been initiated,
timestamped, and have executed their physical write to memory, but
have not terminated yet. It is an important component of our auxiliary
state design that when a write event terminates, it is moved from
$\histJ$ to the invoking thread's $\histS$, to indicate the
\emph{ownership} of the write by the invoking thread.
%
We name by $\hist$ the union $\histS \hunion \histO \hunion \histJ$,
which is the global history of the data structure. As common in
separation logic, the union is \emph{disjoint}, \ie, it is undefined
if the components contain duplicate timestamps. By the semantics of
our specs, $\hist$ is always defined, thus $\histS$, $\histO$ and
$\histJ$ never duplicate timestamps.

The real-time ordering of the timestamped events is the natural
numbers ordering on the timestamps. To track the \emph{logical}
ordering, we need further auxiliary notions.
%
The first is the auxiliary variable $\ordlist$, whose type is a
mathematical sequence. The sequence $\ordlist$ is a permutation of
timestamps from $\hist$ showing the logical ordering of the events in
$\hist$. We write $t_1 \tleq t_2$, and say that $t_1$ is logically
ordered before $t_2$, if $t_1$ appears before $t_2$ in $\ordlist$. The
sequence $\ordlist$ resides in joint state, and can be dynamically
modified by any thread. For example, the execution of the scanner may
reorder $\ordlist$, as shown in
Figure~\ref{fig:reorder:after}. Because $\ordlist$ is a sequence, the
order $\tleq$ is linear.

%
%In this sense, $\ordlist$ is the linkage in time for the snapshot
%algorithm, and it will be dynamically modified. 
%

Because sequence $\ordlist$ changes dynamically under interference, it
is not appropriate for specifications. Thus, our second auxiliary
notion is the \emph{partial} order $\stableorder$, a suborder of
$\tleq$ that is \emph{stable} in the following sense. It relates the
timestamps of events whose logical order has been determined,
\emph{and will not change in the future}. Thus $\stableorder$ can grow
over time, to add new relations between previously unrelated
timestamps, but cannot change the old relations.

%we require further a subset of ${\tleq}$, that is \emph{stable} in the
%following sense. This order relates the timestamps of events whose
%logical order has already been fixed, and will not change in the
%future. One can say that such events have been ``linearized''.  Being
%stable, $\stableorder$ can only grow over time to add new relations
%between previously unrelated timestamps, but cannot change the old
%relations.
%

To illustrate the distinction between the two orders, we refer to
Figure~\ref{fig:reorder:before}. There, $\ordlist$ represents the
linear order $5{-}0{-}2{-}3{-}1$, which changes in
Figure~\ref{fig:reorder:after} to $5{-}0{-}2{-}1{-}3$.
%
Since $1$ and $3$ exchange places, the stable order $\stableorder$
cannot initially relate the two. Thus, in
Figure~\ref{fig:reorder:before}, $\stableorder$ is represented by the
Hasse diagram $5{-}0{-}2{<}\!\begin{array}[c]{c}1\\ 3\end{array}$. In
Figure~\ref{fig:reorder:after}, the relation $1{-}3$ is added to this
partial order, making it the linear order $5{-}0{-}2{-}1{-}3$. Note
how the previous relations remain unchanged.

\newcommand{\scanned}[1]{\mathsf{scanned}\,#1}

The third auxiliary notion is the set $\scanned\stableorder$ of
timestamps. A write's timestamp is placed in $\scanned\stableorder$,
if that write has been observed by some scanner; that is, the written
value is returned in some snapshot, or has been rewritten by another
value that is returned in some snapshot. To illustrate, in the above
example, $\{5, 0, 2\} \subseteq \scanned\stableorder$.  Intuitively,
we want to model that after a write has been observed, the ordering of
the events logically preceding the write must be stabilized, and
moreover, must be a sequence. Thus,
%If $t \in \scanned\stableorder$, then all the events logically
%preceding $t$ should also satisfy the above property, thus
$\scanned\stableorder$ is a \emph{linearly ordered subset} of
$\stableorder$.\footnote{In terminology of linearizability, one may
  say that $\scanned\stableorder$ is the set of ``linearized''
  writes.}  The set $\scanned\stableorder$ can also be seen as
\emph{representing all the scans that have already been
  executed}. Such representation of scans allows us to avoid tracking
scan events directly in the history.

In the sequel, we concretize $\stableorder$ and
$\scanned\stableorder$ in terms of $\ordlist$ and other auxiliary
state. However, we keep the notions abstract in the method specs and
in client reasoning. This enables the use of different snapshot
algorithms, with the same specs, without invalidating the client
proofs. We also mention that $\ordlist$, $\stableorder$ and
$\scanned\stableorder$ can be encoded as user-level concepts in FCSL,
and require no new logic to be developed.

%%
%% \begin{equation}
%% \begin{array}{l}
%% \mathtt{write}\ (p : \mathtt{ptr}, n : \mathtt{int}) : 
%% \!\!\!\begin{array}[t]{l}
%% \{\hists = \hempty \wedge h \subseteq \histO
%%            \wedge \omega \subseteq {\stableorder}\}\\[3pt]
%% \{\exists t\ldot\, \hists = t \mapsto (p, n) \wedge h \subseteq \histO \wedge
%%   \omega \subseteq {\stableorder}\,{\wedge}\, h \subseteq H^{\hbox{}\sqsubsetneq_\ordlist t}\}
%% \end{array}\\
%% %
%% \mathtt{scan} : 
%% \!\!\!\begin{array}[t]{l}
%% \{\hists = \hempty \wedge\, h \subseteq \histO \wedge\,
%%           \omega \subseteq {\stableorder}\}\\[3pt]
%% \{\hists = \hempty \wedge\, h \subseteq \histO \wedge\, \omega \subseteq {\stableorder} \wedge\, \exists\, t\ldot\, %
%%            h \subseteq H^{\hbox{}\sqsubseteq_\ordlist t} \wedge\,
%%            \mathsf{chain}\ H^{\hbox{}\sqsubseteq_\ordlist t} \wedge\,
%%            r = \mathsf{eval}\ {H^{\hbox{}\sqsubseteq_\ordlist t}}\}
%% \end{array}
%% \end{array}
%% \label{eq:specs}
%% \end{equation}
%

\subparagraph*{Snapshot specification.}
% The following version saves a little more space
\begin{figure}
%
\centering
\[
\begin{array}{l}
\mathtt{write}\ (p : \mathtt{ptr}, n : \mathtt{int}) : 
\!\!\!\begin{array}[t]{l}
\{\hists = \hempty \wedge h \subseteq \histO
           \wedge \omega \subseteq {\jleq}\}\\[3pt]
\{\exists t\ldot\, \hists = t \mapsto (p, n) \wedge h \subseteq \histO \wedge
  \omega \subseteq {\jleq}\,{\wedge}\, h \subseteq H^{\hbox{}\sqsubsetneq_\ordlist t}\}
\end{array}\\
%
\mathtt{scan} : 
\!\!\!\begin{array}[t]{l}
\{\hists = \hempty \wedge\, h \subseteq \histO \wedge\,
          \omega \subseteq {\jleq}\}\\[3pt]
          \{\hists = \hempty \wedge\, h \subseteq \histO \wedge\,
            \omega \subseteq {\jleq} \wedge\, \exists\, t\ldot\, %
             h \subseteq H^{\hbox{}\sqsubseteq_\ordlist t} \wedge\,
             \mathsf{chain}\ H^{\hbox{}\sqsubseteq_\ordlist t} \wedge\,
             r = \mathsf{eval}\ {H^{\hbox{}\sqsubseteq_\ordlist t}}\}
       \end{array}
\end{array}
\]
\caption{\label{fig:specs} Specification of {\tt write} and {\tt scan}
  methods.}
\end{figure}
 
%
Figure~\ref{fig:specs} presents our specs for \jyscan~and
\jywrite. These are partial correctness specs that describe how
the methods change the state from the precondition (first braces) to
the postcondition (second braces), possibly influencing the value $r$
that the procedure returns. We use VDM-style notation with unprimed
variables for the state before, and primed variables for the state
after the method executes. We use Greek letters for state-dependent
values that can be mutated by the method, and Latin letters for
immutable variables.
%
The component $C$ is a state transition system (STS) that describes the
state space of the algorithm, i.e, the invariants on the auxiliary and
real state, and the transitions, i.e., the allowed atomic mutations of
the state. For now, we keep $C$ abstract, but will define it in
Sections~\ref{sc:auxiliaries} and~\ref{sc:implementation}.
%
We denote by $\ideal{\stableorder}{t}$ the downward-closed set of
timestamps
$\ideal{\stableorder}{t} = \{ s \mid s\ {\stableorder}\ t \}$. Let
$\sideal{\stableorder}{t} = (\ideal{\stableorder}{t})\setminus\{t\}$.

%downward-closed wrt.~$\stableorder$:
%\begin{align*}
%%\label{eq:ideal-order} 
%\alekideal{\stableorder}{t} \eqdef \{ s \mid s\ {\stableorder}\ t \}\qquad\mbox{and}\qquad
%%\label{eq:sideal-order} 
%\aleksideal{\stableorder}{t} \eqdef
 %  \{ s \mid s\ {\stableorder}\ t \wedge s \neq te\}
%\end{align*}

The spec for \jywrite\ says the following. The precondition starts
with the empty self history $\histS$, indicating that the procedure
has not made any writes. In the postcondition, a new write event $t
\mapsto (p, v)$ has been placed into $\histS'$. Thus, a call to
\jywrite\ wrote $v$ into pointer $p$. The timestamp $t$
is fresh, because $\histP$ does not contain duplicate timestamps.
%
%\histS$ and $\hist$ \ab{?} are always disjoint (by the
%semantics of FCSL triples). 
%
Moreover, the write appears as if it occurred atomically at time
$t$, thus capturing the logical atomicity of \jywrite.

The next conjunct, $\dom{\histO} \cup \scanned\stableorder \subseteq
\sideal{\stableorderP}{t}$, positions the write $t$ into the context
of other events. In particular, if $s \in \dom{\histO}$, \ie, if $s$
finished prior to invoking \jywrite, then $s$ is logically ordered
strictly before $t$. In other words, \jywrite\ cannot reorder prior
events that did not overlap with it. The definition of linearizability
contains a similar prohibition on reordering non-overlapping events,
but here, we capture it using a Hoare-style spec. For similar reasons,
we require that $\scanned\stableorder \subseteq
\sideal{\stableorderP}{t}$. As mentioned before,
$\scanned\stableorder$ represents all the scans that finished prior to
the call to \jywrite. Consequently, they do not overlap with
\jywrite\ in real time, and have to be logically ordered before $t$.

Notice what the spec of \jywrite\ \emph{does not prevent}. It is
possible that some event, say with a timestamp $s$, finishes in real
time before the call of \jywrite~at time $t$. Events $s$ and $t$ do
not overlap, and hence cannot be reordered; thus
$s\ {\stableorder}\ t$ always. However, the relationship of $s$ with
other events that ran concurrently with $s$, may be fixed only later,
thus supporting implementation of ``future-dependent'' nature, such as
Jayanti's.

%For example, in Figure~\ref{fig:reorder:before}, the event $3$ is
%ordered after $5$, $0$, and $2$, but its order wrt.~$1$ is determined
%only later.

%For example, if $s$ overlapped with some other event $s'$, it is
%possible for $s$ and $s'$ to not be immediately related in
%$\stableorder$. The relative ordering of $s$ and $s'$ may be
%determined after either of the events has terminated.  \ab{Can we give
%  an example here?}

In the case of \jyscan, we start and terminate with an empty $\histS$,
because \jyscan\ does not create any write events, and we do not track
scan events. However, when \jyscan\ returns the pair $r = (r_\x,
r_\y)$, we know that there exists a timestamp $t$ that describes when
the scan took place. This $t$ is the timestamp of the last write
preceding the call to \jyscan.

The postcondition says that $t$ is the moment in which the snapshot
was logically taken, by the conjunct $r =
\eval\ t\ \stableorderP\ \histP$.  Here, $\eval$ is a pure,
specification-level function that works as follows. First, it reorders
the entire real-time post-history $\histP$ according to logical
post-ordering $\stableorderP$. Then, it computes and returns the
values of $x$ and $y$ that would result from executing the write
events of such reordered history up to the timestamp $t$. For example,
if $t$ is the timestamp of event $1$ in
Figure~\ref{fig:reorder:after}, then $\eval\ t\ \stableorderP\ \histP$
would return $(2, 1)$. Hence, the conjunct says that
\jyscan\ performed a scan of $x$ and $y$, consistent with the ordering
$\stableorderP$, and returned the read values into $r$. The scan
appears as if it occurred atomically, immediately after time $t$, thus
capturing the atomicity of \jyscan.

%\ab{Can we connect with Figure 3b here?}

The next conjunct, $\dom{\hist} \subseteq \ideal{\stableorderP}{t}$,
says that the scanner returned a snapshot that is current, rather than
corresponding to an outdated scan. For example, referring to
Figure~\ref{fig:reorder}, if \jyscan\ is invoked after the events $2$
and $1$ have already executed, then \jyscan\ should not return the
pair $(5, 0)$ and have $t$ be the timestamp of the event $0$, because
that snapshot is outdated. Specifically, the conjunct says that the
write events from $\hist$ are ordered no later than $t$, similar to
the postcondition of \jywrite. However, while in \jywrite\ we
constrained the events from $\dom{\histO} \cup \scanned\stableorder$,
here we constrain the full global history $\hist = \histO \hunion
\histJ$. The addition of $\histJ$ shows that the scanner will observe
and order all of the write events that have been timestamped and
recorded in $\histJ$ (and thus, that have written their value to
memory), prior to the invocation of \jyscan.

Lastly, the conjunct $t \in \scanned\stableorderP$ explicitly says
that $t$ has been observed by the just finished call to scan.

Again, it is important what the spec does not prevent. It is possible
that the timestamp $t$ identified as the moment of the scan,
corresponds to a write that has been initiated, but has not yet terminated.
Despite being ongoing, $t$ is placed into $\scanned\stableorderP$
(\ie, $t$ is ``linearized''). Also, notice that the postcondition of
\jyscan\ actually specifies the ``linearization'' order of events that
are initiated by another method, namely \jywrite, thus supporting
implementations of ``non-local'' nature, such as Jayanti's.

We close the section with a brief discussion of how the specs are
used. Because $C$, $\stableorder$ and $\scanned$ are abstracted from
the clients, we need to provide an interface to work with them. The
interface consists of a number of properties showing how various
assertions interact, summarized in the statements below.

The first statement presents the invariants on the transitions of STS
$C$, often referred to as 2-state invariants.
%
Another way of working with such invariants is to include them in the
postcondition of every method.\footnote{In fact, this is what we
  currently do in our Coq files.} For simplicity, here we agglomerate
the properties, and use them implicitly in proofs as needed.
%
\begin{invariant}[Transition invariants]\label{inv:mono}
In any program respecting the transitions of $C$: 
\begin{enumerate}
\item\label{inv:mono:hist} $\hist \subseteq \histP$, $\histS \subseteq
  \histSP$, and $\histO \subseteq \histOP$.

\item \label{inv:mono:stable} $\stableorder \subseteq \stableorderP$
  and $\scanned{\stableorder} \subseteq \scanned{\stableorderP}$.

\item\label{inv:mono:ideal} For every $s \in \scanned{\stableorder}$,
  $\ideal{\stableorder}{s} = \ideal{\stableorderP}{s}$.
\end{enumerate}
\end{invariant}

Invariant~\ref{inv:mono}.\ref{inv:mono:hist} says that histories only
grow, but does not insist that $\histJ \subseteq \histJP$, as
timestamps can be removed from $\histJ$ and transferred to $\histS$.
%
Invariant~\ref{inv:mono}.\ref{inv:mono:stable} states that
$\stableorder$ is monotonic, and the same applies for
$\scanned{\stableorder}$. This is a fundamental stability requirement
for our system: no transition in the STS $C$ can change the relations
between write events in $\stableorder$ and, moreover, write events
which have been observed by the scanner--- and thus are in
$\scanned{\stableorder}$--- cannot be unobserved.
%
Invariant~\ref{inv:mono}.\ref{inv:mono:ideal} says that if a new event
is added to increase $\stableorder$ to $\stableorderP$, that event
appears logically later than any $s \in \scanned{\stableorder}$. In
other words, once events are observed by a scanner, and placed into
$\scanned{\stableorder}$ in a certain order, we cannot insert new
events among them to modify the past observation.

The second statement exposes the properties of $\stableorder$,
$\scanned$, and $\mathsf{eval}$ that are used for client reasoning:
%
\begin{invariant}[Relating $\scanned$ and snapshots]\label{lem:scanned}
The set $\scanned\, \stableorder$ satisfies the following properties:
%
\begin{enumerate}
 \item\label{lem:scanned:total} if $ t_1 \in
   \scanned{\stableorder}$ and $t_2 \in \scanned{\stableorder}$, then
   $ t_1 \mathrel{\stableorder} t_2 \vee t_2 \mathrel{\stableorder}
   t_1 $ (linearity).
  
 \item \label{lem:scanned:wkn} if $ t_2 \in
   \scanned{\stableorder}$ and $ t_1 \mathrel{\stableorder} t_2$, then
   $t_1 \in \scanned{\stableorder}$ (downward closure).

%% \item \label{lem:scan:prefix} if $t \in
%%   \scanned{\stableorder}$ and $ \ideal{\stableorder}{t} =
%%   \ideal{\stableorderP}{t}$, then $t \in \scanned{\stableorderP}$ (principality).
%% This one belongs in inv:monotonicity.
   
\item \label{lem:scanned:eval} if $t \in \scanned{\stableorder}$,
  $\hist \subseteq \histP$, $\stableorder \subseteq \stableorderP$,
  $\scanned{\stableorder} \subseteq \scanned{\stableorderP}$, and $
  \ideal{\stableorder}{t} = \ideal{\stableorderP}{t}$ then $ \eval\,
  t\, \stableorder\, \hist = \eval\, t\, \stableorderP\,
  \histP$. (snapshot preservation).
\end{enumerate}
\end{invariant}

The first two properties merely state that the subset
$\scanned{\stableorder}$ is totally-ordered
(\ref{lem:scanned}.\ref{lem:scanned:total}) and also downward closed
(\ref{lem:scanned}.\ref{lem:scanned:total}). The last property is the
most interesting: it entails that once a snapshot is observed by
\jyscan, its validity will not be compromised by future or ongoing
calls to \jywrite. Thus, snapshots returned from previous calls to
\jyscan~are still valid and observable in the future.

% \gad{Principality should be moved to Invariant~\ref{inv:monotonicity} Done! }
