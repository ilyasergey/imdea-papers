\documentclass[a4paper,USenglish]{darts}
\usepackage{microtype}%if unwanted, comment out or use option "draft"
\bibliographystyle{plainurl}% the recommended bibstyle

% ARTIFACT: Include the following input command here
% Commands for artifact descriptions
% Written by Camil Demetrescu and Erik Ernst
% April 8, 2014

% ARTIFACT: This entire file should be used as-is; it defines standard
% headings to be included in the artifact description, and it will be 'input'
% into the document file such that you can use the environments defined
% below

\newenvironment{scope}{\section{Scope}}{}
\newenvironment{content}{\section{Content}}{}
\newenvironment{getting}{\section{Getting the artifact} The artifact 
endorsed by the Artifact Evaluation Committee is available free of 
charge on the Dagstuhl Research Online Publication Server (DROPS).}{}
\newenvironment{platforms}{\section{Tested platforms}}{}
\newcommand{\license}[1]{{\section{License}#1}}
\newcommand{\mdsum}[1]{{\section{MD5 sum of the artifact}#1}}
\newcommand{\artifactsize}[1]{{\section{Size of the artifact}#1}}



% Author macros::begin
% %%%%%%%%%%%%%%%%%%%%%%%%%%%%%%%%%%%%%%%%%%%%%%%% ARTIFACT: Please
% use the same title as the corresponding ECOOP paper and append the
% text "(Artifact)" ARTIFACT: Add as a footnote the reference to the
% corresponding ECOOP paper
\title{Concurrent Data Structures Linked in Time
  (Artifact)\footnote{This artifact is a companion of the paper:
    Germ\'{a}n Andr\'{e}s Delbianco, Ilya Sergey, Aleksandar Nanevski
    and Anindya Banerjee, ``Concurrent Data Structures Linked in
    Time'', Proceedings of the 31st European Conference on
    Object-Oriented Programming (ECOOP 2017), June 18-23, 2017,
    Barcelona, Spain. This research is partially supported by EPSRC
    grant EP/P009271/1, the ERC consolidator grant
    Mathador--DLV--724464, and the US National Science Foundation
    (NSF). Any opinion, findings, and conclusions or recommendations
    expressed in the material are those of the authors and do not
    necessarily reflect the views of~NSF.}}

\titlerunning{Concurrent Data Structures Linked in Time (Artifact)}

% ARTIFACT: Authors may not be exactly the same as the ECOOP paper,
% e.g., you may want to include authors who contributed to the
% preparation of the artifact, but not to the ECOOP companion paper
%% Please provide for each author the \author and \affil macro, even
%% when authors have the same affiliation, i.e. for each author there
%% needs to be the \author and \affil macro

\author[1,2]{Germ\'{a}n Andr\'{e}s Delbianco} 
\author[3]{Ilya Sergey}
\author[1]{Aleksandar Nanevski}
\author[1]{Anindya Banerjee}

\affil[1]{IMDEA Software Institute, Madrid, Spain\\
  {\texttt{\{german.delbianco,
      aleks.nanevski,anindya.banerjee\}@imdea.org}}}

\affil[2]{Universidad Polit\'{e}cnica de Madrid, Spain}

\affil[3]{University College London, United Kingdom\\
  {\texttt{i.sergey@ucl.ac.uk}}}

\authorrunning{G.\,A. Delbianco and I. Sergey and A. Nanevski and %
  A. Banerjee}

\Copyright{Germ\'{a}n Andr\'{e}s Delbianco and Ilya Sergey and %
  Aleksandar Nanevski and Anindya Banerjee}

\keywords{Separation logic, Linearization Points, Concurrent
  snapshots, FCSL}

%% \subjclass{F.3.1 Specifying and Verifying and Reasoning about
%%   Programs, D.2.4 Software/Program Verification, F.1.2: Parallelism
%%   and concurrency, D.1.3 Concurrent Programming}

\subjclass{F.3.1, D.2.4, F.1.2: Parallelism and concurrency, D.1.3}


%% \author[1]{Germ\'{a}n Andr\'{e}s Delbianco} \author[2]{Alice
%% W. Source\footnote{Core artifact developer.}}  \affil[1]{Dummy
%% University Computing Laboratory, Address/City,
%% Country\\ \texttt{open@dummyuniversity.org}} \affil[2]{Department
%% of Informatics, Dummy College, Address/City,
%% Country\\ \texttt{access@dummycollege.org}}
%% \authorrunning{J.\,Q. Open and A.\,W. Source} %mandatory. First:
%% Use abbreviated first/middle names. Second (only in severe cases):
%% Use first author plus 'et. al.'

%Editor-only macros:: begin (do not touch as
%author)%%%%%%%%%%%%%%%%%%%%%%%%%%%%%%%%%%
\Volume{3}
\Issue{2}
\Article{1}

\RelatedConference{European Conference on Object-Oriented Programming
  (ECOOP 2017), June 18-23, 2017, Barcelona, Spain}
% Editor-only macros::end
% %%%%%%%%%%%%%%%%%%%%%%%%%%%%%%%%%%%%%%%%%%%%%%%

\begin{document}

\maketitle

\begin{abstract}
This artifact provides the full mechanization in FCSL of the
developments in the companion paper, ``Concurrent Data Structures
Linked in Time''. In the latter, we propose a new method, based on a
separation-style logic, for reasoning about concurrent objects with
such linearization points. We embrace the dynamic nature of
linearization points, and encode it as part of the data structure's
\emph{auxiliary state}, so that it can be dynamically modified in
place by auxiliary code, as needed when some appropriate run-time
event occurs. We illustrate the method by verifying (mechanically in
FCSL) an intricate optimal snapshot algorithm due to Jayanti, as well
as some clients.

FCSL is the first completely formalized framework for mechanized
verification of full functional correctness of fine-grained concurrent
programs. It is implemented as an embedded domain-specific language
(DSL) in the dependently-typed language of the Coq proof assistant,
and is powerful enough to reason about programming features such as
higher-order functions and local thread spawning. By incorporating a
uniform concurrency model, based on state-transition systems and
partial commutative monoids, FCSL makes it possible to build proofs
about concurrent libraries in a thread-local, compositional way, thus
facilitating scalability and reuse: libraries are verified just once,
and their specifications are used ubiquitously in client-side
reasoning.
\end{abstract}

% ARTIFACT: please stick to the structure of 7 sections provided below

% ARTIFACT: section on the scope of the artifact (what claims of the
% paper are intended to be backed by this artifact?)
\begin{scope}
  The scope of this artifact is to provide the full mechanization of
  the developments described in the companion paper, ``Concurrent Data
  Structures Linked in Time''. The mechanization consists in the
  development of the infrastructure to implement the \emph{``linking
    in Time''} technique as an FCSL library, the implementation of
  Jayanti's snapshot object in FCSL, together with the verification of
  the library methods and the clients presented in the
  paper. Moreover, for completeness sake, we have bundled this new
  development together with previous case studies in FCSL. Finally, if
  you are interested in finding out more details about the
  implementation of FCSL~\cite{NanevskiLSD+esop14, FCSL:Project} and
  several previous case
  studies~\cite{SergeyNB+ESOP15,SergeyNB+PLDI15,SergeyNBD+OOPSLA16},
  we encourage you to check the FCSL reference
  manual~\cite{FCSL:Manual}.

\end{scope}

% ARTIFACT: section on the contents of the artifact (code, data, etc.)
\begin{content}

  This artifact extends the previous release of FCSL with the Coq
  source files mechanizing the proofs of the developments presented in
  the associated research paper. For the sake of brevity, we list only
  the new contributions, and refer the reader to the FCSL user manual
  for details about previous case studies and the development of FCSL.

\begin{description}
  
\item [{\bf \large Folder:}\texttt{Examples/Relink}] The folder
  \texttt{Examples/Relink} contains the main case study of the {\it
    Linking in Time} technique for verifying concurrent data
  structures with non-fixed linearization points introduced in the
  paper. It consists of several files implementing the verification in
  FCSL of Jayanti's single writer/single scanner snapshot
  construction:

\begin{itemize}

 \item \texttt{jayanti\_snapshot.v} --- Preliminaries. Lemmas for
       reasoning about coloring patterns of histories, and other
       history invariants.

 \item \texttt{jayanti\_ghoststate.v} --- Implementation of the
 auxiliary state for the snapshot object, its transitions, invariants
 and associated invariant preservation proofs.

 \item \texttt{jayanti\_concurroid.v} --- Implementation of the FCSL
 concurrent resource ---a.k.a. {\it concurroid} for the snapshot.

 \item \texttt{jayanti\_actions.v} --- FCSL atomic actions for the
 snapshot object concurrent resource.

 \item \texttt{jayanti\_stability.v} --- Assertions used in the proof
 outline of the main methods and proofs of their stability under FCSL
 transitions.

 \item \texttt{jayanti\_library.v} --- Implementation and verification
 of the atomic commands (auxiliary code) and the snapshot library
 (i.e. scan and write procedures).

 \item \texttt{jayanti\_clients.v} --- Verification of the clients of
   the snapshot data structure, as presented in Section 4 of the
   paper.
\end{itemize}

In the sequel, we relate the different parts of the development with
their corresponding presentation in the paper:

The files \texttt{jayanti\_ghoststate.v} and
\texttt{jayanti\_concurroid.v} define the invariants described in
Section~5, together with the implementation of the concurrent resource
for the snapshot object. Moreover, in the later file, there is also
the implementation of the auxiliary code functions (a.k.a transitions
in FCSL jargon) from Section~6. The proof of the 2-state invariants
from Invariant~1, and those from Invariant~2, together with the
definitions of the main assertions (e.g.\ the {\it stable order}
$\Omega$) are carried out in \texttt{jayanti\_stability.v}. The
verification of the \texttt{write} and \texttt{scan} files, as
presented in Section~7, are carried our in the file
\texttt{jayanti\_library.v}. Finally, the clients in Section~4 are
implemented and verified in \texttt{jayanti\_clients.v}. We refer the
reader to the descriptions in the \texttt{README.md} file of the
project and the FCSL manual for further detail.

  \item [{\bf \large Folder:} \texttt{Core}] Two new libraries were
    added to the Core of existing FCSL libraries: a new library for
    {\it relinkable} histories, added to \texttt{histories.v} and an
    extended theory of algebraic surgery operations for lists, which
    were necessary to prove the properties of the {\it relink}
    operation. The latter is implemented in \texttt{seq\_extras.v}
\end{description}
\end{content} 

% ARTIFACT: section containing links to sites holding the
% latest version of the code/data, if any
\begin{getting}
% leave empty if the artifact is only available on the DROPS server.
% otherwise, provide links to the latest version of the artifact
% (e.g., on github)
  The latest version of this artifact is available also on the FCSL
  project website: \url{http://software.imdea.org/fcsl}.
\end{getting} 

% ARTIFACT: section specifying the platforms on which the artifact is
% known to work, including requirements beyond the operating system
% such as large amounts of memory or many processor cores
\begin{platforms}
  The artifact is known to work on any platform running Oracle
  VirtualBox version~5\footnote{\url{https://www.virtualbox.org/}}
  with at least 5~GB of free space on disk and at least 4~GB of free
  space in RAM. Alternatively, FCSL can be built from the sources
  available from the project's webpage. The latter can be done in any
  system capable of installing and executing
  Coq~v.8.5pl3\footnote{\url{https://coq.inria.fr/coq-85}}
  and~Ssreflect~1.6.\footnote{\url{https://math-comp.github.io/math-comp/}}
\end{platforms}

% ARTIFACT: section specifying the license under which the artifact is
% made available
\license{GPLv3 (\url{https://www.gnu.org/licenses/gpl})}

% ARTIFACT: section specifying the md5 sum of the artifact master file
% uploaded to the Dagstuhl Research Online Publication Server,
% enabling downloaders to check that the file is the expected version
% and suffered no damage during download.
\mdsum{95b0fc97f64958028ef3d1b31a4ffa4e}

% ARTIFACT: section specifying the size of the artifact master file
% uploaded to the Dagstuhl Research Online Publication Server
\artifactsize{2.75 GB}

\bibliography{references}
\end{document}
