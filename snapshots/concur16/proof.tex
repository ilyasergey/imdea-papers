
\def\tleqP{\mathrel{\leq_{\ordlist'}}}
\def\tleP{\mathrel{<_{\ordlist'}}}
\def\jleqP{\mathrel{\sqsubseteq_{\ordlist'}}}
%% \subparagraph*{Correctness of write and scan's specifications}%

We devote the rest of the section to illustrating the structure of our
proofs of {\tt write} and {\tt scan}, including aspects of
mechanization in FCSL/Coq. For lack of space, we cannot present the
full theory, but just focus on the structure and a few characteristic
properties.

The first stage of proof (and the mechanization) involves defining the
state space of the algorithm; that is, the definitions of the
auxiliaries, and the invariants that relate them. We presented
selected invariants in Section~\ref{sc:formal}. There is a number of
properties that need to be established under the state space, e.g.,
that it is closed under exchanging history entries between $\histS$
and $\histO$. The latter facilitates dynamic thread creation, because
it allows the reasoning to focus on individual threads in a large
parallel composition.
%
The second stage involves defining the auxiliary code, as in
Figure~\ref{fig:auxcode}, and proving a number of its properties. In
addition to correctness, the properties involve \emph{erasure} (i.e.,
auxiliary code only mutates auxiliary state, but not the real
one), \emph{framing} (i.e., if auxiliary code is executable in some
input heap and input history, then executing it in an extended heap
and history leads to the same results, and the extensions remain
invariant), preservation of invariants (from Section~\ref{sc:formal}),
and \emph{termination}. The correctness aspect of auxiliary code
involves proving that the code preserves the state-space invariants,
and that the precondition and postcondition hold. For example, the
correctness proof of $\aux{relink}$, relies on the following two
helper lemmas.

%The first lemma asserts that $\aux{inspect}$ correctly determines the
%``offending'' timestamp; the second lemma asserts that $\aux{push}$
%modifies the logical ordering in a way that makes the pair $(r_x,
%r_y)$ a valid snapshot.

\begin{lemma}[Correctness of $\aux{inspect}$]\label{lem:inspect}
If $t_x$, $t_y$ are timestamps for write events of $r_x$, $r_y$, then
$\aux{inspect}\ t_x\ t_y\ \ordlist\ \C$ correctly determines that
$(r_x, r_y)$ is a valid snapshot under ordering $\tleq$ and coloring
$C$, or otherwise, returns an ``offending'' timestamp. More formally,
if $\sss = \sOff\ \toff, \sx =\TT, \sy =\TT$, and for each
$p \in \{x,y\}$, $ t_p \hpts (p, r_p) \in \hist$ and $(t_p
= \aux{last\_green}\ \histp \vee \C(t_p) = \mathsf{yellow})$, the
following are exhaustive possibilities.
% if $\aux{inspect}\
%t_x\ t_y\ \ordlist\ \C = \mathsf{No}$ then $(r_x,r_y)$ is a valid
%snapshot. Else, if $\aux{inspect}\ t_x\ t_y\ \ordlist\ \C
%= \mathsf{Yes}\ p\ t'$ then $(r_x,r_y)$ is not a valid snapshot and
%moreover $t'$ is the offending timestamp in $\histp$. Furthermore, by
%exhausting the hypotheses about $t_x$ and $t_y$ we can infer that:
\begin{enumerate}
 \item If $t_x \tle t_y$ and $ \C(t_x) = {\sf yellow}$, then
      $\aux{inspect}\ t_x\ t_y\ \ordlist\ \C
      = \mathsf{No}$. Symmetrically for $t_y \tle t_x$.

 \item If $ t_x \tle t_y $, $ t_x = \aux{last\_green}\ \histx$, and
       $\forall t' \in \histx\ldot\ t_x \tle t'\,{\implies}\,t_y \tle
       t'$, then \\ $\aux{inspect}\ t_x\ t_y\ \ordlist\ \C
       = \mathsf{No}$. Symmetrically for $t_y \tle t_x$.

 \item If $ t_x <_{\ordlist} t_y $, $ t_x = \aux{last\_green}\ \histx
      $, $t' \in \histx$, and $t_x \tle t' \tle t_y$, then
      $\aux{inspect}\ t_x\ t_y\ \ordlist\ \C = \mathsf{Yes}\ \x\ t'$
      and $\C(t') = {\sf yellow}$.  Symmetrically for $t_y \tle t_x$.
\end{enumerate}
\end{lemma}

\begin{lemma}[Correctness of $\aux{push}$]\label{lem:push}
If $\sss = \sOff\ \toff, \sx =\TT, \sy =\TT$, and for $p \in \{x,y\}$,
we have $t_p \hpts (p, r_p) \in \hist$ and $(t_p
= \aux{last\_green}\ \histp \vee \C(t_p) = \mathsf{yellow})$, and
$\aux{inspect}\ t_x\ t_y\ \ordlist\ \C = {\sf Yes}\ p\ t'$, then:
\begin{enumerate}
 \item If $p = x$, then $(r_x, r_y)$ is a valid snapshot under
 $\leq_{\aux{push}\ t'\ t_y\ \ordlist}$. Symmetrically for $p = y$.
\item If $p = x$, then $\ordlist' = \aux{push}\ t'\ t_y\ \ordlist$ and 
$\C' = \C[t_x, t_y \hpts \mathsf{green}]$ satisfy the state space
 invariants
 (Propositions~\ref{inv:overlap}-\ref{inv:redzone}). Symmetrically for
 $p = y$.
\end{enumerate}
\end{lemma}

\begin{comment}
%
%One then has to prove that the state space remains valid under
%exhcanging of history entries between $\histS$ and $\histO$. Such
%exchanging is important because it govens the behavior of auxiliary
%state when threads fork and join, as follows. Let the parent thread be
%$e_1 \parallel e_2$, with \emph{self} history $\histS$
%and \emph{other} history $\histO$.  Upon forking, $\histS$ will be
%split non-deterministically as $\histS = \histS_1 \hunion \histS_2$,
%with $\histS_1$ assigned as a self history of $e_1$, and dually for
%$e_2$. On the other hand, the \emph{other} history of $e_1$ becomes
%$\histO \hunion \histS_2$, to reflect that upon forking $e_2$ becomes
%environment for $e_1$ (and symmetrically for $e_2$).

\subparagraph*{Correctness And Mechanization}%
We finish this section highlighting the most interesting details of
the mechanization of the snapshot data structure in FCSL. Our aim is to
give a general feel of the process of proving a concurrent data
structure correct in the logic, from the definition of the {\it
concurrent resource} for Jayanti's snapshot construction and its
invariants, to the final proof of correctness of the specification of
{\tt write} and {\tt scan} methods from Figure~\ref{fig:specs}. We
will focus mostly on properties of, and assertions about,
$\aux{relink}$ and we refer the reader to Appendix~\ref{sc:coq-code}
and the Coq mechanization~\cite{FCSL:Project} for further detail.

%% \gad{
%% What does an FCSL proof involve:
%% \begin{itemize}
%% \item building the concurrent resource, proving FCSl's meta-theoretic requirements.
%% \item Proving auxiliary code satisfies the invariants i.e. different
%% propositions given before.
%% \item Proving stability of the assertions
%% \item Proving the specs. for the methods
%% \end{itemize}
%% }

% First, build theory for re-linkable histories, auxiliary state,
% invariants and lemmas.

%\gad {Actually, shouldn't we be saying that the proof starts in the
%whitebord?}

The first stage of the mechanization involved building the meta-theory
for re-linkable histories/orders together with the specific auxiliary
state for the snapshot algorithm described above, its properties and
invariants, and several lemmas for reasoning about them. Second, we
implemented the auxiliary code and proved each of them correct,
wrt. to the invariants of the auxiliary code. This entails proving
that each of them satisfies Propositions~\ref{inv:color} --
~\ref{inv:redzone}, among others. We first highlight an intermediate
result about $\aux{inspect}$:

\begin{lemma}[Inspect]\label{lem:inspect}
$\aux{inspect}\ t_x\ t_y\ \ordlist\ \C$ determines if $(r_x, r_y)$ is
a valid snapshot, else finds the \emph{offending} timestamp. More
formally, if $\sss = \sOff\ \toff, \sx =\TT, \sy =\TT$, given $r_x$
and $r_y$ such that for $p \in \{x,y\}$ if $ t_p \hpts (p,
r_p) \in \hist$ and $(t_p = \aux{last\_green}\ \histp \vee \C(t_p)
= \mathsf{yellow}$), then if $\aux{inspect}\ t_x\ t_y\ \ordlist\ \C
= \mathsf{No}$ then $(r_x,r_y)$ is a valid snapshot. Else, if
$\aux{inspect}\ t_x\ t_y\ \ordlist\ \C = \mathsf{Yes}\ p\ t'$ then
$(r_x,r_y)$ is not a valid snapshot and moreover $t'$ is the offending
timestamp in $\histp$. Furthermore, by exhausting the hypotheses about
$t_x$ and $t_y$ we can infer that:
\begin{enumerate}
 \item If $t_x \tle t_y$ and $ \C(t_x) = {\sf yellow}$ then
      $\aux{inspect}\ t_x\ t_y\ \ordlist\ \C = \mathsf{No}$. {\it
      Idem} for $t_x \tle t_y$ and $ \C(t_x) = {\sf yellow})$.

 \item If $ t_x \tle t_y $ and $ t_x = \aux{last\_green}\ \histx $ and
      any timestamp $t'$ that appears after $t_x$ in $\histx $, \ie
      $t_x \tle t' $, also appears after $t_y$ in $\tle$, \ie $
      t_y \tle t' $, then $\aux{inspect}\ t_x\ t_y\ \ordlist\ \C
      = \mathsf{No}$. The same for $ t_y \tle t_x$\ldots, {\it mutatis
      mutandis}.

 \item If $ t_x <_{\ordlist} t_y $ and $ t_x
      = \aux{last\_green}\ \histx $ and there exists a timestamp $t'$
      in $\histx$ such that it appears in $\tleq$ in between $t_x$ and
      $t_y$, \ie $ t_x \tleq t' \tleq t_y $ then $\aux{inspect}\ t_x\
      t_y\ \ordlist\ \C = \mathsf{Yes}\ \x\ t'$. The same for $
      t_y \tle t_x$ \ldots, {\it mutatis mutandis}.
\end{enumerate}
\end{lemma}

%% \begin{lemma}[Inspect]\label{lem:inspect}
%% If $\sss = \sOff\ \toff, \sx =\TT, \sy =\TT$, given $r_x$ and $r_y$
%% such that for $p \in \{x,y\}$ if $ t_p \hpts (p, r_p) \in \hist$ and
%% $(t_p = \aux{last\_green}\ \histp \vee \C(t_p) = \mathsf{yellow}$),
%% then:
%% \begin{enumerate}
%% \item If $(t_x <_{\ordlist} t_y \wedge \C(t_x) = {\sf yellow}) \vee
%%       (t_y <_{\ordlist} t_x \wedge \C(t_y) = {\sf yellow})$ then
%%       $\aux{inspect}\ t_x\ t_y\ \ordlist\ \C = \mathsf{No}$.

%% \item If $ t_x <_{\ordlist} t_y $ and $ t_x
%%       = \aux{last\_green}\ \histx $ and $ \forall\
%%       t' \in \histx\ldot t_x <_{\ordlist} t' \implies t_y <_{\ordlist} t'
%%       $ then $\aux{inspect}\ t_x\ t_y\ \ordlist\ \C
%%       = \mathsf{No}$. Conversely, when $ t_y <_{\ordlist} t_x \ldots
%%       $, {\it mutatis mutandis}.

%% \item If $ t_x <_{\ordlist} t_y $ and $ t_x
%%       = \aux{last\_green}\ \histx $ and $ \exists\ t' \in \histx\ldot
%%       t_x <_{\ordlist} t' <_{\ordlist} t_y $ then $\aux{inspect}\ t_x\
%%       t_y\ \ordlist\ \C = \mathsf{Yes}\ \x\ t'$. Conversely, when $
%%       t_y <_{\ordlist} t_x \ldots $, {\it mutatis mutandis}.
%% \end{enumerate}
%% Moreover, if $\aux{inspect}\ t_x\ t_y\ \ordlist \C = \mathsf{No}$ then
%% $(r_x,r_y)$ is a valid snapshot.
%% \end{lemma}

The lemma above states that $\aux{inspect}$ asserts whether the
timestamps $t_x$ and $t_y$, pointing to the return values $r_x$ and
$r_y$, define a snapshot or not, and in the later case determines the
``offending'' timestamp that needs to be reordered. We consider cases
where $t_x <_{\ordlist} t_y$, as the others are defined conversely. In
the first case, if $t_x <_{\ordlist} t_y$ and $\C(t_x) = {\sf
yellow}$, it follows from $t_y$ being the last green or yellow
timestamp of $\histy$ and Proposition~\ref{inv:redzone} that any
$t' \in \histx$ s.t. $ t_x <_{\ordlist} t'$ will be {\sf red} and
hence $ t_y <_{\ordlist} t'$. Then, $\aux{inspect}\ t_x\
t_y\ \ordlist\ \C = \mathsf{No}$ and $(r_x,r_y)$ will be a valid
snapshot. In the second case, $t_x$ is the last green timestamp of
$\histx$ and, if there is a yellow timestamp $t' \in \histx$ and $ t_y
<_{\ordlist} t'$, then $\aux{inspect}\ t_x\ t_y\ \ordlist\ \C
= \mathsf{No}$, and clearly, $(r_x, r_y)$ is a valid snapshot. In the
last case, there exists a yellow timestamp $t'$ s.t. $ t' <_{\ordlist}
t_y$, so $t'$ invalidates $(r_x, r_y)$ as a snapshot and consequently
$\aux{inspect}\ t_x\ t_y\ \ordlist\ \C = {\sf Yes}\ \x\ t'$, denoting
that $t'$ has to be reordered in $\tle$.
%
%% We refer the reader to the {\tt Coq} mechanization for the formalities
%% of Lemma~\ref{lem:inspect}.
We next prove that, in the later case,
%
%% We next prove that when $\aux{inspect}\ t_x\ t_y\ \ordlist\ \C = {\sf
%% Yes}\ p\ t'$,
the result of $\aux{pushl}\ t'\ t_y\ \ordlist$ fixes the order $\tleq$
such that $(r_x, r_y)$ is a valid snapshot in $\leq_{(\aux{push}\ t'\
t_y\ \ordlist)}$:

\begin{lemma}[Push]\label{lem:push} $\aux{push}\ t'\ t_p$ fixes $\tleq$
so that $(r_x,t_y)$ is a correct snapshot in $\leq_{\aux{push}\ t'\
t_{p}\ \ordlist}$. More formally, if $\sss = \sOff\ \toff, \sx
=\TT, \sy =\TT$, given $r_x$ and $r_y$ such that for $p \in \{x,y\}$
if $ t_p \hpts (p, r_p) \in \hist$ and $(t_p
= \aux{last\_green}\ \histp \vee \C(t_p) = \mathsf{yellow}$) then:
\begin{enumerate}

\item If $\aux{inspect}\ t_x\ t_y\ \ordlist\ \C = {\sf Yes}\ x\ t'$
then $\aux{push}\ t'\ t_{y}\ \ordlist$ makes $(r_x, r_y)$ a valid
snapshot under $\leq_{\aux{push}\ t'\ t_{x}\ \ordlist}$. The same for
${\sf Yes}\ y\ t'$, {\it mutatis mutandis}.

 \item The auxiliary state as specified by the postcondition of
$\aux{relink}$ in Figure~\ref{fig:scanauxcode}, with $\ordlist'
= \aux{push}\ t'\ t_{p}\ \ordlist$ and $ \C' = \C[t_x,
t_y \hpts \mathsf{green}]$ satisfies the invariants of the concurrent
resource from Propositions~\ref{inv:overlap} to~\ref{inv:redzone}.
\end{enumerate}
\end{lemma}

%% \begin{lemma}[Push]\label{lem:push}
%% If $\sss = \sOff\ \toff, \sx =\TT, \sy =\TT$, given $r_x$ and $r_y$
%% such that for $p \in \{x,y\}$ if $ t_p \hpts (p, r_p) \in \hist$ and
%% $(t_p = \aux{last\_green}\ \histp \vee \C(t_p) = \mathsf{yellow}$)
%% then, if $\aux{inspect}\ t_x\ t_y\ \ordlist\ \C = {\sf Yes}\ p\ t'$
%% then:
%% \begin{enumerate}
%%  \item $\aux{push}\ t'\ t_{z}\ \ordlist$, with $z = y$ if $p = x$ and,
%%  conversely, $z = y$ if $p = y$, makes $(r_x, r_y)$ a valid snapshot
%%  under $\leq_{\aux{push}\ t'\ t_{z}\ \ordlist}$.
%% \item The resulting auxiliary state in the post-condition of relink
%%  above, with $\ordlist' = \aux{push}\ t'\ t_{z}\ \ordlist$ and $ \C'
%%  = \C[t_x, t_y \hpts \mathsf{green}]$ satisfies the invariants of the
%%  concurrent resource from Propositions~\ref{inv:overlap}
%%  to~\ref{inv:redzone}.
%% \end{enumerate}
%% \end{lemma}

We consider the first item: by Lemma~\ref{lem:inspect}, if
$\aux{inspect}\ t_x\ t_y\ \ordlist\ \C = {\sf Yes}\ x\ t'$ then $t'$
is the yellow timestamp of $\histx$. Then, all elements $t$ in the
chain $t' \tle t \tleq t_y$ belong to $\histy$, by virtue of $t_y$
being yellow or the last green in $\histy$ and
Invariants~\ref{inv:color} and~\ref{inv:redzone}. Thus, in $\ordlist'
= \aux{push}\ t'\ t_y\ \ordlist$, $\forall\ t' \in t_x \tleP t' \tleP
t_y\ldot. t' \in {\histy'}$, hence $(r_x,r_y)$ is a valid snapshot by
$\tleqP$. As for the second item, the proof of the preservation of
most invariants is quite straightforwards, the crux of the matter
being the observation that all $t$ in the chain $t' \tle t
\tle t_y$ are {\it overlapping} and, as a result, changing the linking
of $t'$ in the logical order does not contradict
Proposition~\ref{inv:overlap}.
\end{comment}

% Stability.
%We refer the reader to the {\tt Coq} source for the formal details
%about Lemmas~\ref{lem:inspect} and~\ref{lem:push}. 

We now have the following main correctness result.
%%
%% Final Correctness of the methods
\begin{theorem}[Correctness of {\tt write} and {\tt scan}]\label{thm:specs}
The implementations for {\tt scan} and {\tt write} in
Figure~\ref{fig:fcsl-snapshot} can be ascribed the specification given
in Figure~\ref{fig:specs}.
\end{theorem}
%
The theorem is proved following a customary style of Floyd-Hoare
symbolic evaluation using the inference rules of
FCSL~\cite{Nanevski-al:ESOP14}, a variant of Hoare logic. We have
mechanized this proof in FCSL/Coq, including all the properties from
the first two stages described above. The proof is subject to the
following helper stability properties that we have not yet mechanized,
but have proved manually, and their proofs are in
Appendix~\ref{sc:coq-code}.

%The stability proofs are carried out by induction on the number of
%steps that the environment takes.  The permitted steps of the
%environment are precisely the auxiliary procedures
%from~\ref{fig:fcsl-snapshot}, except that \emph{self} and \emph{other}
%variables are interchanged to capture that it is the environment doing
%the stepping. 

%% Out of the assertions presented in Figure~\ref{fig:specs}, we
%% highlight the proof of monotonicity of the {\it stable} dynamic order
%% $\jleq$:

\begin{lemma}[Stability of Assertions]\label{lem:menvs}
If by environment stepping $L$ evolves to $L'$, then:
\begin{enumerate}
 \item\label{lem:menvs:jleq} $ \jleq \subseteq\ \jleqP$ \ie the
 logical ordering is monotone.

 \item \label{lem:menvs:drs} $H^{\hbox{}\sqsubseteq_\ordlist
 t} \subseteq H^{\hbox{}\sqsubseteq_{\ordlist'} t}$. 
%
% \ie the restriction of $\hist$ to the elements in the $\jleq$ up to
% $t$ grows monotonically. 
Similarly for $\sqsubsetneq$.
 
 \item \label{lem:menvs:chain} If $\mathsf{chain}\
 H^{\hbox{}\sqsubseteq_\ordlist t}$ then $\mathsf{chain}\
 H^{\hbox{}\sqsubseteq_{\ordlist'} t}$.

 \item \label{lem:menvs:eval} If $\mathsf{eval}\
 H^{\hbox{}\sqsubseteq_\ordlist t}$ then $\mathsf{eval}\
 H^{\hbox{}\sqsubseteq_{\ordlist'} t}$.
\end{enumerate}
\end{lemma}
%
%The proof of Lemma~\ref{lem:menvs}, items~\ref{lem:menvs:jleq}
%and~\ref{lem:menvs:drs} have been mechanized. The (manual) proofs of
%items \ref{lem:menvs:chain} and~\ref{lem:menvs:eval} are given in the
%Appendix~\ref{sc:coq-code}, but are not yet mechanized. 
%
%Finally, we have the main correctness result, mechanized relative to
%the proofs of Lemma~\ref{lem:menvs}.\ref{lem:menvs:chain}
%and~\ref{lem:menvs}.\ref{lem:menvs:eval}. 

