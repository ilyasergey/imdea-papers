\section{Formal proof structures}
\label{sc:formal}

\def\histx{\hist_\x}
\def\histy{\hist_\y}
\def\histp{\hist_p}
\def\ordlist{L}
\renewcommand{\tleq}{\mathrel{\leq_\ordlist}}
\renewcommand{\tle}{\mathrel{<_\ordlist}}
\renewcommand{\jleq}{\mathrel{\sqsubseteq_\ordlist}}
\newcommand{\E}{E}
\newcommand{\C}{C}
\newcommand{\sx}{S_\x}
\newcommand{\sy}{S_\y}
\newcommand{\spp}{S_p}
\newcommand{\sss}{S_s}
\newcommand{\wx}{W_\x}
\newcommand{\wy}{W_\y}
\newcommand{\wpp}{W_p}
\newcommand{\admissible}{\mathsf{fine}}

% Names for writer/scanner-states
\def\toff{t_{\mathsf{off}}}
\newcommand{\wInit}{\mathsf{W_{off}}}
\newcommand{\wWrite}{\mathsf{New}}
\newcommand{\wDirty}{\mathsf{NeedsFwd}}
\newcommand{\wClean}{\mathsf{Done}}
\newcommand{\sOn}{\mathsf{S_{on}}}
\newcommand{\sOff}{\mathsf{S_{off}}}

%% \gad{ Leave the beginning of the section as it is up to the auxiliary
%%   state, and then introduce Jayanti's FPs. Then follow to describe how
%%   we are implementing them and how they relate to
%%   Invariants~\ref{inv:forward},~\ref{inv:scanner}
%%   and~\ref{inv:redzone}.}

\subparagraph*{Specification.}
%
For the purposes of specification and proof, we record a history of
the snapshot object as a set of entries of the form $t \mapsto (p,
v)$. The entry says that at time $t$ (a natural number), the value $v$
was written into the pointer $p$. Notice that we identify a write
event with a \emph{single} moment in time $t$, in contrast to
linearizability which keeps a time interval. As we shall see,
internally, our proofs record intervals too, but we hide the interval
end-points from the clients. Also, unlike linearizability, we do not
need to (though we could) keep track of scan events. Scans do not
modify the object state in ways observable by clients, and may thus be
omitted from specifications. These two points illustrate the
improvement in information hiding over linearizability that we
mentioned in Section~\ref{sc:intro}.

%Unlike in the linearizability proof of the snapshot structure, we keep
%only write events in the history. The scan events do not modify the
%abstract state of the structure observable by clients, and can thus be
%omitted to simplify the reasoning process.

The auxiliary variables $\histS$ and $\histO$, which are local to the
specification, record the \emph{finished} write events of each
procedure, and of the interfering environment, respectively.  We also
have a local variable $\histJ$, which records the set of write events
\emph{in progress}. When a call to {\tt write} is initiated, an
auxiliary code places a write entry into $\histJ$; when a call
finishes, the entry is moved from $\histJ$ to $\histS$.
%
%When one thread changes its own $\histS$, this immediately changes
%every other thread's $\histO$---a discipline provided automatically by
%FCSL. 
%
We name by $\hist$ the union $\histS \hunion \histO \hunion \histJ$,
which is the history of the overall snapshot data structure. This is a
\emph{disjoint union}, enforcing that $\histS$, $\histO$ and $\histJ$
do not contain common timestamps.

The timestamps in $\hist$ determine the \emph{real-time} ordering of
write events. We record the \emph{logical} ordering in another
auxiliary variable $\ordlist$, whose type is list (i.e., mathematical
sequence). This list stores the permutation of timestamps from
$\hist$, and we write $t_1 \tleq t_2$ if $t_1$ appears before $t_2$ in
$\ordlist$. The list $\ordlist$ is the ``linking-in-time'' for the
snapshot algorithm, and can be dynamically modified by the threads.
%
%In this sense, $\ordlist$ is the linkage in time for the snapshot
%algorithm, and it will be dynamically modified. 
%
The order $\tleq$ is total (i.e., a chain), but as $\tleq$ is dynamic,
we introduce further notation, and name by $\jleq$ a \emph{partial
  suborder} of $\tleq$ that is \emph{stable}, i.e., it can only grow
over time to add new relations, but cannot change the old ones. 
%
For example, in Figure~\ref{fig:reorder}, the initial value of $\tleq$ is
(in lattice notation) $5{-}0{-}2{-}3{-}1$, which changes to $5{-}0{-}2{-}1{-}3$. Stable
order $\jleq$ is the lattice
$5{-}0{-}2{<}\!\!\!\begin{array}[c]{c}1\\ 3\end{array}\!\!\!$: $1$ and $3$ are unrelated.

We will formally define $\jleq$ later, but we can already use it to
give Hoare triples for {\tt write} and {\tt scan}. In fact, it will be
important for client reasoning (Section~\ref{sec:clients}) to keep the
definition of $\jleq$ \emph{hidden}, so that different snapshot
algorithms can provide different definitions, without influencing the
clients.
%
% The following version saves a little more space
\begin{figure}
%
\centering
\[
\begin{array}{l}
\mathtt{write}\ (p : \mathtt{ptr}, n : \mathtt{int}) : 
\!\!\!\begin{array}[t]{l}
\{\hists = \hempty \wedge h \subseteq \histO
           \wedge \omega \subseteq {\jleq}\}\\[3pt]
\{\exists t\ldot\, \hists = t \mapsto (p, n) \wedge h \subseteq \histO \wedge
  \omega \subseteq {\jleq}\,{\wedge}\, h \subseteq H^{\hbox{}\sqsubsetneq_\ordlist t}\}
\end{array}\\
%
\mathtt{scan} : 
\!\!\!\begin{array}[t]{l}
\{\hists = \hempty \wedge\, h \subseteq \histO \wedge\,
          \omega \subseteq {\jleq}\}\\[3pt]
          \{\hists = \hempty \wedge\, h \subseteq \histO \wedge\,
            \omega \subseteq {\jleq} \wedge\, \exists\, t\ldot\, %
             h \subseteq H^{\hbox{}\sqsubseteq_\ordlist t} \wedge\,
             \mathsf{chain}\ H^{\hbox{}\sqsubseteq_\ordlist t} \wedge\,
             r = \mathsf{eval}\ {H^{\hbox{}\sqsubseteq_\ordlist t}}\}
       \end{array}
\end{array}
\]
\caption{\label{fig:specs} Specification of {\tt write} and {\tt scan}
  methods.}
\end{figure}

%% \begin{equation}
%% \begin{array}{l}
%% \mathtt{write}\ (p : \mathtt{ptr}, n : \mathtt{int}) : 
%% \!\!\!\begin{array}[t]{l}
%% \{\hists = \hempty \wedge h \subseteq \histO
%%            \wedge \omega \subseteq {\jleq}\}\\[3pt]
%% \{\exists t\ldot\, \hists = t \mapsto (p, n) \wedge h \subseteq \histO \wedge
%%   \omega \subseteq {\jleq}\,{\wedge}\, h \subseteq H^{\hbox{}\sqsubsetneq_\ordlist t}\}
%% \end{array}\\
%% %
%% \mathtt{scan} : 
%% \!\!\!\begin{array}[t]{l}
%% \{\hists = \hempty \wedge\, h \subseteq \histO \wedge\,
%%           \omega \subseteq {\jleq}\}\\[3pt]
%% \{\hists = \hempty \wedge\, h \subseteq \histO \wedge\, \omega \subseteq {\jleq} \wedge\, \exists\, t\ldot\, %
%%            h \subseteq H^{\hbox{}\sqsubseteq_\ordlist t} \wedge\,
%%            \mathsf{chain}\ H^{\hbox{}\sqsubseteq_\ordlist t} \wedge\, r = \mathsf{eval}\ {H^{\hbox{}\sqsubseteq_\ordlist t}}\}
%% \end{array}
%% \end{array}
%% \label{eq:specs}
%% \end{equation}
%
Figure~\ref{fig:specs} presents our specifications for {\tt write} and
{\tt scan}. These are partial correctness Hoare triples, describing
how the program changes the state from the precondition (first braces)
to the postcondition (second braces), possibly influencing the return
result $r$. %, in the postcondition of {\tt scan}.
%
The expression $H^{\hbox{}\jleq t}$ selects the entries in the history
$H$ \emph{up to and including} the timestamp $t$, according to the
ordering $\jleq$. Likewise, $H^{\hbox{}\sqsubsetneq_\ordlist t}$ is
defined to exclude $t$.
%
We describe the specifications next, before commenting on their
relationship to linearizability.

The specification for {\tt write} says the following. We start with
the empty self history indicating that the procedure did not yet make
any writes. We use the variable $h$ to name an arbitrary subset of the
initial value of $\histO$ (hence, of \emph{completed} write events of
\emph{all other threads}), and the variable $\omega$ for a subset of
the initial ordering $\jleq$. The postcondition says that when {\tt
  write} terminates, one write event $t \mapsto (p, v)$ has finished,
and is hence placed into $\histS$.  $\histO$ and $\jleq$ may have
changed from the previous values, due to interference, but they still
include $h$ and $\omega$ as subsets. That is, $\histO$ could only
grow, because other threads could create new write events, and
similarly for $\jleq$. Lastly, the conjunct $h \subseteq
H^{\hbox{}\sqsubsetneq_\ordlist t}$ says that the write events that
have been completed before the call to {\tt write} (and are hence
stored in $h$) will be \emph{ordered before} the event $t$ in $\jleq$,
no matter how $L$ is permuted by interfering threads. In other words,
we cannot logically reorder past events, which is a basic soundness
requirement. But notice how stating it requires that $\histO$ is in
scope, thus directly relating {\tt write} to the interfering threads.

The specification for {\tt scan} starts with the same precondition. In
the postcondition, it says that $\histS$ is empty, because {\tt scan}
itself does not create write events. However, by the time {\tt scan}
returns the pair $r$ as a snapshot, we know that there exists a
timestamp $t$ in the collective history $H$ at which the snapshot
\emph{appears} to be taken. First, the conjunct $h \subseteq
H^{\hbox{}\jleq t}$ indicates that the snapshot is taken \emph{after}
the call to {\tt scan}, because the finished events stored in $h$ are
ordered before (or at best, at) timestamp $t$.
%
%% Second, the set $H^{\hbox{}\sqsubseteq_\ordlist t}$ is
%% a subchain in $\sqsubseteq_\ordlist$ (technically, all its entries are
%% ordered in $\sqsubseteq_\ordlist$).
%
%
Second, the timestamps from $H^{\hbox{}\jleq t}$ form a sub-chain in
$\jleq$,  % (\ie, $\forall\, a,b \in H^{\hbox{}\sqsubseteq_\ordlist
%  t}\ldot\ a \sqsubseteq_{\ordlist} b \vee b \sqsubseteq_{\ordlist}
%a$.
%
encoding how the writes progressed logically in time up to the
snapshot moment $t$. Moreover, since $\jleq$ is stable, this view
of the time will not change in the future to invalidate the result $r$
as a snapshot. Finally, if the chain of writes is evaluated in the
order given by $\jleq$, it produces $r$.

%\gad{Actually, proving that chain is stable entails proving something
%  stronger: that the sub-chain $H^{\hbox{}\sqsubseteq_\ordlist t}$ is
%  ``maximal'' or complete i.e. that no more entries are going to be
%  added to that sub-history. This is (will be) proven by the fact that
%  in the atomic moment after we have executed re-link,
%  $H^{\hbox{}\sqsubseteq_\ordlist t}$ defines a green prefix from the
%  very beginning up to the timestamp t, and such prefixes are stable.}

Notice how these specifications directly capture what linearizability
would have given us: in both procedures we get a time moment $t$ at
which the procedure executed \emph{logically instantaneously}, as part
of a larger \emph{sequence} $H$. Additionally, {\tt write} performed a
write event at $t$, and {\tt scan} returned a snapshot consistent with
scanning at $t$. In the case of linearizability, the last property
requires showing a simulation between the concurrent implementations
of {\tt write} and {\tt scan}, and some sequential equivalents. If we
then want to verify Hoare triples of clients, we need to establish
Hoare triples for the sequential equivalents. In our method, the
process is streamlined by immediately giving Hoare triples for the
concurrent implementations, avoiding intermediary sequential
equivalents and simulation proofs.

\subparagraph*{Internal auxiliary state.} 
The proofs of {\tt write} and {\tt scan} require further auxiliary
state, which does not feature in the specifications, and is hence
hidden from clients.

First, we track the point of execution in which {\tt write} and {\tt
  scan} are, but instead of line numbers in the code, we use
datatypes, to encode extra information in the constructors.
%
For example, the scanner's state is a triple $(\sss, \sx,
\sy)$. $\sss$ is drawn from $\{ \sOn, \sOff\, t\}$. If $\sOn$, then
the scanner is in lines 8--10 in Figure~\ref{fig:jayanti-snapshot}. If
$\sOff\ t$, the the scanner reached line 11 at time $t$, and is now in
11--15.  $\sx$ is a boolean, set when the scanner clears $\fx$ in
line~9, and reset upon scanner's termination (dually for $\sy$ and
$\fy$).
%
Writers' state for $x$ is tracked by the auxiliary $\wx$ (dually,
$\wy$). These are drawn from $\{\wInit, \wWrite\ t, \wDirty\ t,
\wClean\ t\}$, where $t$ marks the beginning of the write. If
$\wInit$, then no write is in progress. If $\wWrite\ t$, then the
writer is in line~2. If $\wDirty\ t$, then $b$ has been set in line~3,
triggering forwarding. If $\wClean\ t$, the writer is free to exit.

Second, like in linearizability, we record the ending times of
events. We use the auxiliary variable $\E$, which is a function
mapping a timestamp $t$ of a \emph{finished} write event from $\hist$,
to a timestamp identifying the event's \emph{ending time}. Events
$t_1$ and $t_2$ which are non-overlapping (i.e., $E(t_1) < t_2$ or
$E(t_2) < t_1$), will never be reordered, thus {\tt write} and {\tt
  scan} cannot modify the past history.

%Formally, the
%following is an invariant of the algorithm.
%\vspace{-2mm}

\begin{proposition}\label{inv:overlap}%
%The logical order $\tle$ preserves the real time order of {\it
%  non-overlapping} past write events. Formally, 
$\forall t_1 \in \dom{E}, t_2 \in \dom{\hist}$, if $E(t_1) < t_2$
  then $t_1 \tle t_2$.
\end{proposition}

Third, we track the rearrangement status of write events wrt.~an
ongoing \emph{active} scan, by \emph{colors}. A scan is \emph{active}
if it has cleared the forwarding pointers in line~9, and is ready to
read $x$ and $y$. The auxiliary variable $\C$ is a function mapping
each timestamp in $\hist$ to a color, as follows.
%
%% It is always relative to the ongoing scan, or the last finished one,
%% if no scans are active.
%
\begin{itemize}
\item {\sf Green} timestamps identify events whose position in the
  logical order is fixed in the following sense: if $\C(t_1) =
  \mathsf{green}$ and $t_1 \tle t_2$, then $t_1 <_{\ordlist'} t_2$ for
  every $\ordlist'$ to which $\ordlist$ may step by auxiliary code
  execution (Section~\ref{sc:implementation}). For example, since we
  only reorder overlapping events, and only the scanner reorders
  events, every event that finished before the active scan started
  will be green. Also, a green timestamp never changes the color.
\item {\sf Yellow} timestamps identify events whose order is not fixed
  yet (i.e., they are not linearized), but which \emph{may} be
  manipulated by the ongoing active scan, as follows.  The scan can
  \emph{push} a yellow timestamp in logical time, \emph{past} another
  green or yellow timestamp, but not past a red one. \emph{This is the
    only way the logical ordering can be modified.}

\item {\sf Red} timestamps identify events whose order is not fixed
  yet, but which will \emph{not} be manipulated by the active scan,
  and are left for the next one.
%
%% They
%% will be put after the green ones.
\end{itemize}
%

There is a number of invariants that relate colors and timestamps. For
brevity, we only list the most important ones, and defer to the Coq
code for the others~\cite{CoqFiles}. In the sequel, we use
$\histp$ to denote the set of writes into the pointer $p$ that appear
in the history $\hist$.

% Color invariant
\begin{proposition}[Color Invariant]\label{inv:color}%
The colors of $\histp$ are described by the regular expression $\GYR$:
there is a non-empty prefix of green timestamps, followed by \emph{at
  most} one yellow, and arbitrary number of reds.
\end{proposition}

This proposition identifies, by the yellow color, the write event for
$p$ that is the unique candidate for reordering by the ongoing active
scan. Moreover, all the writes into $p$ prior to the yellow one, will
have already been painted green (i.e., fixed in time, linearized),
whether they overlapped with the scanner or not.

\begin{comment}
\begin{proposition}[First Forwarding Principle]\label{inv:fwd1}%
If the scanner state is $\sss = \sOff\ \toff, \spp = \TT$, then
$\forall\ t \in \histp\ldot\ t \leq \E(t)< \toff \implies \C(t)=
\mathsf{green}$.
\end{proposition}
%
The above is our mathematical formulation approximating the
equally-named, but only informally stated property of
Jayanti~\cite{Jayanti:STOC05}, which says that if {\tt scan} misses
the value of a concurrent write in line~10 of
Figure~\ref{fig:jayanti-snapshot}, but the write finishes before the
scanner goes through line~11 (the scan's linearization point), then
the scanner will catch the value in the forwarding pointer in
line~12. In our setting, this is captured by saying that a write event
that finished before $\toff$, is green. The write event may have been
yellow in the past, but the act of forwarding will paint it green. We
will see in Section~\ref{sc:implementation} that auxiliary code for
forwarding will do just that.

%It is easy to see how this proposition corresponds to Jayanti's First
%Forwarding Priciple: in our seeting a write event which is visible,
%ergo never going to be missed is {\sf green}. Then the Proposition
%above says that any write event that finished strictly before $\toff$
%will be green. This means that in the case that a write event was
%committed to memory after the scanner read in line 10-- in which case
%it was originally painted {\sf yellow} ---it must have been forwarded,
%as forwarding paints entries from yellow to green. As for the Second
%Forwarding Principle, we capture it through the two following
%propositions:

Conversely, Jayanti's Second Principle states that any non-$\bot$
value read in line~12 comes from a write event that is concurrent with
the scan. Moreover, any later write to the same pointer will finish
after the linearization point of {\tt scan} in line 11, and hence will
be missed by the scanner. While we do not formally capture exactly
this property, we approximate it with the following two invariants,
which are sufficient.

\end{comment}

%% %% Key Inv for fp
\begin{proposition}[Green/Yellow Forwarded Values]\item\label{inv:readFP} 
If $\sss= \sOff\ \toff$ and $\spp = \TT$ (i.e., scanner is in lines
12--14), and a value $v \neq \bot$ has been forwarded to $p$ (i.e.,
$\mathit{fp} = v$), then the event of writing $v$ into $p$ is in the
history, i.e., exists $t$ such that $t \mapsto (p, v) \in
\histp$. Moreover, $t$ is the last green, or the yellow timestamp in
$\histp$.
%(It can then be
%derived that $t$ must be overlapping with the scan).
\end{proposition}

This proposition restricts the set of events that could have forwarded
a value to the scanner, to only two: the event with the (unique)
yellow timestamp, or the one with the last green timestamp. By
Proposition~\ref{inv:color}, the two timestamps are consecutive in
$\histp$.
 
%% RedZone invariant
\begin{proposition}[Red-Zone Invariant]\label{inv:redzone}%
If $\sss = \sOff\ \toff, \sx = \TT, \sy = \TT$, then $\hist$ satisfies
the $\RZ$ pattern. Moreover, for every $t' \in \mathsf{dom}(H)$:
%
\begin{itemize}
\item $\C(t') = \mathsf{green} \implies t' \leq \toff$
\item $\C(t') = \mathsf{yellow} \implies t' \leq \toff \leq \E (t')$
\item $\C(t') = \mathsf{red} \implies \toff < t'$  
\end{itemize}
\end{proposition}

This proposition restricts the global history $\hist$ (not the
projections $\histp$). First, the red events in $H$ do not mix with
the green and yellow ones. Thus, when a scanner pushes a yellow event
past a green or yellow event, it will not ``jump over'' any reds, thus
keeping the invariant that ordering on reds is not manipulated by the
scan. Second, the proposition relates the colors to the time $\toff$
at which the scanner was turned off (in line~11,
Figure~\ref{fig:jayanti-snapshot}). This moment is important for the
algorithm; e.g., it is the linearization point for {\tt scan} in
Jayanti's proof~\cite{Jayanti:STOC05}.
%
In our proof, the inequalities wrt.~$\toff$ are used to prove that the
events reordered by the scanner \emph{do} overlap, as per
Proposition~\ref{inv:overlap}.

\begin{comment}
Proposition~\ref{inv:readFP} states that any non-$\bot$ value read
from the forwarding pointer $\mathit{fp}$ at line 12 might have been
written to $p$ after the scanner cleared the forwarding pointers, and
and hence, be newer than the values read in line 10. This captures the
essence of the first part of Jayanti's Second forwarding principle.
%
Proposition~\ref{inv:redzone} states that when the scanner is in
between lines 12-15 in Figure~\ref{fig:jayanti-snapshot}, then $\hist$
can be partitioned into a non-empty prefix with yellow and/or green
timestamps, and a reds-only suffix. Moreover, if a new write event $t$
occurs at this point, with $ \toff < t <= \E (t)$, it will be colored
red and hence ignored by the current scanner. This later fact captures
the last part of the second Forwarding Principle. Later on, this
proposition will be crucial into proving that relink does not affect
the order of write events that are meant to missed, by keeping them in
the red suffix, and also that those being reordered are overlapping.

\gad{Explain better? The second part of this invariant might also be
  connected to Jayanti's ``Correctness'' Theorem, Lemma 5 in his
  Appendix. I'm not sure however if I need to express its connection
  here.}
\end{comment}

We can now define the stable logical order $\jleq$, which relies on
the internal auxiliary state.
%
\begin{definition}[Logical Order]\label{def-jleq}
%\begin{equation*}
$t_1 \jleq t_2 \eqdef t_1 = t_2 \vee E\,(t_1) < t_2 \vee (t_1 \tle t_2 \wedge \C(t_1) = \mathsf{green})$.
%\end{equation*}
\end{definition}
The proposition $t_1 \jleq t_2$ is stable (i.e., invariant under
interference), since threads do not change the ending times $E$, the
color of green events, or the order of green events in $\tle$.

%Stability holds because the first two disjuncts do not relate mutable
%values, and the third is stable because green timestamps are not
%reordered nor recolored.
%\gad{Explain stability here}
%
%\gad{Add some comment about Forwarding Principles back in, either here
%  or in Related Work.}
%
%\gad{Add instantiation of the ideal $H^{\hbox{}\sqsubseteq_\ordlist t}
%  $ w.r.t Figures~\ref{fig:weird:exec} and~\ref{fig:reorder}}

%% %
%% \gad{I've changed the definition so that it is closer to its
%%   implementation in the Coq development.}

%% \begin{equation}
%%  t_1 \jleq t_2  = (t_1 = t_2) \vee (E\,(t_1) < t_2) \vee
%%  (t_1\tle t_2 \wedge \C(t_1) = \mathsf{green}) \label{def-jleq}
%% \end{equation}
%

%% \gad{Introduce Jayanti's forwarding principles and how we implement
%%   them with the colors.  Use this to motivate the invariants in
%%   particular~\ref{inv:wrtP},~\ref{inv:readFP}, and~\ref{inv:redzone},
%%   which are the ones that correspond closely to the forwarding
%%   principles.}

%% \subparagraph*{Forwarding Principles}
%% Jayanti introduces two {\it Forwarding
%%   Principles}~\cite{Jayanti:STOC05} that guide the design of his
%% snapshot algorithm and provide an intuition for the correctness of the
%% proof of their linearizability. In essence, the First Principle says
%% that, in Figure~\ref{fig:jayanti-snapshot} if {\tt scan} misses the
%% value of a concurrent write when reads in line~10, if the said writer
%% finishes before line~11 (the scanner method's linearization point),
%% the scanner will catch it by reading the forwarding pointer in
%% line~12.


%% The Second Principle then dually states that any non-$\bot$
%% value read in the latter line comes from a write event's timestimap
%% $t$ concurrent with {\tt scan} and that for any other write event such
%% that $ t < t'$, $t'$ finishes after line 11 in {\tt scan}, and thus
%% has to be missed. We capture the essence of these forwarding
%% principles through a series of propositions and invariants expressed
%% in terms of the auxiliary state described above:

%% %% Formalizing the Forwarding Principles in full detail requires keeping
%% %% track of otherwise unnecesary scanner events in the history. Instead,
%% %% we capture their essence with the following three propositions, based
%% %% on the colors of the write events and the scanner state.

%% %
%% \gad{We actually don't have this statement in the development, but we
%%   can prove it out of the refined colorinvariant color\_inv and
%%   Proposition~\ref{inv:redzone} below. I will add it to the Coq files
%%   to-do list.}

%% It is easy to see how this proposition corresponds to Jayanti's First
%% Forwarding Priciple: in our seeting a write event which is visible,
%% ergo never going to be missed is {\sf green}. Then the Proposition
%% above says that any write event that finished strictly before $\toff$
%% will be green. This means that in the case that a write event was
%% committed to memory after the scanner read in line 10-- in which case
%% it was originally painted {\sf yellow} ---it must have been forwarded,
%% as forwarding paints entries from yellow to green. As for the Second
%% Forwarding Principle, we capture it through the two following
%% propositions:

%% \begin{comment}
%% %%%%%%%%%  
%%  Jayanti's Forwarding Principles: in quote:
%%  '' The first principle ensures that if the scanner misses a Write
%%  at A[i], then it will catch that Write at B[i]. More precisely:

%%   Forwarding Principle 1:Suppose that a Scan operation S misses
%%   aWrite(i, v)operation W because S reads A[i] before W writes in
%%   A[i]. If W completes before S performs Line 7, then W will have
%%   surely informed S of v by writing v in B[i].

%%   The second principle ensures that any non-\bot value that the
%%   scanner may find in B[i]is current. More precisely:

%%   Forwarding Principle 2:Suppose that a Scan opera- tion S reads at
%%   Line 9 a non-\bot value written in B[i]by some Write(i, v)operation
%%   W.Then, (1) W is concurrent with S, and (2) If W′ is any Write(i,
%%   ∗)operation thatis executed after W,then W′ completes only after S
%%   performs Line 7. (...)''
%% %%%%%%%%
%% \end{comment}

%% %% Key Inv for P
%% \begin{proposition}[Green/Yellow Writes]\item\label{inv:wrtP}%
%% For each of $\histx, \histy$, if the scanner state is $\sss= \sOn,
%% \spp = \TT$, i.e., it is in lines 9--11 in
%% Figure~\ref{fig:jayanti-snapshot}, and $ \mathit{p} = v$ in the
%% physical state, then this is recorded by $\histp$: $\exists t\ldot t
%% \mapsto (p, v) \in \histp$. Moreover, $t$ will be considered by the
%% scanner, i.e., $t$ is the last green or the yellow timestamp in
%% $\histp$.
%% \end{proposition}

%% \gad{I'm not sure Proposition~\ref{inv:wrtP} is helpful here}
  

%% \gad{We might need to explain the new part of the red-zone invariant
%%   i.e. the one that explains the relation between the timestamps of
%%   the writes and $t_{\mathsf{off}}$. This is important for
%%   establishing that re-link only changes the order of overlapping
%%   events.}

%% In order to prove the correctness of these specs, we introduce the
%% invariants imposed on our axiliary state, as well as the
%% implementation in FCSL of the algorithm, decorated with auxiliary
%% code:

%% \subparagraph*{Invariants}%
%% The following are (selected) properties relevant to the proof of the
%% specs, that the state (real and auxiliary) of the snapshot algorithm
%% satisfies throughout execution.
%% \an{Some are mostly obvious
%%   book-keeping properties, but some are essential. We should present
%%   only the important ones.}

%% % Self/other histories record finish
%% writes.  \item\label{inv:gapless} $\hist$ is gapless, i.e.,
%% contains all the timestamps between $0$ and the maximal
%% one. Moreover, each timestamp is associated with either the pointer
%% $x$ or the pointer $y$; if we denote by $\histx$ and $\histy$ the
%% entries in $\hist$ writing into $x$ and $y$ respectively, then
%% $\hist = \histx \hunion \histy$. \an{removable?} \gad{yes}

%% % Self/other histories record finish
%% writes.  \item\label{inv:finished} $\histS$ and $\histO$ record
%% finished writes: $\dom{E} = \dom{\histS \hunion
%% \histO}$. \an{removable?} \gad{yes}

% Color invariant
%% \item\label{inv:color} The colors of the timestamps in each of
%%   $\histx$ and $\histy$ can be described by the regular expression
%%   $\GYR$: They have a non-empty prefix consisting of green timestamps,
%%   followed by \emph{at most} one yellow timestamp, and zero or more
%%   red timestamps in the tail.
%%   %
%%   %% Importantly, there is always a (green)
%%   %%  entry describing the initial write into each $x$ and $y$.
%
%  i.e. the coloring of each of the pointer-view histories satisfies
%  the pattern: many (at least one) green, at most one yellow, zero or
%  more red. \gad{Will I need the refinements?}. This pattern entails
%  that there is an initial value in each of $\histx$ and $\histy$ and,
%  moreover, that initial write is {\bf green}. One of those initial
%  timestamps will be $0$, i.e. $C(0)= green$. In addition, each
%  pointer-view history is sorted in real time -- naturally sorted--
%  and its elements don't overlap.

%% Non-Overlapping are sorted
%% \item\label{inv:overlap} The ending real time of a finished event
%%   appears after the initial real time: $\forall\, t \in
%%   \dom{E}\ldot\ t \leq E(t)$. \an{removable?}  Moreover, events that
%%   do not overlap in real time are ordered in logical time: $\forall
%%   t_1 \in \dom{E}, t_2 \in \dom{\hist}$, if $E(t_1) < t_2$ then $t_1
%%   \tle t_2$.
  
%% %% Last keys
%% \item\label{inv:lastkey} The entry of the last timestamp in $\histx$
%%   contains the physical value of $\x$: $\x = \histx (\mathsf{last}\,
%%   \histx)$. Similarly for $y$. \an{removable?} \gad{yes}.

%% %% Key Inv for fp
%% \item\label{inv:forward} If the scanner has cleared $\mathit{fx}$
%%   ($\sx = \TT$) but a non-$\bot$ value $v$ has been forwarded since
%%   ($\mathit{fx} = v$), then this is recorded by $\histx$:
%%   $\exists t\ldot t \mapsto (x, v) \in \histx$. Moreover, $t$ will be
%%   considered by the scanner, i.e., $t$ is the last green or the yellow
%%   timestamp in $\histx$. Dually for $\y$.

%% %  For each $\mathtt{p} \in \{\mathtt{x},\mathtt{y}\}$, if $(S_p \wedge
%% %  \mathtt{f\_of\ p}= v)$ then exists $t_p$ such that $t_p \hpts (p,v)
%% %  \in \hist$ and $t_p$ is the last green or yellow timestamp of
%% %  $\hist_{\mathtt{p}}$ i.e. $t_x = \mathsf{last\_green}\ \histx \vee
%% %  t_p = \mathsf{yellow\_ts}\ \hist_{\mathtt{p}}$. This property
%% %  entails that after the scanner has cleared the forwarding pointers,
%% %  if some new value has been forwarded, its timestamp is the last
%% %  green or the yellow of the pointer's view-history.

%% %% Key Inv for p
%% \item\label{inv:scanner} If the scanner has cleared $\mathit{fx}$
%%   ($\sx = \TT$), but still has not turned off $\s$ (i.e., it is in
%%   lines 9-11 in Figure~\ref{fig:jayanti-snapshot}), and $\x = v$ in
%%   physical state, then this is recorded by $\histx$:
%%   $\exists t\ldot t \mapsto (x, v) \in \histx$. Moreover, $t$ will be
%%   considered by the scanner, i.e., $t$ is the last green or the yellow
%%   timestamp in $\histx$. Dually for $\y$.

%  For each $\mathtt{p} \in \{\mathtt{x},\mathtt{y}\}$, if $(S_p
%  \wedge \mathtt{S})$ then exists $t_p$ such that $t_p \hpts (p,v) \in
%  \hist$ and $t_p$ is, again, the last green or yellow timestamp of
%  $\hist_{\mathtt{p}}$ i.e. $t_x = \mathsf{last\_green}\ \histx \vee
%  t_p = \mathsf{yellow\_ts}\ \hist_{\mathtt{p}}$. This property
%  entails that after the in the scanner has cleared the forwarding
%  pointers and before it unsets the scanner bit {\tt S}, the current
%  value of the pointer was committed by the last green or the yellow
%  timestamp of the pointer's view-history.

%% %% RedZone invariant
%% \item\label{inv:redzone} 
%% %
%% If $\s = \FF, \sx = \TT, \sy = \TT$, i.e., the scanner is in lines
%% 12-15 in Figure~\ref{fig:jayanti-snapshot}, then $\hist$ satisfies the
%% $\RZ$ pattern. In other words, if a new write events occur at this
%% point, it will be colored red, and hence ignored by the scanner.

%If $\wstate{S} = (\FF,\TT,\TT)$ then $\hist$ satisfies the $\RZ$
%coloring pattern: if there is a red timestamp in $\hist$, then then
%the {\sf first} red timestamp partitions $\hist$ so that everything
%before it is yellow or green and everything starting from it is red.
  
%\end{enumerate}
%\gad{to do: I'm not sure how deep I should get into the invariants?
%  Won't know for sure until I finish with the proof sketch. We might
%  cut down on them later.}

%\gad{Also: I'm thinking of pushing the invariants down together with
%  the proof sketch, or even fork them together as a separate section
%  sketching the ``correctness'' of our approach}

%% Jayanti's Forwarding Principles are guidelines for the design of the
%% algorithm and also the definition of the Linearization
%% Points. Instead, our implementation of them, as provided by
%% Propositions~\ref{inv:fwd1},~\ref{inv:readFP}, and~\ref{inv:redzone},
%% are formally implemented as invariants of the auxiliary state of the
%% data structure--- or {\it concurrent resource} in FCSL's jargon. In
%% the following we present the atomic auxiliary code operations that
%% implement {\tt write} and {\tt scan} in FCSL. These will be required
%% to satisfy the propositions introduced above, amongst other
%% requirements.

%% \an{Should we make the point that the auxiliary code constructivelly
%%   builds and evolves the various total and partial orderings that are
%%   abstracted by the specifications. It is thus an inherent component
%%   of the proof, in the same way that in type theory, existentials
%%   require a witness. There's probably no space to bring type theory
%%   into the mix.}


