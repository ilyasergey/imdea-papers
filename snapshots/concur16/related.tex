\section{Related work}
\label{sc:related}

% Early times

The proof method for establishing linearizability of concurrent
objects based on the notion of \emph{linearization points} has been
presented in the original paper by Herlihy and
Wing~\cite{Herlihy-Wing:TOPLAS90}. The first Hoare-style logic,
employing this method for compositional proofs of linearizability was
introduced in Vafeiadis' PhD thesis~\cite{Vafeiadis:PhD}. However,
that logic, while being inspired by the combination of Rely-Guarantee
reasoning and Concurrent Separation
logic~\cite{Vafeiadis-Parkinson:CONCUR07} with syntactic treatment of
linearization points~\cite{Vafeiadis-al:PPoPP06}, did not have a
soundness proof with respect to any program semantics. Furthermore,
the work~\cite{Vafeiadis:PhD} did not connect reasoning about
linearizability to the verification of client programs that make use
of linearizable objects in a concurrent environment (\cf
Section~\ref{sec:clients}).

% Modern logics for linearizability

Both these shortcomings were addressed in more recent works on program
logics for linearizability~\cite{Liang-Feng:PLDI13}, or,
equivalently~\cite{Filipovic-al:TCS10}, \emph{observational
  refinement}~\cite{Turon-al:ICFP13}, which provided semantically
sound methodologies for verifying linearizability/refinement of
concurrent objects \emph{as well as} for giving the objects
Hoare-style specifications.
%
However, in~\cite{Liang-Feng:PLDI13,Turon-al:ICFP13}, these two 
properties are established separately, thus doubling the proving
effort.
%
What is more important, none of these approaches supports reasoning
about non-regional LPs, as they require the linearization order of a
procedure being verified to be determined (with respect to other LPs)
by its end.
%
% No-linearizability

A series of more recent Hoare logics focused on specifying concurrent
behavior \emph{without} resorting to
linearizability~\cite{Sergey-al:ESOP15,Svendsen-Birkedal:ESOP14,ArrozPincho-al:ECOOP14,Jung-al:POPL15}.
%
This paper continues the same line of thinking, building
on~\cite{Sergey-al:ESOP15}, which explored patterns of assigning
Hoare-style specifications with self/other auxiliary histories to
concurrent objects, including \emph{higher-order} ones (e.g., {flat
  combiner}~\cite{Hendler-al:SPAA10}) in
FCSL~\cite{Nanevski-al:ESOP14}, but has not considered
future-dependent linearization points, as required by Jayanti's
algorithm.
%
The alternative logics
~\cite{ArrozPincho-al:ECOOP14,Jung-al:POPL15,Svendsen-Birkedal:ESOP14}
would have to make use of \emph{prophecy variables}, in order to
tackle future-dependent LP. Supporting prophecies would require
revising these logics' metatheory, and, to the best of our knowledge,
it has not been done yet.

The key novelty of the current work with respect to previous results
on Hoare logics with
histories~\cite{Fu-al:CONCUR10,Liang-Feng:PLDI13,Gotsman-al:ESOP13,Bell-al:SAS10,Sergey-al:ESOP15,Hemed-al:DISC15}
is the idea of dynamically changing (i.e., \emph{re-linking}) the
auxiliary logical histories to enable effective specification and
constructive reasoning about dynamic linearization points.
%
Since re-linking is just a manipulation of otherwise standard
auxiliary state, we were able to use FCSL \emph{off the shelf}, with
no extensions to its metatheory. Furthermore, we expect to be able to
use FCSL's higher-order features to reason about higher-order (\ie,
parametrized by another data structure) snapshot-based concurrent
constructions~\cite{Petrank-Timnat:DISC13}, which is our immediate
future work.
%

Related to our result, O'Hearn \etal have shown how to employ
history-based reasoning and Hoare-style logic to
\emph{non-constructively} prove the existence of linearization points
for concurrent objects out of the data structure
invariants~\cite{OHearn-al:PODC10}---the result is known as \emph{the
  Hindsight Lemma}. The reasoning principle presented in this paper
generalizes that idea, since the Hindsight Lemma is only applicable to
``pure'' concurrent methods (\eg, concurrent set's
\texttt{contains}~\cite{Heller-al:OPODIS05}) that do not influence the
position of other threads' linearization points. In contrast, our
history re-linking handles such cases, as showcased by Jayanti's
construction, where the linearization point of \texttt{write} depends
on the (future) outcome of \texttt{scan}.

Finally, we note that our proof of Jayanti's algorithm seems very
different from his original proof. Jayanti relies on
so-called \emph{forwarding principles}, as a key property of the
proof. For example, Jayanti's First Forwarding Principle says (in
paraphrase) that if {\tt scan} misses the value of a concurrent write
in line~10 of Figure~\ref{fig:jayanti-snapshot}, but the write
finishes before the scanner goes through line~11 (the scan's
linearization point), then the scanner will catch the value in the
forwarding pointer in line~12.
%
Instead of forwarding principles, we rely on colors to algorithmically
construct the status of each write event as it progresses through
time, and express our assertions using formal logic. For example, though we 
did not use the First Forwarding
Principle, we nevertheless can express and prove a property similar to
it: %(whose proof follows from Propositions~1-4).
%\begin{proposition}\label{inv:fwd1}%
%\hbox{}\hfil
If $\sss = \sOff\ \toff$ and $\spp = \TT$, then $\forall t \in
\histp\ldot\ t \leq \E(t)< \toff \implies \C(t)= \mathsf{green}$.
%\hfil
%\end{centering}
%\end{proposition}

% Most recent related work relies on parametrization to avoid reasoning
% about linearizability. But, that has its drawbacks. In particular,
% while it can handle situations in which linearization points are
% placed in different places, depending on the run-time infomration
% (speculiation), it is not currently strong enough to formalize
% examples where linearizatin points appear in different
% proceedures.\an{Hmm, are we super sure of this?} Thus, we don't
% believe they can handle Jayanti's algorithm.  \is{How about this time
%   we just mention these people briefly and instead do a comparison to
%   the PODC crowd and their reasoning methods, which are all about
%   harnessing the vanialla definition of liearizability. This way, the
%   whole discussion will be more relevant to the audience, as nobody
%   knows the concurrency logics anyway, so we'll just waste valuable
%   space, talking about them in detail.}  \an{I agree that we should
%   just mention them briefly.}

% Independently of us, Kyzha et al. have developed an a method whereby
% linearizability is proved by reordering time-stamped histories,
% similar to the basis of our approach. However, there are many
% differences. 

% \begin{enumerate}
% \item While linearizability does not say anything about clioent side
%   proofs, beyond the ability to replace the two programs in it, our
%   method also gives a way to reason about clients, as we illustrated
%   in Section 4.

% \item While they present a new logic, for us, it is all a mode of use
%   of auxiliary state.

% \item Our PCM of histories let us reuse separation logic in
%   infrastructure (e.g., frame rule) to reason about histories locally,
%   whereas Kyzha et al. use global histories only. Our setting also
%   immediatley lends itself to higher-order programming. This is
%   particularly important for snaphsots, as one of their major
%   application is in iterators -- a prototypical higher-order
%   program~\cite{PetrankT+disc13}. However, we don't explore iterators
%   in this particular paper.

% \item Kyzha et al. track the ordering of timestamps quite differnetly
%   from us. Where we keep an ``existential'' witness for the total
%   ordering of timestamps, at all stages of evaluation, they do so
%   ``universially''. Thus, they require proving that all possible
%   completions of a partial order into a total order are valid for
%   establishing the relation with the linearization program. We believe
%   this leads to larger proofs and more complicated proofs than
%   necessary.
% \end{enumerate}

% Liang et al~\cite{LiangF+pldi13} present a dedicated meta-theory to
% reason with future-dependent linearization points based on
% speculations. \gad{Are we sure they can't do Jayanti here? Need to
%   think what to say about their works as they do have a program logic
%   as we do}
  
