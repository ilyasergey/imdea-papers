\section{Formal proof structures}
\label{sc:formal}

\def\histx{\hist_\x}
\def\histy{\hist_\y}
\def\histp{\hist_p}
\def\ordlist{L}
\renewcommand{\tleq}{\mathrel{\leq_\ordlist}}
\renewcommand{\tle}{\mathrel{<_\ordlist}}
\renewcommand{\jleq}{\mathrel{\sqsubseteq_\ordlist}}
\newcommand{\E}{E}
\newcommand{\C}{C}
\newcommand{\sx}{S_\x}
\newcommand{\sy}{S_\y}
\newcommand{\spp}{S_p}
\newcommand{\sss}{S_s}
\newcommand{\wx}{W_\x}
\newcommand{\wy}{W_\y}
\newcommand{\wpp}{W_p}
\newcommand{\admissible}{\mathsf{fine}}

% Names for writer/scanner-states
\newcommand{\wInit}{\mathsf{W_{off}}}
\newcommand{\wWrite}{\mathsf{New}}
\newcommand{\wDirty}{\mathsf{NeedsFwd}}
\newcommand{\wClean}{\mathsf{Done}}
\newcommand{\sOn}{\mathsf{S_{on}}}
\newcommand{\sOff}{\mathsf{S_{off}}}


\gad{ Leave the beginning of the section as it is up to the auxiliary
  state, and then introduce Jayanti's FPs. Then follow to describe how
  we are implementing them and how they relate to
  Invariants~\ref{inv:forward},~\ref{inv:scanner}
  and~\ref{inv:redzone}.}

\subparagraph*{Specification.}
%
We record a history of the snapshot data structure as a set of entries
of the form $t \mapsto (p, v)$. The entry says that at time $t$ (a
natural number), the value $v$ was written into pointer $p$. Unlike in
the linearizability proof of the snapshot structure, we keep only
write events in the history. The scan events do not modify the
abstract state of the structure observable by clients, and can thus be
omitted to simplify the reasoning process.

We keep the following \emph{auxiliary} variables, which are local to
each thread: $\histS$ keeps the finished write events of the
\emph{self} thread, $\histO$ keeps the finished write events of all
other threads, while $\histJ$ keeps the write events that are in
progress. When a call to {\tt write} is initiated, an auxiliary code
will place an appropriate write entry into $\histJ$; when a call
finishes, the entry is moved from $\histJ$ to $\histS$. When one
thread changes its own $\histS$, this immediately changes every other
thread's $\histO$---a discipline provided automatically by FCSL. We
name by $\hist$ the union $\histS \hunion \histO \hunion \histJ$,
which is the history of the overall snapshot data structure. Moreover,
the latter is a \emph{disjoint union}, enforcing that $\histS$, $\histO$
and $\histJ$ do not contain common timestamps.

The timestamps in $\hist$ determine the \emph{real-time} ordering of
write events. We record the \emph{logical} ordering in the auxiliary
variable $\ordlist$. This is a list, storing a permutation of
timestamps from $\hist$. The permutation corresponds to the logical
ordering, and we write $t_1 \tleq t_2$ if $t_1$ appears before $t_2$
in $\ordlist$. In this sense, $\ordlist$ is the linkage in time for
the snapshot algorithm, and it will be dynamically modified. The
ordering $\tleq$ is a total order (i.e., a chain), but as $\tleq$ is
dynamic, and thus subject to change by the algorithm, we introduce
further notation and name by $\jleq$ a \emph{partial suborder} of
$\tleq$ that is stable and monotonic (i.e., it can only grow over
time, preserving the order). We will define $\jleq$ later, but we can
already use it to give Hoare triples for {\tt write} and {\tt scan}
below. In fact, it will be important for client reasoning
(Section~\ref{sec:clients}) to keep the definition of $\jleq$ hidden,
so that different snapshot algorithms can provide different
definitions, without influencing the clients.
%
\begin{equation}
\begin{array}{l}
\mathtt{write}\ (p : \mathtt{ptr}, n : \mathtt{int}) : 
\!\!\!\begin{array}[t]{l}
\{\hists = \hempty \wedge h \subseteq \histO
           \wedge \omega \subseteq {\jleq}\}\\[3pt]
\{\exists t\ldot\, \hists = t \mapsto (p, n) \wedge h \subseteq \histO \wedge
  \omega \subseteq {\jleq}\,{\wedge}\, h \subseteq H^{\hbox{}\sqsubsetneq_\ordlist t}\}
\end{array}\\
%
\mathtt{scan} : 
\!\!\!\begin{array}[t]{l}
\{\hists = \hempty \wedge\, h \subseteq \histO \wedge\,
          \omega \subseteq {\jleq}\}\\[3pt]
\{\hists = \hempty \wedge\, h \subseteq \histO \wedge\, \omega \subseteq {\jleq} \wedge\, \exists\, t\ldot\, %
           h \subseteq H^{\hbox{}\sqsubseteq_\ordlist t} \wedge\,
           \mathsf{chain}\ H^{\hbox{}\sqsubseteq_\ordlist t} \wedge\, r = \mathsf{eval}\ {H^{\hbox{}\sqsubseteq_\ordlist t}}\}
\end{array}
\end{array}
\label{eq:specs}
\end{equation}
%
These are partial correctness Hoare triples, describing how the
program changes the state from the precondition (first braces) to the
postcondition (second braces), possibly influencing the return result
$r$.
%
The expression $H^{\hbox{}\sqsubseteq_\ordlist t}$ selects the entries
in the history $H$ \emph{up to and including} the timestamp $t$,
according to the ordering $\sqsubseteq_\ordlist$. Likewise,
$H^{\hbox{}\sqsubsetneq_\ordlist t}$ is defined to exclude $t$, {\it
  mutatis~mutandis}.
%

With that in mind, the spec for {\tt write} says the following. We
start with the empty self history indicating that the procedure did
not yet make any writes. We use the variable $h$ to name an arbitrary
subset of the initial value of $\histO$ (hence, of \emph{completed}
write events), and the variable $\omega$ for a subset of the initial
ordering $\jleq$. The postcondition says that when {\tt write}
terminates, one write event $t \mapsto (p, v)$ has finished, and is
hence placed into $\histS$.  $\histO$ and $\jleq$ may have changed
from the previous values, but they still include $h$ and $\omega$ as
subsets. That is, $\histO$ could \emph{only grow}, because other
threads could have created new write events. Similarly for $\jleq$:
other threads could change the ordering in $\ordlist$, but only in a
way which add more relationships between timestamps in
$\jleq$. Lastly, the conjunct
$h \subseteq H^{\hbox{}\sqsubsetneq_\ordlist t}$ says the write events
that have been completed before the call to {\tt write} (and are hence
stored in $h$) will be \emph{ordered before} the event $t$ in
$\sqsubsetneq_\ordlist$. Other threads can permute $L$ to reorder some
entries, but in $L$, $t$ will remain \emph{strictly} after all the
entries from $h$.

The spec for {\tt scan} starts with the same precondition. In the
postcondition, it says that $\histS$ is empty, because {\tt scan}
itself does not execute write events. However, by the time {\tt scan}
returns the pair $r$ as a snapshot, we know that there exists a
timestamp $t$ in the collective history $H$ of the data structure at
which the snapshot \emph{appears} to be taken. First, the conjunct
$h \subseteq H^{\hbox{}\sqsubseteq_\ordlist t}$ indicates that the
snapshot is taken \emph{after} the call to {\tt scan}, because the
finished events stored in $h$ are ordered before (or at best, at)
timestamp $t$.
%
%% Second, the set $H^{\hbox{}\sqsubseteq_\ordlist t}$ is
%% a subchain in $\sqsubseteq_\ordlist$ (technically, all its entries are
%% ordered in $\sqsubseteq_\ordlist$).
%
%
Second, the set $H^{\hbox{}\sqsubseteq_\ordlist t}$ is a sub-chain in
$\sqsubseteq_\ordlist$, \ie $\forall\, a,b \in
H^{\hbox{}\sqsubseteq_\ordlist t}\ldot\ a \sqsubseteq_{\ordlist} b
\vee b \sqsubseteq_{\ordlist} a$.
%
Intuitively, the chain provides a way to see how the history logically
progressed up to the snapshotting moment $t$, and moreover, this view
will not change in the future in a way which invalidates $r$ as a
snapshot (since $\sqsubseteq_\ordlist$ is stable). Finally, if the
chain of writes is evaluated in the order given by $L$, it produces
the snapshot $r$.

\gad{Actually, proving that chain is stable entails proving something
  stronger: that the sub-chain $H^{\hbox{}\sqsubseteq_\ordlist t}$ is
  ``maximal'' or complete i.e. that no more entries are going to be
  added to that sub-history. This is (will be) proven by the fact that
  in the atomic moment after we have executed re-link,
  $H^{\hbox{}\sqsubseteq_\ordlist t}$ defines a green prefix from the
  very beginning up to the timestamp t, and such prefixes are stable}.

\subparagraph*{Additional auxiliary state.} 
We require yet further structure to describe the admissible ways in
which $\ordlist$ can change while respecting the
specs~(\ref{eq:specs}).

First, we will introduce auxiliary state variables in order to record
the point in which an executing writer or a scanner currently are. We
refer to them as the writer and, respectively, scanner state. The
latter is defined by the triple $(\sss, \sx, sy)$. $\sss$ belongs to
the set $\{ \sOn, \sOff\, t\}$: $\sOn$ denotes that an ongoin scanner
is in between lines 8--10, whereas $\sOff\ t$ states that the scanner
executed line 11 at logical time\ $t$ and its execution is in between
lines 11--15. Morover, the auxiliary bits $\sx$ and $\sy$ will be set
when the scanner has cleared the respective forwarding pointer
(line~9, Figure~\ref{fig:jayanti-snapshot}), and reset upon scanner's
termination. As for the writer, variables $\wx$ and $\wy$ will range
over the set $\{ \wInit, \wWrite\ t, \wDirty\ t, \wClean\ t\}$, where
$t$ is the logical timestamp of an ongoing write event. $\wInit$
indicates that there is no write operation is in progress;
$\wWrite\ t$ says that writer is in line~2 in
Figure~\ref{fig:jayanti-snapshot}; $\wDirty\ t$ says that line~3 has
finished and $b$ is set (hence, forwarding in line 5 will occur);
$\wClean\ t$ says that we do not need to track the writer's state
anymore, and it is free to exit.

Second, we need to record how events overlap in time. As typical in
linearizability, events that do not overlap will \emph{not} be
reordered wrt.~each other. In the proofs, this is required in order to
establish that the timestamp $t$ in the postcondition of {\tt write}
and {\tt scan} always appears after the events in $h$. To address this
issue, we keep an auxiliary variable $\E$, storing a function that
maps each timestamp $t$ of a \emph{finished} write events in $\hist$,
to a timestamp identifying the \emph{ending time} of the event.
%
%\gad{I will try to move the invariant for Endpoints here} 
%

Third, we need to track the visibility of write events with regard to
an ongoing scan method (i.e., is the write missed by the scan). This
will inform how events are reordered in $\ordlist$. We do so by
introducing \emph{colors} for timestamps: $\mathsf{green}$,
$\mathsf{yellow}$ and $\mathsf{red}$. We keep an auxiliary variable
$\C$, storing a function that maps each timestamp in $\hist$ to a
color.
%
The intuition behind the colors is as follows. It is always relative
to the ongoing scan, or the last finished one, if no scans are active.
%
\begin{itemize}
\item Green timestamps identify write events that the scan is
  guaranteed to see. This includes the events that finished before the
  scan started. The order of green timestamps in $\ordlist$ is fixed
  in the following sense: if $\C(t_1) = \mathsf{green}$ and
  $t_1 \tle t_2$, then $t_2$ will never be reordered before $t_1$ in
  $\ordlist$.

\item Yellow timestamps are undecided. The scan may or may not see
  them, hence they may be reorder in $\ordlist$. Due to the
  single-writer/single-scanner assumption, $\hist$ contains at most
  one yellow timestamp per pointer.

\item Red timestamps are guaranteed to be missed by the scan. They
  will be put after the green ones.
\end{itemize}
%\gad{Can I put color invariant here?}

%Color patters turn out to be a very flexible and maleable abstraction
%in order to reasoning about the state of the resource. \gad{I'm
%  looking for a formal way of saying that programming/reasoning about
%  the color lemmas is fun, at least for a FP-minded guy!}

%% In order to prove the correctness of these specs, we introduce the
%% invariants imposed on our axiliary state, as well as the
%% implementation in FCSL of the algorithm, decorated with auxiliary code:

We can now define $\jleq$ as follows:
\begin{equation}
 t_1 \jleq t_2 = t_1\tleq t_2 \wedge ( E\,(t_1) < t_2) \vee \C(t_1) =
 \mathsf{green} \vee (t_1 = t_2) ) \label{def-jleq}
\end{equation}
%% %
%% \gad{I've changed the definition so that it is closer to its
%%   implementation in the Coq development.}

%% \begin{equation}
%%  t_1 \jleq t_2  = (t_1 = t_2) \vee (E\,(t_1) < t_2) \vee
%%  (t_1\tle t_2 \wedge \C(t_1) = \mathsf{green}) \label{def-jleq}
%% \end{equation}
  
Intuitively, the ordering $t_1 \jleq t_2$ is stable because either the
events associated with $t_1$ and $t_2$ are equal, or non-overlapping,
or $t_1$ is green, and hence fixed, as commented previously.

\gad{Introduce Jayanti's forwarding principles and how we implement
  them with the colors.  Use this to motivate the invariants in
  particular 6,7 and 8, which are the ones that correspond closely to
  the forwarding principles. We use the auxiliary code to implement
  Jayanti's forwarding principles. explain with a bit less
  detail. Maybe move invariants down.}

\subparagraph*{Invariants}%
The following are (selected) properties relevant to the proof of
specs~\eqref{eq:specs}, that the state (real and auxiliary) of the snapshot
algorithm satisfies throughout execution.
%% \an{Some are mostly obvious
%%   book-keeping properties, but some are essential. We should present
%%   only the important ones.}
\begin{enumerate}

%% % Self/other histories record finish
%% writes.  \item\label{inv:gapless} $\hist$ is gapless, i.e.,
%% contains all the timestamps between $0$ and the maximal
%% one. Moreover, each timestamp is associated with either the pointer
%% $x$ or the pointer $y$; if we denote by $\histx$ and $\histy$ the
%% entries in $\hist$ writing into $x$ and $y$ respectively, then
%% $\hist = \histx \hunion \histy$. \an{removable?} \gad{yes}

%% % Self/other histories record finish
%% writes.  \item\label{inv:finished} $\histS$ and $\histO$ record
%% finished writes: $\dom{E} = \dom{\histS \hunion
%% \histO}$. \an{removable?} \gad{yes}

% Color invariant
\item\label{inv:color} The colors of the timestamps in each of
  $\histx$ and $\histy$ can be described by the regular expression
  $\GYR$: They have a non-empty prefix consisting of green timestamps,
  followed by \emph{at most} one yellow timestamp, and zero or more
  red timestamps in the tail.
  %
  %% Importantly, there is always a (green)
  %%  entry describing the initial write into each $x$ and $y$.
%
%  i.e. the coloring of each of the pointer-view histories satisfies
%  the pattern: many (at least one) green, at most one yellow, zero or
%  more red. \gad{Will I need the refinements?}. This pattern entails
%  that there is an initial value in each of $\histx$ and $\histy$ and,
%  moreover, that initial write is {\bf green}. One of those initial
%  timestamps will be $0$, i.e. $C(0)= green$. In addition, each
%  pointer-view history is sorted in real time -- naturally sorted--
%  and its elements don't overlap.

%% Non-Overlapping are sorted
%% \item\label{inv:overlap} The ending real time of a finished event
%%   appears after the initial real time: $\forall\, t \in
%%   \dom{E}\ldot\ t \leq E(t)$. \an{removable?}  Moreover, events that
%%   do not overlap in real time are ordered in logical time: $\forall
%%   t_1 \in \dom{E}, t_2 \in \dom{\hist}$, if $E(t_1) < t_2$ then $t_1
%%   \tle t_2$.

\item\label{inv:overlap} The logical order $\tle$ preserves the real
  time order of {\it non-overlapping} past write events. Formally,
  $\forall t_1 \in \dom{E}, t_2 \in \dom{\hist}$, if $t_1 < t_2$ and
  $E(t_1) < t_2$ then $t_1 \tle t_2$.
  
%% %% Last keys
%% \item\label{inv:lastkey} The entry of the last timestamp in $\histx$
%%   contains the physical value of $\x$: $\x = \histx (\mathsf{last}\,
%%   \histx)$. Similarly for $y$. \an{removable?} \gad{yes}.

%% Key Inv for fp
\item\label{inv:forward} If the scanner has cleared $\mathit{fx}$
  ($\sx = \TT$) but a non-$\bot$ value $v$ has been forwarded since
  ($\mathit{fx} = v$), then this is recorded by $\histx$:
  $\exists t\ldot t \mapsto (x, v) \in \histx$. Moreover, $t$ will be
  considered by the scanner, i.e., $t$ is the last green or the yellow
  timestamp in $\histx$. Dually for $\y$.

%  For each $\mathtt{p} \in \{\mathtt{x},\mathtt{y}\}$, if $(S_p \wedge
%  \mathtt{f\_of\ p}= v)$ then exists $t_p$ such that $t_p \hpts (p,v)
%  \in \hist$ and $t_p$ is the last green or yellow timestamp of
%  $\hist_{\mathtt{p}}$ i.e. $t_x = \mathsf{last\_green}\ \histx \vee
%  t_p = \mathsf{yellow\_ts}\ \hist_{\mathtt{p}}$. This property
%  entails that after the scanner has cleared the forwarding pointers,
%  if some new value has been forwarded, its timestamp is the last
%  green or the yellow of the pointer's view-history.

%% Key Inv for p
\item\label{inv:scanner} If the scanner has cleared $\mathit{fx}$
  ($\sx = \TT$), but still has not turned off $\s$ (i.e., it is in
  lines 9-11 in Figure~\ref{fig:jayanti-snapshot}), and $\x = v$ in
  physical state, then this is recorded by $\histx$:
  $\exists t\ldot t \mapsto (x, v) \in \histx$. Moreover, $t$ will be
  considered by the scanner, i.e., $t$ is the last green or the yellow
  timestamp in $\histx$. Dually for $\y$.

%  For each $\mathtt{p} \in \{\mathtt{x},\mathtt{y}\}$, if $(S_p
%  \wedge \mathtt{S})$ then exists $t_p$ such that $t_p \hpts (p,v) \in
%  \hist$ and $t_p$ is, again, the last green or yellow timestamp of
%  $\hist_{\mathtt{p}}$ i.e. $t_x = \mathsf{last\_green}\ \histx \vee
%  t_p = \mathsf{yellow\_ts}\ \hist_{\mathtt{p}}$. This property
%  entails that after the in the scanner has cleared the forwarding
%  pointers and before it unsets the scanner bit {\tt S}, the current
%  value of the pointer was committed by the last green or the yellow
%  timestamp of the pointer's view-history.

%% RedZone invariant
\item\label{inv:redzone} 
%
If $\s = \FF, \sx = \TT, \sy = \TT$, i.e., the scanner is in lines
12-15 in Figure~\ref{fig:jayanti-snapshot}, then $\hist$ satisfies the
$\RZ$ pattern. In other words, if a new write events occur at this
point, it will be colored red, and hence ignored by the scanner.

\gad{Explain better. We might need to explain the new part of the
  red-zone invariant i.e. the one that explains the relation between
  the timestamps of the writes and $t_{\mathsf{off}}$. This is
  important for establishing that re-link only changes the order of
  overlapping events.}

%If $\wstate{S} = (\FF,\TT,\TT)$ then $\hist$ satisfies the $\RZ$
%coloring pattern: if there is a red timestamp in $\hist$, then then
%the {\sf first} red timestamp partitions $\hist$ so that everything
%before it is yellow or green and everything starting from it is red.
  
\end{enumerate}
%\gad{to do: I'm not sure how deep I should get into the invariants?
%  Won't know for sure until I finish with the proof sketch. We might
%  cut down on them later.}

%\gad{Also: I'm thinking of pushing the invariants down together with
%  the proof sketch, or even fork them together as a separate section
%  sketching the ``correctness'' of our approach}

%% scanner transitions
%% A stub for the re-linking figure.
\begin{figure}
\begin{subfigure}[t]{0.48\textwidth}
\includegraphics[height=5cm]{res/before-relink-trans}
\caption{\label{fig:relink:before} Before re-link}
\end{subfigure} \hfill
\begin{subfigure}[t]{0.48\textwidth}
\includegraphics[height=4.5cm]{res/after-relink-trans}
\caption{\label{fig:relink:after} After re-link}
\end{subfigure}%
%
\caption{\label{fig:relink} Atomic re-link: $(rx,ry)$ points to the
  snapshot that will be returned by {\tt scan}}
\end{figure}

%\gad{FIX ME: subfigure (b) is incorrectly drawn! 1 and 3 are swapped
%  in Real Time as well!!!}

%Fix me: The yellow should be painted red after relink, if we decide
%to go for the (green - red) split for t\left t\right after re-link.
%Though, this would not be stable after we release the lock for scan,
%so why bother. Painting the t|left chain green will suffice, as it
%will be stable.

% Fix me: red marbles have different font colors :(


%\gad{Up to here!}

We next describe the auxiliary code for {\tt scan}.
\[
\begin{array}{l@{\, :\ }l}
%% clear
  \aux{clear}(p) &
      \{\s = \TT,\ \spp = \FF,\ \C = c \}\quad
      \{\s = \TT,\ \spp = \TT,\
      \C = c[\histp \hpts \mathsf{green}] \}\\[3pt]
%% re-link
   \aux{relink}(rx, ry) &
  \!\!\! \begin{array}[t]{l}
    \{\s = \FF, \sx =\TT, \sy = \TT, \C = c, \ordlist = l,\, %
%      t_x = \admissible\ \histx, t_y = \admissible\ \histy, rx = \histx(t_x), ry = \histy(t_y) \}\\
        \hist = h \hunion t_x \hpts (x, rx) \hunion t_y \hpts (y, ry),\\
      \ (t_x = \aux{last\_green}\ \histx \vee
          t_x = \aux{yellow\_of}\ \histx), (t_y = \aux{last\_green}\ \histy \vee
          t_y = \aux{yellow\_of}\ \histy) \}\\
    \{\s = \FF, \sx=\FF, \sy=\FF,%
        \C = c[t_x, t_y \hpts \mathsf{green}],\,\\
      \ \ordlist = \mathsf{if}\ (d = \mathsf{Yes}\ x\ t_r)\
                \mathsf{then}\ \aux{push}\ t_r\, t_y\, l\
                 \mathsf{else\ if}\
                 (d = \mathsf{Yes}\ y\ t_r)\ \mathsf{then}\
                 \aux{push}\ t_r\, t_x\, l\ \mathsf{else}\ l\}\\
  \quad \mbox{where $d = \aux{inspect}\ t_x\, t_y\, l\, \C$}
  \end{array}
\end{array}
\]
%
The procedure $\aux{clear}(p)$ is executed in line~9 simultaneously
with clearing the forward pointer for $p$. In addition to recording
that the scanner passed line~9 by setting the $\spp$ bit, it paints
the subhistory $\histp$ green. Thus, all the previous writes to $p$
will be visible, and hence accounted for in the remaining machinery of
the scan, with their ordering being fixed forever.

Finally, $\aux{relink}$ is the most important among the auxiliary
procedures. It captures the essence of our approach, as it decides how
to change the logical order given in $\ordlist$ (if it needs to change
at all), to justify $(rx,ry)$ as a snapshot.
%
It starts by finding the timestamps $t_x$ and $t_y$ that are
responsible for writing $rx$ and $ry$ into $\histx$ and $\histy$,
respectively. There may be many such timestamps, but we focus on the
ones that are yellow, or last green in the respective subhistories of
their pointer. It follows from invariants (\ref{inv:forward}) and
(\ref{inv:scanner}) that such must exist.
%
Next, $\aux{relink}$ calls the helper procedure $\aux{inspect}$ to
decide if $t_x$ and $t_y$ determine a valid snapshot; that is, there
are no other events between them that make the pair $(rx, ry)$ look
like it is not a snapshot. $\aux{Inspect}$ tests if $t_x < t_y$ and
then makes a decision based on the color of each, and the color of
events between them. The definition has several cases, but we only
describe one here, referring to the example in
Figure~\ref{fig:reorder}. In that figure, we have $rx = 2$, $ry = 1$,
and $t_x$, $t_y$ stand for the events of writing $2$ and $1$,
respectively. Let us assume that both are the last green events in
their respective subhistories, and $t_x < t_y$ in the real
time. However, there is a yellow timestamp in $\histx$ coming afer
$t_x$, corresponding to the write of $3$. Let us name that timestamp
$t'_x$. Because $t_x < t'_x < t_y$, the pair $(rx, ry)$ is not a
snapshot. To make it so, we have to move $t'_x$ out of the way. The
output of $\aux{inspect}$ in this case will precisely be the value
$\mathsf{Yes}\ x\ t'_x$, indicating that $t'_x$ needs to move in the
logical history $\histx$. The move is accomplished by invoking yet
another helper function $\aux{push}$, which reorders $t'_x$ right
after $t_y$ in $\ordlist$.
%
%Notice that the reordering may cause $t'_y$ to jump over 
%several timestamps, thought that is not shown in 
%Figure~\ref{fig:relink}. 
Notice that the move of $t'_x$ respects the invariant that green
timestamps remain fixed. This is a general property of $\aux{relink}$,
which is responsible for stability of $\jleq$.
%
Finally, as the last move in $\aux{relink}$, the color of $tx$ and
$ty$ is changed to green (if it were not so already), thus preventing
future reordering. We can then prove that $(rx, ry)$ will be a valid
snapshot wrt.~$L$ under any interference.


\begin{comment}
%
$\aux{relink}$ works as follows: first, the auxiliary function 
$\mathsf{decide}$ checks $l$ to see if $t_x$ and $t_y$ form a valid 
snapshot, in which case it will return $\mathsf{No}$ and then 
$\ordlist = l$, else it will return $\mathsf{Yes}\ p\ t_r$, indicating 
that there is a miss and that $t_r \in \hist_{p}$--- and is such that 
$t_r \leq t_{\neq p}$ ---should be {\it pushed} in $\ordlist$ past 
$t_{\neq p}$, the latter being $t_y$ if $p=x$, and vice 
versa. Finally, the returned timestamps are painted green.



In Figure~\ref{fig:relink}, we revisit the example from
Section~\ref{sc:overview}, adding the colors to the time-stamps. There
we see in Figure~\ref{fig:relink:before} that $(rx,ry)$ points to
$(2,1)$ and both timestamps are green. We notice that there is a
yellow in between, $\mathsf{yellow}\, \histx$.  In
Figure~\ref{fig:relink:after}, we see that it has been pushed after
$ry$.

$\aux{relink}$ works as follows: first, the auxiliary function 
$\mathsf{decide}$ checks $l$ to see if $t_x$ and $t_y$ form a valid 
snapshot, in which case it will return $\mathsf{No}$ and then 
$\ordlist = l$, else it will return $\mathsf{Yes}\ p\ t_r$, indicating 
that there is a miss and that $t_r \in \hist_{p}$--- and is such that 
$t_r \leq t_{\neq p}$ ---should be {\it pushed} in $\ordlist$ past 
$t_{\neq p}$, the latter being $t_y$ if $p=x$, and vice 
versa. Finally, the returned timestamps are painted green.

We clarify how $\mathsf{decide}$ works: without any loss of
generality, assume $t_x \tleq t_y$. The pre-condition of
$\aux{relink}$ says $t_x$ and $t_y$ are, respectively, the last green
or yellow timestamps of $\histx$ and respectively $\histy$. From the
latter facts facts and Invariant~\ref{inv:redzone}, we know that every
timestamp in the chain from $t_x \tleq t_y$ will be green or yellow
and there will be, at most, two yellow timestamps, one for each
$\histx$ and $\histy$. Then, if $t_x$ is yellow, $\mathsf{decide}$
will return $\mathsf{No}$-- if there are further elements in $\histx$,
they will be in the red tail, and outside the chain-- e.g. the token 7
in Figure~\ref{fig:relink}. Now, if $t_x$ is green and there is no
yellow key in $\histx$, $\mathsf{decide}$ will return
$\mathsf{No}$. If there is, we need to compare it with $t_y$: if
$\mathsf{yellow}\ \histx \tle t_y$, as in Figure~\ref{fig:relink},
$\mathsf{decide}$ will return $\mathsf{Yes}\ x\
(\mathsf{yellow}\, \histx)$ and the latter will be pushed right after
$t_y$ in \ordlist-- $\mathsf{push}$ implements the pointer-swing-like
manipulation on $l$. Last, if $ t_y \tle \mathsf{yellow}\ \histx$
there will be no $\mathsf{push}$ either.

\end{comment}

%% \gad{The figure might change to include the red-zone invariant as well,
%%   as the endpoints if we need to}. Here give a proof-sketch on how
%% re-link satisfies the invariants.


\def\tleqP{\mathrel{\leq_{\ordlist'}}}
\def\tleP{\mathrel{<_{\ordlist'}}}
\def\jleqP{\mathrel{\sqsubseteq_{\ordlist'}}}
%% \subparagraph*{Correctness of write and scan's specifications}%

We devote the rest of the section to illustrating the structure of our
proofs of {\tt write} and {\tt scan}, including aspects of
mechanization in FCSL/Coq. For lack of space, we cannot present the
full theory, but just focus on the structure and a few characteristic
properties.

The first stage of proof (and the mechanization) involves defining the
state space of the algorithm; that is, the definitions of the
auxiliaries, and the invariants that relate them. We presented
selected invariants in Section~\ref{sc:formal}. There is a number of
properties that need to be established under the state space, e.g.,
that it is closed under exchanging history entries between $\histS$
and $\histO$. The latter facilitates dynamic thread creation, because
it allows the reasoning to focus on individual threads in a large
parallel composition.
%
The second stage involves defining the auxiliary code, as in
Figure~\ref{fig:auxcode}, and proving a number of its properties. In
addition to correctness, the properties involve \emph{erasure} (i.e.,
auxiliary code only mutates auxiliary state, but not the real
one), \emph{framing} (i.e., if auxiliary code is executable in some
input heap and input history, then executing it in an extended heap
and history leads to the same results, and the extensions remain
invariant), preservation of invariants (from Section~\ref{sc:formal}),
and \emph{termination}. The correctness aspect of auxiliary code
involves proving that the code preserves the state-space invariants,
and that the precondition and postcondition hold. For example, the
correctness proof of $\aux{relink}$, relies on the following two
helper lemmas.

%The first lemma asserts that $\aux{inspect}$ correctly determines the
%``offending'' timestamp; the second lemma asserts that $\aux{push}$
%modifies the logical ordering in a way that makes the pair $(r_x,
%r_y)$ a valid snapshot.

\begin{lemma}[Correctness of $\aux{inspect}$]\label{lem:inspect}
If $t_x$, $t_y$ are timestamps for write events of $r_x$, $r_y$, then
$\aux{inspect}\ t_x\ t_y\ \ordlist\ \C$ correctly determines that
$(r_x, r_y)$ is a valid snapshot under ordering $\tleq$ and coloring
$C$, or otherwise, returns an ``offending'' timestamp. More formally,
if $\sss = \sOff\ \toff, \sx =\TT, \sy =\TT$, and for each
$p \in \{x,y\}$, $ t_p \hpts (p, r_p) \in \hist$ and $(t_p
= \aux{last\_green}\ \histp \vee \C(t_p) = \mathsf{yellow})$, the
following are exhaustive possibilities.
% if $\aux{inspect}\
%t_x\ t_y\ \ordlist\ \C = \mathsf{No}$ then $(r_x,r_y)$ is a valid
%snapshot. Else, if $\aux{inspect}\ t_x\ t_y\ \ordlist\ \C
%= \mathsf{Yes}\ p\ t'$ then $(r_x,r_y)$ is not a valid snapshot and
%moreover $t'$ is the offending timestamp in $\histp$. Furthermore, by
%exhausting the hypotheses about $t_x$ and $t_y$ we can infer that:
\begin{enumerate}
 \item If $t_x \tle t_y$ and $ \C(t_x) = {\sf yellow}$, then
      $\aux{inspect}\ t_x\ t_y\ \ordlist\ \C
      = \mathsf{No}$. Symmetrically for $t_y \tle t_x$.

 \item If $ t_x \tle t_y $, $ t_x = \aux{last\_green}\ \histx$, and
       $\forall t' \in \histx\ldot\ t_x \tle t'\,{\implies}\,t_y \tle
       t'$, then \\ $\aux{inspect}\ t_x\ t_y\ \ordlist\ \C
       = \mathsf{No}$. Symmetrically for $t_y \tle t_x$.

 \item If $ t_x <_{\ordlist} t_y $, $ t_x = \aux{last\_green}\ \histx
      $, $t' \in \histx$, and $t_x \tle t' \tle t_y$, then
      $\aux{inspect}\ t_x\ t_y\ \ordlist\ \C = \mathsf{Yes}\ \x\ t'$
      and $\C(t') = {\sf yellow}$.  Symmetrically for $t_y \tle t_x$.
\end{enumerate}
\end{lemma}

\begin{lemma}[Correctness of $\aux{push}$]\label{lem:push}
If $\sss = \sOff\ \toff, \sx =\TT, \sy =\TT$, and for $p \in \{x,y\}$,
we have $t_p \hpts (p, r_p) \in \hist$ and $(t_p
= \aux{last\_green}\ \histp \vee \C(t_p) = \mathsf{yellow})$, and
$\aux{inspect}\ t_x\ t_y\ \ordlist\ \C = {\sf Yes}\ p\ t'$, then:
\begin{enumerate}
 \item If $p = x$, then $(r_x, r_y)$ is a valid snapshot under
 $\leq_{\aux{push}\ t'\ t_y\ \ordlist}$. Symmetrically for $p = y$.
\item If $p = x$, then $\ordlist' = \aux{push}\ t'\ t_y\ \ordlist$ and 
$\C' = \C[t_x, t_y \hpts \mathsf{green}]$ satisfy the state space
 invariants
 (Propositions~\ref{inv:overlap}-\ref{inv:redzone}). Symmetrically for
 $p = y$.
\end{enumerate}
\end{lemma}

\begin{comment}
%
%One then has to prove that the state space remains valid under
%exhcanging of history entries between $\histS$ and $\histO$. Such
%exchanging is important because it govens the behavior of auxiliary
%state when threads fork and join, as follows. Let the parent thread be
%$e_1 \parallel e_2$, with \emph{self} history $\histS$
%and \emph{other} history $\histO$.  Upon forking, $\histS$ will be
%split non-deterministically as $\histS = \histS_1 \hunion \histS_2$,
%with $\histS_1$ assigned as a self history of $e_1$, and dually for
%$e_2$. On the other hand, the \emph{other} history of $e_1$ becomes
%$\histO \hunion \histS_2$, to reflect that upon forking $e_2$ becomes
%environment for $e_1$ (and symmetrically for $e_2$).

\subparagraph*{Correctness And Mechanization}%
We finish this section highlighting the most interesting details of
the mechanization of the snapshot data structure in FCSL. Our aim is to
give a general feel of the process of proving a concurrent data
structure correct in the logic, from the definition of the {\it
concurrent resource} for Jayanti's snapshot construction and its
invariants, to the final proof of correctness of the specification of
{\tt write} and {\tt scan} methods from Figure~\ref{fig:specs}. We
will focus mostly on properties of, and assertions about,
$\aux{relink}$ and we refer the reader to Appendix~\ref{sc:coq-code}
and the Coq mechanization~\cite{FCSL:Project} for further detail.

%% \gad{
%% What does an FCSL proof involve:
%% \begin{itemize}
%% \item building the concurrent resource, proving FCSl's meta-theoretic requirements.
%% \item Proving auxiliary code satisfies the invariants i.e. different
%% propositions given before.
%% \item Proving stability of the assertions
%% \item Proving the specs. for the methods
%% \end{itemize}
%% }

% First, build theory for re-linkable histories, auxiliary state,
% invariants and lemmas.

%\gad {Actually, shouldn't we be saying that the proof starts in the
%whitebord?}

The first stage of the mechanization involved building the meta-theory
for re-linkable histories/orders together with the specific auxiliary
state for the snapshot algorithm described above, its properties and
invariants, and several lemmas for reasoning about them. Second, we
implemented the auxiliary code and proved each of them correct,
wrt. to the invariants of the auxiliary code. This entails proving
that each of them satisfies Propositions~\ref{inv:color} --
~\ref{inv:redzone}, among others. We first highlight an intermediate
result about $\aux{inspect}$:

\begin{lemma}[Inspect]\label{lem:inspect}
$\aux{inspect}\ t_x\ t_y\ \ordlist\ \C$ determines if $(r_x, r_y)$ is
a valid snapshot, else finds the \emph{offending} timestamp. More
formally, if $\sss = \sOff\ \toff, \sx =\TT, \sy =\TT$, given $r_x$
and $r_y$ such that for $p \in \{x,y\}$ if $ t_p \hpts (p,
r_p) \in \hist$ and $(t_p = \aux{last\_green}\ \histp \vee \C(t_p)
= \mathsf{yellow}$), then if $\aux{inspect}\ t_x\ t_y\ \ordlist\ \C
= \mathsf{No}$ then $(r_x,r_y)$ is a valid snapshot. Else, if
$\aux{inspect}\ t_x\ t_y\ \ordlist\ \C = \mathsf{Yes}\ p\ t'$ then
$(r_x,r_y)$ is not a valid snapshot and moreover $t'$ is the offending
timestamp in $\histp$. Furthermore, by exhausting the hypotheses about
$t_x$ and $t_y$ we can infer that:
\begin{enumerate}
 \item If $t_x \tle t_y$ and $ \C(t_x) = {\sf yellow}$ then
      $\aux{inspect}\ t_x\ t_y\ \ordlist\ \C = \mathsf{No}$. {\it
      Idem} for $t_x \tle t_y$ and $ \C(t_x) = {\sf yellow})$.

 \item If $ t_x \tle t_y $ and $ t_x = \aux{last\_green}\ \histx $ and
      any timestamp $t'$ that appears after $t_x$ in $\histx $, \ie
      $t_x \tle t' $, also appears after $t_y$ in $\tle$, \ie $
      t_y \tle t' $, then $\aux{inspect}\ t_x\ t_y\ \ordlist\ \C
      = \mathsf{No}$. The same for $ t_y \tle t_x$\ldots, {\it mutatis
      mutandis}.

 \item If $ t_x <_{\ordlist} t_y $ and $ t_x
      = \aux{last\_green}\ \histx $ and there exists a timestamp $t'$
      in $\histx$ such that it appears in $\tleq$ in between $t_x$ and
      $t_y$, \ie $ t_x \tleq t' \tleq t_y $ then $\aux{inspect}\ t_x\
      t_y\ \ordlist\ \C = \mathsf{Yes}\ \x\ t'$. The same for $
      t_y \tle t_x$ \ldots, {\it mutatis mutandis}.
\end{enumerate}
\end{lemma}

%% \begin{lemma}[Inspect]\label{lem:inspect}
%% If $\sss = \sOff\ \toff, \sx =\TT, \sy =\TT$, given $r_x$ and $r_y$
%% such that for $p \in \{x,y\}$ if $ t_p \hpts (p, r_p) \in \hist$ and
%% $(t_p = \aux{last\_green}\ \histp \vee \C(t_p) = \mathsf{yellow}$),
%% then:
%% \begin{enumerate}
%% \item If $(t_x <_{\ordlist} t_y \wedge \C(t_x) = {\sf yellow}) \vee
%%       (t_y <_{\ordlist} t_x \wedge \C(t_y) = {\sf yellow})$ then
%%       $\aux{inspect}\ t_x\ t_y\ \ordlist\ \C = \mathsf{No}$.

%% \item If $ t_x <_{\ordlist} t_y $ and $ t_x
%%       = \aux{last\_green}\ \histx $ and $ \forall\
%%       t' \in \histx\ldot t_x <_{\ordlist} t' \implies t_y <_{\ordlist} t'
%%       $ then $\aux{inspect}\ t_x\ t_y\ \ordlist\ \C
%%       = \mathsf{No}$. Conversely, when $ t_y <_{\ordlist} t_x \ldots
%%       $, {\it mutatis mutandis}.

%% \item If $ t_x <_{\ordlist} t_y $ and $ t_x
%%       = \aux{last\_green}\ \histx $ and $ \exists\ t' \in \histx\ldot
%%       t_x <_{\ordlist} t' <_{\ordlist} t_y $ then $\aux{inspect}\ t_x\
%%       t_y\ \ordlist\ \C = \mathsf{Yes}\ \x\ t'$. Conversely, when $
%%       t_y <_{\ordlist} t_x \ldots $, {\it mutatis mutandis}.
%% \end{enumerate}
%% Moreover, if $\aux{inspect}\ t_x\ t_y\ \ordlist \C = \mathsf{No}$ then
%% $(r_x,r_y)$ is a valid snapshot.
%% \end{lemma}

The lemma above states that $\aux{inspect}$ asserts whether the
timestamps $t_x$ and $t_y$, pointing to the return values $r_x$ and
$r_y$, define a snapshot or not, and in the later case determines the
``offending'' timestamp that needs to be reordered. We consider cases
where $t_x <_{\ordlist} t_y$, as the others are defined conversely. In
the first case, if $t_x <_{\ordlist} t_y$ and $\C(t_x) = {\sf
yellow}$, it follows from $t_y$ being the last green or yellow
timestamp of $\histy$ and Proposition~\ref{inv:redzone} that any
$t' \in \histx$ s.t. $ t_x <_{\ordlist} t'$ will be {\sf red} and
hence $ t_y <_{\ordlist} t'$. Then, $\aux{inspect}\ t_x\
t_y\ \ordlist\ \C = \mathsf{No}$ and $(r_x,r_y)$ will be a valid
snapshot. In the second case, $t_x$ is the last green timestamp of
$\histx$ and, if there is a yellow timestamp $t' \in \histx$ and $ t_y
<_{\ordlist} t'$, then $\aux{inspect}\ t_x\ t_y\ \ordlist\ \C
= \mathsf{No}$, and clearly, $(r_x, r_y)$ is a valid snapshot. In the
last case, there exists a yellow timestamp $t'$ s.t. $ t' <_{\ordlist}
t_y$, so $t'$ invalidates $(r_x, r_y)$ as a snapshot and consequently
$\aux{inspect}\ t_x\ t_y\ \ordlist\ \C = {\sf Yes}\ \x\ t'$, denoting
that $t'$ has to be reordered in $\tle$.
%
%% We refer the reader to the {\tt Coq} mechanization for the formalities
%% of Lemma~\ref{lem:inspect}.
We next prove that, in the later case,
%
%% We next prove that when $\aux{inspect}\ t_x\ t_y\ \ordlist\ \C = {\sf
%% Yes}\ p\ t'$,
the result of $\aux{pushl}\ t'\ t_y\ \ordlist$ fixes the order $\tleq$
such that $(r_x, r_y)$ is a valid snapshot in $\leq_{(\aux{push}\ t'\
t_y\ \ordlist)}$:

\begin{lemma}[Push]\label{lem:push} $\aux{push}\ t'\ t_p$ fixes $\tleq$
so that $(r_x,t_y)$ is a correct snapshot in $\leq_{\aux{push}\ t'\
t_{p}\ \ordlist}$. More formally, if $\sss = \sOff\ \toff, \sx
=\TT, \sy =\TT$, given $r_x$ and $r_y$ such that for $p \in \{x,y\}$
if $ t_p \hpts (p, r_p) \in \hist$ and $(t_p
= \aux{last\_green}\ \histp \vee \C(t_p) = \mathsf{yellow}$) then:
\begin{enumerate}

\item If $\aux{inspect}\ t_x\ t_y\ \ordlist\ \C = {\sf Yes}\ x\ t'$
then $\aux{push}\ t'\ t_{y}\ \ordlist$ makes $(r_x, r_y)$ a valid
snapshot under $\leq_{\aux{push}\ t'\ t_{x}\ \ordlist}$. The same for
${\sf Yes}\ y\ t'$, {\it mutatis mutandis}.

 \item The auxiliary state as specified by the postcondition of
$\aux{relink}$ in Figure~\ref{fig:scanauxcode}, with $\ordlist'
= \aux{push}\ t'\ t_{p}\ \ordlist$ and $ \C' = \C[t_x,
t_y \hpts \mathsf{green}]$ satisfies the invariants of the concurrent
resource from Propositions~\ref{inv:overlap} to~\ref{inv:redzone}.
\end{enumerate}
\end{lemma}

%% \begin{lemma}[Push]\label{lem:push}
%% If $\sss = \sOff\ \toff, \sx =\TT, \sy =\TT$, given $r_x$ and $r_y$
%% such that for $p \in \{x,y\}$ if $ t_p \hpts (p, r_p) \in \hist$ and
%% $(t_p = \aux{last\_green}\ \histp \vee \C(t_p) = \mathsf{yellow}$)
%% then, if $\aux{inspect}\ t_x\ t_y\ \ordlist\ \C = {\sf Yes}\ p\ t'$
%% then:
%% \begin{enumerate}
%%  \item $\aux{push}\ t'\ t_{z}\ \ordlist$, with $z = y$ if $p = x$ and,
%%  conversely, $z = y$ if $p = y$, makes $(r_x, r_y)$ a valid snapshot
%%  under $\leq_{\aux{push}\ t'\ t_{z}\ \ordlist}$.
%% \item The resulting auxiliary state in the post-condition of relink
%%  above, with $\ordlist' = \aux{push}\ t'\ t_{z}\ \ordlist$ and $ \C'
%%  = \C[t_x, t_y \hpts \mathsf{green}]$ satisfies the invariants of the
%%  concurrent resource from Propositions~\ref{inv:overlap}
%%  to~\ref{inv:redzone}.
%% \end{enumerate}
%% \end{lemma}

We consider the first item: by Lemma~\ref{lem:inspect}, if
$\aux{inspect}\ t_x\ t_y\ \ordlist\ \C = {\sf Yes}\ x\ t'$ then $t'$
is the yellow timestamp of $\histx$. Then, all elements $t$ in the
chain $t' \tle t \tleq t_y$ belong to $\histy$, by virtue of $t_y$
being yellow or the last green in $\histy$ and
Invariants~\ref{inv:color} and~\ref{inv:redzone}. Thus, in $\ordlist'
= \aux{push}\ t'\ t_y\ \ordlist$, $\forall\ t' \in t_x \tleP t' \tleP
t_y\ldot. t' \in {\histy'}$, hence $(r_x,r_y)$ is a valid snapshot by
$\tleqP$. As for the second item, the proof of the preservation of
most invariants is quite straightforwards, the crux of the matter
being the observation that all $t$ in the chain $t' \tle t
\tle t_y$ are {\it overlapping} and, as a result, changing the linking
of $t'$ in the logical order does not contradict
Proposition~\ref{inv:overlap}.
\end{comment}

% Stability.
%We refer the reader to the {\tt Coq} source for the formal details
%about Lemmas~\ref{lem:inspect} and~\ref{lem:push}. 

We now have the following main correctness result.
%%
%% Final Correctness of the methods
\begin{theorem}[Correctness of {\tt write} and {\tt scan}]\label{thm:specs}
The implementations for {\tt scan} and {\tt write} in
Figure~\ref{fig:fcsl-snapshot} can be ascribed the specification given
in Figure~\ref{fig:specs}.
\end{theorem}
%
The theorem is proved following a customary style of Floyd-Hoare
symbolic evaluation using the inference rules of
FCSL~\cite{Nanevski-al:ESOP14}, a variant of Hoare logic. We have
mechanized this proof in FCSL/Coq, including all the properties from
the first two stages described above. The proof is subject to the
following helper stability properties that we have not yet mechanized,
but have proved manually, and their proofs are in
Appendix~\ref{sc:coq-code}.

%The stability proofs are carried out by induction on the number of
%steps that the environment takes.  The permitted steps of the
%environment are precisely the auxiliary procedures
%from~\ref{fig:fcsl-snapshot}, except that \emph{self} and \emph{other}
%variables are interchanged to capture that it is the environment doing
%the stepping. 

%% Out of the assertions presented in Figure~\ref{fig:specs}, we
%% highlight the proof of monotonicity of the {\it stable} dynamic order
%% $\jleq$:

\begin{lemma}[Stability of Assertions]\label{lem:menvs}
If by environment stepping $L$ evolves to $L'$, then:
\begin{enumerate}
 \item\label{lem:menvs:jleq} $ \jleq \subseteq\ \jleqP$ \ie the
 logical ordering is monotone.

 \item \label{lem:menvs:drs} $H^{\hbox{}\sqsubseteq_\ordlist
 t} \subseteq H^{\hbox{}\sqsubseteq_{\ordlist'} t}$. 
%
% \ie the restriction of $\hist$ to the elements in the $\jleq$ up to
% $t$ grows monotonically. 
Similarly for $\sqsubsetneq$.
 
 \item \label{lem:menvs:chain} If $\mathsf{chain}\
 H^{\hbox{}\sqsubseteq_\ordlist t}$ then $\mathsf{chain}\
 H^{\hbox{}\sqsubseteq_{\ordlist'} t}$.

 \item \label{lem:menvs:eval} If $\mathsf{eval}\
 H^{\hbox{}\sqsubseteq_\ordlist t}$ then $\mathsf{eval}\
 H^{\hbox{}\sqsubseteq_{\ordlist'} t}$.
\end{enumerate}
\end{lemma}
%
%The proof of Lemma~\ref{lem:menvs}, items~\ref{lem:menvs:jleq}
%and~\ref{lem:menvs:drs} have been mechanized. The (manual) proofs of
%items \ref{lem:menvs:chain} and~\ref{lem:menvs:eval} are given in the
%Appendix~\ref{sc:coq-code}, but are not yet mechanized. 
%
%Finally, we have the main correctness result, mechanized relative to
%the proofs of Lemma~\ref{lem:menvs}.\ref{lem:menvs:chain}
%and~\ref{lem:menvs}.\ref{lem:menvs:eval}. 










