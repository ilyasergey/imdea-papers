\documentclass[blockstyle,preprint,nocopyrightspace]{sigplanconf}

\usepackage{amssymb,amsmath,amsthm}
\usepackage{latexsym}
\usepackage{graphicx}
\usepackage{color}
\usepackage{listings}
\usepackage{float}
\usepackage[scaled=.90]{helvet}
\usepackage[noend]{algorithmic}
\usepackage{mathrsfs}
\usepackage{mathpartir}
\usepackage{dsfont}
\usepackage{stmaryrd}
\usepackage{url}
\usepackage{textcomp} 
\usepackage{hyperref}
\usepackage{titlesec}
\usepackage{parskip}
\usepackage{alltt}
\usepackage{xspace}
\usepackage{verbatim}
\usepackage{enumitem}
\usepackage{lipsum}

%\usepackage{natbib}
% \bibpunct();A{},
% \let\cite=\citep
 
% remarks
\newcommand{\todo}[1]{\textcolor{red}{({#1})}}
\newcommand{\is}[1]{\textcolor{blue}{(Ilya: {#1})}}
\newcommand{\an}[1]{\textcolor{red}{(Aleks: {#1})}}

% useful macros
\newcommand{\asgn}{\leftarrow}
\newcommand{\code}[1]{\lstinline{#1}}
\newcommand{\ccode}[1]{\code{#1}}
\newcommand{\var}[1]{\({#1}\)} 
\newcommand{\num}[1]{\({\text{{\scriptsize{#1}}}}\)}
\newcommand{\etc}{\emph{etc}}
\newcommand{\ie}{\emph{i.e.}\xspace}
\newcommand{\Ie}{\emph{I.e.}\xspace}
\newcommand{\eg}{\emph{e.g.}\xspace}
\newcommand{\Eg}{\emph{E.g.}\xspace}
\newcommand{\vs}{\emph{vs.}\xspace}
\newcommand{\etal}{\emph{et~al.}\xspace}
\newcommand{\adhoc}{\emph{ad hoc}\xspace}
\newcommand{\viz}{\emph{viz.}\xspace}
\newcommand{\dom}[1]{\mathsf{dom}(#1)}
\newcommand{\aka}{\textit{a.k.a.}\xspace}
\newcommand{\cf}{\textit{cf.}\xspace}
\newcommand{\wrt}{\emph{wrt.}\xspace}
\newcommand{\loef}{L\"{o}f}
\newcommand{\sep}{\textasteriskcentered}
\newcommand{\res}{\mathsf{res}} 
\newcommand{\ret}{\mathsf{ret}} 
\newcommand{\fix}{\mathsf{fix}} 

%specs
\newcommand{\specK}[1]{\ensuremath{\textcolor{blue}{#1}}}
\newcommand{\spec}[1]{\specK{\left\{{#1}\right\}}}
\newcommand{\specQ}[4]{[#1 , #2 , #3] \, #4}%\specQ{mL}{gL}{gE}{p}
\newcommand{\drspec}[1]{\specK{\langle{#1}\rangle}}
\newcommand{\sspec}[1]{\specK{\{{#1}\}}}




% Keep footnotes on one page
\interfootnotelinepenalty=10000 

\setlength{\parindent}{0.15in}
\setlength{\topsep}{0cm}
\setlength{\parskip}{0pt}
\titlespacing*{\section}{0pt}{*1}{*1}
\titlespacing*{\subsection}{0pt}{*1.0}{*0.5}
\titlespacing*{\subsubsection}{0pt}{*1.0}{*0.5}
\titlespacing*{\paragraph}{0pt}{*0.5}{*0.5}

% SSReflect listings 
\input{lstcoq}
\lstset{style=Coq}


\begin{document}

\special{papersize=8.5in,11in}
\setlength{\pdfpageheight}{\paperheight}
\setlength{\pdfpagewidth}{\paperwidth}


\authorinfo{Ilya Sergey}
           {IMDEA Software Institute}
           {ilya.sergey@imdea.org}
%
\authorinfo{Aleksandar Nanevski}
           {IMDEA Software Institute}
           {aleks.nanevski@imdea.org}

\authorinfo{Anindya Banerjee}
           {IMDEA Software Institute}
           {anindya.banerjee@imdea.org}

\title{
Towards Structured Mechanized Verification\\
of Fine-Grained Concurrent Programs 
}

\subtitle{Extended Abstract}

\maketitle
 
It has been long recognized that efficient parallelization is of
crucial importance for high-performant software. Unfortunately,
reasoning about correctness of concurrent programs, in which several
computations can be executed in parallel and, thus, overlap in time,
is challenging due to the large number of possible interactions
between concurrent processes/threads on shared data structures.

One way to deal with the complexity of verifying concurrent code is to
employ the mechanisms of so-called \emph{coarse-grained} concurrency,
\ie, \emph{locks}. By making use of locks in his code, the programmer
ensures \emph{mutually-exclusive} access to critical resources, so
that at most one thread will be accessing the shared data structure at
the same moment of time, therefore, reducing the proof of correctness
of the concurrent code to the proof of correctness of the
\emph{sequential} code. While sound, this approach to concurrency
prevents one from taking full advantage of parallel computations. An
alternative is to implement shared data structures in a
\emph{lock-free} (\ie, \emph{fine-grained}) manner, so the threads
manipulating such structures would be reaching a consensus through the
active use of non-blocking read-modify-write operations (\eg, CAS)
instead of locks.

Despite the clear practical advantages of the fine-grained approach to
the implementation of concurrent data structures, it requires a
significant expertise to devise such structures and establish their
correctness with respect to standard criteria, such as
\emph{linearizability}~\cite{Herlihy-Wing:TOPLAS90,Burckhardt-al:PLDI10}.
%
Even worse, in the presence of the advanced programming features, such
as higher-order functions, pointer aliasing, and internal parallelism,
linearizability by itself might not be strong enough as a criteria to
ensure the full functional correctness of a concurrent program, and
generalizations of linearizability to match these programming features
are a topic of an ongoing active
research~\cite{Cerone-al:ICALP14,Gotsman-Yang:CONCUR12}.

An alternative approach to specify the behaviour of imperative, and,
in particular, concurrent programs as well as to verify them is to
make use of \emph{program logics}. In such logical frameworks, program
specifications are represented by means of \emph{Hoare triples}
$\spec{P}~c~\spec{Q}$, where $c$ is a program being specified, $P$ is
a \emph{precondition}, \ie, a predicate that describes a state, in
which the program should be run, and $Q$ is a \emph{postcondition},
which describes a state upon the program's termination.
%
The logic-based approach has a number of appealing characteristics.
First, checking the program correctness in a Hoare-style program logic
can be done \emph{structurally}, \ie, by means of systematically
applying syntax-directed inference rules, until its specification is
derived.
%
Second, verification via a program logic is \emph{compositional}: once
a library procedure is verified against a suitable specification in
terms of pre- and postconditions, its code is not required to be
re-examined ever again: all reasoning about the client code that makes
use of that procedure can be carried out solely out of the procedure's
specification, which is coherent with good practices of code
encapsulation.
%
Finally, modern program logics are sufficiently \emph{expressive} in
the sense that they can give specifications to the programs that
operate with first-class executable code, locally-spawned threads and
other features inherent for modern programming languages.

Since the process of structural program verification in a Hoare-style
logic is largely mechanical, there has been a number of research
projects recently, targeting to mechanize and automate it for
\emph{sequential} and \emph{coarse-grained concurrent} programs by
means embedding it into a general-purpose proof assistant, \eg,
Coq~\cite{Chlipala:PLDI11,Nanevski-al:POPL10}, or implementing a
standalone verification tool~\cite{Jacobs-al:NFM11}.

The situation is different with the logic-based verification of
\emph{fine-grained} concurrency, which requires reasoning about a
number of concepts, that don't have direct analogues in the sequential
or coarse-grained concurrent programming:

\begin{itemize}[leftmargin=*]

\item \textbf{Custom resource protocols.} Each data structure (\ie,
  \emph{resource}), which can be used by several threads concurrently,
  needs to be supplied with a specific definition of an ``evolution
  protocol'', in order to enforce the preservation the structure's
  internal consistency. In contrast with the coarse-grained case,
  where the protocol is fixed and corresponds to locking/unlocking,
  fine-grained data structures require defining custom protocols with
  multiple states and transitions.
  
\item \textbf{Interference and stability.} For the sake of local
  reasoning about the shared data structure's state from a single
  thread's perspective, one should formalize what are the admissible
  changes that can be made by other threads, \emph{interfering} with
  the current one. That is, every thread-local assertion about the
  structure's state should be \emph{stable}, \ie, invariant under
  possible concurrent modifications.

\item \textbf{Work stealing} is a common concurrent pattern, appearing
  in fine-grained programs due to relaxing the \emph{mutual exclusion}
  policy, so several threads can simultaneously operate with a single
  shared resource. The ``stealing'' happens when a thread is scheduled
  for a particular task involving the resource, but the task is then
  is accomplished by \emph{another} thread; however, the
  ``contribution'' is nevertheless ascribed to the initially assigned
  thread.

\end{itemize}
%
\noindent
%
In addition, Hoare-style reasoning about coarse- or fine-grained
concurrency requires a form of \textbf{auxiliary state} in order to
partially expose the internal threads' behavior and relate local
program assertions to global invariants, accounting for specific
threads' contributions into a resource.
%

These four aspects, paramount for Hoare-style verification of
fine-grained concurrent programs, have been recognized and formalized
in one form or another in a series of recently published works by
various
authors~\cite{LeyWild-Nanevski:POPL13,Svendsen-al:ESOP13,Svendsen-Birkedal:ESOP14,Turon-al:ICFP13,Vafeiadis-Parkinson:CONCUR07,DinsdaleYoung-al:ECOOP10,ArrozPincho-al:ECOOP14,Jacobs-Piessens:POPL11,Feng-al:ESOP07,Feng:POPL09},
providing logics of increasing expressivity and compositionality.
% 
However, in the formal proofs for concurrent libraries, performed in
all these logical systems, most of the complexity comes \emph{not}
from the libraries' size in terms of lines of code, but from the
intricacy of the corresponding data structure invariant and the
presence of thread interference and work stealing. In particular, the
later fact, in contrast with proofs about sequential and
coarse-grained concurrent programs, requires one to establish
stability of \emph{every} intermediate assertion in the program
verified.
%
Needless to say that in such setting paper-and-pencil verification of
fine-grained concurrent programs becomes a highly challenging and
error-prone task, as it's too easy for a human prover to overlook a
piece of resource-specific invariant or an assertion, unstable under
interference, hence, rendering the whole reasoning unsound.

In this work, we present a framework for \emph{mechanized}
specification and verification of fine-grained concurrent programs
based on the recently proposed \emph{Fine-grained Concurrent
  Separation Logic} (FCSL) by Nanevski
\etal~\cite{Nanevski-al:ESOP14}.\footnote{Hereinafter, we will be
  using the acronym FCSL to refer both to the Nanevski \etal's logical
  framework and to our implementation of it.}
% 
FCSL is implemented as a library and an embedded domain-specific
language (DSL) in the dependently-typed language of the Coq proof
assistant~\cite{Coq-manual}.
%
Due to its design and the choice of a logical foundation, FCSL, as a
verification tool for fine-grained concurrency, is:

\begin{itemize}[leftmargin=*]

\item \textbf{Uniform:} FCSL's specification model is based on two
  basic constructions from computer science: \emph{state-transition
    systems} (STSs) and \emph{partial commutative monoids} (PCMs). The
  former ones are used for describing concurrent protocols and thread
  interference, whereas the later ones provide a generic treatment of
  shared resources and thread contributions, making it possible to
  encode, in particular, the work stealing pattern. In the talk, we
  will demonstrate how these two components are sufficient to specify
  a large spectrum of concurrent algorithms as well as to make the
  proofs of verification obligations to be uniform.

\item \textbf{Expressive}: FCSL's \emph{specification} fragment is
  based on the propositional fragment of Calculus of Inductive
  Constructions (CIC), which makes it capable of expressing a large
  variety of mathematical theories (\eg, graphs, arrays, event traces,
  \etc) as well as to encode a number of algebraic structures, all
  occurring in the program specifications in one way or another.

\item \textbf{Realistic:} In addition to the ability inherited from
  CIC to employ higher-order predicates in the specifications, the
  \emph{programming} fragment of FCSL features a complete toolset of
  modern programming abstractions, including user-specified algebraic
  datatypes, higher-order functions and pattern matching, since
  \emph{any} Coq program is also an FCSL program.
  % 
  The monadic nature of FCSL's embedding into Coq makes it possible to
  encode a number of computational effects, such as thread spawning
  and general recursion. All this makes writing programs in FCSL to be
  very similar to programming in a higher-order language, such as ML
  or Haskell.

\item \textbf{Foundational:} The soundness of FCSL as a logic has been
  proven in Coq with respect to a version of denotational semantics
  for concurrent programs in the spirit of
  Brookes~\cite{Brookes:TCS07}. Moreover, thanks to the design choice
  of representing FCSL specifications as mere Coq types, the soundness
  result scales to the \emph{whole} language of Coq, not just a toy
  ``core'' calculus, therefore, ensuring the absence of bugs in the
  whole verification tool and, as a consequence, in any program, which
  is verified in it.

\end{itemize}

\section*{Structure of the talk}
\label{sec:structure-talk}

In the proposed talk, we will introduce the FCSL verification
framework by example, specifying and verifying full functional
correctness of a characteristic fine-grained program---concurrent
graph-manipulating algorithm.
%
Starting from the basic intuition behind the formalization of the
algorithm and its specification, we will demonstrate the common stages
and patterns of structuring program verification in FCSL.
%
While doing so, we will show how the verification process, while not
fully automated at this stage, nevertheless does not put a lot of
proof burden to the human prover, thanks to the fact that, as an
embedded DSL, FCSL inherits a number of features from the common
infrastructure of Coq, enhanced by Gonthier \etal's Ssreflect
extension~\cite{Gonthier-al:TR}: dependent records, boolean reflection
and canonical structures.
%
We will next elaborate on some important design choices made in the
implementation of FCSL with respect to encoding of concurrent
protocols and program specifications.
%
We will then report on our experience of verifying in FCSL a number of
benchmark concurrent programs and data structures: locks, simple
memory allocator, concurrent stack and its client programs, atomic
snapshot, non-blocking universal constructions, and graph-manipulating
algorithms.

\bibliographystyle{abbrv}
\bibliography{bibmacros,references,proceedings-short}
 
\end{document}


