\section{Introduction}
\label{sec:intro}
%In this work we present a model that distills the essential
%ingredients required for compositional specification and verification
%of shared-memory concurrent programs.
%
There are two main styles of program logics for shared-memory
concurrency, customarily divided according to the supported kind of
granularity of program interference. Logics for coarse-grained
concurrency such as Concurrent Separation Logic
(CSL)~\cite{OHearn07,Reynolds02} restrict the interference to critical
sections only, but generally lead to more modular specifications and
simpler proofs of program correctness. Logics for fine-grained
concurrency, such as Rely-Guarantee (RG)~\cite{jones83} admit
arbitrary interference, but their specifications have traditionally
been more monolithic, as we shall illustrate. In this paper, we identify the essential
ingredients required for compositional specification of concurrent
programs, and combine them in a novel way to reconcile the two
approaches. We present a semantic model and a logic that enables
specification and reasoning about fine-grained programs, but in the
style of CSL.
%
To describe our contribution more precisely, we first compare the
relevant properties of CSL and RG.
%Logics for coarse-grained concurrency allow shared memory to be
%accessed by a thread only after the thread acquires exclusive
%ownership of it. Such a protocol is limiting, but it enables
%encapsulating the interference between threads to a well-circumscribed
%region of memory, typically called a \emph{shared resource}, with an
%associated \emph{resource
%  invariant}~\cite{Owicki-Gried:CACM76}. Resources are an abstraction
%which modularize the specification and reasoning. One may verify a
%program wrt.~the smallest set of resources required, then introduce
%additional resources without invalidating the existing verification.
%Logics for fine-grained concurrency allow arbitrary protocols on
%shared memory, but the protocol specification has historically been
%monolithic, requiring the user to predict the full context  in which 
%
%For example, one may create and
%discard resources on the fly, while verifying programs wrt.~the
%smallest set of resources required.
%
%
% the kind of programs that can be written and verified, it provides a
% useful concept of a \emph{shared resource} and associated
% \emph{resource invariants}~\cite{Owicki-Gries:CACM76}, which abstract
% the interference between threads, and help modularize the
% reasoning. For example, one may freely introduce new resources,
% discard old ones, verify programs wrt.~the smallest set of resources
% it requires.
%
%\emph{resource invariants}, which modularize the reasoning about
%interference between threads.
%
%While following this protocol on shared state limits
%the kind of programs that can be verified, this kind of discipline has
%
%\an{Describe the topic of the paper here. Strengthen the point about 
%composing concurrent systems, which neither CSL nor RG do. Strengthen
%the point about identifying components \emph{and properties} that
%enable concurrent systems composition.}

%\subsubsection{Coarse-grained concurrency and Concurrent Separation Logic}
%\label{sec:coarse-grain-conc}
%The main representative of logics for \emph{coarse-grained
%  concurrency}, 
CSL employs \emph{shared resources} and associated
\emph{resource invariants}~\cite{Owicki-Gries:CACM76}, to abstract the
interference between threads. A resource $r$ is a chunk of shared state,
and a resource invariant $I$ is a predicate over states, which holds
of $r$ whenever all threads are outside the critical section. By
mutual exclusion, when a thread enters a critical section for $r$, it
acquires ownership and hence exclusive access to $r$'s state. The
thread may mutate the shared state and violate the invariant $I$, but
it must restore $I$ before releasing $r$ and leaving the critical
section, as given by the following CSL rule~\cite{Brookes:TCS07}.
\begin{mathpar}
{\small
\inferrule*[right={CritSecCSL}]
  {\Gamma \vdash \{p \lsep I\}\ c\ \{q \lsep I\}}
  {\Gamma, r : I \vdash \{p\}\ \mathsf{with}\ r\ \mathsf{do}\ c\ \{q\}}
}
\end{mathpar}
%
$\Gamma$ is a context of currently existing resources. The rule for
parallel composition assumes that forked threads don't share any state
beyond that of the resources in $\Gamma$, and may divide the private
state of the parent thread disjointly among the children.
%
\begin{mathpar}
\begin{small}
\inferrule*[right={ParCSL}]
  {\Gamma \vdash \{p_1\}\ c_1\ \{q_1\} \and \Gamma \vdash \{p_2\}\ c_2\ \{q_2\}}
  {\Gamma \vdash \{p_1 \lsep p_2\}\ c_1 \parallel c_2\ \{q_1 \lsep
    q_2\}}
\end{small}
\end{mathpar}
%
\noindent
A private heap of a thread may be promoted into a freshly allocated
shared resource in a scoped manner by the following rule.
\begin{mathpar}
{\small

\inferrule*[right={ResourceCSL}]
  {\Gamma, r : I \vdash \{p\}\ c\ \{q\}}
  {\Gamma \vdash \{p \lsep I\}\ \mathsf{resource}\ r\ \mathsf{in}\ c\ \{q \lsep I\}}
}
\end{mathpar}
%
One may see from these rules that resources are abstractions that
promote modularity. In particular, one may verify a thread wrt.~the
smallest resource context required. By context weakening, the
introduction of new resources will not invalidate the existing
verification. Thread-local resources can be hidden from the
environment by the \textsc{ResourceCSL} rule.
%One difficulty with resource invariants is that they can't directly
%specify non-invariant properties. For example, a reasource invariant
%can express that an integer value stored in $r$ is non-negative
%throughout program execution, but not that $r$ increments by some
%specified amount. Expressing the latter requires \emph{auxiliary
%  state}~\cite{owi+gri, r-l+n}.

%\subsubsection{Fine-grained concurrency and Rely-Guarantee reasoning}
%\label{sec:fine-grain-conc}
%The main representative of logics for \emph{fine-grained concurrency},
%where interference may occur between any two memory operations, is
%Rely/Guarantee (RG)~\cite{Jones83}. 
In RG, the interaction between threads is directly specified by the
rule for parallel composition.\footnote{In the presence of heaps, the
  rule is more
  complicated~\cite{Vafeiadis-Parkinson:CONCUR07,Feng-al:ESOP07}, but
  we elide the issue here.} 
%
\begin{mathpar}
{\small
\inferrule*[right={ParRG}]
  {R \oor G_2, G_1 \vdash \{p\}\ c_1\ \{q_1\} \and
   R \oor G_1, G_2 \vdash \{p\}\ c_2\ \{q_2\}}
  {R, G_1 \oor G_2 \vdash \{p\}\ c_1 \parallel c_2\ \{q_1 \aand q_2\}}
}\end{mathpar}
%
The rely transition $R$ and guarantee transitions $G_1$ and $G_2$ are
relations on states. A rely specifies the thread's expectations of
state transitions made by its environment. A guarantee specifies the
state transitions made by the thread itself. The disjunctive
combinations of $R$ and $G$'s in the rule captures the idea we call
\emph{forking shuffle}, whereby upon forking, the thread $c_1$ becomes
part of the environment for $c_2$ and vice-versa.

RG is more expressive than CSL because transitions can encode
arbitrary protocols on shared state, whereas CSL is specialized to a
fixed mutual exclusion protocol on critical sections. But, CSL is more
compositional in manipulating resources. Where a CSL resource
invariant specifies the behavior of an individual chunk of shared
state, the transitions in RG treat the whole state as monolithically shared. Feng's
work on Local Rely Guarantee (LRG)~\cite{Feng:POPL09} has made first
steps in improving RG in this respect.
% (and we compare with it in
%Section~\ref{sec:related}).
%The recent work on combining RG with
%CSL~\cite{Vafeiadis-Parkinson:CONCUR07,Feng-al:ESOP07,Feng:POPL09} has
%made the first steps in improving the former in this respect.

\subsection{Contributions}
\label{sec:our-approach}
We propose that a logic for fine-grained concurrency can be based on a
notion of a \emph{fine-grained resource}. Fine-grained resources serve
as buildings blocks for program specification, and generalize
CSL-style coarse-grained resource management.
%
%
%Instead of improving on RG, in this paper we propose that attaining
%the best of both reasoning styles can be achieved by generalizing CSL
%with a novel notion of \emph{fine-grained resource}. 
%
A fine-grained resource is specified by a resource invariant, as in
CSL, but it also adds transitions in the form of relations between
resource states. Thus, it is best viewed as a \emph{state transition
  system} (STS), where the resource invariant specifies the state
space. We identify a number of properties that an STS has to satisfy
to specify a fine-grained resource, and refer to such STSs as
\emph{concurroids}. We refer to our generalization of CSL as
Fine-grained CSL (\SCST).

There are two main ideas that we build on in \SCST, and which separate
\SCST from LRG and other recent related
work~\cite{DinsdaleYoung-al:ECOOP10,Svendsen-al:ESOP13,Turon-al:ICFP13}
(see Section~\ref{sec:related} for details):
(a)~\emph{subjectivity} and (b)~\emph{communication}.
%
Subjectivity~\cite{LeyWild-Nanevski:POPL13} means that each state of a
concurroid STS describes not only the shared resource, but also
\emph{two} abstractions of it that represent the views of the state by
the thread, and by its environment,
respectively. Subjectivity will enable us to capture the idea of
forking shuffle by a rule for parallel composition akin to
\textsc{ParCSL} (but with a somewhat generalized notion of separating
conjunction~$(\lsep)$~\cite{LeyWild-Nanevski:POPL13}), rather than in
the monolithic style of \textsc{ParRG}.

%encode the idea of the forking shuffle \emph{at the level of state},
%rather than by disjoining transitions. Hence, the \SCST rule for
%parallel composition will be in the compositional style of
%\textsc{ParCSL}, with a somewhat generalized notion of separating

To compositionally build large systems out of a number of smaller
ones, we make concurroids communicate.
%Intuitively, communication
%generalizes the folklore notion of the \emph{ownership
%  transfer}~\cite{OHearn07}. 
In addition to standard for STS \emph{internal} transitions between
states, concurroids contain \emph{external} transitions. These may be
thought of as ``wires'' whose one end is connected to a state in the
STS, but whose other end is dangling, representing either an ``input''
into or an ``output'' out of the STS. Concurroids can be
\emph{entangled}, \ie, composed by interconnecting their dangling wires
of opposite polarity, where the interconnections serve to transfer heap
ownership between concurroids. Communication and entanglement endow
\SCST with the compositionality of CSL. For example, entanglement
generalizes the notion of adding a resource to the context $\Gamma$ in
\textsc{ResourceCSL}. 
%However, while CSL resources are independent
%from each other and don't interact, \SCST resources can exchange
%information by communication. 
We also rely on entanglement to
formulate a rule generalizing the scoped resource allocation of
\textsc{ResourceCSL}.
%
% \SCST provides a specific inference rule \textsc{Inject} which
% generalizes the corresponding notion of context weakening, and allows
% that verifications carried out wrt.~a concurroid $U$ can be lifted to
% an entanglement $U \entangle V$. \SCST also provides an inference rule
% \textsc{Hide} for creating a resource out of private state in a scoped
% manner, thus generalizing \textsc{ResourceCSL}.
%
%\subsection{Contributions}
%\label{sec:contr-outl}
%
%\begin{itemize}
More precisely, our contributions are:
\begin{itemize}
\item We identify STSs with subjectively-shaped states (concurroids)
  and a number of algebraic properties, as a natural model for
  scalable concurrency verification.  We show how communication
  enables composing larger STSs out of smaller ones.
\item We present \SCST---a simple and expressive logic for
  fine-grained resources that combines expressivity of RG with the
  compositional resource management of CSL.
\item We illustrate \SCST by showing how to implement a coarse-grained
  resource of CSL by a fine-grained resource of \SCST in which an
  explicit spin lock protects the resource's state. We also
  implemented examples such as ticketed locks, that go beyond
  coarse-grained CSL resources, and present them in the extended
  version~\cite{fcsl-coqscripts}.
%
%in the appendix.
%
\item We implemented \SCST~\cite{fcsl-coqscripts} as a shallow
  embedding within the type theory of the Calculus of Inductive
  Constructions (\ie, Coq~\cite{coq-team,Bertot-Casteran:04}). Thus,
  \SCST naturally reconciles with features such as higher-order
  functions, abstract predicates, modules and functors. We formally
  instantiated the whole stack of abstractions: the semantic model is
  formalized in Coq, \SCST is built on top of the semantic model, CSL
  is built on top of \SCST, and then verified programs are built on
  top of CSL.
\end{itemize}





