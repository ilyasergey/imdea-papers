\section{Actions}\label{sec:actions}

Actions are the primive objects for programming in the language of
SCSP. They perform \emph{atomic} steps from state to state, and have
the lowest granularity. An action can operationally be thought of as a
single memory operation, such as e.g., reading from, writing into, or
CAS-ing over a heap cell. 
%
\is{This point should be emphasized even more: it is a crucial
  difference from CAP and its descendants, who allow one to use
  arbitrary actions, without any undeerlying intuition about
  ``machine-level'' atmoicity.}
%
An action may also be idle. Additionally,
actions may return a value of some given type $A$, as well as manipule
the auxiliary state, and perform ownership transfer (i.e., reallign
the boundaries between self, joint and other components).

An action is always associated with some concurroid $V$ in the
following sense. If an action modifies state $s$ into state $s'$, then
$(s, s') \in \mathsf{internal}_V$.\footnote{In our formalization, we
  also have actions that are associated with $\mathsf{acquire}$ and
  $\mathsf{release}$ transitions. However, these are not used for
  programming, but only for building other $\mathsf{internal}$-related
  actions. Thus, we don't present them here.}  

\an{Do we want to use the notation $a : \mathsf{action}\ V\ A$, to
  denote that the action $a$ is associated with the concurroid $V$,
  and returns a value of type $A$. We should probably leave the
  notation for the next section, where it will also mean that the
  action components satisfy the properties.}

Similar to related work on CAP, the development of SCLP is
\emph{parametrized} by the set of actions. The user of SCLP is free to
choose any set of actions associated with any concurroids, provided
that the chosen actions satisfy a number of properties which we
describe next.

Each action $a$ is \emph{specified} by 4 components: (1) the
concurroid $V$ whose internal transition it respects, (2) the type $A$
of the action's return value, (3) the predicate $\mathsf{safe}_a$ (on
states) describing the states in which the action is safe to execute,
and (4) the relation $\mathsf{step}_a$ relating the initial state, the
ending state, and the ending result of $a$.\an{Of course, we actually
  denotationally \emph{model} actions precisely as that pair of
  components, but I want to keep references to denotational semantics
  for the model section.}\is{I didn't understand this remark. Could
  you elaborate more on this?}

To illustrate, we present in more detail the action
$\mathsf{write}\ (x : \mathsf{ptr})\ (v : A)$ (associated with
$\mathsf{priv}\ p$ concurroid), which writes $v$ into the pointer $x$,
and the action $\mathsf{trylock}$ with no arguments (associated with
$\mathsf{priv}\ p \bowtie \mathsf{spinlock}\ sl$ concurroid), which
tries to acquire the spinlock of the concurroid
$\mathsf{spinlock}\ sl$, and transfer the heap protected by the
spinlock into the ownership of $\mathsf{priv}\ p$.

Let $V = \mathsf{priv}\ p$. The action of $\mathsf{write}\ x\ v$ is
associated with the concurroid $V$, returns a value of $\mathsf{unit}$
type, and its $\mathsf{safe}$ predicate and $\mathsf{step}$ relation
are defined as follows.
\[
\begin{array}{l}
\mathsf{safe}_{\mathsf{write}\ x\ v} = 
  \{s | s \in \mathsf{coh}_V \aand \exists w:B\ldot h:\mathsf{heap}\ldot s.self = p \mapsto (x \mapsto w \oplus h)\}\\
\mathsf{step}_{\mathsf{write}\ x\ v} = 
  \{(s, s', ()) | \!\!\!\begin{array}[t]{l}
                s, s' \in \mathsf{coh}_V \aand \exists w\ h\ldot \hbox{}\\
                s.self = p \mapsto (x \mapsto w) \oplus h \aand \hbox{}\\
                s'.self = p \mapsto (x \mapsto v) \oplus h\}
             \end{array}
\end{array}
\]

The $\mathsf{write}\ x\ v$ action is safe if the pointer $x$ exists in
the private heap of the concurroid, storing some value $w$ of arbitary
type $B$. The step of the action is to replace $w$ with $v$, while
leaving the rest $h$ of the private heap unchanged. 


Let $W = \mathsf{priv}\ p \bowtie \mathsf{spinlock}\ sl$. The action
$\mathsf{trylock}$ is associated with the concurroid $W$, returns a
value of type $\mathsf{bool}$ and its $\mathsf{safe}$ predicate and
$\mathsf{step}$ relation are defined as follows.
\[
\begin{array}{l}
\mathsf{safe}_{\mathsf{trylock}} = \mathsf{coh}_W\\
\mathsf{step}_{\mathsf{trylock}} = 
  \{(s, s', r) | s, s' \in \mathsf{coh}_W \aand \exists \hL\ \hE\ \mL\ \mE\ b\ h\ldot\hbox{}\\
\quad \!\!\!\begin{array}[t]{l}
    s = p \mapsto [self=\hL,joint=(),other=\hE] \oplus \hbox{}\\
    \hphantom{s =} sl \mapsto [self=\mL, joint=lk \mapsto b \oplus h, other=\mE] \aand \hbox{}\\
    \mbox{if}\ b\ \mbox{then}\ r = \mathsf{false} \aand s' = s\\
   \mbox{else}\ \!\!\!\begin{array}[t]{l}
                 r = \mathsf{true} \aand \hbox{}\\
                 s' = p \mapsto [self=\hL \oplus h, joint=(), other=\hE] \oplus \hbox{}\\
                 \hphantom{s' =} sl \mapsto [self=\lockOwn, joint=lk \mapsto \mathsf{true}, other=\lockNown]\}
               \end{array}
    \end{array}
\end{array}
\]

The intuition is that $\mathsf{trylock}$ is a denotational encoding of
the behavior of the CAS operation. The $\mathsf{step}$ relation
specifies that $\mathsf{trylock}$ inspects the lock implemented as a
pointer $lk$. If $lk$ is $\mathsf{false}$ (representing unlocked), the
action atomically tries to change $lk$ to $\mathsf{true}$
(representing locked). The return value $r = \mathsf{true}$ indicates
succesful locking, and the output state $s'$ is changed by transfering
the ownership of the heap $h$ from the joint to the self component. If
$lk$ is initially $\mathsf{true}$, then $\mathsf{trylock}$ terminates
without changing anything.

The action is safe to perform on all coherent states of $W$, as each
such states contains the pointer $lk$ representing the lock.


\paragraph{Actions abstractly}

The components of the action $a$ associated with a concurroid $V$, and
return type $A$ satisfy the following properties.

\[
\begin{array}{rcl}
\mathsf{safe_C} & : & s \in \mathsf{safe}_a \implies s \in \mathsf{coh}_V\\
\mathsf{safe_Z} & : & s \zag t \in \mathsf{safe}_a \implies s \zig t \in \mathsf{safe}_a\\
\mathsf{step_S} & : & (s, s', v) \in \mathsf{step}_a \implies s \in \mathsf{safe}_a\\
\mathsf{step_T} & : & (s, s', v) \in \mathsf{step}_a \implies (s, s') \in \mathsf{internal}_V\\
%\mathsf{step_Z} & : & s.other = s'.other \implies \hbox{}\\
%& & \quad (s \zag t, s' \zag t, v) \in \mathsf{step}_a \implies (s \zig t, s' \zig t, v) \in \mathsf{step}_a\\
\mathsf{step_F} & : & s \zag t \in \mathsf{safe}_a \implies \hbox{}\\
& & \quad (s \zig t, s', v) \in \mathsf{step}_a \implies \hbox{}\\
& & \quad \exists s''\ldot s' = s'' \zig t \aand  (s \zag t, s'' \zag t, v) \in \mathsf{step}_a\\
\mathsf{step_E} & : &  \mathsf{valid} (\flatten s \oplus h) \implies \hbox{}\\
& &  \flatten s \oplus h = \flatten{s'} \oplus h' \implies \hbox{}\\
& &  (s, t, v) \in \mathsf{step}_a \implies (s', t', v') \in \mathsf{step}_a \implies \hbox{}\\
& &  v = v' \aand \flatten t \oplus h = \flatten{t'} \oplus h'
\end{array}
\]


The properties $\mathsf{safe_C}$, $\mathsf{step_S}$, and and
$\mathsf{step_T}$ are straightforward.

$\mathsf{safe_Z}$ states that if the action is safe in a state with a
smaller self component (because the other component is enlarged by
$t$), the action is also safe if we increase the self component by
$t$. This property corresponds to \emph{safety monotonicity} in
abstract separation logic.

The $\mathsf{step_F}$ property says that if $a$ steps in a state with
a large self component $s \zig t$, but is already safe to step in a
state with a smaller self component $s \zag t$, then the result state
and value obtained by stepping in $s \zig t$, can be obtained by
stepping in $s \zag t$, and moving $t$ afterwards. This property
corresponds to \emph{framing property} in abstract separation logic.
\footnote{In addition to $\mathsf{step_F}$, we have been proving an
  additional property for each action:
\[
\begin{array}[t]{rcl}
\mathsf{step_Z} & : & s.other = s'.other \implies \hbox{}\\
& & \quad (s \zag t, s' \zag t, v) \in \mathsf{step}_a \implies (s \zig t, s' \zig t, v) \in \mathsf{step}_a
\end{array}
\]
This property corresponds to what we have taken up calling
\emph{locality}, though it's not at all equivalent to locality in the
sense of abstract separation logic. The property is \emph{not}
provable from the rest, unless we require that each action is
deterministic (which right now seems unecessary). Thus, if we want it,
it has to be required as primitive, which is what we have done.

However, I just realized that the property is not used \emph{anywhere}
in the development. It's unused in the model, and it's unused in the
always lemmas. Thus, I'm suppressing it here.

On the other hand, the similar locality property for
\emph{transitions} is useful in the always lemmas, as alredy
commented!}

The property $\mathsf{step}_E$ shows that the behavior of the action
on the concrete input state obtained after erasing the auxiliary
fields and the logical partition, doesn't depend on the erased
auxiliary fields and the logical partition. In other words, if the
input state have \emph{compatible} erasures (that is, erasures which
are subheaps of a common heap), then executing the action in the two
states results in equal values, and final states that also have
compatible erasures. This is a standard property proved in concurrency
logics that deal with auxiliary state and
code~\cite{Owicki-Gries:CACM76,Brookes:TCS07}.\footnote{Though, these
  other papers don't deal with compatibility, as far as I can tell.}


\paragraph{Stability}

We require two further properties of actions, requiring stability of
the safety and stepping action components wrt. to the associated
concurroid $V$.\an{I separate these, because they are not actually
  implemented in Coq. However, I think that if we impose them, the
  section on logic will be easier to present.}

\[
\begin{array}{rcl}
\mathsf{safe}_{ST} & : & \mathsf{safe}_a\ i \implies (i, j) \in \mathsf{env\_steps}_V \implies \mathsf{safe}_a\ j\\
\mathsf{step}_{ST} & : & (i, j) \in \mathsf{step}_a \implies (j, k) \in \mathsf{env\_steps}_V \implies \hbox{}\\
                  &   & (i, k) \in \mathsf{step}_a
\end{array}
\]



