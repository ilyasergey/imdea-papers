\section{Language and logic}\label{sec:logic}

In this section we discuss the language and the logic, and in
particular, with the notions of injection and retraction.\is{Those
  haven't been given any intuition in the previous sections. \frownie}

In the tradition of axiomatic program logics, the language of SCLP
splits into purely functional expressions $e$ ($v$ when the expression
is a value), and commands $C$ with the effects of divergence, state,
concurrency. We also include procedures $F$, for commands with
arguments.

In the forthcoming sections, we will semantically interpret the syntax
into CiC, thus shallowly embedding it into CiC. That justifies us in
freely using all of CiC as the purely functional fragment of SCLP (in
particular, the inductive types, as well as abtract types and
predicates). We will use these additional features in our examples in
Section~\ref{sec:examples}. In this section, we confine ourselves to
treating the side-effecting fragment of commands and procedure, as it
is this fragment that's relevant for concurrency. 

The syntax is as follows. 

\[
\begin{array}{rcl}
C & ::= & x \leftarrow C_1; C_2 \mid C_1 \parallel C_2 \mid \mathsf{if}\ e\ \mathsf{then}\ C_1\ \mathsf{else}\ C_2 \mid F(e) \mid \hbox{}\\
& & \mathsf{ret}\ v \mid \mathsf{act}\ a \mid \mathsf{inject}\ S\ C \mid \mathsf{retract}\ S\ C\\
F & ::= & f \mid \mathsf{fix}\ f\ldot x\ldot C
\end{array}
\]

\paragraph{Description of commands}
Commands and procedures include primitive actions $a$, a monadic unit
$\mathsf{ret}\ v$ that returns $v$ and termintes, a monadic bind
(i.e., sequential composition) $x \leftarrow C_1; C_2$ that runs $C_1$
then substitutes its result $v_1$ for $x$ to run $C_2$ (we write $C_1;
C_2$ when $x \not\in \mathsf{FV}(C_2)$), fork-join parallel
composition $C_1 \parallel C_2$, a conditional, a procedure
application $F(e)$, a procedure variable $f$, a fixed-point construct
for recursion. In the injection construct $\mathsf{inject}\ S\ C$, and
retraction $\mathsf{retract}\ S C$, $S$ is the structure of
side-conditions.\footnote{I have to make this precise.}


\paragraph{Assertions}

We assume our assertions are \emph{higher-order logic} predicates over
states.\is{What does it mean exactly?} Thus, we do not present a
separate syntax for them, and do not give the definition of the
$\models$ relation. In fact, we use $\models$ as a notational
convenience, and, for a given state $s$, we write $s \models p$
interchangeably with $p\ s$.

We overload the standard logical connectives and quantifiers to work
over states in a pointwise manner. For example, as customary in
separation logic, given two state predicates $p$ and $q$, we write $p
\aand q$ for the predicate that is the characteristic function of the
set $\{s ~|~ s \models p \aand s \models q\}$. When we need to
differentiate, we write $\aand_p$ for the conjunction over predicates,
and keep $\aand$ for the usual conjunction of propositions, and
similarly for disjunction, implication, and quantifiers.

Given a state $s$, we write $\mathsf{this}\ s$ for the characteristic
function of $\{s' ~|~ s = s'\}$.

Given state predicates $p_1$ and $p_2$, we define $p_1 \sstar p_2$ as
follows. Given a state $s$, $s \models p_1 \sstar p_2$ iff there
exist components $lp_1$, $lp_2$, $sh$ and $ep$, such that
\[s = [self=lp_1 \otimes lp_2, joint=sh,  other=ep]\] and
\[[self = lp_1, joint=sh, other=ep \otimes lp_2] \models p_1\]
and
\[[self = lp_2, joint=sh, other=ep \otimes lp_1] \models p_2.\]


Given state predicates $p_1$ and $p_2$, we define $p_1 \odot p_2$ as
follows. Given a state $s$, $s \models p_1 \odot p_2$ iff there exists
$s_1$ and $s_2$ such that $s = s_1 \oplus s_2$ and $s_1 \models p_1$
and $s_2 \models p_2$.

One can see that $\odot$ is similar to the usual separation logic $*$,
in that it separates the states into portions which are
\emph{label}-disjoint. However, it doesn't do so for the components
that may be stored into self, other and joint portion associated with
such labels. We use $\sstar$ to perform such field-wise splitting.
\an{When working with states with only one label (e.g. the coherent
  states of the $\mathsf{priv}$ concurroid), it is $\odot$ that may be
  viewed as a generalization of separation logic $\sstar$. Do I need
  to elaborate on this?}\is{Yes, please, do. So far I miss the full
  intuition of $\odot$ defined through $\oplus$. At least, you need to
  say how to extend the PCM structure to the arbitrary states (and if
  it's indeed the case, why do you need a separate operator $\odot$
  then?)}

\paragraph{Specifications}

A command $C$ satisfies the Hoare tuple $\stconcTy{p}{C}{A}{q}{V}$ if
its effect on states respects the internal transition of the
concurroid $V$, is memory-safe when executed from a state satisfying
$p$, and concurrently with {\em any} environment that also respects
the transitions of $V$. Furthermore, if $C$ terminates, it returns a
value of type $A$ in a state satisfying $q$. $q$ may use dedicated
variable $\mathsf{res}$ of type $A$ to name the return result.

We use a {\em procedure tuple},
$\forall{x}{:}{B}\ldot\stconcTy{p}{F\,(x)}{A}{q}{V}$, to specify a
potentially recursive procedure $F$ taking an argument $x$ of type $B$
to a result of type $A$. The assertions $p$ and $q$ may depend on
$x$. We use Cartesian products $A_1\,{\times}\,A_2$ for functions with
more than one argument, but curry the function for readability.

The \SCSL judgments are {\em hypothetical} under a context $\Gamma$
that maps \emph{program variables} $x$ to their type and
\emph{procedure variables} $f$ to their specification. Each
specification is allowed to depend on the variables declared to the
left.  
%
\[
\figsize
\Gamma ::= \cdot\mid  \Gamma, \mathit{x}{:}A \mid \Gamma,\forall \mathit{x}{:} B. \stconcTy{p}{\mathit{f} (\mathit{x})}{A}{q}{V}
\]
%
$\Gamma$ does not bind logical variables. In first-order Hoare logics,
logical variables are implicitly universally quantified with global
scope. In \SCSL, we limit their scope to the Hoare tuples in which
they appear.  This is required for specifying recursive procedures,
where a logical variable may be instantiated differently in each
recursive call~\cite{Kleymann:FAC99}.  We also assume a formation
requirement on Hoare tuples $\mathsf{FLV}(p) \supseteq
\mathsf{FLV}(q)$, \ie, that all free logical variables of the
postcondition also appear in the precondition.

We use an additional hypothetical judgment $\Gamma \vdash a :
\mathsf{action}\ V\ A$ for actions to denote that $a$ is an action
associated with the concurroid $V$\footnote{$a$ may depend on the
  variabels in $\Gamma$, such as e.g. in the case of
  $\mathsf{write}\ x\ v$, where $x{:}\mathsf{ptr}$ and $v{:}A$ should
  be elements of $\Gamma$}. 

This judgment is valid if the action $a$, with its specification
components $\mathsf{safe}_a$ and $\mathsf{step}_a$ satisfies the
properties listed in Section~\ref{sec:actions}. We intend these
properties to be proved semantically in the higher-order logic over
states (the assertion logic of SCLP), so we don't present any
inference rules for this judgmnet that may streamline such reasoning.

Finally, the rules $\mathsf{Conseq}$ and $\mathsf{Action}$ use the
judgment $\Gamma \vdash (p_1, q_1) \sqsubseteq (p_2, q_2)$ which
generalizes the customary Hoare logic side conditions $p_2 \implies
p_1$ and $q_1 \implies q_2$, and which we describe in
Section~\ref{sec:model}.

\paragraph{Inference rules}

The inference rules are presented in Figure~\ref{fig:rules}. 

The rules themselves seem to be completely agnostic as to whether they
are in large-footprint or small footprint style. The only rule that's
sensitive to such issues is the rule for actions! Right now, the way
things have been presented, that rule is in \emph{large footprint}
style, but it will be fairly simple, I think, to change that.
\an{Furthermore, some rules have \emph{not} been checked in Coq, e.g.,
  frame rule (no need so far, because everything else has been large
  footprint). These are TODOs.}

We also note that the specifications are always wrt. the concurroid
$V$. The assertions will explicitly mention all the components of the
concurroid, so in that respect, one can think of the rules as being in
large footprint style. That said, there is a frame rule for the
addition of new concurroids, which is the injection rule.

\begin{figure*}[ht]
\centering
\begin{mathpar}
\figsize
\inferrule*[Right={Seq}]
 {\Gamma \vdash \stconcTy{p}{C_1}{B}{q}{V} \\ 
  \Gamma, x:B \vdash \stconcTy{[x/res]q}{C_2}{A}{r}{V} \\
  x \not\in \mathsf{FV}(r)}
 {\Gamma \vdash x\leftarrow \stconcTy{p}{C_1; C_2}{A}{r}{V}}
\and
\inferrule*[Right={Par}]
  {\Gamma \vdash \stconcTy{p_1}{C_1}{A_1}{q_1}{V} \\
   \Gamma \vdash \stconcTy{p_2}{C_2}{A_2}{q_2}{V}}
  {\Gamma \vdash \stconcTy{p_1 \sstar p_2}{C_1 \parallel C_2}{A_1 \times A_2}{[res.1/res]q_1 \sstar [res.2/res]q_2}{V}}
\and

\inferrule*[Right={Conseq}]
  {\Gamma \vdash \stconcTy{p_1}{C}{A}{q_1}{V} \\
   \Gamma \vdash (p_1, q_1) \sqsubseteq (p_2, q_2)}
  {\Gamma \vdash \stconcTy{p_2}{C}{A}{q_2}{V}}
\and
\inferrule*[Right={If}]
  {\Gamma \vdash \stconcTy{e = \mathsf{true}\aand p}{C_1}{A}{q}{V}\\
   \Gamma \vdash \stconcTy{e = \mathsf{false}\aand p}{C_2}{A}{q}{V}}
  {\Gamma \vdash \stconcTy{p}{\mathsf{if}\ e\ \mathsf{then}\ C_1\ \mathsf{else}\ C_2}{A}{q}{V}}
\and
\inferrule*[Right={Conj}]
  {\Gamma \vdash \stconcTy{p_1}{C}{A}{q_1}{V}\\
   \Gamma \vdash \stconcTy{p_2}{C}{A}{q_2}{V}}
  {\Gamma \vdash \stconcTy{p_1 \aand p_2}{C}{A}{q_1 \aand q_2}{V}}
\and
\inferrule*[Right={Exist}]
  {\Gamma \vdash \stconcTy{p}{C}{A}{q}{V}\\ 
   \alpha \not\in\mathsf{dom}\ \Gamma}
  {\Gamma \vdash \stconcTy{\exists \alpha{:}B\ldot p}{C}{A}{\exists \alpha{:}B\ldot q}{V}}
\and
\inferrule*[Right={Frame}]
  {\Gamma \vdash \stconcTy{p}{C}{A}{q}{V}}
  {\Gamma \vdash \stconcTy{p\sstar r}{C}{A}{q \sstar r}{V}}
\and
\inferrule*[Right={Hyp}]
  {\forall x{:}B\ldot \stconcTy{p}{f(x)}{A}{q}{V} \in \Gamma}
  {\Gamma \vdash \forall x{:}B\ldot \stconcTy{p}{f(x)}{A}{q}{V}}
\and
\inferrule*[Right={Fix}]
  {\Gamma, \forall x{:}B\ldot\stconcTy{p}{f(x)}{A}{q}{V}, x{:}B \vdash \stconcTy{p}{C}{A}{q}{V}}
  {\Gamma \vdash \forall x{:}B\ldot\stconcTy{p}{(\mathsf{fix}\ f\ldot x\ldot C)(x)}{A}{q}{V}}
\and
\inferrule*[Right={App}]
  {\Gamma \vdash \forall x{:}B\ldot \stconcTy{p}{F(x)}{A}{q}{V}\\
   \Gamma \vdash e : B}
  {\Gamma \vdash \stconcTy{[e/x]p}{F(e)}{A}{[e/x]q}{V}}
\and
\inferrule*[Right={Action}]
  {\Gamma \vdash a : \mathsf{action}\ V\ A\\
   \Gamma \vdash (\mathsf{safe}_a \aand \mathsf{this}\ i, \mathsf{fun}\ m\ldot (i, m, \mathsf{res}) \in \mathsf{step}_a) \sqsubseteq (p, q)}
  {\Gamma \vdash \stconcTy{p}{\mathsf{act}\ a}{A}{q}{V}}
\and
\inferrule*[Right={Inject}]
  {\Gamma \vdash \stconcTy{p}{C}{A}{q}{V}\\
    \mbox{$r$ stable under $W$}}
  {\Gamma \vdash \stconcTy{p \odot r}{\mathsf{inject}\ C}{A}{q \odot r}{V \bowtie W}}
\and
\inferrule*[Right={Retract}]
  {\Gamma \vdash \stconcTy{p \odot k}{C}{A}{q \odot k'}{V \bowtie W}\\  
    \mbox{side conditions}}
  {\Gamma \vdash \stconcTy{p}{\mathsf{retract}\ S\ \gL\ \gE\ C}{A}{q}{V}}
\end{mathpar}
\caption{SCLP inference rules.}\label{fig:rules}
\end{figure*}


\paragraph{Actions}

The construct $\mathsf{act}\ a$ coerces the action $a$ into a command.
For example, the command for reading from the pointer $x$ may be
written as:
\[
\mathsf{act}\ (\mathsf{read}\ x)
\]
As the action $a$ is specified by means of its $\mathsf{spec}_a$ and
$\mathsf{step}_a$, the $\mathsf{Action}$ rule has to transform the
specification into the form of Hoare tuples, which we use for
commands.

Thus, the rule $\mathsf{Action}$ contains a premise that the
specification composed of the $\mathsf{safe}_A$ and $\mathsf{step}_a$
relations can be precondition-weakened and postcondition-strenghtened
into the pair $(p, q)$, before $(p, q)$ may be ascribed as a
specification of $\mathsf{act}\ a$.

The premise uses $\mathsf{safe}_A$ as a main component of the
precondition to be weakened. The conjunction with $\mathsf{this}\ i$
serves to explicitly name the initial state of the command with a
fresh variable $i$. The variable $i$ is used to instantiate the
$\mathsf{step}_a$ relation in the postcondition to be strenghtened,
thus turning $\mathsf{step}_a$ into a predicate over only the ending
state only.

The rule is sound wrt. the described semantics of Hoare tuples,
because $p$ and $q$ are stable under interference from programs in the
$V$-concurroid.\an{The last stamenents looks unmotivated.}


\paragraph{Injection}

The Hoare tuple associates a command with a concurroid $V$ which
specifies the command's transitions. When programs are composed
sequentially or in parallel, the rules $\mathsf{Seq}$ and
$\mathsf{Par}$ require that the Hoare tuples of the composed programs
come with same concurroid. When this is not the case, the commands
have to be explicitly coerced to a common concurroid.

For example, the command for reading from the pointer $x$, $C =
\mathsf{act}\ (\mathsf{read}\ x)$, is associated witha the concurroid
$\mathsf{priv}\ p$. But if we want to execute this command in a
program that afterwards goes on to acquire the spinlock $sl$, then $C$
has to be coerced into the concurroid $\mathsf{priv}\ p \bowtie
\mathsf{spinlock}\ sl$.

The $\mathsf{inject}\ C$ command performs this explicit coercion.

When read bottom-up, the inference rule $\mathsf{Inject}$ lets us
focus the verification of $C$ against the concurroid $V \bowtie W$, to
the minimal concurroid $V$ that's necessary for the task. The state
belonging to the framing concurroid $W$ is not touched by $C$, and is
thus described by a predicate $r$ in both the precondition and the
postcondition of the conclusion.

Independently of $C$, the environment threads may act on the state
encompased by both $p$ and $r$. However, the interference with the
environment with the portion $p$ is accounted for in the premise
related to the program $C$. By the semantics of the Hoare tuple, the
postcondition $q$ holds no matter what interference the environment
does.

The requirement for stability of $r$ wrt. $W$ accounts for the
interference of the environemnt on the portion of the initial state
described by $r$. By choosing $r$ to be stable wrt. $W$, we make sure
that $r$ remains true no matter the steps the environment executed
concurrently with $C$.

Notice that this is the \emph{only} inference rule of SCPL where we
have to reason about stability of predicates. This in contrast to the
existing logics for shared-memory concurrency in the Rely-Guarantee
style (RGSep, SAGL, CAP, HOCAP), where the burden for reasoning about
stability, in our experience, is much more overwhelming.

\is{I think, it is a good place to say how we can simplify the
  reasoning even in this case, i.e., which lemmas are sufficient to
  prove for a particular entanglement in order to have stability (we
  also need to compare it with a classical definition by
  Vafeiadis~\cite{Vafeiadis:PHD}).}


NOTE: The assymetry of $\bowtie$ is very annoying. Ilya's found an
example in the ticketed locks module, where an action over a ticketed
lock needs to be extended to (priv + ticketedlock). I.e., it needs to
be injected from $W$ into $V \bowtie W$, which is not covered by
$\mathsf{inject}$ rule, as that rule allows only the injection into a
left component.\is{Is it a conceptual restriction? Or this is to be
  fixed soon?}

There's two ways out of this. Either (1) we inject the action by hand,
or (2) we make $\bowtie$ symmetric. But if $\bowtie$ is made
symmetric, all hell may break lose, so that doesn't sound like an
option.

The option (1), of injecting the action by hand into a larger module,
may work, but it goes against the modularity principle. Even though
the action is completely self-contained, and doesn't need private
state, we need to entangle it with private state, to make up for the
deficiency of $\bowtie$. This is clearly not satisfactory.

Yet another option may be to bake the Priv into the semantics of the
Hoare triple. Then we can argue that every action, even if in-and-off
itself doesn't deal with private state, has to be made part of some
priv-like module, so that may provide some fig-leaf for the deficiency
of $\bowtie$.

If that works (and it has to be checked), we don't really need
$\bowtie$ in any inference rules, in particular, in the rule for
injection. The concurroid annotation in the Hoare tuple is then merely
a list of concurroids. That may take some appeal away from the logic,
as the $\bowtie$ operator then merely appears in the semantics!  It
also has to be checked if the injection rule still works (though I
think it should).

\paragraph{Retraction}

This one is complicated, may require a full separate section.

The annoying part is that the rule, as implemented in the semantics,
actually doesn't use $\bowtie$ anywhere, but is very general, and
works for any two concurroids which ``refine'' one another. But, one
of those concurroids is always going to be ``priv-like'' as it
provides the private state into which the ``discharched'' concurroid's
state is collapsed. So, even though in the semantics we did much more
work and provided for a general refinement story, in practice, we're
always retracting into a priv-like concurroid, so why not have a
simpler inference rule which codifies that practice?



The rule is as follows

\[
\begin{array}[t]{c}
  {\Gamma \vdash \stconcTy{p \odot hp@sf\mapsto h \odot \pi\ g}{C}{A}{p' \odot hp@sf\mapsto h' \odot \pi\ g'}{(\mathsf{priv}\ hp \bowtie V) \bowtie W}}\\\hline
  {\Gamma \vdash \stconcTy{p \odot \phi\ g\ h}{\mathsf{retract}_{\pi, g} C}{A}{p' \odot \phi\ g'\ h'}{\mathsf{priv}\ hp \bowtie V}}
\end{array}
\]

Where $\phi\ g\ h$ is the predicate, written in ``modal'' notation
\[
\exists i{:}\mathsf{state}. \pi\ g\ i \aand hp@sf\mapsto (h \oplus \flatten {i}) \aand
 \forall s\ldot s \in \mathsf{coh}_{\mathsf{priv} hp \bowtie V} \arrow \mathsf{valid}(s \oplus i)
\]

Here $\pi$ is the decoration function which relates the auxiliary $g$
(the abstract state of the concurroid STS) with the concrete
state. It's is required to satsisfy the following properties

\[
\begin{array}[t]{l}
i \in \pi\ g \arrow i \in \mathsf{coh}\ W\\
i \in \pi\ g_1 \arrow i \in \pi\ g_2 \arrow g_1 = g_2\\
i \in \mathsf{coh}\ W \arrow \exists g, i \in \pi\ g\\
i_1 \in \pi\ g_1 \arrow i_2 \in \pi\ g_2 \arrow \mathsf{env}\ i_1 = \mathsf{env}\ i_2\\
i_1 \in \pi\ g \arrow \i_2 \in \pi\ g \arrow \flatten {i_1} \oplus h_1 = \flatten {i_2} \oplus h_2 \arrow i_1 = i_2
\end{array}
\]

The onto condition (third one) is actually superfluous and not used in
the implementation. But, if I remove it, I will almost never be able
to satisfy the precondition, for any program. Actually, it's not true;
that depends on the specific $\pi$'s.

The alternative, stronger, formulation is to take $\pi\ g$ from the
premise and put it into pre and post of the conclusion. It roughly
goes as follows, informally, modulo the notation $p(\hL)$, which I
used here to simplify things.

\begin{mathpar}
\inferrule*[Right={Retract}]
  {\Gamma \vdash \stconcTy{p(\hL) \odot \mathsf{this}\ i}{C}{A}{p'(\hL') \odot \mathsf{this}\ m'}{\mathsf{priv}\ hp \bowtie V}}
  {\Gamma \vdash \stconcTy{p(\hL \oplus \flatten i) \aand \pi\ g\ i}{\mathsf{retract}_{\pi, g} C}{A}{p'(\hL' \oplus \flatten {m'}) \aand \exists g'. \pi\ g'\ m'}{\mathsf{priv}\ hp \bowtie V}}
\end{mathpar}

Or even better, have the premise be just $\stconcTy{p}{C}{A}{q}{...}$,
and let the conclusion do all the work, but then I'll need magic
wands, and that's a bit too much.
