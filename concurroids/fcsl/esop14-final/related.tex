\section{Related Work}\label{sec:related}

\SCST builds on the previous work on subjective auxiliary state and
SCSL logic~\cite{LeyWild-Nanevski:POPL13}. The SCSL logic contained
the distinction between \self and \other views, which was essential
for compositional implementation of \emph{auxiliary state}. However,
it contained exactly one coarse-grained resource, with no ability to
create and dispose new resources. In contrast, \SCST can introduce any
number of fine-grained resources in a scoped way.

The work on Concurrent Abstract Predicates
(CAP)~\cite{DinsdaleYoung-al:ECOOP10} introduces a notion of
\emph{shared region} that serves a similar purpose as concurroids, in
that regions circumscribe a chunk of shared heap with a protocol
governing its evolution. A \emph{protocol} is defined by a set of
atomic actions, which are RG-style transitions on private state and a
region. In addition to heaps, regions may contain abstract
capabilities that identify enabled actions. Thus there is a subtle
mutual recursion in a protocol definition between an action and the
capability to perform the action. A recurring pattern for this
approach is quantification over \emph{all} possible capabilities and
placing them in a shared region, to be used up if needed in the
execution of the protocol. The CAP framework could atomically change
only one region; a restriction lifted in the recent work on
Views~\cite{DinsdaleYoung-al:POPL13} and
HOCAP~\cite{Svendsen-al:ESOP13} that introduced \emph{view shifts} to
synchronize changes in several~regions.
%
Once allocated, CAP's regions have dynamically-scoped lifetime, and
they can be disposed by a particular thread if it collects all
corresponding region's capabilities. To the best of our knowledge,
HOCAP does not allow the removal or scoped hiding of a shared~region.
%
% To the best of our knowledge, neither CAP nor HOCAP allow the removal
% or scoped hiding of a shared region.

In contrast with CAP and their successors, \SCST does not require
capabilities to perform actions, as these are naturally represented in
the \self and \other views associated with a resource (and can also be
seen as auxiliary state). Such auxiliary state is simpler than
capabilities; it is not subject to ownership transfer, and there is no
need to quantify over all capabilities. In our experience, this
simplicity extends to the specification of invariants and transitions,
and to the proofs of stability. In \SCST, synchronizing changes over a
number of concurroids is achieved directly at the level of transitions
by means of entanglement, and at the level of programs by allowing
actions to be defined over any concurroid, including entangled
ones. Thus, no view shifts are required.
%
The burden of stability proofs is further reduced in \SCST by
formulating private heaps as a separate concurroid that one may, but
need not, entangle with. Thus, when an action manipulates only the
internal state of a resource, the attendant stability proofs can
ignore private heaps, \eg, the \textsf{take} action of a ticketed
lock~\cite{DinsdaleYoung-al:ECOOP10,fcsl-coqscripts}.
%
Moreover, the communication in \SCST makes it possible for concurroids
to pass heaps between each other directly, rather than going through
private state. While the current paper does not present examples that
exploit this ability, we have found it useful when verifying in \SCST
a more advanced example of readers-writers, which we will present in
future~work.

CaReSL~\cite{Turon-al:ICFP13} uses the same notion of shared region as
CAP, though it specifies the transitions in a manner closer to \SCST,
namely by means of STS's. CaReSL does not directly provide subjective
\self and \other views of a resource, but it provides a notion of
\emph{tokens}, whose ownership is exchanged between a thread and its
environment. CaReSL assertions explicitly allow statements only about
self-owned tokens, not \other-owned ones. Thus, reasoning about the
lack of logical changes to environment-owned data has to be encoded
with a level of indirection, potentially quantifying over all tokens,
similar to CAP's quantification over capabilities. A frequent side
condition in CaReSL rules is that various assertions are
\emph{token-pure}, which does not have a direct correspondent in
\SCST.
%
Similar to CAP, CaReSL currently allows actions that work over only a
single region, and will require an extension akin to view shifts to
enable synchronized updates. 
%
CaReSL does not consider removal or scoped hiding of shared regions,
although it can be emulated by introducing an empty ``final'' protocol
state.
%
Instead of stability checks in \SCST, in CaReSL one may stabilize
assertions by composing them with environment stepping. In our
experience, this does not change the proofs: the same obligations
reappear in proofs out of stabilized hypotheses.
%
On the other hand, CaReSL can reason about fine-grained data structure
by means of refinement (a generalization of linearizability). \SCST
supports higher-order functions by means of shallow embedding into
CiC~\cite{Bertot-Casteran:04,coq-team}, but we have not considered
linearizability so far, which is future work.

Feng's Local Rely-Guarantee (LRG)~\cite{Feng:POPL09} is, to the best
of our knowledge, the first work that reconciled fine-grained
reasoning in the style of RG with framing and hiding at the level of
transitions (similar to our \textsc{Inject} and \textsc{Hide}). We
differ from LRG in that we introduce communication and subjectivity
into the mix; thus our injection and hiding rules take \self and
\other views into account. The latter are a compositional form of
auxiliary state, whereas LRG in practice has to use the classical,
non-compositional form of auxiliary
state~\cite{Owicki-Gries:CACM76,LeyWild-Nanevski:POPL13}.
%
% Additionally, \SCST parallel composition rule is in the style of
% CSL, rather than RG.
