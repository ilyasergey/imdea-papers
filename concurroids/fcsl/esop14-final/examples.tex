\section{Examples}
\label{sec:examples}

In this section we present a number of examples of reasoning in the
framework of Concurrent Subjective Protocol Logic.

% \subsection{CAS lock}
% \label{sec:cas-lock}

% \is{Do we need it here?}


\subsection{Ticketed lock}
\label{sec:ticketed-lock}

In this section, we construct from scratch a concurroid for another
famous locking algorithm: a ticketed lock\footnote{sometimes also
  referred to as a version of Lamport's \emph{Bakery
    Algorithm}~\cite[Section 2.6]{Herlihy-Shavit:08}}. Unlike the CAS-based
spin lock, this algorithms, although simple, provides fairness
guarantees for competing threads. We start by giving a simplified
implementation of the algorithm and building the intuition about the
\emph{shape} of the concurroid, implementing its protocol. 

\paragraph{Reference implementation}
\label{sec:refer-impl}

As a reference implementation of the ticketed lock, we use the one
from the work on Concurrent Abstract
Predicates~\cite{DinsdaleYoung-al:ECOOP10}. 

\begin{code}
 lock = {                       unlock = {
   n := INCRNXT;                INCROWN;         
   while (!TRY(n)) skip;       }
 }  
\end{code} 

The three \emph{atomic} actions \texttt{INCRNXT}, \texttt{TRY(n)}
and \texttt{INCROWN} have the following operational meaning:

\begin{code}
 INCRNXT  = { tmp := next; next++; return tmp; }
 TRY(n)     = { return (n = owner); }
 INCROWN = { owner++; }  
\end{code}

As a minor difference, we assume the pointer to the lock structure
being fixed and omit it from the code, referring instead to the two
its essential fields: \texttt{owner} and \texttt{next}. These two
fields are natural numbers, describing the state of the ticketed lock
system. The intuition is similar to the organisation of the service in
a bakery: \texttt{next} corresponds to the current ticket available at
the dispenser (but not yet taken!), whereas \texttt{owner} indicates a
ticket number, an owner of which is currently being served (or just
about to be). Hence, the \texttt{lock} procedure corresponds to
picking a ticked and an attempt to be served, whereas \texttt{unlock}
corresponds to throwing the ticket away, so the next client could have
a service (if there is one in a queue).
%
We also deliberately omit a procedure for \emph{making} the lock, as
it can be naturally implemented and verified using the
\emph{injection} mechanism, described above, so we postpone this part
of the implementation till the end of this section. \is{Don't forget
  to do it!}

\paragraph{Intuition and the structure of the concurroid}

The intuition for the construction of a ticketed lock concurroid can
be built from the idea of \emph{splitting} the resources in the spirit
of the Subjective Concurrent Separation
Logic~\cite{LeyWild-Nanevski:POPL13}. What are the primary resources
that a ticketed lock deals with? Those are tickets, of course! Some of
them can be taken by a current thread, some by others. If we make an
assumption that each ticket is thrown away right after its owner has
been served, we will see a simple invariant: all tickets in the system
are from the range $[n_1, n_2)$, where $n_1$ and $n_2$ are natural
numbers, corresponding to the values of \texttt{owner} (currently
being served or about to be) and \texttt{next} (next ticked to be
issued), hence non-inclusion on the right side of the range.

The concurroid for the ticketed lock is, therefore, is defined by the
following state (omitting the labels \emph{self}, \emph{joint} and
\emph{other} for the sake of brevity):
%
\[
tl \mapsto [(\fs, \gs), \owner \mapsto n_1 \oplus \nxt \mapsto n_2 \oplus \mr
\oplus \hr, (\fo, \go)]
\]
%
Similarly to the spin lock, the joint component containing the protected
shared heap $\hr$ and also the two pointers $\owner$ and $\nxt$
describing the state of the dispenser and the queue. Moreover, the
joint state features a \emph{ghost} element $\mr$, which is used to
indicate whether the lock is currently in the \emph{unlocked} or
\emph{locked} state, i.e., whether it \emph{owns} or not its protected
heap $\hr$. 

The \emph{self} and \emph{other} components have the same structure
and are pairs of two PCMs: the PCM of \emph{finite sets} of natural
numbers, accounting to the tickets ($\fs$ and $\fo$) and a generic PCM
accounting in order to the invariant associated with this particular
lock (see~\cite{LeyWild-Nanevski:POPL13} for more details). Using
finite sets of natural numbers is very natural. Essentially $\fs \cup
\fo$ corresponds to \emph{all} tickets now being issued by the
dispenser. Of course, in order to make it a real PCM, we need to
introduce some restriction to the $\oplus$ operation, namely 
%
\[
\mathsf{f}_1 \oplus \mathsf{f}_2 = 
\left\{
\begin{array}{cl}
\mathsf{f}_1 \cup \mathsf{f}_2  & \text{if }  \mathsf{f}_1 \cap \mathsf{f}_2 = \emptyset \\
\bot & \text{otherwise}
\end{array}
\right.
\]

The \emph{ghost} element of the joined component $\mr \in
\set{\lockOwn, \lockNown}$ should be familiar from the development of
the CAS-based spin locks. 

\paragraph{Coherence of the ticketed lock}

The coherence predicate for the ticketed lock should define what the
valid state of the system are. It is also parametrized by the
invariant $I$ over a protected heap.  The definition of the coherence
predicate (Figure~\ref{fig:tlock-trans}) is intuitive. First, at any time the last ticked served (or
being served) should be less or equal to the next ticket available at
the dispenser ($n_1 \le n_2$). In fact, the equality means that there
is currently \emph{no} clients being served or waiting for the
service, and the lock is just idling. In particular, this is the case
at the very beginning of the work. Second, all current tickets are
captured by the union $\fs \oplus \fo$ (which is empty if $n_1 =
n_2$). Finally, the system can be in one of the three \emph{disjoint}
states: $\locked$, $\transit$ or $\idle$. The first one means that
some client with a ticket $n_1$ is being correctly served at the
moment; the second means that there are clients in the system, but no
service is now being performed (so-called ``transition state''); the
third one indicates exactly the situation that the module is idling:
there are no clients being served or waiting for the service. The
disjointness of these three states is trivial to check.

\paragraph{Transitions}
\label{sec:transitions}


\begin{figure*}
\small

\fbox{Coherence}
\vspace{-12pt}

\[
\begin{array}{l}
\coh_{\tlock~\tl} = 
\left\{
tl \mapsto [(\fs, \gs), \owner \mapsto n_1 \oplus \nxt \mapsto n_2
\oplus \mr \oplus \hr, (\fo, \go)] 
\left|
\begin{array}{lc}
n_1 \le n_2 & \wedge  \\
\fs \oplus \fo = \set{n ~|~ n_1 \le n < n_2}  & \wedge  \\
\either (\fs \oplus \fo, \gs \oplus \go, \mr, n_1, n_2, \hr) 
\end{array} 
\right. \right\} 
\end{array}
\]

where
\vspace{-18pt}

\[
\begin{array}{ll}
\begin{array}{l}
\either(\textsf{f}, \textsf{g}, \textsf{m}, n_1, n_2, h)  = 
\left(
\begin{array}{ll}
\locked(\textsf{f}, n_1, \textsf{m}, h) & \vee \\
\transit(n_1, n_2, \textsf{g}, \textsf{m}, h) & \vee  \\
\idle(n_1, n_2, \textsf{g}, \textsf{m}, h) 
\end{array}
\right)
\end{array}
&
\transit(n_1, n_2, \textsf{g}, \textsf{m}, h) =
\left(
\begin{array}{l}
n_1 < n_2 ~\wedge~ n_1 \in \mathsf{f} ~\wedge~ \\
\textsf{m} = \lockNown ~\wedge~ I(\mathsf{g}, h)
\end{array}
\right)
\\ 
\\
\locked(\textsf{f}, n_1, \textsf{m}, h) =
(n_1 \in \mathsf{f} ~\wedge~ h = \emp ~\wedge~ \textsf{m} = \lockOwn)
&
\idle(n_1, n_2, \textsf{g}, \textsf{m}, h) =
(n_1 = n_2 ~\wedge~ \textsf{m} = \lockNown ~\wedge~ I(\mathsf{g}, h))
\end{array}
\]

\line(1,0){500}
\vspace{5pt}

\fbox{Transitions}

\vspace{-17pt}

\[
\begin{array}{lcl}
\internal_{\tlock~\tl} & = &
\left\{ 
(s, s') ~\left|~ 
\begin{array}{ll}
    s = tl \mapsto [(\fs, \gs), \owner \mapsto n_1 \oplus \nxt
   \mapsto n_2 \oplus \mr \oplus \hr, (\fo, \go)] & \wedge \\
    s' = tl \mapsto [(\fs, \gs), \owner \mapsto n_1 \oplus \nxt
   \mapsto n'_2 \oplus \mr \oplus \hr, (\fo, \go)]  & \wedge \\
    s \in \coh_{\tlock~\tl} ~\wedge~ s' \in \coh_{\tlock~\tl}
   ~\wedge~ \gbm{ n_2 \le n'_2 }
\end{array} \right. \right\}
\\ \\
\acquire_{\tlock~\tl} & = &
\left\{ 
(s, s') ~\left|~ 
\begin{array}{ll}
    s = tl \mapsto [(\fs, \gs), \owner \mapsto n_1 \oplus \nxt
   \mapsto n_2 \oplus \gbm{\lockOwn}, (\fo, \go)] & \wedge \\
    s' = tl \mapsto [(\gbm{ \fs \setminus \set{n_1}}, \gs'), \owner \mapsto
    \gbm{ n_1 + 1 } \oplus \nxt
   \mapsto n'_2 \oplus \gbm{\lockNown} \oplus \gbm{\hr}, (\fo, \go)]  & \wedge \\
    s \in \coh_{\tlock~\tl} ~\wedge~ \gbm{ n_1 \in \fs } ~\wedge~
    \gbm{ (\gs' \oplus \go, \hr) }
\end{array}  
\right. \right\}
\\ \\
\release_{\tlock~\tl} & = &
\left\{ 
(s, s') ~\left|~ 
\begin{array}{ll}
    s = tl \mapsto [(\fs, \gs), \owner \mapsto n_1 \oplus \nxt
   \mapsto n_2 \oplus \gbm{\lockNown} \oplus \gbm{\hr}, (\fo, \go)] & \wedge \\
    s' = tl \mapsto [(\fs, \gs), \owner \mapsto n_1 \oplus \nxt
   \mapsto n'_2 \oplus \gbm{\lockOwn}, (\fo, \go)]  & \wedge \\
    s \in \coh_{\tlock~\tl} ~\wedge~ \gbm{ n_1 \in \fs }
\end{array}  
\right. \right\}
\end{array}
\]

\line(1,0){500}
\vspace{5pt}

\fbox{Atomic actions}
\vspace{-20pt}

\[
\begin{array}{l}
\safe_{\incrnxt} = \set{s ~|~ s \in \coh_{\tlock~\tl}}
\quad\quad
\step_{\incrnxt} = \left\{ (s, s', n_2) \left|
\begin{array}{ll}
s \in \coh_{\tlock~\tl} \wedge \exists n_1, \fs, \fo, \gs, \go, \hr, \mr. &  \\
s.\joint = \tl \mapsto \owner \mapsto n_1 \oplus \nxt  \mapsto \gbm{n_2} \oplus \mr \oplus \hr &  \wedge \\
s.\joint = \tl \mapsto \owner \mapsto n_1 \oplus \nxt  \mapsto \gbm{n_2 + 1} \oplus \mr \oplus \hr &  \wedge \\   
s.\self = \tl \mapsto (\gbm{\fs}, \gs) \wedge s'.\self = \tl \mapsto (\gbm{\fs \oplus \set{n_2}}, \gs) &  
\end{array}
\right.\right\}
\\ 
\\
\begin{array}{l}
\safe_{\incrown} = \left\{s ~\left|~ 
  \begin{array}{ll}
    s \in \coh_{(\priv~p) \bowtie (\tlock~\tl)} ~\wedge~ \exists n_1,
    n_2, \fs, \gs, \hs, \ho, \hr.~~\gbm{n_1 \in \fs} & \wedge \\
    s = p \mapsto [\hs \gbm{\oplus \hr}, (), \ho] \oplus \tl \mapsto [(\fs,
    \gs), \owner \mapsto n_1 \oplus \nxt  \mapsto n_2 \oplus
    \gbm{\lockOwn} ,(\fo,\go)]  & \wedge \\     
    \gbm{\valid(\owner \mapsto (n_1 + 1) \oplus \nxt  \mapsto n_2
      \oplus \hr)} \wedge \gbm{I(\gs \oplus \go, \hr)}
  \end{array} 
\right. \right\}
\\ 
\\
\step_{\incrown} = \left\{(s, s', ()) ~\left|~ 
  \begin{array}{ll}
    s \in \coh_{(\priv~p) \bowtie (\tlock~\tl)} ~\wedge~ \exists n_1,
    n_2, \fs, \fo, \gs, \go, \hs, \ho, \hr.~~\gbm{n_1 \in \fs} & \wedge \\
    s = p \mapsto [\hs \gbm{\oplus \hr}, (), \ho] \oplus \tl \mapsto [(\fs,
    \gs), \owner \mapsto n_1 \oplus \nxt  \mapsto n_2 \oplus
    \gbm{\lockOwn} ,(\fo,\go)]  & \wedge \\     
    s' =  p \mapsto [\hs, (), \ho] \oplus \tl \mapsto \left[(\gbm{\fs \setminus \set{n_1}},\gs),
      \begin{array}{l}
       \owner \mapsto \gbm{n_1 + 1} \oplus \\
       \nxt  \mapsto n_2 \oplus \gbm{\lockNown \oplus \hr}
      \end{array}
    , (\fo,\go) \right]
& \wedge \\     
    \gbm{\valid(\owner \mapsto (n_1 + 1) \oplus \nxt  \mapsto n_2
      \oplus \hr)} \wedge \gbm{I(\gs \oplus \go, \hr)}
  \end{array}
\right. \right\}
\end{array}
\\
\\
\begin{array}{l}
\safe_{\tryn~n} = \left\{s ~\left|~ 
  \begin{array}{ll}
    s \in \coh_{(\priv~p) \bowtie (\tlock~\tl)} ~\wedge~ \exists n_1, n_2,
    \fs, \fo, \gs, \go, \hs, \ho, \hr.~~\gbm{n \in \fs} & \wedge \\
    s = p \mapsto [\hs, (), \ho] \oplus \tl \mapsto [(\fs,
    \gs), \owner \mapsto n_1 \oplus \nxt  \mapsto n_2 \oplus \mr
    \oplus \hr ,(\fo,\go)]  
  \end{array} 
\right. \right\}
\\
\\
\step_{\tryn~n} = \left\{(s, s', \res) ~\left|~ 
  \begin{array}{ll}
    s \in \coh_{(\priv~p) \bowtie (\tlock~\tl)} ~\wedge~ \exists n_1, n_2,
    \fs, \fo, \gs, \go, \hs, \ho, \hr.~~\gbm{n \in \fs} & \wedge \\
    s = p \mapsto [\hs, (), \ho] \oplus \tl \mapsto [(\fs, \gs),
    \owner \mapsto n_1 \oplus \nxt  \mapsto n_2 \oplus \mr \oplus \hr
    ,(\fo,\go)]   & \wedge \\
    (s', \res) =
    \left\{\begin{array}{l}
       (p \mapsto [\hs \oplus \hr, (), \ho] \oplus \tl \mapsto [(\fs, \gs), \owner
        \mapsto n_1 \oplus \nxt  \mapsto n_2 \oplus \gbm{\lockOwn}
        ,(\fo,\go)], \True)   \\   
        \quad\quad \text{if } \gbm{\mr = \lockNown \wedge n = n_1} \\
        (p \mapsto [\hs, (), \ho] \oplus \tl \mapsto [(\fs, \gs),
        \owner \mapsto n_1 \oplus \nxt  \mapsto n_2 \oplus \mr
        \oplus \hr,(\fo,\go)], \False)  \\
        \quad\quad \text{otherwise} 
    \end{array}\right.
  \end{array} 
\right. \right\}
\end{array}
\end{array}
\]




\caption{Ticketed lock concurroid: coherence predicate, transitions
  and actions. The essential conditions are highlighted by grey
  boxes. }
\label{fig:tlock-trans}
\end{figure*}

The internal, acquire and release transition of the ticketed lock
concurroid are presented in the middle part of
Figure~\ref{fig:tlock-trans}.

The internal transition relation accounts to the changes of the
dispenser: the number of the next ticket to be issued ($n_2$) may only
grow monotonically, hence the restriction on $n_2$ and $n'_2$. Also,
the initial and the resulting state should conform to the coherence
predicate.\footnote{The requirement $s' \in \coh_{\tlock~\tl}$ is a
  design decision: we could have made it deducible from the structure
  of $s'$, but it would needlessly complicate the definition of the
  transition. }
%
The acquire relation is more interesting, as it corresponds to
\emph{releasing} the lock by a client. Following the intuition, this
transition happens simultaneously with incrementing the value of the
$\owner$ owner cell, so the relation captures the fact that the served
ticked is ``thrown away'' by updating the appropriate ghosts. It also
requires the ticket $n_1$ to be present in $\fs$, to justify the fact
of ``throwing away''. Naturally, the $\mr$ token changes from
$\lockOwn$ to $\lockNown$, the heap $\hr$ appears in the resulting
state, and the invariant $I$ is checked for the acquired heap and the
augmented ghosts $\gs' \oplus \go$, which is standard for
Owicki-Gries-style reasoning about critical sections. The coherence of
$s'$ is omitted, as it can be inferred.
%
Finally, the release transition justifies the ownership transfer over
the heap to the client if the current ticket to be served $n_1$ is in
its local component.

\paragraph{Designing actions}
\label{sec:designing-actions}

The way we designed transitions of the concurroid makes it easy to
figure out the axioms for the atomic actions.  As is follows from the
reference implementation at the beginning of the section, there should
be three actions: for taking the lock, trying to acquire the resource
and releasing the lock. 

A major yet, we believe, justified design restriction is that
\emph{all} of the actions should look like primitive \emph{read},
\emph{write} of \emph{CAS}. Although our framework does not enforce
this restriction, we follow it faithfully. The reason for this is
clear: the listed primitives are well-understood and their power for
establishing a consensus is known~\cite[Chapter
5]{Herlihy-Shavit:08}. Moreover, this level of atomicity makes them
easy to implement in hardware.\footnote{ The recent work on Concurrent
  Abstract Predicate and its descendants
  ~\cite{DinsdaleYoung-al:ECOOP10,Svendsen-al:ESOP13} do not
  distinguish specific \emph{atomic} actions, allowing instead make
  any command atomic by indicating that syntactically and proving the
  stability with respect to the interference right in place. We
  consider this approach indiscreet and cheating, as it, in fact,
  equivalent to defining locks in terms of \emph{other} (implicit)
  locks!  }

Another important thing to note is the way the atomic actions are
going to be used. Of course, we are going to employ them to implement
the reference algorithm, but applications are possible. However, in
any case, the actions will preserved the coherent structure of the
concurroid and respect the transitions.

For instance, one can think of an ``abusive'' lock algorithm that
fetches several tickets consequently (by running \texttt{INCRNXT}) and
then distributes them among its children.\footnote{One of the authors
  witnessed this situation in a real bakery.} In fact, this scenario
is perfectly valid for the concurroid, and we should design the
actions in a way that make it possible, too.

What about ``forging'' the tickets? What is one tries to be served
without actually having a ticket? Thinking of this phenomenon in terms
of the formal program verification gives an understanding that the
precondition for the atomic action \texttt{TRY(n)} should check that
the ticket number $n$ is actually in the possession of a thread
(client) trying to be served (and this is what the
$\release_{\tlock~\tl}$ transition accounts for, too).

Actions are defined as relations of the form $\set{(s, s', \res)
  ~|~ \ldots}$, where $s$ and $s'$ are initial and final states and
$\res$ is a result of an action.
%
The action $\incrnxt$ (Figure~\ref{fig:tlock-trans}) just increments
$n_2$ and update the appropriate map $\fs$ in the \emph{self} part of
the state. Its stepping relation is naturally subsumed by the internal
transition of the concurroid $\tlock~\tl$ and its precondition is just
coherence. The action $\incrown$ is defined on the entanglement of the
modules $\priv~p$ and $\tlock~\tl$. It increments the $\owner$ cell,
but its specification is more subtle, as it essentially described the
transfer of the ownership over the heaplet $\hr$ back to the ticketed
lock concurroid. Naturally, it ensures that the current ticket $n_1$
is in the local set $\fs$ of the thread as well as the invariant
$I(\gs \oplus \go, \hr)$ being restored. The action $\tryn~n$ is the
most interesting one. Its precondition do not restrict $\mr$ or $\hr$,
but require that $n \in \fs$, accounting to the fact that one can
\emph{try to get a service} with a ticket $n$ only when possessing
such a ticket. It works like a CAS on a two-field cell $(\mr, \owner)$
The stepping relation may have two outcomes:

\begin{enumerate}
\item $\mr = \lockNown \wedge n = n_1$ indicates exactly  the situation
  when the resource $\hr$ is going to be acquired by the owner of the
  ticket $n$, so the heap is transferred to $(\priv~p)$ and the result
  is $\True$;
\item Otherwise, it's either the very same thread trying to acquire
  the resource again (that is $\mr = \lockOwn\ wedge n = n_1$), which
  is prohibited, or \emph{another} thread is being or about to be
  served (that is, $n \neq n_1$). In both cases nothing is transferred
  to $(\priv~p)$ and the result is $\False$.
\end{enumerate}

It is not that difficult to check that $\incrnxt$ is a subset of the
internal transitions of $(\tlock~\tl)$, whereas $\incrown$ and
$\tryn~n$ fit to the internal transitions of $(\priv~p) \bowtie
(\tlock~\tl)$. Also, one can check that all $\step_{-}$
specifications are immune to \emph{erasure}, i.e., they are not
affected operationally by different values of the \emph{ghost}
elements, such as $\fs$ or $\gs$: those are only used in preconditions
in order to check that the operation is \emph{safe} to run.

\paragraph{Allocating and disposing a lock}
\label{sec:alloc-disp-lock}

\is{Elaborate on the injection rule}

\subsection{Readers-writers}
\label{sec:readers-writers}

\is{Only sketch the protocol, if enough space. }

