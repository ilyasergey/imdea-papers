\section{The rest}\label{sec:concurroids}
%\an{Heh, are the last two names foreboding?}

Notes for revision (but there's too much material here):
\begin{enumerate}
\item Finite maps, heaps, states, PCMs, state, zipping, zigging,
  zagging, erasure. Broccoli definition of $\jpts$, $\spts$, $\opts$,
  $\lsep$, $\ssep$.
\item Concurroids abstractly as n-tuples: properties of coherence predicates and
  transitions. Motivation for the various properties.
\item Examples: concurroid for private state, and spin locks: state
  and transitions formally.
\item Entanglement. The properties for injection rule to be valid. The
  CSL concurroid as the entanglement $P \entangle (L_1 \apart L_2
  \cdots \apart L_n)$.
\item Actions. List the properties, and implement CAS and unlock.
\item Language and logic. The rules. Not much discussion about rules
  already covered in overview. Rather, just point out how they
  correspond to CSL. I.e., the hiding rule gives us the implementation
  of let-resource, but endowed with auxiliary state.
\item Concurroid for ticketed locks.
\end{enumerate}

How should we refer to a pair of acquire-release transitions? It can
be a channel, or better yet, a wire, since it may be dangling.

\paragraph{Main mathematical concepts}

We use finite maps in many different guises in the formulation of
SCLP, so we introduce some common notation for them.

\emph{Heaps} are finite maps from non-null pointers to values of an
arbitrary type. \emph{PCM components} are finite maps from labels to
values of an arbitrary partial commutative monoid (PCM). \emph{Type
  components} are finite maps from labels to values of an arbitrary
type. Pointers and labels are isomorphic to natural numbers.

We overload the notation $x \mapsto v$ to denote a heap (resp. pcm or
type component), containing a single pointer $x$ (resp. label x),
storing the value $v$. If we want to make the type $A$ of $v$
explicit, we write $x \mapsto_A v$. Finite maps can be nested in a
stratified manner, e.g., we can have a component storing a heap,
\[
\mathsf{label} \mapsto_{\mathsf{heap}} (x \mapsto_{\mathsf{nat}} 3),\]
%
but heaps can't store components. We use $\mathsf{empty}$ for the
empty finite map/heap/component. Given a finite map $f$, we write
$\mathsf{dom}\ f$ for the set of arguments on which $f$ is defined.


\paragraph{PCMs}
A PCM consists of a type $U$, and a partial binary operation $\oplus$
on $U$ with a neutral element $1_U$. The operation $\oplus$ is
commutative and associative. The PCM also contains a boolean predicate
$\mathsf{valid}$ on $U$, which selects a subset of $U$. All the
elements out of that subset are considered \emph{undefined}. The
partiality of $\oplus$ is modeled by having it return an undefined
element.

Finite maps (with an addition of an undefined element) form a PCM with
the operation of \emph{disjoint union}, which is undefined if the maps
overlap on some argument, and with $\mathsf{empty}$ as the unit
element.

\paragraph{Subjective state and joining}

The state of a concurroid is divided into three parts, which we call
self, joint, and other, part.

The intuition is that when a thread $t$ is ascribed a concurroid as
its specification, the self part of the concurroid state is treated by
$t$ as a chunk of private state. Other is the chunk state that the
thread can't modify, as it is private to the threads running in
parallel with $t$. Joint is the state that is shared by $t$ and
other. Thus, the specification via concurroids are always specific to
the view-point of the thread being specified, and we will frequently
refer to this thread as ``self'', overloading the terminoliogy with
state.  The totality of all threads running in parallel with the
``self'' thread, is called ``other'', again, overloading the
terminology.

We refer to such an approach with 3-way split of the state, and the
thread-relative style of specification in terms of ``self'' and
``other'' as \emph{subjectivity}, following in the style of Subjective
Concurrent Separation
Logic~\cite{LeyWild-Nanevski:POPL13}.\footnote{The existing approaches
  to concurrency logics divide the state into only private and shared
  portion, and ignore the other portion, as the thread $t$ can't
  modify it. In SCPL we expose the other part. As argued by the
  previous work on SCSL, the other part is important for specifying
  the global invariants of the state. But this point is better made in
  the related work section.}

Self and Other are PCM components, and Joint is a type component.
Self and Other are PCM components, because the contents of their
labels will often have to be \emph{joined} by $\oplus$ in order to
describe the global state invariants. We won't need such operation
with the Joint component, so we don't impose that its labels must
store values that belong to a PCM.

We use the following notation for the describing the 3 components of a
state.
\[
\begin{array}[t]{l}
[self = \mathsf{label}_{11} \mapsto x_{11}, \mathsf{label}_{12} \mapsto x_{12}, \ldots\\ 
 joint = \mathsf{label}_{21} \mapsto x_{21}, \mathsf{label}_{22} \mapsto x_{22}, \ldots\\
 other = \mathsf{label}_{31} \mapsto x_{31}, \mathsf{label}_{32} \mapsto x_{32}, \ldots]
\end{array}
\]
where all the labels may be different. 

When the components contain a unique label, we use
\[
\mathsf{label} \mapsto [self = x, joint = y, other = z]
\]
to abbreviate
\[
[self = \mathsf{label} \mapsto x, joint = \mathsf{label} \mapsto y, other = {label} \mapsto z]
\]

We use special notation for states where one component contains a
single label, but others are empty.

\[
\begin{array}{rcl}
label@self \mapsto x & = & [self = label \mapsto x, joint=empty, other=empty]\\
label@joint \mapsto x & = & [self = empty, joint=label \mapsto x, other=empty]\\
label@other \mapsto x & = & [self = empty, joint=empty, other=label \mapsto x]
\end{array}
\]


As states are triples of finite maps, they form a PCM with the
operation of \emph{pointwise} disjoint union. We are thus justified in
denoting this operation with the PCM notation $\oplus$. Similarly, we
use $1_\mathsf{state}$ (or simply $1$ when the PCM is determined from
the context), to denote the state consisting of three empty finite
maps.

Further notation: Given a state $s$, we write $s.self$, $s.joint$ and
$s.other$ for the three components of $s$.


\paragraph{Zipping, zigging, zagging}

In addition to $\oplus$ operation on PCM-components, which joins up
the finite maps disjointly, we need an operation $\otimes$ (pronounced
zip) which joins the \emph{contents} of equally named labels.

\[
\begin{array}{rcl}
\mathsf{empty} \otimes \mathsf{empty} & = & \mathsf{empty}\\
(\mathsf{label} \mapsto_U v_1), t_1 \otimes (\mathsf{label} \mapsto_U v_2), t_2 & = & 
   (\mathsf{label} \mapsto v_1 \oplus v_2), t_1 \otimes t_2
\end{array}
\]

Notice that $t_1 \otimes t_2$ may be undefined if $t_1$ and $t_2$
contain different labels, or if some label contains values of
different types in $t_1$ and $t_2$.

Given a state $s = [self=t_S, joint=t_J, other=t_O]$ and a component
$t$, we define \emph{zigging} $s \zig t$ and \emph{zagging}
$s \zag t$ of $s$ and $t$, as follows
\[
\begin{array}{rcl}
s \zig t = [self = t_s \otimes t, joint=t_J, other=t_O]\\
s \zag t = [self = t_s, joint=t_J, other=t_O \otimes t]\\
\end{array}
\]


\paragraph{Erasure}

The split into the various components, and label fields within them is
a logical representation of the actual physical state in which the
programs are executed. The logical representation contains much more
information than needed for execution. For example, some labels may
store values that correspond to \emph{auxiliary} state that should be
erased before execution. We assume that the labels storing values of
non-heap types are auxiliary, and introduce \emph{erasure} operators
to remove them: ${\flatten -}_\mathsf{comp}$ which works on components
(pcm or type ones), and ${\flatten -}_{\mathsf{state}}$ which works on
states.
\[
\begin{array}{rcl}
\flatten{\mathsf{empty}}_{\mathsf{comp}} & = & \mathsf{empty}_\mathsf{heap}\\
\flatten{\mathsf{label}\mapsto_A v, t}_{\mathsf{comp}} & = & 
  (\mbox{if} A = \mathsf{heap}\ \mbox{then}\ v\ \mbox{else}\ \mathsf{empty}_\mathsf{heap}) \oplus {\flatten t}_{\mathsf{comp}}
\end{array}
\]

\[
\flatten{[self=t_1, joint=t_2, other=t_3]}_{\mathsf{state}} = \flatten{t_1}_{\mathsf{comp}} \oplus \flatten{t_2}_{\mathsf{comp}} \oplus \flatten{t_3}_{\mathsf{comp}}
\]
When there is no confusion, we will simply write $\flatten -$ without
the index.

\begin{lemma}\label{lemma:join-valid}
 Let $s_1, s_2$ be either components or states. Then: $\flatten{s_1
   \oplus s_2} = \flatten{s_1} \oplus \flatten{s_2}$ if
 $\mathsf{valid} (s_1 \oplus s_2)$, and is undefined otherwise.
\end{lemma}



We proceed to illustrate the subjective division of state, by
presenting two basic concurroids: $\mathsf{priv}$ for private state
only, and $\mathsf{spinlock}$ for spinlocks in the style of CSL (see
some figure).

Both concurroids are indexed by a label which identifies their
individual states, in a larger, compounded state. Thus, for example,
we can have concurroids $\mathsf{spinlock}\ l_1$ and
$\mathsf{spinlock}\ l_2$, to reprent two different spinlocks.

\paragraph{Private state}

The concurroid $\mathsf{priv}$ for private state (with the label $p$),
splits the state as follows:
\[
p \mapsto [self = \hL, joint = (), other = \hE]
\]

The self and other components are two heaps $\hL$ and $\hE$, and the
joint component is the (unique) value of the unit type, symbolizing
that the concurroid doesn't deal with shared state. The heap $\hL$ is
the local heap for threads specified by $\mathsf{priv}$, and $\hE$ is
the heap belonging to the environment.

One condition on the state space of the $\mathsf{priv}$ concurroid is
that $\hE$ and $\hL$ are disjoint heaps, i.e., that no location can be
private to two different concurrent threads at the same time. Using
the notation for PCMs, we write this condition as
$\mathsf{valid}\ (\hL \oplus \hE)$.

We denote the set of states of a concurroid by $\mathsf{coh}$ (for
\emph{coherent} states). When we want to make the concurroid explicit,
we index $\mathsf{coh}$ with the name of the concurroid, but omit the
index when it can be inferred.

In the case of $\mathsf{priv}$, the coherent states are defined as:

\[
\mathsf{coh}_{(\mathsf{priv}\ p)} = \{p \mapsto [self = \hL, joint = (), other = \hE] | \mathsf{valid}\ (\hL \oplus \hE)\}
\]


\paragraph{Spin lock}

The concurroid for spin locks (over the label $\mathsf{sl}$) splits
the state as follows.
\[
sl \mapsto [self = \mL, joint = (lk \mapsto b) \oplus \hS, other = \mE]
\]

The joint component contains the shared \emph{heap}, consisting of the
actual lock address $lk$ which is a pointer storing the boolean $b$,
and the heap $\hS$ that the lock protects.

The concurroid encodes the mutual exclusion protocol following the CSL
and Owicki-Gries' resource invariant method~\cite{LeyWild-Nanevski:POPL13},
whereby a thread that acquires the lock also acquires ownership of the
heap $\hS$. When a thread releases the lock, the heap $\hS$ is moved
from the thread's private state, back into the shared portion. When
the lock is free, the shared heap $\hS$ satisfies a predetermined
invariant $I$. More formally, the components of $\mathsf{spinlock}$
state satisfy the property:
\[
\mbox{if}\ b\ \mbox{then}\ I\ \hS\ \mbox{else}\ \hS = \mathsf{empty}
\]

While the boolean $b$ specifies whether the lock taken or not, it
doesn't specify who owns it -- the ``self'' or ``other'' threads. The
fields $\mL$ and $\mE$ from the self and other components, are
\emph{auxiliary state} that provides this information. They can be
either $\lockOwn$ or $\lockNown$. For example, if $\mL = \lockOwn$
(resp. $\lockNown$), then the ``self'' thread owns (resp. doesn't own)
the lock, and similarly for $\mE$. $\mL$ and $\mE$ can't both be equal
$\lockOwn$ at the same time, as the lock can't be owned by two
different concurrent threads at the same time. However, if the lock is
taken, i.e. $b = true$ than one of $\mL$ and $\mE$ must be $\lockOwn$.

The type $\mathsf{mutex} = \{\lockOwn, \lockNown\}$ forms a PCM with
the following partial operation which describes \emph{joining} of lock
ownership:
\[
\begin{array}{rcl}
\lockOwn \oplus \lockNown & = & \lockNown \oplus \lockOwn = \lockOwn,\\
\lockNown \oplus \lockNown & = & \lockNown\\
\lockOwn \oplus \lockOwn & = & \mbox{undefined}
\end{array}
\]
Collecting the described conditions, the coherence set for the
$\mathsf{spinlock}$ concurroid is formally defined as:
\[
\begin{array}{l}
\mathsf{coh}_{(\mathsf{spinlock}\ sl)} = \\
\quad \{\!\!\!\begin{array}[t]{l}
  sl \mapsto [self = \mL, joint = (lk \mapsto b) \oplus \hS, other = \mE] \mid \\
  \mathsf{valid} (\mL \oplus \mE) \aand \mathsf{valid} ((lk \mapsto b) \oplus \hS) \aand \hbox{}\\
  \mbox{if}\ b\ \!\!\!\begin{array}[t]{l}
            \mbox{then}\ \hS = \mathsf{empty} \aand \mL \oplus \mE = \lockOwn\\
            \mbox{else}\ I\ \hS \aand \mL \oplus \mE = \lockNown\}
            \end{array}
  \end{array}
\end{array}
\]

\paragraph{Transitions}

Transitions are relations on states describing how the state of
concurroid may change. Each concurroid has three different
transitions: (1) internal, (2) acquire, and (3) release transition.

The internal transition describes the modifications to the contents of
the concurroid state, such as, e.g. changing the value stored into
some pointer of the private state. Acquire and release transitions are
indexed by a heap, and describe how the concurroid can receive (dually
send) a chunk of heap from/to the outside world.

\paragraph{Transitions of the priv concurroid}

The internal transition of the $\mathsf{priv}$ concurroid allows that
the self heap $\hL^1$ in the input state may change to an arbitrary
$\hL^2$ in the result state, \emph{as long as $\hL^1$ and $\hL^2$ have
  the same domains (i.e., contain the same pointers, possibly storing
  different values)}. The other state must remain unchanged, as that
state is private to other threads.
\[
\mathsf{internal}_{\mathsf{priv}\ p} = \{
\!\!\!\begin{array}[t]{l}
 (\!\!\!\begin{array}[t]{l}
    p \mapsto [self = \hL^1, joint=(), other=\hE], \\\relax
    p \mapsto [self = \hL^2, joint=(), other=\hE]) \mid 
 \end{array}\\\relax
    \mathsf{dom}\ \hL^1 = \mathsf{dom}\ \hL^2 \aand \hbox{}\\\relax
    p \mapsto [self = \hL^i, joint=(), other=\hE] \in \mathsf{coh}_{\mathsf{priv}\ p}\}
  \end{array}
\]

The acquire transition states that $\mathsf{priv}$ can receive a heap
$h$, only if $h$ is disjoint from the two heaps already in the
concurroid.

\[
\mathsf{acquire}_{\mathsf{priv}\ p}\ h = \{
\!\!\!\begin{array}[t]{l}
 (\!\!\!\begin{array}[t]{l}
    p \mapsto [self = \hL, joint=(), other=\hE], \\\relax
    p \mapsto [self = \hL \oplus h, joint=(), other=\hE]) \mid 
 \end{array}\\\relax
    \mathsf{valid} (\hL \oplus h \oplus \hE) \aand \hbox{}\\\relax
    \mathsf{p} \mapsto [self = \hL, joint=(), other=\hE] \in \mathsf{coh}_{\mathsf{priv}\ p}\}
  \end{array}
\]

Dually, the release transitionstates that $\mathsf{priv}$ can give
away a heap $h$ only if $h$ is contained within its private heap.

\[
\begin{array}{l}
\mathsf{release}_{\mathsf{priv}\ p}\ h = \\
\qquad \{
\!\!\!\begin{array}[t]{l}
 (\!\!\!\begin{array}[t]{l}
    p \mapsto [self = \hL \oplus h, joint=(), other=\hE], \\\relax
    p \mapsto [self = \hL, joint=(), other=\hE]) \mid 
 \end{array}\\\relax
    p \mapsto [self = \hL \oplus h, joint=(), other=\hE] \in \mathsf{coh}_{\mathsf{priv}\ p}\}
  \end{array}
\end{array}
\]

In a sense, the acquire and release transitions may be seen as
generalization of allocation and deallocation, as they account for a
concurroid (and the programs the concurroid may specify) increasing
and decreasing the number of pointer in its state. And indeed, SCSP
doesn't contain primitive operations for allocation and deallocation,
as these will be encoded inside SCSP as interactions of the
$\mathsf{priv}$ concurroid with another concurroid for allocation.
However, acquire and release transitions are much more general. 

\paragraph{Transitions of the $\mathsf{spinlock}$ concurroid}

For example, we use them to encode that upon locking and unlocking,
the lock-protected resource changes ownership, switching from shared
to private state, and back. We overloading the transition names, and
for the $\mathsf{spinlock}$ concurroid, define them as follows.

\[
\begin{array}{l}
\mathsf{acquire}_{\mathsf{spinlock}\ sl}\ h = \\
\qquad \{
\!\!\!\begin{array}[t]{l}
 (\!\!\!\begin{array}[t]{l}
   sl \mapsto [self = \lockOwn, joint=lk \mapsto \mathsf{true}, other=\lockNown], \\\relax
   sl \mapsto [self = \lockNown, joint=lk \mapsto \mathsf{false} \oplus h, other=\lockNown]) \mid 
 \end{array}\\\relax
    \mathsf{valid} (lk \mapsto \mathsf{false} \oplus h) \aand I\ h\}
  \end{array}
\end{array}
\]

\[
\begin{array}{l}
\mathsf{release}_{\mathsf{spinlock}\ sl}\ h = \\
\qquad \{
\!\!\!\begin{array}[t]{l}
 (\!\!\!\begin{array}[t]{l}
   sl \mapsto [self = \lockNown, joint=lk \mapsto \mathsf{false} \oplus h, other=\lockNown], \\\relax
   sl \mapsto [self = \lockOwn, joint=lk \mapsto \mathsf{true}, other=\lockNown]) \mid 
 \end{array}\\\relax
    \mathsf{valid} (lk \mapsto \mathsf{false} \oplus h) \aand I\ h\}
  \end{array}
\end{array}
\]

In other words, $\mathsf{spinlock}$ acquires a heap $h$ only upon
unlocking the lock; that is, setting the contents of the lock $lk$ to
false and the lock self status $\mL$ to $\lockNown$. The acquired heap
$h$ is disjoint from the lock $lk$ and satisfies the invariant
$I$. Dually, the lock-protected shared heap $h$ is released only upon
a succesful locking, which sets the contents of $lk$ to true, and the
self lock status $\mL$ to $\lockOwn$.

The internal transition of $\mathsf{spinlock}$ is trivially the
identity, symbolizing that lock's heap and auxiliary state can only be
changed by means of acquire and release tranitions.

\[
\mathsf{internal}_{\mathsf{spinlock}\ sl} = \{(s, s) \mid s \in \mathsf{coh}_{\mathsf{spinlock}\ sl}\}
\]

\paragraph{Entanglement}

Given concurroids $V$ and $W$, their entaglement $V \bowtie W$ is a
new concurroid that unions the states of $V$ and $W$, while also
\emph{connecting} the transitions of $V$ and $W$ of opposite polarity
(as we define shortly).

Intiutively, the idea is that $V \bowtie W$ is a product STS for $V$
and $W$, and is used to specify programs that logically perform an
ownership transfer between their components.

For example, in the concurroid $(\mathsf{priv}\ p) \bowtie
(\mathsf{spinlock}\ sl)$, the $\mathsf{acquire} h$ transition of
$\mathsf{priv}\ p$ will be ``connected'' with the
$\mathsf{release}\ h$ transitions of $\mathsf{release}\ h$.  In
particular, upon locking, the heap $h$ will be \emph{transfered} out
of the joint portion of $\mathsf{spinlock}\ sl$, and received into the
self component of $\mathsf{priv}\ p$. Such a connection thereby
encodes the logical protocol of Owicki-Gries style resources (also
used in CSL), whereby upon locking, the shared heap protected by the
lock is transfered into the private ownership of the locking thread.


\paragraph{Entangled states}

The coherent states of $V \bowtie W$ are those that can be obtained as
a union of coherent states of $V$ and of $W$, \emph{with disjoint
  erasures}.
\[
\mathsf{coh}_{V \bowtie W} = \{s_1 \oplus s_2 \mid 
s_1 \in \mathsf{coh}_V \aand s_2 \in \mathsf{coh}_W \aand \mathsf{valid} \flatten {s_1 \oplus s_2}\}
\]

By Lemma~\ref{lemma:join-valid}, the condition $\mathsf{valid}
\flatten {s_1 \oplus s_2}$ implies that $\flatten{s_1}$ and
$\flatten{s_2}$ are disjoint heaps. It also implies that $s_1$ and
$s_2$ \emph{have disjoint labels}. Indeed, had $s_1$ and $s_2$ shared
labels, then $s_1 \oplus s_2$ would be undefinded, making $\flatten
{s_1 \oplus s_2}$ undefined as well, by Lemma~\ref{lemma:join-valid}.

\paragraph{Entangled transitions}

The internal transition of $V \bowtie W$ is defined as:

\[
\begin{array}{l}
\mathsf{internal}_{V \bowtie W} = \{(s_1 \oplus s_2, s'_1 \oplus s'_2) \mid s_1 \oplus s_2, s'_1 \oplus s'_2 \in \mathsf{coh}_{V\bowtie W} \aand \hbox{}\\
\quad\!\!\!\begin{array}[t]{l}
  (s_1, s'_1) \in \mathsf{internal}_V \aand s_2 = s'_2  \oor \hbox{}\\
  s_1 = s'_1 \aand (s_2, s'_2) \in \mathsf{internal}_W \oor \hbox{}\\
  \exists h : \mathsf{heap}. \hbox{}\\
  \quad  (s_1, s'_1) \in \mathsf{acquire}_V\ h \aand (s_2, s'_2) \in \mathsf{release}_W\ h \oor \hbox{}\\
  \quad  (s_1, s'_1) \in \mathsf{release}_V\ h \aand (s_2, s'_2) \in \mathsf{acquire}_W\ h \}
  \end{array}
\end{array}
\]

The conjunct $s_1 \oplus s_2, s'_1 \oplus s'_2 \in
\mathsf{coh}_{V\bowtie W}$ ensures that the transition only applies to
states that belong to the state space of the $V \bowtie W$ STS.  After
that, $V \bowtie W$ can internally step whenever $V$ steps but $W$
stays idle ($(s_1 \oplus s'_1) \in \mathsf{internal}_V \aand s_2 =
s'_2$), or whenever $W$ steps but $V$ stays idle ($s_1 = s'_1 \aand
(s_2 \oplus s'_2) \in \mathsf{internal}_W$, or when $V$ and $W$
exchange the ownership of some heap $h$.


The entangled concurroid $V \bowtie W$ should itself have a release
and an acquire transition in order to communicate and entangle with
additional concurroids. For example, entangling $(\mathsf{priv}\ p)
\bowtie (\mathsf{spinlock}\ sl_1)$ gives us a concurroid to describe
the behavior of programs with state and \emph{one} spinlock $sl_1$. If
we want to add another spinlock $sl_2$, we'd have construct
$(\mathsf{priv}\ p) \bowtie (\mathsf{spinlock}\ sl_1) \bowtie
(\mathsf{spinlock}\ sl_2)$.

In this ``iterated'' entanglement, we want to avoid that the
concurroids for spinlocks $sl_1$ and $sl_2$ communicate between
themselves. Indeed, we don't want the locking of $sl_2$ to transfer
the ownership of the $sl_2$'s shared heap to $sl_1$; we want the heap
to be transfered to the private state.

To disable such communication, we de-symmetrize the definition of the
acquire and release transitions of $V \bowtie W$, so that they inherit
the behavior of the acquire and release transitions of $V$, but ignore
those of $W$. Formally:

\[
\begin{array}{l}
\mathsf{acquire}_{V \bowtie W}\ h = \{(s_1 \oplus s_2, s'_1 \oplus s'_2) \mid 
\!\!\!\begin{array}[t]{l}
  s_1 \oplus s_2, s'_1 \oplus s'_2 \in \mathsf{coh}_{V\bowtie W} \aand \hbox{}\\
  (s_1, s'_1) \in \mathsf{acquire}_V\ h \aand s_2 = s'_2\}
\end{array}
\end{array}
\]


\[
\begin{array}{l}
\mathsf{release}_{V \bowtie W}\ h = \{(s_1 \oplus s_2, s'_1 \oplus s'_2) \mid 
\!\!\!\begin{array}[t]{l}
  s_1 \oplus s_2, s'_1 \oplus s'_2 \in \mathsf{coh}_{V\bowtie W} \aand \hbox{}\\
  (s_1, s'_1) \in \mathsf{release}_V\ h \aand s_2 = s'_2\}
\end{array}
\end{array}
\]


\begin{lemma}[Reordering properties of $\bowtie$]\label{lemma:entangleAC}
\[
\begin{array}{rcl}
\mathsf{coh}_{V \bowtie W} & = & \mathsf{coh}_{W \bowtie
  V}\\ \mathsf{coh}_{(U \bowtie V) \bowtie W} & = & \mathsf{coh}_{U \bowtie (V \bowtie W)}\\
\end{array}
\]

\[
\begin{array}{rcl}
\mathsf{internal}_{V \bowtie W} & = & \mathsf{internal}_{W \bowtie V}\\
\mathsf{internal}_{(U \bowtie V) \bowtie W} & = & \mathsf{internal}_{(U \bowtie W) \bowtie V}\\
\mathsf{acquire}_{(U \bowtie V) \bowtie W} & = & \mathsf{acquire}_{(U \bowtie W) \bowtie V}\\
\mathsf{release}_{(U \bowtie V) \bowtie W} & = & \mathsf{release}_{(U \bowtie W) \bowtie V}
\end{array}
\]

\end{lemma}


\subsection{Concurroids abstractly}

A concurroid is a tuple $V = (D, \mathsf{coh},
\mathsf{internal}, \mathsf{acquire}, \mathsf{release})$ satsisfying a
number of properties (listed below).

$D$ is a set of labels, refered to as the \emph{label footprint} of $A$,
listing the labels that appear in the well-formed
states. $\mathsf{coh}$ is the set of well-formed states, the
transition $\mathsf{internal}$ is a relation on states, and so are
$\mathsf{acquire}\ h$ and $\mathsf{release}\ h$ for every heap $h$.


\paragraph{coherence-set properties}

The state sets of a concurroid satisfy the following properties, for
every state $s$.
\[
\begin{array}{rcl}
\mathsf{coh_D} & : & s \in \mathsf{coh} \implies \mathsf{dom}(s.self) = \mathsf{dom}(s.joint) = \mathsf{dom}(s.other) = D\\
\mathsf{coh_H} & : & s \in \mathsf{coh} \implies \mathsf{valid}{\flatten s}\\
\mathsf{coh_{SO}} & : & s \in \mathsf{coh} \implies \mathsf{valid}(s.self \otimes s.other)\\
\mathsf{coh_Z} & : & \forall t\ldot s \zig t \in \mathsf{coh} \iff s \zag t \in \mathsf{coh}.
\end{array}
\]

Properties $\mathsf{coh_D}$ and $\mathsf{coh_H}$ are straightforward,
so we don't discuss them.  

$\mathsf{coh_{SO}}$ requires that composition of the self and other
view of the concurroid state is well-formed. To illustrate, in the
case of the concurroid $\mathsf{priv}$, $\mathsf{coh_{SO}}$
corresponds to the requirement that $\hL$ and $\hE$ are disjoint
heaps, i.e. that the private states of parallel threads can't
overlap. In the case of the concurroid $\mathsf{spinlock}$,
$\mathsf{coh_{SO}}$ corresponds to the requirement that the mutexes
$\mL$ and $\mE$ are not both $\lockOwn$, i.e. that parallel threads
can't simultanously own one and the same lock. In the case of
entanglement $\mathsf{priv}\ p \bowtie \mathsf{spinlock}\ sl$,
$\mathsf{coh_{SO}}$ corresponds to the conjunction of the
$\mathsf{coh_{SO}}$ properties for the individual concurroids.

$\mathsf{coh_Z}$ allows that the boundary between the self and other
components can be realigned, as the component $t$ added onto the self
(i.e., zigged) can always be moved onto the other (i.e. zagged), and
vice-versa.  Concretely, $\mathsf{coh_Z}$ in the case of
$\mathsf{priv}$ shows that the state $[self=\hL \oplus h,
  joint=(),other=\hE]$ is coherent iff the state $[self=\hL,
  joint=(),other=\hE + h]$ is coherent too.

The $\mathsf{coh_Z}$ property will be important when considering
properties of fork-join composition of threads. Following CSL, in SCPL
we consider that the private state of a thread is divided among its
children upon forking. The left child becomes part of the environment
for the right child, and vice versa. Thus, the state assigned to the
left child's self component, is zagged onto the right child's other
component (and vice versa). $\mathsf{coh_Z}$ property thus ensures
that upon forking, a thread that is in a coherent state leaves both of
its children in coherent states too, albeit the particular alignments
of the boundary between self and other may differ among the children
themselves and the parent thread.

\paragraph{Transition properties}

Let $G$ vary over a transitions $\mathsf{internal}$, and
$\mathsf{acquire}\ h$ and $\mathsf{release}\ h$ (for every heap
$h$). Then $G$ is required to satisfy the following properties.

\[
\begin{array}{rcl}
\mathsf{trans_C} & : & (s, s') \in G \implies s \in coh \aand s' \in coh\\
\mathsf{trans_E} & : & (s, s') \in G \implies s.other = s'.other\\
\mathsf{trans_Z} & : & \forall t\ldot \!\!\!\begin{array}[t]{l}
  s.other = s'.other \implies \hbox{}\\
  (s \zag t, s' \zag t) \in G \implies (s \zig t, s' \zig t) \in G
  \end{array}
\end{array}
\]

The $\mathsf{trans_C}$ property restricts $G$ to only relate states
that are coherent, that is, they belong to the state space of the STS.

The $\mathsf{trans_E}$ property restricts that $G$ can't modify the
other component of the concurroid. Thus, $G$ only bounds how the
concurroid can change the self and joint component. In relation to the
work on Rely-Guarantee
reasoning~\cite{Jones:IFIP83,Feng-al:ESOP07,Feng:POPL09,Vafeiadis-Parkinson:CONCUR07},
the $\mathsf{trans_E}$ property implies that the transitions of
concurroids are \emph{guarantee} transitions.
%
Given a concurroid $V$, the transitions that correspond to rely
transitions in Rely-guarantee reasoning (that is, they bound the steps
of the environment threads), may be computed as the internal, acquire
and release transitions of the \emph{transposed} concurroid $V^\top$,
in which the self and other components of the states in the definition
of $\mathsf{coh}$ set and the relations are interchanged.\is{This is
  an important and not entirely intuitive definition. Shall we state
  it more formally?} The transposition of concurroids will be used in
the definition of the important concept of \emph{stability}.

The $\mathsf{trans_Z}$ property show shows that a chunk $t$ may be
moved from the other components of related states (zagged) into the
self components (zigged). 

More concretely, in the case of $\mathsf{priv}$ after the expansion of
the definitions of $\zag$ and $\zig$, and some simplification, the
$\mathsf{trans_Z}$ property states that, if:
\[
(\!\!\!\begin{array}[t]{l}
  [self=\hL, joint=(),other=\hE + h], \\\relax
  [self=\hL', joint=(),other=\hE + h]) \in G
 \end{array}
\]
then 
\[
(\!\!\!\begin{array}[t]{l}
 [self=\hL \oplus h, joint=(),other=\hE],\\\relax
 [self=\hL' \oplus h, joint=(),other=\hE]) \in G
\end{array}
\]

Thus $\mathsf{trans_Z}$ represents a generalized notion of
\emph{locality} of transitions, whereby if a transition relates states
with certain self components, then it also relates states in which the
self components have been \emph{enlarged} (i.e., framed by) by a fixed
quantity.

Whereas related work on separation logic only considers framing by a
heap, $\mathsf{trans_Z}$ considers framing by any component.

Whereas in related work the framing heap is materialized out of
nowhere, $\mathsf{trans_Z}$ requires that the framing component $t$ is
appropriated from the environment; that is, $t$ has to be part of
other in the premise of $\mathsf{trans_Z}$, and is removed from other
in the conclusion of $\mathsf{trans_Z}$.

\paragraph{Specific properties of transitions}

Additionally, there's a few properties that only $\mathsf{internal}$
transitions, or only $\mathsf{acquire}$ and $\mathsf{release}$
transitions has to satisfy.

\[
\begin{array}{rcl}
\mathsf{inter\_refl} & : & s \in \mathsf{coh} \implies (s, s) \in \mathsf{internal}_A\\
\mathsf{inter\_foot} & : & \!\!\!\begin{array}[t]{l}
           (s, s') \in \mathsf{internal}_A\ h \implies \hbox{}\\
           \quad \mathsf{valid} \flatten{s} = \mathsf{valid} \flatten{s'} \aand \hbox{}\\
           \quad \mathsf{dom} \flatten{s} = \mathsf{dom} \flatten{s'}
           \end{array}\\
\mathsf{acq\_foot} & : & \forall h{:}\mathsf{heap}\ldot \!\!\!\begin{array}[t]{l}
           (s, s') \in \mathsf{acquire}_A\ h \implies \hbox{}\\
           \quad \mathsf{valid} (\flatten s \oplus h) = \mathsf{valid} \flatten{s'} \aand \hbox{}\\
           \quad \mathsf{dom} (\flatten s \oplus h) = \mathsf{dom} \flatten{s'}
           \end{array}\\
\mathsf{rel\_foot} & : & \forall h{:}\mathsf{heap}\ldot \!\!\!\begin{array}[t]{l}
           (s, s') \in \mathsf{release}_A\ h \implies \hbox{}\\
           \quad \mathsf{valid} {\flatten s} = \mathsf{valid} (h \oplus \flatten{s'}) \aand \hbox{}\\
           \quad \mathsf{dom} {\flatten s} = \mathsf{dom} (h \oplus \flatten{s'})
           \end{array}
\end{array}
\]

$\mathsf{inter\_refl}$ shows that the internal transition is
reflexive, thus allowing the programs specified by the concurroid to
be \emph{idle}, that is, transition to their initial state.

$\mathsf{inter\_foot}$ requires that internal transition preserves the
heap obtained by erasure of states. Related states may move the
ownership of various heaps between self and joint components; they may
also change the auxiliary state, and change the \emph{contents} of
individual heap pointers. However, they can't add new pointers or
remove old ones. Adding new pointers or removing old ones is the job
of $\mathsf{acquire}$ and $\mathsf{release}$ transitions, as
formalized by $\mathsf{acq\_foot}$ and $\mathsf{rel\_foot}$
properties.


\begin{lemma}[Entanglement]
If $V$ and $W$ are concurroids, then $V \bowtie W$ is a concurroid as
well, whose label footprint is the union of the label footprints of
$V$ and $W$.
\end{lemma}


\paragraph{Stability}

\begin{definition}[Environment stepping]
Let $V$ be a concurroid, and $W = V^\top$ its transpose. The
\emph{environment stepping} relation for $V$ is defined as the
following reflexive-transitive closure.
\[
\begin{array}{rcl}
\mathsf{env\_step}_V & = &
  \{(s, s') | (s, s') \in \mathsf{internal}_W \oor \hbox{}\\
& & \qquad \exists h\ldot (s, s') \in \mathsf{acquire}_W\ h \cup 
                           \mathsf{relase}_W\ h\}\\
\mathsf{env\_steps}_V & = & (\mathsf{env\_step}_V)^*
\end{array}
\]
\end{definition}


\begin{definition}[Stability]
A state predicate $p$ is \emph{stable} under a concurroid $V$ iff
\[
\forall i\ j{:}state, p\ i \implies (i, j) \in \mathsf{env\_steps}_V \implies p\ j.
\]
A state relation $r$ is \emph{stable} under a concurroid $V$ iff
\[
\forall i\ j\ k{:}state, (i, j) \in r \implies (j, k) \in \mathsf{env\_steps}_V \implies (i, k) \in r.
\]
\an{Maybe it's better to write these in a point-free way, with
  composition? Though one can do that for relations, there's no
  standard way to do it for predicates, as there's no standard
  operator for composing a predicate with a relation.}

\end{definition}


