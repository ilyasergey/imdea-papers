
\section*{Optional Appendices}

In the optional appendices we describe in detail some technical
insights behind \SCST and the model of concurroids. We also provide an
extended overview of another case study, which was omitted in the
paper body due to space constraints. The remainder of the paper is
structured as follows. \textbf{Appendix~\ref{sec:actions}} describes a
number of properties and requirements of the actions, presented in
Section~\ref{sec:logic} of the main paper
body. \textbf{Appendix~\ref{sec:stability}} formalizes the familiar
notions of \emph{Rely} and \emph{Guarantee} relations, as well as a
notion of \emph{stability}, in terms of concurroids and their
transitions. \textbf{Appendix~\ref{sec:phi-properties}} lists the
necessary properties of the abstraction functions $\Phi$, used in the
$\textsc{Hide}$ rule of \SCST. \textbf{Appendix~\ref{sec:ordering}}
defines Hoare ordering $(p_1, q_1) \sqsubseteq (p_2, q_2)$, used in
the rule $\textsc{Conseq}$ of the
logic. \textbf{Appendix~\ref{sec:ticketed}} presents development of a
large example: a concurroid for the ticketed lock algorithm. Finally,
in \textbf{Appendix~\ref{sec:semantics}} we present the denotational
model of \SCST, based on \emph{action trees} and \emph{schedules}, and
prove the main technical result: soundness of \SCST as a shallow
embedding into the Calculus of Inductive Constructions. 


\section{Abstract Properties of Actions}
\label{sec:actions}

An action is a 4-tuple $a = (U, A, \sigma, \mu)$. $U$ is a concurroid
whose internal transition $a$ is supposed to respect. $A$ is the type
of the return value of $a$. $\sigma$ is the \emph{safety predicate} (over
states of $U$) describing under which conditions $a$ may be
executed. $\mu$ is the \emph{stepping relation}; it relates the input
state, output state, and the result of $a$.

\begin{definition}[Action erasures] Given an action $a$, the erasures $\flatten \sigma$ and 
$\flatten \mu$ of $a$'s safety predicate and stepping relation are relations on heaps defined as follows.  
%
{\small
\[
\begin{array}{lcl}
\flatten {w} \in \flatten{\sigma} & \iff & w \in \sigma\\
(\flatten{w}, \flatten{w'}, r) \in \flatten{\mu} & \iff & (w, w', r) \in \mu
\end{array}
\]}
\end{definition}

An \emph{atomic} is a triple $\alpha = (A, \sigma, \mu)$. It's a
special kind of actions, but over concrete heaps, rather than over
states. States differ from heaps in that they are decorated with
additional information such as auxiliary state and partitioning
between \self, \joint and \other.  As with actions, $A$ is the return
type, $\sigma$ is the safety predicate and $\mu$ is the stepping
relation, but they all range over heaps.

We consider four different (parametrized classes of) atomics,
corresponding to the four (parametrized) primitive memory operations
that we consider.

\begin{definition}[Primitive atomic actions]
\label{def:actions}
{\small
\[
\begin{array}{lcl}
  \mathsf{Read}^A_x & = & (A, (x \hpts_A -) \hunion h, (x \hpts v) \hunion h \rightsquigarrow (x \hpts v) \hunion h \aand \result = v)\\
  \mathsf{Write}\ x\ v & = & (\mathsf{unit}, (x \hpts -) \hunion h, (x
  \hpts -) \hunion h \rightsquigarrow (x \hpts v) \hunion h)\\
  \mathsf{Skip} & = & (\mathsf{unit}, h, h \rightsquigarrow h) 
  \\
  \mathsf{RMW}^{A~B}_{x~f~g} & = & (B, (x \hpts_A -) \hunion h, (x
  \hpts v) \hunion h \rightsquigarrow (x \hpts f(v)) \hunion h \aand \result = g(v))
\end{array}
\label{eq:actions}
\]}
  
\end{definition}

The last class $\mathsf{RMW}^{A~B}_{x~f~g}$ corresponds to the family of
\emph{Read-Modify-Write} operations: they all atomically replace the
current register value $v$ with $f(v)$ for some pure function $f$, and
return the result according to the function $g$~\cite[\S
5.6]{Herlihy-Shavit:08}. One particular representative of this family
is the CAS operation, which instantiates the parameters of $\mathsf{RMW}$ as
follows:
%
{\small
\[
\begin{array}{rcl}
\text{CAS}_{A~x~v_1~v_2} & \eqdef &
\mathsf{RMW}^{A~\mathsf{bool}}_{x~f({v_1},{v_2})~g({v_1},{v_2})}, \text{where}   
\\ \\
f({v_1},{v_2})(v) & = & \mathsf{if}~ (v = v_1) ~\mathsf{then}~ v_2
~\mathsf{else}~ v_1 \\
g({v_1},{v_2})(v) & = & (v = v_1)
\end{array}
\]}
%
% \[
% \begin{array}{lcl}
% CAS_A\ x\ v_1\ v_2 & = & (\mathsf{bool}, x \hpts_A -, x \hpts v \rightsquigarrow \begin{array}[t]{l}
%                                           \mathsf{if}\ v = v_1\ \mathsf{then}\ x \hpts v_2, r = \mathsf{true}\\
%                                           \mathsf{else}\ x \hpts v_1,
%                                           r = \mathsf{false})
%                                           \end{array}  
% \end{array}
% \]
%
Another representative of the $\mathsf{RMW}$ class is the atomic
\emph{fetch-and-increment} functions, which we will employ in
Appendix~\ref{sec:ticketed}, defined as follows:
%
\[
\begin{array}{c}
\text{FAI}_{x} \eqdef
\mathsf{RMW}^{\mathsf{nat}~\mathsf{nat}}_{x~\mathsf{inc}~\mathsf{id}}, \text{where}   
\\ \\
\mathsf{inc} (v) =  v + 1 ~~~~
\mathsf{id} (v) = v
\end{array}
\]


\begin{definition}[Operational action] 
\label{def:opact}
%
An action $a$ is \emph{operational} if its erasure corresponds to one
of the atomics, \ie, if there exists $b \in \set{\mathsf{Read}^A_x, \mathsf{Write}\ x\ v, \mathsf{Skip},
\mathsf{RMW}^{A~B}_{x~f~g}}$ such that
{\small
\[
\flatten{\sigma_a} \subseteq \sigma_b \aand 
\forall h \in \flatten{\sigma_a}\ h'\ r\ldot (h, h', r) \in \flatten{\mu_a} \implies (h, h', r) \in \mu_b
\]}
\end{definition}
In our examples we only consider operational actions, though the
inference rules and the implementation in Coq don't currently enforce
this requirement (the operationality of actions in the examples has
been proved by hand).

Let $U = ({\cal L}, {\cal W}, \tau, {\cal E})$. The action $a = (U, A,
\sigma, \mu)$ is required to satisfy the following properties.
{\small
\[
\begin{array}{rcl}
\textit{Coherence} & : & w \in \sigma \implies w \in {\cal W}\\
\textit{Safety monotonicity} & : & w \zag t \in \sigma \implies w \zig t \in \sigma\\
\textit{Step safety} & : & (w, w', r) \in \mu \implies w \in \sigma\\
\textit{Internal stepping} & : & (w, w', r) \in \mu \implies (w, w') \in \tau\\
%\mathsf{step_Z} & : & s.other = s'.other \implies \hbox{}\\
%& & \quad (s \zag t, s' \zag t, v) \in \mathsf{step}_a \implies (s \zig t, s' \zig t, v) \in \mathsf{step}_a\\
\textit{Framing} & : & w \zag t \in \sigma \implies (w \zig t, w', r) \in \mu \implies \hbox{}\\
& & \quad \exists w''\ldot w' = w'' \zig t \aand  (w \zag t, w'' \zag t, v) \in \mu\\
\textit{Erasure} & : &  \mathsf{defined} (\flatten w \hunion h) \implies \flatten w \hunion h = \flatten{w'} \hunion h' \implies \hbox{}\\
& &  (w, w_1, r) \in \mu \implies (w', w'_1, r') \in \mu \implies \hbox{}\\
& &  r = r' \aand \flatten w_1 \hunion h = \flatten{w'_1} \hunion h'\\
\textit{Totality} & : & \forall w\ldot w \in \sigma \implies \exists w'\ v\ldot (w, w', v) \in \mu
\end{array}
\]}
The properties of Coherence, Step safety and Internal stepping are
straightforward.  
%
Safety monotonicity states that if the action is safe in a state with
a smaller \self component (because the other component is enlarged by
$t$), the action is also safe if we increase the \self component by
$t$.
%The property is analogous to the equally named property in
%abstract separation logic.
%

Framing property says that if $a$ steps in a state with a large \self
component $w \zig t$, but is already safe to step in a state with a
smaller \self component $w \zag t$, then the result state and value
obtained by stepping in $w \zig t$ can be obtained by stepping in $w
\zag t$, and moving $t$ afterwards.

The Erasure property shows that the behavior of the action on the
concrete input state obtained after erasing the auxiliary fields and
the logical partition, doesn't depend on the erased auxiliary fields
and the logical partition. In other words, if the input state have
\emph{compatible} erasures (that is, erasures which are subheaps of a
common heap), then executing the action in the two states results in
equal values, and final states that also have compatible
erasures. This is a standard property proved in concurrency logics
that deal with auxiliary state and
code~\cite{Owicki-Gries:CACM76,Brookes:TCS07}.

The Totality property shows that an action whose safety predicate is
satisfied always produces a result state and value. It doesn't loop
forever, and more importantly, it doesn't crash. We will use this
property of actions in the semantics of programs to establish that if
the program's precondition is satisfied, then all of the
approximations in the program's denotation are either done stepping,
or can actually make a step (i.e., they make progress).

So far, the actions are defined in a so-called \emph{large footprint}
style, \eg, the definition of the action $\mathsf{release}$ in
Section~\ref{sec:logic} mentions existentially-quantified heap $\hL$
explicitly,
%
To enable writing various actions in a \emph{small footprint} style,
we also enforce the property {\small
\[
\begin{array}[t]{rcl}
\textit{Locality} & : & w.\mathother = w'.\mathother \implies (w \zag t, w' \zag t, v) \in \mu \implies (w \zig t, w' \zig t, v) \in \mu
\end{array}
\]}
%This property corresponds to what we have taken up calling
%\emph{locality}, though it's not at all equivalent to locality in the
%The property is \emph{not} provable from the rest, unless we require
%that each action is deterministic (which right now seems
%unecessary). Thus, if we want it, it has to be required as primitive,
%which is what we have done.
Curiously, if the default use of the logic is in a large footprint
notation, then this property is not necessary as it is not used in any
proofs.
%
%However, I just realized that the property is not used \emph{anywhere}
%in the development. It's unused in the model, and it's unused in the
%always lemmas. Thus, I'm suppressing it here.
%
%On the other hand, the similar locality property for
%\emph{transitions} is useful in the always lemmas, as alredy
%commented!}
%\end{comment}


\section{Self/Environment Stepping and Stability}
\label{sec:stability}

\begin{definition}[Self-stepping (Guarantee)]
\label{def:guarantee}

Given a concurroid $U = ({\cal L}, {\cal W}, \tau, {\cal E})$, the
\emph{environment stepping} relation $\guar{U}$ of $U$ unions its
internal and all external transitions of $U$, and then takes a
transitive closure.

{\small
\[
  \guar{U} = (\tau_U \cup \bigcup_{h, (\alpha, \rho)\in {\cal E}_X} (\alpha~h) \cup (\rho~h))^*.
\]
}
\end{definition}


\begin{definition}[Environment stepping (Rely)]
\label{def:rely}

Given a concurroid $U = ({\cal L}, {\cal W}, \tau, {\cal E})$, the
\emph{environment stepping} relation $\rely{U}$ of $U$ unions the
transposed versions of the internal and all external transitions of
$U$, and then takes a transitive closure.
{\small
\[
\rely{U} = (\tau^\top \cup \bigcup_{h, (\alpha, \rho)\in {\cal E}}\ (\alpha\ h)^\top \cup (\rho\ h)^\top)^*
\]}
\end{definition}


\begin{definition}[Stability]
An assertion (\ie, a predicate over states) $p$ is \emph{stable}
under concurroid $U$ iff for every $w, w'$ such that $(w, w') \in
\rely{U}$, $w \models p \implies w' \models p$.
\end{definition}

\begin{comment}
We require two further properties of actions, requiring stability of
the safety and stepping action components wrt. to the associated
concurroid $V$.\an{I separate these, because they are not actually
  implemented in Coq. However, I think that if we impose them, the
  section on logic will be easier to present.}
{\small
\[
\begin{array}{rcl}
\mathsf{safe}_{ST} & : & \mathsf{safe}_a\ i \implies (i, j) \in \mathsf{env\_steps}_V \implies \mathsf{safe}_a\ j\\
\mathsf{step}_{ST} & : & (i, j) \in \mathsf{step}_a \implies (j, k) \in \mathsf{env\_steps}_V \implies \hbox{}\\
                  &   & (i, k) \in \mathsf{step}_a
\end{array}
\]}
\end{comment}


\section{Properties of $\Phi$ Functions from the Hiding Rule}
\label{sec:phi-properties}

The abstraction function $\Phi$ is a user-specified annotation on the
hide command. It maps values $g : G$ (where $G$ is also user
specified) to assertions, that is, predicates over states
(equivalently, sets of states) of a concurroid $V$. For the soundness
of the hiding rule, $\Phi$ is required to satisfy the following
properties.  {\small
\[
\begin{array}[t]{lcl}
\textit{Coherence} & : & w \in \Phi(g) \implies w \in {\cal W}_V\\
\textit{Injectivity} & : & w \in \Phi(g_1) \implies w \in \Phi(g_2) \implies g_1 = g_2\\
%& : & i \in \mathsf{coh}\ W \arrow \exists g, i \in \Phi(g)\\
\textit{Surjectivity} & : & w_1 \in \Phi(g_1) \implies w_2 \in {\cal W}_W \implies w_1.\mathother = w_2.\mathother \implies \exists g_2\ldot w_2 \in \Phi(g_2)\\
\textit{Guarantee} & : & w_1 \in \Phi(g_1) \implies w_2 \in \Phi(g_2) \implies w_1.\mathother = w_2.\mathother\\
\textit{Precision} & : & w_1 \in \Phi(g) \implies w_2 \in \Phi(g) \implies \flatten {w_1} \hunion h_1 = \flatten {w_2} \hunion h_2 \implies w_1 = w_2
\end{array}
\]}
Coherence and Injectivity are obvious. Surjectivity states that for
every state $w_2$ of the concurroid $W$ one can find an image $g$,
under the condition that the \other component of $w_2$ is well-formed
according to $\Phi$ (typically, that the \other component is equal to
the unit of the PCM-map monoid for $W$). Guarantee is the property
stated in the paper at the end of Section~\ref{sec:reasoning}, but
through assertions, rather than semantically. This property formalizes
that environment of $\mathsf{hide}$ can't interference on $V$, as $V$
is installed locally. Thus, whatever the environment does, it can't
influence the \other component of the states $w$ described by $\Phi$.

Precision is a technical property common to separation-style logics,
though here it has a somewhat different flavor. Precision ensures that
for every value $g$, $\Phi(g)$ precisely describes the underlying
heaps of its circumscribed states; that is, each state $\Phi(g)$ is
uniquely determined by its heap erasure.

\section{Definition of Hoare Ordering %
  $(p_1, q_1) \sqsubseteq (p_2, q_2)$}
\label{sec:ordering}

Given preconditions $p_1, p_2$ and postconditions $q_1, q_2$, the
judgment $(p_1, q_1) \sqsubseteq (p_2, q_2)$ generalizes the usual
Hoare logic side conditions on the rule of consequence about
strengthening the precondition $p_2\,{\implies}\,p_1$ and weakening
the postcondition $q_1\,{\implies}\,q_2$.  The generalization is
required in \SCST because of the local scope of logical variable. In
first order Hoare logics, the logical variables have global scope, so
the above implications over $p_1, p_2$ and $q_1, q_2$ suffice. In
\SCST, the logical variables have scope locally over Hoare triples,
and this scope has to be reflected in the semantic definition of
$\sqsubseteq$ by introducing quantifiers. The definition is similar to
the one of Kleymann~\cite{Kleymann:FAC99}.
%
{\small
\[
\begin{array}{l}
(p_1, q_1) \sqsubseteq (p_2, q_2) \iff \hbox{}\\
\qquad \forall w\ w'\ldot \begin{array}[t]{l}
(w \models \exists \bar{v}_2\ldot p_2 \implies w \models \exists \bar{v}_1\ldot p_1) \aand \hbox{}\\
((\forall \bar{v}_1\ \result\ldot w \models p_1 \implies w' \models q_1) \implies (\forall \bar{v}_2\ \result\ldot w \models p_2 \implies w' \models q_2))
\end{array}
\end{array}
\]}
where $v_i = \mathsf{FLV}(p_i, q_i)$ are the free logical variables.
%
The definition makes it apparent that the Hoare triple
$\stconc{p}{c}{q}{U}$ is essentially a syntactic sugar for a different
kind of Hoare triple, which may be written as:
{\small
\[
\stconc{w\ldot\exists \bar{v}\ldot w \models p}{c}{\result\ w\ w'\ldot \forall \bar{v}\ldot w \models p \implies w' \models q}{U}
\]}
where $v = \mathsf{FLV}(p,q)$. In this alternative Hoare triple, the
postconditions are predicates ranging over input and output states $w$
and $w'$ (they are thus called binary postconditions). The advantage
of the alternative Hoare triple is that the logical variables are
explicitly bound, making their scoping explicit. In our Coq
implementation we use this alternative formulation of Hoare triples.

