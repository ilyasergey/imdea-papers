\subsection{Soundness Result}
\label{sec:soundness}

We culminate this section with the proof of soundness of the
interpretation. 
%
The main soundness theorem (Theorem~\ref{thm:soundness}), which we
present at the end of this section, is a key result of our formal
development. It states that \SCST is sound as a logic with respect to
its translation into CiC (\ie shallow embedding), that is, for any
judgment $J$, $\Gamma \vdash J$ implies $\mean{\Gamma \vdash J}$. 
%
In particular, it means that if a judgment of the form $\Gamma \vdash
\spec{p}\,c\, \ty \,A\,\spec{q}@U$ is derivable in \SCST, then its
translation $\mean{\Gamma} \vdashCiC \mean{c} : \ST{p}{A}{q}{U}$ is
derivable in CiC. 
%
Next, let us apply Definition~\ref{def:htypes} of a Hoare type to
$\ST{p}{A}{q}{U}$ and recall that $\mean{c}$ is just a set of action
trees, corresponding to the command $c$ (Figure~\ref{fig:den}).
%
Therefore, by the derivation under $\vdashCiC$, all action trees from
$\mean{c}$ will belong to $\ST{p}{A}{q}{U}$, which automatically
implies that for any state $w$ and each tree from $\mean{c}$, if $w$
satisfies $p$ and $w \in \coh{U}$, then $\mathsf{after}_U~w~t~q$ also
holds. The later means (Definition~\ref{def:modpred}) that $t$ is
$\mathsf{safe}$ and $q~v~w'$ eventually holds if $t$ steps completely
to $\mathsf{Ret}~v$ in some final configuration $w'$. Since $\mathsf{after}$
employs the $\mathsf{always}$ predicate, it also implies the progress
property (Lemma~\ref{thm:progress}) for all action trees from $\mean{c}$.

\subsubsection{Auxiliary modal lemmas}
\label{sec:auxil-modal-lemm}

The proof of the soundness result will rely on a number of auxiliary
modal lemmas, which are listed below.
%
For lack of space, we omit their proofs. They are relatively
straightforward, usually by an induction on the schedule $\Paths$
used by $\mathsf{always}$ predicate.
%
We have carried all of proofs for these lemmas in
Coq~\cite{fcsl-coqscripts}, they usually proceed by an induction on
the schedule $\Paths$.
%
In the statements of the lemmas below, we always assume $U$ to be some
fixed concurroid, $t$ ranges over action trees and $w$ ranges over
states.

The Universal lemma states that the modal predicate $\mathsf{always}$
commutes with universal quantification, which yields to the soundness
of an infinitary \textsc{Conj}unction rule (Figure~\ref{fig:rules}). %
The assumption
$\mathsf{always}_U~w~t~(\lambda~w'~t'\ldot\mathsf{True})$ makes the
lemma hold when the quantification over $x$ is vacuous.

% Notation alwsafe s p := (always s p (fun _ _ => True)).
%
% Lemma alwA A B s (p : tree A) (P : B -> state -> tree A -> Prop) : 
%         alwsafe s p ->
%         (always s p (fun s' p' => forall x, P x s' p') <->
%          forall x, always s p (fun s' p' => P x s' p')).

\begin{lemma}[Universal]
\label{lm:universal}
%
If $\mathsf{always}_U~w~t~(\lambda~w'~t'\ldot\mathsf{True})$, then
$\mathsf{always}$ commutes with universal quantification:
%
\[
\mathsf{always}_U~w~t~(\lambda w'~t' \ldot \forall x \ldot P~x~w'~t')
\iff
\forall x\ldot \mathsf{always}_U~w~t~(\lambda w'~t' \ldot P~x~w'~t')
\]  
\end{lemma}

The implication lemma corresponds to weakening the postcondition,
which is necessary for the proof of \textsc{Conseq} rule.

% Lemma aft_imp A s (p : tree A) (P1 P2 : A -> state -> Prop) : 
%         (forall v s, s \In coherent -> 
%                      P1 v s -> P2 v s) -> 
%         after s p P1 -> after s p P2.

\begin{lemma}[Implication for \textsf{after}]
\label{lm:aft-imp}
%
If $\mathsf{after}_U~w~t~Q_1$ and $Q_1~v~w \implies Q_2~v~w$, for all
$v$ and $w \in \coh{U}$, then $\mathsf{after}_U~w~t~Q_2$.  
\end{lemma}


Closure under sequential composition justifies the $\mathsf{Seq}$
rule: $q$ holds at the end of the composed tree if final configuration
of the prefix $t_1$ can be used as an initial configuration for the
suffix to show $q$ holds $\mathsf{after}$.

% Lemma aft_bnd A B (p1 : tree A) (p12 : tree B) pp2 s1 P :
%         p12 \In tcat p1 pp2 -> 
%         after s1 p1 (fun v s => 
%           forall p, p \In pp2 v -> after s p P) ->
%         after s1 p12 P.

\begin{lemma}[Closure under sequential composition]
\label{lm:aft-bnd}
%
For some action tree $t_1 \ty \mathsf{tree}~U~B$ and $K \ty B \to
\mathcal{P}(\mathsf{tree}~U~A)$, if $t_2 \in \set{\mathsf{Seq}~t_1~k
  ~|~ \forall x \ldot k~x \in K~x}$ and
\[
\mathsf{after}_U~w_1~t_1~(\lambda v~w. ~\forall t \ldot ~(t \in K~v)
\implies \mathsf{after}_U~w~t~q),
\]
then $\mathsf{after}_U~w_1~t_2~q$.
\end{lemma}

Closure under parallel composition justifies the \textsc{Par}
rule. 
%
Intuitively, it holds because when (an approximation $t_2$ of) $c_2$
takes a step over its private and shared state, it amounts to
$\rely{U}$ environment interference on (an approximation $t_1$ of)
$c_1$, and vice versa. Note that the pattern of rearranging subjective
\self/\other components recurs at the level of triples $w =
\state{s}{j}{o}$: the parallel composition uses $\State{s_1 \zip
  s_2}{j}{o}$ and the left and right child threads use
$\State{s_1}{j}{o \zip s_2}$ and $\State{s_2}{j}{o \zip s_1}$,
respectively.

% Lemma aft_par lp1 lp2 sh ep (t1 : tree A1) (t2 : tree A2) P1 P2 :
%         after ([lp1, sh, ep] \> lp2) t1 P1 ->
%         after ([lp2, sh, ep] \> lp1) t2 P2 -> 
%         after [lp1 \* lp2, sh, ep] (Par t1 t2 Ret)
%           (fun v' s' => 
%              exists lp1' lp2' sh' ep',
%                [/\ s' = [lp1' \* lp2', sh', ep'], 
%                    P1 v'.1 ([lp1', sh', ep'] \> lp2') &
%                    P2 v'.2 ([lp2', sh', ep'] \> lp1')]).

\begin{lemma}[Closure under parallel composition]
\label{lm:aft-par}
If $\mathsf{after}_U~(\state{s_1 \zip s_2}{j}{o})~t_1~Q_1$ and 
$\mathsf{after}_U~(\state{s_2 \zip s_1}{j}{o})~t_2~Q_2$, then 
\[
\begin{array}{rrl}
\mathsf{after}_U & (\state{s_1 \zip s_2}{j}{o}) &
(\mathsf{Par}~t_1~t_2~(\lambda x\ldot \mathsf{Ret}~x)) 
\\ & &
(\lambda v'~w'\ldot \exists s_1'~s_2'~j'~o' \ldot  
\begin{array}[t]{ll}
w' = \state{s_1' \zip s_2'}{j'}{o'} & ~\aand \\ 
P_1~\pi_1~v'~\State{s_1' }{j'}{o' \zip s_2'} & ~\aand \\ 
P_2~\pi_2~v'~\State{s_2' }{j'}{o' \zip s_1'} ).
\end{array}

\end{array}
\]
\end{lemma}

Closure under injection is targeted to justify the \textsc{Inject}
rule. It states that if $\mathsf{after}$ for some predicate $Q$ holds
on a state $w_U$ from a ``small'' concurroid $V$, it will hold in a
large concurroid $U = V \opentangle W$ with an additional state,
reachable by the $\rely{W}$ relation of an addition $W$
(Definition~\ref{def:bbt}).

% Lemma aft_extend (p : tree V A) (P : A -> state -> Prop) i j :
%         i \+ j \In Mod.coh W ->
%         after i p P ->
%         after (i \+ j) (Extend w p)
%           (fun v m => exists i' j', 
%              [/\ m = i' \+ j', i' \In Mod.coh V, 
%                  menv_steps W2 j j' & P v i']).


\begin{lemma}[Closure under injection]
\label{lm:aft-extend} 
%
For the states $w_V$ and $w_0$, such that $U = V \opentangle W$, $w_V
\in \coh{U}$, $w_0 \in \coh{W}$ and $w_V \hunion w_0 \in \coh{V}$.
%
If $\mathsf{after}_V~w_V~t~Q$, then 
%
\[
\begin{array}{rll}
\mathsf{after}_U~(w_V \hunion w_0)~(\mathsf{Inject}~t) &
  (\lambda~v'~w'\ldot~ \exists w_V'~w_0'\ldot & 
  w' = w_V' \hunion w_0' \aand w_V' \in \coh{U} ~\aand \\
  && (w_0, w_0') \in \rely{W} \aand Q~v'~w_V').
\end{array}
\]
%
\end{lemma}

Closure under hiding justifies the rule \textsc{Hide}. It is important
to notice that in the final state $w_V'$ of the refined execution in
$V$ is related to the final state $w_U'$ of the coarse execution in
$U$ by the same refinement function $\Ghi$
(Definition~\ref{def:hid-pred}), but with different auxiliaries $g'$,
hence the conjunct $\Ghi(g')(w_U', w_V')$.

% In the statement, we assume that the refinement function $\Ghi$ and
% the hiding function $\Phi$ a related by Lemma~\ref{lm:hiding}.

% Lemma aft_hide (p : tree V A) (P : A -> state -> Prop) i i' gL : 
%         f gL i i' -> 
%         after i' p P ->
%         after i (Hide (Retracts gL f) p) (fun v m =>
%            exists m' gL', f gL' m m' /\ P v m').


\begin{lemma}[Closure under hiding]
\label{lm:aft-hide}
%
If~~$\Ghi(g)(w_U, w_V)$, such that $\mathsf{refines}~U~V~\Ghi$, and 
$\mathsf{after}_V~w_V~t~Q$, then
%
\[
\mathsf{after}_U~w_U~(\mathsf{Hide}~\Phi~g~t)~
(\lambda v~w_U'\ldot~ \exists w_V'~g' \ldot \Ghi(g')(w_U', w_V') \aand Q~v~w_V')
\]
  
\end{lemma}

%\is{TODO: frame-lemma}

% Lemma aft_zagzig A s p (t : tree A) Q : 
%         after (s \> p) t Q ->
%         after (s \< p) t (fun v s => 
%           exists s', s = s' \< p /\ Q v (s' \> p)).




Finally, we can formulate and prove the main theorem. The theorem
essentially states that if a judgment $\Gamma \vdash J$ can be
derived in \SCST, its translation (Figure~\ref{fig:denj}) can be also derived in CiC.

\begin{theorem}[Soundness]
\label{thm:soundness}
If~~$\Gamma \vdash J$, then $\mean{\Gamma \vdash J}$.
\end{theorem}
\begin{proof}
  The proof goes by induction on the derivation of $J$. The proof for
  pure expression is straightforward, since $\mean{\Gamma \vdash e:
    A}$ is just $\mean{\Gamma} \vdashCiC e: A$. 
  % 
  So let us focus on the judgments of the form $\Gamma \vdash
  \spec{p}\,c\, \ty \,A\,\spec{q}@U$ that check the specifications of
  \SCST commands (Figure~\ref{fig:rules}).

  Each basic command (\eg, $\mathsf{return}$, $\mathsf{act}$) is sound
  because its pre- and postconditions are stable under environment
  interference, the precondition implies the command is
  $\mathsf{safe}$ (by Definition~\ref{def:htypes} and through the
  \textsf{after} predicate), and the resulting configuration satisfies
  the postcondition.
  %
  The \textsc{Seq}, \textsc{Inject} and \textsc{Hide} rules are sound by
  Lemmas~\ref{lm:aft-bnd},~\ref{lm:aft-extend} and~\ref{lm:aft-hide}.
  %
  The \textsc{Par} rule is sound by Lemma~\ref{lm:aft-par}, as the
  \self/\other exchange in the lemma accounts exactly for the
  interpretation of the subjective separate conjunction $\ssep$
  (Figure~\ref{fig:broccoli}), and the rule \textsc{Frame} is just a
  particular case of \textsc{Par} with $c_2$ being an idle thread
  (\eg, $\mathsf{return}~();$), hence the stability requirement.
  % 
  The $\mathsf{fix}$ rule is sound by the Knaster-Tarski theorem. 
  % 
  The \textsc{Conj} rule is sound by Lemma~\ref{lm:universal}, and
  the \textsc{Conseq} rule by Lemma~\ref{lm:aft-imp}. 
  %
  The \textsc{Exist}ential rule and \textsc{If} rules are derivable. 
  % 
  Since \SCSL procedures are interpreted as (monadic) CiC functions.
  the procedure \textsc{App}lication and \textsc{Hyp}othesis rules are
  sound by the function application and hypothesis rules of CiC.  
  % 
  \qed
\end{proof}
