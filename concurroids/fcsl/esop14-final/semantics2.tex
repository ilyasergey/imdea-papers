\section{Semantics and Soundness}
\label{sec:semantics}

The semantic model for \SCST programs largely relies on the
denotational semantic of \emph{action
  trees}~\cite{LeyWild-Nanevski:POPL13}, which implement finite,
partial approximations of the behavior of \SCST commands.  Action
trees are a generalization of the Brookes' action traces in the
following sense. Where action trace semantics approximate a program by
a set of traces, we approximate with a set of trees. A tree differs
from a trace in the following: a trace is a sequence of actions and
their results, whereas a tree contains an action followed by a
\emph{continuation} which itself is a tree parametrized wrt.~the
output of the action. In this appendix we introduce a number of
notions in order to formalize the safety of \SCST.

The high-level view of the development is as follows. We first define
semantic behavior (\ie, operational semantics) for action trees wrt.~a
program state (Section~\ref{sec:opsem}). We consider program state
that, in addition to concrete heaps includes the logical information
such as auxiliary state, or division of state between concurroid
labels. Thus, the operational semantics of trees is
\emph{instrumented}. In Section~\ref{sec:ersem}, we prove the standard
erasure result of shared-memory concurrency logics, whereby logical
information in program states doesn't influence the result of the
computation.

Taking the low-level operational semantics of threes from
Section~\ref{sec:opsem} as a base for our soundness result, we relate
it to the high-level transitions of a concurroid by an
$\mathsf{always}$ predicate (Section~\ref{sec:modal-pred}) that
ensures a tree is resilient to any amount of a concurroid's
rely-interference, and that all operational steps by a tree are
\emph{safe} (\ie, providing an analogue to \emph{preservation}
property for the \emph{action safety} predicate) and correspond to
concurroid interference. %

The denotational semantics (Section~\ref{sec:denot-sem}) interprets
Hoare tuple judgments by the \emph{monadic Hoare type}
$\ST{p}{A}{q}{U}$, which is a \emph{complete lattice} of trees that
are $\mathsf{always}$-safe to run from any initial configuration that
satisfies precondition $p$ and if they terminate produce a final
configuration that satisfies postcondition $q$ (under possible
interference from programs respecting the transitions of the
concurroid $U$). The complete lattice structure makes the semantic
domain suitable for modeling recursion.  A command is denoted by the
set of its tree approximations, and a procedure is denoted by a
function into a set of trees.
%
The \emph{soundness} of \SCST (Section~\ref{sec:soundness}) follows
from showing that the denotations of the commands listed in
Section~\ref{sec:logic} satisfy the appropriate $\mathsf{always}$
predicate (\ie, adhere to the \emph{progress-and-preservation}
property), and that $\mathsf{always}$ satisfies certain closure
conditions. 

We choose the Calculus of Inductive Constructions
(CiC)~\cite{coq-team,Bertot-Casteran:04} as our meta logic.  This has
several important benefits.  First, we can define a \emph{shallow
  embedding} of \SCST into CiC that allows us to program and prove
directly \emph{with the semantic objects}, thus immediately lifting
\SCST to a full-blown programming language and verification system
with higher-order functions, abstract types, abstract predicates, and
a module system.  We also gain a powerful dependently-typed
$\lambda$-calculus, which we use to formalize all semantic definitions
and metatheory, including the definition of action trees by
\emph{iterated inductive definitions}~\cite{coq-team},
specification-level functions, and programming-level higher-order
procedures.
%
Finally, we were able to mechanize the entire semantics and
metatheory~\cite{fcsl-coqscripts} in the Coq proof assistant
implementation of~CiC.

Following~\cite{LeyWild-Nanevski:POPL13}, we use set-theoretic and
type-theoretic notation as appropriate.
%
The reader unconcerned with the fine points of the type theory, may
read a typing judgment $x \ty A$ as a set membership predicate $x \in
A$.


\subsection{Action trees and their operational semantics}
\label{sec:opsem}
%
The type family $\mathsf{tree}~U~A$ of $A$-returning action trees is
defined by an iterated inductive definition, as follows.

\begin{definition}[Action trees]
\label{def:trees}
\[
\begin{array}{lcr}
  \mathsf{Unfinished} & : & \mathsf{tree}~U~A
  \\
  \mathsf{Ret}\ (v \ty A) & : & \mathsf{tree}~U~A
  \\
  \mathsf{Act}~(a \ty (U, A, \sigma, \mu)) & : & \mathsf{tree}~U~A
  \\
  \mathsf{Seq}~(t \ty \mathsf{tree}~U~B)~(k \ty B \arrow \mathsf{tree}~U~A) & : & 
  \mathsf{tree}~U~A
  \\
  \mathsf{Par}~(t_1 \ty \mathsf{tree}~U~B_1)\ (t_2 \ty \mathsf{tree}~U~B_2)\ (k \ty B_1 \times B_2 \arrow \mathsf{tree}~U~A) & : & 
  \mathsf{tree}~U~A 
  \\
  \mathsf{Inject}~(t \ty \mathsf{tree}~V~A) &:&
  \mathsf{tree}~U~A 
  \\
  \mathsf{Hide}_{\Ghi,g}~(t \ty \mathsf{tree}~V~A) &:&
  \mathsf{tree}~U~A 
\end{array}
\]  
\end{definition}

Most of the constructors in Definition~\ref{def:trees} are
self-explanatory.
%
Since trees have finite depth, they can only \emph{approximate}
divergent computations, thus the $\mathsf{Unfinished}$ tree indicates
an incomplete approximation. %
%
$\mathsf{Ret}\ v$ is a terminal computation that returns value
$v\,{:}\,A$. %
%
The constructor $\mathsf{Act}$ takes as a parameter an action $(U, A,
\sigma, \mu)$, defined in concurroid $U$~(Section~\ref{sec:logic}).
%
$ \mathsf{Seq}~t~k$ sequentially composes a $B$-returning tree $t$
with a continuation $k$ that takes $t$'s return value and generates
the rest of the approximation.
%
$\mathsf{Par}\ t_1\ t_2\ k$ is the parallel composition of trees $t_1$
and $t_2$, and a continuation $k$ that takes the pair of their results
when they join. %
CiC's iterated inductive definition permits the recursive occurrences
of $\mathsf{tree}$ to be \emph{nonuniform} (\eg, $\mathsf{tree}\ B_i$
in $\mathsf{Par}$) and \emph{nested} (\eg, the \emph{positive}
occurrence of $\mathsf{tree}\ A$ in the continuation). %
Since the CiC function space $\arrow$ includes case-analysis, the
continuation may branch upon the argument, which captures the pure
computation of conditionals. %
This closely corresponds to the operational intuition and leads to a
straightforward denotational semantics.
%

The $\mathsf{Inject}$ constructor embeds a tree $t \ty
\mathsf{tree}~V~A$ (of a \emph{different}, \ie, smaller, concurroid
$V$) for the underlying computation and generates a tree in the
concurroid $U$.  $V$ and $U$ are not arbitrary, but it has to be
possible to \emph{inject} $V$ into $U$, as we shall define shortly.
In particular, we will be able to prove that we can inject $V$ into $U
= V \entangle W$ for any $W$. 

%
The $\mathsf{Hide}$ constructor embeds a \emph{generalized refinement
  function} $\Ghi$, an abstract state $g$ and a tree $t \ty
\mathsf{tree}~V~A$ for the underlying computation and generates a tree
in the concurroid $U$. Again, $U$ and $V$ are not arbitrary but it has
to be possible to \emph{refine} $V$ into $U$. In particular, we will
be able to prove the we can refine $U = P \entangle V_1 \entangle V_2$
into $V = P \entangle V_1$.

In general, in Coq, the proof that $V$ injects into $U$, or that $V$
refines $U$, has to be an annotation on the $\mathsf{Inject}$ and
$\mathsf{Hide}$ constructors, but we omit the annotations for brevity.



\subsection{Injection and Refinement}
\label{sec:prop-inject-hiding}

We define the predicate $\mathsf{injects}~V~U$, which intuitively
means that the ``larger'' concurroid $U$ can be considered as an
entanglement of the ``smaller'' concurroid $V$ with some additional
concurroid $W$. 

\begin{definition}[Injection predicate]
\label{def:inj-pred}

$\mathsf{injects}~V~U$ $\iff$ there exists $W$, such that the
following three statements hold:

\begin{enumerate}

\item $w \in {\cal W}_U \iff w = w_1 \hunion w_2 \aand w_1 \in
  {\cal W}_V \aand w_2 \in  {\cal W}_W$;

\item $\forall w_1~w_2~w.~w_1 \hunion w \in {\cal W}_U \aand w_2
  \hunion w \in {\cal W}_U \aand (w_1, w_2) \in \tau_V \implies
  (w_1 \hunion w, w_2 \hunion w) \in \tau_U$;

\item $\forall w_1~w_1'~w_2~w_2'.$ 

\[
\begin{array}{cl}
s_1 \in {\cal W}_V \aand s_2 \in
  {\cal W}_V \aand (s_1 \hunion s_1', s_2 \hunion s_2') \in \guar{U} &
  \implies \\
  (s_1, s_2) \in \guar{V} \aand (s_1', s_2') \in \guar{W}, & \text{ where }
\end{array}
\]
%
$\guar{X}$ is a \emph{guarantee} relation of a concurroid $X$
(Definition~\ref{def:guarantee}).

\end{enumerate}

\end{definition}
The first requirement ensures that the states of $U$ can be
constructed as Cartesian products of states of $V$ and $W$. The second
requirement ensures that internal transitions of the smaller
concurroid~$V$ do not break the coherence of it enclosing
concurroid~$U$. The third requirement ensures that the transitions
within $U$ can be always uniquely split to the transitions done by $V$
and by the ``addition'' $W$.
%
Notably, the first requirement ensures that if $\mathsf{injects}~V~U$
holds, there always exists a ``smaller'' state $w_V$, such that $w_V
\in {\cal W}_V$ for any $w_U \in {\cal W}_U$.

We will also introduce another auxiliary definition in order to have a
handle to the existential concurroid $W$ in
Definition~\ref{def:inj-pred}.

\begin{definition}
\label{def:bbt}
%
We will write $U = V \opentangle W$, when $\mathsf{injects}~V~U$ and
$W$ is an additional concurroid from Definition~\ref{def:inj-pred}. 
\end{definition}

The notation $\opentangle$ hints that already familiar entanglement
operators $\entangle$ and $\rentangle$ can be seen as particular cases
of $\opentangle$.

\begin{lemma}$\mathsf{injects}~V~{(V \entangle W)}$ for every $V$ and $W$ with disjoint label sets. \end{lemma}

We next define the refinement relation on concurroids. The concurroid
$U$ is refined into $V$, if we can, intuitively, elaborate the states
of $U$ into those of $V$. In other words, if $V$'s states can be seen
to contain the states of $U$, plus some other additional
state. Abstractly, we capture the dependence by positing an
elaboration predicate $\Ghi$ between the state spaces of $U$ and
$V$, with a number of properties shown below. Additionally, $\Ghi$
takes a value $g : G$ of user-specified type, which is an abstraction
of the mentioned additional state. Thus, we read $\Ghi(g)(w, w')$
to say that $g$ elaborates $w$ into $w'$ (via $\Ghi$).

\begin{definition}[Elaboration predicate]
\label{def:hid-pred}

Given a type $G$ of abstract states and a function $\Ghi: G
\rightarrow ({\cal W}_U, {\cal W}_V) \rightarrow \mathsf{prop}$,
$\mathsf{refines}~U~V~{\Ghi}$ $\iff$ the following statements
hold:

\begin{enumerate}
%1
%\forall g, w, w_1, w_2, 
\item $\Ghi(g)(w, w_1) \aand \Ghi(g)(w, w_2) 
\implies w_1 = w_2;$

%2 
%\forall g_1, w_1, g_2, w_2, w, 
\item $\Ghi(g_1)(w_1, w) \aand
  \Ghi(g_1)(w_2, w) \implies g_1 = g_2 \aand w_1 = w_2;$

%3
%\forall g, w, w', 
\item $\Ghi(g)(w, w') \implies w \in {\cal W}_U
  \aand w' \in {\cal W}_V;$

%4
% \forall g, w_1, w_1', w_2, 
\item $\Ghi(g)(w_1, w_1') \aand (w_1, w_2)
  \in \rely{U} \implies \exists w_2', \Ghi(g)(w_2, w_2') \aand (w_1',
  w_2') \in~\rely{V};$

%5
% \forall g1, w1, w1', g2 w2, w2',
\item $\Ghi(g_1)(w_1, w_1') \aand (w_1',
  w_2') \in \tau_V \implies \exists g_2\ w_2\ldot \Ghi(g_2)(w_2, w_2') \aand (w_1, w_2) \in \tau_U;$
%\item $\Ghi(g_1)(w_1, w_1') \aand \Ghi(g_2)(w_2, w_2') \aand (w_1',
%  w_2') \in \tau_V \implies (w_1, w_2) \in \tau_U;$

%6
% \forall g1, w1, w1', g2, w2, w2'
\item $\Ghi(g_1)(w_1, w_1') \aand \Ghi(g_2)(w_2, w_2') \implies
  (w_1.\mathother = w_1.\mathother \iff w_1'.\mathother = w_2'.\mathother);$

%7 
\item $\forall p. \exists q.$
\[
\begin{array}{llc}  
\forall g~w~w'. & \Ghi(g)(w \zag p, w') \implies
\left(\exists w_1', w' = w_1' \zag q \aand \Ghi(g)(w \zig p, w_1' \zig
  q) \right)
& ~\aand \\
\forall g~w~w'. & \Ghi(g)(w, w' \zig q) \implies
\left(\exists w_1, w = w_1 \zig p \aand \Ghi(g)(w_1 \zag p, w' \zag
  q) \right); 
\end{array}
\]

%8
\item 
% \forall g, w, w',
$\Ghi(g)(w, w') \implies \flatten{w} = \flatten{w'};$


\end{enumerate}

\end{definition}

If the properties in Definition~\ref{def:hid-pred} are satisfied, we
call $\Ghi$ and elaboration predicate for $U$ and $V$.  In
Definition~\ref{def:hid-pred}, (1) states that $\Ghi$ uniquely
determines refined states (however, some states in $U$ need not refine
into anything in $V$, \ie, $\Ghi$ is a partial function); (2)
states that the $\Ghi$ is injective: abstracting back from a
refinement is unique. Thus, (1) and (2) state that $\Ghi$ is a
partial bijection. (3) ensures that the elaboration maps well-formed
states to well-formed states; (4) specifies that environment steps in
the ``coarse'' world do not change introduced abstractions in the
``fine'' world, as environment steps in the coarse world can't see the
refinement; (5) states that ``abstracting back'' preserves internal
transitions. The properties (4) and (5) can be seen as stating
commutativity of the apparent categorical diagrams. They may also be
seen as stating simulation properties. In (4), a rely transitions of
$U$ can always be matched by a rely transition of $V$. In (5), an
internal transition of $V$ may always be matched by an internal
transition of $U$.  (6) postulates that refinement realigns \self and
\joint parts, but doesn't move things into/out of \other; (7) is a
version of framing, namely, every every extension to \other of a
coarse state is is uniquely refined into an addition to fine state;
finally, (8) ensures that refinement only deals with abstract parts of
the state and does not change the heap.

\begin{lemma}[Hiding and refinement]
\label{lm:hiding}

  Let $U$ and $V$ be concurroids, and let $\Phi$ be an abstraction
  function from a \textsc{Hide}; that is, $\Phi$ satisfies the
  properties about concurroid $V$ from
  Section~\ref{sec:phi-properties}. Let $\Ghi$ be constructed as
  follows.
\[
\begin{array}{l}
\Ghi (g)(w, w') \iff 
 \exists h\ w_1\ w_2,
   \begin{array}[t]{lcl}
   w & = & (\hpriv \hpts (h \hunion \flatten {w_1})) \hunion w_2\aand\hbox{}\\
   w' & = & (\hpriv \hpts h) \hunion w_1 \hunion {w_2}\aand\hbox{}\\
   w_1 & \in & \Phi(g)\aand\hbox{}\\
   w & \in & {\cal W}_{P \entangle U} \aand w' \in {\cal W}_{{P \entangle U}\entangle V}
   \end{array}
\end{array}
\]
Then $\Ghi$ is a valid elaboration predicate, that is
$\mathsf{refines}~(P \entangle U)~(P \entangle U \entangle
V)~\Ghi$.
\end{lemma}

\subsection{Operational semantics of action trees}
The judgment for small-step operational semantics of action trees has
the form $w; t \tsteps{\Path} w', t'$ (Figure~\ref{fig:tsteps}). It
operates on program states $w$ and paths $\Path$ to step the tree $t$
from initial state $w$ to a reduced tree $t'$ in ending state $w'$.
Intuitively, the path $\Path$ identifies the position in the tree to
be reduced, which may be an action, or a beta redex.
\[
\begin{array}{rclclclcl}
\Path & ~::=~ & \ChoiceAct &~~|~~& \SeqRet &~~|~~& \SeqStep~\Path &~~|~~&   \\
&&                   \ParRet &~~|~~& \ParL~\Path &~~|~~& \ParR~\Path &~~|~~& \tag{\arabic{tags}}\refstepcounter{tags} \\
&&                   \HideStep~\Path &~~|~~& \HideRet &~~|~~&  \InjStep~\Path &~~|~~& \InjRet .
\end{array}
\label{eq:path}
\]
%
Stepping is undefined for the $\mathsf{Unfinished}$ and $\mathsf{Ret}$
trees. %
For the $\mathsf{Seq}$ and $\mathsf{Par}$ trees, the path $\Path$
selects a $\beta$-redex and performs the appropriate reduction. 
%
For the $\mathsf{Inject}$ tree in the non-$\mathsf{Ret}$ case the
reduction of the tree $t_1$ is performed for a smaller state $w_V$
instead of the large state $w_U$, and the resulted state $w_V'$ is
``plugged'' back using the addition $w_0$ in order to obtain the
resulting large state $w_U'$. The situation for $\mathsf{Hide}$ is
symmetric: the coarse-grained state $w_U$ and the fine-grained state
$w_V$ are related by the elaboration $\Ghi$. Note that the
modifications of the changes to the abstract values $g$ are recorded
into $g'$.

\begin{figure}[t] 
\figsize
\figStepTree
\caption{Tree stepping $w; t \tsteps{\Path} w'; t'$ with respect to
  the path $\Path$}
\label{fig:tsteps}
\end{figure} 

The decorated semantics in Figure~\ref{fig:tsteps} may not make a step
for two different reasons. The first, benign, reason is that the the
chosen path $\Path$ doesn't actually determine an action or a redex in
the tree $t$. For example, we may have $t = \mathsf{Unfinished}$ and
$\Path = \mathsf{ParR}$. But we can choose the right side of a
parallel composition only in a tree whose head constructor is
$\mathsf{Par}$, which is not the case with $\mathsf{Unfinished}$.  We
consider such paths that don't determine an action or a redex in a
tree ill-formed.

The second reason arises when $\Path$ is actually well-formed. In that
case, the constructors of the path uniquely determine a number of
rules of the operational semantics that should be applied to step the
tree. However, what is some of the rules can't be applied because
their side-conditions are not satisfies by the current state.

We next state a form of the progress property, showing that the latter
case can't occur in trees that approximate well-proved programs. We
first formally define paths which are well-formed wrt.~a tree.

\begin{definition}[Good paths]
\label{def:good}
Let $t \ty \mathsf{tree}~U~A$ and $\Path$ be a path.  Then the
predicate $\mathsf{good}~t~\Path$ is defined as follows:
\[
\begin{array}{rllcl}
  \mathsf{good} & (\mathsf{Act}~(U, A, \sigma, \mu)) & \ChoiceAct
  & ~\eqdef~ & \mathsf{true}   
  \\
  \mathsf{good} & (\mathsf{Seq}~(\mathsf{Ret}~v)~\_) & \SeqRet
  & ~\eqdef~ & \mathsf{true}  
  \\
  \mathsf{good} & (\mathsf{Seq}~t~\_) & \SeqRet~\Path 
  & ~\eqdef~ & \mathsf{good}~t~\Path 
  \\
  \mathsf{good} & (\mathsf{Par}~(\mathsf{Ret}~\_)~(\mathsf{Ret}~\_)~\_) & \ParRet 
  & ~\eqdef~ & \mathsf{true}  
  \\
  \mathsf{good} & (\mathsf{Par}~t_1~t_2~\_) & \ParL~\Path
  & ~\eqdef~ & \mathsf{good}~t_1~\Path
  \\
  \mathsf{good} & (\mathsf{Par}~t_1~t_2~\_) & \ParR~\Path 
  & ~\eqdef~ & \mathsf{good}~t_2~\Path
  \\
  \mathsf{good} & (\mathsf{Inject}~(\mathsf{Ret}~\_)) & \InjRet
  & ~\eqdef~ & \mathsf{true}
  \\
  \mathsf{good} & (\mathsf{Inject}~t) & \InjStep~\Path 
  & ~\eqdef~ & \mathsf{good}~t~\Path 
  \\
  \mathsf{good} & (\mathsf{Hide}~\_~\_~(\mathsf{Ret}~\_)) & \HideRet
  & ~\eqdef~ & \mathsf{true}
  \\
  \mathsf{good} & (\mathsf{Hide}~\_~\_~t) & \HideStep~\Path 
  & ~\eqdef~ & \mathsf{good}~t~\Path 
  \\
  \mathsf{good} & t & \Path 
  & ~\eqdef~ & \mathsf{false} \text{~~otherwise}
\end{array}
\]

\end{definition}

We next require a notion of safety for a tree and a path.

\begin{definition}[Safe path for an action tree]
\label{def:safe}
Let $t \ty \mathsf{tree}~U~A$, $\Path$ be a path $\Path$, and state $w
\in {\cal W}_U$.  $\mathsf{safe}~t~\Path~w$ is defined as follows:
\[
\begin{array}{rlllcl}
  \mathsf{safe} & (\mathsf{Act}~(U, A, \sigma, \mu)) & \ChoiceAct & w 
  & ~\eqdef~ & (w \in \sigma) 
  \\
 % \mathsf{safe} & (\mathsf{Ret}~v) & \mathsf{nil} & w 
 % & ~\eqdef~ & \mathsf{true}  
 % \\ 
  \mathsf{safe} & (\mathsf{Seq}~(\mathsf{Ret}~v)~\_) & \SeqRet & w 
  & ~\eqdef~ & \mathsf{true}  
  \\
  \mathsf{safe} & (\mathsf{Seq}~t~\_) & \SeqStep~\Path & w 
  & ~\eqdef~ & \mathsf{safe}~t~\Path~w  
  \\
  \mathsf{safe} & (\mathsf{Par}~(\mathsf{Ret}~\_)~(\mathsf{Ret}~\_)~\_) &\ParRet & w 
  & ~\eqdef~ & \mathsf{true}  
  \\
  \mathsf{safe} & (\mathsf{Par}~t_1~t_2~\_) & \ParL~\Path~ & w 
  & ~\eqdef~ & \mathsf{safe}~t_1~\Path~w
  \\
  \mathsf{safe} & (\mathsf{Par}~t_1~t_2~\_) & \ParR~\Path~ & w 
  & ~\eqdef~ & \mathsf{safe}~t_2~\Path~w
  \\
  \mathsf{safe} & (\mathsf{Inject}~(\mathsf{Ret}~\_)) & \InjRet & w 
  & ~\eqdef~ & \mathsf{true}
  \\
  \mathsf{safe} & (\mathsf{Inject}~(t \ty \mathsf{tree}~V~A)) & \InjRet~\Path & w 
  & ~\eqdef~ & \mathsf{injects}~V~U  \aand w = w_V \hunion w_0 ~\aand \\ 
  &&&&& \mathsf{safe}~t~\Path~w_V
  \\
  \mathsf{safe} & (\mathsf{Hide}~\_~\_~(\mathsf{Ret}~\_)) & \HideRet & w 
  & ~\eqdef~ & \mathsf{true}
  \\
  \mathsf{safe} & (\mathsf{Hide}~\Phi~g~(t \ty \mathsf{tree}~V~A)) & \HideStep~\Path & w 
  & ~\eqdef~ & \mathsf{hides}~U~V~\Phi \aand \Phi(g)(w, w_V) ~\aand \\
  &&&&& \mathsf{safe}~t~\Path~w_V
  \\
  \mathsf{safe} & t & \Path & w 
  & ~\eqdef~ & \mathsf{true}  \text{~~otherwise}
  % \\
  % \mathsf{safe} & t & \Path & w 
  % & ~\eqdef~ & \mathsf{true} \text{~~otherwise}
\end{array}
\]

\end{definition}

The definition traverses the path and the tree, and in those cases
when the tree and the path match, collects the side-conditions on the
\emph{initial states} of the operational semantics rules.

\begin{lemma}[Progress on action trees]
\label{thm:progress}
For any concurroid $U$, state $w \in {\cal W}_U$, action tree $t \ty
\mathsf{tree}~U~A$ and a path $\Path$, if $\mathsf{safe}~t~\Path~w$,
then either of the following holds
%
\begin{enumerate}
\item $\neg~ \mathsf{good}~t~\Path$;
\item $\exists w'~t',~ w;t \tsteps{\Path} w';t'$.
\end{enumerate}
\end{lemma}

The lemma shows that if we can match the tree $t$ with a path $\Path$
and thus manage to pick an action or a redex (\ie,
$\mathsf{good}~t~\Path$), then the safety predicate implies the side
conditions of the inference rule determined by $\Path$.

The import of the lemma is in allowing that safety and stepping
relations may be defined independently of each other. The difficulty
of the proof consists in showing that the various conditions from the
reduction rules, that have \emph{not} been incorporated into the
safety predicate, are nevertheless satisfied. For example, the
reduction rules for \textsf{Inject} and \textsf{Hide} contain such
side conditions on the \emph{ending} states of their premisses. We
have proved that even though these conditions are not incorporated by
the $\mathsf{safe}$ predicate, they are implied by it. The proof
relies on the general properties of injection and refinement listed in
Section~\ref{sec:prop-inject-hiding}.

Defining safety independently from stepping will enable us to develop
a semantics in the style of Milner and Wright and Felleisen, whereby
we only give meaning to programs that are well-proved in \SCST,
analogously to how Milner and Wright-Felleisen semantics give meaning
to only well-typed programs. As the type of the program, we will use
its Hoare-tuple specification, and the denotation of a program wrt.~a
specification with precondition $p$, will only contain approximating
action trees whose safety predicate is implied by $p$. The fact that
the safety predicate actually means safety, \ie, it allows stepping,
is a consequence of the Progress lemma above.

\subsection{Erased \SCST action trees}
\label{sec:ersem}

% \begin{figure}[t] 
%  \figsize
% \figStepAction
% \rule{1.0\textwidth}{0.2mm}
% \figStepETree
% \caption{Semantics of actions $h; a \asteps h'; v$ (top) and erased
%   trees $h; t \etsteps{\Path} h'; t'$ (bottom).}
% \label{fig:esteps}
% \end{figure} 

% The operational semantics of primitive actions $h; a\,\asteps\,h'; v$
% (Figure~\ref{fig:esteps}, top) mimics the specifications of actions
% given in Definition~\ref{def:actions} (Section~\ref{sec:actions}),
% assuming that all logical actions we consider are \emph{operational}
% (Definition~\ref{def:opact}).
% %
% More precisely, steps an $A$-returning action $a$ relative to an
% initial heap $h$, and produces an ending heap $h'$ and result $v$ of
% type $A$. %
% The operational semantics of the basic memory operations is
% standard. %
% The \emph{read-modify-write} action $\mathsf{RMW}^{A~B}_{x~f~g}$
% atomically replaces the current value $v$ of the pointer $x$ with
% $f(v)$ of type $A$ for some pure function $f$, and returns the result
% according to the function $g$ of type $B$. As usual, we will use the
% notation $\hpExt{h}{x}{v}$ to indicate the strong update of the
% pointer $x$ in the heap $h$ with value $v$.

We next define an erased version of action trees that remove the
various logical annotations. One example of such erased logical
information is the proofs of injection and refinement that we
implicitly elided from the \textsc{Inject} and \textsc{Hide}
constructors. Another example is the logical information from the
actions embedded into action trees, which, as we have shown, operate
on \emph{decorated} states. 

We then prove that such logical information doesn't influence the
stepping of the tree. In other words, states that erase to equal
concrete heaps, produce output states that also erase to equal
concrete heaps.

\begin{definition}[Tree erasure]
\label{def:tree-erasure}

Given a tree $t \ty \mathsf{tree}~U~A$, its erasure $\flatten{t} \ty
\mathsf{etree}~A$ is defined as follows:

\[
\begin{array}{lcl}
  \flatten{\mathsf{Unfinished}}  & \eqdef &  \mathsf{Unfinished} 
  \\
  \flatten{\mathsf{Ret}~v}  & \eqdef &  \mathsf{Ret}~v 
  \\
  \flatten{\mathsf{Act}~(U, A, \sigma, \mu)}  & \eqdef &  
  \mathsf{Act}~(A, \flatten{\sigma}, \flatten{\mu}) 
  \\
  \flatten{\mathsf{Seq}~t~k} & \eqdef & 
  \mathsf{Seq}~\flatten{t}~(\lambda v. \flatten{k~v})
  \\
  \flatten{\mathsf{Par}~t_1~t_2~k} & \eqdef &
  \mathsf{Par}~\flatten{t_1}~\flatten{t_1}~(\lambda v. \flatten{k~v}) 
  \\
  \flatten{\mathsf{Inject}~t} & \eqdef &
  \mathsf{Inject}~\flatten{t}
  \\
  \flatten{\mathsf{Hide}~\_~\_~t}  & \eqdef &
  \mathsf{Hide}~\flatten{t}
  \\
\end{array}
\] 
  
\end{definition}

\begin{theorem}[Erasure irrelevance]
\label{thm:erasure-irrel}
For a concurroid $U$, states $w_1, w_2 \in {\cal W}_U$ and action
trees $t_1, t_2 \ty \mathsf{tree}~U~A$, if $\flatten{w_1} =
\flatten{w_2}$, $\flatten{t_1} = \flatten{t_2}$, 
%
$w_1; t_1 \tsteps{\Path} w_1'; t_1'$ and
%
$w_2; t_2 \tsteps{\Path} w_2'; t_2'$, then 
%
$\flatten{w_1'} = \flatten{w_2'}$ and $\flatten{t_1'} = \flatten{t_2'}$.
\end{theorem}

Ideally, we will need to prove a stronger result, stating that if the
tree and state evaluate under semantics from Figure~\ref{fig:tsteps},
so do their erased counterparts under some sort of erased
semantics. However, we have not carried out this exercise due to time
constraints and decided to leave it for the future work.

% The small-step operational semantics of erased trees $h; t
% \etsteps{\Path} h', t'$ (Figure~\ref{fig:esteps}) operates on
% flattened heap $h$ rather than rich states $w$. As its rich
% counterpart, it is defined inductively on $\Path$, and steps a tree
% $t$ from initial heap $h$ to a reduced tree $t'$ in ending heap~$h'$. 


\subsection{Auxiliary modal predicates}
\label{sec:modal-pred}

In this section we next define a number of \emph{modal} predicates
over \emph{all} possible steps of execution to relate the operational
semantics over heaps and concurroid transitions over admissible
states.

\begin{definition}[Modal predicates]
  \label{def:modpred} 
%
  The predicates $\mathsf{always}^{\Paths}_U~w~t~P$,
  $\mathsf{always}_U~w~t~P$ and $\mathsf{after}_U~w~t~Q$ are defined
  relative to a schedule $\Paths$, concurroid $U$, state $w$,
  $A$-returning rich tree $t \ty \mathsf{tree}~U~A$ and arbitrary
  predicates $P \ty \mathsf{state} \to \mathsf{tree}~U~A \to
  \mathsf{prop}$ and $Q \ty A \to \mathsf{state} \to \mathsf{prop}$:

\[
\begin{array}{l}
\begin{array}{rcl}
\mathsf{always}^{\Paths}_U~w~t~P & \eqdef & 
\mathsf{if}~\Paths = \Path :: \Paths' 
\\ 
&& \mathsf{then}~\forall w_2. (w, w_2) \in \rely{U} \implies 
\\
&& ~~~~
   ~~~~~~~
   \begin{array}{l}\mathsf{safe}~t~\Path~w_2 ~~\aand \\
   \left(\forall w_3~t_2.~~ w_2; t \tsteps{\Path} w_3; t_2 \implies
   \mathsf{always}^{\Paths'}_U~w_3~t_2~P \right)
   \end{array}
\\
&& \mathsf{else}~ \forall w_2. (w, w_2) \in \rely{U} \implies P~w_2~t
\end{array}
\\
\\
\mathsf{always}_U~w~t~P  \eqdef \forall \Paths.
\mathsf{always}^{\Paths}_U~w~t~P
\\
\\
~~~\mathsf{after}_U~w~t~Q \eqdef \mathsf{always}_U~w~t~(\lambda~w'~t.~\forall v'. (t' = \mathsf{Ret}~v') \implies Q~v'~w') 
\end{array}
\]

\end{definition}

The predicate $\mathsf{always}_U~w~t~P$ expresses the fact that
starting from the state $w$, the tree $t$ remains memory-safe and the
user-chosen predicate $P$ holds of all intermediate configurations and
trees, for any schedule $\Paths$ and under any environment of the
concurroid $U$. %
The helper predicate $\mathsf{always}^{\Paths}_U~w~t~P$ is defined by
induction on $\Paths$: the concurroid $U$ is allowed to make arbitrary
rely-transitions from $w$ to $w_2$, in the resulting configuration the
predicate $P~w_2~t$ holds; moreover, if the schedule is $\Path {::}
\Paths'$, then the resulting configuration must have a state that is
\textsf{safe} for $t$, $\Path$ and $w_2$ (so, according to
Theorem~\ref{thm:progress}, it can step), and, if $t$ steps to $w_3$
and $t_2$, then the predicate recurses on $\Paths'$, $w_3$ and
$t_2$. One can notice that if the predicate
$\mathsf{always}^{\Paths}_U~w~t~P$, it automatically implies
``preservation'' of the $\mathsf{safe}$ predicate at each step of
reduction of the tree $t$.

The predicate $\mathsf{after}_U~w~t~Q$ encodes that $t$ is \textsf{safe};
however, $Q\ v'\ w'$ only holds if $t$ steps completely to
$\mathsf{Ret}\ v'$ in configuration $w'$.

