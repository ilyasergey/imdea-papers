
\section{A Concurroid for the Ticketed Lock}
\label{sec:ticketed}

In this section, we instantiate the presented framework by
constructing from scratch a concurroid for another famous fine-grained
locking algorithm---a ticketed lock~\cite{DinsdaleYoung-al:ECOOP10}.
Unlike the CAS-based spin lock, which we took as a running example,
this algorithm, although simple, provides fairness guarantees for
competing threads. We start by giving a simplified implementation of
the algorithm and building the intuition about the states and
transitions of the concurroid, implementing its protocol.

\subsection{Reference implementation}
\label{sec:refer-impl}

As a reference implementation of the ticketed lock, we use the one
from the work on Concurrent Abstract
Predicates~\cite{DinsdaleYoung-al:ECOOP10}. 

\begin{code}
 lock = {                        unlock = {
   n := TAKE();                   DROP_TKT();         
   while (!TRY(n)) skip;         }
 }  
\end{code} 

The three \emph{atomic} actions \texttt{TAKE}, \texttt{TRY(n)} and
\texttt{DROP\_TKT} have the following operational meaning:

\begin{code}
 TAKE()     = { fetch_and_increment(next); }
 TRY(n)     = { return (n == owner); }
 DROP_TKT() = { fetch_and_increment(owner); }  
\end{code} 

As a minor difference, we assume a pointer to the lock structure being
fixed and omit it from the code, referring instead to its two
essential fields: \texttt{owner} and \texttt{next}. These two fields
are natural numbers, describing the state of the ticketed lock system.
%
We implemented \texttt{TAKE()} and \texttt{DROP\_TKT()} as
\texttt{fetch\_and\_increment}, which is is a standard atomic operation
implemented in most of the modern architectures, operationally
equivalent to the following sequence of commands:

\begin{code}
 fetch_and_increment(ptr) = { tmp := ptr; ptr++; return tmp; }  
\end{code}


The intuition behind the protocol comes from the idea of the
organization of the service in a bakery: \texttt{next} corresponds to
the current ticket available at the dispenser (but not yet taken!),
whereas \texttt{owner} indicates a ticket number, an owner of which is
called to be served, or is currently being served. Hence, the
\texttt{lock} procedure corresponds to drawing a ticket from the
dispenser and a subsequent repetitive attempt to be served, whereas
\texttt{unlock} corresponds to throwing the ticket away and increasing
the ticket counter, so the next client could be served (if there is
one in a waiting line).

\subsection{States of the concurroid}
\label{sec:states-concurroid}

The intuition for the construction of a ticketed lock concurroid can
be built from the idea of \emph{splitting} the resources in the spirit
of the Subjective Concurrent Separation
Logic~\cite{LeyWild-Nanevski:POPL13}. What are the primary resources
that a ticketed lock deals with? Those are tickets, of course! Some of
them can be taken by a current thread, some by others. If we make an
assumption that each ticket is thrown away right after its owner has
been served, we will see a simple invariant: all tickets in the system
are from the range $[n_1, n_2)$, where $n_1$ and $n_2$ are natural
numbers, corresponding to the values of \texttt{owner} (currently
being served or about to be) and \texttt{next} (next ticked to be
issued), hence non-inclusion on the right side of the range.

As in the case of the CAS-based spin lock
(Section~\ref{sec:overview}), the \self and \other components of the
ticketed lock concurroid are pairs of two PCMs: the PCM of
\emph{finite sets} of natural numbers, accounting for the tickets
($\tL$ and $\tE$), and a generic PCM accounting in order to the heap
invariant associated with this particular lock: $\gL$ and
$\gE$. Similarly to the CAS-based spin lock, the ticketed lock is
parametrized by a resource invariant~$I$.
%
We will denote the whole concurroid with the label $\tlock$, pointers
$\own$ and $\nxt$ and invariant $I$ by $\TL$. Let us start by defining
a predicate, describing valid states of the ticketed lock concurroid.

\begin{definition}[$\TL$ state invariant ${\cal W}_T$] 
\label{def:tlock}
\[
\begin{array}{rcl} 
{\cal W}_T &\eqdef & \tlock \spts (\tL, \gL) \aand \tlock \opts (\tE, \gE) \aand 
\tlock \jpts (\own \hpts \angled{n_1, b}) \hunion (\nxt \hpts n_2) \hunion h ~ \aand \\ 
&& n_1 \le n_2 \aand (\tL \join \tE) = \set{n ~|~ n_1 \le n < n_2} ~\aand \\
&& 
\left(\begin{array}{l}
n_1 \in (\tL \join \tE) \aand b = \mathsf{true} \aand h = \hempty ~\oor \\
\mathsf{if}~~ n_1 = n_2 ~~\mathsf{then}~~ b = \mathsf{false} \aand I\
(\gL \join \gE)\ h \\
~~~~~~~~~~~~~~~~~~\mathsf{else}~~n_1 \in (\tL \join \tE) \aand b = \mathsf{false} \aand I~(\gL \join \gE)\ h
\end{array}\right)
\end{array}
\]
\end{definition}

Definition~\ref{def:tlock} highlights a number of important invariants
of the ticketed lock:

\begin{itemize}

\item The first line of the conjunction specifies the shape of the
  concurroid states. Since the \joint component is always a heap, we
  needed to introduce a ``ghost'' value $b$, such that $\own \hpts
  \angled{n_1, b}$, which tracks if the lock is taken or not.  This is
  a purely logical artifact, which simplifies the reasoning and it can
  be safely erased in the actual implementation, making this entry
  look just as $(\own \hpts n_1)$. To make it clear that $b$ is
  auxiliary state, we could have introduced a separate label whose
  \joint part contains only $b$. In our implementation, \joint
  components need not contain only heaps, but can contain arbitrary
  non-heap types, whose values are by default considered
  auxiliary. Thus, the addition of a new label with a boolean contents
  $b$ would suffice to declare $b$ as an auxiliary state. However, for
  simplicity, we omitted such generalization.

\item The second line states the invariant with respect to the
  distributed tickets. Indeed, the number of a ticket currently being
  (or just about to be) served is less than or equal to the number of
  the ticket in the dispenser, hence, $n_1 \le n_2$. The tickets in
  the queue range from $n_1$ to $n_2$, and $n_2$ has not yet been
  assigned to any thread.

\item The last three lines outline the essential ``states'' of the
  system. 

\begin{itemize}

\item One case is the lock being locked. It corresponds to a thread
  with some ticket $n_1$ (captured either by $\tL$ or $\tE$) being
  served. Therefore, the ownership over the heap is being transferred
  to the client ($h = \hempty$) and the logical flag $b$ is indicating
  the locking ($b = \mathsf{true}$). We will refer to this state as
  \emph{locked}.

  \item Another case corresponds to the situation when there are no
    more threads awaiting for service or being served. This is the
    case $n_1 = n_2$. Therefore, the lock is not taken ($b =
    \mathsf{false}$) and has the ownership over the heap with the lock
    invariant holding over it ($I~(\gL \join \gE)\ h$). We will refer
    to this state as \emph{unlocked}.

  \item The least obvious case is the situation when \emph{no} thread
    is being served, but one is just about to be. This is the thread
    having the ticket $n_1$ (hence $n_1 \in (\tL \join \tE)$), but the
    lock is not yet taken ($b = \mathsf{false}$) and has the
    ownership over the heap ($I~(\gL \join \gE)\ h$). We will refer to
    such state as \emph{transit}. 

\end{itemize}

\end{itemize}

It is important to notice that the awkward addition of the logical
flag $b$ to the \joint state is a byproduct of co-existence of
``locked'' and ``transit'' states of the concurroid and our generality
with respect to the invariant $I$. In particular, had we assumed
$I~(\gL \join \gE)\ h \implies h \neq \hempty$, we could use this fact
in order to distinguish between the ``locked'' and ``transit''
states. Finally, as it will be shown, the flag $b$ plays an important
role only when verifying the body of the \texttt{lock} procedure, and
can be abstracted away for convenience of the clients of \texttt{lock}
and \texttt{unlock}.

\subsection{Transitions of the concurroid}
\label{sec:trans-conc}

In contrast with the CAS-based spin lock, whose internal transitions
do not allow any components of the state to change, the internal
transitions of the ticketed lock concurroid account for \emph{taking}
the tickets from the dispenser and changing the corresponding counter
$\nxt$ in the \joint part of the state. Again, describing the
transitions on states $w$ and $w'$, we implicitly assume $w. \mathother = w'.\mathother$.

\begin{definition}[$\TL$ internal transitions $\tau_T$]
\label{def:tlock-int}
\[
\begin{array}{lcl}
(w, w') \in \tau_T & \iff & 
\begin{array}[t]{llll}
w.\mathself & = \tlock \hpts (\tL, \gL), & w.\mathjoint & = \tlock
\hpts (\nxt \hpts n_2 \hunion h), \\
w'.\mathself & = \tlock \hpts (\tL', \gL), & w.\mathjoint & = \tlock
\hpts (\nxt \hpts n'_2 \hunion h), \\
n_2 \le n'_2
\end{array}\\  
\end{array}
\]
  
\end{definition}

Notice that in Definition~\ref{def:tlock-int} we don't specify the
rest of the \joint heap $h$. What is important is that it remains
unchanged. The $\nxt$ counter may only increase, and this change is
reflected by changing the \self auxiliary component $\tL$ to
$\tL'$, assuming that the resulting state still adheres to the state
predicate ${\cal W}_T$ (Definition~\ref{def:tlock}).

\begin{definition}[$\TL$ external transitions, $\alpha_T$ and $\rho_T$]
\label{def:tlock-ext}

\[
\begin{array}[t]{lcl}
(w, w') \in \alpha_T~h & \iff & 
\begin{array}[t]{ll}
w.\mathself & = \tlock \hpts (\tL \cup \set{n_1}, \gL), \\
w.\mathjoint & = \tlock \hpts (\own \hpts \angled{n_1, \mathsf{true}} \hunion \nxt
\hpts n_2), \\
w'.\mathself & = \hlock \hpts (\tL, \gL'), \\
w'.\mathjoint & = \tlock \hpts (\own \hpts \angled{n_1 + 1, \mathsf{false}} \hunion \nxt
\hpts n_2 \hunion h)
\end{array}
\\ \\
(w, w') \in \rho_T~h & \iff & 
\begin{array}[t]{ll}
w.\mathself & = \hlock \hpts (\tL \cup \set{n_1}, \gL), \\
w.\mathjoint & = \tlock \hpts (\own \hpts \angled{n_1, \mathsf{false}} \hunion \nxt
\hpts n_2 \hunion h),
 \\
w'.\mathself & = \hlock \hpts (\tL \cup \set{n_1}, \gL), \\
w'.\mathjoint & = \tlock \hpts (\own \hpts \angled{n_1, \mathsf{true}} \hunion \nxt
\hpts n_2)
\end{array}
\end{array}
\]

\end{definition}

In Definition~\ref{def:tlock-ext}, the acquire transition $\alpha_T$
transfers the heap $h$ to the \joint part of the state $w'.j$,
increases the service counter, hence $n_1 + 1$, and flips the flag $b$
from \textsf{true} to \textsf{false}. The transition is only possible
when the ticket number $n_1$ is ``registered'' in the \self-set $\tL$
in the initial state. Similarly, the release transition $\rho_T$ is
only possible when the ticked being called is owned by the \self set
$\tL$. In such a case, the ownership of the heap $h$ is given away to
the calling thread.

We now have all ingredients to define the concurroid for ticketed
locks.

\begin{definition}[The concurroid for the ticketed lock]

  $\TL = (\{\tlock\}, {\cal W}_T, \tau_T, (\alpha_T, \rho_T))$, where
  ${\cal W}_T$ is given by Definition~\ref{def:tlock} and $\tau_T,
  \alpha_T$ and $\rho_T$ are
  defined in Definitions~\ref{def:tlock-int}~and~\ref{def:tlock-ext}.
\end{definition}

\subsection{Atomic actions of a ticketed lock}
\label{sec:atom-acti-tick}

We now define the three atomic actions, used in the implementation of
the locking/unlocking procedures in Section~\ref{sec:refer-impl}.

The action \textsf{take} corresponds to internal transitions of the
concurroid $\TL$ and returns the number of the drawn ticket as its
result. Hence the following specifications, according to the
definitions of Section~\ref{sec:logic}:
%
\begin{definition}[\textsf{take} action]
\label{def:take-act}
%
$\mathsf{take} = (\TL, \mathsf{nat}, \sigma, \mu)$, where
\[
{\small
\begin{array}{lclcl}
w \in \sigma & \iff & 
  w & = &\tlock \hpts \state{(\tL, \gL)}{\nxt \hpts n_2 \hunion h}{(\tE,
    \gE)}\\ \\
(w, w', \result) \in \mu & \iff & 
  w &= &\tlock \hpts \state{(\tL,~~~~~~~~~~~ \gL)}{\nxt \hpts n_2 ~~~~~~~~~\hunion h}{(\tE,
    \gE)} ~\aand\\
&&  w' &=& \tlock \hpts \state{(\tL \cup \set{n_2}, \gL)}{\nxt \hpts (n_2 + 1) \hunion h}{(\tE,
    \gE)} ~\aand\\
&&  \result &=& n_2
\end{array}}
\]
\end{definition}
%
By doing \emph{flattening}, one can see that operationally
$\mathsf{take}$ corresponds to an atomic \emph{fetch-and-increment} of
the pointer $\nxt$. Formally, the fetch-and-increment operation can be
is defined in Section~\ref{sec:actions}.
%
Two other actions, \textsf{try} and \textsf{droptkt} account for the
heap ownership transfer and, hence, correspond to internal transitions
of the entanglement $P \entangle \TL$.

\begin{definition}[\textsf{try} action]  
\label{def:try-act}
%
$\forall n:\mathsf{nat}\ldot~\mathsf{try}(n) = (P \entangle \TL, \mathsf{bool}, \sigma, \mu)$, where
%
\[
{\small
\begin{array}{l}
w \in \sigma \iff
\\
\begin{array}{lcl}
~~~~~~  w & = & \priv ~\hpts \state{\hempty}{\hempty}{\hE} \hunion \\
  &&\tlock \hpts \state{(\tL \cup \set{n}, \gL)}{\own \hpts \angled{n_1, b}
    \hunion \nxt \hpts n_2 \hunion h}{(\tE, \gE)}\\ \\
\end{array}    
\\ \\
(w, w', \result) \in \mu \iff 
\\
\begin{array}{llcl}
 &  w & = & \priv ~\hpts \state{\hempty}{\hempty}{\hE} \hunion \\
 &&&\tlock \hpts \state{(\tL \cup \set{n}, \gL)}{\own \hpts \angled{n_1, b}
     \hunion \nxt \hpts n_2 \hunion h}{(\tE, \gE)} ~\aand\\  
\mathsf{if}~ n = n_1 \\
\mathsf{then} &  
w' & = & \priv ~\hpts \state{h}{\hempty}{\hE} \hunion \\
  &&&\tlock \hpts \state{(\tL \cup \set{n}, \gL)}{\own \hpts \angled{n_1, \mathsf{true}}
    \hunion \nxt \hpts n_2}{(\tE, \gE)} ~\aand\\  
  & \result &=& \mathsf{true}  ~\aand~ I~(\gL \join \gE)~h \\
\mathsf{else} &  
w' & = & w ~\aand~ \result = \mathsf{false} \\
\end{array}
\end{array}
}
\]
\end{definition}
%
One can see that the action \textsf{try}$(n)$ corresponds to reading
from the heap cell $\own \hpts -$ (hence, it's operational), but it
has a interesting effect on the auxiliary state. In particular, if the
comparison succeeds (\textsf{then}-branch), the ownership over heap
$h$ is transferred to the \self-part of $\hpriv$, and logical flag is
flipped, which corresponds to the internal transition of the
entanglement $P \entangle \TL$. Otherwise, the thread has approached
the counter with a ticket that is not called, and thus is denied
service for now. The state therefore doesn't change, which corresponds
to the reflexivity of the internal transition. The action can only be
performed if the ticket $n$, which is being checked for the service,
is present in the \self part, hence $\tL \cup \set{n}$ (that is the
thread indeed has right to \emph{try} claim the service).

\begin{definition}[\textsf{droptkt} action]  
\label{def:incrown-act}
%
$\mathsf{droptkt} = (\TL, \mathsf{unit}, \sigma, \mu)$, where
\[
{\small
\begin{array}{lclcl}
w \in \sigma & \iff & 
  w & = & \priv \hpts \state{h}{\hempty}{\hE} \hunion \\   
  &&&& \tlock \hpts \state{(\tL \cup \set{n_1}, \gL)}{\own \hpts \angled{n_1, b} \hunion \nxt \hpts n_2}{(\tE, \gE)} ~\aand\\ 
  && \multicolumn{3}{l}{I~(f(\gL) \join \gE)~h ~\aand~ f~\text{is local}}
\\ \\
(w, w', \result) \in \mu & \iff & 
  w & = & \priv \hpts \state{h}{\hempty}{\hE} \hunion \\   
  &&&& \tlock \hpts \state{(\tL \cup \set{n_1}, \gL)}{\own \hpts \angled{n_1, b} \hunion \nxt \hpts n_2}{(\tE, \gE)} ~\aand\\ 
  && \multicolumn{3}{l}{I~(f(\gL) \join \gE)~h ~\aand~ f~\text{is
      local} ~\aand}\\ 
&& w' &=& \priv \hpts \state{\hempty}{\hempty}{\hE} \hunion \\   
  &&&& \tlock \hpts \state{(\tL, \gL)}{\own \hpts
    \angled{n_1 + 1, \mathsf{false}} \hunion \nxt \hpts n_2 \hunion
    h}{(\tE, \gE)}
\end{array}
}
\]
\end{definition}

In the definition of \textsf{droptkt}, the \self-contribution
with respect to the abstract state $\gL$ is captured via some
\emph{auxiliary function} $f$, which is required to be
\emph{local}~\cite{LeyWild-Nanevski:POPL13}. Another important change
is removing a ticket $n_1$ from the \self-set $\tL$, once the
service is finished (\ie, the lock is unlocked).

\subsection{Lock predicates}
\label{sec:spec-textttl-texttt}

In order to verify the \texttt{lock} and \texttt{unlock} procedures,
we first introduce a number of auxiliary predicates to simplify the
presentation of the reasoning.

\begin{definition}[$\TL$ predicates]
\label{def:predicates}
\[
\small{
\begin{array}{rcl}
  \islock(\gL) & \eqdef  &  \tlock \spts (\tL, \gL) \\
  \\
  \cantry(n, \gL) & \eqdef & \tlock \spts (\tL \cup \set{n}, \gL) \aand
  \tlock \jpts (\own \hpts \angled{n_1, b} \hunion h) ~\aand\\
  & & (n = n_1) \implies \neg b \\
  \\
  \locked(\gL, \gE) & \eqdef  &  \tlock \spts (\tL \cup \set{n_1}, \gL) \aand
  \tlock \jpts (\own \hpts \angled{n_1, \mathsf{true}} \hunion h) \\
\end{array}
}
\]
\end{definition}

In Definition~\ref{def:predicates} and further we implicitly treat all
non-bound logical variables as existentially quantified.
%
% \an{Hmm, this is not the folklore. If you don't want to mention a
%   variable, just put a $-$ instead of it.} 
%
The predicate $\islock$ identifies a state $w$ as one satisfying to
the structure of the $\TL$ concurroid. The predicate $\cantry$ asserts
that the current thread can make an attempt to acquire the lock with
the ticket $n$, but it might fail. The auxiliary condition $(n = n_1)
\implies \neg b$ is essential for verifying the body of \texttt{lock}
in order to exclude the bogus situation when the same thread tries to
re-acquire the lock which it already owns.

Curiously, the $\islock$ and $\locked$ predicates can be considered as
implementation of the \emph{abstract} interface for locks, as
presented by the work on CAP~\cite{DinsdaleYoung-al:ECOOP10}: we
deliberately picked the same name for ours. In particular, we can
prove the similar properties for ours predicates with respect to the
locking intuition:

\begin{lemma}
\label{lm:lk1}
$\locked(\gL, \gE) \implies \islock(\gL)$.  
\end{lemma}

\begin{lemma}[Mutual exclusion for Ticketed Lock]
\label{lm:lk2}

$w \models \locked(\gL, \gE) ~\aand~ w^{\top} \models \locked(\gE,
\gL) 
\implies \mathsf{False}$.  
\end{lemma}

Lemma~\ref{lm:lk2} is a ``subjective'' formulation of the mutual
exclusion property, stating that the lock can be simultaneously
acquired by at most one thread. We address the interested reader to
our formal development in Coq~\cite{fcsl-coqscripts} for the proofs of
these and some other lemmas about locks as well as for implementation
of the same abstract interface for the CAS-based spin lock.

\subsection{Verifying locking procedures}
\label{sec:verify-lock-proc}

We culminate the development of this appendix with the verification of
the \texttt{lock} procedure for the ticketed lock
(Section~\ref{sec:refer-impl}).\footnote{Verification of
  \texttt{unlock} is significantly simpler, and we address the reader
  to our formal development for the details~\cite{fcsl-coqscripts}.} 
\[
\begin{array}{c}
\spec{\priv \spts \hempty \lsep \islock(\gL)}
\\
\mathsf{lock} \tag{\arabic{tags}}\refstepcounter{tags}
\\
\spec{\exists h, \gE, 
\priv \spts h \lsep \locked(\gL, \gE) ~\aand~
I~(\gL \join \gE)~h}
\end{array}
\label{lock-spec}
\]

Since in our logic imperative \texttt{while}-loops are implemented as
a fixed point iteration of a tail-recursive function, for such functions
a specification should be defined by the programmer (same is true for
loop invariants in the standard Hoare logic). We begin by assuming for
each auxiliary element of PCM $\gL$ and a ticket number $n$ a family
of curried functions $\mathit{loop}(n: \mathsf{nat}): \mathsf{unit}
\rightarrow \mathsf{unit}$ with the following \SCST specification:

\[
\begin{array}{c}
\spec{\priv \spts \hempty \lsep \cantry(n, \gL)}
\\
\mathit{loop}(n)(\_)  \tag{\arabic{tags}}\refstepcounter{tags}
\\
\spec{\exists h, \gE, 
\priv \spts h \lsep \locked(\gL, \gE) ~\aand~
I~(\gL \join \gE)~h}
\end{array}
\label{loop}
\]

We proceed by unfolding the definition of the locking procedure and
presenting the proof outline.

\[
\begin{array}{l}
\spec{\priv \spts \hempty \lsep \islock(\gL)}
\\
n \leftarrow \mathsf{inject'}(\mathsf{take});
\\
\spec{
\begin{array}{l}
  \priv \spts \hempty \lsep  \tlock \spts (\tL \cup \set{n_2}, \gL)
  \aand \tlock \jpts (\own \hpts - \hunion \nxt \hpts
  n'_2 \hunion h')~\aand \\
  n = n_2
  \aand n_2 \le n'_2
\end{array}
}
\\
\spec{\priv \spts \hempty \lsep \cantry(n, \gL)}
\\
\mathsf{fix}.\mathit{loop}(n). ~\_~. ~(~
\\
~~ \mathsf{if}~ (\mathsf{try}(n))~\mathsf{then}
\\
~~~~\spec{{
  \priv \spts h \lsep \tlock \hpts \state{(\tL, \gL)}{\own \hpts \angled{n, \mathsf{true}}
  \hunion \nxt \hpts n_2''}{(\tE', \gE)} \aand I~(\gL \join \gE)~h}
}
\\
~~~~\mathsf{return}~();
\\
~~~~\spec{\priv \spts h \lsep \locked(\gL, \gE) ~\aand~
I~(\gL \join \gE)~h}
\\
~~\mathsf{else}
\\
~~~~\spec{\priv \spts \hempty \lsep \cantry(n, \gL)}
\\
~~~~\mathit{loop}(n)(\_);
\\
~~~~\spec{\exists h, \gE, \priv \spts h \lsep \locked(\gL, \gE) ~\aand~
I~(\gL \join \gE)~h}
\\
)(\_)
\\
\spec{\exists h, \gE, 
\priv \spts h \lsep \locked(\gL, \gE) ~\aand~
I~(\gL \join \gE)~h}
\end{array}
\]

In fact, the proof has been implicitly split into two parts: (a)
proving the correctness of the \emph{loop} specification~\eqref{loop}
in the body of the locking procedure by the rule \textsc{Fix}
(Figure~\ref{fig:rules}) and (b) proving the specification of the
locking itself, assuming the \emph{loop} spec. In the outline we have
blended them both together, as the proof of~\eqref{loop} is
straightforward: just by reading the specification of the
$\mathsf{try}$ action, we know that in \textsf{then}-branch the heap
$h$ has been transferred to \self-part of $\priv$.  Moreover, in the
proof outline we omitted some reasoning about stability, which,
indeed, has been taken into account. In particular, in the assertion
right after the line ``$\mathsf{if}~
(\mathsf{try}(n))~\mathsf{then}$'', we mention $n_2''$ rather than
$n_2'$, since this counter could have been changed by the \other
threads. These subtle details become clear in the mechanized proofs
and are omitted above for the sake of brevity.

Finally, we have employed a slightly different version of injection,
which we emphasized by using the $\mathsf{inject'}$ command. It
differs from one we described in Section~\ref{sec:reasoning} by the
form of entanglement it handles. The original rule \textsc{Inject}
enabled concurroid framing with respect to the right operand of
$\entangle$. In contrast, the following rule enables concurroid
framing with respect to the left operand of $\entangle$, or right
operand of the ``reversed'' operator $\rentangle$:
%
% \an{Hmm, the same notation is used for both left and right
%   entanglement.}
%
\begin{mathpar}
\inferrule*[Right={Inject'}]
  {\stconc{p}{c}{q}{U}\\
   \mbox{$r$ stable under $V$}}
  {\stconc{p \lsep r}{\mathsf{inject'}\ c}{q \lsep r}{U \rentangle V}}
\end{mathpar}
%
The rule was, indeed, proved sound as well~\cite{fcsl-coqscripts}. In
general, we have actually proved the injection rule sound for any
operator satisfying a number of general properties, and then shown
that $\entangle$ and $\rentangle$ (as well as some other combinators),
satisfy these general properties.


