\subsection{Denotational semantics of \SCST}
\label{sec:denot-sem}

In this section, we define the denotational semantics of \SCST as a
shallow embedding into CiC. For the sake of simplicity, one can think
of the embedding as of a macro-expansion, which compositionally
replaces \SCST programs by expressions and values of CiC.

The semantics of commands and judgments are defined simultaneously.
The mutual recursion is necessary because the denotation of judgments
depends on the denotation of commands and procedures, while the
denotation of a fixed point procedure depends on the denotation of its
procedure triple to determine the lattice in which to take the fixed
point.

We denote \SCST programs as \emph{sets} $T$ of trees of increasing
precision including the $\mathsf{Unfinished}$ tree, which is the coarsest
possible approximation of any program: 
%
\[
\mathsf{prog}~U~A \eqdef \{T
\subseteq \mathcal{P}(\mathsf{tree}~U~A) \mid \mathsf{Unfinished} \in
T\}.
\]

To model recursion, we construct a complete lattice of Hoare types to
get fixed points. We use the $\mathsf{after}$ predicate
(Definition~\ref{def:modpred}) to ensure the tree approximations are
memory safe, respect mutual exclusion, and satisfy their \SCST
specifications. %
 
\begin{definition}[\SCST Hoare types]
\label{def:htypes}
%
For a concurroid $U$ and type $A$, fix a precondition $p \ty \mathsf{state}
\to \prop$ and a postcondition $q \ty A \to \mathsf{state} \to \prop$, with
free logical variables $\flv(p, q)$. The \SCST Hoare type
$\ST{p}{A}{q}{U}$ is defined as follows:
%
\[
\begin{array}{rcl}
\ST{p}{A}{q}{U} & \eqdef &   
\left\{ 
T \in \mathsf{prog}~U~A ~\left|~
  \begin{array}{l}
    \forall \flv(p,q)~w~(t \in T) \ldot
    \\ 
    w \models p \aand w \in \coh{U} \implies \mathsf{after}_U~w~t~q

  \end{array}
\right.
\right\} 
\end{array}
\]
\end{definition}

% Definition has_spec W A (s : spec A) := 
%   [Pred T : prog W A | forall i t, 
%      s.1 i -> i \In Mod.coh W -> t \In T -> after i t (s.2 i)]. 

\begin{figure}[t]
\figsize
\figDenJdg
%\rule{1.0\textwidth}{0.2mm}
\caption{Denotational semantics of \SCST judgments.}
\label{fig:denj}
\end{figure}


Intuitively, the denotation of a \SCST judgment $\prog{p}{c \ty A}{q}{U}$
is the set of trees $T$ denoting the command $c$, together with a
proof that for any initial configuration $w$ that satisfies the
precondition $p$, then $\mathsf{after}$ executing any tree $t \in T$
from $c$ produces some result value and final configuration that
satisfy postcondition $q$. The definition quantifies over the free
logical variables of $p$ and $q$ in order to give these variables
local scope, as stipulated in Section~\ref{sec:logic}.
%
\SCST assertions (Section~\ref{sec:reasoning}) are arbitrary CiC
predicates of type $\mathsf{state} \to \prop$; and we model the
relation $w \models p$ of \SCST as the application $p\ w$ in CiC.

\begin{lemma}
  The type $\ST{p}{A}{q}{U}$ is a complete lattice, with set
  union as the join operator, and $\{\mathsf{Unfinished}\}$ as the unit
  element.

  The type $\forall{x \ty B}\ldot \ST{p}{A}{q}{U}$ of functions
  mapping $x\,{:}\,B$ into $\ST{p}{A}{q}{U}$, where $A, p, q$
  may depend on $x$, is also a complete lattice, with the join
  operator on functions defined point-wise, and the constant
  $\{\mathsf{Unfinished}\}$ function as the unit element.
\end{lemma}

% \paragraph{Denotations of Judgements}
The \textbf{denotation of judgments} $\mean{\Gamma \vdash J}$
(Figure~\ref{fig:denj}) turns \SCST judgments into CiC typing
judgments ($\vdashCiC$). 
% 
A command specification $\prog{p}{- \ty A}{q}{U}$ is denoted by the CiC
type $\ST{p}{A}{q}{U}$, and a procedure specification $\forall x \ty
B\ldot \prog{p}{- \ty A}{q}{U}$ is denoted by the CiC dependent function
type $\forall{x \ty B}\ldot \ST{p}{A}{q}{U}$. %
%
The Hoare ordering $(p_1, q_1) \sqsubseteq (p_2, q_2)$ judgment is
explained in Section~\ref{sec:ordering}, so here we just literally
repeat the definition given there.

The \textbf{denotation of commands and procedures}
(Figure~\ref{fig:den}) is subsidiary to that of judgments because the
fixed-point construction is indexed by the argument and return types,
and the pre- and postconditions. %
An $A$-returning command $c$ is denoted by a set of approximating
trees in $\mathsf{prog}~U~A$, and an $A$-returning procedure $F$ with
argument $B$ is denoted by a set of trees in $B \arrow
\mathsf{prog}~U~A$. 

In the semantic translation for the sequential composition $\mean{x
  \leftarrow c_1; c_2}$, we employ the fact that the translation
$\mean{c_2}$ can produce \emph{open} values (\ie, trees $t_2$), where
a variable $x$ of CiC is not bound. We subsequently close them by
constructing a continuation $k$ as a CiC function, in an indirect way,
that is, quantifying over all possible inputs, hence $\forall x, k~x
\in \mean{c_2}$.
%
For the denotational semantics of the hiding statement
$\mean{\mathsf{hide}_{\Phi,g}~c}$, we appeal to Lemma~\ref{lm:hiding}
(Section~\ref{sec:prop-inject-hiding}), which give a constructive way
to construct an elaboration predicate $\Ghi$ from a given abstraction
function $\Phi$.
%
Since all \SCST program constructors preserve \emph{monotonicity}, the
$\mathsf{fix}\ f.x.c$ procedure can take the least fixed point
$\mathsf{lfp}$ of the function $\lambda f\ldot \lambda x\ldot
\mean{c}$ by the Knaster-Tarski theorem.

\begin{figure}[t]
\figsize
\figDenCmd
\caption{Denotational semantics of \SCST commands.}
\label{fig:den}
\end{figure}

 
