% General words
%
This manual describes the design and implementation of Fine-grained
Separation Logic (FCSL)~\cite{Nanevski-al:ESOP14} as well as the case
studies conducted in
it~\cite{Sergey-al:ESOP15,Sergey-al:PLDI15,Sergey-al:OOPSLA16,
Delbianco-al:ECOOP17}.

Section~\ref{sec:prerequisites} of this manuscript lists the
prerequisites for reading and executing FCSL code,
Section~\ref{sec:structure} outlines the structure of the FCSL
project, and Section~\ref{sec:compilation} provides general
instructions on compiling FCSL and measuring the line counts of the
development.
%
Section~\ref{sec:case-studies} serves as a guide through the
mechanized implementation of the case studies.
%
% In \S \ref{sec:fcsl-implementation} we elaborate on some parts of the
% core implementation of the FCSL logical framework, including its
% denotational domain, modal predicates and structural lemmas.

\tableofcontents

\vspace{20pt}

\section{Prerequisites}
\label{sec:prerequisites}
%
In the remainder of this document, we assume that the reader is using
the provided VirtualBox VM image or has successfully installed
Coq/Ssreflect as well as GNU Emacs/Aquamacs and Proof General. The VM image and
the set-up instructions for the artifact are available by the
following URL:

\begin{center}
    \url{http://software.imdea.org/fcsl}
\end{center}

In the case of the VM image, you can find the code of the project in
the folder \texttt{\textasciitilde/fcsl-ecoop17} (use \code{fcsl} as
username and password whenever required).  We do not assume the reader
to have any familiarity with Coq/Ssreflect to browse through the
examples. In the case if the reader is feeling curious to experiment
on her own and modify the proof scripts, we recommend to take a look
at the lecture notes~\cite{Sergey:PnP} for the basics of interactive
proof development in Coq/Ssreflect.

For the description of semantic foundations of FCSL in prose, we
address the reader to the appendices of the original FCSL
paper~\cite{Nanevski-al:ESOP14}.

\section{Structure of the development}
\label{sec:structure}

The project's structure is completely flat, with all files located
currently in the same folder (\code{fcsl}). Besides the
\code{README} file, which summarizes the contents of particular files,
and the \code{Makefile}, the folder contains the files, corresponding
to case studies, and to the core development of the FCSL logic itself.
%
Below, we provide brief descriptions for the most interesting parts of
the source codebase.

\subsection{Case Studies}
\label{sec:case-studies-files}

The following files implement the case studies/example programs,
together with their specifications and verification scripts (more
details are available in the corresponding referred sections further
in the document).

%\vspace{10pt}

\subsubsection{Basic examples}
\label{sec:basic-examples}

Basic examples, described in the initial work on
FCSL~\cite{Nanevski-al:ESOP14}, described in detail in
Section~\ref{sec:basic-examples}.

Folder: \texttt{Examples/Basic}

\begin{itemize}

\item \code{locks.v} --- generic interface for locking structures~(\S
  \ref{sec:locks}).

\item \code{caslock.v} --- an implementation of a CAS-based spin lock
  concurroid, its actions, lock/unlock procedures and lemmas for
  Hoare-style reasoning~(\S \ref{sec:cas-lock}).

\item \code{ticketed.v} --- An implementation of a ticketed lock
  concurroid, its actions, lock/unlock procedures and lemmas for
  Hoare-style reasoning~(\S \ref{sec:ticketed-lock}).

\item \code{incrementor.v} --- a toy example of a coarse-grained
  incrementor~(\S \ref{sec:incrementor}).

\item \code{alloc.v} --- a simple coarse-grained memory allocator,
  parametrized by a lock implementation (\S \ref{sec:allocator}).

\end{itemize}

\subsubsection{Linearizable concurrent objects and universal
  constructions}
\label{sec:line-conc-objects}

Examples, motivating the use of subjective auxiliary histories,
presented in the work~\cite{Sergey-al:ESOP15}, described in
Section~\ref{sec:hist}.

Folder: \texttt{Examples/Histories}

\begin{itemize}

\item \code{readpair.v} --- an implementation of an atomic
  pair-snapshot data structure (\S \ref{sec:atomic-snapshot}).

\item \code{treiber.v} --- an implementation of Treiber's non-blocking
  stack~(\S \ref{sec:treiber-stack}).

\item \code{stack_framed.v} --- application of FCSL's frame rule applied to the Treiber
  stack specifications~(\S \ref{sec:fram-stack-spec})

\item \code{stack_seq.v} --- a derivation of a sequential specification
  for the Treiber stack "push" and "pop" operations~(\S
  \ref{sec:sequential-stack}).

\item \code{stack_client.v} --- a simple stack client: two-thread
  producer/consumer~(\S \ref{sec:producerconsumer}).

\item \code{flatcombine.v} --- a generic implementation of the flat
  combiner, parametrized by specifications of sequential procedures,
  requiring exclusive access to the critical resource~(\S
  \ref{sec:flat-combiner}).

\item \code{stack_combine.v} --- instantiation of the flat combiner
  with \emph{sequential} stack push/pop procedures~(\S
  \ref{sec:inst-flat-comb}).

\end{itemize}

\subsubsection{Concurrent graph manipulations}
\label{sec:conc-graph-manip}

Main example from the work~\cite{Sergey-al:PLDI15}, demonstrating
handling of deep sharing in FCSL, described in
Section~\ref{sec:spanning-tree}.

Folder: \texttt{Examples/Graphs}

\subsubsection{Non-linearizable concurrent objects and their clients}
\label{sec:non-line-conc}

Examples from the paper on handling non-linearizable concurrent data
structures in FCSL~\cite{Sergey-al:OOPSLA16}, described in detail in
Section~\ref{sec-nonlinaerizable}.

Folder: \texttt{Examples/NonLinearizable}

\begin{itemize}

\item \code{exchange.v} --- an implementation of an elimination-based
  exchanger, specified via subjective histories, storing ``twin''
  contributions. %(\S \ref{sec:exchange}).

\item \code{exchange\_client.v} --- A client of the elimination-based
  exchanger, implementing the fair exchange of two sequences between
  two threads running in parallel in the absence of additional
  external interference. %(\S \ref{sec:exchange-client}).

\item \code{qcounter.v} --- an implementation and verification of a
  simple counting network, built via one binary balancer b and two
  atomic counters, c0 and c1, such that each of the counters stores
  only values of the corresponding parity. The given specification
  incorporates information about interference, observed in the past,
  and, hence, enables quantitative reasoning about the counter's
  clients. %(\S \ref{sec:qcounter}).

\item \code{qcounter\_client1.v} --- a client of the counting network,
  exercising the quiescent consistency and proving that in the
  presence of a quiescent state between two calls to the
  (quiescently-consistent) counter, the obtained results will come out
  in the natural order. %(\S \ref{sec:qcounter-client1}).

\item \code{qcounter\_client2.v} --- A client of the counting network,
  used to demonstrate quantitative reasoning out of the provided spec,
  derived in the qcounter.v file. The specification ensures that the
  results of the two subsequent calls in the presence of interference
  come out of order by no more that 2 times the ``size'' of the
  interference. %(\S \ref{sec:qcounter-client2}).

\end{itemize}

\subsubsection{Concurrent Data Structures Linked in Time}
\label{sec:relink}

Folder: \texttt{Examples/Relink}

This folder contains the main case study of the {\it Linking in Time}
technique for verifying concurrent data structures with non-fixed
linearization points introduced in~\cite{Delbianco-al:ECOOP17}. It
consists of several files implementing the verification in FCSL of
Jayanti's single writer/single scanner snapshot
construction~\cite{Jayanti+STOC05}.


\begin{itemize}

 \item \code{jayanti\_snapshot.v} --- Preliminaries. Lemmas for
       reasoning about coloring patterns of histories, and other
       history invariants.

 \item \code{jayanti\_ghoststate.v} --- Implementation of the
 auxiliary state for the snapshot object, its transitions, invariants
 and associated invariant preservation proofs.

 \item \code{jayanti\_concurroid.v} --- Implementation of the FCSL
 concurrent resource ---a.k.a. {\it concurroid} for the snapshot.

 \item \code{jayanti\_actions.v} --- FCSL atomic actions for the
 snapshot object concurrent resource.

 \item \code{jayanti\_stability.v} --- Assertions used in the proof
 outline of the main methods and proofs of their stability under FCSL
 transitions.

 \item \code{jayanti\_library.v} --- Implementation and verification
 of the atomic commands (auxiliary code) and the snapshot library
 (i.e. scan and write procedures).

 \item \code{jayanti\_clients.v} --- Verification of the clients of
 the snapshot data structure, as presented in Section 4 of the
 paper~\cite{Delbianco-al:ECOOP17}.

\end{itemize}

\subsection{FCSL metatheory}
\label{sec:fcsl-metatheory}

For the formalization of the metatheory, we have borrowed a number of
libraries (\code{pred.v}, \code{prelude.v}, \code{ordtype.v},
\code{finmap.v}, \code{pperm.v}, \code{domain.v}, \code{array.v}) from
the development of the verification framework for the sequential Hoare
Type Theory~\cite{Nanevski-al:POPL10} (which can be found in the
folder \texttt{Heaps} of the project), so we don't describe them here.

Below, we list the libraries (contained in the folder \texttt{Core}),
essential for the formulation of FCSL's semantics and the proof of
soundness of its rules. The intuition behind the implementation can be
acquired from the referred parts of the preceding papers on FCSL.


%\vspace{10pt}
\begin{itemize}

\item \code{feaps.v} --- PCM of heap maps.

\item \code{state.v} --- resource states implemented as triples of maps.

\item \code{zigzag.v} --- definition and helper lemmas about $\zig$
  (\code{<\\}) and $\zag$ (\code{/>}) operators used in the fork-join
  closure~\cite{Nanevski-al:ESOP14}.

\item \code{worlds.v} --- definition of concurroids, with properties on
  states and transitions. %(\S \ref{sec:concurroids}).

\item \code{auto_split.v} --- some automation using overloaded
    lemmas.

\item \code{inject.v} --- meta theory required for proving the
    injection rule~\cite{Nanevski-al:ESOP14}.
                       
\item\code{atomic.v} --- definition and properties of atomic actions~\cite[Appendix A]{Nanevski-al:ESOP14}.
   % (\S \ref{sec:atomic-actions}).

\item\code{rely.v} --- definition of the \emph{Rely} predicate for a
  concurroid~\cite[Appendix~B]{Nanevski-al:ESOP14}.

\item \code{privstate.v} --- concurroid for thread-local
  state~\cite[\S 4,~Example~1]{Nanevski-al:ESOP14}. %(\S \ref{sec:priv}).

\item \code{footprint.v} ---            helper lemmas about disjointness of labels.

\item \code{retract.v} --- meta-theory required for proving the hiding
  rule~\cite[Appendix~F.2]{Nanevski-al:ESOP14}.

\item \code{schedule.v} --- datatype of execution paths.

\item \code{process.v} --- definition of action trees~\cite[Appendices
  F.1, F.3 and F.4]{Nanevski-al:ESOP14}.

\item \code{always.v} --- definition of the modal predicates for
  adherence to a protocol~\cite[Appendices~F.5 and~F.7]{Nanevski-al:ESOP14}.

\item \code{model.v} --- implementation of denotational semantics of
  programs as sets of trees~\cite[Appendix F.6]{Nanevski-al:ESOP14}. %~(\S \ref{sec:semantics}).

\item \code{lemmas.v} --- soundness of the rules; essentially each
  lemma is a case in the proof of the main soundness theorem
  from~\cite[Appendix F.7]{Nanevski-al:ESOP14}. The lemmas also
  immediately serve as proof rules. %(\S\ref{sec:structural-lemmas}).

\item \code{histories.v} ---  libraries for history PCMs the used in the specifications of
optimistic fine-grained algorithms
in~\cite{Sergey-al:ESOP15,Sergey-al:OOPSLA16,
Delbianco-al:ECOOP17}. The {\it standard} PCM history library is
described in~\cite{Sergey-al:ESOP15}, whereas the {\it relinkable}
history library is presented in~\cite{Delbianco-al:ECOOP17}.

\item \code{stsep.v} --- notation for traditional-style unary Hoare
  pre/postconditions. % (\S \ref{sec:getters}).

\item \code{getter.v} --- generic getters for concurroid components.% (\S\ref{sec:getters}).

 \item \code{seq\_extras.v} --- Algebraic surgery lemmas for lists,
 extending the original Ssreflect \texttt{seq} library.

\end{itemize}

%\vspace{10pt}

%%%%%%%%%%%%%%%%%%%%%%%%%%%%%%%%%%%%%%%%%%

\section{Compilation}
\label{sec:compilation}

\subsection{Building the project}
\label{sec:building-project}

In order to compile the whole project from scratch, which will
\emph{\textbf{automatically check the proofs of all theorems and
    specifications}}, execute the following command from the command
line in the project folder.

\begin{lstlisting}
make clean; make
\end{lstlisting}


%% The build script will also display compilation times for particular
%% files, generated via the standard \code{time} utility.

% \subsection{Counting the lines of code}
% \label{sec:counting-lines-code}

% We supply the project with the \code{cloc}
% utility\footnote{\url{http://cloc.sourceforge.net}} to facilitate
% computing the lines of code in particular files.
% %
% Since the syntax of Coq is similar to syntax of the languages from the
% ML family, the lines for \code{*.v}-files are computed using the
% \code{cloc} settings for OCaml. 
% %
% To compute LOCs for individual files, we recommend to use the
% predefined script \code{count}, located in the project folder. For
% example, the following command will compute the number of LOCs in the
% implementation of the spanning tree example:

% \begin{lstlisting}
% ./count spanning.v
% \end{lstlisting}

% Due to some minor proof optimizations since the PLDI 2015 paper
% submission deadline, the line counts in the supplied FCSL code base
% might have slightly decreased comparing to what is reported in the
% accepted draft of the paper.

%%%%%%%%%%%%%%%%%%%%%%%%%%%%%%%%%%%%%%%%%%


%%%%%%%%%%%%%%%%%%%%%%%%%%%%%%%%%%%%%%%%%%


% \section{Implementation of the FCSL semantic model and logic}
% \label{sec:fcsl-implementation}

% \subsection{Concurroids}
% \label{sec:concurroids}

% TODO

% \subsection{Atomic actions}
% \label{sec:atomic-actions}

% TODO

% \subsection{Semantics of FCSL commands}
% \label{sec:semantics}

% TODO

% \subsection{Thread-local state}
% \label{sec:priv}

% TODO

% \subsection{Modal predicates and Hoare types}
% \label{sec:modal-pred-hoare}

% TODO: explain \code{always} and \code{after} predicates in the logic and the way they are implemented. 

% \subsection{Structural lemmas}
% \label{sec:structural-lemmas}

% TODO: probably, refer to Dijkstra's predicate transformers~\cite{Dijkstra:CACM75}.

% \subsection{State getters and unary postconditions}
% \label{sec:getters}

% TODO: Explain this stuff
