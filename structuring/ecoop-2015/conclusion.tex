
\section{Conclusion}
\label{sec:conclusion}

% In this paper we have presented FCSL---a mechanized framework, based
% on a Hoare-style logic, for specification and verification of
% fine-grained concurrent programs.

Our experience with implementing a number of concurrent data
structures in FCSL indicates a recurring pattern, exhibited by the
formal proof development.
%
Verification of a new library in FCSL starts from describing its
invariants and evolution in terms of an STS. It's common to consider
parts of real or auxiliary state, which are a subject of a logical
split between parallel threads, as elements of a particular PCM.
%
Such representation of resources makes the verification to be uniform
and compositional, as it internalizes the library protocol, so the
clients can reason out of the specifications.

This observation indicates that STSs and PCMs can be a robust basis
for understanding, formalizing and verifying existing fine-grained
programs. We conjecture that the same foundational insights will play
a role in future designs and proofs of correctness of novel concurrent
algorithms.

% \ab{Shouldn't we re-emphasize the 4 points in the intro., uniformity,
%   expressivity, ...?}
%
% \is{Okay, I incorporated the scalability here}

% \ab{Have we convinced the reader that the self/other dichotomy is
%   essential? May be a sentence or 2 needs to be added earlier and
%   refer to the arXiv paper?}
% %
% \is{Hmm. I don't think that the PCMs are the main message of this
%   paper. It's more about the whole methodology and a tool, which are
%   scalable and bring new insights. So we say in the conclusion.}
