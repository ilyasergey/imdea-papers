\section{Introduction}
\label{sec:intro}

Reasoning about correctness of concurrent programs, in which several
computations can be executed in parallel and overlap in time, is
difficult due to the large number of possible interactions between
concurrent processes/threads on shared data structures.


% \ab{Can we also make the point that: fine-grained concurrent
%   algorithms are heavily used in concurrent data structure libraries
%   (cite?) and calls to such library functions in client code is
%   ubqiquitous (cite?). Therefore it is essential that the libraries be
%   proved correct. ``Verify once, use often''.}

% \is{I put this point into the abstract and added later in the intro to
%   the ``compositional" bullet of the framework description.}

One way to deal with the complexity of verifying concurrent code is to
employ the mechanisms of so-called \emph{coarse-grained}
synchronization, \ie, locks. By making use of locks in the code, the
programmer ensures mutually-exclusive thread access to critical
resources, therefore, reducing the proof of correctness of concurrent
code to the proof of correctness of \emph{sequential} code. While
sound, this approach to concurrency prevents one from taking full
advantage of parallel computations. An alternative is to implement
shared data structures in a \emph{fine-grained} (\ie, lock-free)
manner, so the threads manipulating such structures would be reaching
a consensus via the active use of non-blocking read-modify-write
operations (\eg, compare-and-swap) instead of locks.

Despite the clear practical advantages of the fine-grained approach to
the implementation of concurrent data structures, it requires
significant expertise to devise such structures and establish
correctness of their behavior.
%
%  with respect to standard criteria, such as
% \emph{linearizability}~\cite{Herlihy-Wing:TOPLAS90}.
% %
% Moreover, in the presence of advanced programming features, such
% as higher-order functions, pointer aliasing, and internal parallelism,
% linearizability by itself might not be sufficient as a criteria to
% ensure the full functional correctness of a concurrent program, and
% its generalizations to match these programming features are a topic of
% ongoing active
% research~\cite{Cerone-al:ICALP14,Gotsman-Yang:CONCUR12}.

In this paper, we focus on \emph{program logics} as a generic approach
to specify a program and formally prove its correctness \wrt the given
specification.
%
In such logics, program specifications (or specs) are represented by
Hoare triples $\spec{P}~c~\spec{Q}$, where $c$ is a program being
described, $P$ is a precondition that constrains a state in which the
program is safe to run, and $Q$ is a postcondition, describing a state
upon the program's termination.
%
Modern logics are sufficiently \emph{expressive}: they can reason
about programs operating with first-class executable code,
locally-spawned threads and other features omnipresent in modern
programming.
%
Verifying a program in a Hoare-style program logic can be done
\emph{structurally}, \ie, by means of systematically applying
syntax-directed inference rules, until the spec is proven.


Importantly, logic-based verification of fine-grained concurrency
requires reasoning about a number of concepts that don't have direct
analogues in reasoning about sequential or coarse-grained concurrent
programs:

\begin{itemize}[itemindent=0pt] 

\item[(1)] \textbf{Custom resource protocols.} Each shared data
  structure (\ie, a \emph{resource}) that can be used by several
  threads concurrently, requires a specific ``evolution protocol'', in
  order to enforce preservation of the structure's consistency.
% 
  In contrast with coarse-grained case, where the protocol is fixed to
  be locking/unlocking, a fine-grained resource comes with its own
  notion of consistency and protocol.
  
\item[(2)] \textbf{Interference and stability.} Absent locking, local reasoning
  about a shared resource from a single thread's perspective should manifest
  the admissible changes that
  can be made by other threads that interfere with the current one. Every thread-local assertion about a fine-grained data structure's state 
  should be \emph{stable}, \ie, invariant under possible concurrent
  modifications of the resource.
%
%   \ab{This doesn't differentiate between coarse grained and fine
%     grained, like the others do.}
% %
% \is{Ok, I've added a note about absence of locking, which is exactly
%   the difference between CG and FG cases.}

\item[(3)] \textbf{Work stealing.} This common concurrent pattern
  appears in fine-grained programs due to relaxing the {mutual
    exclusion} policy; thus several threads can simultaneously operate
  with a single shared resource. The ``stealing'' happens when a
  thread is scheduled for a particular task involving the resource,
  but the task is then accomplished by \emph{another} thread; however,
  the result of the work is nevertheless ascribed to the initially
  assigned thread.
%
  % \ab{Here it isn't clear what ascribing ``contribution'' to initially
  %   assigned thread means. Or why such contribution is important.}
%
  %   \is{Okay, changed the contribution to the ``result of the
  %   work''.}

\end{itemize}
%
\noindent
%
In addition, Hoare-style reasoning about coarse- or fine-grained
concurrency requires a form of (4) \textbf{auxiliary state} to
partially expose the internal threads' behavior and relate local
program assertions to global invariants, accounting for specific
threads' contributions into a resource~\cite{Jones:TR09}.
%

These aspects, critical for Hoare-style verification of fine-grained
concurrent programs, have been recognized and formalized in one form
or another in a series of recently published works by various
authors~\cite{Feng:POPL09,DinsdaleYoung-al:ECOOP10,Jacobs-Piessens:POPL11,Turon-al:ICFP13,Svendsen-Birkedal:ESOP14,ArrozPincho-al:ECOOP14},
providing logics of increasing expressivity and compositionality.
% 
In formal proofs of correctness of concurrent libraries, that are
based upon these logical systems, the complexity is \emph{not} due to
the libraries' sizes in terms of lines of code, but predominantly due
to the intricacy of the corresponding data structure invariant, and
the presence of thread interference and work stealing. This fact, in
contrast to proofs about sequential and coarse-grained concurrent
programs, requires one to establish stability of every intermediate
verification assertion.
%
Needless to say, manual verification of fine-grained concurrent
programs therefore becomes a challenging and error-prone task, as it's
too easy for a human prover to forget about a piece of
resource-specific invariant or to miss an assertion that is unstable
under interference; thus the entire reasoning can be rendered
unsound. 
%
% \ab{This is nice, but I think this motivation needs to appear early to
%   situate the work.}
% %
%   \is{Yes, I've sad more to thin earlier in the abstract}
%

Since the process of structural program verification in a Hoare-style
logic is largely mechanical, there have been a number of recent
research projects that target mechanization and automation of the
verification process by means of embedding it into a general-purpose
proof
assistant~\cite{Nanevski-al:ICFP08,Nanevski-al:POPL10,Shao:CACM10,Chlipala:PLDI11},
or implementing a standalone verification
tool~\cite{Leino-Muller:ESOP09,Cohen-al:TPHOL09,Jacobs-al:NFM11}. However,
to the best of our knowledge, none of the existing tools has yet
adopted the logical foundations necessary for compositional reasoning
about all of the aspects (1)--(4) of fine-grained concurrency. This is
the gap which we intend to fill in this work.

In this paper, we present a framework for mechanized verification of
fine-grained concurrent programs based on the recently proposed
\emph{Fine-grained Concurrent Separation Logic} (FCSL)
by~\citet{Nanevski-al:ESOP14}.\footnote{Hereinafter, we will be using
  the acronym FCSL to refer both to the Nanevski et al.'s logical
  framework and to our implementation of it.}
% 
FCSL is a library and an embedded domain-specific language (DSL) in
the dependently-typed language of
Coq~\cite{Coq-manual}.
% 
Due to its logical foundations, FCSL, as a verification tool and
methodology for fine-grained concurrency,~is:

\begin{itemize}[leftmargin=*]

\item \textbf{Uniform:} FCSL's specification model is based on two
  basic constructions: \emph{state-transition systems} (STSs) and
  \emph{partial commutative monoids} (PCMs). The former describe
  concurrent protocols and thread interference, whereas the latter
  provide a generic treatment of shared resources and thread
  contributions, making it possible to encode, in particular, the work
  stealing pattern. Later in this paper, we will demonstrate how these
  two components are sufficient to specify a large spectrum of
  concurrent algorithms, data structures, and synchronization
  mechanisms, as well as to make the proofs of verification
  obligations to be uniform.

\item \textbf{Expressive}: FCSL's \emph{specification} fragment is
  based on the propositional fragment of Calculus of Inductive
  Constructions (CIC)~\cite{Bertot-Casteran:BOOK}. Therefore, FCSL can
  accommodate and compose arbitrary mathematical theories, \eg, PCMs,
  heaps, arrays, graphs, \etc.
%
% , as well as encoding arbitrary
%   algebraic structures. 

\item \textbf{Realistic:} FCSL's \emph{programming} fragment features
  a complete toolset of modern programming abstractions, including
  user-defined algebraic datatypes, first-class functions and pattern
  matching. That is, any Coq program is also a valid FCSL program.
  % 
  The monadic nature of FCSL's embedding into
  Coq~\cite{Nanevski-al:ICFP06} makes it possible to encode a number
  of computational effects, \eg, thread spawning and general
  recursion. This makes programming in FCSL similar to programming in
  ML or Haskell.

\item \textbf{Compositional:} Once a library is verified in FCSL
  against a suitable spec, its code is not required to be re-examined
  ever again: all reasoning about the client code of that library can
  be conducted out of the specification. 
  %
  The approach is thus scalable: even though the proofs for
  libraries might be large, they are done just once.

\item \textbf{Interactive:} FCSL benefits from the infrastructure,
  provided by Coq's fragment for mechanized reasoning, enhanced by
  Ssreflect extension~\cite{Gonthier-al:TR}.
% 
  % \ab{We discussed replacing interactive by compositional on Friday. I
  %   think a discussion on compositionality/spec reuse is more critical
  %   here. We can talk about interactive verification as commentary at
  %   the end of the bullets.}
%
  % \is{Okay, I've added a separate point, as I think both these things
  %   should be mentioned.}
%
  While the verification process can't be fully automated (as full
  functional correctness of concurrent programs often requires stating
  specs in terms of higher-order predicates, hence, making their proof
  undecidable in general), the human prover nevertheless can take
  advantage of all of Coq's tools to discharge proof obligations.

\item \textbf{Foundational:} The soundness of FCSL as a logic has been
  proven in Coq with respect to a version of denotational semantics
  for concurrent programs in the spirit
  of~\citet{Brookes:TCS07}. Moreover, since FCSL program specs are
  encoded as Coq types, the soundness result scales to the
  \emph{entire language} of Coq, not just a toy core calculus. This
  ensures the absence of bugs in the whole verification tool and, as a
  consequence, in any program, which is verified in~it.

\end{itemize}

\noindent
In the remainder of the paper, we will introduce the FCSL framework by
example, specifying and verifying full functional correctness of a
characteristic fine-grained program---a concurrent spanning tree
algorithm.
%
Starting from the intuition behind the algorithm, we will demonstrate
the common stages of program verification in FCSL.
%
% While doing so, we will show how the verification process, while not
% fully automated, nevertheless does not impose a lot of proof burden to
% the human prover, thanks to the fact that, 
%
We next explain some design choices, made in the implementation
of FCSL, and report on our experience of verifying a number of
benchmark concurrent programs and data structures: locks, memory
allocator, concurrent stack and its clients, an atomic snapshot
structure and non-blocking universal constructions.
%
We conclude with a comparison to related logical frameworks
and tools, and a discussion on future work.
