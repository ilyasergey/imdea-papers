\documentclass[runningheads]{llncs}

%% Save the class definition of \subparagraph
\let\llncssubparagraph\subparagraph
%% Provide a definition to \subparagraph to keep titlesec happy
\let\subparagraph\paragraph
%% Load titlesec
\usepackage[compact]{titlesec}
%% Revert \subparagraph to the llncs definition
\let\subparagraph\llncssubparagraph

\usepackage{amssymb,amsmath}
\usepackage{latexsym}
\usepackage{graphicx}
\usepackage{textcomp}
\usepackage{txfonts} %% for various points-to symbols. use pxfonts for 5-tip stars
\usepackage{verbatim} %% for comment environment
\usepackage{xspace} %% for comment environment
\usepackage{cancel} % for crossing Owns 
\usepackage{mathpartir}
\usepackage{soul}
\usepackage[usenames,dvipsnames]{color}
\usepackage{multirow}
\usepackage{stmaryrd} 
% % sorting citation
\usepackage{url}
\usepackage{float} 
\usepackage{framed}
\usepackage{hyperref}
% Fonts
\usepackage{bbm}
\usepackage{alltt}
\usepackage{verbdef}
\usepackage{titlesec}
\usepackage{parskip}
\usepackage{wrapfig}
\usepackage{cite}
\usepackage{listings}
\usepackage{enumitem}
% For turned column headers 
\usepackage{adjustbox} 
\usepackage{array}
\usepackage{booktabs}
\usepackage{multirow}
\usepackage{pifont}


% \setlength{\intextsep}{5pt}%
\setlength{\parindent}{0.15in}
% \setlength{\topsep}{0cm}
\setlength{\parskip}{0pt}
% \titlespacing*{\section}{0pt}{*1}{*1}
% \titlespacing*{\subsection}{0pt}{*0.7}{*0.5}
% \titlespacing*{\paragraph}{0pt}{*0.5}{*0.5}
\renewcommand{\figurename}{Figure} 

\let\citet=\cite

% remarks
\newcommand{\todo}[1]{\textcolor{red}{({#1})}}
\newcommand{\is}[1]{\textcolor{blue}{(Ilya: {#1})}}
\newcommand{\an}[1]{\textcolor{red}{(Aleks: {#1})}}

% useful macros
\newcommand{\asgn}{\leftarrow}
\newcommand{\code}[1]{\lstinline{#1}}
\newcommand{\ccode}[1]{\code{#1}}
\newcommand{\var}[1]{\({#1}\)} 
\newcommand{\num}[1]{\({\text{{\scriptsize{#1}}}}\)}
\newcommand{\etc}{\emph{etc}}
\newcommand{\ie}{\emph{i.e.}\xspace}
\newcommand{\Ie}{\emph{I.e.}\xspace}
\newcommand{\eg}{\emph{e.g.}\xspace}
\newcommand{\Eg}{\emph{E.g.}\xspace}
\newcommand{\vs}{\emph{vs.}\xspace}
\newcommand{\etal}{\emph{et~al.}\xspace}
\newcommand{\adhoc}{\emph{ad hoc}\xspace}
\newcommand{\viz}{\emph{viz.}\xspace}
\newcommand{\dom}[1]{\mathsf{dom}(#1)}
\newcommand{\aka}{\textit{a.k.a.}\xspace}
\newcommand{\cf}{\textit{cf.}\xspace}
\newcommand{\wrt}{\emph{wrt.}\xspace}
\newcommand{\loef}{L\"{o}f}
\newcommand{\sep}{\textasteriskcentered}
\newcommand{\res}{\mathsf{res}} 
\newcommand{\ret}{\mathsf{ret}} 
\newcommand{\fix}{\mathsf{fix}} 

%specs
\newcommand{\specK}[1]{\ensuremath{\textcolor{blue}{#1}}}
\newcommand{\spec}[1]{\specK{\left\{{#1}\right\}}}
\newcommand{\specQ}[4]{[#1 , #2 , #3] \, #4}%\specQ{mL}{gL}{gE}{p}
\newcommand{\drspec}[1]{\specK{\langle{#1}\rangle}}
\newcommand{\sspec}[1]{\specK{\{{#1}\}}}




% SSReflect listings 
\input{lstcoq}
\lstset{style=Coq}

% Hyphenation
\hyphenation{Veri-Fast}

% Bibtgex tweaks

\begin{document}

%\special{papersize=8.5in,11in}


% \authorinfo{Ilya Sergey}
%            {IMDEA Software Institute}
%            {ilya.sergey@imdea.org}
% %
% \authorinfo{Aleksandar Nanevski}
%            {IMDEA Software Institute}
%            {aleks.nanevski@imdea.org}

% \authorinfo{Anindya Banerjee}
%            {IMDEA Software Institute}
%            {anindya.banerjee@imdea.org}

\title{
Mechanized Verification\\ 
of Fine-grained Concurrent Programs 
}
 
% \author{Ilya Sergey \and 
%   Aleksandar Nanevski \and
%   Anindya Banerjee}

% \institute{IMDEA Software Institute, Spain\\ 
% \email{\{ilya.sergey, aleks.nanevski, anindya.banerjee\}@imdea.org}}

\author{}
\institute{}

\maketitle

\begin{abstract}

  Efficient concurrent programs and data structures rarely employ
  coarse-grained synchronization mechanisms (\ie,~locks); instead,
  they implement custom synchronization patterns via fine-grained
  primitives, such as \emph{compare-and-swap}.
  % 
  Due to sophisticated thread interference scenarios, reasoning about
  such programs is challenging and error-prone, and can benefit from
  mechanization.
  
  In this paper, we present the first completely formalized framework for
  mechanized verification of \emph{full functional correctness} of
  fine-grained concurrent programs. Our tool is based on the recently
  proposed program logic FCSL.
  %
  It is implemented as an embedded domain-specific language in the
  dependently-typed language of the Coq proof assistant, and is
  powerful enough to reason about programming features such as
  higher-order functions and local thread spawning. By incorporating a
  uniform concurrency model, based on \emph{state-transition systems}
  and \emph{partial commutative monoids}, FCSL makes it possible to
  build proofs about concurrent libraries in a thread-local,
  compositional way, thus facilitating scalability and reuse:
  libraries are verified \emph{just once}, and their specifications
  are used ubiquitously in client-side reasoning.
  %
  We illustrate the proof layout in FCSL by example, and report on
  our experience of using FCSL to verify a number of concurrent
  algorithms and data structures.

\end{abstract}


\section{Introduction}
\label{sec:introduction}

Formally defining the behavior of highly parallel concurrent objects
is a fundamental challenge, which requires the program designer to
find a trade-off between the desired performance on a multicore
processor, possibly enabled by reduced contention, and the safety
guarantees, implied by the chosen correctness conditions.
 
Traditionally, the correctness of concurrent objects is defined using
\emph{event histories},\footnote{Here, \emph{event} stands for a call
  to or a return from an object's method.}  by (\emph{i}) providing a
\emph{specification set}, describing all ``basic'' behaviors that the
object's client might observe when using it, and (\emph{ii}) fixing a
\emph{consistency condition} that relates the concurrently observable
behaviors to the ones in the specification set.
% 
In the majority of the cases, the specification set is taken to be the
histories of the concurrent object's \emph{sequential} behaviors, in
which the calls are immediately followed by the corresponding returns.
 
The most well-known concurrent correctness condition is
\emph{linearizability}~\cite{Herlihy-Wing:TOPLAS90}, which requires
for each concurrent history of an object to exist a mapping to a
sequential history, such that the ordering of two operations, defined
as matching call/return pairs, is preserved if they are performed by
the same thread, or if they do not overlap.
% 
In the twenty five years since it invention, linearizability has been
shown to be remarkably scalable as a correctness condition, widely
applicable to capture the behavior of implementations of multiple
concurrent objects with intuitive sequential specifications, such as
stacks, queues, sets, locks, snapshots and their combinations, and
even suitable for automatic synthesis of some concurrent
objects~\cite{Vechev-Yahav:PLDI08}. Thanks to the compositional proof
method, based on \emph{linearization points}, proofs of
linearizability in most of the cases are amenable for practical
computer-aided
verification~\cite{Burckhardt-al:PLDI10,Derrick-al:TOPLAS11,Vafeiadis:CAV10,Amit-al:CAV07,Shacham-al:OOPSLA11}.
%
Furthermore, linearizability with respect to a sequential
specification implies observational
refinement~\cite{Filipovic-al:TCS10,Emmi-al:PLDI15}, which allows one
to verify the client code of linearizable concurrent objects
efficiently by replacing the actual object implementation with its
\emph{atomic} counterpart.

However, in order to be linearizable with respect to a sequential
specification, an implementation of a concurrent object should
inherently incorporate certain costly synchronization
primitives~\cite{Attiya-al:POPL11}, which then become points of high
contention and prevent efficient
parallelization~\cite{Shavit:CACM11}. The way to remedy this situation
for enabling better scalability is to change the semantics of a
concurrent object, \ie, the correctness criterion, redefining its
admissible behaviors.

The outlined \emph{history-based} approach for concurrency
specification allows one to relax the semantics of a concurrent object
by either changing the specification set to include other histories
than just the sequential ones, or by changing the consistency
condition, typically quantifying over the possible interference
patterns, which can be observed while using the object. The first
option is taken, for example, in the
works~\cite{Hemed-Rinetzky:PODC14,Henzinger-al:POPL13}, resulting in
such correctness-defining frameworks as \emph{concurrency-aware
  linearizability} and \emph{quantitative relaxation},
correspondingly. The second option is exemplified by such concurrency
correctness criteria as \emph{quiescent
  consistency}~\cite{Aspnes-al:JACM94,Derrick-al:FM14},
\emph{quantitative quiescent
  consistency}~\cite{Jagadeesan-Riely:ICALP14},
\emph{quasi-linearizability}~\cite{Afek-al:OPODIS10}, and \emph{local
  linearizability}~\cite{Haas-al-local15}.

Adopting a new correctness condition for concurrent objects comes with
a price. First, in order to make the newly developed criterion
\textbf{(a)} \emph{scalable}, one has to prove it to be compositional
with respect to combination of multiple concurrent data
structures. Second, in order to make it \textbf{(b)} \emph{practical},
one needs to supply it with a principled proof technique, enabling
efficient correctness checking. Finally, to make it \textbf{(c)}
\emph{relevant for the program verification at large}, one has to
devise a method for exploiting the provided safety guarantees for the
sake of reasoning about client code that uses the concurrent object.
%
While compositionality was formally established for most of the listed
above
criteria~\cite{Aspnes-al:JACM94,Jagadeesan-Riely:ICALP14,Afek-al:OPODIS10,Haas-al-local15},
only few of them were equipped with proof
techniques~\cite{Derrick-al:FM14,Zhang-al:ASE13}, and we are not aware
of any correctness conditions other than linearizability being
employed for client-side reasoning. This state of affairs makes us
pose the following research question:

\vspace{4pt}
\noindent
\textbf{Q:} \emph{Is there a generic specification approach that
  allows one to formally define adequate correctness conditions for
  arbitrary concurrent objects, immediately enjoying the properties
  \emph{\textbf{(a)}--\textbf{(c)}}}?
%
\vspace{4pt}

\noindent
We seek an answer to this question by exploring the opportunities,
opened by the recent advances in the \emph{state-based} approach for
concurrency verification using \emph{Hoare-style program logics}.

In modern program logics for
concurrency~\cite{Feng-al:ESOP07,Vafeiadis-Parkinson:CONCUR07,Feng:POPL09,DinsdaleYoung-al:ECOOP10,Nanevski-al:ESOP14,Svendsen-Birkedal:ESOP14,ArrozPincho-al:ECOOP14,Jung-al:POPL15,Raad-al:ESOP15,Fu-al:CONCUR10},
specifications (or \emph{specs}) for methods of concurrent objects
(and programs in general) are represented by \emph{Hoare tuples}
$\spec{P}~e~\spec{Q}@\ucon$, where $e$ is a program, $P$ is a
precondition that constrains a state in which the program is safe to
run, and $Q$ is a postcondition, describing a state upon the program's
termination. The last component of the tuple, $\ucon$, whose exact
syntax is different for various program logics, is typically referred
to as a \emph{concurrent protocol} or \emph{concurrent resource}.
%
% \an{Why not call $U$ a \emph{resource}, instead of concurrent
%   protocol.  Protocol is the name that has been pushed by CAP, but
%   resource is what everyone \emph{before} such as Brookes and O'Hearn,
%   have used for the concept.}
% 
% \is{Okay, I added the concurrent resource as an alternative. I don't
%   think it's important which name has appeared earlier, but in my
%   opion the world ``protocol" better explains the purpose of the
%   concept, whereas ``resource'' can be confused with the actual state,
%   e.g., heap.}
%
It defines the invariants of the shared state, that are respected by
all threads working with it
concurrently~\cite{Nanevski-al:ESOP14,OHearn:TCS07} and/or the allowed
state changes that the threads can make depending on the roles they
take in the protocol~\cite{Jones:TOPLAS83}.
%
To ensure thread-locality of the concurrent Hoare-style specs, the
assertions $P$ and $Q$, should be proven \emph{stable} with respect to
the concurrent protocol, \ie, they should be invariant under possible
changes that interfering environment threads can make to the state
according to $\ucon$.

Program logics provide a naturally compositional way to specify
concurrency: once a concurrent object is verified against a suitable
spec, its code is not required to be re-examined ever
again. Therefore, specification and verification of the client
programs (including other data structures that use the object) can be
performed out of the object's spec.
%
Program logics were used with a great success to specify and verify
such concurrent data structures and algorithms as
barriers~\cite{Dodds-al:POPL11,Hobor-Gherghina:ESOP11},
indices~\cite{ArrozPincho-al:OOPSLA11}, flat
combiner~\cite{Turon-al:ICFP13,Sergey-al:ESOP15}, shared graph
manipulation~\cite{Raad-al:ESOP15,Sergey-al:PLDI15}, as well as their
multiple client programs.
%
Program verification in such logics is done structurally, \ie, by
systematically applying syntax-directed inference rules, until the
spec is proven, and by now several mechanized tools for logic-based
reasoning about concurrency has been
released~\cite{Sergey-al:PLDI15,Appel-al:BOOK14}.

That is, the program logics-based approach seems like an answer to the
question we posed above, as logical specs satisfy the properties
\textbf{(a)}--\textbf{(c)}. But in order to fully adopt it, we first
need to figure out how to express the existing patterns of
\emph{history-based} reasoning in terms of \emph{state-based} logical
specs.
%
Several attempts to do so have been taken
recently~\cite{Fu-al:CONCUR10,Gotsman-al:ESOP13,Sergey-al:ESOP15,OHearn-al:PODC10},
however, all these works were focusing exclusively on specifying
\emph{linearizable} concurrent objects, thus, making us to rephrase
the question \textbf{Q} as follows:

\vspace{4pt}
\noindent
\textbf{Q$'$:} \emph{Can we employ existing program logics for
  specifying \textbf{non-linearizable} concurrent objects and
  reasoning about their clients?}
% 
\vspace{4pt}

\vspace{-8pt}
\noindent
In this paper we answer to this question affirmatively.

\subsection{Logic-based specifications for concurrent objects}
\label{sec:logic-based-spec}

We build our solution on the recently made observation that histories,
describing relevant atomic changes in the logical state of a
concurrent object, can be expressed as an instance of \emph{auxiliary
  state}~\cite{Sergey-al:ESOP15} (a generalization of auxiliary, or
\emph{ghost}, variables~\cite{Owicki-Gries:CACM76}, customary in
logic-based concurrency verification).
%
That is, reasoning about histories follows exactly the same patterns
that reasoning about \emph{heaps} follows in separation
logic~\cite{Reynolds:LICS02}.
%
We then use the expressive power of concurrent protocols to leverage
this observation for specifying and verifying non-linearizable
concurrent objects.  
%
% \an{Throw in the word "separation logic" somewhere here, to make the
%   point of our observation more precise. Strictly speaking, that you
%   can have histories as ghosts was an old observation (they even call
%   them "history variables" in the model checking works). Our
%   observation was that \emph{separation logic} can natually and
%   compositionally handle history variables, using the algebraic
%   structure of PCMs.}
%
% \is{okay, done}
%

In particular, we show how to define invariants of a concurrent object
in a way that they constrain the real and auxiliary state, capturing a
suitable specification set of histories (\cf (\emph{i})), \eg,
\emph{concurrency-aware} one.
%
The freedom to define histories in any way we need (as they are just
an instance of auxiliary state) allows us to record additional logical
information into them, capturing quantitative aspects of the expected
interference. In combination with a possibility to describe the
allowed changes in the real and auxiliary state of an object, it
provides us with a technique to express and verify diverse consistency
conditions on the histories (\cf (\emph{ii})).  

What is crucial for capturing the essence of most of these conditions
is the ability to formally quantify in \emph{thread-local} assertions
over the arbitrary effects produced by \emph{other} interfering
threads and the ``shape'' of the environment (\eg, a number of threads
running concurrently with the one being specified).
%
% \an{It sounds like we also need to work in the word
%   \emph{subjectivity}, in order to express what's new in our approach,
%   and why the previous work didn't quite succeed in capturing what we
%   propose. The point should be that quiescent consistency, and others,
%   just naturally want to have access to the contributions of others.}
%
% \is{okay, check the paragraphs above and below.}
%
Such ability, dubbed \emph{subjectivity}, has been introduced in the
recently developed \emph{Subjective} and \emph{Fine-grained Concurrent
  Separation Logics}, SCSL~\cite{LeyWild-Nanevski:POPL13} and
FCSL~\cite{Nanevski-al:ESOP14}. This is the reason why we have chosen
to use FCSL as a specification and verification framework.
%
% While some other logics could have been employed for this role (see
% Section~\ref{sec:related} for discussion on alternatives),

In addition to the native support of subjectivity, FCSL was appealing
to us because of its uniform model of thread-local resources, which is
based on \emph{partial commutative monoids} (PCMs) and can be
instantiated to reason about arbitrary state, auxiliary or real, such
as heaps, thread capabilities, and, indeed, histories.
%
Finally, FCSL has been implemented as a mechanized tool for concurency
verification~\cite{Sergey-al:PLDI15}, enabling provably sound
computer-aided reasoning about highly optimized concurrent objects,
whose state invariants tend to be quite complicated.

% \an{This section reads meekly, but it's the
%   most important one.  It should be strenghtened by bringing up the
%   points about subjectivity further above, as I said.  Subjectivity as
%   a crucial and most important idea, which makes all the
%   difference. If we don't mention it, peple like Noam Rinetzky or
%   Cesar Sanches will not get the feel of what's different.
%   Subjectivity is the delta that makes us succeed where everyone else
%   went into wrong directions.  Also, we should be more proud of our
%   work. It doesn't matter if "other logics" could have been
%   used. Subjectivity has been invented by SCSL and in the fine-grained
%   setting by FCSL. Just because some copy-cats later decided to
%   implement it in their logics doesn't mean we should bow to them. So
%   I think we don't need to mention "other logics can do it" part. Just
%   say that we use FCSL to explain the ideas of this paper, because
%   FCSL is based on, and introduced the idea of subjectivity.  Of
%   course, this should be done after subjectivity has been promoted
%   further above as the main idea that makes everything fly.}

\subsection{Contributions and paper outline}
\label{sec:contr-paper-outl}

In the remainder of the paper, we demonstrate the viability of the
logic-based approach for defining correctness conditions for highly
parallel concurrent objects by formally specifying and verifying two
concurrent data structures: an \emph{elimination-based exchange
  channel}~\cite{Scherer-al:SCOOL05} (or simply \emph{exchanger}) and
a simple \emph{counting network}~\cite{Aspnes-al:JACM94}, whose
behavior was previously described only in terms of dedicated
criteria~\cite{Hemed-Rinetzky:PODC14,Derrick-al:FM14,Jagadeesan-Riely:ICALP14}. We
then argue for the adequacy of the provided specs by modularly
verifying a series of concurrent client programs, which employ these
data structures.

\noindent
Specifically, in this work we make the following contributions:

\vspace{2pt}

\begin{itemize}

\item We describe a series of novel reasoning patterns that unify
  state-based and history-based approaches for specification and
  verification of concurrent objects.

\item We provide the first formal logic-based specification of an
  \emph{elimination-based concurrent
    exchanger}~\cite{Scherer-al:SCOOL05} in the spirit of
  concurrency-aware linearizability~\cite{Hemed-Rinetzky:PODC14}
  (Section~\ref{sec:overview});

\item We specify and formally verify of a realistic client of the
  exchanger adapted directly from the \code{java.util.concurrent}
  library documentation~\cite{ExchangerClass} (Section~\ref{sec:cal}).

\item We give the first logical specification to a simple
  \emph{counting network}~\cite{Aspnes-al:JACM94} and verify two its
  clients, exploiting the implications of the proved spec in the
  spirit of \emph{quiescent consistency}~\cite{Derrick-al:FM14} and
  \emph{quantitative quiescent
    consistency}~\cite{Jagadeesan-Riely:ICALP14}
  (Section~\ref{sec:counting}).

\item We supply all examples from the paper with proof scripts that
  were mechanically checked in the Coq proof
  assistant~\cite{Coq-manual,Bertot-Casteran:BOOK,Sergey-al:PLDI15}.

\end{itemize}

\vspace{2pt}

\noindent
Section~\ref{sec:background} provides necessary minimal background on
program logics for concurrency and key concepts of
FCSL~\cite{Nanevski-al:ESOP14}. In Section~\ref{sec:discussion} we
discuss other possible applications of the proposed approach for
reasoning about concurrent objects. We cover the relevant related work
in Section~\ref{sec:related} and conclude in
Section~\ref{sec:conclusion}.

\an{We should have in the intro with a paragraph like follows: That
  linearizability can adequatly be replaced by Hoare-style reasoning
  has been already argued by the previous work on HOCAP and ICAP,
  which employ the method of parametrization (to be skethed in the
  related work section). In this paper, we argue that similar
  replacement can be carried out for three other alternative
  consistency criteria such as Concurrency-aware linearizability
  (CAL), quiescent consistency, and quantitative quiescent
  consistency. Moreover, the last two consistency criteria seem
  impossible to address by a method of parametrization, at least
  without some significant and complicated meta-theoretic additions
  (e.g., prophecy variables), whereas here we show how they can be
  easily supplanted using reasoning based on subjectivity combined
  with histories.}

\is{I disagree with the remark by Aleks almost entirely. First, I'm
  sick and tired of giving credit to CAP-like approaches for something
  they don't have enev a slight idea how to do. While I'll most
  certainly put something on them into the related work, I don't want
  to have enything on them in the intro, otherwise there will be
  another round of what we've seen already a year ago.
%  
  Second, I don't think that remark on prophecies is sound: the
  counting network example doesn't have anything reminiscent to
  prophecy-requiring linearization points. That is, any comparison to
  LP-based methods, as they're done in CAP, is dangerous and
  misleading, as (a) they might be able to do it via some other
  callback-related mumbo-jumbo (or they might not, but we'll get a
  strong reject on that grounds anyway, just like the last tim), or
  (b) they indeed cannot do it, but this fact has nothing to with with
  their prophecies-related troubles. 
%
  Finally, the only people who will be able to understand this
  comments are those who will give us strong reject because of it
  (just like the last time). That is why I suggest us to focus bringin
  Henzinger and Rinetzky-like crown on our side by speaking the
  language they understand.
%
  So, the bottom line: I'm strongly against any specific on
  CAP-related logics or any of this stuff in the intro.
}

\an{Gee, that's a strong sentiment. I will just reiterate that HOCAP
  and ICAP have suggested that Hoare logic should replace
  linearizability. It seems prudent to be generous and give them
  credit for that, especially as it doesn't cost us anything
  (basically, just one line of space), and doesn't diminish from our
  contribution at all. A year ago, we got rejected form POPL precise
  because we didn't have a forward-pointer in the intro to the related
  work section, where the comparison was done. They accused us in the
  after-rebutal comment of setting '`wrong expectations'' in our
  intro. After that, in the ESOP version, we did put the forward
  pointer. So, I really don't think a forward pointer would hurt us
  here, and can only help. As for prophecies, I just put them as a
  side comment. I don't think we really have to mention them. We could
  simply say that its unclear that parametrization can be used to
  handle quiescense and quantitativeness, wihtout speculiating what's
  needed to fix that. I do personally think that prophecies will
  probably be OK for that, if they can get them. But then, I think
  they'll never get them, precisely because subjectivity and histories
  is what you need here :-). As for Henzinger and Rinetzky crowd:
  well, Philippa will be reviewing this, so expect to get someone from
  the CAP crowd too.}


%\lipsum[1-4]


%% \gad Macros to refer to snapshot's pointers and line-numbers are
%% defined together with the Figure in

%\newcommand{\fx}{\text{fx}}
%\newcommand{\fy}{\text{fy}}
%\newcommand{\x}{\text{x}}
%\newcommand{\y}{\text{y}}
%\newcommand{\s}{\text{S}}

\newcommand{\fx}{\mathit{fx}}
\newcommand{\fy}{\mathit{fy}}
\newcommand{\x}{x}
\newcommand{\y}{y}
\newcommand{\s}{S}

%%\begin{wrapfigure}[9]{r}[0pt]{0.4\textwidth} 
%% \begin{figure}
%% %
%% \centering
%% \begin{tabular}{l l l}
%% %
%% %  
%% \begin{minipage}[l]{.30\textwidth}
%% \begin{alltt}
%% \num{1}  write (p, v): () \{
%% \num{2}    \act{write} (p, v);
%% \num{3}    b <- \act{read} (S);
%% \num{4}    \textbf{if} b 
%% \num{5}    \textbfthen \act{transfer} (p, v);
%% \num{6}    \textbf{else skip};\}
%% \end{alltt} 
%% \end{minipage}
%% %
%% & \hfill
%% %
%% \begin{minipage}[l]{.6\textwidth}
%% \begin{alltt}
%% \num{1}  scan (): \(A {\times} A\)  \{
%% \num{2}    \act{write} (S, true);
%% \num{3}    \act{write} (fx,\( \bot\));
%% \num{4}    \act{write} (fy,\( \bot\));
%% \num{5}    vx <- \act{read} (x);
%% \num{6}    vy <- \act{read} (y);
%% \num{7}    \act{write}(S, false);
%% \num{8}    ox <- \act{read} (fx);
%% \num{9}    oy <- \act{read} (fy);
%% \num{10}   \textbf{let} rx = \textbf{if} ox \(\neq \bot\) \textbfthen ox \textbf{else} vx;  
%% \num{11}   \textbf{let} ry = \textbf{if} oy \(\neq \bot\) \textbfthen oy \textbf{else} vy;  
%% \num{11}   \act{relink}(rx, ry);
%% \num{12}   \textbf{return} (rx, ry);
%% \end{alltt} 
%% \end{minipage}
%% %
%% \end{tabular}
%% %
%% \caption{Jayanti's single-scanner, single-writer snapshot algorithm}
%% \label{fig:jayanti}
%% \end{figure}
%\end{wrapfigure}

\newcommand{\actwrite}[2]{{#1}\,{:=}\,{#2}}

% The following version saves a little more space
\begin{figure}
%
\centering
\begin{tabular}{c@{\ \ \ \ \ }c}
%  
\begin{minipage}[t][3.7cm][t]{.5\textwidth}
\small
\begin{alltt}
\num{1} write (p : ptr, v : \(A\)) \{
\num{2}  \actwrite{p}{v};
\num{3}  b \tbnd \act{read}(S);
\num{4}  if b 
\num{5}  then \actwrite{(f_of p)}{v}
\num{6}  {else return} \}

  f_of (p : ptr) \{
   return p = x ? fx : fy \}
\end{alltt}
\end{minipage}
%
&
\begin{minipage}[t][3.7cm][t]{.5\textwidth}
\small
\begin{alltt}
\num{ 7} scan (): \(A {\times} A\)  \{
\num{ 8}  \actwrite{S}{true};
\num{ 9}  \actwrite{fx}{\(\bot\)}; \actwrite{fy}{\(\bot\)};
\num{10}  vx \tbnd \act{read}(x); vy \tbnd \act{read}(y);
\num{11}  \actwrite{S}{false};
\num{12}  ox \tbnd \act{read}(fx); oy \tbnd \act{read}(fy);
\num{13}  rx \tbnd if (ox \(\neq\bot\)) then ox {else} vx;  
\num{14}  ry \tbnd if (oy \(\neq\bot\)) then oy {else} vy;  
\num{15}  return (rx, ry) \}
\end{alltt} 
\end{minipage}
%
\end{tabular}
%
\caption{Jayanti's single-scanner/single-writer snapshot algorithm.}
\label{fig:jayanti-snapshot}
\end{figure}



\newcommand{\jywrite}{\texttt{write}\xspace}
\newcommand{\jyscan}{\texttt{scan}\xspace}

\section{Verification challenge and main ideas}
\label{sc:overview}


Jayanti's snapshot algorithm~\cite{Jayanti+STOC05} provides the
functionality of a shared array of size $m$, operated on by two
procedures: \jywrite, which stores a given value into an element, and
\jyscan, which returns the array's contents. We use the
\emph{single-writer}/\emph{single-scanner} version of the algorithm.
which assumes that at most one thread writes into an element, and at
most one thread invokes the scanner, at any given time. In other
words, there is a scanner lock and $m$ per-element locks. A thread
that wants to scan, has to acquire the scanner lock first, and a
thread that wants to write into element $i$ has to acquire the $i$-th
element lock. However, scanning and writing into different elements
can proceed concurrently.
% 
%where a thread acquires a writer lock for a particular element before
%writing into it, and a scanner before scanning. A scanner lock does
%not preclude writing, and a writer lock for an element does not
%preclude scanning, or writing into other elements. 
This is the simplest of Jayanti's algorithms, but it already exhibits
linearization points of dynamic nature. We also restrict the array
size to $m\,{=}\,2$ (\ie, we consider two pointers $\x$ and $\y$,
instead of an array). This removes some tedium from verification, but
exhibits the same conceptual challenges.
 
The difficulty in this snapshot algorithm is ensuring that the scanner
returns the most recent snapshot of the memory. A na\"{i}ve scanner, which
simply reads $\x$ and $\y$ in succession, is unsound. To see why,
consider the following scenario, starting with $\x=5$, $\y=0$. The
scanner reads $\x$, but before it reads $\y$, another thread preempts
it, and changes $\x$ to $2$ and, subsequently, $\y$ to $1$. The
scanner continues to read $\y$, and returns $\x=5, \y=1$, which was
never the contents of the memory. Moreover, $(\x, \y)$, changed from
$(5,0)$ to $(2, 0)$ to $(2, 1)$ as a result of distinct
non-overlapping writes; thus, it is impossible to find a linearization
point for the scan because linearizability only permits reordering of
overlapping operations.

%\ab{Remove rest?} by dynamically reordering non-overlapping
%operations, as permitted by linearizability (though we show further
%below a scenario when {\jyscan} is justified in returning a pair that
%was not the contents of the memory).

%\gad{Do we make the latter example a graph/ figure somehow? We have
%  done so for the slides}

To ensure a sound snapshot, Jayanti's algorithm internally keeps
additional \emph{forwarding pointers} $\fx$ and $\fy$, and a boolean
\emph{scanner bit} $\s$. The implementation is given in
Figure~\ref{fig:jayanti-snapshot}.\footnote{Following Jayanti, we
  simplify the presentation and omit the locking code that ensures the
  single-writer/single-scanner setup. Of course, in our Coq
  development~\cite{CoqFiles}, we make the locking explicit.}
%
The intuition is as follows. A writer storing $v$ into $p$
(line~\lineWrtWrt), will additionally store $v$ into the forwarding
pointer for $p$ (line~\lineWrtFwd), provided $S$ is set. If the
scanner missed the write and instead read the old value of $p$
(lines~\lineScanReadsX--\lineScanReadsY), it will have a chance to
catch $v$ via the forwarding pointer
(lines~\lineScanReadsFX--\lineScanReadsFY). The scanner bit $S$ is
used by writers (line~\lineWrtChk) to detect a scan in progress, and
forward $v$.

{
%\setlength{\belowcaptionskip}{-5pt} 
\begin{figure}[t]
%
\captionsetup[subfigure]{justification=centering}
\centering  
\begin{subfigure}[t]{1\textwidth}
\centering
\begin{tabular}{l || l || l}
  \texttt{l: }\texttt{write (x,2);}\quad &
   \multirow{2}{*}{\texttt{c: scan ()}}\quad & 
    \multirow{2}{*}{\texttt{r: write (x,3)}}  \\
  \phantom{\texttt{l: }}\texttt{write (y,1)} & &   
\end{tabular}
\caption{\label{fig:weird:code}Parallel composition of three threads \texttt{l, c, r}.}
\end{subfigure}\\

\begin{subfigure}[b]{1\textwidth}
\begin{tabular}{l@{\hfill} l@{\hfil}}
\begin{minipage}[t]{0.5\textwidth}
\begin{alltt}
 \num{1}  c: \actwrite{S}{true}
 \num{2}  c: \actwrite{fx}{\(\bot\)}
 \num{3}  c: \actwrite{fy}{\(\bot\)}
 \num{4}  c: \act{read}(x)  // vx <- 5
 \num{5}  c: \act{read}(y)  // vy <- 0
 \num{6}  l: \actwrite{x}{2}
 \num{7}  l: \act{read}(S)  // b <- true
 \num{8}  l: \actwrite{fx}{2} 
 \num{9}  l: return ()
\num{10}  r: \actwrite{x}{3}
\end{alltt}
\end{minipage}
&
\begin{minipage}[t]{0.33\textwidth}
\begin{alltt}
\num{11} l: \actwrite{y}{1}
\num{12} l: \act{read}(S)  // b <- true
\num{13} l: \actwrite{fy}{1}
\num{14} l: return ()
\num{15} c: \actwrite{S}{false}
\num{16} r: \act{read}(S)  // b <- false
\num{17} r: return ()
\num{18} c: \act{read}(fx) // ox <- 2
\num{19} c: \act{read}(fy) // oy <- 1
\num{20} c: return (2,1)
\end{alltt} 
\end{minipage}
%
\end{tabular}
\caption{\label{fig:weird:exec} A possible interleaving of the threads
  in~(\subref{fig:weird:code}).}
\end{subfigure}
\caption{\label{fig:weird} An example leading to a scanner miss.%
}
\end{figure}
}

 
As Jayanti proves, this implementation \emph{is} linearizable. Informally,
every overlapping calls to \jywrite~and \jyscan~can be rearranged to
appear as if they occurred sequentially.  To illustrate, consider the
program in Figure~\ref{fig:weird:code}, and one possible interleaving
of its primitive memory operations in Figure~\ref{fig:weird:exec}. The
threads {\tt l}, {\tt c}, and {\tt r}, start with $\x = 5, \y = 0$.
%
The thread {\tt c} is scheduled first, and through lines~1--5 sets the
scanner bit, clears the forwarding pointers, and reads $\x = 5, \y =
0$. Then {\tt l} intervenes, and in lines~6--9, overwrites
$\x$ with $2$, and seeing $\s$ set, forwards $2$ to $\fx$. Next, {\tt
  r} and {\tt l} overlap, writing $3$ into $\x$ and $1$ into
$\y$. However, while $1$ gets forwarded to $\fy$ (line 13), $3$ is not
forwarded to $\fx$, because $\s$ was turned off in line 15 (\ie, the
scan is no longer in progress). Hence, when {\tt c} reads the
forwarded values (lines 18, 19), it returns $\x = 2, \y = 1$.

While $\x\,{=}\,2, \y\,{=}\,1$ was never the contents of the memory,
returning this snapshot is nevertheless justified because we can
\emph{pretend} that the scanner \emph{missed} {\tt r}'s write of
$3$. Specifically, the events in Figure~\ref{fig:weird:exec} can be
\emph{reordered} to represent the following sequential execution:
%
\begin{equation}
\hfill \mathtt{write\, (x, 2);\ write\, (y,1);\ scan\, ();\ write\, (x,
  3)}\hfill \label{eq:lin}
\end{equation}
%
Importantly, the client programs have no means to discover that a
different scheduling actually took place in real time, because they
can access the internal state of the algorithm only via interface
methods, \jywrite~and \jyscan.

This kind of temporal reordering is the most characteristic aspect of
linearizability proofs, which typically describe the reordering by
listing the linearization points of each procedure. At a linearization
point, the procedure's operations can be spliced into the execution
history as an uninterrupted chunk. For example, in Jayanti's proof,
the linearization point of \jyscan~is at line~\lineScanUnsetsS\ in
Figure~\ref{fig:jayanti-snapshot}, where the scanner bit is unset. The
linearization point of \jywrite, however, may vary. If
\jywrite~starts before an overlapping \jyscan's line~\lineScanUnsetsS,
and moreover, the \jyscan~misses the \jywrite---note the dynamic and
future-dependent nature of this property---, then \jywrite~should
appear after {\tt scan}; that is, the \jywrite's linearization point
is right after \jyscan's linearization point at line~\lineScanUnsetsS.
%
Otherwise, \jywrite's linearization point is at line~\lineWrtWrt.
%
In the former case, \jywrite~exactly has a non-local and
future-dependent linearization point, because the decision on the
logical order of this \jywrite~depends on the execution of \jyscan~in
a different thread. This decision takes effect on
lines~\lineScanReadsFX--\lineScanReadsFY, which can take place
\emph{after} the execution of \jywrite~has terminated.
%
For instance, in Figure~\ref{fig:weird:exec} the execution
of \jywrite~in \texttt{r} terminates at step 17, yet, in Jayanti's
proof, the decision to linearize this \jywrite\ after the
overlapping \jyscan\ is taken at line~18, when the \jyscan\ reads the
value from the previous \jywrite.

%%\gad{Rephrased the paragraph above to answer {\sf R1.Q5}}

%% \gad{Well, the non-regional argument is subtle and here is used with
%%   the wrong example: In this particular case, although the scanner bit
%%   is unset later at 15, the LP' of {\tt l} is fixed at line~11
%%   regardless of the future -- witnessed by the fact that it finishes
%%   green. The non-regionality argument has to be made about the
%%   position of the write to \x done by {\tt r}, which is the write that
%%   is relinked.}

%% \gad{When {\tt r} finishes in line~18, it's position in the final
%%   order is not settled as it depends on the scanners future: this
%%   write is missed by the scanner, and has to be relinked. This
%%   example, though showcases why relink is needed and how it works it
%%   does not showcase non-regionality: when 18 finishes, you have the
%%   information in lines 1-18 to determine that his position will be
%%   changed by scan before the end, so it can be linearized in line 18.}

\begin{figure}[t]
%\captionsetup[subfigure]{justification=centering}
\begin{subfigure}[t]{0.49\textwidth}
\includegraphics[width=6.1cm]{relink-before3.pdf}
\caption{\label{fig:reorder:before}} % Logical $=$ Real Time order, not a snapshot}
\end{subfigure} \hfill
\begin{subfigure}[t]{0.49\textwidth}
\includegraphics[width=6.1cm]{relink-after3.pdf}
\caption{\label{fig:reorder:after}} % Logical $\neq$ Real Time order, snapshot OK}
\end{subfigure}%
%
\caption{\label{fig:reorder} Changing the logical ordering (solid line
  $\ordlist$) of write events from (5, 0, 2, 3, 1) in
  (\subref{fig:reorder:before}) to (5, 0, 2, 1, 3) in
  (\subref{fig:reorder:after}), to reconcile with {\tt scan} returning
  the snapshot $\x=2, \y=1$, upon missing the write of $3$. Dashed
  lines $\hist$ represent real-time ordering.}
\end{figure}


Obviously, the high-level pattern of the proof requires tracking the
\emph{logical ordering} of the \jywrite\ and \jyscan\ events, which
differs from their \emph{real-time ordering}. As the logical ordering
is inherently dynamic, depending on properties such as
\jyscan\ missing a \jywrite, we formalize it in Hoare logic, by
keeping it as a list of events in auxiliary state that can be
dynamically reordered as needed. For example, Figure~\ref{fig:reorder}
shows the situation in the execution of \jyscan~that we reviewed
above. We start with the (initializing) writes of $5$ and $0$ already
executed, and our program performs the writes of $2$, $3$ and $1$ in
the real time order shown by the position of the events on the dashed
lines. In Figure~\ref{fig:reorder:before}, the logical order
$\ordlist$ coincides with real-time order, but is unsound for the
snapshot $\x=2, \y=1$ that \jyscan~wants to return. In that case, the
auxiliary code with which we annotate \jyscan, will change the
sequence $\ordlist$ in-place, as shown in
Figure~\ref{fig:reorder:after}.

Our specification and verification challenge then lies in reconciling
the following requirements. First, we have to posit specs that
say that \jywrite\ performs a write, and \jyscan\ performs a scan of
the memory, with the operations executing in a single logical
moment. Second, we need to implement the event reordering discipline
so that a method call only reorders events that overlap with it; the
logical order of the past events should be preserved. This will be
accomplished by introducing yet further structures into the auxiliary
state and code. Finally, the specs must hide the specifics of
the reordering discipline, which should be internal to the snapshot
object. Different snapshot implementations should be free to implement
different reorderings, without changing the method specs.


%Our challenge then lies in reconciling the following two conflicting
%requirements. First, we need to implement the reordering discipline so
%that the subsequent calls to \jywrite~and \jyscan~preserve the
%established logical order of the past events. This will be
%accomplished by introducing yet further structures into the auxiliary
%state and code. Second, we have to engineer Hoare triples for
%\jywrite~and \jyscan~to be \emph{intuitive} and \emph{helpful} to
%clients, but also to \emph{not expose} the specifics of the reordering
%discipline, which is internal to the snapshot object\footnotemark.
%%We discuss these issues next.
%\footnotetext{\ie we want to give the methods {\it principal} specifications}




\section{Outline of the Mechanized Development}
\label{sec:devel}

We next discuss how the above informal overview is mechanized in
Coq. We start with the definition of \code{span_tp} and proceed to
explain all of its components.
%
%
The specifications and code shown will be very similar to what's in
our Coq files, though, to improve presentation, we occasionally take
liberties with the order of definitions and notational
abbreviations. We do not assume any familiarity with Coq, and explain
the code displays as they appear. We also omit the proofs and
occasional auxiliary definitions, which can be found in the FCSL code,
accompanying the paper~\cite{Sergey-al:PLDI15TR}.

{
\setlength{\belowcaptionskip}{-10pt} 
\begin{figure}[t!]
{\centering 
\begin{lstlisting}[basicstyle=\footnotesize\ttfamily]
Definition span_tp := forall (x : ptr),
 {i (g1 : graph (joint i))}, STsep [SpanTree sp] 
   (*  precondition predicate   *)
   (fun s1 => i = s1 /\ 
               (x == null \/ x \in dom (joint s1)), 
   (*  postcondition predicate *)
    fun (r : bool) s2 => exists g2 : graph (joint s2), 
      subgraph g1 g2 /\ 
      if r then x != null /\ exists t,
        self s2 = self i \+ t /\ tree g2 x t /\ 
        maximal g2 t /\ front g1 t (self s2 \+ other s2)
      else (x == null \/ mark g2 x) /\ self s2 = self i). 
\end{lstlisting}
\vspace{-7pt}   
}
\caption{Specification \code{span_tp} of the \code{span} procedure.}
\label{fig:span_tp}
\end{figure}
}


\subsection{The Definition of the Type~\code{span_tp}} 

The type \code{span_tp} is described in Figure~\ref{fig:span_tp}.
%
It is an example of a \emph{dependent type}, as it takes formal
arguments in the form of variables \code{x}, \code{i} and \code{g1},
that the body of the type can use, \ie, \emph{depend on}.  The roles of
the variables differ depending on the keyword that binds them.
%
For example, the Coq keyword \code{forall} binds the variable \code{x}
of type \code{ptr}, and indicates that \code{span_tp} is a
specification for a procedure that has \code{x} as input. Indeed,
\code{span} is exactly such a procedure, as apparent from
Section~\ref{sec:overview}. Using \code{forall} to bind \code{x}
allows \code{x} to be used in the body of \code{span_tp}, \emph{but
  also in the body} of \code{span} (Figure~\ref{fig:coq-span}).
%
On the other hand, \code{i} and \code{g1} are bound by FCSL binder
\code{\{...\}}. This binding is different; it allows \code{i} and
\code{g1} to be used in the body of \code{span_tp}, \emph{but not in
  the procedure} \code{span}. In terminology of Hoare-style logic,
\code{i} and \code{g1} are \emph{logical variables}
(aka.~\emph{ghosts}), which are used in specs, but not in the code.
%
\code{STsep} is a Coq macro, defined by FCSL announcing that what
follows is a Hoare-style partial correctness specification for a
concurrent program. The component \code{SpanTree sp} in the brackets
is the concurroid whose protocol \code{span_tp} respects. We will
define \code{SpanTree} shortly. Finally, the parentheses
include the precondition and the postcondition (defined as Coq's
\code{fun}ctions) that we want to ascribe to \code{span}. The
precondition is a predicate over the pre-state \code{s1}. The
postcondition is a predicate over the boolean result \code{r} and
post-state \code{s2}. As customary in many programming languages, Coq
included, we omit the types of various variables when the system can
infer them (\eg, the variables \code{i}, \code{s1} and \code{s2} are
all of type \code{state}).

The precondition says that the input \code{x} is either \code{null}
(since \code{span} can be called on a leaf node), or belongs to the
domain of the input heap, and hence is a valid node in the
heap-represented graph. The heap is computed as the projection
\code{joint} out of the input state \code{s1}, which \code{i}
snapshots.  The projections \code{self} and \code{other} are sets of
marked nodes, belonging to the caller of \code{span} and to its
environment, respectively.

The postcondition says that in the case the return result is \code{r =
  false}, the pointer \code{x} was either \code{null} or already
marked. Otherwise, there is a set of nodes \code{t} which is freshly
marked by the call to \code{span}; that is, \code{self s2} is a
\emph{disjoint union} (\code{\\+}) of \code{t} with the set of nodes
marked in the pre-state \code{self i}. The set \code{t} satisfies
several important properties.
%
First, \code{t} is a subtree in the graph, \Code{g2}, of the
post-state \code{s2}, with root \code{x}. Second, the tree \code{t} is
\emph{maximal}, \ie, it cannot be extended into a larger tree by
adding more nodes from \code{g2}, as all the edges between \code{t}
and the rest of the graph have been severed by \code{span}. Third, all
the nodes immediately reachable from \code{t} in the \emph{initial}
state \code{i} (\ie, \code{t}'s front) are marked in \code{g2} either
by this or some other thread (\code{self s2 \\+ other s2}). That is,
\code{span} did not leave any reachable nodes unmarked; if such nodes
existed, \code{span} would not have terminated.
%
Finally, in both cases, \code{subgraph g1 g2} states that the final
graph \code{g2} is obtained by marking nodes and removing edges from
the initial graph \code{g1}; no new edges are added, no nodes are
un-marked.

We close the description of \code{span_tp} by noting its interesting
\emph{bi-directional} nature. It contains properties such as
\code{tree} and \code{maximal}, stated over the post-state graph
\code{g2} (forward direction), but also the property \code{front}
which is stated of the pre-state graph \code{g1}, and can be stated
only in relation to \code{s2} (backward
direction). The backward direction is a crucial component in the proof
that the top-most call to \code{span}, shielded from interference by
\textsf{hide}, indeed marked all the nodes and, hence, constructed a
spanning tree.

%The proof of this spec of the \code{span} program is about 200 LOC of
%Coq proof script, which is mostly dedicated to splitting
%\emph{self}-contributions between parallel threads and combining them
%upon their termination. Importantly, the specifications for recursive
%calls relate the initial/final graph \wrt the recursive calls
%themselves, but \emph{not} the initial/final graph of the enclosing
%call. This is why we need to reason in terms of forwards/backwards
%stability, described previously, to derive, in particular, the
%assertion about \code{front} at the \emph{end} \code{span}'s body from
%the postcondition of a recursive call to \code{loop}.

\subsection{Representing Graphs in a Heap}

Next we define the predicate \code{graph h}, which appears in
\code{span_tp} (Figure~\ref{fig:span_tp}), and says when a heap
\code{h} represents a graph. It does so if every pointer \code{x} in
\code{h} stores some triple \code{(b, xl, xr)}, where \code{b} is the
``marked'' bit, and \code{xl}, \code{xr} are pointers in the domain of
\code{h} (and, hence, are \code{x} itself or other nodes), or
\code{null} if \code{x} has no successors. 
%
\begin{lstlisting}
Definition graph (h : heap) := valid h /\ 
  forall x, x \in dom h -> exists (b : bool) (xl xr : ptr),
      h = x :-> (b, xl, xr) \+ free x h /\
      {subset [:: xl; xr] <= [:: null] ++ dom h}.   
\end{lstlisting}
%
The conjunct \code{valid h} says that the heap \code{h} doesn't contain duplicate
pointers. The notation \code{\\+} is overloaded and used for disjoint
union of sets of nodes in \code{span_tp}, and for disjoint union of
heaps in \code{graph}. In general, we use \code{\\+} for any PCM
$\pcmF$ operation.  \code{free x h} is the heap obtained by
deallocating \code{x} from \code{h}.  Finally, the last line is
concrete syntax for $\small{\{\mathtt{xl, xv}\} \subseteq \{\mathtt{null}\}
\cup \mathtt{dom\ h}}$.
%
%\an{Maybe remove \code{free x h} in favor of an extra heap variable.}

The \code{graph} predicate illustrates \emph{certified programming} in
Coq~\cite{Chlipala:BOOK}, \ie, the ability to use propositions as
types, and pass variables such as \code{g1} and \code{g2} that stand
for \emph{proofs} of the \code{graph} property, as inputs to other
types (\eg, \code{span_tp}) or functions. This ability enables
formally defining \emph{partial} functions over heaps that are
undefined when the heap doesn't encode a valid graph. An alternative
to this somewhat unique capability of dependent types is to encode
partial functions as relations, but that usually results in increase
in proof tedium and size.

Here are a few examples of such partial functions. Given a node (\ie,
a pointer) \code{x} and a proof that the heap \code{h} represents a
graph (written (\code{g : graph h})), we name \code{mark g x},
\code{edgl g x} and \code{edgr g x} the three components stored in the
pointer \code{x} in the heap (\ie, the ``marked'' bit, left, right
successor), and write \code{cont g x} for the whole triple.  By
default, these values are \code{false}, \code{null}, \code{null} if
\code{x} is not in the heap.
%
Each of these functions takes \code{h} as an argument; \ie, one could
also write \code{mark h g x} \etc., but we omit \code{h} as it can be
inferred from \code{g}'s type, following Coq's standard notational
abbreviation.

We can now define the remaining predicates used in \code{span_tp} in
Figure~\ref{fig:span_tp}. For all of the definitions, we assume that variables
\code{h} and (\code{g : graph h}) are in scope, and omit them. We also use
\code{ptr_set} as an alias for finite maps from pointers to the unit
type.\footnote{This is a bit expedient way of implementing finite
  sets, but it saves work by reusing an extensive library of finite
  maps, also used for formalizing heaps.}

First, we define the function \code{edge}, which represents the
incidence relation for the graph \code{g}.
%
\begin{lstlisting}
Definition edge (x y : ptr) := (x \in dom h) && 
  (y != null) && (y \in [:: edgl g x; edgr g x]).
\end{lstlisting}
% 
Second, \code{tree x t} requires that \code{t} contains \code{x}, and
for any node \code{y}~$\in$~\code{t}, there exists a unique path (\ie,
a list of nodes) \code{p} from \code{x} to \code{y} via \code{edge}'s
links, which lies within the tree (\ie, the nodes \code{p} are a subset
of~\code{t}). Note how \code{edge} is \emph{curried}, \ie, passed to
\code{path} as a function, abstracted over arguments. This illustrates
that even simple mathematical mechanizations require higher-order
functions in order to~work.
%
\begin{lstlisting}
Definition tree (x : ptr) (t : ptr_set) := x \in dom t /\
  forall y, y \in dom t -> exists !p, 
    path edge x p /\ y = last x p /\ {subset p <= dom t}.   
\end{lstlisting} 
%
%
Third, \code{front\ t t'}, determines if the nodes reachable from
\code{t} in \emph{zero or one step} are included in \code{t'}.
%
\begin{lstlisting}
Definition front (t t' : ptr_set) :=
  {subset dom t <= dom t'} /\
  forall x y, x \in dom t -> edge x y -> y \in dom t'. 
\end{lstlisting}
%
Fourth, a tree \code{t} is maximal if it includes its front. A graph
is connected if there's a path from \code{x} to every other node
\code{y} in it.
%
\begin{lstlisting}
Definition maximal (t : ptr_set) := front t t.
Definition connected (x : ptr) (t : ptr_set) := forall y, 
  y \in dom t -> exists p, path edge x p /\ last x p = y.
\end{lstlisting}
%
Finally, \code{subgraph} codifies a number of properties between
pre-state \code{s1} and post-state \code{s2}, and their graphs
\code{g1}, \code{g2}. In particular: \code{g1}, \code{g2} contain the
same nodes (\code{=i} is equality on lists modulo permutation), the
set of self-marked and other-marked nodes only increase, edges out of
a node \code{y} can be changed only if the node is marked, and the
only change to the edges is nullification (that is, removal).
\begin{lstlisting}
Definition subgraph s1 s2 
  (g1 : graph (joint s1)) (g2 : graph (joint s2)) :=
  dom (joint s1) =i dom (joint s2) /\
  {subset dom (self s1) <= dom (self s2)} /\
  {subset dom (other s1) <= dom (other s2)} /\
  (forall y, ~~(mark g2 y) -> cont g1 y = cont g2 y) /\
  (forall x, (edgl g2 x \in [:: null; edgl g1 x]) /\
              (edgr g2 x \in [:: null; edgr g1 x])).
\end{lstlisting}

We close the description of the predicates used in \code{span_tp}, by
listing two important lemmas that relate them. The first lemma,
%
%\code{front_leq} lemma says that \code{front} is monotone \wrt its
%second argument. 
\code{max_tree2}, says that if \code{y1} and \code{y2} are
successors of \code{x} (\ie, \code{edge x} equals the set \code{[::
  y1; y2]} %\ab{Avoid the Coq syntax?} \an{Hmm. If I did it may be
% unclear what the syntax is refering to in the listing}
modulo permutation), and \code{ty1} and \code{ty2} are maximal trees
rooted in \code{y1} and \code{y2}, and moreover, \code{ty1} and
\code{ty2} are disjoint, then the set of nodes built from \code{x},
\code{ty1} and \code{ty2} by disjoint union (\code{\\+}) is a tree
itself, \ie, no edges connect \code{ty1} and \code{ty2} (the notation
\code{#x} is concrete syntax for the singleton finite map containing
node \code{x}).  This lemma is essential in proving that \code{span}
produces a tree, as mentioned in Section~\ref{sec:overview} for the
case $r_l\,{=}\,r_r\,{=}\,\mathtt{\small{true}}$.
%
%Lemma front_leq t t1 t2 : 
%  {subset dom t1 <= dom t2} -> front t t1 -> front t t2.
%
\begin{lstlisting}
Lemma max_tree2 x y1 y2 ty1 ty2 : 
  edge x =i [:: y1; y2] ->  tree y1 ty1 -> maximal ty1 ->
  tree y2 ty2 -> maximal ty2 -> valid (ty1 \+ ty2) -> 
  tree x (#x \+ ty1 \+ ty2).
\end{lstlisting}
The second lemma shows that \code{subgraph} is monotone \wrt the
stepping of environment threads in the \code{SpanTree} concurroid.
%
\begin{lstlisting}
Lemma subgraph_steps s1 s2
  (g1 : graph (joint s1)) (g2 : graph (joint s2)) :
  env_steps (SpanTree sp) s1 s2 -> subgraph g1 g2.
\end{lstlisting}
We used this lemma as the main tool in establishing a number of
stability properties in Coq, related to the conjuncts from the
definition of \code{subgraph g1 g2}. For example, the lemma implies
that if \code{x} is a node of \code{joint s1}, then it is so in a
stable manner; that is, \code{x} is a node in \code{joint s2} for any
\code{s2}, obtained from \code{s1} by changes via environment
interference.

\subsection{\code{SpanTree} Concurroid}
\label{sec:graph-conc-coher}

Next we define the \code{SpanTree} concurroid. Being an STS, the
definition includes the specification of the state space, and
transitions between states. In the case of concurroids, we have an
additional component: \emph{labels} (semantically, natural numbers)
that differentiate instances of the concurroid. Thus the
definition of \code{SpanTree} is parametrized by the variable
\code{sp}, which makes it possible to use several instances of
\code{SpanTree} with different labels in a specification of a single
program. For example, say we want to run two \code{span} procedures
in parallel on disjoint heaps. Such a program could be specified by a
Cartesian product of \code{SpanTree sp1} and \code{SpanTree sp2},
where the different labels \code{sp1} and \code{sp2} instantiate the
variable \code{sp}.

The state space of \code{SpanTree} is defined by the following state
predicate \code{coh}, which we call \emph{coherence predicate}.

%
%spanning tree concurroid \code{SpanTree} has already appeared in the
%specifications of the \code{trymark} action and the \code{span_tp}
%spec (Figure~\ref{fig:span_tp}) of \code{span}.
%%
%Its presence in the above types constrains the state space of
%the programs being specified by defining which states are
%considered to be \emph{coherent}, and what changes (\ie,
%\emph{transitions}) can be made in the resource by the interfering
%threads.
%
%A concurroid definition is parametrized by a \emph{label} (a natural
%number) \code{sp}, making it possible to use several of its instances
%with different labels in a single program (\eg, running two
%\code{span} procedures in parallel on disjoint heaps).
%%
%The state space is defined via the \emph{coherence} predicate
%\code{coh}:
%
\begin{lstlisting}
Variable sp : nat.
Definition coh s := exists g : graph (joint s), 
  s = sp ->> [self s, joint s, other s] /\ 
  valid (self s \+ other s) /\ 
  forall x, x \in dom (self s \+ other s) = mark g x.  
\end{lstlisting}
%
The coherence predicate codifies that the state \code{s} is a triple,
\code{[self s,} \code{joint s,} \code{other s]}, and that it is
labelled by \code{sp}. The proof \code{g} is a witness that the
\code{joint} component is a graph-shaped heap.
%
The conjunct \code{valid (self s \\+ other s)} says that the
\emph{self} and \emph{other} components of the auxiliary state are
disjoint; their union is a finite map which is \code{valid}, \ie,
doesn't contain duplicate keys.
%
Finally, the most important invariant is that a node \code{x} is
contained in either \emph{self} or \emph{other} subjective view \Iff
it's marked in the joint graph.
%
%In all of the previously shown specifications and action definitions,
%we implicitly assumed the subject state to be coherent in the sence of
%\code{coh}.

The metatheory of FCSL~\cite[\S4]{Nanevski-al:ESOP14} requires the
coherence predicates to satisfy several properties that we omit
here, but prove in our implementation. The most important property is
the \emph{fork-join closure}, stating that the state space is closed
under realignment of \emph{self} and \emph{other} components. In other
words, one may subtract a value from \emph{self} and add it to
\emph{other} (and vice versa), without changing the coherence of the
underlying state.

%As a requirement imposed by FCSL's metatheory
%soundness~\cite[\S4]{Nanevski-al:ESOP14}, a concurroid's coherence
%predicate should satisfy a number of properties: the validity of
%\emph{self} $\pcmF$ \emph{other} and of the captured heap, as well as
%\emph{fork-join closure}, stating that the state-space is closed under
%realignment of \emph{self} and \emph{other} (\ie, subtracting a part
%of a state from one and adding is to another, and vice versa).

%
\code{SpanTree sp} contains two non-idle transitions. The first
transition \code{marknode_trans}, parametrized by the node \code{x},
describes how an unmarked \code{x} is physically marked in the joint
graph, and simultaneously added to the \emph{self} component.  The
second transition \code{nullify_trans} is parametrized by node
\code{x} and the direction \code{c}, indicating the successor of
\code{x} (left or right) that must be cut off from the graph. We omit
the definitions of the functions \code{mark_node} and \code{null_edge}
that describe the physical changes performed by the two transitions to
the underlying shared graph. These can be found in the accompanying
Coq code~\cite{Sergey-al:PLDI15TR}.
%
%
\begin{lstlisting}
Definition marknode_trans x s s' := 
 exists g : graph (joint s), ~~(mark g x) /\ 
   joint s' = mark_node g x /\ self s' = #x \+ self s /\ 
   other s' = other s /\ coh s /\ coh s'.  

Definition nullify_trans x (c : side) s s' := 
 exists g : graph (joint s), x \in dom (self s) /\ 
   joint s' = null_edge g c x /\ self s' = self s /\ 
   other s' = other s, coh s /\ coh s'.  
\end{lstlisting} 
%
The FCSL metatheory requires that transitions also satisfy several
properties. For example, \code{marknode_trans} and
\code{nullify_trans} preserve the \emph{other}-component and the
coherence predicate, as immediately apparent from their
definitions. They also preserve the footprint of the underlying state,
\ie, they don't add or remove any pointers. Adding and removing heap
parts can be accomplished by \emph{communication} between concurroids,
which we will briefly discuss in Section~\ref{sec:more-examples} of
this paper.

%The other properties to be proved for each of these transitions are:
%\emph{footprint preservation} (\ie, they don't alter the domain of the
%concurroid's heap) and \emph{locality}, which ensures that the
%transition can be ``framed''.\footnote{In FCSL, terminology both
%  \ccode{marknode_trans} nad \ccode{nullify-trans} are \emph{internal}
%  transitions, as they don't change the ownership over the heap. A
%  different class of \emph{external} transitions (with different
%  properties) serves for composing concurroids and is featured in
%  other exmaples.}

The coherence predicate, the transitions, and the proofs of their
properties are packaged into a \emph{dependent record}\footnote{A
  type-theoretic variant of a C \ccode{struct}, where fields can
  contain proofs.} \code{SpanTree sp}, which encapsulates all that's
important about a concurroid. Thus, we use the power of dependent
types in an essential way to build mathematical abstractions, such as
concurroids, that are critical for reusing proofs.
%

% The coherence predicate and the transitions, together with the proofs
% of their properties, are used to instantiate the concurroid
% ``interface''.
% %
% % \footnote{Proofs are first-class values in Coq, so one can pass them
% %   as arguments.}
% %
% Such interface (omitted here for brevity) is implemented as Coq's
% \emph{dependent record}, so it ``packages'' the STS definition
% together with the proofs of its
% properties. Since the provided proofs
% become the values of the record's fields, they can be later extracted
% and employed in the reasoning about specific concurrent programs,
% operating with the defined concurroid.\footnote{\ccode{[Pred...]}-like
%   combinators construct predicates for generally undecidable
%   properties, \ie, those whose proofs should be explicitly
%   constructed.}
% %
% \begin{lstlisting}
% Canonical SpanCore : mod_core := 
%   Mod.Core [PredU [Pred r | exists x, r = marknode_trans x] & 
%                     [Pred r | exists x c, r = nullify_trans x c]].  
% Canonical SpanTree : mod := Mod.Make SpanCore Pred0.   
% \end{lstlisting}


\subsection{Defining Atomic Actions}
\label{sec:atom-acts}

%
% \ab{This paragraph starts out well but becomes handwavy, due to lack
% of space.  Perhaps we should first write the paragraph on
% \Code{SpanTree} concurroid and then put Atomic action.}
%
We next illustrate the mechanism for defining atomic actions in FCSL.
The role of atomic actions is to perform a single physical memory
operation on the real heap, simultaneously with an arbitrary
modification of the auxiliary part of the state.
%Many related logics
%\ab{Citations} \an{Heh, gotta save space; PLDI format wastes a lot of space on citations.}
%
%specify these two different aspects of the atomic
%action as two separate operations. 
In FCSL, we treat the real and auxiliary state uniformly as they both
satisfy the same PCM laws. We specify their effects in one common
step, but afterwards prove a number of properties that separate
them. For instance, for each atomic action we always prove the
\emph{erasure property} that says that the effect of the action on the
auxiliary state doesn't affect the real state.

Specifically, the effect of the \code{trymark} action is defined by
the following relation between the input pointer \code{x}, the
pre-state \code{s1}, post-state \code{s2} and the return result
\code{r} of type \code{bool}.
%
\begin{lstlisting}
Definition trymark_step (x : ptr) s1 s2 (r : bool) := 
 exists g : graph (joint s1), 
   x \in dom (joint s1) /\ other s2 = other s1 /\ 
   if mark g x 
   then r = false /\ joint s2 = joint s1 /\ 
         self s2 = self s1
   else r = true /\ joint s2 = mark_node g x /\ 
         self s2 = #x \+ self s1.
\end{lstlisting}
The relation requires that \code{x} is a node in the pre-state graph
(\code{x \\in dom (joint s1)}). If \code{x} is unmarked in this graph,
then the action returns \code{true}, together with marking the node
physically in the real state (employing the function \code{mark_node}
already used in \code{marknode_trans}). Otherwise, the state remains
unchanged, and the action's result is \code{false}.
%
Notice that when restricted to the real heap, \ie, if we ignore the
auxiliary state in \code{self s1} and \code{other s1}, the relation
essentially describes the effect of the \code{CAS} command on the mark
bit of \code{x}. Thus, \code{trymark} {\em erases} to \code{CAS}.

There are several other components that go into the definition of an
atomic action. In particular, one has to prove that transitions are
\emph{total}, \emph{local}, and \emph{frameable} in the sense of
Separation Logic, and then ascribe to each action a stable
specification. However, the most important aspect of action
definitions is to identify their behavior with some transition in the
underlying concurroid. For example, \code{trymark} behaves like
\code{marknode_trans} transition of \code{SpanTree} if it succeeds,
and like \code{idle} if it fails.  Actions may also change state of a
number of concurroids simultaneously, as we will discuss in
Section~\ref{sec:more-examples}.  We omit the formal definition of all
these properties, but they can be found in the
accompanying~code~\cite{Sergey-al:PLDI15TR}.

\subsection{Scoped Concurroid Allocation and Hiding}

The \code{span_tp} type from Figure~\ref{fig:span_tp} operates under
\emph{open-world assumption} that \code{span} runs in an environment
of interfering threads, which, however, respect the transitions of the
\code{SpanTree} concurroid. If one wants to protect \code{span} from
interference, and move to \emph{closed-world assumption}, the top-most
call must be enclosed within $\mathsf{hide}$. We next show how to
formally do so.

The $\mathsf{hide}$ construct allocates a new lexically-scoped
concurroid from a local state of a particular thread.  The
thread-local state is modelled in FCSL by a basic concurroid
\ccode{Priv pv} with a label
\ccode{pv}~\cite[\S4]{Nanevski-al:ESOP14}, and its
\emph{self}/\emph{other} components are retrieved via \ccode{pv_self}
and \ccode{pv_other} projections.
%
%
The description of how much local heap should be ``donated'' to the
concurroid creation is provided by the user-supplied predicate $\Phi$,
called \emph{decoration} predicate. In addition to the heap, the
predicate scopes over the auxiliary \emph{self} value, while the
auxiliary \emph{other} is fixed to the PCM unit, to signal that
there's no interference from outside threads. In the case of
\code{span}, the decoration predicate is the following one.
%
\begin{lstlisting}
Definition graph_dec sp (g : heap * ptr_set) s := coh s /\
  exists (pf : graph g.1), s = sp ->> [g.2, g.1, Unit]. 
\end{lstlisting}
%
%
We can now write out a new type \code{span_root_tp}, to specify the
top-most call to \code{span}, under the closed-world assumption that
there's no interference. Parametrizing \wrt the locally-scoped
variable \code{h1 : heap} that snapshots the initial heap, the type is
as follows.
%
\begin{lstlisting}
Definition span_root_tp (x : ptr) :=
 {g1 : graph h1}, STsep [Priv pv] 
  (* precondition predicate *)
  (fun s1 => (forall y, ~~(mark g1 y)) /\
     pv_self s1 = h1 /\ x \in dom h1 /\ connected g1 x, 
  (* postcondition predicate *)
   fun _ s2 => exists (g2 : graph (pv_self s2)) t, 
     (forall x, (edgl g2 x \in [:: null; edgl g1 x]) /\
                 (edgr g2 x \in [:: null; edgr g1 x])) /\
     tree g2 x t /\ dom t =i dom h1).
\end{lstlisting}
The precondition says that the argument \code{x} is the root of the
graph \code{g1} stored in \code{h1}, and all the nodes of \code{g1}
are reachable from \code{x}. 
%
The postcondition says that the final heap's topology is a tree
\code{t}, whose edges are a subset of the edges of \code{g1}, but
whose nodes include \emph{all} the nodes of \code{g1}. Thus, the tree
is a spanning one.
%
The program satisfying this spec is a call to \code{span}, wrapped
into $\mathsf{hide}$, annotated with the decorating functions. We also
supply \code{h1} as the initial heap, and \code{Unit} of the PCM of
finite sets (hence, the empty set), as the initial value for
\emph{self}, which indicates that \code{span} is invoked with the
empty set of marked nodes.
% 
\begin{lstlisting}
Program Definition span_root x : span_root_tp x := 
 Do (priv_hide pv (graph_dec sp) (h1, Unit) [span sp x]).  
\end{lstlisting}
%
Coq will emit a proof obligation that the pre and post of
\code{span_tp} can be weakened into those of \code{span_root_tp} under
the closed-world assumption that \code{other s2 = Unit}. This proof is
in the development, accompanying this paper~\cite{Sergey-al:PLDI15TR}.

\section{More Examples}
\label{sec:more-examples}

We next briefly illustrate two additional features of FCSL that our
implementation uses extensively: concurroid composition and reasoning
about higher-order concurrent structures with helping.

\subsection{Composing Concurrent Resources}
\label{sec:comp-conc-reso}

The \code{span} algorithm uses only one concurroid \code{SpanTree},
allocated by $\mathsf{hide}$ out of the concurroid \code{Priv} for
thread-local state.
%
In general, FCSL specs can span multiple primitive concurroids, of the
same or different kinds, which are \emph{entangled} by interconnecting
special \emph{channel}-like transitions~\cite{Nanevski-al:ESOP14}. The interconnection
implements synchronized communication, by which concurroids 
exchange heap ownership.
%
Entangling several concurroids yields a new concurroid.
%
Omitting the formal details of the entanglement
operators, let us demonstrate a program
whose spec uses a composite concurroid.
%
\begin{lstlisting}
Definition alloc : 
 {h : heap}, STsep [entangle (Priv pv) ALock] 
 (fun s1 => pv_self s1 = h, 
  fun r s2 => exists B (v : B), pv_self s2 = r :-> v \+ h)
:=  ffix (fun (loop : unit -> alloc_tp) (_ : unit) =>
      Do (res <-- try_alloc; 
          if res is Some r then ret r else loop tt)) tt.  
\end{lstlisting}
%
The \code{alloc} procedure implements a pointer allocator. Its
postcondition says that the initial heap \code{h} is augmented by a
new pointer \code{r} storing some value \code{v} (\code{r :-> v}). The
heap \code{h} is part of the \code{Priv} concurroid, as evident by the
projection \code{pv_self} in the precondition. The pointer \code{r} is
logically transferred from the concurroid \code{ALock} which
implements a coarse-grained (\ie, lock-protected) concurrent
allocator. Hence, the whole procedure \code{alloc} uses the composed
concurroid \code{[entangle (Priv pv) ALock]}.
%
The body of \code{alloc} implements a simple spin-loop, trying to
acquire the pointer by invoking the \code{try_alloc} procedure, omitted
here.

Whereas separation logic~\cite{Reynolds:LICS02} always assumes
allocation as a primitive operation, the above example illustrates
that in FCSL, allocation is definable. One can also define a new
variant of the \code{STsep} type that automatically entangles the
underlying concurroid with \code{ALock}, thus enabling allocation
without the user having to explicitly do so herself.

\subsection{Higher-Order Specifications}
\label{sec:high-order-spec}

Due to embedding in Coq, FCSL is also capable of specifying and
verifying higher-order concurrent data structures, which we illustrate
by an example of a universal non-blocking construction of \emph{flat
  combining} by Hendler~\etal~\cite{Hendler-al:SPAA10}.\footnote{For
  simplicity, we present here a specification that is much weaker than
  what we have actually verified in our implementation.}

A flat combiner (FC) is a higher-order structure, whose method
\code{flat_combine} takes a sequential state-modifying
function~\code{f} and its argument~\code{v}, and works as
follows. While for the client, invoking \code{flat_combine(f, v)} looks
like a sequence \texttt{\small{lock; f(v); unlock}}, in reality, the
structure implements a sophisticated concurrent
behavior. Instead of expensive locking and unlocking, the calling
thread doesn't run \code{f}, but only \emph{registers} \code{f} to be
executed on \code{v}. One of the threads then becomes a
\emph{combiner} and executes the registered methods on behalf of
everyone else. Since only the combiner needs exclusive access to the
data structure, this reduces contention and improves cache
locality. This design pattern is known as \emph{helping} or
\emph{work stealing}: a thread can complete its task even without accessing
the shared resource.

To specify FC, we parametrize it by a sequential data structure and a
\emph{validity predicate} \code{fc_R}, which relates a function
\code{f} (from a fixed set of allowed operations), the argument of
type \code{fc_inT f}, result of type \code{fc_outT f} and the
\emph{contribution} of type \code{fc_pcm}. The last entry is a
description of what \code{f} \emph{does} to the shared state,
expressed in abstract algebraic terms as an element from a
user-supplied PCM~\code{fc_pcm}. 
%
\begin{lstlisting}
Variable fc_R : forall f, 
          fc_inT f -> fc_outT f -> fc_pcm -> Prop.
\end{lstlisting}
%
The spec of the \code{flat_combine} is then given in the context of
three entangled concurroids: \code{Priv} for thread-local state, a
lock-based allocator \code{Alloc}, adapted from the previous example
(since a sequential function \code{f} might allocate new memory), and
a separate concurroid \code{FlatCombine}.
%
\begin{lstlisting}
Definition PA := (entangle (Priv pv) Alloc).
Program Definition flat_combine f (v : fc_inT f) : 
 STsep [entangle PA (FlatCombine fc)] 
 (fun s1 => pv_self s1 = Unit /\ fc_self s1 = Unit,
  fun (w : fc_outT f) s2 => exists g, pv_self s2 = Unit /\ 
       fc_self s2 = g /\ fc_R f v w g) := ...
\end{lstlisting}
% 
The precondition says that \code{flat_combine} executes in the empty
initial heap (\code{pv_self s1 = Unit}), and hence by framing, in any
initial heap. Similarly, the initially assumed effects of the calling
thread on the shared data structure are empty (\code{fc_self s1 =
  Unit}), but can be made arbitrary by applying FCSL's \emph{frame
  rule} to the spec of \code{flat_combine}.
%
The postcondition says that there exists an abstract PCM value
\code{g} describing the effect of \code{f} in terms of PCM elements
(\code{fc_R f v w g}). Moreover, the effect of \code{g} is attributed
to the invoking thread (\code{fc_self s2 = g}), even though in reality
\code{f} could be executed by the combiner, \emph{on behalf} of the
calling thread.
%
In our Coq implementation, we instantiated the FC structure with a sequential
stack, showing that the result has the same spec as a concurrent stack
implementation.


\section{Elements of FCSL infrastructure}
\label{sec:underhood}

In this section we sketch two important parts of FCSL machinery, used
to simplify construction of proofs.

\subsection{Extracting concurroid structure via getters}
\label{sec:getters}

When working with compositions of multiple concurroids, as in examples
listed in Section~\ref{sec:more-examples}, one frequently has to
select the \emph{self}, \emph{joint} or \emph{other} components that
belong to one of the composed concurroids. A na\"{i}ve way of doing this
is to describe the state space of the composition concurroid using 
existentials that abstract over the concurroid-specific
fields. \emph{E.g.}, in the case of flat combiner, which composes
three concurroids \code{Priv}, \code{Alloc} and \code{FlatCombine}, we
could use three existentials to abstract over
\emph{self}/\emph{joint}/\emph{other} fields for \code{Priv}, another
three for \code{Alloc}, and three more for \code{FlatCombine}. To access
any of the fields, we have to destruct all
nine of the existentials. This quickly becomes tedious and results in
proofs that are obscured by such existential destruction.

Our alternative approach develops a systematic way of projecting the
fields associated with each concurroid, based on the concurroid's
label. Thus, for example, we can write \code{self\ pv\ s} to obtain
the \emph{self} component of \code{s}, associated with a concurroid
whose label is \code{pv}. The identifier \code{pv_self} we used in the
spec for \code{span_root} and for \code{flat_combine} is a notational
abbreviation for exactly this projection. While this is a simple and
obvious idea, its execution required a somewhat involved use of
dependently-typed programming, and an intricate automation by
canonical structures and lemma
overloading~\cite{Gonthier-al:ICFP11,Mahboubi-Tassi:ITP13}.

%As it was shown in Section~\ref{sec:more-examples}, most of the
%programs ``span'' a set of resources, described by a composite
%concurroid. To make proofs in such cases as simple as proofs about
%primitive (one-label) concurroids, we used the power of
%dependently-typed programming and implemented a library of \emph{state
%  getters}.
%
%Given the label of an arbitrary primitive concurroid, the type of its
%self/other and the joint component (which is typically \code{heap}),
%and the default value for the joint component, the function
%\code{Getter} will provide a dependent record with the three getters
%for this concurroid's state components.
%
%The following code shows an instantiation of the \code{Getter}
%interface for the graph concurroid, with a label \code{sp}, auxiliary
%state's type \code{ptr_set} and joint component to be a heap.
%
%\begin{lstlisting}
%Notation ptr_set := (ptrmap_pcm_Encoded [encoded_set of unit]).  
%Definition spg := Getter sp ptr_set heap_tp_Encoded Unit.
%
%Notation self s := (self spg s).
%Notation other s := (other spg s). 
%Notation joint s := (joint spg s).
%\end{lstlisting}

\subsection{Structural lemmas}
\label{sec:lemmas}

The proofs in FCSL are structured to facilitate systematic application
of Floyd-style structural rules, one for each program command. All the
rules are proved sound from first principles, and are applied as
lemmas to advance the verification.
%
As the first step of every proof, the system implicitly applies the
weakening rule to the automatically synthesized weakest pre- and
strongest postconditions~\cite{Dijkstra:CACM75}, essentially
converting the program into the continuation-passing style (CPS)
representation and sequentializing its structure.
%
Every statement-specific structural rule ``symbolically evaluates''
the program by one step, and replaces the goal with a new one to be
verified.

For example, the following lemma \code{step}, corresponding to the
rule of sequential composition, reduces the verification of a program
$(y \asgn e_1; (e_2~y))$ with continuation~$k$, to the verification
of the program $e_1$ and the program $e_2~y~k$, where $y$ corresponds
to a symbolic result of evaluating $e_1$, constrained according to
$e_1$'s postcondition. One can apply it several times until $e_1$ is
reduced to some primitive action, at which point one can apply the
structural rule for that action.
%
\begin{lstlisting}
Lemma step W A B (e1 : ST W A) (e2 : A -> ST W B) i (k : cont B): 
     verify i e1 (fun y m => verify m (e2 y) k) -> 
     verify i (y <-- e1; e2 y) k.  
\end{lstlisting}
%
\code{ST} is a type synonym for \code{STsep}, hiding its pre- and
postconditions.





\section{Mechanization and Evaluation}
\label{sec:evaluation}

% \ab{This is an important section. It needs to convey what was
%   difficult, what was easy in the implementation. How were the
%   difficulties mitigated via the techniques developed in the previous
%   sections? What interesting proof engineering that needed to be done?
%   Ideally, these discussions should provide context/understanding of
%   the numbers in the Table. What are the current limitations of the
%   implementation? Make a separate paragraph for limitations perhaps
%   starting ``At the moment'' below.}
%
% \is{Mostly addressed in revised writing.}

We have mechanized the specs and the proofs of all the examples from
this paper~\cite{fcsl-site}, taking advantage of the fact that FCSL
has been recently implemented as a tool for concurrency
verification~\cite{Sergey-al:PLDI15} on top of the Coq proof
assistant~\cite{Coq-manual}.

Table~\ref{tab:locs} summarizes the statistics with respect to our
mechanization in terms of lines of code and compilation times. 
%
The examples were proof-checked on a 3.1~GHz Intel Core~i7 OS~X
machine with 16 Gb RAM, using Coq~8.5pl2 and Ssreflect
1.6~\cite{Gonthier-al:TR}.
%
As the table indicates, a large fraction of the implementation is
dedicated to proofs of preservation of resource
invariants~(\textsf{Inv}), \ie, checking that the actual
implementations do not ``go wrong''.
%
In our experience, these parts of the development are the most tricky,
as they require library-specific insights to define and reason about
auxiliary histories.
%
Since FCSL is a general-purpose verification framework, which does not
target any specific class of programs or properties, we had to prove
problem-specific facts, \eg, lemmas about histories of a particular
kind (\textsf{Facts}), and to establish the specs of interest stable
(\textsf{Stab}). Once this infrastructure has been developed, the
proofs of main procedures turned out to be relatively small
(\textsf{Main}).
%
% \gad{I don't see the point of having the {\bf Build} column in
%   Table~\ref{tab:locs}: First of all, I don't think that time to
%   compile a proof in Coq is relevant at all. From an engineering
%   perspective, it might perhaps make more sense to estimate the amount
%   of man-hours that took to discharge them. Second, because we then
%   would have to explain why the exchanger takes half the time than the
%   counter, given that they are similar in size. And I don't think we
%   want to get into describing the design, the differences in ``style''
%   between the two, and discussing why that might be affecting the
%   buildtime in detail here.}
%
% \is{Sorry, but since it's PLDI, we have to give build times, even
%   though we're out of space to explain some specific performance
%   phenomena}

{
%\setlength{\belowcaptionskip}{-1pt} 
\begin{table}
{%\footnotesize
\sffamily\small % tabular data either 10pt times, or 9pt helvetica
\centering
\begin{tabular}{|@{\ }l@{\ }||@{\ }c@{\ }|@{\ }c@{\ }|@{\ }c@{\ }|@{\ }c@{\ }|@{\ }c@{\ }||@{\ }r@{\ }|}
  \hline
  \textbf{Program} &  
                     {Facts} & {Inv} &
                                       {Stab} & {Main} & \textbf{Total}
  & \textbf{Build~~~}    
  \\ \hline \hline 
  Exchanger \hfill (\S \ref{sec:exchanger}) & 365 & 1085 & 446 & 162 & 2058 & 4m~~46s
  \\
  Exch. Client \hfill (\S \ref{sec:cal}) & 258 & -- &--& 182 & 440 & 57s
  \\
  Count. Netw. \hfill (\S \ref{sec:counting}) & 379 & 785 & 688 & 27 & 1879 & 12m~23s
  \\
  CN Client 1 \hfill (\S \ref{sec:qc-client}) & 141 &--&--& 180  & 321 & 3m~11s
  \\
  CN Client 2 \hfill (\S \ref{sec:qqc-client})& 115 &--&--& 259 & 374 & 3m~~~9s 
  \\[2pt] \hline
\end{tabular}
}
\caption{
  Mechanization of the examples: lines of code for program-specific facts \intab{Facts},
  resource invariants and transitions \intab{Inv}, 
  stability proofs for desired specs \intab{Stab}, spec and proof sizes for main
  functions \intab{Main}, total LOC count \intab{\textbf{Total}}, and build
  times \intab{\textbf{Build}}. The ``--'' entries indicate the
  components that were not needed for the example.
} 
\label{tab:locs}
%\vspace{-15pt}
\end{table}}

Fortunately, trickiness in libraries is invisible to clients, as FCSL
proofs are compositional. Indeed, because specs are encoded as Coq
types~\cite{Sergey-al:PLDI15}, the substitution principle
automatically applies to programs \emph{and proofs}.
%
%Thus, the trickiness of library proofs is not visible to the clients.
%
At the moment, our goal was not to optimize the proof sizes, but to
demonstrate that FCSL as a tool is suitable \emph{off-the-shelf} for
machine-checked verification of properties in the spirit of novel
correctness
conditions~\cite{Hemed-al:DISC15,Aspnes-al:JACM94,Jagadeesan-Riely:ICALP14}.
Therefore, we didn't invest into building advanced
tactics~\cite{McCreight:TPHOL09} for specific classes of
programs~\cite{Zee-al:PLDI08} or
properties~\cite{Dragoi-al:CAV13,Vafeiadis:CAV10,Bouajjani-al:POPL15}.
%
% and we leave developing such automation for future work.

% Our verification is compositional, because the proofs of the clients'
% specs are derived only from the specifications of concurrent objects,
% without relying on their implementation.


\section{Related work}
\label{sc:related}

% Early times

The proof method for establishing linearizability of concurrent
objects based on the notion of \emph{linearization points} has been
presented in the original paper by Herlihy and
Wing~\cite{Herlihy-Wing:TOPLAS90}. The first Hoare-style logic,
employing this method for compositional proofs of linearizability was
introduced in Vafeiadis' PhD thesis~\cite{Vafeiadis:PhD}. However,
that logic, while being inspired by the
combination~\cite{Vafeiadis-Parkinson:CONCUR07} of Rely-Guarantee
reasoning~\cite{Jones:TOPLAS83} and Concurrent Separation
logic~\cite{OHearn:TCS07} with syntactic treatment of linearization
points~\cite{Vafeiadis-al:PPoPP06}, did not have a soundness proof
with respect to any program semantics. Furthermore, the
work~\cite{Vafeiadis:PhD} did not connect reasoning about
linearizability to the verification of client programs, which make use
of linearizable objects in a concurrent environment (\cf
Section~\ref{sec:clients}).

% Modern logics for linearizability

Both these shortcomings were addressed in more recent works on program
logics for establishing linearizability~\cite{Liang-Feng:PLDI13}, or,
equivalently~\cite{Filipovic-al:TCS10}, \emph{observational
  refinement}~\cite{Turon-al:ICFP13}, which provided semantically
sound methodologies for (a) verifying linearizability/refinement of
concurrent objects \emph{as well as} for (b) giving the objects
Hoare-style specifications, useful for the clients.
%
However, in the both
approaches~\cite{Liang-Feng:PLDI13,Turon-al:ICFP13} establishing (a)
and (b) essentially requires one to prove \emph{two different} facts
about a program, and, if one is interested only in the Hoare-style
reasoning by means of composing program specifications, verifying
linearizability (a) is a detour, which might be avoided.

% No-linearizability

This observation has been recognized in a series of more recent works
on program logics for concurrency that all focused on establishing
Hoare-style specifications for concurrent objects (b) without
resorting to
linearizability~\cite{Sergey-al:ESOP15,Svendsen-Birkedal:ESOP14,ArrozPincho-al:ECOOP14,Jung-al:POPL15}.
%
In this paper, we are following the same way of thinking, building on
the ideas from the prior work~\cite{Sergey-al:ESOP15}, which explored
some patterns of assigning \emph{subjective} Hoare-style concurrent
specifications with auxiliary histories to concurrent objects
(including \emph{higher-order} ones, such as {flat
  combiner}~\cite{Hendler-al:SPAA10}) in
FCSL~\cite{Nanevski-al:ESOP14}. The work~\cite{Sergey-al:ESOP15} has
generalized earlier results on history-based Hoare-style
logics~\cite{Fu-al:CONCUR10, Gotsman-al:ESOP13,Bell-al:SAS10}, yet it
has not provided a way to reason about concurrent objects, featuring
future-dependent linearization points.
%

The key novelty of this work with respect to previous results
involving Hoare-style reasoning about histories~\cite{Fu-al:CONCUR10,
  Gotsman-al:ESOP13,Bell-al:SAS10,Sergey-al:ESOP15,Hemed-al:DISC15} is
the idea of dynamically \emph{re-linking} the auxiliary histories,
enabling efficient constructive reasoning about non-local and
future-dependent linearization points.
%
Since re-linking as we presented it is just manipulation with
otherwise standard auxiliary state, we did not have to extend the
metatheory of FCSL, and were able to use it
\emph{off-the-shelf}. Furthermore, relying on the auxiliary state
makes it possible to extend our verification method for reasoning
about higher-order (\ie, parametrized by another data structure)
snapshot-based concurrent constructions~\cite{Petrank-Timnat:DISC13},
which is our immediate future work.
%
In contrast, alternative modern programming
logics~\cite{ArrozPincho-al:ECOOP14,Jung-al:POPL15,Svendsen-Birkedal:ESOP14}
would require introduction of prophecy variables in order to verify
Jayanti's snapshot construction, and, to the best of our knowledge,
none of these extensions has been implemented yet.

Related to our result, O'Hearn \etal have demonstrated how to employ
history-based reasoning and Hoare-style logic for proving
\emph{non-constructively} existence of linearization points for
concurrent objects out of the data structure
invariants~\cite{OHearn-al:PODC10}---the result is known as \emph{the
  Hindsight Lemma}. The reasoning principle presented in this paper
generalizes that idea, since the Hindsight Lemma is only applicable to
``pure'' concurrent methods (\eg, concurrent set's
\texttt{contains}~\cite{Heller-al:OPODIS05}), which do not determine
position of other threads' linearization points. In contrast, our idea
of re-linking histories also handles the structures, where a
linearization point of a method call (e.g., \texttt{write}) might
depend on the (future) outcome of another operation (e.g.,
\texttt{scan}), as was showcased by Jayanti's construction.



% Most recent related work relies on parametrization to avoid reasoning
% about linearizability. But, that has its drawbacks. In particular,
% while it can handle situations in which linearization points are
% placed in different places, depending on the run-time infomration
% (speculiation), it is not currently strong enough to formalize
% examples where linearizatin points appear in different
% proceedures.\an{Hmm, are we super sure of this?} Thus, we don't
% believe they can handle Jayanti's algorithm.  \is{How about this time
%   we just mention these people briefly and instead do a comparison to
%   the PODC crowd and their reasoning methods, which are all about
%   harnessing the vanialla definition of liearizability. This way, the
%   whole discussion will be more relevant to the audience, as nobody
%   knows the concurrency logics anyway, so we'll just waste valuable
%   space, talking about them in detail.}  \an{I agree that we should
%   just mention them briefly.}

% Independently of us, Kyzha et al. have developed an a method whereby
% linearizability is proved by reordering time-stamped histories,
% similar to the basis of our approach. However, there are many
% differences. 

% \begin{enumerate}
% \item While linearizability does not say anything about clioent side
%   proofs, beyond the ability to replace the two programs in it, our
%   method also gives a way to reason about clients, as we illustrated
%   in Section 4.

% \item While they present a new logic, for us, it is all a mode of use
%   of auxiliary state.

% \item Our PCM of histories let us reuse separation logic in
%   infrastructure (e.g., frame rule) to reason about histories locally,
%   whereas Kyzha et al. use global histories only. Our setting also
%   immediatley lends itself to higher-order programming. This is
%   particularly important for snaphsots, as one of their major
%   application is in iterators -- a prototypical higher-order
%   program~\cite{PetrankT+disc13}. However, we don't explore iterators
%   in this particular paper.

% \item Kyzha et al. track the ordering of timestamps quite differnetly
%   from us. Where we keep an ``existential'' witness for the total
%   ordering of timestamps, at all stages of evaluation, they do so
%   ``universially''. Thus, they require proving that all possible
%   completions of a partial order into a total order are valid for
%   establishing the relation with the linearization program. We believe
%   this leads to larger proofs and more complicated proofs than
%   necessary.
% \end{enumerate}

% Liang et al~\cite{LiangF+pldi13} present a dedicated meta-theory to
% reason with future-dependent linearization points based on
% speculations. \gad{Are we sure they can't do Jayanti here? Need to
%   think what to say about their works as they do have a program logic
%   as we do}
  

%\vspace{-4pt}

\section{Conclusion and Future Work}
\label{sec:conclusion}

%\vspace{-2pt}

We have presented a number of formalization techniques, enabling
specification and verification of highly scalable non-linearizable
concurrent objects and their clients in Hoare-style program logics.
%
In particular, we have explored several reasoning patterns, all
involving the idea of formulating execution histories as auxiliary
state, capturing the expected concurrent object behavior.
%
We have discovered that quantitative logic-based reasoning about
concurrent behaviors can be done by storing relevant information about
interference directly into the entries of a logical history.

We believe that our results help to bring the Hoare-style reasoning
into the area of non-linearizable concurrent objects and open a number
of exciting opportunities for the field of mechanized logic-based
concurrency verification.

For instance, in this paper we have deliberately chosen to focus on
simple client programs to showcase the specs we gave to concurrent
libraries. However, any larger program incorporating these examples
can be verified compositionally in FCSL, out of \emph{these clients'
  specs}, via the substitution principles of
FCSL~\cite{Nanevski-al:ESOP14,Sergey-al:PLDI15}, without the need to
deal with concepts such as histories and tokens that are specific to
particular libraries. Given the bounds, which we formally proved in
Section~\ref{sec:qclients}, we believe that the reasoning patterns we
have described will be useful for mechanical verification of larger
weakly-synchronized approximate parallel
computations~\cite{Rinard:RACES}, exploiting the QC and QQC-like
behavior.

Furthermore, by ascribing interference-sensitive quantitative specs in
the spirit of~\eqref{eq:qc-spec} to relaxed concurrent
libraries~\cite{Henzinger-al:POPL13}, one can assess the applicability
of a library implementation for its clients: the clients should
tolerate the anomalies caused by interference, as long as they can
logically infer the desired safety assertions from a library spec,
which is fine-tuned for particular usage scenarios.


% Since logical approaches enable reasoning about higher-order
% concurrent data
% structures~\cite{Svendsen-al:ESOP13,Turon-al:ICFP13,Sergey-al:ESOP15},
% we envision the possibility of giving parametric logical specs to such
% generic relaxed constructions as diffracting/elimination
% trees~\cite{Shavit-Touitou:TCS97,Shavit:CACM11} that, once
% instantiated with suitably specified stacks or pools on the leaves,
% would yield a provably correct, highly scalable concurrent container
% implementation.


% Acknowledgements:

% Michael Emmi
% Pierre Ganty
% Andrea Cerone
% Anton Podkopaev

% \todo{Generalizing the construction of the counting network to
%   arbitrary diffracting trees}

% \todo{Elimination and diffracting trees~\cite{Shavit-Touitou:TCS97}.}
\paragraph{Acknowledgements}
We thank the anonymous reviewers from OOPSLA'16 PC and AEC for their feedback. We are also grateful to Sophia Drossopoulou for shepherding our paper. This research was partially supported by \ab{Please fill in} and the US National Science Foundation (NSF). Any opinion, findings, and conclusions or recommendations expressed in the material are those of the authors and do not necessarily reflect the views of NSF.

% \acks
% \todo{Acknowledgments, if needed.}

%\newpage

\bibliographystyle{abbrv}
\bibliography{bibmacros,references,proceedings}

\end{document}


