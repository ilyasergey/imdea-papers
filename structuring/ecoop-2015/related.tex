
\section{Related and future work}
\label{sec:related-work}

Using the Coq proof assistant as a uniform platform for implementation
of logic-based program verification tools is a well-established
approach, which by now has been successfully employed in a number of
projects on certified compilers~\cite{Leroy:POPL06,Appel-al:BOOK14}
and verified low-level
code~\cite{Shao:CACM10,Chlipala:PLDI11,Jensen-al:POPL13}, although,
with no specific focus on abstractions for fine-grained concurrency,
such as protocols and auxiliary state.
%
% Our FCSL development inherits a lot of ideas from works on
% verification of sequential imperative programs,
% e.g.,~\cite{Nanevski-al:ICFP08,Nanevski-al:POPL10}.


\subsection{Related program logics}
\label{sec:relat-logic-fram}

The FCSL logic has been designed as a generalization of the classical
Concurrent Separation Logic by~\citet{OHearn:TCS07}, combining the
ideas of local concurrent protocols with arbitrary
interference~\cite{Jones:TOPLAS83,Feng:POPL09} and compositional
auxiliary state~\cite{LeyWild-Nanevski:POPL13} with the possibility to
compose protocols.
%
% \ab{Do we need to say the rest of the sentence?} by means of
% entanglement via transitions, implementing communication channels.
%
Other concurrency logics, close to FCSL in their expressive power, are
iCAP~\cite{Svendsen-Birkedal:ESOP14}, CoLoSL~\cite{Raad-al:colosl},
and CaReSL~\cite{Turon-al:ICFP13}.
 
iCAP leverages the idea, originated by~\citet{Jacobs-Piessens:POPL11},
of parametrizing specs for fine-grained concurrent data types by
client-provided auxiliary code, which can be seen as a ``callback''. A
form of composition of concurrent resources can be encoded in iCAP
using fractional permissions~\cite{Bornat-al:POPL05} and
view-shifts~\cite{DinsdaleYoung-al:POPL13}.
%
Since iCAP doesn't have explicit subjective dichotomy of the auxiliary
state, encoding of thread-specific contributions in it is less direct
comparing to FCSL.

CoLoSL defines a different notion of thread-local views to a shared
resource, and uses \emph{overlapping
  conjunction}~\cite{Hobor-Villard:POPL13} to reconcile the
permissions and capabilities, residing in the shared state, between
different threads.
%
Overlapping conjunction affords a description of the shared structure
mirroring the recursive calls in the structure's methods. In FCSL,
such machinery isn't required, as \emph{self} and \emph{other} suffice
to represent the thread-specific views, and \emph{joint} state doesn't
need to be divided between threads.
%
In our opinion, this leads to simpler specs and proofs. For example,
CoLoSL's proof that \code{span} constructs \emph{a tree} involves
abstractions such as shared capabilities for marking nodes and
extension of the graph with virtual edges, none of which is required
in FCSL. Moreover, CoLoSL doesn't prove that the tree is spanning,
which we achieve in FCSL via hiding.

% \begin{comment}
% CoLoSL defines an alternative notion of a thread-specific view to a
% shared resource, defined \wrt the real heap (in contrast to FCSL's
% subjective views, which are parametrized over a generic PCM structure).  CoLoSL
% makes use of the \emph{overlapping
%   conjunction}~\cite{Hobor-Villard:POPL13} to reconcile ``local''
% views of several parallel threads---an idea, generalized by FCSL's
% notion of subjectivity.
% %
% CoLoSL was used to verify the concurrent spanning tree algorithm,
% proving that its result is \emph{a tree}. It was done via logical
% action capabilities and instrumenting the graph with virtual edges.
% %
% However, the CoLoSL paper doesn't prove the resulting tree to be a
% spanning one, which we achieve in FCSL via hiding.
% %
% % \ab{Should the above be: However, the CoLoSL paper doesn't establish
% % that the resulting tree is a \emph{spanning} tree, which we achieve
% % in FCSL using hiding.}
% \end{comment} 

CaReSL combines the Hoare-style reasoning and proofs about
\emph{contextual refinement}.  Similarly to FCSL, CaReSL employs
resource protocols, although, targeting the ``life stories'' of
particular memory locations instead of describing a whole concurrent
data structure by means of an STS.  While FCSL is not equipped with
abstractions for contextual refinement, in our experience it was never
required to prove the desired Hoare-style specs for fine-grained data
structures.

Reasoning in all the three alternative logics follows the tradition of
Hoare-style logics,
%
% \ab{Shouldn't it be Hoare logic?} \is{Okay, made it Hoare-style.}
%
so the specs never mention explicitly the heap and state
components. In contrast, FCSL assertions use explicit variables to
bind heap and auxiliary state, as well as their components. In our
experience, working directly with the state model is pleasant, and has
to be done in Coq anyway, since Coq lacks support for contexts of
bunched implications, as argued by~\citet{Nanevski-al:POPL10}.

None of iCAP, CoLoSL or CaReSL features a mechanized metatheory,
nor any of these logics has been implemented in a form of a mechanized
verification tool.

%\noindent\textbf{\emph{Related tools for concurrency
%verification}}

\subsection{Related tools for concurrency verification}
%
RGSep, one of the first logics for modular reasoning about
fine-grained concurrency~\cite{Vafeiadis-Parkinson:CONCUR07}, inspired
creation of semi- and fully-automated verification tools:
SmallfootRG~\cite{Calcagno-al:SAS07} and
Cave~\cite{Vafeiadis:VMCAI10}.
%
These tools target basic safety properties of first-order code,  such
as data integrity and absence of memory leaks.

Chalice~\cite{Leino-Muller:ESOP09} is an experimental first-order
concurrent language, supplied with a tool, which generates
verification conditions (VCs) for client-annotated Chalice
programs. Such VCs are suitable for automatic discharge by SMT
solvers.
%
% \ab{Check above rewording.}
%
% Chalice has been recently employed as a backend in a number of
% verification frameworks for concurrency, such as
% VerCors~\cite{Blom-Huisman:FM14}.
%
For local reasoning, Chalice employs fractional
permissions~\cite{Bornat-al:POPL05}, implicit dynamic
frames~\cite{Smans-al:ECOOP09}, and auxiliary
state~\cite{Leino-al:FOSAD09}, which, unlike the one of FCSL, is not a
subject of PCM laws, and thus is not compositional, as its shape
should match the forking pattern of the client program being verified.
%
Chalice also supports a form of \emph{symmetric} Rely/Guarantee
reasoning, \ie, not allowing the threads to take different roles in a
protocol (which is expressible in FCSL via \emph{self}-enabled
transitions).
%
Chalice's specification fragment is a first-order logic, whereas FCSL
admits higher-order functions and predicates in specs, therefore,
enabling program composition and proof reuse, as shown in
Figure~\ref{fig:deps}.

VCC~\cite{Cohen-al:TPHOL09} is a tool for verifying low-level
concurrent C code. VCC doesn't support reasoning about custom
interference or compositional auxiliary state, and, similarly to
Chalice, allows specifications only in a first-order logic to support
an SMT-based automation back-end.

VeriFast~\cite{Jacobs-al:NFM11} is a tool for deductive verification
of sequential and concurrent programs, based on separation
logic~\cite{Reynolds:LICS02}.
% 
% The specification language of VeriFast provides a possibility to
% implement custom inductive definitions and prove propositions about
% them, although, without a possibility to quantify over values in
% propositions, due to the lack of dependent types.
%
To specify and verify fine-grained concurrent algorithms, VeriFast
employs fractional permissions and a form of (non-compositional)
first-class auxiliary state~\cite{Jacobs-Piessens:POPL11}. VeriFast
has been recently extended with Rely/Guarantee
reasoning~\cite{Jacobs-al:sharedboxes}, although, without a
possibility to compose resources. To the best of our knowledge,
soundness of VeriFast core has been formally proved only
partially~\cite{Vogels:PhD}.

% Other shallow-embedding-based approaches~\cite{Brady-Hammond:FI10}

% Several projects has been conducted recently by means of \emph{deep}
% embedding (\ie, by implementing a standalone language) of the XCAP
% framework into
% Coq~\cite{Chlipala:PLDI11,Feng-al:PLDI08,Ni-Shao:POPL06}, targeting
% verification of different aspects of low-level code, without
% introducing specific abstractions for fine-grained concurrency.

% \citeyearpar{Gordon:PhD}

Rely-Guarantee references (\textsc{RGRef}s) by~\citet{Gordon:PhD} are
a mechanism to prove transition invariants of concurrent data
structures.
%
% Comparing to FCSL, the \textsc{RGRef}s approach follows an opposite
% path to concurrency verification, considering code as a primary entity
% and showing that it preserves certain transitions.
%
While the language of \textsc{RGRef}s is implemented as a Coq DSL, the
system's soundness is proved by hand using the Views
framework~\cite{DinsdaleYoung-al:POPL13}.
%
Since \textsc{RGRef}s focuses on correctness of data structures \wrt
specific protocols and doesn't provide auxiliary state, it's unclear
how to employ it for client-side reasoning.


\subsection{Future work}
\label{sec:future-work}

In the future, we plan to augment FCSL with the program extraction
mechanism~\cite{Letouzey:CIE08} and implement proof automation for
stability-related facts via lemma
overloading~\cite{Gonthier-al:ICFP11}.
%
Due to the limitations of Coq's model \wrt impredicativity, at this
moment, FCSL doesn't support higher-order heaps (\ie, the possibility
to reason about arbitrary storable effectful procedures). While programs
requiring this feature are rare, we hope that it will be possible to
encode and verify the characteristic ones, once the development is
migrated to Coq 8.5, featuring universe
polymorphism~\cite{Sozeau-Tabareau:ITP14}.

