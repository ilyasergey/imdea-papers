\section{More examples}
\label{sec:more-examples}

We next briefly illustrate two additional features of FCSL that our
implementation uses extensively: concurroid composition and reasoning
about higher-order concurrent structures with work stealing.

\paragraph{Composing concurrent resources.~~}
\label{sec:comp-conc-reso}

The \code{span} algorithm uses only one concurroid \code{SpanTree},
allocated by $\mathsf{hide}$ out of the concurroid \code{Priv} for
thread-local state.
%
In general, FCSL specs can span multiple primitive concurroids, of the
same or different kinds, which are \emph{entangled} by interconnecting
special \emph{channel}-like transitions~\cite{Nanevski-al:ESOP14}. The interconnection
implements synchronized communication, by which concurroids 
exchange heap ownership.
%
Entangling several concurroids yields a new concurroid.
%
Omitting the formal details of the entanglement
operators, let us demonstrate a program
whose spec uses a composite concurroid.
%
\begin{lstlisting}
Definition alloc : {h : heap}, STsep [entangle (Priv pv) ALock] 
   (fun s1 => pv_self s1 = h, 
    fun r s2 => exists B (v : B), pv_self s2 = r :-> v \+ h) := 
  ffix (fun (loop : unit -> alloc_tp) (_ : unit) =>
    Do (res <-- try_alloc; 
        if res is Some r then ret r else loop tt)) tt.  
\end{lstlisting}
%
The \code{alloc} procedure implements a pointer allocator. Its
postcondition says that the initial heap \code{h} is augmented by a
new pointer \code{r} storing some value \code{v} (\code{r :-> v}). The
heap \code{h} is part of the \code{Priv} concurroid, as evident by the
projection \code{pv_self} in the precondition. The pointer \code{r} is
logically transferred from the concurroid \code{ALock} which
implements a coarse-grained (\ie, lock-protected) concurrent
allocator. Hence, the whole procedure \code{alloc} uses the composed
concurroid \code{[entangle (Priv pv) ALock]}.
%
The body of \code{alloc} implements a simple spin-loop, trying to
acquire the pointer by invoking the \code{try_alloc} procedure, omitted
here.

Whereas separation logic~\cite{Reynolds:LICS02} always assumes
allocation as a primitive operation, the above example illustrates
that in FCSL, allocation is definable. One can also define a new
variant of the \code{STsep} type that automatically entangles the
underlying concurroid with \code{ALock}, thus enabling allocation
without the user having to explicitly do so herself.

\paragraph{Higher-order specifications.~~}
\label{sec:high-order-spec}

Due to embedding in Coq, FCSL is also capable of specifying and
verifying higher-order concurrent data structures, which we illustrate
by an example of a universal non-blocking construction of \emph{flat
  combining} by~\citet{Hendler-al:SPAA10}.\footnote{For simplicity, we
  present here a specification that is much weaker than what we have
  actually verified in our implementation.}

A flat combiner (FC) is a higher-order structure, whose method
\code{flat_combine} takes a sequential state-modifying
function~\code{f} and its argument~\code{v}, and works as
follows. While for the client, invoking \code{flat_combine(f, v)} looks
like a sequence \texttt{\small{lock; f(v); unlock}}, in reality, the
structure implements a sophisticated concurrent
behavior. Instead of expensive locking and unlocking, the calling
thread doesn't run \code{f}, but only \emph{registers} \code{f} to be
executed on \code{v}. One of the threads then becomes a
\emph{combiner} and executes the registered methods on behalf of
everyone else. Since only the combiner needs exclusive access to the
data structure, this reduces contention and improves cache
locality. This design pattern is known as \emph{work stealing} or
\emph{helping}: a thread can complete its task even without accessing
the shared resource.

To specify FC, we parametrize it by a sequential data structure and a
\emph{validity predicate} \code{fc_R}, which relates a function
\code{f} (from a fixed set of allowed operation), the argument of type
\code{fc_inT f}, result of type \code{fc_outT f} and the
\emph{contribution} of type \code{fc_pcm}. The last entry is a
description of what \code{f} \emph{does} to the shared state,
expressed in abstract algebraic terms as a value from a user-supplied
PCM.
%
\begin{lstlisting}
Variable fc_R : forall f, fc_inT f -> fc_outT f -> fc_pcm -> Prop.
\end{lstlisting}
%
The spec of the \code{flat_combine} is then given in the context of
three entangled concurroids: \code{Priv} for thread-local state, a
lock-based allocator \code{Alloc}, adapted from the previous example
(since a sequential function \code{f} might allocate new memory), and
a separate concurroid \code{FlatCombine}.
%
\begin{lstlisting}
Definition PA := (entangle (Priv pv) Alloc).
Definition W  := (entangle PA (FlatCombine fc)).  
Program Definition flat_combine f (v : fc_inT f) : STsep [W] 
 (fun s1 => pv_self s1 = Unit /\ fc_self s1 = Unit,
  fun (w : fc_outT f) s2 => exists g,
     pv_self s2 = Unit /\ fc_self s2 = g /\ fc_R f v w g) := ...
\end{lstlisting}
% 
The precondition says that \code{flat_combine} executes in the empty
initial heap (\code{pv_self s1 = Unit}), and hence by framing, in any
initial heap. Similarly, the initially assumed effects of the calling
thread on the shared data structure are empty (\code{fc_self s1 =
  Unit}), but can be made arbitrary by applying FCSL's \emph{frame
  rule} to the spec of \code{flat_combine}.
%
The postcondition says that there exists an abstract PCM value
\code{g} describing the effect of \code{f} in terms of PCM elements
(\code{fc_R f v w g}). Moreover, the effect of \code{g} is attributed
to the invoking thread (\code{fc_self s2 = g}), even though in reality
\code{f} could be executed by the combiner, \emph{on behalf} of the
calling thread.
%
In our Coq implementation, we instantiated the FC structure with a sequential
stack, showing that the result has the same spec as a concurrent stack
implementation.
