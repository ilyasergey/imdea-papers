
\section{Elements of FCSL infrastructure}
\label{sec:underhood}

In this section we sketch two important parts of FCSL machinery, used
to simplify construction of proofs.

\paragraph{Extracting concurroid structure via getters.~~}
\label{sec:getters}

When working with compositions of multiple concurroids, as in examples
listed in Section~\ref{sec:more-examples}, one frequently has to
select the \emph{self}, \emph{joint} or \emph{other} components that
belong to one of the composed concurroids. A na\"{i}ve way of doing this
is to describe the state space of the composition concurroid using 
existentials that abstract over the concurroid-specific
fields. \emph{E.g.}, in the case of flat combiner, which composes
three concurroids \code{Priv}, \code{Alloc} and \code{FlatCombine}, we
could use three existentials to abstract over
\emph{self}/\emph{joint}/\emph{other} fields for \code{Priv}, another
three for \code{Alloc}, and three more for \code{FlatCombine}. To access
any of the fields, we have to destruct all
nine of the existentials. This quickly becomes tedious and results in
proofs that are obscured by such existential destruction.

Our alternative approach develops a systematic way of projecting the
fields associated with each concurroid, based on the concurroid's
label. Thus, for example, we can write \code{self\ pv\ s} to obtain
the \emph{self} component of \code{s}, associated with a concurroid
whose label is \code{pv}. The identifier \code{pv_self} we used in the
spec for \code{span_root} and for \code{flat_combine} is a notational
abbreviation for exactly this projection. While this is a simple and
obvious idea, its execution required a somewhat involved use of
dependently-typed programming, and an intricate automation by
canonical structures and lemma
overloading~\cite{Gonthier-al:ICFP11,Mahboubi-Tassi:ITP13}.

%As it was shown in Section~\ref{sec:more-examples}, most of the
%programs ``span'' a set of resources, described by a composite
%concurroid. To make proofs in such cases as simple as proofs about
%primitive (one-label) concurroids, we used the power of
%dependently-typed programming and implemented a library of \emph{state
%  getters}.
%
%Given the label of an arbitrary primitive concurroid, the type of its
%self/other and the joint component (which is typically \code{heap}),
%and the default value for the joint component, the function
%\code{Getter} will provide a dependent record with the three getters
%for this concurroid's state components.
%
%The following code shows an instantiation of the \code{Getter}
%interface for the graph concurroid, with a label \code{sp}, auxiliary
%state's type \code{ptr_set} and joint component to be a heap.
%
%\begin{lstlisting}
%Notation ptr_set := (ptrmap_pcm_Encoded [encoded_set of unit]).  
%Definition spg := Getter sp ptr_set heap_tp_Encoded Unit.
%
%Notation self s := (self spg s).
%Notation other s := (other spg s). 
%Notation joint s := (joint spg s).
%\end{lstlisting}

\paragraph{Structural lemmas.~~}
\label{sec:lemmas}

The proofs in FCSL are structured to facilitate systematic application
of Floyd-style structural rules, one for each program command. All the
rules are proved sound from first principles, and are applied as
lemmas to advance the verification.
%
As the first step of every proof, the system implicitly applies the
weakening rule to the automatically synthesized weakest pre- and
strongest postconditions~\cite{Dijkstra:CACM75}, essentially
converting the program into the continuation-passing style (CPS)
representation and sequentializing its structure.
%
Every statement-specific structural rule ``symbolically evaluates''
the program by one step, and replaces the goal with a new one to be
verified.

For example, the following lemma \code{step}, corresponding to the
rule of sequential composition, reduces the verification of a program
$(y \asgn e_1; (e_2~y))$ with continuation~$k$, to the verification
of the program $e_1$ and the program $e_2~y~k$, where $y$ corresponds
to a symbolic result of evaluating $e_1$, constrained according to
$e_1$'s postcondition. One can apply it several times until $e_1$ is
reduced to some primitive action, at which point one can apply the
structural rule for that action.
%
\begin{lstlisting}
Lemma step W A B (e1 : ST W A) (e2 : A -> ST W B) i (k : cont B): 
        verify i e1 (fun y m => verify m (e2 y) k) ->
        verify i (y <-- e1; e2 y) k.  
\end{lstlisting}
%
\code{ST} is a type synonym for \code{STsep}, hiding its pre's and
post's.


% \paragraph{Soundness of the shallow embedding.~}
% \todo{Explain the semantics of Hoare types}



