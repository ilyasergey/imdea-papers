
\section{Elements of FCSL Infrastructure}
\label{sec:underhood}

In this section we sketch the ideas behind implementation of FCSL as
an embedding into Coq, and describe important parts of FCSL machinery,
used to simplify construction of proofs.

\subsection{FCSL Model and Embedding into Coq}
\label{sec:shallow-embedding}

Programs in FCSL are encoded as their values in the denotational
semantics of sets of \emph{action trees}~\cite[Appendix F of extended
version]{Nanevski-al:ESOP14}. The trees are a structured version of
\emph{action traces} by Brookes~\cite{Brookes:TCS07}. They implement
finite, partial approximations of the behavior of FCSL commands. Basic
commands, such as \code{ret} and atomic actions are given semantics as
finite sets of primitive trees. The semantics of program combinators,
such as sequential or parallel composition, is defined as a
set-comprehension over the elements of their constituents' semantics,
and \code{ffix} combinator is defined by taking Tarski's fixed point
of its argument function's monotone closure over the powerset lattice.
%
Therefore, each FCSL program (\eg, \code{span} in
Figure~\ref{fig:coq-span}) is a value in Coq, obtained by composing
primitive commands and atomic actions via FCSL combinators and native
Coq expressions (\eg, functions and conditionals).

In our implementation, the definition of each FCSL command (\ie, its
denotational semantics) is packaged, as a dependent record, together
with a specification, corresponding to its \emph{weakest pre-} and
\emph{strongest postconditions}~\cite{Dijkstra:CACM75} \wrt the
natural safety predicate,\footnote{The predicate states that a program
  can execute (possibly, concurrently) in a state, satisfying a given
  precondition, without crashing~\cite[\S F.5]{Nanevski-al:ESOP14}.}
and with a proof that such specification is a valid one for the
corresponding program's semantics.
%
We also defined a number of implicit coercions that inject a command
into the corresponding record instance, containing the command's
spec. This allowed us to exploit the type-checking machinery of Coq,
which is capable of composing coercions, thus \emph{synthesizing}
weakest pre- and strongest postconditions for arbitrarily complex
well-typed FCSL programs.

\subsection{Verification Conditions and Structural Lemmas}
\label{sec:lemmas}

Even though each well-typed FCSL program has already the strongest
specification inferred automatically by the system, such spec isn't
what one would typically like to ascribe to a program and expose to
the clients. Therefore, verification in FCSL reduces to
\emph{weakening} the derived strongest spec to the one of interest.

The verification conditions for weakening are encoded by wrapping a
program into the \code{Do} constructor (\eg, in
Figure~\ref{fig:coq-span}), which emits a corresponding proof
obligation.
%
Such proofs in FCSL are structured to facilitate systematic
application of Floyd-style structural rules, one for each command. All
the rules are proved sound from first principles, and are applied as
lemmas to advance the verification.
%
As the first step of every proof, the system implicitly applies the
weakening rule to the automatically synthesized weakest pre- and
strongest postconditions, essentially converting the program into the
continuation-passing style (CPS) representation and sequentializing
its structure.
%
Every statement-specific structural rule ``symbolically evaluates''
the program by one step, and replaces the goal with a new one (or
several ones) to be verified.

For example, the following lemma \code{step}, corresponding to the
rule of sequential composition, reduces the verification of a program
$(y \asgn e_1; (e_2~y))$ with continuation~$k$, to the verification
of the program $e_1$ and the program $e_2~y~k$, where $y$ corresponds
to a symbolic result of evaluating $e_1$, constrained according to
$e_1$'s postcondition. One can apply it several times until $e_1$ is
reduced to some primitive action, at which point one can apply the
structural rule for that action.
%
\begin{lstlisting}
Lemma step W A B (e1 : ST W A) 
                   (e2 : A -> ST W B) i (k : cont B): 
        verify i e1 (fun y m => verify m (e2 y) k) ->
        verify i (y <-- e1; e2 y) k.  
\end{lstlisting}
%
\code{ST} is a type synonym for \code{STsep}, hiding its pre's and
post's.

\subsection{Extracting Concurroid Structure via Getters}
\label{sec:getters}

When working with compositions of multiple concurroids, as in examples
listed in Section~\ref{sec:more-examples}, one frequently has to
select the \emph{self}, \emph{joint} or \emph{other} components that
belong to one of the composed concurroids. A na\"{i}ve way of doing
this is to describe the state space of the composition concurroid
using existentials that abstract over the concurroid-specific
fields. For instance, in the case of flat combiner, which composes
three concurroids \code{Priv}, \code{Alloc} and \code{FlatCombine}, we
could use three existentials to abstract over
\emph{self}/\emph{joint}/\emph{other} fields for \code{Priv}, another
three for \code{Alloc}, and three more for \code{FlatCombine}. To
access any of the fields, we have to destruct all nine of the
existentials. This quickly becomes tedious and results in proofs that
are obscured by such existential destruction.

Our alternative approach develops a systematic way of projecting the
fields associated with each concurroid, based on the concurroid's
label. Thus, for example, we can write \code{self\ pv\ s} to obtain
the \emph{self} component of \code{s}, associated with a concurroid
whose label is \code{pv}. The identifier \code{pv_self} we used in the
spec for \code{span_root} and for \code{flat_combine} is a notational
abbreviation for exactly this projection. While this is an obvious
idea, its execution required a somewhat involved use of
dependently-typed programming, and an intricate automation by
employing canonical structures and lemma
overloading~\cite{Gonthier-al:ICFP11,Mahboubi-Tassi:ITP13}.

