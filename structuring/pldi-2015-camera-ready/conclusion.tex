\vspace{-5pt}

\section{Conclusion}
\label{sec:conclusion}

Our experience with implementing a number of concurrent data
structures in FCSL indicates a recurring pattern, exhibited by the
formal proof development.
%
Verification of a new library in FCSL starts from describing its
invariants and evolution in terms of an STS. It's common to consider
parts of real or auxiliary state, which are a subject of the logical
split between parallel threads, as elements of a particular PCM.
%
Such representation of resources makes the verification uniform
and compositional, as it internalizes the library protocol, so the
clients can reason out of the specifications.

This observation indicates that STSs and PCMs can be a robust basis
for understanding, formalizing and verifying existing fine-grained
programs. We conjecture that the same foundational insights will play
a role in future designs and proofs of correctness of novel concurrent
algorithms.

\paragraph{Acknowledgements}
% \vspace{-5pt}
%\acks

We thank the anonymous reviewers from PLDI'15 PC and AEC for their
feedback. We are also grateful to Steve Blackburn for his efforts as
PLDI PC chair. This research is partially supported by Ramon y Cajal
grant RYC-2010-0743 and the US National Science Foundation (NSF). Any
opinion, findings, and conclusions or recommendations expressed in
this material are those of the authors and do not necessarily reflect
the views of NSF.

\vspace{-5pt}
